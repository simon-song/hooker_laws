% \chapter*[The Fifth Book]{THE FIFTH BOOK. 
% OF THEIR FOURTH ASSERTION, THAT TOUCHING THE SEVERAL PUBLIC DUTIES OF CHRISTIAN
% RELIGION, THERE IS AMONGST US MUCH SUPERSTITION RETAINED IN THEM; AND CONCERNING
% PERSONS WHICH FOR PERFORMANCE OF THOSE DUTIES ARE ENDUED WITH THE POWER OF
% ECCLESIASTICAL ORDER, OUR LAWS AND PROCEEDINGS ACCORDING THEREUNTO ARE MANY WAYS
% HEREIN ALSO CORRUPT.}
% \label{chap:book5}
% \addcontentsline{toc}{chapter}{THE FIFTH BOOK}

\chapter*[The Fifth Book]{THE FIFTH BOOK}
\label{chap:book5}
\addcontentsline{toc}{chapter}{THE FIFTH BOOK}

OF THEIR FOURTH ASSERTION, THAT TOUCHING THE SEVERAL PUBLIC DUTIES OF CHRISTIAN
RELIGION, THERE IS AMONGST US MUCH SUPERSTITION RETAINED IN THEM; AND CONCERNING
PERSONS WHICH FOR PERFORMANCE OF THOSE DUTIES ARE ENDUED WITH THE POWER OF
ECCLESIASTICAL ORDER, OUR LAWS AND PROCEEDINGS ACCORDING THEREUNTO ARE MANY WAYS
HEREIN ALSO CORRUPT.

\vspace*{2cm}

\PRLsep

%%%%%%%%%%%%%%%%%%%%%%%%%%%%%%%%%%%%%%%%%%%%%%%%%%%%%%%%
% BOOK V1. Dedication.

\begin{center}
  TO THE  \par
  MOST REVEREND FATHER IN GOD,  \par
  MY VERY GOOD LORD,  \par
  THE LORD ARCHBISHOP OF CANTERBURY  \par
  HIS GRACE,  \par
  PRIMATE AND METROPOLITAN OF ALL ENGLAND.
\end{center}

\vspace*{1cm}

\noindent
Most Reverend in Christ,

\vspace*{0.5cm}

The long-continued and more than ordinary favour which hitherto your Grace hath been pleased to shew towards me may justly claim at my hands some thankful acknowledgment thereof. In which consideration, as also for that I embrace willingly the ancient received course and conveniency of that discipline, which teacheth inferior degrees and orders in the Church of God to submit their writings to the same authority, from which their allowable dealings whatsoever in such affairs must receive approbation, I nothing fear but that your accustomed clemency will take in good worth the offer of these my simple and mean labours, bestowed for the necessary justification of laws heretofore made questionable, because as I take it they were not perfectly understood.

[2]For surely I cannot find any great cause of just complaint, that good laws have so much been wanting unto us, as we to them. To seek reformation of evil laws is a commendable endeavour; but for us the more necessary is a speedy redress of ourselves. We have on all sides lost much of our first fervency towards God; and therefore concerning our own degenerated ways we have reason to exhort with St. Gregory, Ὅπερ ἠ̑μεν γενώμεθα, “Let us return again unto that which we sometime were:” but touching the exchange of laws in practice with laws in device, which they say are better for the state of the Church, if they might take place, the farther we examine them the greater cause we find to conclude, μένωμεν ὅπερ ἐσμέν, “although we continue the same we are, the harm is not great.” These fervent reprehenders of things established by public authority are always confident and bold-spirited men. But their confidence for the most part riseth from too much credit given to their own wits, for which cause they are seldom free from error. The errors which we seek to reform in this kind of men are such as both received at your own hands their first wound, and from that time to this present have been proceeded in with that moderation, which useth by patience to suppress boldness, and to make them conquer that suffer.

[3]Wherein considering the nature and kind of these controversies, the dangerous sequels whereunto they were likely to grow, and how many ways we have been thereby taught wisdom, I may boldly aver concerning the first, that as the weightiest conflicts the Church hath had were those which touched the Head, the Person of our Saviour Christ; and the next of importance those questions which are at this day between us and the Church of Rome about the actions of the body of the Church of God; so these which have lastly sprung up for complements, rites, and ceremonies of church actions, are in truth for the greatest part such silly things, that very easiness doth make them hard to be disputed of in serious manner. Which also may seem to be the cause why divers of the reverend prelacy, and other most judicious men, have especially bestowed their pains about the matter of jurisdiction. Notwithstanding led by your Grace’s example myself have thought it convenient to wade thorough the whole cause, following that method which searcheth the truth by the causes of truth.

[4]Now if any marvel how a thing in itself so weak could import any great danger, they must consider not so much how small the spark is that flieth up, as how apt things about it are to take fire. Bodies politic being subject as much as natural to dissolution by divers means, there are undoubtedly more estates overthrown through diseases bred within themselves than through violence from abroad; because our manner is always to cast a doubtful and a more suspicious eye towards that over which we know we have least power; and therefore the fear of external dangers causeth forces at home to be the more united; it is to all sorts a kind of bridle, it maketh virtuous minds watchful, it holdeth contrary dispositions in suspense, and it setteth those wits on work in better things which would else be employed in worse: whereas on the other side domestical evils, for that we think we can master them at all times, are often permitted to run on forward till it be too late to recall them. In the mean while the commonwealth is not only through unsoundness so far impaired as those evils chance to prevail, but further also through opposition arising between the unsound parts and the sound, where each endeavoureth to draw evermore contrary ways, till distraction in the end bring the whole to ruin.

[5]To reckon up how many causes there are, by force whereof divisions may grow in a commonwealth, is not here necessary. Such as rise from variety in matter of religion are not only the farthest spread, because in religion all men presume themselves interessed alike; but they are also for the most part hotlier prosecuted and pursued than other strifes, forasmuch as coldness, which in other contentions may be thought to proceed from moderation, is not in these so favourably construed. The part which in this present quarrel striveth against the current and stream of laws was a long while nothing feared, the wisest contented not to call to mind how errors have their effect many times not proportioned to that little appearance of reason whereupon they would seem built, but rather to the vehement affection or fancy which is cast towards them and proceedeth from other causes. For there are divers motives drawing men to favour mightily those opinions, wherein their persuasions are but weakly settled; and if the passions of the mind be strong, they easily sophisticate the understanding; they make it apt to believe upon very slender warrant, and to imagine infallible truth where scarce any probable show appeareth.

[6]Thus were those poor seduced creatures, Hacket and his other two adherents, whom I can neither speak nor think of but with much commiseration and pity, thus were they trained by fair ways, first accounting their own extraordinary love to this Discipline a token of God’s more than ordinary love towards them; from hence they grew to a strong conceit, that God, which had moved them to love his Discipline more than the common sort of men did, might have a purpose by their means to bring a wonderful work to pass, beyond all men’s expectation, for the advancement of the throne of Discipline by some tragical execution, with the particularities whereof it was not safe for their friends to be made acquainted; of whom they did therefore but covertly demand, what they thought of extraordinary motions of the Spirit in these days, and withal request to be commended unto God by their prayers whatsoever should be undertaken by men of God in mere zeal to his glory and the good of his distressed Church. With this unusual and strange course they went on forward, till God, in whose heaviest worldly judgments I nothing doubt but that there may lie hidden mercy, gave them over to their own inventions, and left them made in the end an example for headstrong and inconsiderate zeal no less fearful, than Achitophel for proud and irreligious wisdom. If a spark of error have thus far prevailed, falling even where the wood was green and farthest off to all men’s thinking from any inclination unto furious attempts; must not the peril thereof be greater in men whose minds are of themselves as dry fuel, apt beforehand unto tumults, seditions, and broils? But by this we see in a cause of religion to how desperate adventures men will strain themselves, for relief of their own part, having law and authority against them.

[7]Furthermore let not any man think that in such divisions either part can free itself from inconveniences, sustained not only through a kind of truce, which virtue on both sides doth make with vice during war between truth and error; but also in that there are hereby so fit occasions ministered for men to purchase to themselves well-willers, by the colour under which they oftentimes prosecute quarrels of envy or inveterate malice: and especially because contentions were as yet never able to prevent two evils; the one a mutual exchange of unseemly and unjust disgraces offered by men whose tongues and passions are out of rule; the other a common hazard of both to be made a prey by such as study how to work upon all occurrents with most advantage in private. I deny not therefore, but that our antagonists in these controversies may peradventure have met with some not unlike to Ithacius; who mightily bending himself by all means against the heresy of Priscillian, the hatred of which one evil was all the virtue he had, became so wise in the end, that every man careful of virtuous conversation, studious of Scripture, and given unto any abstinence in diet, was set down in his calendar of suspected Priscillianists, for whom it should be expedient to approve their soundness of faith by a more licentious and loose behaviour. Such proctors and patrons the truth might spare. Yet is not their grossness so intolerable, as on the contrary side the scurrilous and more than satirical immodesty of Martinism; the first published schedules whereof being brought to the hands of a grave and a very honourable knight, with signification given that the book would refresh his spirits, he took it, saw what the title was, read over an unsavoury sentence or two, and delivered back the libel with this answer: “I am sorry you are of the mind to be solaced with these sports, and sorrier you have herein thought mine affection to be like your own.”

[8]But as these sores on all hands lie open, so the deepest wounds of the Church of God have been more softly and closely given. It being perceived that the plot of Discipline did not only bend itself to reform ceremonies, but seek farther to erect a popular authority of Elders, and to take away episcopal jurisdiction, together with all other ornaments and means whereby any difference or inequality is upheld in the ecclesiastical order; towards this destructive part they have found many helping hands, divers, although peradventure not willing to be yoked with elderships, yet contented (for what intent God doth know) to uphold opposition against bishops; not without greater hurt to the course of their whole proceedings in the business of God and her Majesty’s service, than otherwise much more weighty adversaries had been able by their own power to have brought to pass. Men are naturally better contented to have their commendable actions suppressed, than the contrary much divulged. And because the wits of the multitude are such, that many things they cannot lay hold on at once, but being possest with some notable either dislike or liking of any one thing whatsoever, sundry other in the meantime may escape them unperceived: therefore if men desirous to have their virtues noted do in this respect grieve at the fame of others, whose glory obscureth and darkeneth theirs; it cannot be chosen but that when the ears of the people are thus continually beaten with exclamations against abuses in the Church, these tunes come always most acceptable to them, whose odious and corrupt dealings in secular affairs both pass by that mean the more covertly, and whatsoever happen do also the least feel that scourge of vulgar imputation, which notwithstanding they most deserve.

[9]All this considered as behoveth, the sequel of duty on our part is only that which our Lord and Saviour requireth, harmless discretion; the wisdom of serpents tempered with the innocent meekness of doves. For this world will teach them wisdom that have capacity to apprehend it. Our wisdom in this case must be such as doth not propose to itself τὸ ἴδιον, our own particular, the partial and immoderate desire whereof poisoneth wheresoever it taketh place; but the scope and mark which we are to aim at is τὸ κοινὸν, the public and common good of all; for the easier procurement whereof, our diligence must search out all helps and furtherances of direction, which scriptures, councils, fathers, histories, the laws and practices of all churches, the mutual conference of all men’s collections and observations may afford: our industry must even anatomize every particle of that body, which we are to uphold sound. And because be it never so true which we teach the world to believe, yet if once their affections begin to be alienated, a small thing persuadeth them to change their opinions, it behoveth that we vigilantly note and prevent by all means those evils whereby the hearts of men are lost: which evils for the most part being personal do arm in such sort the adversaries of God and his Church against us, that, if through our too much neglect and security the same should run on, soon might we feel our estate brought to those lamentable terms, whereof this hard and heavy sentence was by one of the ancient uttered upon like occasions, “Dolens dico, gemens denuncio, sacerdotium quod apud nos intus cecidit, foris diu stare non poterit.”

[10]But the gracious providence of Almighty God hath I trust put these thorns of contradiction in our sides, lest that should steal upon the Church in a slumber, which now I doubt not but through his assistance may be turned away from us, bending thereunto ourselves with constancy; constancy in labour to do all men good, constancy in prayer unto God for all men: her especially whose sacred power matched with incomparable goodness of nature hath hitherto been God’s most happy instrument, by him miraculously kept for works of so miraculous preservation and safety unto others, that as, “By the sword of God and Gedeon,” was sometime the cry of the people of Israel, so it might deservedly be at this day the joyful song of innumerable multitudes, yea, the emblem of some estates and dominions in the world, and (which must be eternally confest even with tears of thankfulness) the true inscription, style, or title, of all churches as yet standing within this realm, “By the goodness of Almighty God and his servant Elizabeth we are.” That God who is able to make mortality immortal give her such future continuance, as may be no less glorious unto all posterity than the days of her regiment past have been happy unto ourselves; and for his most dear anointed’s sake grant them all prosperity, whose labours, cares, and counsels, unfeignedly are referred to her endless welfare: through his unspeakable mercy, unto whom we all owe everlasting praise. In which desire I will here rest, humbly beseeching your Grace to pardon my great boldness, and God to multiply his blessings upon them that fear his name.

\vspace*{1cm}

\hfill Your Grace’s in all duty,

\hfill RICHARD HOOKER.

%%%%%%%%%%%%%%%%%%%%%%%%%%%%%%%%%%%%%%%%%%%%%%%%%%%%%%%%
\newpage

\noindent
MATTER CONTAINED IN THIS FIFTH BOOK.

\vspace{0.1in}

I. True religion is the root of all true virtues and the stay of all well-ordered commonwealths.

II. The most extreme opposite to true Religion is affected Atheism.

III. Of Superstition, and the root thereof, either misguided zeal, or ignorant fear of divine glory.

IV. Of the redress of superstition in God’s Church, and concerning the question of this book.

V. Four general propositions demanding that which may reasonably be granted, concerning matters of outward form in the exercise of true Religion. And, fifthly, of a rule not safe nor reasonable in these cases.

VI. The first proposition touching judgment what things are convenient in the outward public ordering of church affairs.

VII. The second proposition.

VIII. The third proposition.

IX. The fourth proposition.

X. The rule of men’s private spirits not safe in these cases to be followed.

XI. Places for the public service of God.

XII. The solemnity of erecting Churches condemned, the hallowing and dedicating of them scorned by the adversary.

XIII. Of the names whereby we distinguish our Churches.

XIV. Of the fashion of our Churches.

XV. The sumptuousness of Churches.

XVI. What holiness and virtue we ascribe to the Church more than other places.

XVII. Their pretence that would have Churches utterly razed.

XVIII. Of public teaching or preaching, and the first kind thereof, catechising.

XIX. Of preaching by reading publicly the books of Holy Scripture; and concerning supposed untruths in those Translations of Scripture which we allow to be read; as also of the choice which we make in reading.  

XX. Of preaching by the public reading of other profitable instructions; and concerning books Apocryphal.

XXI. Of preaching by Sermons, and whether Sermons be the only ordinary way of teaching whereby men are brought to the saving knowledge of God’s truth.

XXII. What they attribute to Sermons only, and what we to reading also.

XXIII. Of Prayer.

XXIV. Of public Prayer.

XXV. Of the form of Common Prayer.

XXVI. Of them which like not to have any set form of Common Prayer.

XXVII. Of them who allowing a set form of prayer yet allow not ours.

XXVIII. The form of our Liturgy too near the papists’, too far different from that of other reformed Churches, as they pretend.

XXIX. Attire belonging to the service of God.

XXX. Of gesture in praying, and of different places chosen to that purpose.

XXXI. Easiness of praying after our form.

XXXII. The length of our service.

XXXIII. Instead of such prayers as the primitive Churches have used, and those that the reformed now use, we have (they say) divers short cuts or shreddings, rather wishes than prayers.

XXXIV. Lessons intermingled with our prayers.

XXXV. The number of our prayers for earthly things, and our oft rehearsing of the Lord’s Prayer.

XXXVI. The people’s saying after the minister.

XXXVII. Our manner of reading the Psalms otherwise than the rest of the Scripture.

XXXVIII. Of Music with Psalms.

XXXIX. Of singing or saying Psalms, and other parts of Common Prayer wherein the people and the minister answer one another by course.

XL. Of Magnificat, Benedictus, and Nunc Dimittis.

XLI. Of the Litany.

XLII. Of Athanasius’s Creed, and Gloria Patri.

XLIII. Our want of particular thanksgiving.

XLIV. In some things the matter of our prayer, as they affirm, is unsound.

XLV. “When thou hadst overcome the sharpness of death, thou didst open the Kingdom of Heaven unto all believers.”

XLVI. Touching prayer for deliverance from sudden death.

XLVII. Prayer that those things which we for our unworthiness dare not ask, God for the worthiness of his Son would vouchsafe to grant.

XLVIII. Prayer to be evermore delivered from all adversity.

XLIX. Prayer that all men may find mercy.

L. Of the name, the author, and the force of Sacraments, which force consisteth in this, that God hath ordained them as means to make us partakers of him in Christ, and of life through Christ.

LI. That God is in Christ by the personal incarnation of the Son, who is very God.  

LII. The misinterpretations which heresy hath made of the manner how God and man are united in one Christ.

LIII. That by the union of the one with the other nature in Christ, there groweth neither gain nor loss of essential properties to either.

LIV. What Christ hath obtained according to the flesh, by the union of his flesh with Deity.

LV. Of the personal presence of Christ every where, and in what sense it may be granted he is every where present according to the flesh.

LVI. The union or mutual participation which is between Christ and the Church of Christ in this present world.

LVII. The necessity of Sacraments unto the participation of Christ.

LVIII. The substance of Baptism, the rites or solemnities thereunto belonging, and that the substance thereof being kept, other things in Baptism may give place to necessity.

LIX. The ground in Scripture whereupon a necessity of outward Baptism hath been built.

LX. What kind of necessity in outward Baptism hath been gathered by the words of our Saviour Christ; and what the true necessity thereof indeed is.

LXI. What things in Baptism have been dispensed with by the fathers respecting necessity.

LXII. Whether baptism by Women be true Baptism, good and effectual to them that receive it.

LXIII. Of Interrogatories in Baptism touching faith and the purpose of a Christian life.

LXIV. Interrogatories proposed unto infants in Baptism, and answered as in their names by godfathers.

LXV. Of the Cross in Baptism.

LXVI. Of Confirmation after Baptism.

LXVII. Of the Sacrament of the body and blood of Christ.

LXVIII. Of faults noted in the form of administering that holy Sacrament.

LXIX. Of Festival Days, and the natural causes of their convenient institution.

LXX. The manner of celebrating festival days.

LXXI. Exceptions against our keeping of other festival days besides the Sabbath.

LXXII. Of days appointed as well for ordinary as for extraordinary Fasts in the Church of God.

LXXIII. The celebration of Matrimony.

LXXIV. The Churching of Women.

LXXV. The Rites of Burial.

LXXVI. Of the nature of that Ministry which serveth for performance of divine duties in the Church of God, and how happiness not eternal only but also temporal doth depend upon it.

LXXVII. Of power given unto men to execute that heavenly office, of the gift of the Holy Ghost in Ordination; and whether conveniently the power of order may be sought or sued for.  

LXXVIII. Of Degrees whereby the power of Order is distinguished, and concerning the Attire of ministers.

LXXIX. Of Oblations, Foundations, Endowments, Tithes, all intended for perpetuity of religion; which purpose being chiefly fulfilled by the clergy’s certain and sufficient maintenance, must needs by alienation of church livings be made frustrate.

LXXX. Of Ordination lawful without Title, and without any popular Election precedent, but in no case without regard of due information what their quality is that enter into holy orders.

LXXXI. Of the Learning that should be in ministers, their Residence, and the number of their Livings.


%%%%%%%%%%%%%%%%%%%%%%%%%%%%%%%%%%%%%%%%%%%%%%%%%%%%%%%%%%%%%
\section*{True religion is the root of all true virtues and the stay of all well-ordered commonwealths.}

I. Few there are of so weak capacity, but public evils they easily espy; fewer so patient, as not to complain, when the grievous inconveniences thereof work sensible smart. Howbeit to see wherein the harm which they feel consisteth, the seeds from which it sprang, and the method of curing it, belongeth to a skill, the study whereof is so full of toil, and the practice so beset with difficulties, that wary and respective men had rather seek quietly their own, and wish that the world may go well, so it be not long of them, than with pain and hazard make themselves advisers for the common good. We which thought it at the very first a sign of cold affection towards the Church of God, to prefer private ease before the labour of appeasing public disturbance, must now of necessity refer events to the gracious providence of Almighty God, and, in discharge of our duty towards him, proceed with the plain and unpartial defence of a common cause. Wherein our endeavour is not so much to overthrow them with whom we contend, as to yield them just and reasonable causes of those things, which, for want of due consideration heretofore, they misconceived, accusing laws for men’s oversights, imputing evils, grown through personal defects unto that which is not evil, framing unto some sores unwholesome plaisters, and applying other some where no sore is.
To make therefore our beginning that which to both parts is most acceptable, We agree that pure and unstained religion ought to be the highest of all cares appertaining to  public regiment: as well in regard of that aid and protection which they who faithfully serve God confess they receive at his merciful hands; as also for the force which religion hath to qualify all sorts of men, and to make them in public affairs the more serviceable, governors the apter to rule with conscience, inferiors for conscience’ sake the willinger to obey. It is no peculiar conceit, but a matter of sound consequence, that all duties are by so much the better performed, by how much the men are more religious from whose abilities the same proceed. For if the course of politic affairs cannot in any good sort go forward without fit instruments, and that which fitteth them be their virtues, let Polity acknowledge itself indebted to Religion; godliness being the chiefest top and wellspring of all true virtues, even as God is of all good things.
So natural is the union of Religion with Justice, that we may boldly deem there is neither, where both are not. For how should they be unfeignedly just, whom religion doth not cause to be such; or they religious, which are not found such by the proof of their just actions? If they, which employ their labour and travail about the public administration of justice, follow it only as a trade, with unquenchable and unconscionable thirst of gain, being not in heart persuaded that justice is God’s own work, and themselves his agents in this business, the sentence of right God’s own verdict, and themselves his priests to deliver it; formalities of justice do but serve to smother right, and that, which was necessarily ordained for the common good, is through shameful abuse made the cause of common misery.
The same piety, which maketh them that are in authority desirous to please and resemble God by justice, inflameth  every way men of action with zeal to do good (as far as their place will permit) unto all. For that, they know, is most noble and divine. Whereby if no natural nor casual inability cross their desires, they always delighting to inure themselves with actions most beneficial to others, cannot but gather great experience, and through experience the more wisdom; because conscience, and the fear of swerving from that which is right, maketh them diligent observers of circumstances, the loose regard whereof is the nurse of vulgar folly, no less than Salomon’s attention thereunto was of natural furtherances the most effectual to make him eminent above others. For he gave good heed, and pierced every thing to the very ground, and by that mean became the author of many parables.
Concerning fortitude; sith evils great and unexpected (the true touchstone of constant minds) do cause oftentimes even them to think upon divine power with fearfullest suspicions, which have been otherwise the most secure despisers thereof; how should we look for any constant resolution of mind in such cases, saving only where unfeigned affection to God-ward hath bred the most assured confidence to be assisted by his hand? For proof whereof, let but the acts of the ancient Jews be indifferently weighed; from whose magnanimity, in causes of most extreme hazard, those strange and unwonted resolutions have grown, which for all circumstances no people under the roof of heaven did ever hitherto match. And that which did always animate them was their mere religion.
Without which, if so be it were possible that all other ornaments of mind might be had in their full perfection, nevertheless the mind that should possess them divorced from piety could be but a spectacle of commiseration; even as that body is, which adorned with sundry other admirable beauties, wanteth eyesight, the chiefest grace that nature hath in that kind to bestow. They which commend so much the felicity of that innocent world, wherein it is said that men of their own accord did embrace fidelity and honesty, not for fear of the magistrate, or because revenge was before their eyes, if at any time they  should do otherwise, but that which held the people in awe was the shame of ill-doing, the love of equity and right itself a bar against all oppressions which greatness of power causeth; they which describe unto us any such estate of happiness amongst men, though they speak not of Religion, do notwithstanding declare that which is in truth her only working. For, if Religion did possess sincerely and sufficiently the hearts of all men, there would need no other restraint from evil. This doth not only give life and perfection to all endeavours wherewith it concurreth; but what event soever ensue, it breedeth, if not joy and gladness always, yet always patience, satisfaction, and reasonable contentment of mind. Whereupon it hath been set down as an axiom of good experience, that all things religiously taken in hand are prosperously ended; because whether men in the end have that which religion did allow them to desire, or that which it teacheth them contentedly to suffer, they are in neither event unfortunate.
But lest any man should here conceive, that it greatly skilleth not of what sort our religion be, inasmuch as heathens, Turks, and infidels, impute to religion a great part of the same effects which ourselves ascribe thereunto, they having ours in the same detestation that we theirs; it shall be requisite to observe well, how far forth there may be agreement in the effects of different religions. First, by the bitter strife which riseth oftentimes from small differences in this behalf, and is by so much always greater as the matter is of more importance; we see a general agreement in the secret opinion of men, that every man ought to embrace the religion which is true, and to shun, as hurtful, whatsoever dissenteth from it, but that most, which doth farthest dissent. The generality of which persuasion argueth, that God hath imprinted it by nature, to the end it might be a spur to our industry in searching and maintaining that religion, from which as to swerve in the least points is error, so the capital enemies thereof God hateth as his deadly foes, aliens, and, without repentance, children of endless perdition. Such therefore touching man’s immortal state after this life are not likely to reap benefit by their  religion, but to look for the clean contrary, in regard of so important contrariety between it and the true religion.
Nevertheless, inasmuch as the errors of the most seduced this way have been mixed with some truths, we are not to marvel, that although the one did turn to their endless woe and confusion, yet the other had many notable effects as touching the affairs of this present life. There were in these quarters of the world, sixteen hundred years ago, certain speculative men, whose authority disposed the whole religion of those times. By their means it became a received opinion, that the souls of men departing this life do flit out of one body into some other. Which opinion, though false, yet entwined with a true, that the souls of men do never perish, abated the fear of death in them which were so resolved, and gave them courage unto all adventures.
The Romans had a vain superstitious custom, in most of their enterprises to conjecture beforehand of the event by certain tokens which they noted in birds, or in the entrails of beasts, or by other the like frivolous divinations. From whence notwithstanding as oft as they could receive any sign which they took to be favourable, it gave them such hope, as if their gods had made them more than half a promise of prosperous success. Which many times was the greatest cause that they did prevail, especially being men of their own natural inclination hopeful and strongly conceited, whatsoever they took in hand. But could their fond superstition have furthered so great attempts without the mixture of a true persuasion concerning the unresistible force of divine power?
Upon the wilful violation of oaths, execrable blasphemies, and like contempts, offered by deriders of religion even unto false gods, fearful tokens of divine revenge have been known to follow. Which occurrents the devouter sort did take for manifest arguments, that the gods whom they worshipped were of power to reward such as sought unto them, and would plague those that feared them not. In this they erred. For (as the wise man rightly noteth concerning such) it was not the power of them by whom they sware, but the vengeance of them that sinned, which punished the offences of the ungodly. It was their hurt untruly to attribute so great power  unto false gods. Yet the right conceit which they had, that to perjury vengeance is due, was not without good effect as touching the course of their lives, who feared the wilful violation of oaths in that respect.
And whereas we read so many of them so much commended, some for their mild and merciful disposition, some for their virtuous severity, some for integrity of life, all these were the fruits of true and infallible principles delivered unto us in the word of God as the axioms of our religion, which being imprinted by the God of nature in their hearts also, and taking better root in some than in most others, grew, though not from, yet with and amidst the heaps of manifold repugnant errors; which errors of corrupt religion had also their suitable effects in the lives of the selfsame parties.
Without all controversy, the purer and perfecter our religion is, the worthier effects it hath in them who steadfastly and sincerely embrace it, in others not. They that love the religion which they profess, may have failed in choice, but yet they are sure to reap what benefit the same is able to afford; whereas the best and soundest professed by them that bear it not the like affection, yieldeth them, retaining it in that sort, no benefit. David was a “man after God’s own heart,” so termed because his affection was hearty towards God. Beholding the like disposition in them which lived under him, it was his prayer to Almighty God, “O keep this for ever in the purpose and thoughts of the heart of this people.” But when, after that David had ended his days in peace, they who succeeded him in place for the most part followed him not in quality; when those kings (some few excepted) to better their worldly estate, (as they thought,) left their own and their people’s ghostly condition uncared for; by woful experience they both did learn, that to forsake the true God of heaven, is to fall into all such evils upon the face of the earth, as men either destitute of grace divine may commit, or unprotected from above endure.
Seeing therefore it doth thus appear that the safety of all estates dependeth upon religion; that religion unfeignedly loved perfecteth men’s abilities unto all kinds of virtuous services in the commonwealth; that men’s desire is in general to  hold no religion but the true; and that whatsoever good effects do grow out of their religion, who embrace instead of the true a false, the roots thereof are certain sparks of the light of truth intermingled with the darkness of error, because no religion can wholly and only consist of untruths: we have reason to think that all true virtues are to honour true religion as their parent, and all well-ordered commonweals to love her as their chiefest stay.

\section*{The most extreme opposite to true Religion is affected Atheism.}

II. They of whom God is altogether unapprehended are but few in number, and for grossness of wit such, that they hardly and scarcely seem to hold the place of human being. These we should judge to be of all others most miserable, but that a wretcheder sort there are, on whom whereas nature hath bestowed riper capacity, their evil disposition seriously goeth about therewith to apprehend God as being not God. Whereby it cometh to pass that of these two sorts of men, both godless, the one having utterly no knowledge of God, the other study how to persuade themselves that there is no such thing to be known. The fountain and wellspring of which impiety is a resolved purpose of mind to reap in this world what sensual profit or pleasure soever the world yieldeth, and not to be barred from any whatsoever means available thereunto. And that this is the very radical cause of their atheism, no man I think will doubt which considereth what pains they take to destroy those principal spurs and motives unto all virtue, the creation of the world, the providence of God, the resurrection of the dead, the joys of the kingdom of heaven, and the endless pains of the wicked, yea above all things the authority of Scripture, because on these points it evermore beateth, and the soul’s immortality, which granted, draweth easily after it the rest as a voluntary train. Is it not wonderful that base desires should so extinguish in men the sense of their own excellency, as to make them willing that their souls should be like to the souls of beasts, mortal and corruptible with their bodies? Till some admirable or unusual accident happen (as it hath in some) to work the beginning of a better alteration in their minds, disputation about the  knowledge of God with such kind of persons commonly prevaileth little. For how should the brightness of wisdom shine, where the windows of the soul are of very set purpose closed? True religion hath many things in it, the only mention whereof galleth and troubleth their minds. Being therefore loth that inquiry into such matters should breed a persuasion in the end contrary unto that they embrace, it is their endeavour to banish as much as in them lieth quite and clean from their cogitation whatsoever may sound that way.
But it cometh many times to pass (which is their torment) that the thing they shun doth follow them, truth as it were even obtruding itself into their knowledge, and not permitting them to be so ignorant as they would be. Whereupon inasmuch as the nature of man is unwilling to continue doing that wherein it shall always condemn itself, they continuing still obstinate to follow the course which they have begun, are driven to devise all the shifts that wit can invent for the smothering of this light, all that may but with any the least show of possibility stay their minds from thinking that true, which they heartily wish were false, but cannot think it so without some scruple and fear of the contrary.
Now because that judicious learning, for which we commend most worthily the ancient sages of the world, doth not in this case serve the turn, these trencher-mates (for such the most of them be) frame to themselves a way more pleasant; a new method they have of turning things that are serious into mockery, an art of contradiction by way of scorn, a learning wherewith we were long sithence forewarned that the miserable times whereinto we are fallen should abound. This they study, this they practise, this they grace with a wanton superfluity of wit, too much insulting over the patience of more virtuously disposed minds.
For towards these so forlorn creatures we are (it must be confest) too patient. In zeal to the glory of God, Babylon hath excelled Sion. We want that decree of Nabuchodonosor;  the fury of this wicked brood hath the reins too much at liberty; their tongues walk at large; the spit-venom of their poisoned hearts breaketh out to the annoyance of others; what their untamed lust suggesteth, the same their licentious mouths do every where set abroach.
With our contentions their irreligious humour also is much strengthened. Nothing pleaseth them better than these manifold oppositions about the matter of religion, as well for that they have hereby the more opportunity to learn on one side how another may be oppugned, and so to weaken the credit of all unto themselves; as also because by this hot pursuit of lower controversies amongst men professing religion, and agreeing in the principal foundations thereof, they conceive hope that about the higher principles themselves time will cause altercation to grow.
For which purpose, when they see occasion, they stick not sometime in other men’s persons, yea sometime without any vizard at all, directly to try, what the most religious are able to say in defence of the highest points whereupon all religion dependeth. Now for the most part it so falleth out touching things which generally are received, that although in themselves they be most certain, yet because men presume them granted of all, we are hardliest able to bring such proof of their certainty as may satisfy gainsayers, when suddenly and besides expectation they require the same at our hands. Which impreparation and unreadiness when they find in us, they turn it to the soothing up of themselves in that cursed fancy, whereby they would fain believe that the hearty devotion of such as indeed fear God is nothing else but a kind of harmless error, bred and confirmed in them by the sleights of wiser men.
For a politic use of religion they see there is, and by it they would also gather that religion itself is a mere politic device, forged purposely to serve for that use. Men  fearing God are thereby a great deal more effectually than by positive laws restrained from doing evil; inasmuch as those laws have no farther power than over our outward actions only, whereas unto men’s inward cogitations, unto the privy intents and motions of their hearts, religion serveth for a bridle. What more savage, wild, and cruel, than man, if he see himself able either by fraud to overreach, or by power to overbear, the laws whereunto he should be subject? Wherefore in so great boldness to offend, it behoveth that the world should be held in awe, not by a vain surmise, but a true apprehension of somewhat, which no man may think himself able to withstand. This is the politic use of religion.
In which respect there are of these wise malignants some, who have vouchsafed it their marvellous favourable countenance and speech, very gravely affirming, that religion honoured, addeth greatness, and contemned, bringeth ruin unto commonweals; that princes and states, which will continue, are above all things to uphold the reverend regard of religion, and to provide for the same by all means in the making of their laws.
But when they should define what means are best for that purpose, behold, they extol the wisdom of Paganism; they give it out as a mystical precept of great importance, that princes, and such as are under them in most authority or credit with the people, should take all occasions of rare events, and from what cause soever the same do proceed, yet wrest  them to the strengthening of their religion, and not make it nice for so good a purpose to use, if need be, plain forgeries. Thus while they study how to bring to pass that religion may seem but a matter made, they lose themselves in the very maze of their own discourses, as if reason did even purposely forsake them, who of purpose forsake God the author thereof. For surely a strange kind of madness it is, that those men who though they be void of piety, yet because they have wit cannot choose but know that treachery, guile, and deceit are things, which may for a while but do not use long to go unespied, should teach that the greatest honour to a state is perpetuity; and grant that alterations in the service of God, for that they impair the credit of religion, are therefore perilous in commonweals, which have no continuance longer than religion hath all reverence done unto it; and withal acknowledge (for so they do) that when people began to espy the falsehood of oracles, whereupon all Gentility was built, their hearts were utterly averted from it; and notwithstanding counsel princes in sober earnest, for the strengthening of their states to maintain religion, and for the maintenance of religion not to make choice of that which is true, but to authorize that they make choice of by those false and fraudulent means which in the end must needs overthrow it. Such are the counsels of men godless, when they would shew themselves politic devisers, able to create God in man by art.


\section*{Of Superstition, and the root thereof, either misguided zeal, or ignorant fear of divine glory.}

III. Wherefore to let go this execrable crew, and to come to extremities on the contrary hand; two affections there are, the forces whereof, as they bear the greater or lesser sway in man’s heart, frame accordingly the stamp and character of his religion; the one zeal, the other fear.
Zeal, unless it be rightly guided, when it endeavoureth most busily to please God, forceth upon him those unseasonable offices which please him not. For which cause, if they who this way swerve be compared with such sincere, sound,  and discreet, as Abraham was in matter of religion; the service of the one is like unto flattery, the other like the faithful sedulity of friendship. Zeal, except it be ordered aright, when it bendeth itself unto conflict with things either in deed, or but imagined to be opposite unto religion, useth the razor many times with such eagerness, that the very life of religion itself is thereby hazarded; through hatred of tares the corn in the field of God is plucked up. So that zeal needeth both ways a sober guide.
Fear on the other side, if it have not the light of true understanding concerning God, wherewith to be moderated, breedeth likewise superstition. It is therefore dangerous, that in things divine we should work too much upon the spur either of zeal or fear. Fear is a good solicitor to devotion. Howbeit, sith fear in this kind doth grow from an apprehension of Deity endued with irresistible power to hurt, and is of all affections (anger excepted) the unaptest to admit any conference with reason; for which cause the wise man doth say of fear that it is a betrayer of the forces of reasonable understanding; therefore except men know beforehand what manner of service pleaseth God, while they are fearful they try all things which fancy offereth. Many there are who never think on God but when they are in extremity of fear; and then, because what to think or what to do they are uncertain, perplexity not suffering them to be idle, they think and do as it were in a phrensy they know not what.
Superstition neither knoweth the right kind, nor observeth the due measure, of actions belonging to the service of God, but is always joined with a wrong opinion touching things divine. Superstition is, when things are either abhorred or observed with a zealous or fearful, but erroneous, relation to God. By means whereof, the superstitious do sometimes serve, though the true God, yet with needless offices, and defraud him of duties necessary; sometime load others than him with such honours as properly are his. The one their oversight, who miss in the choice of that wherewith; the other theirs, who fail in the election of him towards whom they shew devotion: this, the crime of idolatry, that, the fault of voluntary either niceness or superfluity in religion.
The Christian world itself being divided into two grand parts, it appeareth by the general view of both, that with matter of heresy the West hath been often and much troubled; but the East part never quiet, till the deluge of misery, wherein now they are, overwhelmed them. The chiefest cause whereof doth seem to have lien in the restless wits of the Grecians, evermore proud of their own curious and subtile inventions; which when at any time they had contrived, the great facility of their language served them readily to make all things fair and plausible to men’s understanding. Those grand heretical impieties therefore, which most highly and immediately touched God and the glorious Trinity, were all in a manner the monsters of the East. The West bred fewer a great deal, and those commonly of a lower nature, such as more nearly and directly concerned rather men than God; the Latins being always to capital heresies less inclined, yet unto gross superstition more.
Superstition such as that of the Pharisees was, by whom divine things indeed were less, because other things were more divinely esteemed of than reason would; the superstition that riseth voluntarily, and by degrees which are hardly discerned mingleth itself with the rites even of very divine service done to the only true God, must be considered of as a creeping and encroaching evil, an evil the first beginnings whereof are commonly harmless, so that it proveth only then to be an evil when some farther accident doth grow unto it, or itself come unto farther growth. For in the Church of God sometimes it cometh to pass as in over battle grounds, the fertile disposition whereof is good; yet because it exceedeth due proportion, it bringeth forth abundantly, through too much rankness, things less profitable; whereby that which principally it should yield, being either prevented in place, or defrauded of nourishment, faileth. This (if so large a discourse were necessary) might be exemplified even by heaps of rites and customs now superstitious in the greatest part of the Christian world, which in their first original beginnings, when the strength of virtuous,  devout, or charitable affection bloomed them, no man could justly have condemned as evil.

\section*{Of the redress of superstition in God’s Church, and concerning the question of this book.}
IV. But howsoever superstition do grow, that wherein unsounder times have done amiss, the better ages ensuing must rectify, as they may. I now come therefore to those accusations brought against us by pretenders of reformation; the first in the rank whereof is such, that if so be the Church of England did at this day therewith as justly deserve to be touched, as they in this cause have imagined it doth, rather would I exhort all sorts to seek pardon even with tears at the hands of God, than meditate words of defence for our doings, to the end that men might think favourably of them. For as the case of this world, especially now, doth stand, what other stay or succour have we to lean unto, saving the testimony of our conscience, and the comfort we take in this, that we serve the living God (as near as our wits can reach unto the knowledge thereof) even according to his own will, and do therefore trust that his mercy shall be our safeguard against those enraged powers abroad, which principally in that respect are become our enemies? But sith no man can do ill with a good conscience, the consolation which we herein seem to find, is but a mere deceitful pleasing of ourselves in error, which at the length must needs turn to our greater grief, if that which we do to please God most be for the manifold defects thereof offensive unto him. For so it is judged, our prayers, our sacraments, our fasts, our times and places of public meeting together for the worship and service of God, our marriages, our burials, our functions, elections and ordinations ecclesiastical, almost whatsoever we do in the exercise of our religion according to laws for that purpose established, all things are some way or other thought faulty, all things stained with superstition.
Now although it may be the wiser sort of men are not greatly moved hereat, considering how subject the very best things have been always unto cavil, when wits possessed either with disdain or dislike thereof have set them up as their mark to shoot at: safe notwithstanding it were not therefore to  neglect the danger which from hence may grow, and that especially in regard of them, who desiring to serve God as they ought, but being not so skilful as in every point to unwind themselves where the snares of glosing speech do lie to entangle them, are in mind not a little troubled, when they hear so bitter invectives against that which this church hath taught them to reverence as holy, to approve as lawful, and to observe as behoveful for the exercise of Christian duty. It seemeth therefore at the least for their sakes very meet, that such as blame us in this behalf be directly answered, and they which follow us informed plainly in the reasons of that we do.
On both sides the end intended between us, is to have laws and ordinances such as may rightly serve to abolish superstition, and to establish the service of God with all things thereunto appertaining in some perfect form.
There is an inward reasonable, and there is a solemn outward serviceable worship belonging unto God. Of the former kind are all manner virtuous duties that each man in reason and conscience to Godward oweth. Solemn and serviceable worship we name for distinction sake, whatsoever belongeth to the Church or public society of God by way of external adoration. It is the later of these two whereupon our present question groweth.
Again, this later being ordered, partly, and as touching principal matters, by none but precepts divine only; partly, and as concerning things of inferior regard, by ordinances as well human as divine: about the substance of religion wherein God’s only law must be kept there is here no controversy; the crime now intended against us is, that our laws have not ordered those inferior things as behoveth, and that our customs are either superstitious, or otherwise amiss, whether we respect the exercise of public duties in religion, or the functions of persons authorized thereunto.

\section*{Four general propositions demanding that which may reasonably be granted, concerning matters of outward form in the exercise of true Religion. And, fifthly, of a rule not safe nor reasonable in these cases.}
V. It is with teachers of mathematical sciences usual, for us in this present question necessary, to lay down first certain reasonable demands, which in most particulars following are to serve as principles whereby to work, and therefore must be beforehand considered. The men whom we labour to  inform in the truth perceive that so to proceed is requisite. For to this end they also propose touching customs and rites indifferent their general axioms, some of them subject unto just exceptions, and, as we think, more meet by them to be farther considered, than assented unto by us. As that, “In outward things belonging to the service of God, reformed churches ought by all means to shun conformity with the church of Rome;” that, “the first reformed should be a pattern whereunto all that come after ought to conform themselves;” that, “sound religion may not use the things which being not commanded of God have been either devised or abused unto superstition.” These and the rest of the same consort we have in the book going before examined.
Other canons they allege and rules not unworthy of approbation; as that, “In all such things the glory of God, and the edification or ghostly good of his people, must be sought;” “That nothing should be undecently or unorderly done.” But forasmuch as all the difficulty is in discerning what things do glorify God and edify his Church, what not; when we should think them decent and fit, when otherwise: because these rules being too general, come not near enough unto the matter which we have in hand; and the former principles being nearer the purpose, are too far from truth; we must propose unto all men certain petitions incident and very material in causes of this nature, such as no man of moderate judgment hath cause to think unjust or unreasonable.

\section*{The first proposition touching judgment what things are convenient in the outward public ordering of church affairs.}
VI. The first thing therefore which is of force to cause approbation with good conscience towards such customs or rites as publicly are established, is when there riseth from the due consideration of those customs and rites in themselves apparent reason, although not always to prove them better than any other that might possibly be devised, (for who did ever require this in man’s ordinances?) yet competent to shew their conveniency and fitness, in regard of the use for which they should serve.
Now touching the nature of religious services, and the manner of their due performance, thus much generally we know to be most clear; that whereas the greatness and dignity  of all manner actions is measured by the worthiness of the subject from which they proceed, and of the object whereabout they are conversant, we must of necessity in both respects acknowledge, that this present world affordeth not any thing comparable unto the public duties of religion. For if the best things have the perfectest and best operations, it will follow, that seeing man is the worthiest creature upon earth, and every society of men more worthy than any man, and of societies that most excellent which we call the Church; there can be in this world no work performed equal to the exercise of true religion, the proper operation of the Church of God.
Again, forasmuch as religion worketh upon him who in majesty and power is infinite, as we ought we account not of it, unless we esteem it even according to that very height of excellency which our hearts conceive when divine sublimity itself is rightly considered. In the powers and faculties of our souls God requireth the uttermost which our unfeigned affection towards him is able to yield. So that if we affect him not far above and before all things, our religion hath not that inward perfection which it should have, neither do we indeed worship him as our God.
That which inwardly each man should be, the Church outwardly ought to testify. And therefore the duties of our religion which are seen must be such as that affection which is unseen ought to be. Signs must resemble the things they signify. If religion bear the greatest sway in our hearts, our outward religious duties must shew it as far as the Church hath outward ability. Duties of religion performed by whole societies of men, ought to have in them according to our power a sensible excellency, correspondent to the majesty of him whom we worship. Yea then are the public duties of religion best ordered, when the militant Church doth resemble by sensible means, as it may in such cases, that hidden  dignity and glory wherewith the Church triumphant in heaven is beautified.
Howbeit, even as the very heat of the sun itself, which is the life of the whole world, was to the people of God in the desert a grievous annoyance, for ease whereof his extraordinary providence ordained a cloudy pillar to overshadow them: so things of general use and benefit (for in this world what is so perfect that no inconvenience doth ever follow it?) may by some accident be incommodious to a few. In which case, for such private evils remedies there are of like condition, though public ordinances, wherein the common good is respected, be not stirred.
Let our first demand be therefore, that in the external form of religion such things as are apparently, or can be sufficiently proved, effectual and generally fit to set forward godliness, either as betokening the greatness of God, or as beseeming the dignity of religion, or as concurring with celestial impressions in the minds of men, may be reverently thought of; some few, rare, casual, and tolerable, or otherwise curable inconveniences notwithstanding.

\section*{The second proposition.}
VII. Neither may we in this case lightly esteem what hath been allowed as fit in the judgment of antiquity, and by the long continued practice of the whole Church; from which unnecessarily to swerve, experience hath never as yet found it safe. For wisdom’s sake we reverence them no less that are young, or not much less, than if they were stricken in years. And therefore of such it is rightly said that their ripeness of understanding is “grey hair,” and their virtues “old age.” But because wisdom and youth are seldom joined in one, and the ordinary course of the world is more according to Job’s observation, who giveth men advice to seek “wisdom amongst the ancient, and in the length of days, understanding;” therefore if the comparison do stand between man and man, which shall hearken unto other; sith the aged for the most part are best experienced, least subject to rash and unadvised passions, it hath been ever judged reasonable that their sentence in matter of counsel should be better trusted, and more relied upon than other men’s. The goodness of God having furnished man with two chief instruments both necessary for this life, hands to execute and a mind to devise great things; the one is not profitable longer than the vigour of youth doth strengthen it, nor the other greatly till age and experience have brought it to perfection. In whom therefore time hath not perfected knowledge, such must be contented to follow them in whom it hath. For this cause none is more attentively heard than they whose speeches are as David’s were, “I have been young and now am old,” much I have seen and observed in the world. Sharp and subtile discourses of wit procure many times very great applause, but being laid in the balance with that which the habit of sound experience plainly delivereth, they are overweighed. God may endue men extraordinarily with understanding as it pleaseth him. But let no man presuming thereupon neglect the instructions, or despise the ordinances of his elders, sith He whose gift wisdom is hath said, “Ask thy father and he will shew thee; thine ancients and they shall tell thee.”
It is therefore the voice both of God and nature, not of learning only, that especially in matters of action and policy, “The sentences and judgments of men experienced, aged and wise, yea though they speak without any proof or demonstration, are no less to be hearkened unto, than as being demonstrations in themselves; because such men’s long observation is as an eye, wherewith they presently and plainly behold those principles which sway over all actions.” Whereby we are taught both the cause wherefore wise men’s judgments should be credited, and the mean how to use their judgments to the increase of our own wisdom. That which sheweth them to be wise, is the gathering of principles out of their own particular experiments. And the framing of our particular experiments according to the rule of their principles shall make us such as they are.
If therefore even at the first so great account should be made of wise men’s counsels touching things that are publicly done, as time shall add thereunto continuance and approbation  of succeeding ages, their credit and authority must needs be greater. They which do nothing but that which men of account did before them, are, although they do amiss, yet the less faulty, because they are not the authors of harm. And doing well, their actions are freed from prejudice of novelty. To the best and wisest, while they live, the world is continually a froward opposite, a curious observer of their defects and imperfections; their virtues it afterwards as much admireth. And for this cause many times that which most deserveth approbation would hardly be able to find favour, if they which propose it were not content to profess themselves therein scholars and followers of the ancient. For the world will not endure to hear that we are wiser than any have been which went before. In which consideration there is cause why we should be slow and unwilling to change, without very urgent necessity, the ancient ordinances, rites, and long approved customs, of our venerable predecessors. The love of things ancient doth argue stayedness, but levity and want of experience maketh apt unto innovations. That which wisdom did first begin, and hath been with good men long continued, challengeth allowance of them that succeed, although it plead for itself nothing. That which is new, if it promise not much, doth fear condemnation before trial; till trial, no man doth acquit or trust it, what good soever it pretend and promise. So that in this kind there are few things known to be good, till such time as they grow to be ancient. The vain pretence of those glorious names, where they could not be with any truth, neither in reason ought to have been so much alleged, hath wrought such a prejudice against them in the minds of the common sort, as if they had utterly no force at all; whereas (especially for these observances which concern our present question) antiquity, custom, and consent in the Church of God, making with that which law doth establish, are themselves most sufficient reasons to uphold the  same, unless some notable public inconvenience enforce the contrary. For a small thing in the eye of law is as nothing.
We are therefore bold to make our second petition this, That in things the fitness whereof is not of itself apparent, nor easy to be made sufficiently manifest unto all, yet the judgment of antiquity concurring with that which is received may induce them to think it not unfit, who are not able to allege any known weighty inconvenience which it hath, or to take any strong exception against it.

\section*{The third proposition.}
VIII. All things cannot be of ancient continuance, which are expedient and needful for the ordering of spiritual affairs: but the Church being a body which dieth not hath always power, as occasion requireth, no less to ordain that which never was, than to ratify what hath been before. To prescribe the order of doing in all things, is a peculiar prerogative which Wisdom hath, as queen or sovereign commandress over other virtues. This in every several man’s actions of common life appertaineth unto Moral, in public and politic secular affairs unto Civil wisdom. In like manner, to devise any certain form for the outward administration of public duties in the service of God, or things belonging thereunto, and to find out the most convenient for that use, is a point of wisdom Ecclesiastical.
It is not for a man which doth know or should know what order is, and what peaceable government requireth, to ask, “why we should hang our judgment upon the Church’s sleeve;” and “why in matters of order, more than in matters of doctrine.” The Church hath authority to establish that for an order at one time, which at another time it may abolish, and in both do well. But that which in doctrine the Church doth now deliver rightly as a truth, no man will say that it may hereafter recall, and as rightly avouch the contrary. Laws touching matter of order are changeable, by the power of the Church; articles concerning doctrine not so. We read often in the writings of catholic  and holy men touching matters of doctrine, “this we believe, this we hold, this the Prophets and Evangelists have declared, this the Apostles have delivered, this Martyrs have sealed with their blood, and confessed in the midst of torments, to this we cleave as to the anchor of our souls, against this, though an Angel from heaven should preach unto us, we would not believe.” But did we ever in any of them read, touching matters of mere comeliness, order, and decency, neither commanded nor prohibited by any Prophet, any Evangelist, any Apostle, “Although the church wherein we live, do ordain them to be kept, although they be never so generally observed, though all the churches in the world should command them, though Angels from heaven should require our subjection thereunto, I would hold him accursed that doth obey?” Be it in matter of the one kind or of the other, what Scripture doth plainly deliver, to that the first place both of credit and obedience is due; the next whereunto is whatsoever any man can necessarily conclude by force of reason; after these the voice of the Church succeedeth. That which the Church by her ecclesiastical authority shall probably think and define to be true or good, must in congruity of reason overrule all other inferior judgments whatsoever.
To them which ask why we thus hang our judgment on the Church’s sleeve, I answer with Salomon, because “two are better than one.” “Yea simply (saith Basil) and universally, whether it be in works of Nature, or of voluntary choice and counsel, I see not any thing done as it should be, if it be wrought by an agent singling itself from consorts.” The Jews have a sentence of good advice, “Take not upon thee to be a judge alone; there is no sole judge but one only; say not to others, Receive my sentence, when their authority is above thine.” The bare consent of the whole Church should itself in these things stop their mouths,  who living under it, dare presume to bark against it. “There is (saith Cassianus) no place of audience left for them, by whom obedience is not yielded to that which all have agreed upon.” Might we not think it more than wonderful, that nature should in all communities appoint a predominant judgment to sway and overrule in so many things; or that God himself should allow so much authority and power unto every poor family for the ordering of all which are in it; and the city of the living God, which is his Church, be able neither to command nor yet to forbid any thing, which the meanest shall in that respect, and for her sole authority’s sake, be bound to obey?
We cannot hide or dissemble that evil, the grievous inconvenience whereof we feel. Our dislike of them, by whom too much heretofore hath been attributed unto the Church, is grown to an error on the contrary hand; so that now from the Church of God too much is derogated. By which removal of one extremity with another, the world seeking to procure a remedy, hath purchased a mere exchange of the evil which before was felt.
Suppose we that the sacred word of God can at their hands receive due honour, by whose incitement the holy ordinances of the Church endure every where open contempt? No; it is not possible they should observe as they ought the one, who from the other withdraw unnecessarily their own or their brethren’s obedience.
Surely the Church of God in this business is neither of capacity, I trust, so weak, nor so unstrengthened, I know, with authority from above, but that her laws may exact obedience at the hands of her own children, and enjoin gainsayers silence, giving them roundly to understand, That where our duty is submission, weak oppositions betoken pride.
We therefore crave thirdly to have it granted, That where neither the evidence of any law divine, nor the strength of any invincible argument otherwise found out by the light of reason, nor any notable public inconvenience, doth make  against that which our own laws ecclesiastical have although but newly instituted for the ordering of these affairs, the very authority of the Church itself, at the least in such cases, may give so much credit to her own laws, as to make their sentence touching fitness and conveniency weightier than any bare and naked conceit to the contrary; especially in them who can owe no less than child-like obedience to her that hath more than motherly power.

\section*{The fourth proposition.}
IX. There are ancient ordinances, laws which on all sides are allowed to be just and good, yea divine and apostolic constitutions, which the church it may be doth not always keep, nor always justly deserve blame in that respect. For in evils that cannot be removed without the manifest danger of greater to succeed in their rooms, wisdom, of necessity, must give place to necessity. All it can do in those cases is to devise how that which must be endured may be mitigated, and the inconveniences thereof countervailed as near as may be; that when the best things are not possible, the best may be made of those that are.
Nature than which there is nothing more constant, nothing more uniform in all her ways, doth notwithstanding stay her hand, yea, and change her course, when that which God by creation did command, he doth at any time by necessity countermand. It hath therefore pleased himself sometime to unloose the very tongues even of dumb creatures, and to teach them to plead this in their own defence, lest the cruelty of man should persist to afflict them for not keeping their wonted course, when some invincible impediment hath hindered.
If we leave Nature and look into Art, the workman hath in his heart a purpose, he carrieth in mind the whole form which his work should have, there wanteth not in him skill and desire to bring his labour to the best effect, only the matter which he hath to work on is unframable. This necessity excuseth him, so that nothing is derogated from his credit, although much of his work’s perfection be found wanting.
Touching actions of common life, there is not any defence more favourably heard than theirs, who allege sincerely for  themselves, that they did as necessity constrained them. For when the mind is rightly ordered and affected as it should be, in case some external impediment crossing well advised desires shall potently draw men to leave what they principally wish, and to take a course which they would not if their choice were free; what necessity forceth men unto, the same in this case it maintaineth, as long as nothing is committed simply in itself evil, nothing absolutely sinful or wicked, nothing repugnant to that immutable law, whereby whatsoever is condemned as evil can never any way be made good. The casting away of things profitable for the sustenance of man’s life, is an unthankful abuse of the fruits of God’s good providence towards mankind. Which consideration for all that did not hinder St. Paul from throwing corn into the sea, when care of saving men’s lives made it necessary to lose that which else had been better saved. Neither was this to do evil, to the end that good might come of it: for of two such evils being not both evitable, the choice of the less is not evil. And evils must be in our construction judged inevitable, if there be no apparent ordinary way to avoid them; because where counsel and advice bear rule, of God’s extraordinary power without extraordinary warrant we cannot presume.
In civil affairs to declare what sway necessity hath ever been accustomed to bear, were labour infinite. The laws of all states and kingdoms in the world have scarcely of any thing more common use. Should then only the Church shew itself inhuman and stern, absolutely urging a rigorous observation of spiritual ordinances, without relaxation or exception what necessity soever happen? We know the contrary practice to have been commended by him, upon the warrant of whose judgment the Church, most of all delighted with merciful and moderate courses, doth the oftener condescend unto like equity, permitting in cases of necessity that which otherwise it disalloweth and forbiddeth.
Cases of necessity being sometime but urgent, sometime extreme, the consideration of public utility is with very good  advice judged at the least equivalent with the easier kind of necessity.
Now that which causeth numbers to storm against some necessary tolerations, which they should rather let pass with silence, considering that in polity as well ecclesiastical as civil, there are and will be always evils which no art of man can cure, breaches and leaks more than man’s wit hath hands to stop; that which maketh odious unto them many things wherein notwithstanding the truth is that very just regard hath been had of the public good; that which in a great part of the weightiest causes belonging to this present controversy hath ensnared the judgments both of sundry good and of some well learned men, is the manifest truth of certain general principles, whereupon the ordinances that serve for usual practice in the Church of God are grounded. Which principles men knowing to be most sound, and that the ordinary practice accordingly framed is good, whatsoever is over and besides that ordinary, the same they judge repugnant to those true principles. The cause of which error is ignorance what restraints and limitations all such principles have, in regard of so manifold varieties as the matter whereunto they are appliable doth commonly afford. These varieties are not known but by much experience, from whence to draw the true bounds of all principles, to discern how far forth they take effect, to see where and why they fail, to apprehend by what degrees and means they lead to the practice of things in show though not in deed repugnant and contrary one to another, requireth more sharpness of wit, more intricate circuitions of discourse, more industry and depth of judgment, than common ability doth yield. So that general rules, till their limits be fully known (especially in matter of public and ecclesiastical affairs), are, by reason of the manifold secret exceptions which lie hidden in them, no other to the eye of man’s understanding than cloudy mists cast before the eye of common sense. They that walk in darkness know not  whither they go. And even as little is their certainty, whose opinions generalities only do guide. With gross and popular capacities nothing doth more prevail than unlimited generalities, because of their plainness at the first sight: nothing less with men of exact judgment, because such rules are not safe to be trusted over far. General laws are like general rules of physic, according whereunto as no wise man will desire himself to be cured, if there be joined with his disease some special accident, in regard whereof that whereby others in the same infirmity but without the like accident recover health, would be to him either hurtful, or at the least unprofitable; so we must not, under a colourable commendation of holy ordinances in the Church, and of reasonable causes whereupon they have been grounded for the common good, imagine that all men’s cases ought to have one measure.
Not without singular wisdom therefore it hath been provided, that as the ordinary course of common affairs is disposed of by general laws, so likewise men’s rarer incident necessities and utilities should be with special equity considered. From hence it is, that so many privileges, immunities, exceptions, and dispensations, have been always with great equity and reason granted; not to turn the edge of justice, or to make void at certain times and in certain men, through mere voluntary grace or benevolence, that which continually and universally should be of force, (as some understand it,) but in very truth to practise general laws according to their right meaning.
We see in contracts and other dealings which daily pass between man and man, that, to the utter undoing of some, many things by strictness of law may be done, which equity and honest meaning forbiddeth. Not that the law is unjust, but unperfect; nor equity against, but above, the law, binding men’s consciences in things which law cannot reach unto. Will any man say, that the virtue of private equity is opposite and repugnant to that law the silence whereof it supplieth in all such private dealing? No more is public equity against the law of public affairs, albeit the one permit unto some in special considerations, that which the other  agreeably with general rules of justice doth in general sort forbid. For sith all good laws are the voices of right reason, which is the instrument wherewith God will have the world guided; and impossible it is that right should withstand right: it must follow that principles and rules of justice, be they never so generally uttered, do no less effectually intend, than if they did plainly express, an exception of all particulars, wherein their literal practice might any way prejudice equity.
And because it is natural unto all men to wish their own extraordinary benefit, when they think they have reasonable inducements so to do; and no man can be presumed a competent judge what equity doth require in his own case: the likeliest mean whereby the wit of man can provide, that he which useth the benefit of any special benignity above the common course of others may enjoy it with good conscience, and not against the true purpose of laws which in outward show are contrary, must needs be to arm with authority some fit both for quality and place, to administer that which in every such particular shall appear agreeable with equity. Wherein, as it cannot be denied but that sometimes the practice of such jurisdiction may swerve through error even in the very best, and for other respects where less integrity is: so the watchfullest observers of inconveniences that way growing, and the readiest to urge them in disgrace of authorized proceedings, do very well know, that the disposition of these things resteth not now in the hands of Popes, who live in no worldly awe or subjection, but is committed to them whom law may at all times bridle, and superior power control; yea to them also in such sort, that law itself hath set down to what persons, in what causes, with what circumstances, almost every faculty or favour shall be granted, leaving in a manner nothing unto them, more than only to deliver what is already given by law. Which maketh it by many degrees less reasonable, that under pretence of inconveniences so easily stopped, if any did grow, and so well prevented that none may, men should be altogether barred of the liberty that law with equity and reason granteth.
These things therefore considered, we lastly require that it may not seem hard, if in cases of necessity, or for common utility’s sake, certain profitable ordinances sometime  be released, rather than all men always strictly bound to the general rigour thereof.

\section*{The rule of men’s private spirits not safe in these cases to be followed.}
X. Now where the word of God leaveth the Church to make choice of her own ordinances, if against those things which have been received with great reason, or against that which the ancient practice of the Church hath continued time out of mind, or against such ordinances as the power and authority of that Church under which we live hath itself devised for the public good, or against the discretion of the Church in mitigating sometimes with favourable equity that rigour which otherwise the literal generality of ecclesiastical laws hath judged to be more convenient and meet; if against all this it should be free for men to reprove, to disgrace, to reject at their own liberty what they see done and practised according to order set down; if in so great variety of ways as the wit of man is easily able to find out towards any purpose, and in so great liking as all men especially have unto those inventions whereby some one shall seem to have been more enlightened from above than many thousands, the Church did give every man license to follow what himself imagineth that “God’s Spirit doth reveal” unto him, or what he supposeth that God is likely to have revealed to some special person whose virtues deserve to be highly esteemed: what other effect could hereupon ensue, but the utter confusion of his Church under pretence of being taught, led, and guided by his Spirit? The gifts and graces whereof do so naturally all tend unto common peace, that where such singularity is, they whose hearts it possesseth ought to suspect it the more, inasmuch as if it did come of God, and should for that cause prevail with others, the same God which revealeth it to them, would also give them power of confirming it unto others, either with miraculous operation, or with strong and invincible remonstrance of sound Reason, such as whereby it might appear that God would indeed have all men’s judgments give place unto it; whereas now the error and unsufficiency of their arguments do make it on the contrary side against them a strong presumption, that God hath not moved their hearts to think such things as he hath not enabled them to prove.
And so from rules of general direction it resteth that  now we descend to a more distinct explication of particulars, wherein those rules have their special efficacy.

\section*{Places for the public service of God.}
XI. Solemn duties of public service to be done unto God, must have their places set and prepared in such sort, as beseemeth actions of that regard. Adam, even during the space of his small continuance in Paradise, had where to present himself before the Lord. Adam’s sons had out of Paradise in like sort whither to bring their sacrifices. The Patriarchs used altars, and mountains, and groves, to the selfsame purpose.
In the vast wilderness when the people of God had themselves no settled habitation, yet a moveable tabernacle they were commanded of God to make. The like charge was given them against the time they should come to settle themselves in the land which had been promised unto their fathers, “Ye shall seek that place which the Lord your God shall choose.” When God had chosen Jerusalem, and in Jerusalem Mount Moria, there to have his standing habitation made, it was in the chiefest of David’s desires to have performed so good a work. His grief was no less that he could not have the honour to build God a temple, than their anger is at this day, who bite asunder their own tongues with very wrath, that they have not as yet the power to pull down the temples which they never built, and to level them with the ground. It was no mean thing which he purposed. To perform a work so majestical and stately was no small charge. Therefore he incited all men unto bountiful contribution, and procured towards it with all his power, gold, silver, brass, iron, wood, precious stones, in great abundance. Yea, moreover, “Because I have (saith David) a joy in the house of my God, I have of mine own gold and silver, besides all that I have prepared for the house of the sanctuary, given to the house of my God three thousand talents of gold, even the gold of Ophir, seven thousand talents of fined silver.” After the overthrow of this first house of God, a second was instead thereof erected; but with so great odds, that they wept which had seen the former, and beheld  how much this later came behind it, the beauty whereof notwithstanding was such, that even this was also the wonder of the whole world. Besides which Temple, there were both in other parts of the land, and even in Jerusalem, by process of time, no small number of synagogues for men to resort unto. Our Saviour himself, and after him the Apostles, frequented both the one and the other.
The Church of Christ which was in Jerusalem, and held that profession which had not the public allowance and countenance of authority, could not so long use the exercise of Christian religion but in private only. So that as Jews they had access to the temple and synagogues, where God was served after the custom of the Law; but for that which they did as Christians, they were of necessity forced other where to assemble themselves. And as God gave increase to his Church, they sought out both there and abroad for that purpose not the fittest (for so the times would not suffer them to do) but the safest places they could. In process of time, some whiles by sufferance, some whiles by special leave and favour, they began to erect themselves oratories; not in any sumptuous or stately manner, which neither was possible by reason of the poor estate of the Church, and had been perilous in regard of the world’s envy towards them. At the length, when it pleased God to raise up kings and emperors favouring sincerely the Christian truth, that which the Church before either could not or durst not do, was with all alacrity performed. Temples were in all places erected. No cost was spared, nothing judged too dear which that way should be spent. The whole world did seem to exult, that it had occasion of pouring out gifts to so blessed a purpose. That cheerful devotion which David this way did exceedingly delight to behold, and wish that the same in the Jewish people might be perpetual, was then in Christian people every where to be seen.
Their actions, till this day always accustomed to be spoken of with great honour, are now called openly into question. They, and as many as have been followers of their example in that thing, we especially that worship God either in temples which their hands made, or which other men  sithence have framed by the like pattern, are in that respect charged no less than with the very sin of idolatry. Our churches, in the foam of that good spirit which directeth such fiery tongues, they term spitefully the temples of Baal, idol synagogues, abominable styes.

\section*{The solemnity of erecting Churches condemned, the hallowing and dedicating of them scorned by the adversary.}
XII. Wherein the first thing which moveth them thus to cast up their poison, are certain solemnities usual at the first erection of churches. Now although the same should be blame-worthy, yet this age thanks be to God hath reasonably well forborne to incur the danger of any such blame. It cannot be laid to many men’s charge at this day living, either that they have been so curious as to trouble bishops with placing the first stone in the Churches they built, or so scrupulous, as after the erection of them to make any great ado for their dedication. In which kind notwithstanding as we do neither allow unmeet, nor purpose the stiff defence of any unnecessary custom heretofore received: so we know  no reason wherefore churches should be the worse, if at the first erecting of them, at the making of them public, at the time when they are delivered as it were in God’s own possession, and when the use whereunto they shall ever serve is established, ceremonies fit to betoken such intents and to accompany such actions be usual, as in the purest times they have been. When Constantine had finished an house for the service of God at Jerusalem, the dedication he judged a matter not unworthy, about the solemn performance whereof the greatest part of the bishops in Christendom should meet together. Which thing they did at the emperor’s motion, each most willingly setting forth that action to their power; some with orations, some with sermons, some with the sacrifice of prayers unto God for the peace of the world, for the Church’s safety, for the emperor’s and his children’s good. By Athanasius the like is recorded concerning a bishop of Alexandria, in a work of the like devout magnificence. So that whether emperors or bishops in those days were church founders, the solemn dedication of churches they thought not to be a work in itself either vain or superstitious. Can we judge it a thing seemly for any man to go about the building of an house to  the God of heaven with no other apparance, than if his end were to rear up a kitchen or a parlour for his own use? Or when a work of such nature is finished, remaineth there nothing but presently to use it, and so an end?
It behoveth that the place where God shall be served by the whole Church, be a public place, for the avoiding of privy conventicles, which covered with pretence of religion may serve unto dangerous practices. Yea, although such assemblies be had indeed for religion’s sake, hurtful nevertheless they may easily prove, as well in regard of their fitness to serve the turn of heretics, and such as privily will soonest adventure to instil their poison into men’s minds; as also for the occasion which thereby is given to malicious persons, both of suspecting and of traducing with more colourable show those actions, which in themselves being holy, should be so ordered that no man might probably otherwise think of them. Which considerations have by so much the greater weight, for that of these inconveniences the Church heretofore had so plain experience, when Christian men were driven to use secret meetings, because the liberty of public places was not granted them. There are which hold, that the presence of a Christian multitude, and the duties of religion performed amongst them, do make the place of their assembly public; even as the presence of the king and his retinue maketh any man’s house a court. But this I take to be an error, inasmuch as the only thing which maketh any place public is the public assignment thereof unto such duties. As for the multitude there assembled, or the duties which they perform, it doth not appear how either should be of force to infuse any such prerogative.
Nor doth the solemn dedication of churches serve only to make them public, but farther also to surrender up that right which otherwise their founders might have in them, and to make God himself their owner. For which cause at the erection and consecration as well of the tabernacle as of the temple, it pleased the Almighty to give a manifest sign that he took possession of both. Finally, it notifieth in solemn manner the holy and religious use whereunto it is intended such houses shall be put.
These things the wisdom of Salomon did not account superfluous. He knew how easily that which was meant should be holy and sacred, might be drawn from the use whereunto it was first provided; he knew how bold men are to take even from God himself; how hardly that house would be kept from impious profanation he knew; and right wisely therefore endeavoured by such solemnities to leave in the minds of men that impression which might somewhat restrain their boldness, and nourish a reverend affection towards the house of God. For which cause when the first house was destroyed, and a new in the stead thereof erected by the children of Israel after their return from captivity, they kept the dedication even of this house also with joy.
The argument which our Saviour useth against profaners of the temple, he taketh from the use whereunto it was with solemnity consecrated. And as the prophet Jeremy forbiddeth the carrying of burdens on the Sabboth, because that was a sanctified day; so because the temple was a place sanctified, our Lord would not suffer no not the carriage of a vessel through the temple. These two commandments therefore are in the Law conjoined, “Ye shall keep my Sabboths, and reverence my sanctuary.”
Out of those the Apostle’s words, “Have ye not houses to eat and drink?”—albeit temples such as now were not then erected for the exercise of the Christian religion, it hath been nevertheless not absurdly conceived that he teacheth  what difference should be made between house and house; that what is fit for the dwelling-place of God, and what for man’s habitation he sheweth; he requireth that Christian men at their own home take common food, and in the house of the Lord none but that food which is heavenly; he instructeth them, that as in the one place they use to refresh their bodies, so they may in the other learn to seek the nourishment of their souls; and as there they sustain temporal life, so here they would learn to make provision for eternal. Christ could not suffer that the temple should serve for a place of mart, nor the Apostle of Christ that the church should be made an inn.
When therefore we sanctify or hallow churches, that which we do is only to testify that we make them places of public resort, that we invest God himself with them, that we sever them from common uses. In which action, other solemnities than such as are decent and fit for that purpose we approve none.
Indeed we condemn not all as unmeet, the like whereunto have been either devised or used haply amongst Idolaters. For why should conformity with them in matter of opinion be lawful when they think that which is true, if in action when they do that which is meet it be not lawful to be like unto them? Are we to forsake any true opinion because idolaters have maintained it? Nor to shun any requisite action only  because we have in the practice thereof been prevented by idolaters. It is no impossible thing but that sometimes they may judge as rightly what is decent about such external affairs of God, as in greater things what is true. Not therefore whatsoever idolaters have either thought or done, but let whatsoever they have either thought or done idolatrously be so far forth abhorred. For of that which is good even in evil things God is author.

\section*{Of the names whereby we distinguish our Churches.}
XIII. Touching the names of Angels and Saints whereby the most of our churches are called; as the custom of so naming them is very ancient, so neither was the cause thereof at the first, nor is the use and continuance with us at this present, hurtful. That churches were consecrated unto none but the Lord only, the very general name itself doth sufficiently shew, inasmuch as by plain grammatical construction, Church doth signify no other thing than the Lord’s house. And because the multitude as of persons so of things particular causeth variety of proper names to be devised for distinction sake, founders of churches did herein that which best liked their own conceit at the present time; yet each intending that as oft as those buildings came to be mentioned, the name should put men in mind of some memorable thing or person. Thus therefore it cometh to pass that all churches have had their names, some as memorials of Peace, some of Wisdom, some in memory of the Trinity itself, some of Christ under sundry titles, of the blessed Virgin not a few, many of one Apostle, Saint or Martyr, many of all.
In which respect their commendable purpose being not of every one understood, they have been in latter ages construed as though they had superstitiously meant, either that those places which were denominated of Angels and Saints should serve for the worship of so glorious creatures, or else those glorified creatures for defence, protection, and  patronage of such places. A thing which the ancient do utterly disclaim. “To them (saith St. Augustine) we appoint no churches, because they are not to us as gods.” Again, “The nations to their gods erected temples, we not temples unto our Martyrs as unto gods, but memorials as unto dead men, whose spirits with God are still living.”
Divers considerations there are, for which Christian churches might first take their names of Saints: as either because by the ministry of Saints it pleased God there to shew some rare effect of his power; or else in regard of death which those saints having suffered for the testimony of Jesus Christ did thereby make the places where they died venerable; or thirdly, for that it liked good and virtuous men to give such occasion of mentioning them often, to the end that the naming of their persons might cause inquiry to be made, and meditation to be had of their virtues. Wherefore seeing that we cannot justly account it superstition to give unto churches those fore-rehearsed names, as memorials either of holy persons or things, if it be plain that their founders did with such meaning name them, shall not we in otherwise taking them offer them injury? Or if it be obscure or uncertain what they meant, yet this construction being more  favourable, charity I hope constraineth no man which standeth doubtful of their minds, to lean to the hardest and worst interpretation that their words can carry.
Yea although it were clear that they all (for the error of some is manifest in this behalf) had therein a superstitious intent, wherefore should their fault prejudice us, who (as all men know) do use but by way of mere distinction the names which they of superstition gave? In the use of those names whereby we distinguish both days and months are we culpable of superstition, because they were, who first invented them? The sign of Castor and Pollux superstitiously given unto that ship wherein the Apostle sailed, polluteth not the Evangelist’s pen, who thereby doth but distinguish that ship from others. If to Daniel there had been given no other name but only Beltshazzar, given him in honour of the Babylonian idol Belti, should their idolatry which were authors of that name cleave unto every man which had so termed him by way of personal difference only? Were it not to satisfy the minds of the simpler sort of men, these nice curiosities are not worthy the labour which we bestow to answer them.


\section*{Of the fashion of our Churches.}
XIV. The like unto this is a fancy which they have against the fashion of our churches, as being framed according to the pattern of the Jewish temple. A fault no less grievous, if so be it were true, than if some king should build his mansion-house by the model of Salomon’s palace. So far forth as our churches and their temple have one end, what should let but that they may lawfully have one form? The temple was for sacrifice, and therefore had rooms to that purpose such as ours have none. Our churches are places provided that the people might there assemble themselves in due and decent manner, according to their several degrees and orders. Which thing being common unto us with Jews,  we have in this respect our churches divided by certain partitions, although not so many in number as theirs. They had their several for heathen nations, their several for the people of their own nation, their several for men, their several for women, their several for the priests, and for the high priest alone their several. There being in ours for local distinction between the clergy and the rest (which yet we do not with any great strictness or curiosity observe neither) but one partition; the cause whereof at the first (as it seemeth) was, that as many as were capable of the holy mysteries might there assemble themselves and no other creep in amongst them: this is now made a matter so heinous, as if our religion thereby were become even plain Judaism, and as though we retained a most holy place, whereinto there might not any but the high priest alone enter, according to the custom of the Jews.

\section*{The sumptuousness of Churches.}
XV. Some it highly displeaseth, that so great expenses this way are employed. “The mother of such magnificence” (they think) “is but only a proud ambitious desire to be spoken of far and wide. Suppose we that God himself delighteth to dwell sumptuously, or taketh pleasure in  chargeable pomp? No; then was the Lord most acceptably served, when his temples were rooms borrowed within the houses of poor men. This was suitable unto the nakedness of Jesus Christ and the simplicity of his Gospel.”
What thoughts or cogitations they had which were authors of those things, the use and benefit whereof hath descended unto ourselves, as we do not know, so we need not search. It cometh we grant many times to pass, that the works of men being the same, their drifts and purposes therein are divers. The charge of Herod about the temple of God was ambitious, yet Salomon’s virtuous, Constantine’s holy. But howsoever their hearts are disposed by whom any such thing is done in the world, shall we think that it baneth the work which they leave behind them, or taketh away from others the use and benefit thereof?
Touching God himself, hath he any where revealed that it is his delight to dwell beggarly? And that he taketh no pleasure to be worshipped saving only in poor cottages? Even then was the Lord as acceptably honoured of his people as ever, when the stateliest places and things in the whole world were sought out to adorn his temple. This most suitable, decent, and fit for the greatness of Jesus Christ, for the sublimity of his gospel; except we think of Christ and his gospel as the officers of Julian did. As therefore the son of Sirach giveth verdict concerning those things which God hath wrought, “A man need not say, ‘this is worse than that, this more acceptable to God, that less;’ for in their season they are all worthy praise:” the like we may also conclude as touching these two so contrary ways of providing in meaner or in costlier sort for the honour of Almighty God, “A man need not say, ‘this is worse than that, this more acceptable to God, that less;’ for with him they are in their season both allowable:” the one when the state of the Church is poor, the other when God hath enriched it with plenty.
When they, which had seen the beauty of the first temple  built by Salomon in the days of his great prosperity and peace, beheld how far it excelled the second which had not builders of like ability, the tears of their grieved eyes the prophets endeavoured with comforts to wipe away. Whereas if the house of God were by so much the more perfect by how much the glory thereof is less, they should have done better to rejoice than weep, their prophets better to reprove than comfort.
It being objected against the Church in the times of universal persecution, that her service done to God was not solemnly performed in temples fit for the honour of divine majesty, their most convenient answer was, that “The best temples which we can dedicate to God, are our sanctified souls and bodies.” Whereby it plainly appeareth how the Fathers, when they were upbraided with that defect, comforted themselves with the meditation of God’s most gracious and merciful nature, who did not therefore the less accept of their hearty affection and zeal, rather than took any great delight, or imagined any high perfection in such their want of external ornaments, which when they wanted, the cause was their only lack of ability; ability serving, they wanted them not. Before the emperor Constantine’s time, under Severus, Gordian, Philip, and Galienus, the state of Christian affairs being tolerable, the former buildings which were but of mean and small estate contented them not, spacious and ample churches they erected throughout every city. No envy was able to be their hinderance, no practice of Satan or fraud of men available against their proceedings herein, while they continued as yet worthy to feel the aid of the arm of God extended over them for their safety. These churches Diocletian caused by  solemn edict to be afterwards overthrown. Maximinus with like authority giving leave to erect them, the hearts of all men were even rapt with divine joy, to see those places, which tyrannous impiety had laid waste, recovered as it were out of mortal calamity, Churches “reared up to an height immeasurable, and adorned with far more beauty in their restoration, than their founders before had given them.” Whereby we see how most Christian minds stood then affected, we see how joyful they were to behold the sumptuous stateliness of houses built unto God’s glory.
If we should, over and besides this, allege the care which was had, that all things about the tabernacle of Moses might be as beautiful, gorgeous, and rich, as art could make them; or what travail and cost was bestowed that the goodliness of the temple might be a spectacle of admiration to all the world: this they will say was figurative, and served by God’s appointment but for a time, to shadow out the true everlasting glory of a more divine sanctuary; whereinto Christ being long sithence entered, it seemeth that all those curious exornations should rather cease. Which thing we also ourselves would grant, if the use thereof had been merely and only mystical. But sith the Prophet David doth mention a natural conveniency which such kind of bounteous expenses have, as well for that we do thereby give unto God a testimony of our cheerful affection which thinketh nothing too dear to be bestowed about the furniture of his service; as also because it serveth to the world for a witness of his almightiness, whom we outwardly honour with the chiefest of outward things, as being of all things himself incomparably the greatest. Besides, were it not also strange, if God should have made such store of glorious creatures on earth, and leave them all to be consumed in secular vanity, allowing none but the baser sort to be employed in his own service? To set forth the  majesty of kings his vicegerents in this world, the most gorgeous and rare treasures which the world hath are procured. We think belike that he will accept what the meanest of them would disdain.
If there be great care to build and beautify these corruptible sanctuaries, little or none that the living temples of the Holy Ghost, the dearly redeemed souls of the people of God, may be edified; huge expenses upon timber and stone, but towards the relief of the poor small devotion; cost this way infinite, and in the meanwhile charity cold: we have in such case just occasion to make complaint as St. Jerome did, “The walls of the Church there are enow contented to build, and to underset it with goodly pillars, the marbles are polished, the roofs shine with gold, the altar hath precious stones to adorn it; and of Christ’s ministers no choice at all.” The same Jerome both in that place and elsewhere debaseth with like intent the glory of such magnificence, (a thing whereunto men’s affection in those times needed no spur,) thereby to extol the necessity sometimes of charity and alms, sometimes of other the most principal duties belonging unto Christian men; which duties were neither so highly esteemed as they ought, and being compared with that in  question, the directest sentence we can give of them both, as unto me it seemeth, is this: “God, who requireth the one as necessary, accepteth the other also as being an honourable work.”

\section*{What holiness and virtue we ascribe to the Church more than other places.}
XVI. Our opinion concerning the force and virtue which such places have is, I trust, without any blemish or stain of heresy. Churches receive as every thing else their chief perfection from the end whereunto they serve. Which end being the public worship of God, they are in this consideration houses of greater dignity than any provided for meaner purposes. For which cause they seem after a sort even to mourn, as being injuried and defrauded of their right, when places not sanctified as they are prevent them unnecessarily in that preeminence and honour. Whereby also it doth come to pass, that the service of God hath not then itself such perfection of grace and comeliness, as when the dignity of place which it wisheth for doth concur.
Again, albeit the true worship of God be to God in itself acceptable, who respecteth not so much in what place, as with what affection he is served; and therefore Moses in the midst of the sea, Job on the dunghill, Ezechias in bed, Jeremy in mire, Jonas in the whale, Daniel in the den, the children in the furnace, the thief on the cross, Peter and Paul in prison, calling unto God were heard, as St. Basil noteth: manifest notwithstanding it is, that the very majesty and holiness of the place, where God is worshipped, hath in regard of us great virtue, force, and efficacy, for that it serveth as a sensible help to stir up devotion, and in that respect no doubt bettereth even our holiest and best actions in this kind. As therefore we every where exhort all men to  worship God, even so for performance of this service by the people of God assembled, we think not any place so good as the church, neither any exhortation so fit as that of David, “O worship the Lord in the beauty of holiness.”

\section*{Their pretence that would have Churches utterly razed.}
XVII. For of our churches thus it becometh us to esteem, howsoever others rapt with the pang of a furious zeal do pour out against them devout blasphemies, crying “Down with them, down with them, even to the very ground: for to idolatry they have been abused. And the places where idols have been worshipped are by the law of God devote to utter destruction. For execution of which law the kings that were godly, Asa, Jehosaphat, Ezechia, Josiah, destroyed all the high places, altars, groves, which had been erected in Juda and Israel. He that said, ‘Thou shalt have no other gods before my face,’ hath likewise said, ‘Thou shalt utterly deface and destroy all these synagogues and places where such idols have been worshipped.’ This law containeth the temporal punishment which God hath set down, and will that men execute, for the breach of the other law. They which spare them therefore do but reserve, as the hypocrite Saul did, execrable things, to worship God withal.”
The truth is, that as no man serveth God, and loveth him not; so neither can any man sincerely love God, and not extremely abhor that sin, which is the highest degree of treason against the Supreme Guide and Monarch of the whole world, with whose divine authority and power it investeth others. By means whereof the state of idolaters is two ways miserable. First in that which they worship they find no succour; and secondly at his hands whom they ought to serve, there is no other thing to be looked for but the effects of most just displeasure, the withdrawing of grace, dereliction in this world, and in the world to come confusion.  Paul and Barnabas, when infidels admiring their virtues went about to sacrifice unto them, rent their garments in token of horror, and as frighted persons ran crying through the press of the people, “O men, wherefore do ye these things?” They knew the force of that dreadful curse whereunto idolatry maketh subject. Nor is there cause why the guilty sustaining the same should grudge or complain of injustice. For whatsoever evil befalleth in that respect, themselves have made themselves worthy to suffer it.
As for those things either whereon or else wherewith superstition worketh, polluted they are by such abuse, and deprived of that dignity which their nature delighteth in. For there is nothing which doth not grieve and as it were even loathe itself, whensoever iniquity causeth it to serve unto vile purposes. Idolatry therefore maketh whatsoever it toucheth the worse. Howbeit, sith creatures which have no understanding can shew no will; and where no will is, there is no sin; and only that which sinneth is subject to punishment: which way should any such creature be punishable by the law of God? There may be cause sometimes to abolish or to extinguish them; but surely never by way of punishment to the things themselves.
Yea farther howsoever the law of Moses did punish idolaters, we find not that God hath appointed for us any definite or certain temporal judgment, which the Christian magistrate is of necessity for ever bound to execute upon offenders in that kind, much less upon things that way abused as mere instruments. For what God did command touching Canaan, the same concerneth not us any otherwise than only as a fearful pattern of his just displeasure and wrath against sinful nations. It teacheth us how God thought good to plague and afflict them: it doth not appoint in what form and manner we ought to punish the sin of idolatry in all others. Unless they will say, that because the Israelites were commanded to make no covenant with the people of that land, therefore leagues and truces made between superstitious persons and such as serve God aright are unlawful altogether; or because God commanded the Israelites to smite the inhabitants of Canaan, and to root them out, that therefore  reformed churches are bound to put all others to the edge of the sword.
Now whereas commandment was also given to destroy all places where the Canaanites had served their gods, and not to convert any one of them to the honour of the true God; this precept had reference unto a special intent and purpose, which was, that there should be but only one place in the whole land, whereunto the people might bring such offerings, gifts, and sacrifices, as their Levitical law did require. By which law, severe charge was given them in that respect not to convert those places to the worship of the living God, where nations before them had served idols, “but to seek the place which the Lord their God should choose out of all their tribes.”
Besides, it is reason we should likewise consider how great a difference there is between their proceedings, who erect a new commonwealth, which is to have neither people nor law, neither regiment nor religion, the same that was; and theirs who only reform a decayed estate by reducing it to that perfection from which it hath swerved. In this case we are to retain as much, in the other as little, of former things as we may.
Sith therefore examples have not generally the force of laws which all men ought to keep, but of counsels only and persuasions not amiss to be followed by them whose case is the like; surely where cases are so unlike as theirs and ours, I see not how that which they did should induce, much less any way enforce us to the same practice; especially considering that groves and hill altars were, while they did remain, both dangerous in regard of the secret access which people superstitiously given might have always thereunto with ease, neither could they, remaining, serve with any fitness unto better purpose: whereas our temples (their former abuse being by order of law removed) are not only free from such peril, but withal so conveniently framed for the people of God to serve and honour him therein, that no man beholding them can choose but think it exceeding great pity they should be ever any otherwise employed.
“Yea but the cattle of Amalek” (you will say) “were fit  for sacrifice; and this was the very conceit which sometime deceived Saul.” It was so. Nor do I any thing doubt but that Saul upon this conceit might even lawfully have offered to God those reserved spoils, had not the Lord in that particular case given special charge to the contrary.
As therefore notwithstanding the commandment of Israel to destroy Canaanites, idolaters may be converted and live: so the temples which have served idolatry as instruments may be sanctified again and continue, albeit to Israel commandment have been given that they should destroy all idolatrous places in their land, and to the good kings of Israel commendation for fulfilling, to the evil for disobeying the same commandment, sometimes punishment, always sharp and severe reproof, hath even from the Lord himself befallen.
Thus much it may suffice to have written in defence of those Christian oratories, the overthrow and ruin whereof is desired, not now by Infidels, Pagans, or Turks, but by a special refined sect of Christian believers, pretending themselves exceedingly grieved at our solemnities in erecting churches, at the names which we suffer them to hold, at their form and fashion, at the stateliness of them and costliness, at the opinion which we have of them, and at the manifold superstitious abuses whereunto they have been put.

\section*{Of public teaching or preaching, and the first kind thereof, catechising.}
XVIII. Places of public resort being thus provided for, our repair thither is especially for mutual conference, and as it were commerce to be had between God and us.
Because therefore want of the knowledge of God is the cause of all iniquity amongst men, as contrariwise the very ground of all our happiness, and the seed of whatsoever  perfect virtue groweth from us, is a right opinion touching things divine; this kind of knowledge we may justly set down for the first and chiefest thing which God imparteth unto his people, and our duty of receiving this at his merciful hands for the first of those religious offices wherewith we publicly honour him on earth. For the instruction therefore of all sorts of men to eternal life it is necessary, that the sacred and saving truth of God be openly published unto them. Which open publication of heavenly mysteries, is by an excellency termed Preaching. For otherwise there is not any thing publicly notified, but we may in that respect, rightly and properly say it is “preached.” So that when the school of God doth use it as a word of art, we are accordingly to understand it with restraint to such special matter as that school is accustomed to publish.
We find not in the world any people that have lived altogether without religion. And yet this duty of religion, which provideth that publicly all sorts of men may be instructed in the fear of God, is to the Church of God and hath been always so peculiar, that none of the heathens, how curious soever in searching out all kinds of outward ceremonies like to ours, could ever once so much as endeavour to resemble herein the Church’s care for the endless good of her children.
Ways of teaching there have been sundry always usual in God’s Church. For the first introduction of youth to the knowledge of God, the Jews even till this day have their Catechisms. With religion it fareth as with other sciences.  The first delivery of the elements thereof must, for like consideration, be framed according to the weak and slender capacity of young beginners: unto which manner of teaching principles in Christianity, the Apostle in the sixth to the Hebrews is himself understood to allude. For this cause therefore, as the Decalogue of Moses declareth summarily those things which we ought to do; the prayer of our Lord whatsoever we should request or desire: so either by the Apostles, or at the leastwise out of their writings, we have the substance of Christian belief compendiously drawn into few and short articles, to the end that the weakness of no man’s wit might either hinder altogether the knowledge, or excuse the utter ignorance of needful things.
Such as were trained up in these rudiments, and were so made fit to be afterwards by Baptism received into the Church,  the Fathers usually in their writings do term Hearers, as having no farther communion or fellowship with the Church than only this, that they were admitted to hear the principles of Christian faith made plain unto them.
Catechising may be in schools, it may be in private families. But when we make it a kind of preaching, we mean always the public performance thereof in the open hearing of men, because things are preached not in that they are taught, but in that they are published.

\section*{Of preaching by reading publicly the books of Holy Scripture; and concerning supposed untruths in those Translations of Scripture which we allow to be read; as also of the choice which we make in reading.}
XIX. Moses and the Prophets, Christ and his Apostles, were in their times all preachers of God’s truth; some by word, some by writing, some by both. This they did partly as faithful Witnesses, making mere relation what God himself had revealed unto them; and partly as careful Expounders, teachers, persuaders thereof. The Church in like case preacheth still, first publishing by way of Testimony or relation the truth which from them she hath received, even in such sort as it was received, written in the sacred volumes of Scripture; secondly by way of Explication, discovering the mysteries which lie hid therein. The Church as a witness preacheth his mere revealed truth by reading publicly the sacred Scripture. So that a second kind of preaching is the reading of Holy Writ.
For thus we may the boldlier speak, being strengthened with the example of so reverend a prelate as saith, that Moses from the time of ancient generations and ages long since past had amongst the cities of the very Gentiles them that preached him, in that he was read every sabboth day. For so of necessity it must be meant, in as much as we know that the Jews have always had their weekly readings of the Law of Moses; but that they always had in like manner their  weekly sermons upon some part of the Law of Moses we nowhere find.
Howbeit still we must here remember, that the Church by her public reading of the book of God preacheth only as a witness. Now the principal thing required in a witness is fidelity. Wherefore as we cannot excuse that church, which either through corrupt translations of Scripture delivereth instead of divine speeches any thing repugnant unto that which God speaketh; or, through falsified additions, proposeth that to the people of God as Scripture which is in truth no scripture: so the blame, which in both these respects hath been laid upon the church of England, is surely altogether without cause.
Touching translations of holy Scripture, albeit we may not disallow of their painful travails herein, who strictly have tied themselves to the very original letter; yet the judgment of the Church, as we see by the practice of all nations, Greeks, Latins, Persians, Syrians, Æthiopians, Arabians, hath been ever that the fittest for public audience are such as following a middle course between the rigour of literal translators and the liberty of paraphrasts, do with greatest shortness and plainness deliver the meaning of the Holy Ghost. Which being a labour of so great difficulty, the exact performance thereof we may rather wish than look for. So that, except between the words of translation and the mind of the Scripture itself there be contradiction, every little difference should not seem an intolerable blemish necessarily to be spunged out.
Whereas therefore the prophet David in a certain  Psalm doth say concerning Moses and Aaron, that they were obedient to the word of God, and in the selfsame place our allowed translation saith they were not obedient; we are for this cause challenged as manifest gainsayers of scripture, even in that which we read for scripture unto the people. But for as much as words are resemblances of that which the mind of the speaker conceiveth, and conceits are images representing that which is spoken of, it followeth that they who will judge of words, should have recourse to the things themselves from whence they rise.
In setting down that miracle, at the sight whereof Peter fell down astonied before the feet of Jesus, and cried, “Depart, Lord, I am a sinner,” the Evangelist St. Luke saith the store of the fish which they took was such that the net they took it in “brake,” and the ships which they loaded therewith sunk; St. John recording the like miracle saith, that albeit the fishes in number were so many, yet the net with so great a weight was “not broken.” Suppose they had written both of one miracle. Although there be in their words a manifest shew of jar; yet none, if we look upon the difference of matter, with regard whereunto they might both have spoken even of one miracle the very same which they spake of divers, the one intending thereby to signify that the greatness of the burden exceeded the natural ability of the instruments which they had to bear it, the other that the weakness thereof was supported by a supernatural and miraculous addition of strength. The nets as touching themselves brake, but through the power of God they held.
Are not the words of the Prophet Micheas touching Bethleem, “Thou Bethleem the least?” And doth not the very Evangelist translate these words, “Thou Bethleem not the least?” the one regarding the quantity of the place, the other the dignity. Micheas attributeth unto it smallness in respect of circuit; Matthew greatness, in regard of honour  and estimation, by being the native soil of our Lord and Saviour Christ.
Sith therefore speeches which gainsay one another must of necessity be applied both unto one and the same subject; sith they must also the one affirm, the other deny, the selfsame thing: what necessity of contradiction can there be between the letter of the Prophet David, and our authorized translation thereof, if he understanding Moses and Aaron do say they were not disobedient; we applying our speech to Pharao and the Egyptians, do say of them, they were not obedient? Or (which the matter itself will easily enough likewise suffer) if the Egyptians being meant by both, it be said that they, in regard of their offer to let go the people when they saw the fearful darkness, disobeyed not the word of the Lord; and yet that they did not obey his word, inasmuch as the sheep and cattle at the selfsame time they withheld. Of both translations the better I willingly acknowledge that which cometh nearer to the very letter of the original verity; yet so that the other may likewise safely enough be read, without any peril at all of gainsaying as much as the least jot or syllable of God’s most sacred and precious truth.
Which truth as in this we do not violate, so neither is the same gainsayed or crossed, no not in those very preambles placed before certain readings, wherein the steps of the Latin service-book have been somewhat too nearly followed. As when we say Christ spake to his disciples that which the Gospel declareth he spake unto the Pharisees. For doth the Gospel affirm he spake to the Pharisees only? doth it mean that they and besides them no man else was at that time spoken unto by our Saviour Christ? If not, then is there in this diversity no contrariety. I suppose it somewhat probable, that St. John and St. Matthew which have recorded those sermons heard them, and being hearers did think themselves as well respected as the Pharisees, in that which their  Lord and Master taught concerning the pastoral care he had over his own flock, and his offer of grace made to the whole world; which things are the matter whereof he treateth in those sermons. Wherefore as yet there is nothing found, wherein we read for the word of God that which may be condemned as repugnant unto his word.
Furthermore somewhat they are displeased in that we follow not the method of reading which in their judgment is most commendable, the method used in some foreign churches, where Scriptures are read before the time of divine service, and without either choice or stint appointed by any determinate order. Nevertheless, till such time as they shall vouchsafe us some just and sufficient reason to the contrary, we must by their patience, if not allowance, retain the ancient received custom which we now observe. For with us the reading of Scripture in the church is a part of our church liturgy, a special portion of the service which we do to God, and not an exercise to spend the time, when one doth wait for another’s coming, till the assembly of them that shall afterwards worship him be complete. Wherefore as the form of our public service is not voluntary, so neither are the parts thereof left uncertain, but they are all set down in such order, and with such choice, as hath in the wisdom of the Church seemed best to concur as well with the special occasions, as with the general purpose which we have to glorify God.

\section*{Of preaching by the public reading of other profitable instructions; and concerning books Apocryphal.}
XX. Other public readings there are of books and writings not canonical, whereby the Church doth also preach, or openly  make known the doctrine of virtuous conversation; whereupon besides those things in regard whereof we are thought to read the Scriptures of God amiss, it is thought amiss that we read in our churches any thing at all besides the Scriptures. To exclude the reading of any such profitable instruction as the Church hath devised for the better understanding of Scripture, or for the easier training up of the people in holiness and righteousness of life, they plead that God in the Law would have nothing brought into the temple, neither besoms, nor flesh-hooks, nor trumpets, but those only which were sanctified; that for the expounding of darker places we ought to follow the Jews’ polity, who under Antiochus, where they  had not the commodity of sermons, appointed always at their meeting somewhat out of the Prophets to be read together with the Law, and so by the one made the other plainer to be understood; that before and after our Saviour’s coming they neither read Onkelos nor Jonathan’s paraphrase, though having both, but contented themselves with the reading only of scriptures; that if in the primitive Church there had been any thing read besides the monuments of the Prophets and Apostles, Justin Martyr and Origen who mention these would have spoken of the other likewise; that the most ancient and best councils forbid any thing to be read in churches saving canonical Scripture only; that when other things were afterwards permitted, fault was found with it, it succeeded but ill, the Bible itself was thereby in time quite and clean thrust out.
Which arguments, if they be only brought in token of the author’s good will and meaning towards the cause which they would set forward, must accordingly be accepted of by them who already are persuaded the same way. But if their drift and purpose be to persuade others, it would be demanded,  by what rule the legal hallowing of besoms and flesh-hooks must needs exclude all other readings in the church save Scripture. Things sanctified were thereby in such sort appropriated unto God, as that they might never afterwards again be made common. For which cause the Lord, to sign and mark them as his own, appointed oil of holy ointment, the like whereunto it was not lawful to make for ordinary and daily uses. Thus the anointing of Aaron and his sons tied them to the office of the priesthood for ever; the anointing, not of those silver trumpets (which Moses as well for secular as sacred uses was commanded to make, not to sanctify), but the unction of the tabernacle, the table, the laver, the altar of God, with all the instruments appertaining thereunto, this made them for ever holy unto him in whose service they were employed. But what of this? Doth it hereupon follow that all things now in the church “from the greatest to the least” are unholy, which the Lord hath not himself precisely instituted? For so those rudiments they say do import. Then is there nothing holy which the Church by her authority hath appointed, and consequently all positive ordinances that ever were made by ecclesiastical power touching spiritual affairs are profane, they are unholy.
I would not wish them to undertake a work so desperate as to prove, that for the people’s instruction no kind of reading is good, but only that which the Jews devised under Antiochus, although even that be also mistaken. For according to Elias the Levite, (out of whom it doth seem borrowed)  the thing which Antiochus forbade was the public Reading of the Law, and not sermons upon the Law. Neither did the Jews read a portion of the Prophets together with the Law to serve for an interpretation thereof, because Sermons were not permitted them; but instead of the Law which they might not read openly, they read of the Prophets that which in likeness of matter came nearest to each section of their Law. Whereupon when afterwards the liberty of reading the Law was restored, the selfsame custom as touching the Prophets did continue still.
If neither the Jews have used publicly to read their paraphrasts, nor the primitive Church for a long time any other writings than Scripture, except the cause of their not doing it were some law of God or reason forbidding them to do that which we do, why should the later ages of the Church be deprived of the liberty the former had? Are we bound while the world standeth to put nothing in practice but only that which was at the very first?
Concerning the council of Laodicea, as it forbiddeth the reading of those things which are not canonical, so it maketh some things not canonical which are. Their judgment in this we may not, and in that we need not follow.
We have by thus many years’ experience found, that exceeding great good, not encumbered with any notable inconvenience, hath grown by the custom which we now observe. As for the harm whereof judicious men have complained in former times; it came not of this, that other things were read besides the Scripture, but that so evil choice was made. With us there is never any time bestowed in divine service without the reading of a great part of the holy Scripture, which we account a thing most necessary. We dare not admit any such form of liturgy as either appointeth no Scripture at all, or very little, to be read in the church. And therefore the thrusting of the Bible out of the house of God is rather there to be feared, where men esteem it a matter so indifferent, whether the same be by solemn appointment read publicly, or not read, the bare text excepted which the preacher haply chooseth out to expound.
But let us here consider what the practice of our fathers before us hath been, and how far forth the same may be followed. We find that in ancient times there was publicly read first the Scripture, as namely, something out of the books of the Prophets of God which were of old; something out of the Apostles’ writings; and lastly out of the holy Evangelists, some things which touched the person  of our Lord Jesus Christ himself. The cause of their reading first the Old Testament, then the New, and always somewhat out of both, is most likely to have been that which Justin Martyr and St. Augustin observe in comparing the two Testaments. “The Apostles,” saith the one, “have taught us as themselves did learn, first the precepts of the Law, and then the Gospels. For what else is the Law but the Gospel foreshewed? What other the Gospel, than the Law fulfilled?” In like sort the other, “What the Old Testament hath, the very same the New containeth; but that which lieth there as under a shadow is here brought forth into the open sun. Things there prefigured are here performed.” Again, “In the Old Testament there is a close comprehension of the New, in the New an open discovery of the Old.” To be short,  the method of their public readings either purposely did tend, or at the leastwise doth fitly serve, “That from smaller things the mind of the hearers may go forward to the knowledge of greater, and by degrees climb up from the lowest to the highest things.”
Now besides the Scripture, the books which they called Ecclesiastical were thought not unworthy sometime to be brought into public audience, and with that name they entitled the books which we term Apocryphal. Under the selfsame name they also comprised certain no otherwise annexed unto the New than the former unto the Old Testament, as a Book of Hermes, Epistles of Clement, and the like. According therefore to the phrase of antiquity, these we may term the New, and the other the Old Ecclesiastical Books or Writings. For we, being directed by a sentence (I suppose) of St. Jerome, who saith, “that all writings not canonical are apocryphal,” use not now the title “apocryphal” as the rest of the Fathers ordinarily have done, whose custom is so to name for the most part only such as might not publicly be read or divulged. Ruffinus therefore having rehearsed the selfsame books of canonical Scripture, which with us are held to be alone canonical, addeth immediately by way of caution, “We must know that other Books there are also, which our forefathers have used to name not canonical but ecclesiastical books, as the Book of Wisdom, Ecclesiasticus, Toby, Judith, the Maccabees, in the Old Testament; in the New, the Book of Hermes, and such others. All which books and writings  they willed to be read in Churches, but not to be alleged as if their authority did bind us to build upon them our faith. Other writings they named Apocryphal, which they would not have read in churches. These things delivered unto us from the Fathers we have in this place thought good to set down.” So far Ruffinus.
He which considereth notwithstanding what store of false and forged writings dangerous unto Christian belief, and yet bearing glorious inscriptions, began soon upon the Apostles’ times to be admitted into the Church, and to be honoured as if they had been indeed apostolic, shall easily perceive what cause the provincial synod of Laodicea might have as then to prevent especially the danger of books made newly Ecclesiastical, and for fear of the fraud of heretics to provide, that such public readings might be altogether taken out of Canonical scripture. Which ordinance respecting but that abuse that grew through the intermingling of lessons human with sacred, at such time as the one both affected the credit and usurped the name of the other (as by the canon of a later council providing remedy for the selfsame evil, and yet allowing the old ecclesiastical books to be read, it doth more plainly and clearly appear,) neither can be construed nor should be urged utterly to prejudice our use of those old ecclesiastical writings; much less of Homilies, which were a third kind of readings usual in former times, a most commendable  institution, as well then to supply the casual, as now the necessary defect of sermons.
In the heat of general persecution, whereunto Christian belief was subject upon the first promulgation thereof throughout the world, it much confirmed the courage and constancy of weaker minds, when public relation was made unto them after what manner God had been glorified through the sufferings of Martyrs, famous amongst them for holiness during life, and at the time of their death admirable in all men’s eyes, through miraculous evidence of grace divine assisting them from above. For which cause the virtues of some being thought expedient to be annually had in remembrance above the rest, this brought in a fourth kind of public reading, whereby the lives of such saints and martyrs had at the time of their yearly memorials solemn recognition in the Church of God. The fond imitation of which laudable custom being in later ages resumed, when there was neither the like cause to do as the Fathers before had done, nor any care, conscience, or wit, in such as undertook to perform that work, some brainless men have by great labour and travail brought to pass, that the Church is now ashamed of nothing more than of saints. If therefore Pope Gelasius did so long sithence see  those defects of judgment even then, for which the reading of the acts of Martyrs should be and was at that time forborne in the church of Rome; we are not to marvel that afterwards legends being grown in a manner to be nothing else but heaps of frivolous and scandalous vanities, they have been even with disdain thrown out, the very nests which bred them abhorring them. We are not therefore to except only Scripture, and to make confusedly all the residue of one suit, as if they who abolish legends could not without incongruity retain in the church either Homilies or those old Ecclesiastical books.
Which books in case myself did think, as some others do, safer and better to be left publicly unread; nevertheless as in other things of like nature, even so in this, my private judgment I should be loth to oppose against the force of their reverend authority, who rather considering the divine excellency of some things in all, and of all things in certain of those Apocrypha which we publicly read, have thought it  better to let them stand as a list or marginal border unto the Old Testament, and though with divine yet as human compositions, to grant at the least unto certain of them public audience in the house of God. For inasmuch as the due estimation of heavenly truth dependeth wholly upon the known and approved authority of those famous oracles of God, it greatly behoveth the Church to have always most especial care, lest through confused mixture at any time human usurp the room and title of divine writings. Wherefore albeit for the people’s more plain instruction (as the ancient use hath been) we read in our churches certain books besides the Scripture, yet as the Scripture we read them not. All men know our professed opinion touching the difference whereby we sever them from the Scripture. And if any where it be suspected that some or other will haply mistake a thing so manifest in every man’s eye, there is no let but that as often as those books are read, and need so requireth, the style of their difference may expressly be mentioned, to bar even all possibility of error.
It being then known that we hold not the Apocrypha for sacred (as we do the holy Scripture) but for human compositions, the subject whereof are sundry divine matters; let there be reason shewed why to read any part of them publicly it should be unlawful or hurtful unto the Church of God. I  hear it said that “many things” in them are very “frivolous,” and unworthy of public audience; yea many contrary, “plainly contrary to the holy Scripture.” Which hitherto is neither sufficiently proved by him who saith it, and if the proofs thereof were strong, yet the very allegation itself is weak. Let us therefore suppose (for I will not demand to what purpose it is that against our custom of reading books not canonical they bring exceptions of matter in those books which we never use to read) suppose I say that what faults soever they have observed throughout the passages of all those books, the same in every respect were such as neither could be construed, nor ought to be censured otherwise than even as themselves pretend: yet as men through too much haste oftentimes forget the errand whereabout they should go; so here it appeareth that an eager desire to rake together whatsoever might prejudice or any way hinder the credit of apocryphal books, hath caused the collector’s pen so to run as it were on wheels, that the mind which should guide it had no leisure to think, whether that which might haply serve to withhold from giving them the authority which belongeth unto sacred Scripture, and to cut them off from the canon, would as effectually serve to shut them altogether out of the church, and to withdraw from granting unto them that public use wherein they are only held as profitable for instruction. Is it not acknowledged that those  books are “holy,” that they are “ecclesiastical” and “sacred,” that to term them “divine,” as being for their excellency next unto them which are properly so termed, is no way to honour them above desert; yea even that the whole Church of Christ as well at the first as sithence hath most worthily approved their fitness for the public information of life and manners; is not thus much I say acknowledged, and that by them, who notwithstanding receive not the same for “any part of canonical Scripture,” by them who deny not but that they are “faulty,” by them who are ready enough to give instances wherein they seem to contain matter “scarce agreeable with holy Scripture?” So little doth such their supposed faultiness in moderate men’s judgment enforce the removal of them out of the house of God, that still they are judged to retain worthily those very titles of commendation, than which there cannot greater be given to writings the authors whereof are men. As in truth if the Scripture itself ascribing to the persons of men righteousness in regard of their manifold virtues, may not rightly be construed as though it did thereby clear them and make them quite free from all faults, no reason we should judge it absurd to commend their writings as reverend, holy, and sound, wherein there are so many singular perfections, only for that the exquisite wits of some few peradventure are able dispersedly here and there to find now a word and then a sentence, which may be more probably suspected than easily cleared of error, by us which have but conjectural knowledge of their meaning.
Against immodest invectives therefore whereby they are charged as being fraught with outrageous lies, we doubt not but their more allowable censure will prevail, who without so passionate terms of disgrace, do note a difference great enough between Apocryphal and other writings, a difference such as Josephus and Epiphanius observe: the one declaring that amongst the Jews books written after the days of Artaxerxes were not of equal credit with them which had gone before, inasmuch as the Jews sithence that time had not the like exact succession of Prophets; the other acknowledging that  they are “profitable,” although denying them to be “divine” in such construction and sense as the Scripture itself is so termed. With what intent they were first published, those words of the nephew of Jesus do plainly enough signify, “After that my grandfather Jesus had given himself to the reading of the Law and the Prophets and other books of our fathers, and had gotten therein sufficient judgment, he purposed also to write something pertaining to learning and wisdom, to the intent that they which were desirous to learn, and would give themselves to these things, might profit much more in living according to the Law.” Their end in writing and ours in reading them is the same. The books of Judith, Toby, Baruch, Wisdom, and Ecclesiasticus, we read, as serving most unto that end. The rest we leave unto men in private.
Neither can it be reasonably thought, because upon certain solemn occasions some lessons are chosen out of those books, and of Scripture itself some chapters not appointed to be read at all, that we thereby do offer disgrace to the word of God, or lift up the writings of men above it. For in such choice we do not think but that Fitness of speech may be more respected than Worthiness. If in that which we use to read there happen by the way any clause, sentence, or speech, that soundeth towards error, should the mixture of a little dross constrain the Church to deprive herself of so much gold, rather than learn how by art and judgment to make separation of the one from the other? To this effect very fitly, from the counsel that St. Jerome giveth Læta, of taking heed how she read the Apocrypha, as also by the help of other learned men’s judgments delivered in like case, we may take direction. But surely the arguments that should bind us not to read them or any part of them publicly at all must be stronger than as yet we have heard any.

\section*{Of preaching by Sermons, and whether Sermons be the only ordinary way of teaching whereby men are brought to the saving knowledge of God’s truth.}
XXI. We marvel the less that our reading of books not canonical is so much impugned, when so little is attributed unto the reading of canonical Scripture itself, that now it hath grown to be a question, whether the word of God be any ordinary mean to save the souls of men, in that it is either privately studied or publicly read and so made known, or else only as the same is preached, that is to say, explained by lively voice, and applied to the people’s use as the speaker in his wisdom thinketh meet. For this alone is it which they use to call Preaching. The public reading of the Apocrypha they condemn altogether as a thing effectual unto evil; the bare reading in like sort of whatsoever, yea even of Scriptures themselves, they mislike, as a thing uneffectual to do that good, which we are persuaded may grow by it.
Our desire is in this present controversy, as in the rest, not to be carried up and down with the waves of uncertain arguments, but rather positively to lead on the minds of the simpler sort by plain and easy degrees, till the very nature of the thing itself do make manifest what is truth. First therefore because whatsoever is spoken concerning the efficacy or necessity of God’s Word, the same they tie and restrain only unto Sermons, howbeit not Sermons read neither (for such they also abhor in the church) but sermons without book, sermons which spend their life in their birth and may have public audience but once; for this cause to avoid ambiguities wherewith they often entangle themselves, not marking what doth agree to the word of God in itself, and what in regard of  outward accidents which may befall it, we are to know that the word of God is his heavenly truth touching matters of eternal life revealed and uttered unto men; unto Prophets and Apostles by immediate divine inspiration, from them to us by their books and writings. We therefore have no word of God but the Scripture. Apostolic sermons were unto such as heard them his word, even as properly as to us their writings are. Howbeit not so our own sermons, the expositions which our discourse of wit doth gather and minister out of the word of God. For which cause in this present question, we are when we name the word of God always to mean the Scripture only.
The end of the word of God is to save, and therefore we term it the word of life. The way for all men to be saved is by the knowledge of that truth which the word hath taught. And sith eternal life is a thing of itself communicable unto all, it behoveth that the word of God, the necessary mean thereunto, be so likewise. Wherefore the word of life hath been always a treasure, though precious, yet easy, as well to attain, as to find; lest any man desirous of life should perish through the difficulty of the way. To this end the word of God no otherwise serveth than only in the nature of a doctrinal instrument. It saveth because it maketh “wise to salvation.” Wherefore the ignorant it saveth not; they which live by the word must know it. And being itself the instrument which God hath purposely framed, thereby to work the knowledge of salvation in the hearts of men, what cause is there wherefore it should not of itself be acknowledged a most apt and a likely mean to leave an Apprehension of things divine in our understanding, and in the mind an Assent thereunto? For touching the one, sith God, who knoweth and discloseth best the rich treasures of his own wisdom, hath by delivering his word made choice of the Scriptures as the most effectual means whereby those treasures might be imparted unto the world, it followeth that to man’s understanding the Scripture must needs be even of itself intended as a full and perfect discovery, sufficient to imprint in us the lively character of all things necessarily required for the attainment of eternal life. And concerning our Assent to the mysteries of heavenly truth,  seeing that the word of God for the Author’s sake hath credit with all that confess it (as we all do) to be his word, every proposition of holy Scripture, every sentence being to us a principle; if the principles of all kinds of knowledge else have that virtue in themselves, whereby they are able to procure our assent unto such conclusions as the industry of right discourse doth gather from them; we have no reason to think the principles of that truth which tendeth unto man’s everlasting happiness less forcible than any other, when we know that of all other they are for their certainty the most infallible.
But as every thing of price, so this doth require travail. We bring not the knowledge of God with us into the world. And the less our own opportunity or ability is that way, the more we need the help of other men’s judgments to be our direction herein. Nor doth any man ever believe, into whom the doctrine of belief is not instilled by instruction some way received at the first from others. Wherein whatsoever fit means there are to notify the mysteries of the word of God, whether publicly (which we call Preaching) or in private howsoever, the word by every such mean even “ordinarily” doth save, and not only by being delivered unto men in Sermons.
Sermons are not the only preaching which doth save souls. For concerning the use and sense of this word Preaching, which they shut up in so close a prison, although more than enough have already been spoken to redeem the liberty thereof, yet because they insist so much and so proudly insult thereon, we must a little inure their ears with hearing how others whom they more regard are in this case accustomed to use the selfsame language with us whose manner of speech they deride. Justin Martyr doubteth not to tell the Grecians, that even in certain of their writings the very judgment to come is preached; nor the council of Vaus to insinuate that presbyters absent through infirmity from their churches might be said to preach by those deputies who in their stead did but  read Homilies; nor the council of Toledo to call the usual public reading of the Gospels in the church Preaching; nor others long before these our days to write, that by him who but readeth a lesson in the solemn assembly as part of divine service, the very office of Preaching is so far forth executed. Such kind of speeches were then familiar, those phrases seemed not to them absurd, they would have marvelled to hear the outcries which we do, because we think that the Apostles in writing, and others in reading to the church those books which the Apostles wrote, are neither untruly nor unfitly said “to preach.” For although men’s tongues and their pens differ, yet to one and the selfsame general if not particular effect, they may both serve. It is no good argument, St. Paul could not “write with his tongue,” therefore neither could he “preach with his pen.” For Preaching is a general end whereunto writing and speaking do both serve. Men speak not with the instruments of writing, neither write with the instruments of speech, and yet things recorded with the one and uttered with the other may be preached well enough with both. By their patience therefore be it spoken, the Apostles preached as well when they wrote as when they spake the Gospel of Christ, and our usual public Reading of the word of God for the people’s instruction is Preaching.
Nor about words would we ever contend, were not their purpose in so restraining the same injurious to God’s most sacred Word and Spirit. It is on both sides confessed that the word of God outwardly administered (his Spirit inwardly concurring therewith) converteth, edifieth, and saveth souls. Now whereas the external administration of his word is as well by reading barely the Scripture, as by explaining the same when sermons thereon be made; in the one they deny that the finger of God hath ordinarily certain principal operations, which we most steadfastly hold and believe that it hath in both.

\section*{What they attribute to Sermons only, and what we to reading also.}
XXII. So worthy a part of divine service we should greatly wrong, if we did not esteem Preaching as the blessed ordinance of God, sermons as keys to the kingdom of heaven, as wings to the soul, as spurs to the good affections of man, unto the sound and healthy as food, as physic unto diseased minds. Wherefore how highly soever it may please them with words of truth to extol sermons, they shall not herein offend us. We seek not to derogate from any thing which they can justly esteem, but our desire is to uphold the just estimation of that from which it seemeth unto us they derogate more than becometh them. That which offendeth us is first the great disgrace which they offer unto our custom of bare reading the word of God, and to his gracious Spirit, the principal virtue whereof thereby manifesting itself for the endless good of men’s souls, even the virtue which it hath to convert, to edify, to save souls, this they mightily strive to obscure; and secondly the shifts wherewith they maintain their opinion of sermons, whereunto while they labour to appropriate the saving power of the Holy Ghost, they separate from all apparent hope of life and salvation thousands whom the goodness of Almighty God doth not exclude.
Touching therefore the use of Scripture, even in that it is openly read, and the inestimable good which the Church of God by that very mean hath reaped; there was, we may very well think, some cause, which moved the Apostle St. Paul to require, that those things which any one church’s  affairs gave particular occasion to write, might for the instruction of all be published, and that by reading.
1. When the very having of the books of God was a matter of no small charge and difficulty, inasmuch as they could not be had otherwise than only in written copies, it was the necessity not of preaching things agreeable with the word, but of reading the word itself at large to the people, which caused churches throughout the world to have public care, that the sacred oracles of God being procured by common charge, might with great sedulity be kept both entire and sincere. If then we admire the providence of God in the same continuance of Scripture, notwithstanding the violent endeavours of infidels to abolish, and the fraudulent of heretics always to deprave the same, shall we set light by that custom of reading, from whence so precious a benefit hath grown?
2. The voice and testimony of the Church acknowledging Scripture to be the law of the living God, is for the truth and certainty thereof no mean evidence. For if with reason we may presume upon things which a few men’s depositions do testify, suppose we that the minds of men are not both at their first access to the school of Christ exceedingly moved, yea and for ever afterwards also confirmed much, when they consider the main consent of all the churches in the whole world witnessing the sacred authority of scriptures, ever sithence the first publication thereof, even till this present day and hour? And that they all have always so testified, I see not how we should possibly wish a proof more palpable, than this manifest received and every where continued custom of reading them publicly as the Scriptures. The reading therefore of the word of God, as the use hath ever been, in open audience, is the plainest evidence we have of the Church’s Assent and Acknowledgment that it is his word.
3. A further commodity this custom hath, which is to furnish the very simplest and rudest sort with such infallible Axioms and Precepts of sacred truth, delivered even in the very Letter of the Law of God, as may serve them for Rules whereby to judge the better all other doctrines and instructions which they hear. For which end and purpose I see not  how the Scripture could be possibly made familiar unto all, unless far more should be read in the people’s hearing, than by a sermon can be opened. For whereas in a manner the whole book of God is by reading every year published, a small part thereof in comparison of the whole may hold very well the readiest interpreter of Scripture occupied many years.
4. Besides, wherefore should any man think, but that reading itself is one of the “ordinary” means, whereby it pleaseth God of his gracious goodness to instil that celestial verity, which being but so received, is nevertheless effectual to save souls? Thus much therefore we ascribe to the reading of the word of God as the manner is in our churches.
And because it were odious if they on their part should altogether despise the same, they yield that reading may “set forward,” but not begin the work of salvation; that faith may be “nourished” therewith, but not bred; that herein men’s attention to the Scriptures, and their speculation of the creatures of God have like efficacy, both being of power to “augment,” but neither to effect belief without sermons; that if any believe by reading alone, we are to account it a miracle, an “extraordinary” work of God. Wherein that which they grant we gladly accept at their hands, and wish that patiently they would examine how little cause they have to deny that which as yet they grant not.
The Scripture witnesseth that when the book of the  Law of God had been sometime missing, and was after found, the king, which heard it but only read, tare his clothes, and with tears confessed, “Great is the wrath of the Lord upon us, because our fathers have not kept his word to do after all things which are written in this book.” This doth argue, that by bare reading (for of sermons at that time there is no mention) true repentance may be wrought in the hearts of such as fear God, and yet incur his displeasure, the deserved effect whereof is eternal death. So that their repentance (although it be not their first entrance) is notwithstanding the first step of their reentrance into life, and may be in them wrought by the word only read unto them.
Besides, it seemeth that God would have no man stand in doubt but that the reading of Scripture is effectual, as well to lay even the first foundation, as to add degrees of farther perfection in the fear of God. And therefore the Law saith, “Thou shalt read this Law before all Israel, that men, women, and children may hear, yea even that their children which as yet have not known it may hear it, and by hearing it so read, may learn to fear the Lord.”
Our Lord and Saviour was himself of opinion, that they which would not be drawn to amendment of life by the testimony which Moses and the Prophets have given concerning the miseries that follow sinners after death, were not likely to be persuaded by other means, although God from the very dead should have raised them up preachers.
Many hear the books of God and believe them not. Howbeit their unbelief in that case we may not impute unto any weakness or unsufficiency in the mean which is used towards them, but to the wilful bent of their obstinate hearts against it. With minds obdurate nothing prevaileth. As well they that preach, as they that read unto such, shall still have cause to complain with the Prophets which were of old, “Who will give credit unto our teaching?” But with whom ordinary means will prevail, surely the power of the word of God, even without the help of interpreters in God’s Church worketh mightily, not unto their confirmation alone which are converted, but also to their conversion which are not.
It shall not boot them who derogate from reading to excuse it, when they see no other remedy, as if their intent were only to deny that aliens and strangers from the family of God are won, or that belief doth use to be wrought at the first in them, without sermons. For they know it is our custom of simple reading not for conversion of infidels estranged from the house of God, but for instruction of men baptized, bred and brought up in the bosom of the Church, which they despise as a thing uneffectual to save such souls. In such they imagine that God hath no ordinary mean to work faith without sermons.
The reason, why no man can attain belief by the bare contemplation of heaven and earth, is for that they neither are sufficient to give us as much as the least spark of light concerning the very principal mysteries of our faith; and whatsoever we may learn by them, the same we can only attain to know according to the manner of natural sciences, which mere discourse of wit and reason findeth out, whereas the things which we properly believe be only such as are received upon the credit of divine testimony. Seeing therefore that he which considereth the creatures of God findeth therein both these defects, and neither the one nor the other in Scriptures, because he that readeth unto us the Scriptures delivereth all the mysteries of faith, and not any thing amongst them all more than the mouth of the Lord doth warrant: it followeth in those two respects that our consideration of creatures and attention unto Scriptures are not in themselves, and without sermons, things of like disability to breed or beget faith.
Small cause also there is, why any man should greatly wonder as at an extraordinary work, if without sermons reading be found to effect thus much. For I would know by some special instance, what one article of Christian faith, or what duty required necessarily unto all men’s salvation there is, which the very reading of the word of God is not apt to notify. Effects are miraculous and strange when they grow by unlikely means. But did we ever hear it accounted for a wonder, that he which doth read, should believe and live according to the will of Almighty God? Reading doth convey to the  mind that truth without addition or diminution, which Scripture hath derived from the Holy Ghost. And the end of all Scripture is the same which St. John proposeth in the writing of that most divine Gospel, namely Faith, and through faith Salvation. Yea all Scripture is to this effect in itself available, as they which wrote it were persuaded; unless we suppose that the Evangelist or others in speaking of their own intent to instruct and to save by writing, had a secret conceit which they never opened unto any, a conceit that no man in the world should ever be that way the better for any sentence by them written, till such time as the same might chance to be preached upon or alleged at the least in a sermon. Otherwise if he which writeth do that which is forcible in itself, how should he which readeth be thought to do that which in itself is of no force to work belief and to save believers?
Now although we have very just cause to stand in some jealousy and fear, lest by thus overvaluing their sermons, they make the price and estimation of Scripture otherwise notified to fall; nevertheless so impatient they are, that being but requested to let us know what causes they leave for men’s encouragement to attend to the reading of the Scripture, if sermons only be the power of God to save every one which believeth; that which we move for our better learning and instruction’s sake, turneth unto anger and choler in them, they grow altogether out of quietness with it, they answer fumingly that they are “ashamed to defile their pens with making answer to such idle questions:” yet in this their mood they cast forth somewhat, wherewith under pain of greater displeasure we must rest contented. They tell us the profit of reading is singular, in that it serveth for a preparative unto sermons; it helpeth prettily towards the nourishment of faith which sermons have once engendered; it is some stay to his mind which readeth the Scripture, when he findeth the same things there which are taught in sermons, and thereby perceiveth how God doth concur in opinion with the preacher; besides it keepeth sermons in memory, and doth in that respect, although not feed the soul of man, yet help the retentive force of that stomach of the mind which receiveth ghostly  food at the preacher’s hand. But the principal cause of writing the Gospel was, that it might be preached upon or interpreted by public ministers apt and authorized thereunto. Is it credible that a superstitious conceit (for it is no better) concerning sermons should in such sort both darken their eyes and yet sharpen their wits withal, that the only true and weighty cause why Scripture was written, the cause which in Scripture is so often mentioned, the cause which all men have ever till this present day acknowledged, this they should clean exclude as being no cause at all, and load us with so great store of strange concealed causes which did never see light till now? In which number the rest must needs be of moment, when the very chiefest cause of committing the sacred Word of God unto books, is surmised to have been, lest the preacher should want a text whereupon to scholy.
Men of learning hold it for a slip in judgment, when offer is made to demonstrate that as proper to one thing which reason findeth common unto more. Whereas therefore they take from all kinds of teaching that which they attribute to sermons, it had been their part to yield directly some strong reason why between sermons alone and faith there should be ordinarily that coherence which causes have with their usual effects, why a Christian man’s belief should so naturally grow from sermons, and not possibly from any other kind of teaching.
In belief there being but these two operations, apprehension  and assent, do only sermons cause belief, in that no other way is able to explain the mysteries of God, that the mind may rightly apprehend or conceive them as behoveth? We all know that many things are believed, although they be intricate, obscure, and dark, although they exceed the reach and capacity of our wits, yea although in this world they be no way possible to be understood. Many things believed are likewise so plain, that every common person may therein be unto himself a sufficient expounder. Finally, to explain even those things which need and admit explication, many other usual ways there are besides sermons. Therefore sermons are not the only ordinary means whereby we first come to apprehend the mysteries of God.
Is it in regard then of sermons only, that apprehending the Gospel of Christ we yield thereunto our unfeigned assent as to a thing infallibly true? They which rightly consider after what sort the heart of man hereunto is framed, must of necessity acknowledge, that whoso assenteth to the words of eternal life, doth it in regard of his authority whose words they are. This is in man’s conversion unto God τὸ ὅθεν ἡ ἀρχὴ τη̑ς κινήσεως, the first step whereat his race towards heaven beginneth. Unless therefore, clean contrary to our own experience, we shall think it a miracle if any man acknowledge the divine authority of the Scripture, till some sermon have persuaded him thereunto, and that otherwise neither conversation in the bosom of the Church, nor religious education, nor the reading of learned men’s books, nor information received by conference, nor whatsoever pain and diligence in hearing, studying, meditating day and night on the Law, is so far blest of God as to work this effect in any man; how would they have us to grant that faith doth not come but only by hearing sermons?
Fain they would have us to believe the Apostle St. Paul himself to be the author of this their paradox, only because he hath said that “it pleaseth God by the foolishness of preaching to save them which believe;” and again, “How shall they call on him in whom they have not believed? how shall they believe in him of whom they have not heard? how shall they hear without a preacher? how shall men preach except they be sent?”
To answer therefore both allegations at once; the very substance of that they contain is in few but this. Life and salvation God will have offered unto all; his will is that Gentiles should be saved as well as Jews. Salvation belongeth unto none but such “as call upon the name of our Lord Jesus Christ.” Which nations as yet unconverted neither do nor possibly can do till they believe. What they are to believe, impossible it is they should know till they hear it. Their hearing requireth our preaching unto them.
Tertullian, to draw even Paynims themselves unto Christian belief, willeth the books of the Old Testament to be searched, which were at that time in Ptolemy’s library. And if men did not list to travel so far though it were for their endless good, he addeth that in Rome and other places the Jews had synagogues whereunto every one which would might resort, that this kind of liberty they purchased by payment of a standing tribute, that there they did openly read the Scriptures; and whosoever “will hear” saith Tertullian, “he shall find God; whosoever will study to  know, shall be also fain to believe.” But sith there is no likelihood that ever voluntarily they will seek instruction at our hands, it remaineth that unless we will suffer them to perish, salvation itself must seek them, it behoveth God to send them preachers, as he did his elect Apostles throughout the world.
There is a knowledge which God hath always revealed unto them in the works of nature. This they honour and esteem highly as profound wisdom; howbeit this wisdom saveth them not. That which must save believers is the knowledge of the cross of Christ, the only subject of all our preaching. And in their eyes what doth this seem as yet but folly? It pleaseth God by “the foolishness of preaching” to save. These words declare how admirable force those mysteries have which the world doth deride as follies; they shew that the foolishness of the cross of Christ is the wisdom of true believers; they concern the object of our faith, the matter preached of and believed in by Christian men. This we know that the Grecians or Gentiles did account foolishness; but that they ever did think it a fond or unlikely way to seek men’s conversion by sermons we have not heard. Manifest therefore it is that the Apostle applying the name of foolishness in such sort as they did must needs by “the foolishness of preaching” mean the doctrine of Christ, which we learn that we may be saved; but that sermons are the only manner of teaching whereby it pleaseth our Lord to save he could not mean.
In like sort where the same Apostle proveth that as well the sending of the Apostles as their preaching to the Gentiles was necessary, dare we affirm it was ever his meaning, that unto their salvation who even from their tender infancy never knew any other faith or religion than only Christian, no kind of teaching can be available saving that which was so needful for the first universal conversion of Gentiles hating Christianity; neither the sending of any sort allowable in the one case, except only of such as had been in the other also most fit and worthy instruments?
Belief in all sorts doth come by hearkening and attending  to the word of life. Which word sometime proposeth and preacheth itself to the hearer; sometime they deliver it whom privately zeal and piety moveth to be instructors of others by conference; sometime of them it is taught whom the Church hath called to the public either reading thereof or interpreting. All these tend unto one effect; neither doth that which St. Paul or other Apostles teach, concerning the necessity of such teaching as theirs was, or of sending such as they were for that purpose unto the Gentiles, prejudice the efficacy of any other way of public instruction, or enforce the utter disability of any other men’s vocation thought requisite in this Church, for the saving of souls, where means more effectual are wanting.
Their only proper and direct proof of the thing in question had been to shew, in what sort and how far man’s salvation doth necessarily depend upon the knowledge of the word of God; what conditions, properties, and qualities there are, whereby sermons are distinguished from other kinds of administering the word unto that purpose; and what special property or quality that is, which being no where found but in sermons, maketh them effectual to save souls, and leaveth all other doctrinal means besides destitute of vital efficacy. These pertinent instructions, whereby they might satisfy us and obtain the cause itself for which they contend, these things which only would serve they leave, and (which needeth not) sometime they trouble themselves with fretting at the ignorance of such as withstand them in their opinion; sometime they fall upon their poor brethren which can but read, and against them they are bitterly eloquent.
If we allege what the Scriptures themselves do usually speak for the saving force of the word of God, not with restraint to any one certain kind of delivery, but howsoever the same shall chance to be made known, yet by one trick  or other they always restrain it unto sermons. Our Lord and Saviour hath said, “Search the Scriptures, in them ye think to have eternal life.” But they tell us, he spake to the Jews, which Jews before had heard his Sermons; and that peradventure it was his mind they should search, not by reading, nor by hearing them read, but by “attending” whensoever the Scriptures should happen to be alleged “in Sermons.”
Furthermore, having received apostolic doctrine, the Apostle St. Paul hath taught us to esteem the same as the supreme rule whereby all other doctrines must for ever be examined. Yea, but inasmuch as the Apostle doth there speak of that he had preached, he “flatly maketh” (as they strangely affirm) “his Preachings or Sermons the rule whereby to examine all.” And then I beseech you what rule have we whereby to judge or examine any? For if sermons must be our rule, because the Apostles’ sermons were so to their hearers; then, sith we are not as they were hearers of the Apostles’ sermons, it resteth that either the sermons which we hear should be our rule, or (that being absurd) there will (which yet hath greater absurdity) no rule at all be remaining for trial, what doctrines now are corrupt, what consonant with heavenly truth.
Again, let the same Apostle acknowledge “all Scripture profitable to teach, to improve, to correct, to instruct in righteousness.” Still notwithstanding we err, if hereby we presume to gather, that Scripture read will avail unto any one of all these uses; they teach us the meaning of the words to be, that so much the Scripture can do if the minister that way apply it in his sermons, otherwise not.
Finally, they never hear sentence which mentioneth the  Word or Scripture, but forthwith their glosses upon it are, the Word “preached,” the Scripture “explained or delivered unto us in sermons.” Sermons they evermore understand to be that Word of God, which alone hath vital operation; the dangerous sequel of which construction I wish they did more attentively weigh. For sith speech is the very image whereby the mind and soul of the speaker conveyeth itself into the bosom of him which heareth, we cannot choose but see great reason, wherefore the word that proceedeth from God, who is himself very truth and life, should be (as the Apostle to the Hebrews noteth) lively and mighty in operation, “sharper than any two-edged sword.” Now if in this and the like places we did conceive that our own sermons are that strong and forcible word, should we not hereby impart even the most peculiar glory of the word of God unto that which is not his word? For touching our sermons, that which giveth them their very being is the wit of man, and therefore  they oftentimes accordingly taste too much of that over corrupt fountain from which they come. In our speech of most holy things, our most frail affections many times are bewrayed.
Wherefore when we read or recite the Scripture, we then deliver to the people properly the word of God. As for our sermons, be they never so sound and perfect, his word they are not as the sermons of the prophets were; no, they are but ambiguously termed his word, because his word is commonly the subject whereof they treat, and must be the rule whereby they are framed. Notwithstanding by these and the like shifts they derive unto sermons alone whatsoever is generally spoken concerning the word.
Again, what seemeth to have been uttered concerning sermons and their efficacy or necessity, in regard of divine Matter, and must consequently be verified in sundry other kinds of teaching, if the Matter be the same in all; their use is to fasten every such speech unto that one only Manner of teaching which is by sermons, that still sermons may be all in all. Thus because Salomon declareth that the people decay or “perish” for want of knowledge, where no “prophesying” at all is, they gather that the hope of life and salvation is cut off, where preachers are not which prophesy by sermons, how many soever they be in number that read daily the word of God, and deliver, though in other sort, the selfsame matter which sermons do. The people which have  no way to come to the knowledge of God, no prophesying, no teaching, perish. But that they should of necessity perish, where any one way of knowledge lacketh, is more than the words of Salomon import.
Another usual point of their art in this present question, is to make very large and plentiful discourses how Christ is by sermons lifted up higher and made more apparent to the eye of faith; how the savour of the word is more sweet being brayed, and more able to nourish being divided by preaching, than by only reading proposed; how sermons are the keys of the kingdom of heaven, and do open the Scriptures, which being but read, remain in comparison still clasped; how God giveth richer increase of grace  to the ground that is planted and watered by preaching, than by bare and simple reading. Out of which premises declaring how attainment unto life is easier where sermons are, they conclude an impossibility thereof where sermons are not.
Alcidamas the sophister hath many arguments, to prove that voluntary and extemporal far excelleth premeditated speech. The like whereunto and in part the same are brought by them, who commend sermons, as having (which all men I think will acknowledge) sundry peculiar and proper virtues, such as no other way of teaching besides hath. Aptness to follow particular occasions presently growing, to put life into words by countenance, voice, and gesture, to prevail mightily in the sudden affections of men, this sermons may challenge. Wherein notwithstanding so eminent properties whereof lessons are haply destitute, yet lessons being free from some inconveniences whereunto sermons are more subject, they may in this respect no less take, than in other they must give the hand which betokeneth preeminence. For there is nothing which is not someway excelled even by that which it doth excel. Sermons therefore and Lessons may each excel other in some respects, without any prejudice unto either as touching that vital force which they both have in the work of our salvation.
To which effect when we have endeavoured as much as in us doth lie to find out the strongest causes wherefore they should imagine that reading is itself so unavailable, the most we can learn at their hand is, that sermons are “the ordinance of God,” the Scriptures “dark,” and the labour of reading “easy.”
First therefore as we know that God doth aid with his grace, and by his special providence evermore bless with happy success those things which himself appointeth, so his Church we persuade ourselves he hath not in such sort given over to a reprobate sense, that whatsoever it deviseth for the good of the souls of men, the same he doth still accurse and make frustrate.
Or if he always did defeat the ordinances of his Church, is not reading the ordinance of God? Wherefore then should we think that the force of his secret grace is accustomed to bless the labour of dividing his word according unto each man’s private discretion in public sermons, and to withdraw itself from concurring with the public delivery thereof by such selected portions of Scripture, as the whole Church hath solemnly appointed to be read for the people’s good, either by ordinary course, or otherwise, according to the exigence of special occasions? Reading (saith Isidore) is to the hearers no small edifying. To them whose delight and meditation is  in the law seeing that happiness and bliss belongeth, it is not in us to deny them the benefit of heavenly grace. And I hope we may presume, that a rare thing it is not in the Church of God, even for that very word which is read to be both presently their joy, and afterwards their study that hear it. St. Augustine speaking of devout men, noteth how they daily frequented the church, how attentive ear they gave unto the lessons and chapters read, how careful they were to remember the same, and to muse thereupon by themselves. St. Cyprian observeth that reading was not without effect in the hearts of men. Their joy and alacrity were to him an argument, that there is in this ordinance a blessing, such as ordinarily doth accompany the administration of the word of life.
It were much if there should be such a difference between the hearing of sermons preached and of lessons read in the church, that he which presenteth himself at the one, and maketh his prayer with the Prophet David, “Teach me O Lord the way of thy statutes, direct me in the path of thy commandments,” might have the ground of usual experience, whereupon to build his hope of prevailing with God, and obtaining the grace he seeketh; they contrariwise not so, who crave the like assistance of his Spirit, when they give ear to the reading of the other. In this therefore preaching and reading are equal, that both are approved as his ordinances, both assisted with his grace. And if his grace do assist them both to the nourishment of faith already bred, we cannot, without some very manifest cause yielded, imagine that in breeding or begetting faith, his grace doth cleave to the one and utterly forsake the other.
Touching hardness which is the second pretended impediment, as against Homilies being plain and popular instructions it is no bar, so neither doth it infringe the efficacy no not of Scriptures although but read. The force of reading, how small soever they would have it, must of necessity be granted sufficient to notify that which is plain or easy to be understood. And of things necessary to all men’s salvation we have been hitherto accustomed to hold (especially sithence the publishing of the Gospel of Jesus Christ, whereby the simplest having now a key unto knowledge which the Eunuch in the Acts did want, our children may of themselves  by reading understand that, which he without an interpreter could not) they are in Scripture plain and easy to be understood. As for those things which at the first are obscure and dark, when memory hath laid them up for a time, judgment afterwards growing explaineth them. Scripture therefore is not so hard, but that the only reading thereof may give life unto willing hearers.
The “easy” performance of which holy labour is in like sort a very cold objection to prejudice the virtue thereof. For what though an infidel, yea though a child may be able to read? There is no doubt, but the meanest and worst amongst the people under the Law had been as able as the priests themselves were to offer sacrifice. Did this make sacrifice of no effect unto that purpose for which it was instituted? In religion some duties are not commended so much by the hardness of their execution, as by the worthiness and dignity of that acceptation wherein they are held with God.
We admire the goodness of God in nature, when we consider how he hath provided that things most needful to preserve this life should be most prompt and easy for all living creatures to come by. Is it not as evident a sign of his wonderful providence over us, when that food of eternal life, upon the utter want whereof our endless death and destruction necessarily ensueth, is prepared and always set in such a readiness, that those very means than which nothing is more easy may suffice to procure the same? Surely if we perish it is not the lack of scribes and learned expounders that can be our just excuse. The word which saveth our souls is near us; we need for knowledge but to read and live. The man which readeth the word of God the word itself doth pronounce blessed, if he also observe the same.
Now all these things being well considered, it shall be no intricate matter for any man to judge with indifferency, on which part the good of the Church is most conveniently sought; whether on ours whose opinion is such as hath been shewed, or else on theirs, who leaving no ordinary way of  salvation for them unto whom the word of God is but only read, do seldom name them but with great disdain and contempt who execute that service in the Church of Christ. By means whereof it hath come to pass, that churches, which cannot enjoy the benefit of usual preaching, are judged as it were even forsaken of God, forlorn, and without either hope or comfort: contrariwise those places which every day for the most part are at sermons as the flowing sea, do both by their emptiness at times of reading, and by other apparent tokens, shew to the voice of the living God this way sounding in the ears of men a great deal less reverence than were meet.
But if no other evil were known to grow thereby, who can choose but think them cruel which doth hear them so boldly teach, that if God (as to Him there is nothing impossible) do haply save any such as continue where they have all other means of instruction, but are not taught by continual preaching, yet this is miraculous, and more than the fitness of so poor instruments can give any man cause to hope for; that sacraments are not effectual to salvation, except men be instructed by preaching before they be made partakers of them; yea, that both sacraments and prayers also, where sermons are not, “do not only not feed, but are ordinarily to  further condemnation?” What man’s heart doth not rise at the mention of these things?
It is true that the weakness of our wits and the dulness of our affections do make us for the most part, even as our Lord’s own disciples were for a certain time, hard and slow to believe what is written. For help whereof expositions and exhortations are needful, and that in the most effectual manner. The principal churches throughout the land, and no small part of the rest, being in this respect by the goodness of God so abundantly provided for, they which want the like furtherance unto knowledge, wherewith it were greatly to be desired that they also did abound, are yet we hope not left in so extreme destitution, that justly any man should think the ordinary means of eternal life taken from them, because their teaching is in public for the most part but by reading. For which cause amongst whom there are not those helps that others have to set them forward in the way of life, such to dishearten with fearful sentences, as though their salvation could hardly be hoped for, is not in our understanding so consonant with Christian charity.  We hold it safer a great deal and better to give them encouragement; to put them in mind that it is not the deepness of their knowledge, but the singleness of their belief, which God accepteth; that they which “hunger and thirst after righteousness shall be satisfied;” that no imbecility of means can prejudice the truth of the promise of God herein; that the weaker their helps are, the more their need is to sharpen the edge of their own industry; and that painfulness by feeble means shall be able to gain that, which in the plenty of more forcible instruments is through sloth and negligence lost.
As for the men, with whom we have thus far taken pains to confer about the force of the word of God, either read by itself, or opened in sermons; their speeches concerning both the one and the other are in truth such, as might give us very just cause to think, that the reckoning is not great which they make of either. For howsoever they have been driven to devise some odd kinds of blind uses, whereunto they may answer that reading doth serve, yet the reading of the word of God in public more than their preachers’ bare text, who will not judge that they deem needless; when if we chance at any time to term it necessary, as being a thing which God himself did institute amongst the Jews for purposes that touch as well us as them; a thing which the Apostles commend under the Old, and ordain under the New Testament; a thing whereof the Church of God hath ever sithence the first beginning reaped singular commodity; a thing which without exceeding great detriment no Church can omit: they only are the  men that ever we heard of by whom this hath been crossed and gainsaid, they only the men which have given their peremptory sentence to the contrary, “It is untrue that simple reading is necessary in the Church.” And why untrue? Because “although it be very convenient which is used in some churches, where before preaching-time the church assembled hath the Scriptures read in such order that the whole canon thereof is oftentimes in one year run through; yet a number of churches which have no such order of simple reading cannot be in this point charged with breach of God’s commandment, which they might be if simple reading were necessary.” A poor, a cold, and an hungry cavil! Shall we therefore to please them change the word “necessary,” and say that it hath been a commendable order, a custom very expedient, or an ordinance most profitable (whereby they know right well that we mean exceedingly behoveful) to read the word of God at large in the church, whether it be as our manner is, or as theirs is whom they prefer before us? It is not this that will content or satisfy their minds. They have against it a marvellous deep and profound axiom, that “Two things to one and the same end cannot but very improperly be said most profitable.” And therefore if preaching be “most profitable” to man’s salvation, then is not reading; if reading be, then preaching is not.
Are they resolved then at the leastwise, if preaching be the only ordinary mean whereby it pleaseth God to save our souls, what kind of preaching it is which doth save? Understand they how or in what respect there is that force and virtue in preaching? We have reason wherefore to make these demands, for that although their pens run all upon preaching and sermons, yet when themselves do practise that whereof they write, they change their dialect, and those words they shun as if there were in them some secret sting. It is not their phrase to say they “preach,” or to give to their own instructions and exhortations the name of sermons; the pain they take themselves in this kind is either “opening,” or “lecturing,” or “reading,” or “exercising,” but in no case “preaching.”  And in this present question they also warily protest, that what they ascribe to the virtue of preaching, they still mean it of “good preaching.” Now one of them saith that a good sermon must “expound” and “apply” a “large” portion of the text of Scripture at one time. Another giveth us to understand, that sound preaching “is not to do as one did at London, who spent the most of his time in invectives against good men, and told his audience how the magistrate should have an eye to such as troubled the peace of the Church.” The best of them hold it for no good preaching “when a man endeavoureth to make a glorious show of eloquence  and learning, rather than to apply himself to the capacity of the simple.”
But let them shape us out a good preacher by what pattern soever it pleaseth them best, let them exclude and inclose whom they will with their definitions, we are not desirous to enter into any contention with them about this, or to abate the conceit they have of their own ways, so that when once we are agreed what sermons shall currently pass for good, we may at the length understand from them what that is in a good sermon which doth make it the word of life unto such as hear. If substance of matter, evidence of things, strength and validity of arguments and proofs, or if any other virtue else which words and sentences may contain; of all this what is there in the best sermons being uttered, which they lose by being read? But they utterly deny that the reading either of scriptures or homilies and sermons can ever by the ordinary grace of God save any soul. So that although we had all the sermons word for word which James, Paul, Peter, and the rest of the Apostles made, some one of which sermons was of power to convert thousands of the hearers unto Christian faith; yea although we had all the instructions, exhortations, consolations, which came from the gracious lips of our Lord Jesus Christ himself, and should read them ten thousand times over, to faith and salvation no man could hereby hope to attain.
Whereupon it must of necessity follow, that the vigour and vital efficacy of sermons doth grow from certain accidents which are not in them but in their maker: his virtue, his gesture, his countenance, his zeal, the motion of his body, and the inflection of his voice who first uttereth them as his own, is that which giveth them the form, the nature, the very essence of instruments available to eternal life. If they like neither that nor this, what remaineth but that their final conclusion be, “sermons we know are the only ordinary means to salvation, but why or how we cannot tell?”
Wherefore to end this tedious controversy, wherein the too great importunity of our over eager adversaries hath constrained us much longer to dwell, than the barrenness of so poor a cause could have seemed at the first likely either to  require or to admit, if they which without partialities and passions are accustomed to weigh all things, and accordingly to give their sentence, shall here sit down to receive our audit, and to cast up the whole reckoning on both sides; the sum which truth amounteth unto will appear to be but this, that as medicines provided of nature and applied by art for the benefit of bodily health, take effect sometimes under and sometimes above the natural proportion of their virtue, according as the mind and fancy of the patient doth more or less concur with them: so whether we barely read unto men the Scriptures of God, or by homilies concerning matter of belief and conversation seek to lay before them the duties which they owe unto God and man; whether we deliver them books to read and consider of in private at their own best leisure, or call them to the hearing of sermons publicly in the house of God; albeit every of these and the like unto these means do truly and daily effect that in the hearts of men for which they are each and all meant, yet the operation which they have in common being most sensible and most generally noted in one kind above the rest, that one hath in some men’s opinions drowned altogether the rest, and injuriously brought to pass that they have been thought, not less effectual than the other, but without the other uneffectual to save souls. Whereas the cause why sermons only are observed to prevail so much while all means else seem to sleep and do nothing, is in truth nothing but that singular affection and attention which the people sheweth every where towards the one, and their cold disposition to the other; the reason hereof being partly the art which our adversaries use for the credit of their sermons to bring men out of conceit with all other teaching besides; partly a custom which men have to let those things carelessly pass by their ears, which they have oftentimes heard before, or know they may hear again whensoever it pleaseth themselves; partly the especial advantages which sermons naturally have to procure attention, both in that they come always new, and because by the hearer it is still presumed, that if they be let slip for the present, what good soever they contain is lost, and that without all hope of recovery. This is the true cause of odds between sermons and other kinds of wholesome instruction.
As for the difference which hath been hitherto so much defended on the contrary side, making sermons the only ordinary means unto faith and eternal life, sith this hath neither evidence of truth nor proof sufficient to give it warrant, a cause of such quality may with far better grace and conveniency ask that pardon which common humanity doth easily grant, than claim in challenging manner that assent which is as unwilling when reason guideth it to be yielded where it is not, as withheld where it is apparently due.
All which notwithstanding, as we could greatly wish that the rigour of this their opinion were allayed and mitigated, so because we hold it the part of religious ingenuity to honour virtue in whomsoever, therefore it is our most hearty desire, and shall be always our prayer unto Almighty God, that in the selfsame fervent zeal wherewith they seem to affect the good of the souls of men, and to thirst after nothing more than that all men might by all means be directed in the way of life, both they and we may constantly persist to the world’s end. For in this we are not their adversaries, though they in the other hitherto have been ours.

\section*{Of Prayer.}
XXIII. Between the throne of God in heaven and his Church upon earth here militant if it be so that Angels have their continual intercourse, where should we find the same more verified than in these two ghostly exercises, the one Doctrine, and the other Prayer? For what is the assembling of the Church to learn, but the receiving of Angels descended from above? What to pray, but the sending of Angels upward? His heavenly inspirations and our holy desires are as so many Angels of intercourse and commerce between God and us. As teaching bringeth us to know that God is our supreme truth; so prayer testifieth that we acknowledge him our sovereign good.
Besides, sith on God as the most high all inferior causes in the world are dependent; and the higher any cause is, the more it coveteth to impart virtue unto things beneath it; how should any kind of service we do or can do find greater acceptance than prayer, which sheweth our concurrence with him in desiring that wherewith his very nature doth most delight?
Is not the name of prayer usual to signify even all the service that ever we do unto God? And that for no other cause,  as I suppose, but to shew that there is in religion no acceptable duty which devout invocation of the name of God doth not either presuppose or infer. Prayers are those “calves of men’s lips;” those most gracious and sweet odours; those rich presents and gifts, which being carried up into heaven do best testify our dutiful affection, and are for the purchasing of all favour at the hands of God the most undoubted means we can use.
On others what more easily, and yet what more fruitfully bestowed than our prayers? If we give counsel, they are the simpler only that need it; if alms, the poorer only are relieved; but by prayer we do good to all. And whereas every other duty besides is but to shew itself as time and opportunity require, for this all times are convenient: when we are not able to do any other thing for men’s behoof, when through maliciousness or unkindness they vouchsafe not to accept any other good at our hands, prayer is that which we always have in our power to bestow, and they never in theirs to refuse. Wherefore “God forbid,” saith Samuel, speaking unto a most unthankful people, a people weary of the benefit of his most virtuous government over them, “God forbid that I should sin against the Lord, and cease to pray for you.” It is the first thing wherewith a righteous life beginneth, and the last wherewith it doth end.
The knowledge is small which we have on earth concerning things that are done in heaven. Notwithstanding thus much we know even of Saints in heaven, that they pray. And therefore prayer being a work common to the Church as well triumphant as militant, a work common unto men with Angels, what should we think but that so much of our lives is celestial and divine as we spend in the exercise of prayer? For which cause we see that the most comfortable visitations, which God hath sent men from above, have taken especially the times of prayer as their most natural opportunities.

\section*{Of public Prayer.}
XXIV. This holy and religious duty of service towards God concerneth us one way in that we are men, and another  way in that we are joined as parts to that visible mystical body which is his Church. As men, we are at our own choice, both for time, and place, and form, according to the exigence of our own occasions in private; but the service, which we do as members of a public body, is public, and for that cause must needs be accounted by so much worthier than the other, as a whole society of such condition exceedeth the worth of any one. In which consideration unto Christian assemblies there are most special promises made. St. Paul, though likely to prevail with God as much as [any] one, did notwithstanding think it much more both for God’s glory and his own good, if prayers might be made and thanks yielded in his behalf by a number of men. The prince and people of Nineveh assembling themselves as a main army of supplicants, it was not in the power of God to withstand them. I speak no otherwise concerning the force of public prayer in the Church of God, than before me Tertullian hath done, “We come by troops to the place of assembly, that being banded as it were together, we may be supplicants enough to besiege God with our prayers. These forces are unto him acceptable.”
When we publicly make our prayers, it cannot be but that we do it with much more comfort than in private, for that the things we ask publicly are approved as needful and good in the judgment of all, we hear them sought for and desired with common consent. Again, thus much help and furtherance is more yielded, in that if so be our zeal and devotion to Godward be slack, the alacrity and fervour of others serveth as a present spur. “For even prayer itself” (saith St. Basil) “when it hath not the consort of many voices to strengthen it, is not itself.” Finally, the good which we do  by public prayer is more than in private can be done, for that besides the benefit which here is no less procured to ourselves, the whole Church is much bettered by our good example; and consequently whereas secret neglect of our duty in this kind is but only our own hurt, one man’s contempt of the common prayer of the Church of God may be and oftentimes is most hurtful unto many. In which considerations the prophet David so often voweth unto God the sacrifice of praise and thanksgiving in the congregation; so earnestly exhorteth others to sing praises unto the Lord in his courts, in his sanctuary, before the memorial of his holiness; and so much complaineth of his own uncomfortable exile, wherein although he sustained many most grievous indignities, and endured the want of sundry both pleasures and honours before enjoyed, yet as if this one were his only grief and the rest not felt, his speeches are all of the heavenly benefit of public assemblies, and the happiness of such as had free access thereunto.

\section*{Of the form of Common Prayer.}
XXV. A great part of the cause, wherefore religious minds are so inflamed with the love of public devotion, is that virtue, force, and efficacy, which by experience they find that the very form and reverend solemnity of common prayer duly ordered hath, to help that imbecility and weakness in us, by means whereof we are otherwise of ourselves the less apt to perform unto God so heavenly a service, with such affection of heart, and disposition in the powers of our souls as is requisite. To this end therefore all things hereunto appertaining have been ever thought convenient to be done with the most solemnity and majesty that the wisest could devise. It is not with public as with private prayer. In this rather secresy is commended than outward show, whereas that being the public act of a whole society, requireth accordingly more care to be had of external appearance. The very assembling of men therefore unto this service hath been ever solemn.
And concerning the place of assembly, although it serve for other uses as well as this, yet seeing that our Lord himself hath to this as to the chiefest of all other plainly sanctified his own temple, by entitling it “the House of Prayer,” what preeminence of dignity soever hath been either by the ordinance or through the special favour and providence of God annexed unto his Sanctuary, the principal cause thereof must needs be in regard of Common Prayer. For the honour and furtherance whereof, if it be as the gravest of the ancient Fathers seriously were persuaded, and do oftentimes plainly teach, affirming that the house of prayer is a Court beautified with the presence of celestial powers; that there we stand, we pray, we sound forth hymns unto God, having his Angels intermingled as our associates; and that with reference hereunto the Apostle doth require so great care to be had of decency for the Angels’ sake; how can we come to the house of prayer, and not be moved with the very glory of the place itself, so to frame our affections praying, as doth best beseem them, whose suits the Almighty doth there sit to hear, and his Angels attend to further? When this was ingrafted in the minds of men, there needed no penal statutes to draw them unto public prayer. The warning sound was no sooner heard, but the churches were presently filled, the pavements covered with bodies prostrate, and washed with their tears of devout joy.
And as the place of public prayer is a circumstance in the outward form thereof, which hath moment to help devotion; so the person much more with whom the people of God do join themselves in this action, as with him that standeth and speaketh in the presence of God for them. The authority of his place, the fervour of his zeal, the piety and gravity of his whole behaviour must needs exceedingly both grace and set forward the service he doth.
The authority of his calling is a furtherance, because if God have so far received him into favour, as to impose upon  him by the hands of men that office of blessing the people in his name, and making intercession to him in theirs; which office he hath sanctified with his own most gracious promise, and ratified that promise by manifest actual performance thereof, when others before in like place have done the same; is not his very ordination a seal as it were to us, that the selfsame divine love, which hath chosen the instrument to work with, will by that instrument effect the thing whereto he ordained it, in blessing his people and accepting the prayers which his servant offereth up unto God for them? It was in this respect a comfortable title which the ancient used to give unto God’s ministers, terming them usually God’s most beloved, which were ordained to procure by their prayers his love and favour towards all.
Again, if there be not zeal and fervency in him which proposeth for the rest those suits and supplications which they by their joyful acclamations must ratify; if he praise not God with all his might; if he pour not out his soul in prayer; if he take not their causes to heart, or speak not as Moses, Daniel, and Ezra did for their people: how should there be but in them frozen coldness, when his affections seem benumbed from whom theirs should take fire?
Virtue and godliness of life are required at the hands of the minister of God, not only in that he is to teach and instruct the people, who for the most part are rather led away by the ill example, than directed aright by the wholesome instruction of them, whose life swerveth from the rule of their own doctrine; but also much more in regard of this other part of his function; whether we respect the weakness of the people, apt to loathe and abhor the sanctuary when they which perform the service thereof are such as the sons of Heli were; or else consider the inclination of God himself, who requireth the lifting up of pure hands in prayer, and hath given the world plainly to understand that the wicked although they cry shall not be heard. They are no fit supplicants to seek his mercy in behalf of others, whose own unrepented sins provoke his just indignation. Let thy Priests therefore, O  Lord, be evermore clothed with righteousness, that thy saints may thereby with more devotion rejoice and sing.
But of all helps for due performance of this service the greatest is that very set and standing order itself, which framed with common advice, hath both for matter and form prescribed whatsoever is herein publicly done. No doubt from God it hath proceeded, and by us it must be acknowledged a work of his singular care and providence, that the Church hath evermore held a prescript form of common prayer, although not in all things every where the same, yet for the most part retaining still the same analogy. So that if the liturgies of all ancient churches throughout the world be compared amongst themselves, it may be easily perceived they had all one original mould, and that the public prayers of the people of God in churches throughly settled did never use to be voluntary dictates proceeding from any man’s extemporal wit.
To him which considereth the grievous and scandalous inconveniences whereunto they make themselves daily subject, with whom any blind and secret corner is judged a fit house of common prayer; the manifold confusions which they fall into where every man’s private spirit and gift (as they term it) is the only Bishop that ordaineth him to this ministry; the irksome deformities whereby through endless and senseless effusions of indigested prayers they oftentimes disgrace in most unsufferable manner the worthiest part of Christian duty towards God, who herein are subject to no certain order, but pray both what and how they list: to him I say which weigheth duly all these things the reasons cannot be obscure, why God doth in public prayer so much respect the solemnity of places where, the authority and calling of persons by whom, and the precise appointment even with what words or sentences his name should be called on amongst his people.


\section*{Of them which like not to have any set form of Common Prayer.}
XXVI. No man hath hitherto been so impious as plainly and directly to condemn prayer. The best stratagem that Satan hath, who knoweth his kingdom to be no one way more  shaken than by the public devout prayers of God’s Church, is by traducing the form and manner of them to bring them into contempt, and so to shake the force of all men’s devotion towards them. From this and from no other forge hath proceeded a strange conceit, that to serve God with any set form of common prayer is superstitious.
As though God himself did not frame to his Priests the very speech wherewith they were charged to bless the people; or as if our Lord, even of purpose to prevent this fancy of extemporal and voluntary prayers, had not left us of his own framing one, which might both remain as a part of the church liturgy, and serve as a pattern whereby to frame all other prayers with efficacy, yet without superfluity of words. If prayers were no otherwise accepted of God than being conceived always new, according to the exigence of present occasions; if it be right to judge him by our own bellies, and to imagine that he doth loathe to have the selfsame supplications often iterated, even as we do to be every day fed without alteration or change of diet; if prayers be actions which ought to waste away themselves in the making; if being made to remain that they may be resumed and used again as prayers, they be but instruments of superstition: surely we cannot excuse Moses, who gave such occasion of scandal to the world, by not being contented to praise the name of Almighty God according to the usual naked simplicity of God’s Spirit for that admirable victory given them against Pharao, unless so dangerous a precedent were left for the casting of prayers into certain poetical moulds, and for the framing of prayers which might be repeated often, although they never had again the same occasions which brought them forth at the first. For that very hymn of Moses grew afterwards to be a part of the ordinary Jewish liturgy; nor only that, but sundry other sithence invented.  Their books of common prayer contained partly hymns taken out of the holy Scripture, partly benedictions, thanksgivings, supplications, penned by such as have been from time to time the governors of that synagogue. These they sorted into their several times and places, some to begin the service of God with, and some to end, some to go before, and some to follow, and some to be interlaced between the divine readings of the Law and Prophets. Unto their custom of finishing the Passover with certain Psalms, there is not any thing more probable, than that the Holy Evangelist doth evidently allude saying, That after the cup delivered by our Saviour unto his apostles, “they sung,” and went forth to the mount of Olives.
As the Jews had their songs of Moses and David and the rest, so the Church of Christ from the very beginning hath both used the same, and besides them other also of like nature, the song of the Virgin Mary, the song of Zachary,  the song of Simeon, such hymns as the Apostle doth often speak of saying, “I will pray and sing with the Spirit:” again, “in psalms, hymns, and songs, making melody unto the Lord, and that heartily.” Hymns and psalms are such kinds of prayer as are not wont to be conceived upon a sudden, but are framed by meditation beforehand, or else by prophetical illumination are inspired, as at that time it appeareth they were when God by extraordinary gifts of the Spirit enabled men to all parts of service necessary for the edifying of his Church.

\section*{Of them who allowing a set form of prayer yet allow not ours.}
XXVII. Now albeit the Admonitioners did seem at the first to allow no prescript form of prayer at all, but thought it the best that their minister should always be left at liberty to pray as his own discretion did serve; yet because this opinion upon better advice they afterwards retracted, their defender and his associates have sithence proposed to the world a form such as themselves like, and to shew their dislike of ours, have taken against it those exceptions, which whosoever doth measure by number, must needs be greatly out of love with a thing that hath so many faults; whosoever by weight, cannot choose but esteem very highly of that, wherein the wit of so scrupulous adversaries hath not hitherto observed any defect which themselves can seriously think to be of moment. “Gross errors and manifest impiety,” they  grant we have “taken away.” Yet many things in it they say are amiss; many instances they give of things in our common prayer not agreeable as they pretend with the word of God. It hath in their eye too great affinity with the form of the Church of Rome; it differeth too much from that which churches elsewhere reformed allow and observe; our attire disgraceth it; it is not orderly read nor gestured as beseemeth: it requireth nothing to be done which a child may not lawfully do; it hath a number of short cuts or shreddings which may be better called wishes than prayers; it intermingleth prayings and readings, in such manner as if supplicants should use in proposing their suits unto mortal princes, all the world would judge them mad; it is too long and by that mean abridgeth preaching; it appointeth the people to say after the minister; it spendeth time in singing and in reading the Psalms by course from side to side; it useth the Lord’s Prayer too oft; the songs of Magnificat, Benedictus, and Nunc Dimittis, it might very well spare; it hath the Litany, the Creed of Athanasius, and Gloria Patri, which are superfluous; it craveth earthly things too much; for deliverance from those evils against which we pray it giveth no thanks; some things it asketh unseasonably when they need not to be prayed for, as deliverance from thunder and tempest when no danger is nigh; some in too abject and diffident manner, as that God would give us that which we for our unworthiness dare not ask; some which ought not to be desired, as the deliverance from sudden death, riddance from all adversity, and the extent of saving  mercy towards all men. These and such like are the imperfections, whereby our form of common prayer is thought to swerve from the word of God.
A great favourer of that part, but yet (his error that way excepted) a learned, a painful, a right virtuous and a good man did not fear sometime to undertake, against popish detractors, the general maintenance and defence of our whole church service, as having in it nothing repugnant to the word of God. And even they which would file away most from the largeness of that offer, do notwithstanding in more sparing terms acknowledge little less. For when those opposite judgments which never are wont to construe things doubtful to the better, those very tongues which are always prone to aggravate whatsoever hath but the least show whereby it may be suspected to savour of or to sound towards any evil, do by their own voluntary sentence clearly free us from “gross errors,” and from “manifest impiety” herein; who would not judge us to be discharged of all blame, which are confessed to have no great fault even by their very word and testimony, in whose eyes no fault of ours hath ever hitherto been accustomed to seem small?
Nevertheless what they seem to offer us with the one hand, the same with the other they pull back again. They grant we err not in palpable manner, we are not openly and  notoriously impious; yet errors we have which the sharp insight of their wisest men doth espy, there is hidden impiety which the profounder sort are able enough to disclose. Their skilful ears perceive certain harsh and unpleasant discords in the sound of our common prayer, such as the rules of divine harmony, such as the laws of God cannot bear.

\section*{The form of our Liturgy too near the papists’, too far different from that of other reformed Churches, as they pretend.}
XXVIII. Touching our conformity with the church of Rome, as also of the difference between some reformed churches and ours, that which generally hath been already answered may serve for answer to that exception which in these two respects they take particularly against the form of our common prayer. To say that in nothing they may be followed which are of the church of Rome were violent and extreme. Some things they do in that they are men, in that they are wise men and Christian men some things, some things in that they are men misled and blinded with error. As far as they follow reason and truth, we fear not to tread the selfsame steps wherein they have gone, and to be their followers. Where Rome keepeth that which is ancienter and better, others whom we much more affect leaving it for newer and changing it for worse; we had rather follow the perfections of them whom we like not, than in defects resemble them whom we love.
For although they profess they agree with us touching “a prescript form of prayer to be used in the church,” yet in that very form which they say is “agreeable to God’s word and the use of reformed churches,” they have by special protestation declared, that their meaning is not it shall be prescribed as a thing whereunto they will tie their minister. “It shall not” (they say) “be necessary for the minister daily to repeat all these things before-mentioned, but beginning with some like confession to proceed to the sermon, which ended, he either useth the prayer for all estates before-mentioned, or else prayeth as the Spirit of God shall move his heart.” Herein therefore we hold it much better with the church of Rome to appoint a prescript form which every  man shall be bound to observe, than with them to set down a kind of direction, a form for men to use if they list, or otherwise to change as pleaseth themselves.
Furthermore, the church of Rome hath rightly also considered, that public prayer is a duty entire in itself, a duty requisite to be performed much oftener than sermons can possibly be made. For which cause, as they, so we have likewise a public form how to serve God both morning and evening, whether sermons may be had or no. On the contrary side, their form of reformed prayer sheweth only what shall be done “upon the days appointed for the preaching of the word;” with what words the minister shall begin, “when the hour appointed for the sermon is come;” what shall be said or sung before sermon, and what after. So that, according to this form of theirs, it must stand for a rule, “No sermon, no service.” Which oversight occasioned the French spitefully to term religion in that sort exercised a mere “preach.” Sundry other more particular defects there are, which I willingly forbear to rehearse, in consideration whereof we cannot be induced to prefer their reformed form of prayer before our own, what church soever we resemble therein.

\section*{Attire belonging to the service of God.}
XXIX. The attire which the minister of God is by order to use at times of divine service being but a matter of mere formality, yet such as for comeliness sake hath hitherto been judged by the wiser sort of men not unnecessary to concur with other sensible notes betokening the different kind or quality of persons and actions whereto it is tied: as we think not ourselves the holier because we use it, so neither should they with whom no such thing is in use think us therefore unholy, because we submit ourselves unto that, which in a matter so  indifferent the wisdom of authority and law have thought comely. To solemn actions of royalty and justice their suitable ornaments are a beauty. Are they only in religion a stain?
“Divine religion,” saith St. Jerome, (he speaketh of the priestly attire of the Law,) “hath one kind of habit wherein to minister before the Lord, another for ordinary uses belonging unto common life.” Pelagius having carped at the curious neatness of men’s apparel in those days, and through the sourness of his disposition spoken somewhat too hardly thereof, affirming that “the glory of clothes and ornaments was a thing contrary to God and godliness;” St. Jerome, whose custom is not to pardon over easily his adversaries if any where they chance to trip, presseth him as thereby making all sorts of men in the world God’s enemies. “Is it enmity with God” (saith he) “if I wear my coat somewhat handsome? If a Bishop, a Priest, a Deacon, and the rest of the ecclesiastical order come to administer the usual sacrifice in a white garment, are they hereby God’s adversaries? Clerks, Monks, Widows, Virgins, take heed, it is dangerous for you to be otherwise seen than in foul and ragged clothes. Not to speak any thing of secular men, which are proclaimed to have war with God, as oft as ever they put on precious and shining clothes.” By which words of Jerome we may take it at the least for a probable collection that his meaning was to draw Pelagius into hatred, as condemning by so general a speech even the neatness of that very garment itself, wherein the clergy did then use to administer publicly the holy Sacrament of Christ’s most blessed Body and Blood. For that they did then use some such ornament, the words of Chrysostom give plain  testimony, who speaking to the clergy of Antioch, telleth them that if they did suffer notorious malefactors to come to the Table of our Lord and not put them by, it would be as heavily revenged upon them, as if themselves had shed his blood; that for this purpose God hath called them to the rooms which they held in the church of Christ; that this they should reckon was their dignity, this their safety, this their whole crown and glory; and therefore this they should carefully intend, and not when the Sacrament is administered imagine themselves called only to walk up and down in a white and shining garment.
Now whereas these speeches of Jerome and Chrysostom do seem plainly to allude unto such ministerial garments as were then in use, to this they answer, that by Jerome nothing can be gathered but only that the ministers came to church in handsome holyday apparel, and that himself did not think them bound by the law of God to go like slovens, but the weed which we mean he defendeth not; that Chrysostom meaneth indeed the same which we defend, but seemeth rather to reprehend than to allow it as we do. Which answer wringeth out of Jerome and Chrysostom that which their words will not gladly yield. They both speak of the same persons, namely the Clergy; and of their weed at the same time, when they administer the blessed Sacrament; and of the selfsame kind of weed, a white garment, so far as we have wit to conceive; and for any thing we are able to see, their manner of speech is not such as doth argue either the thing itself to be different whereof they speak, or their judgments  concerning it different; although the one do only maintain it against Pelagius, as a thing not therefore unlawful, because it was fair or handsome, and the other make it a matter of small commendation in itself, if they which wear it do nothing else but air the robes which their place requireth. The honesty, dignity, and estimation of white apparel in the eastern part of the world, is a token of greater fitness for this sacred use, wherein it were not convenient that any thing basely thought of should be suffered. Notwithstanding I am not bent to stand stiffly upon these probabilities, that in Jerome’s and Chrysostom’s time any such attire was made several to this purpose. Yet surely the words of Salomon are very impertinent to prove it an ornament therefore not several for the ministers to execute their ministry in, because men of credit and estimation wore their ordinary apparel white. For we know that when Salomon wrote those words, the several apparel for the ministers of the Law to execute their ministry in was such.
The wise man, which feared God from his heart, and honoured the service that was done unto him, could not mention so much as the garments of holiness but with effectual signification of most singular reverence and love. Were it not better that the love which men bear to God should make the least things that are employed in his service amiable, than that their overscrupulous dislike of so mean a thing as a vestment should from the very service of God withdraw their hearts and affections? I term it the rather a mean thing, a thing not much to be respected, because even they so account now of it, whose first disputations against it were such as if religion had scarcely any thing of greater weight.
Their allegations were then, “That if a man were assured to gain a thousand by doing that which may offend any one brother, or be unto him a cause of falling, he ought not to do it; that this popish apparel, the surplice especially, hath been by papists abominably abused; that it hath been a mark and a very sacrament of abomination; that remaining, it serveth as a monument of idolatry, and not only edifieth not, but as a dangerous and scandalous  ceremony doth exceeding much harm to them of whose good we are commanded to have regard; that it causeth men to perish and make shipwreck of conscience;” for so themselves profess they mean, when they say the weak are offended herewith; “that it hardeneth Papists, hindereth the weak from profiting in the knowledge of the Gospel, grieveth godly minds, and giveth them occasion to think hardly of their ministers; that if the magistrate may command, or the Church appoint rites and ceremonies, yet seeing our abstinence from things in their own nature indifferent if the weak brother should be offended is a flat commandment of the Holy Ghost, which no authority either of church or commonwealth can make void, therefore neither may the one nor the other lawfully ordain this ceremony, which hath great incommodity and no profit, great offence and no edifying; that by the Law it should have been burnt and consumed with fire as a thing infected with leprosy; that the example of Ezechias beating to powder the brazen serpent, and of Paul abrogating those abused feasts of charity, enforceth upon us the duty of abolishing altogether a thing which hath been and is so offensive; finally, that God by his Prophet hath given an express commandment, which in this case toucheth us no less than of old it did the Jews; ‘Ye shall pollute the ‘covering of the images of silver, and the rich ornament of your images of gold, and cast them away as a stained rag; thou shalt say unto it, Get thee hence.’ ”
These and such like were their first discourses touching that church attire which with us for the most part is usual in public prayer; our ecclesiastical laws so appointing, as well because it hath been of reasonable continuance, and by special choice was taken out of the number of those holy garments which (over and besides their mystical reference) served for “comeliness” under the Law, and is in the number of  those ceremonies which may with choice and discretion be used to that purpose in the Church of Christ; as also for that it suiteth so fitly with that lightsome affection of joy, wherein God delighteth when his saints praise him; and so lively resembleth the glory of the saints in heaven, together with the beauty wherein Angels have appeared unto men, that they which are to appear for men in the presence of God as Angels, if they were left to their own choice and would choose any, could not easily devise a garment of more decency for such a service.
As for those fore-rehearsed vehement allegations against it, shall we give them credit when the very authors from whom they come confess they believe not their own sayings? For when once they began to perceive how many both of them in the two universities, and of others who abroad having ecclesiastical charge do favour mightily their cause and by all means set it forward, might by persisting in the extremity of that opinion hazard greatly their own estates, and so weaken that part which their places do now give them much opportunity to strengthen; they asked counsel as it seemeth from some abroad, who wisely considered that the body is of far  more worth than the raiment. Whereupon for fear of dangerous inconveniences, it hath been thought good to add, that sometimes authority “must and may with good conscience be obeyed, even where commandment is not given upon good grounds;” that “the duty of preaching is one of the absolute commandments of God, and therefore ought not to be forsaken for the bare inconvenience of a thing which in the own nature is indifferent;” that “one of the foulest spots in the surplice is the offence which it giveth in occasioning the weak to fall and the wicked to be confirmed in their wickedness,” yet hereby there is no unlawfulness proved, but “only an inconveniency” that such things should be established, howbeit no such inconveniency neither “as may not be borne with;” that when God doth flatly command us to abstain from things in their own nature indifferent if they offend our weak brethren, his meaning is not we should obey his commandment herein, unless we may do it “and not leave undone that which the Lord hath absolutely  commanded.” Always provided that whosoever will enjoy the benefit of this dispensation to wear a scandalous badge of idolatry, rather than forsake his pastoral charge, do “as occasion serveth teach” nevertheless still “the incommodity of the thing itself, admonish the weak brethren that they be not, and pray unto God so to strengthen them that they may not be offended thereat.” So that whereas before they which had authority to institute rites and ceremonies were denied to have power to institute this, it is now confessed that this they may also “lawfully” but not so “conveniently” appoint; they did well before and as they ought, who had it in utter detestation and hatred, as a thing abominable, they now do well which think it may be both borne and used with a very good conscience; before, he which by wearing it were sure to win thousands unto Christ ought not to do it if there were but one which might be offended, now though it be with the offence of thousands, yet it may be done rather than that should be given over whereby notwithstanding we are not certain we shall gain one: the examples of Ezechias and of Paul, the charge which was given to the Jews by Esay, the strict apostolical prohibition of things indifferent whensoever they may be scandalous, were before so forcible laws against our ecclesiastical attire, as neither church nor commonwealth could possibly make void; which now one of far less authority than either hath found how to frustrate, by dispensing with the breach of inferior commandments, to the end that the greater may be kept.
But it booteth them not thus to soder up a broken cause, whereof their first and last discourses will fall asunder do what they can. Let them ingenuously confess that their invectives were too bitter, their arguments too weak, the matter not so dangerous as they did imagine. If those alleged testimonies of Scripture did indeed concern the matter to such effect as was pretended, that which they should infer were unlawfulness, because they were cited as prohibitions of that thing which indeed they concern. If they prove not our attire unlawful because in truth they concern it not, it followeth that they prove not any thing against it, and consequently not so much as uncomeliness or inconveniency. Unless therefore they be able throughly to resolve themselves that there is no  one sentence in all the Scriptures of God which doth control the wearing of it in such manner and to such purpose as the church of England alloweth; unless they can fully rest and settle their minds in this most sound persuasion, that they are not to make themselves the only competent judges of decency in these cases, and to despise the solemn judgment of the whole Church, preferring before it their own conceit, grounded only upon uncertain suspicions and fears, whereof if there were at the first some probable cause when things were but raw and tender, yet now very tract of time hath itself worn that out also; unless I say thus resolved in mind they hold their pastoral charge with the comfort of a good conscience, no way grudging at that which they do, or doing that which they think themselves bound of duty to reprove, how should it possibly help or further them in their course to take such occasions as they say are requisite to be taken, and in pensive manner to tell their audience, “Brethren, our hearts’ desire is that we might enjoy the full liberty of the Gospel as in other reformed churches they do elsewhere, upon whom the heavy hand of authority hath imposed no grievous burden. But such is the misery of these our days, that so great happiness we cannot look to attain unto. Were it so, that the equity of the Law of Moses could prevail, or the zeal of Ezechias be found in the hearts of those guides and governors under whom we live; or the voice of God’s own prophets be duly heard; or the example of the Apostles of Christ be followed, yea or their precepts be answered with full and perfect obedience: these abominable rags, polluted garments, marks and sacraments of idolatry, which power as you see constraineth us to wear and conscience to abhor, had long ere this day been removed both out of sight and out of memory. But as now things stand, behold to what narrow straits we are driven. On the one side we fear the words of our Saviour Christ, ‘Wo be to them by whom scandal and offence cometh;’ on the other side at the Apostle’s speech we cannot but quake and tremble, ‘If I preach not the Gospel wo be unto me.’ Being thus hardly beset, we see not any other remedy but to hazard your souls the one way, that we may the other way endeavour to save them. Touching the offence of the weak therefore, we must  adventure it. If they perish, they perish. Our pastoral charge is God’s absolute commandment. Rather than that shall be taken from us, we are resolved to take this filth and to put it on, although we judge it to be so unfit and inconvenient, that as oft as ever we pray or preach so arrayed before you, we do as much as in us lieth to cast away your souls that are weak-minded, and to bring you unto endless perdition. But we beseech you, brethren, have care of your own safety, take heed to your steps that ye be not taken in those snares which we lay before you. And our prayer in your behalf to Almighty God is, that the poison which we offer you may never have the power to do you harm.”
Advice and counsel is best sought for at their hands which either have no part at all in the cause whereof they instruct, or else are so far engaged that themselves are to bear the greatest adventure in the success of their own counsels. The one of which two considerations maketh men the less respective, and the other the more circumspect. Those good and learned men which gave the first direction to this course had reason to wish that their own proceedings at home might be favoured abroad also, and that the good affection of such as inclined towards them might be kept alive. But if themselves had gone under those sails which they require to be hoised up, if they had been themselves to execute their own theory in this church, I doubt not but easily they would have seen being nearer at hand, that the way was not good which they took of advising men, first to wear the apparel, that thereby they might be free to continue their preaching, and then of requiring them so to preach as they might be sure they could not continue, except they imagine that laws which permit them not to do as they would, will endure them to speak as they list even against that which themselves do by constraint of laws; they would have easily seen that our people being accustomed to think evermore that thing evil which is publicly under any pretence reproved, and the men themselves worse which reprove it and use it too, it should be to little purpose for them to salve the wound by making protestations in disgrace of their own actions, with plain acknowledgment that they are scandalous, or by using fair  entreaty with the weak brethren; they would easily have seen how with us it cannot be endured to hear a man openly profess that he putteth fire to his neighbour’s house, but yet so halloweth the same with prayer that he hopeth it shall not burn. It had been therefore perhaps safer and better for ours to have observed St. Basil’s advice both in this and in all things of like nature: “Let him which approveth not his governors’ ordinances either plainly (but privately always) shew his dislike if he have λόγον ἰσχυρὸν, strong and invincible reason against them, according to the true will and meaning of Scripture; or else let him quietly with silence do that which is enjoined.” Obedience with professed unwillingness to obey is no better than manifest disobedience.

\section*{Of gesture in praying, and of different places chosen to that purpose.}
XXX. Having thus disputed whether the surplice be a fit garment to be used in the service of God, the next question whereinto we are drawn is, whether it be a thing allowable or no that the minister should say service in the chancel, or turn his face at any time from the people, or before service ended remove from the place where it was begun. By them which trouble us with these doubts we would more willingly be resolved of a greater doubt; whether it be not a kind of taking God’s name in vain to debase religion with such frivolous disputes, a sin to bestow time and labour about them. Things of so mean regard and quality, although necessary to be ordered, are notwithstanding very unsavoury when they come to be disputed of: because disputation presupposeth some difficulty in the matter which is argued, whereas in things of this nature they must be either very simple or very froward who need to be taught by disputation what is meet.
When we make profession of our faith, we stand; when we acknowledge our sins, or seek unto God for favour, we fall down: because the gesture of constancy becometh us best in the one, in the other the behaviour of humility. Some parts of our liturgy consist in the reading of the word of God,  and the proclaiming of his law, that the people may thereby learn what their duties are towards him; some consist in words of praise and thanksgiving, whereby we acknowledge unto God what his blessings are towards us; some are such as albeit they serve to singular good purpose even when there is no communion administered, nevertheless being devised at the first for that purpose are at the table of the Lord for that cause also commonly read; some are uttered as from the people, some as with them unto God, some as from God unto them, all as before his sight whom we fear, and whose presence to offend with any the least unseemliness we would be surely as loth as they who most reprehend or deride that we do.
Now because the Gospels which are weekly read do all historically declare something which our Lord Jesus Christ himself either spake, did, or suffered, in his own person, it hath been the custom of Christian men then especially in token of the greater reverence to stand, to utter certain words of acclamation, and at the name of Jesus to bow. Which harmless ceremonies as there is no  man constrained to use; so we know no reason wherefore any man should yet imagine it an unsufferable evil. It sheweth a reverend regard to the Son of God above other messengers, although speaking as from God also. And against infidels, Jews, Arians, who derogate from the honour of Jesus Christ, such ceremonies are most profitable. As for any erroneous “estimation,” advancing the Son “above the Father and the Holy Ghost,” seeing that the truth of his equality with them is a mystery so hard for the wits of mortal men to rise unto, of all heresies that which may give him superiority above them is least to be feared.
But to let go this as a matter scarce worth the speaking of, whereas if fault be in these things any where justly found, law hath referred the whole disposition and redress thereof to the ordinary of the place; they which elsewhere complain that disgrace and “injury” is offered even to the meanest parish minister, when the magistrate appointeth him what to wear, and leaveth not so small a matter as that to his own discretion, being presumed a man discreet and trusted with the care of the people’s souls, do think the gravest prelates in the land no competent judges to discern and appoint where it is fit for the minister to stand, or which way convenient to look praying. From their ordinary therefore they appeal  to themselves, finding great fault that we neither reform the thing against the which they have so long sithence given sentence, nor yet make answer unto that they bring, which is that St. Luke declaring how Peter stood up “in the midst of the disciples,” did thereby deliver an “unchangeable” rule, that “whatsoever” is done in the church “ought to be  done” in the midst of the church, and therefore not baptism to be administered in one place, marriage solemnized in another, the supper of the Lord received in a third, in a fourth sermons, in a fifth prayers to be made; that the custom which we use is Levitical, absurd, and such as hindereth the understanding of the people; that if it be meet for the minister at some time to look towards the people, if the body of the church be a fit place for some part of divine service, it must needs follow that whensoever his face is turned any other way, or any thing done any other where, it hath absurdity. “All these reasons” they say have been brought, and were hitherto never answered; besides a number of merriments and jests unanswered likewise, wherewith they have pleasantly moved much laughter at our manner of serving God. Such is their evil hap to play upon dull-spirited men. We are still persuaded that a bare denial is answer sufficient to things which mere fancy objecteth; and that the best apology to words of scorn and petulancy is Isaac’s apology to his brother Ismael, the apology which patience and silence maketh. Our answer therefore to their reasons is no; to their scoffs nothing.

\section*{Easiness of praying after our form.}
XXXI. When they object that our Book requireth nothing to be done which a child may not do as “lawfully and as well as that man wherewith the book contenteth itself,” is it their meaning that the service of God ought to be a matter of great difficulty, a labour which requireth great learning and  deep skill, or else that the book containing it should teach what men are fit to attend upon it, and forbid either men unlearned or children to be admitted thereunto? In setting down the form of common prayer, there was no need that the book should mention either the learning of a fit, or the unfitness of an ignorant minister, more than that he which describeth the manner how to pitch a field should speak of moderation and sobriety in diet.
And concerning the duty itself, although the hardness thereof be not such as needeth much art, yet surely they seem to be very far carried besides themselves to whom the dignity of public prayer doth not discover somewhat more fitness in men of gravity and ripe discretion than in “children of ten years of age,” for the decent discharge and performance of that office. It cannot be that they who speak thus should thus judge. At the board and in private it very well becometh children’s innocency to pray, and their elders to say Amen. Which being a part of their virtuous education, serveth greatly both to nourish in them the fear of God, and to put us in continual remembrance of that powerful grace which openeth the mouths of infants to sound his praise. But public prayer, the service of God in the solemn assembly of saints, is a work though easy yet withal so weighty and of such respect, that the great facility thereof is but a slender argument to prove it may be as well and as lawfully committed to children as to men of years, howsoever their ability of learning be but only to do that in decent order wherewith the book contenteth itself.
The book requireth but orderly reading. As in truth what should any prescript form of prayer framed to the minister’s hand require, but only so to be read as behoveth? We know that there are in the world certain voluntary overseers of all books, whose censure in this respect would fall as sharp on us as it hath done on many others, if delivering but a form of prayer, we should either express or include anything, more than doth properly concern prayer. The minister’s greatness or meanness of knowledge to do other things,  his aptness or insufficiency otherwise than by reading to instruct the flock, standeth in this place as a stranger with whom our form of common prayer hath nothing to do.
Wherein their exception against easiness, as if that did nourish ignorance, proceedeth altogether of a needless jealousy. I have often heard it inquired of by many, how it might be brought to pass that the Church should every where have able preachers to instruct the people; what impediments there are to hinder it, and which were the speediest way to remove them. In which consultations the multitude of parishes, the paucity of schools, the manifold discouragements which are offered unto men’s inclinations that way, the penury of the ecclesiastical estate, the irrecoverable loss of so many livings of principal value clean taken away from the Church long sithence by being appropriated, the daily bruises that spiritual promotions use to take by often falling, the want of somewhat in certain statutes which concern the state of the Church, the too great facility of many bishops, the stony hardness of too many patrons’ hearts not touched with any feeling in this case: such things oftentimes are debated, and much thought upon by them that enter into any discourse concerning any defect of knowledge in the clergy. But whosoever be found guilty, the communion book hath surely deserved least to be called in question for this fault. If all the clergy were as learned as themselves are that most complain of ignorance in others, yet our book of prayer might remain the same; and remaining the same it is, I see not how it can be a let unto any man’s skill in preaching. Which thing we acknowledge to be God’s good gift, howbeit no such necessary element that every act of religion should be thought imperfect and lame wherein there is not somewhat exacted that none can discharge but an able preacher.

\section*{The length of our service.}
XXXII. Two faults there are which our Lord and Saviour himself especially reproved in prayer: the one when ostentation did cause it to be open; the other when superstition  made it long. As therefore prayers the one way are faulty, not whensoever they be openly made, but when hypocrisy is the cause of open praying: so the length of prayer is likewise a fault, howbeit not simply, but where error and superstition causeth more than convenient repetition or continuation of speech to be used. “It is not, as some do imagine,” saith St. Augustine, “that long praying is that fault of much speaking in prayer which our Saviour did reprove; for then would not he himself in prayer have continued whole nights.” “Use in prayer no vain superfluity of words as the heathens do, for they imagine that their much speaking will cause them to be heard,” whereas in truth the thing which God doth regard is how virtuous their minds are, and not how copious their tongues in prayer; how well they think, and not how long they talk who come to present their supplications before him.
Notwithstanding forasmuch as in public prayer we are not only to consider what is needful in respect of God, but there is also in men that which we must regard; we somewhat the rather incline to length, lest over-quick despatch of a duty so important should give the world occasion to deem that the thing itself is but little accounted of, wherein but little time is bestowed. Length thereof is a thing which the gravity and weight of such actions doth require.
Besides, this benefit also it hath, that they whom earnest lets and impediments do often hinder from being partakers of the whole, have yet through the length of divine service opportunity left them at the least for access unto some reasonable part thereof.
Again it should be considered, how it doth come to pass  that we are so long. For if that very service of God in the Jewish synagogues, which our Lord did approve and sanctify with the presence of his own person, had so large portions of the Law and the Prophets together with so many prayers and psalms read day by day as equal in a manner the length of ours, and yet in that respect was never thought to deserve blame, is it now an offence that the like measure of time is bestowed in the like manner? Peradventure the Church hath not now the leisure which it had then, or else those things whereupon so much time was then well spent, have sithence that lost their dignity and worth. If the reading of the Law, the Prophets, and Psalms, be a part of the service of God as needful under Christ as before, and the adding of the New Testament as profitable as the ordaining of the Old to be read; if therewith instead of Jewish prayers it be also for the good of the Church to annex that variety which the Apostle doth commend, seeing that the time which we spend is no more than the orderly performance of these things necessarily requireth, why are we thought to exceed in length? Words be they never so few are too many when they benefit not the hearer. But he which speaketh no more than edifieth is undeservedly reprehended for much speaking.
That as “the Devil under colour of long prayer drave preaching out of the Church” heretofore, so we “in appointing so long time of prayers and reading, whereby the less can be spent in preaching, maintain an unpreaching ministry,” is neither advisedly nor truly spoken. They reprove long prayer, and yet acknowledge it to be in itself a thing commendable. For so it must needs be, if the Devil have used it as “a colour” to hide his malicious practices. When malice would work that which is evil, and in working avoid  the suspicion of any evil intent, the colour wherewith it overcasteth itself is always a fair and plausible pretence of seeking to further that which is good. So that if we both retain that good which Satan hath pretended to seek, and avoid the evil which his purpose was to effect, have we not better prevented his malice than if as he hath under colour of long prayer driven preaching out of the Church, so we should take the quarrel of sermons in hand and revenge their cause by requital, thrusting prayer in a manner out of doors under colour of long preaching?
In case our prayers being made at their full length did necessarily enforce sermons to be the shorter, yet neither were this to uphold and maintain an “unpreaching ministry,” unless we will say that those ancient Fathers, Chrysostom, Augustine, Leo, and the rest, whose homilies in that consideration were shorter for the most part than our sermons are, did then not preach when their speeches were not long. The necessity of shortness causeth men to cut off impertinent discourses, and to comprise much matter in few words. But neither doth it maintain inability, nor at all prevent opportunity of preaching, as long as a competent time is granted for that purpose.
“An hour and a half” is, they say, in reformed churches “ordinarily” thought reasonable “for their whole liturgy or service.” Do we then continue as Ezra did in reading the Law from morning till midday? or as the Apostle St. Paul did in prayer and preaching till men through weariness be taken up dead at our feet? The huge length whereof they make such complaint is but this, that if our whole form of prayer be read, and besides an hour allowed for a sermon, we spend ordinarily in both more time than they do by half an hour. Which half-hour being such a  matter as the “age of some and the infirmity of other some are not able to bear;” if we have any sense of the “common imbecility,” if any care to preserve men’s wits from being broken with the very “bent of so long attention,” if any love or desire to provide that things most holy be not with “hazard” of men’s souls abhorred and “loathed,” this half-hour’s tediousness must be remedied, and that only by cutting off the greatest part of our common prayer. For no other remedy will serve to help so dangerous an inconvenience.

\section*{Instead of such prayers as the primitive Churches have used, and those that the reformed now use, we have (they say) divers short cuts or shreddings, rather wishes than prayers.}
XXXIII. The brethren in Egypt (saith St. Augustine, epist. 121,) are reported to have many prayers, but every of them very short, as if they were darts thrown out with a kind of sudden quickness, lest that vigilant and erect attention of mind, which in prayer is very necessary, should be wasted or dulled through continuance, if their prayers were few and long. But that which St. Augustine doth allow they  condemn. Those prayers whereunto devout minds have added a piercing kind of brevity, as well in that respect which we have already mentioned, as also thereby the better to express that quick and speedy expedition, wherewith ardent affections, the very wings of prayer, are delighted to present our suits in heaven, even sooner than our tongues can devise to utter them, they in their mood of contradiction spare not openly to deride, and that with so base terms as do very ill beseem men of their gravity. Such speeches are scandalous, they savour not of God in him that useth them, and unto virtuously disposed minds they are grievous corrosives. Our case were miserable, if that wherewith we most endeavour to please God were in his sight so vile and despicable as men’s disdainful speech would make it.

\section*{Lessons intermingled with our prayers.}
XXXIV. Again, forasmuch as effectual prayer is joined with a vehement intention of the inferior powers of the soul, which cannot therein long continue without pain, it hath been therefore thought good so by turns to interpose still somewhat for the higher part of the mind, the understanding, to work upon, that both being kept in continual exercise with variety, neither might feel any great weariness, and yet each be a spur to other. For prayer kindleth our desire to behold God by speculation; and the mind delighted with that contemplative sight of God, taketh every where new inflammations to pray, the riches of the mysteries of heavenly wisdom continually stirring up in us correspondent desires towards them. So that he which prayeth in due sort is thereby made the more attentive to hear, and he which heareth the more earnest to pray, for the time which we bestow as well in the one as the other.
But for what cause soever we do it, this intermingling of lessons with prayers is in their taste a thing as unsavoury, and as unseemly in their sight, as if the like should be done in suits and supplications before some mighty prince of the world. Our speech to worldly superiors we frame in such sort as serveth best to inform and persuade the minds of them, who otherwise neither could nor would greatly regard our necessities: whereas, because we know that God is indeed a King, but a great king, who understandeth all things beforehand, which no other king besides doth, a king which needeth not to be informed what we lack, a king readier to grant than we to make our requests; therefore in prayer we do not so much respect what precepts art delivereth touching the method of persuasive utterance in the presence of great men, as what doth most avail to our own edification in piety and godly zeal. If they on the contrary side do think that the same rules of decency which serve for things done unto terrene powers should universally decide what is fit in the service of God; if it be their meaning to hold it for a maxim, that the Church must deliver her public supplications unto God in no other form of speech than such as were decent, if suit should be made to the great Turk, or some other monarch, let them apply their own rule unto their own form of common prayer. Suppose that the people of a whole town with some chosen man before them did continually twice or thrice in a week resort to their king, and every time they come first acknowledge themselves guilty of rebellions and treasons, then sing a song, after that explain some statute of the land to the standers-by, and therein  spend at the least an hour, this done, turn themselves again to the king, and for every sort of his subjects crave somewhat of him, at the length sing him another song, and so take their leave. Might not the king well think that either they knew not what they would have, or else that they were distracted in mind, or some other such like cause of the disorder of their supplication? This form of suing unto kings were absurd. This form of praying unto God they allow.
When God was served with legal sacrifices, such was the miserable and wretched disposition of some men’s minds, that the best of every thing they had being culled out for themselves, if there were in their flocks any poor starved or diseased thing not worth the keeping, they thought it good enough for the altar of God, pretending (as wise hypocrites do when they rob God to enrich themselves) that the fatness of calves doth benefit him nothing; to us the best things are most profitable, to him all as one if the mind of the offerer be good, which is the only thing he respecteth. In reproof of which their devout fraud, the Prophet Malachi allegeth that gifts are offered unto God not as supplies of his want indeed, but yet as testimonies of that affection wherewith we acknowledge and honour his greatness. For which cause, sith the greater they are whom we honour, the more regard we have to the quality and choice of those presents which we bring them for honour’s sake, it must needs follow that if we dare not disgrace our worldly superiors with offering unto them such refuse as we bring unto God himself, we shew plainly that our acknowledgment of his greatness is but feigned, in heart we fear him not so much as we dread them. “If ye offer the blind for sacrifice it is not evil. Offer it now unto  thy prince. Will he be content, or accept thy person? saith the Lord of hosts. Cursed be the deceiver which hath in his flock a male, and having made a vow sacrificeth unto the Lord a corrupt thing. For I am a great king, saith the Lord of hosts.” Should we hereupon frame a rule that what form of speech or behaviour soever is fit for suitors in a prince’s court, the same and no other beseemeth us in our prayers to Almighty God?

\section*{The number of our prayers for earthly things, and our oft rehearsing of the Lord’s Prayer.}
XXXV. But in vain we labour to persuade them that any thing can take away the tediousness of prayer, except it be brought to the very same both measure and form which themselves assign. Whatsoever therefore our liturgy hath more than theirs, under one devised pretence or other they cut it off. We have of prayers for earthly things in their opinion too great a number; so oft to rehearse the Lord’s Prayer in so small a time is as they think a loss of time; the people’s praying after the minister they say both wasteth time, and also maketh an unpleasant sound; the Psalms they would not have to be made (as they are) a part of our common prayer, nor to be sung or said by turns, nor such music to be used with them; those evangelical hymns they allow not to stand in our liturgy; the Litany, the Creed of Athanasius, the sentence of Glory wherewith we use to conclude psalms, these things they cancel, as having been instituted  in regard of occasions peculiar to the times of old, and as being therefore now superfluous.
Touching prayers for things earthly, we ought not to think that the Church hath set down so many of them without cause. They peradventure, which find this fault, are of the same affection with Salomon, so that if God should offer to grant them whatsoever they ask, they would neither crave riches, nor length of days, nor yet victory over their enemies, but only an understanding heart: for which cause themselves having eagles’ wings, are offended to see others fly so near the ground. But the tender kindness of the Church of God it very well beseemeth to help the weaker sort, which are by so great odds more in number, although some few of the perfecter and stronger may be therewith for a time displeased.
Ignorant we are not, that of such as resorted to our Saviour Christ being present on earth, there came not any unto him with better success for the benefit of their souls’ everlasting happiness, than they whose bodily necessities gave them the first occasion to seek relief, where they saw willingness and ability of doing every way good unto all.
The graces of the Spirit are much more precious than worldly benefits; our ghostly evils of greater importance than any harm which the body feeleth. Therefore our desires to heavenward should both in measure and number no less exceed than their glorious object doth every way excel in value. These things are true and plain in the eye of a perfect judgment. But yet it must be withal considered, that the greatest part of the world are they which be farthest from perfection. Such being better able by sense to discern the wants of this present life, than by spiritual capacity to apprehend things above sense, which tend to their happiness in the world to come, are in that respect the more apt to apply their minds even with hearty affection and zeal at the least unto those branches of public prayer, wherein their own particular is moved. And by this mean there stealeth upon them a double benefit: first because that good affection, which things of smaller account have once set on work, is by so much the more easily raised higher; and secondly in that the very  custom of seeking so particular aid and relief at the hands of God, doth by a secret contradiction withdraw them from endeavouring to help themselves by those wicked shifts which they know can never have his allowance, whose assistance their prayer seeketh. These multiplied petitions of worldly things in prayer have therefore, besides their direct use, a service, whereby the Church underhand, through a kind of heavenly fraud, taketh therewith the souls of men as with certain baits.
If then their calculation be true, (for so they reckon,) that a full third of our prayers be allotted unto earthly benefits, for which our Saviour in his platform hath appointed but one petition amongst seven, the difference is without any great disagreement; we respecting what men are, and doing that which is meet in regard of the common imperfection; our Lord contrariwise proposing the most absolute proportion that can be in men’s desires, the very highest mark whereat we are able to aim.
For which cause also our custom is both to place it in the front of our prayers as a guide, and to add it in the end of some principal limbs or parts as a complement which fully perfecteth whatsoever may be defective in the rest. Twice we rehearse it ordinarily, and oftener as occasion requireth more solemnity or length in the form of divine service; not mistrusting, till these new curiosities sprang up, that ever any man would think our labour herein mispent, the time wastefully consumed, and the office itself made worse by so repeating that which otherwise would more hardly be made familiar to the simpler sort; for the good of whose souls there is not  in Christian religion any thing of like continual use and force throughout every hour and moment of their whole lives.
I mean not only because prayer, but because this very prayer, is of such efficacy and necessity. For that our Saviour did but set men a bare example how to contrive or devise prayers of their own, and no way bind them to use this, is no doubt an error. John the Baptist’s disciples which had been always brought up in the bosom of God’s Church from the time of their first infancy till they came to the school of John, were not so brutish that they could be ignorant how to call upon the name of God; but of their master they had received a form of prayer amongst themselves, which form none did use saving his disciples, so that by it as by a mark of special difference they were known from others. And of this the Apostles having taken notice, they request that as John had taught his, so Christ would likewise teach them to pray.
Tertullian and St. Augustine do for that cause term it Orationem legitimam, the Prayer which Christ’s own law hath tied his Church to use in the same prescript form of words wherewith he himself did deliver it; and therefore what part of the world soever we fall into, if Christian religion have been there received, the ordinary use of this very prayer hath with equal continuance accompanied the same as one of the principal and most material duties of honour done to Jesus Christ. “Seeing that we have” (saith St. Cyprian) “an Advocate with the Father for our sins, when we that have sinned come to seek for pardon, let us allege unto God the words which our Advocate hath taught. For sith his promise is our plain warrant, that in his name what we ask we shall receive, must we not needs much the rather obtain that for which we sue if not only his name do countenance but also his speech present our requests?”
Though men should speak with the tongues of Angels, yet  words so pleasing to the ears of God as those which the Son of God himself hath composed were not possible for men to frame. He therefore which made us to live hath also taught us to pray, to the end that speaking unto the Father in the Son’s own prescript form without scholy or gloss of ours, we may be sure that we utter nothing which God will either disallow or deny. Other prayers we use many besides this, and this oftener than any other; although not tied so to do by any commandment of Scripture, yet moved with such considerations as have been before set down: the causeless dislike whereof which others have conceived, is no sufficient reason for us as much as once to forbear in any place a thing which uttered with true devotion and zeal of heart affordeth to God himself that glory, that aid to the weakest sort of men, to the most perfect that solid comfort which is unspeakable.

\section*{The people’s saying after the minister.}
XXXVI. With our Lord’s Prayer they would find no fault, so that they might persuade us to use it before or after sermons only (because so their manner is) and not (as all Christian people have been of old accustomed) insert it so often into the liturgy. But the people’s custom to repeat any thing after the minister, they utterly mislike. Twice we appoint that the words which the minister first pronounceth, the whole congregation shall repeat after him. As first in the public confession of sins, and again in rehearsal of our Lord’s Prayer presently after the blessed Sacrament of his  Body and Blood received. A thing no way offensive, no way unfit or unseemly to be done, although it had been so appointed oftener than with us it is. But surely with so good reason it standeth in those two places, that otherwise to order it were not in all respects so well.
Could there be any thing devised better than that we all at our first access unto God by prayer should acknowledge meekly our sins, and that not only in heart but with tongue, all which are present being made ear-witnesses even of every man’s distinct and deliberate assent unto each particular branch of a common indictment drawn against ourselves? How were it possible that the Church should any way else with such ease and certainty provide, that none of her children may as Adam dissemble that wretchedness, the penitent confession whereof is so necessary a preamble, especially to common prayer?
In like manner if the Church did ever devise a thing fit and convenient, what more than this, that when together we have all received those heavenly mysteries wherein Christ imparteth himself unto us, and giveth visible testification of our blessed communion with him, we should in hatred of all heresies, factions, and schisms, the pastor as a leader, the people as willing followers of him step by step declare openly ourselves united as brethren in one, by offering up with all our hearts and tongues that most effectual supplication, wherein he unto whom we offer it hath himself not only comprehended all our necessities, but in such sort also framed every petition, as might most naturally serve for many, and doth though not always require yet always import a multitude of speakers together? For which cause communicants have ever used it, and we at that time by the form of our very utterance do shew we use it, yea every word and syllable of it, as communicants.
In the rest we observe that custom whereunto St. Paul alludeth, and whereof the Fathers of the Church in their writings make often mention, to shew indefinitely what was  done, but not universally to bind for ever all prayers unto one only fashion of utterance.
The reasons which we have alleged induce us to think it still “a good work,” which they in their pensive care for the well bestowing of time account “waste.” As for unpleasantness of sound if it happen, the good of men’s souls doth either deceive our ears that we note it not, or arm them with patience to endure it. We are not so nice as to cast away a sharp knife, because the edge of it may sometimes grate. And such subtle opinions as few but Utopians are likely to fall into, we in this climate do not greatly fear.

\section*{Our manner of reading the Psalms otherwise than the rest of the Scripture.}
XXXVII. The complaint which they make about Psalms and Hymns, might as well be overpast without any answer, as it is without any cause brought forth. But our desire is to content them if it may be, and to yield them a just reason even of the least things wherein undeservedly they have but as much as dreamed or suspected that we do amiss. They seem sometimes so to speak, as if it greatly offended them, that such Hymns and Psalms as are Scripture should in common prayer be otherwise used than the rest of the Scripture is wont: sometime displeased they are at the artificial music which we add unto psalms of this kind, or of any other nature else; sometime the plainest and the most intelligible rehearsal of them yet they savour not, because it is done by interlocution, and with a mutual return of sentences from side to side.
They are not ignorant what difference there is between other parts of Scripture and Psalms. The choice and flower of all things profitable in other books the Psalms do both more briefly contain, and more movingly also express, by reason of that poetical form wherewith they are written. The ancient when they speak of the Book of Psalms use to fall into large discourses, shewing how this part above the rest doth of purpose set forth and celebrate all the considerations and operations which belong to God; it magnifieth the holy  meditations and actions of divine men; it is of things heavenly an universal declaration, working in them whose hearts God inspireth with the due consideration thereof, an habit or disposition of mind whereby they are made fit vessels both for receipt and for delivery of whatsoever spiritual perfection. What is there necessary for man to know which the Psalms are not able to teach? They are to beginners an easy and familiar introduction, a mighty augmentation of all virtue and knowledge in such as are entered before, a strong confirmation to the most perfect amongst others. Heroical magnanimity, exquisite justice, grave moderation, exact wisdom, repentance unfeigned, unwearied patience, the mysteries of God, the sufferings of Christ, the terrors of wrath, the comforts of grace, the works of Providence over this world, and the promised joys of that world which is to come, all good necessarily to be either known or done or had, this one celestial fountain yieldeth. Let there be any grief or disease incident into the soul of man, any wound or sickness named, for which there is not in this treasure-house a present comfortable remedy at all times ready to be found. Hereof it is that we covet to make the Psalms especially familiar unto all. This is the very cause why we iterate the Psalms oftener than any other part of Scripture besides; the cause wherefore we inure the people together with their minister, and not the minister alone to read them as other parts of Scripture he doth.

\section*{Of Music with Psalms.}
XXXVIII. Touching musical harmony whether by instrument or by voice, it being but of high and low in sounds a due proportionable disposition, such notwithstanding is the force thereof, and so pleasing effects it hath in that very part of man which is most divine, that some have been thereby induced to think that the soul itself by nature is or hath in it harmony. A thing which delighteth all ages and beseemeth all states; a thing as seasonable in grief as in joy; as decent being added unto actions of greatest weight and solemnity, as being used when men most sequester themselves from action. The reason hereof is an admirable facility which music hath to express and represent to the mind, more inwardly than any other sensible mean, the very standing, rising, and falling, the  very steps and inflections every way, the turns and varieties of all passions whereunto the mind is subject; yea so to imitate them, that whether it resemble unto us the same state wherein our minds already are, or a clean contrary, we are not more contentedly by the one confirmed, than changed and led away by the other. In harmony the very image and character even of virtue and vice is perceived, the mind delighted with their resemblances, and brought by having them often iterated into a love of the things themselves. For which cause there is nothing more contagious and pestilent than some kinds of harmony; than some nothing more strong and potent unto good. And that there is such a difference of one kind from another we need no proof but our own experience, inasmuch as we are at the hearing of some more inclined unto sorrow and heaviness; of some, more mollified and softened in mind; one kind apter to stay and settle us, another to move and stir our affections; there is that draweth to a marvellous grave and sober mediocrity, there is also that carrieth as it were into ecstasies, filling the mind with an heavenly joy and for the time in a manner severing it from the body. So that although we lay altogether aside the consideration of ditty or matter, the very harmony of sounds being framed in due sort and carried from the ear to the spiritual faculties of our souls, is by a native puissance and efficacy greatly available to bring to a perfect temper whatsoever is there troubled, apt as well to quicken the spirits as to allay that which is too eager, sovereign against melancholy and despair, forcible to draw forth tears of devotion if the mind be such as can yield them, able both to move and to moderate all affections.
The Prophet David having therefore singular knowledge not in poetry alone but in music also, judged them both to be things most necessary for the house of God, left behind him to that purpose a number of divinely indited poems, and was farther the author of adding unto poetry melody in public prayer, melody both vocal and instrumental, for the raising up of men’s hearts, and the sweetening of their affections towards God. In which considerations the Church of Christ doth likewise at this present day retain it as an ornament to God’s service, and an help to our own  devotion. They which, under pretence of the Law ceremonial abrogated, require the abrogation of instrumental music, approving nevertheless the use of vocal melody to remain, must shew some reason wherefore the one should be thought a legal ceremony and not the other.
In church music curiosity and ostentation of art, wanton or light or unsuitable harmony, such as only pleaseth the ear, and doth not naturally serve to the very kind and degree of those impressions, which the matter that goeth with it leaveth or is apt to leave in men’s minds, doth rather blemish and disgrace that we do than add either beauty or furtherance unto it. On the other side, these faults prevented, the force and efficacy of the thing itself, when it drowneth not utterly but fitly suiteth with matter altogether sounding to the praise of God, is in truth most admirable, and doth much edify if not the understanding because it teacheth not, yet surely the affection, because therein it worketh much. They must  have hearts very dry and tough, from whom the melody of psalms doth not sometime draw that wherein a mind religiously affected delighteth. Be it as Rabanus Maurus observeth, that at the first the Church in this exercise was more simple and plain than we are, that their singing was little more than only a melodious kind of pronunciation, that the custom which we now use was not instituted so much for their cause which are spiritual, as to the end that into grosser and heavier minds, whom bare words do not easily move, the sweetness of melody might make some entrance for good things. St. Basil himself acknowledging as much, did not think that from such inventions the least jot of estimation and credit thereby should be derogated: “For” (saith he) “whereas the Holy Spirit saw that mankind is unto virtue hardly drawn, and that righteousness is the less accounted of by reason of the proneness of our affections to that which delighteth; it pleased the wisdom of the same Spirit to borrow from melody that pleasure, which mingled with heavenly mysteries, causeth the smoothness and softness of that which toucheth the ear, to convey as it were by stealth the treasure of good things into man’s mind. To this purpose were those harmonious tunes of psalms devised for us, that they which are either in years but young, or touching perfection of virtue as yet not grown to ripeness, might when they think they sing, learn. O the wise conceit of that heavenly Teacher, which hath by his skill, found out a way, that doing those things wherein we delight, we may also learn that whereby we profit!”

\section*{Of singing or saying Psalms, and other parts of Common Prayer wherein the people and the minister answer one another by course.}
XXXIX. And if the Prophet David did think that the very meeting of men together, and their accompanying one another to the house of God, should make the bond of their love insoluble, and tie them in a league of inviolable amity (Psal. lv. 14); how much more may we judge it reasonable to hope, that the like effects may grow in each of the people towards other, in them all towards their pastor, and in their pastor towards every of them, between whom there daily and interchangeably pass, in the hearing of God himself, and in the presence of his holy Angels, so many heavenly acclamations, exultations, provocations, petitions, songs of comfort, psalms of praise and thanksgiving: in all which particulars, as when the pastor maketh their suits, and they with one voice testify a general assent thereunto; or when he joyfully beginneth, and they with like alacrity follow, dividing between them the sentences wherewith they strive which shall most shew his own and stir up others’ zeal, to the glory of that God whose name they magnify; or when he proposeth unto God their necessities, and they their own requests for relief in every of them; or when he lifteth up his voice like a trumpet to proclaim unto them the laws of God, they adjoining though not as Israel did by way of generality a cheerful promise, “All that the Lord hath commanded we will do,” yet that which God doth no less approve, that which savoureth more of meekness, that which testifieth rather a feeling knowledge of our common imbecility, unto the several  branches thereof, several, lowly and humble requests for grace at the merciful hands of God to perform the thing which is commanded; or when they wish reciprocally each other’s ghostly happiness; or when he by exhortation raiseth them up, and they by protestation of their readiness declare he speaketh not in vain unto them: these interlocutory forms of speech what are they else, but most effectual partly testifications and partly inflammations of all piety?
When and how this custom of singing by course came up in the Church it is not certainly known. Socrates maketh Ignatius the Bishop of Antioch in Syria the first beginner thereof, even under the Apostles themselves. But against Socrates they set the authority of Theodoret, who draweth the original of it from Antioch as Socrates doth; howbeit ascribing the invention to others, Flavian and Diodore, men which constantly stood in defence of the apostolic faith against the Bishop of that church, Leontius, a favourer of the Arians. Against both Socrates and Theodoret, Platina is brought as a witness, to testify that Damasus Bishop of Rome began it in his time. Of the Latin church it may be true which Platina saith. And therefore the eldest of that church which maketh any mention thereof is St. Ambrose, Bishop of  Milan at the same time when Damasus was of Rome. Amongst the Grecians St. Basil having brought it into his church before they of Neocæsarea used it, Sabellius the heretic and Marcellus took occasion thereat to incense the churches against him, as being an author of new devices in the service of God. Whereupon to avoid the opinion of novelty and singularity, he allegeth for that which himself did the example of the churches of Egypt, Libya, Thebes, Palestina, the Arabians, Phœnicians, Syrians, Mesopotamians, and in a manner all that reverenced the custom of singing psalms together. If the Syrians had it then before Basil, Antioch the mother church of those parts must needs have used it before Basil, and consequently before Damasus. The question is then how long before, and whether so long that Ignatius or as ancient as Ignatius may be probably thought the first inventors. Ignatius in Trajan’s days suffered martyrdom. And of the churches in Pontus and Bithynia to Trajan the emperor his own vicegerent there affirmeth, that  the only crime he knew of them was, they used to meet together at a certain day, and to praise Christ with hymns as a God, secum invicem, “one to another amongst themselves.” Which for any thing we know to the contrary might be the selfsame form which Philo Judæus expresseth, declaring how the Essenes were accustomed with hymns and psalms to honour God, sometime all exalting their voices together in one, and sometime one part answering another, wherein as he thought, they swerved not much from the pattern of Moses and Miriam.
Whether Ignatius did at any time hear the angels praising God after that sort or no, what matter is it? If Ignatius did not, yet one which must be with us of greater authority did. “I saw the Lord (saith the Prophet Esay) on an high throne; the Seraphims stood upon it; one cried to another saying, Holy, Holy, Holy, Lord God of Hosts, the whole world is full of his glory.”
But whosoever were the author, whatsoever the time, whencesoever the example of beginning this custom in the Church of Christ; sith we are wont to suspect things only before trial, and afterwards either to approve them as good, or if we find them evil, accordingly to judge of them; their counsel must needs seem very unseasonable, who advise men now to suspect that wherewith the world hath had by their own account twelve hundred years’ acquaintance and upwards, enough to take away suspicion and jealousy. Men know by this time, if ever they will know, whether it be good or evil which hath been so long retained.
As for the Devil, which way it should greatly benefit him to have this manner of singing psalms accounted an invention of Ignatius, or an imitation of the angels of heaven,  we do not well understand. But we very well see in them who thus plead a wonderful celerity of discourse. For perceiving at the first but only some cause of suspicion and fear lest it should be evil, they are presently in one and the selfsame breath resolved, that “what beginning soever it had, there is no possibility it should be good.” The potent arguments which did thus suddenly break in upon them and overcome them are first, that it is not unlawful for the people all jointly to praise God in singing of psalms; secondly, that they are not any where forbidden by the law of God to sing every verse of the whole psalm both with heart and voice quite and clean throughout; thirdly, that it cannot be understood what is sung after our manner. Of which three, forasmuch as lawfulness to sing one way proveth not another way inconvenient, the former two are true allegations, but they lack strength to accomplish their desire; the third so strong that it might persuade, if the truth thereof were not doubtful.
And shall this enforce us to banish a thing which all Christian churches in the world have received; a thing which so many ages have held; a thing which the most approved councils and laws have so oftentime ratified; a thing which was never found to have any inconvenience in it; a thing which always heretofore the best men and wisest governors of God’s people did think they could never commend  enough; a thing, which as Basil was persuaded, did both strengthen the meditation of those holy words which were uttered in that sort, and serve also to make attentive, and to raise up the hearts of men; a thing whereunto God’s people of old did resort, with hope and thirst that thereby especially their souls might be edified; a thing which filleth the mind with comfort and heavenly delight, stirreth up flagrant desires and affections correspondent unto that which the words contain, allayeth all kind of base and earthly cogitations, banisheth and driveth away those evil secret suggestions which our invisible enemy is always apt to minister, watereth the heart to the end it may fructify, maketh the virtuous in trouble full of magnanimity and courage, serveth as a most approved remedy against all doleful and heavy accidents which befall men in this present life, to conclude, so fitly accordeth with the Apostle’s own exhortation, “Speak to yourselves in psalms and hymns and spiritual songs, making melody, and singing to the Lord in your hearts,” that surely there is more cause to fear lest the want thereof be a maim, than the use a blemish to the service of God.
It is not our meaning, that what we attribute unto the Psalms should be thought to depend altogether on that only form of singing or reading them by course as with us the manner is; but the end of our speech is to shew that because the Fathers of the Church, with whom the selfsame custom was so many ages ago in use, have uttered all these things concerning the fruit which the Church of God did then reap, observing that and no other form, it may be justly avouched that we ourselves retaining it and besides it also the other more newly and not unfruitfully devised, do neither want that good which the later invention can afford, nor lose any thing of that for which the ancient so oft and so highly commend the former. Let novelty therefore in this give over endless contradictions, and let ancient custom prevail.

\section*{Of Magnificat, Benedictus, and Nunc Dimittis.}
XL. We have already given cause sufficient for the great conveniency and use of reading the Psalms oftener than other Scriptures. Of reading or singing likewise Magnificat,  Benedictus, and Nunc Dimittis oftener than the rest of the Psalms, the causes are no whit less reasonable, so that if the one may very well monthly the other may as well even daily be iterated. They are songs which concern us so much more than the songs of David, as the Gospel toucheth us more than the Law, the New Testament than the Old. And if the Psalms for the excellency of their use deserve to be oftener repeated than they are, but that the multitude of them permitteth not any oftener repetition, what disorder is it if these few Evangelical Hymns which are in no respect less worthy, and may be by reason of their paucity imprinted with much more ease in all men’s memories, be for that cause every day rehearsed? In our own behalf it is convenient and orderly enough that both they and we make day by day prayers and supplications the very same; why not as fit and convenient to magnify the name of God day by day with certain the very selfsame psalms of praise and thanksgiving? Either let them not allow the one, or else cease to reprove the other.
For the ancient received use of intermingling hymns and psalms with divine readings, enough hath been written. And if any may fitly serve unto that purpose, how should it better have been devised than that a competent number of the old being first read, these of the new should succeed in the place where now they are set? In which place notwithstanding there is joined with Benedictus the hundredth Psalm; with Magnificat the ninety-eighth; the sixty-seventh with Nunc Dimittis, and in every of them the choice left free for the minister to use indifferently the one or the other. Seeing therefore they pretend no quarrel at other psalms, which are in like manner appointed also to be daily read, why do these so much offend and displease their taste? They are the first gratulations wherewith our Lord and Saviour was joyfully received at his entrance into the world by such as in their hearts, arms, and very bowels embraced him; being prophetical  discoveries of Christ already present, whose future coming the other psalms did but foresignify, they are against the obstinate incredulity of the Jews, the most luculent testimonies that Christian religion hath; yea the only sacred hymns they are that Christianity hath peculiar unto itself, the other being songs too of praise and thanksgiving, but songs wherewith as we serve God, so the Jew likewise.
And whereas they tell us these songs were fit for that purpose, when Simeon and Zachary and the Blessed Virgin uttered them, but cannot so be to us which have not received like benefit; should they not remember how expressly Ezechias amongst many other good things is commended for this also, that the praises of God were through his appointment daily set forth by using in public divine service the songs of David and Asaph unto that very end? Either there wanted wise men to give Ezechias advice, and to inform him of that which in his case was as true as it is in ours, namely, that without some inconvenience and disorder he could not appoint those Psalms to be used as ordinary prayers, seeing that although they were songs of thanksgiving such as David and Asaph had special occasion to use, yet not so the whole Church and people afterwards whom like occasions did not befall: or else Ezechias was persuaded as we are that the praises of God in the mouths of his saints are not so restrained to their own particular, but that others may both conveniently and fruitfully use them: first, because the mystical communion of all faithful men is such as maketh every one to be interessed in those precious blessings which any one of them receiveth at God’s hands: secondly, because when any thing is spoken to extol the goodness of God whose mercy endureth for ever, albeit the very particular occasion whereupon it riseth do come no more, yet the fountain continuing the same, and yielding other new effects which are but only in some sort proportionable, a small resemblance between the benefits which we and others have received, may serve to make the  same words of praise and thanksgiving fit though not equally in all circumstances fit for both; a clear demonstration whereof we have in all the ancient Fathers’ commentaries and meditations upon the Psalms: last of all because even when there is not as much as the show of any resemblance, nevertheless by often using their words in such manner, our minds are daily more and more inured with their affections.

\section*{Of the Litany.}
XLI. The public estate of the Church of God amongst the Jews hath had many rare and extraordinary occurrents, which also were occasions of sundry open solemnities and offices, whereby the people did with general consent make show of correspondent affection towards God. The like duties appear usual in the ancient Church of Christ, by that which Tertullian speaketh of Christian women matching themselves with infidels. “She cannot content the Lord with performance of his discipline, that hath at her side a vassal whom Satan hath made his vice-agent to cross whatsoever the faithful should do. If her presence be required at the time of Station or standing prayer, he chargeth her at no time  but that to be with him in his baths; if a fasting-day come he hath on that day a banquet to make; if there be cause for the church to go forth in solemn procession, his whole family have such business come upon them that no one can be spared.”
These processions as it seemeth were first begun for the interring of holy martyrs, and the visiting of those places where they were entombed. Which thing the name itself applied by heathens unto the office of exequies, and partly the speeches of some of the ancient delivered concerning Christian processions, partly also the very dross which superstition thereunto added, I mean the custom of invoking saints in processions, heretofore usual, do strongly insinuate. And as things invented to one purpose are by use easily converted to more, it grew that supplications with this solemnity  for the appeasing of God’s wrath, and the averting of public evils, were of the Greek church termed Litanies;1 Rogations, of the Latin. To the people of Vienna (Mamercus being their Bishop, about 450 years after Christ) there befell many things, the suddenness and strangeness whereof so amazed the hearts of all men, that the city they began to forsake as a place which heaven did threaten with imminent ruin. It beseemed not the person of so grave a prelate to be either utterly without counsel as the rest were, or in a common perplexity to shew himself alone secure. Wherefore as many as remained he earnestly exhorteth to prevent portended calamities, using those virtuous and holy means wherewith others in like case have prevailed with God. To which purpose he perfecteth the Rogations or Litanies before in use, and addeth unto them that which the present necessity required. Their good success moved Sidonius Bishop of Arverna to use the same so corrected Rogations, at such time as he and his people were  after afflicted with famine, and besieged with potent adversaries. For till the empty name of the empire came to be settled in Charles the Great, the fall of the Romans’ huge dominion concurring with other universal evils, caused those times to be days of much affliction and trouble throughout the world. So that Rogations or Litanies were then the very strength, stay, and comfort of God’s Church. Whereupon in the year 506 it was by the council of Aurelia decreed, that the whole Church should bestow yearly at the feast of Pentecost three days in that kind of processionary service. About half an hundred years after, to the end that the Latin churches which all observed this custom might not vary in the order and form of those great Litanies which were so solemnly every where exercised, it was thought convenient by Gregory the First and the best of that name to draw the flower of them all into one.
But this iron began at the length to gather rust. Which thing the synod of Colen saw and in part redressed within that province, neither denying the necessary use for which such Litanies serve, wherein God’s clemency and mercy is desired by public suit, to the end that plagues, destructions, calamities, famines, wars, and all other the like adversities, which for our manifold sins we have always cause to fear, may  be turned away from us and prevented through his grace; nor yet dissembling the great abuse whereunto as sundry other things so this had grown by men’s improbity and malice, to whom that which was devised for the appeasing of God’s displeasure gave opportunity of committing things which justly kindled his wrath. For remedy whereof it was then thought better, that these and all other supplications or processions should be no where used but only within the walls of the house of God, the place sanctified unto prayer. And by us not only such inconveniences being remedied, but also whatsoever was otherwise amiss in form or matter, it now remaineth a work, the absolute perfection whereof upbraideth with error or somewhat worse them whom in all parts it doth not satisfy.
As therefore Litanies have been of longer continuance than that we should make either Gregory or Mamercus the author of them, so they are of more permanent use than that now the Church should think it needeth them not. What dangers at any time are imminent, what evils hang over our heads, God doth know and not we. We find by daily experience that those calamities may be nearest at hand, readiest to break in suddenly upon us, which we in regard of times or circumstances may imagine to be farthest off. Or if they do not indeed approach, yet such miseries as being present all men are apt to bewail with tears, the wise by their prayers should rather prevent. Finally, if we for ourselves had a privilege of immunity, doth not true Christian charity require that whatsoever any part of the world, yea any one of all our brethren elsewhere doth either suffer or fear, the same we account as our own burden? What one petition is there found in the whole Litany, whereof we shall ever be able at any time to say that no man living needeth the grace or benefit therein craved at God’s hands? I am not able to express how much it doth grieve me, that things of principal excellency should be thus bitten at, by men whom God hath endued with graces both of wit and learning for better purposes.

\section*{Of Athanasius’s Creed, and Gloria Patri.}
XLII. We have from the Apostles of our Lord Jesus Christ received that brief confession of faith which hath been  always a badge of the Church, a mark whereby to discern Christian men from Infidels and Jews. “This faith received from the Apostles and their disciples,” saith Irenæus, “the Church though dispersed throughout the world, doth notwithstanding keep as safe as if it dwelt within the walls of some one house, and as uniformly hold, as if it had but one only heart and soul; this as consonantly it preacheth, teacheth, and delivereth, as if but one tongue did speak for all. As one sun shineth to the whole world, so there is no faith but this one published, the brightness whereof must enlighten all that come to the knowledge of the truth.” “This rule,” saith Tertullian, “Christ did institute; the stream and current of this rule hath gone as far, it hath continued as long, as the very promulgation of the Gospel.”
Under Constantine the emperor about three hundred years and upward after Christ, Arius a priest in the church of Alexandria, a subtle-witted and a marvellous fair-spoken man,  but discontented that one should be placed before him in honour, whose superior he thought himself in desert, became through envy and stomach prone unto contradiction, and bold to broach at the length that heresy, wherein the deity of our Lord Jesus Christ contained but not opened in the former creed, the co-equality and co-eternity of the Son with the Father was denied. Being for this impiety deprived of his place by the bishop of the same church, the punishment which should have reformed him did but increase his obstinacy, and give him occasion of labouring with greater earnestness elsewhere to entangle unwary minds with the snares of his damnable opinion. Arius in short time had won to himself a number both of followers and of great defenders, whereupon much disquietness on all sides ensued. The emperor to reduce the Church of Christ unto the unity of sound belief, when other means whereof trial was first made took no effect, gathered that famous assembly of three hundred and eighteen bishops in the council of Nice, where besides order taken for many things which seemed to need redress, there was with common consent for the settling of all men’s minds, that other confession of faith set down which we call the Nicene Creed, whereunto the Arians themselves which were present subscribed also; not that they meant sincerely and in deed to forsake their error, but only to escape deprivation and exile, which they saw they could not avoid openly persisting in their former opinions when the greater part had concluded against them, and that with the emperor’s royal assent. Reserving therefore themselves unto future opportunities, and knowing that it would not boot them to stir again in a matter so composed, unless they could draw the emperor first and by his means the chiefest bishops unto their part, till Constantine’s death and somewhat after they always professed love and zeal to the Nicene faith; yet ceased not in the meanwhile to strengthen that part which in heart they favoured, and to infest by all means under colour of other quarrels their greatest adversaries in this cause: amongst them Athanasius especially, whom by the space of forty-six years, from the time of his consecration to succeed Alexander archbishop in the church of Alexandria till the last hour of his life in this world, they never suffered to enjoy the comfort of a peaceable  day. The heart of Constantine stolen from him. Constantius Constantine’s successor his scourge and torment by all the ways that malice armed with sovereign authority could devise and use. Under Julian no rest given him. And in the days of Valentinian as little. Crimes there were laid to his charge many, the least whereof being just had bereaved him of estimation and credit with men while the world standeth. His judges evermore the selfsame men by whom his accusers were suborned. Yet the issue always on their part, shame; on his, triumph. Those bishops and prelates, who should have accounted his cause theirs, and could not many of them but with bleeding hearts and with watered cheeks behold a person of so great place and worth constrained to endure so foul indignities, were sure by bewraying their affection towards him to bring upon themselves those molestations, whereby if they would not be drawn to seem his adversaries, yet others should be taught how unsafe it was to continue his friends.
Whereupon it came to pass in the end, that (very few excepted) all became subject to the sway of time; other odds there was none amongst them, saving only that some fell sooner away, some later, from the soundness of belief; some were leaders in the host of impiety, and the rest as common soldiers, either yielding through fear, or brought under with penury, or by flattery ensnared, or else beguiled through simplicity, which is the fairest excuse that well may be made for them. Yea (that which all men did wonder at) Osius the ancientest bishop that Christendom then had, the most forward in defence of the Catholic cause and of the contrary part most feared, that very Osius with whose hand the Nicene Creed itself was set down and framed for the whole Christian world to subscribe unto, so far yielded in the end as even with the same hand to ratify the Arians’ confession, a thing which they neither hoped to see, nor the other part ever feared, till with amazement they saw it done. Both were persuaded that although there had been for Osius no way but either presently subscribe or die, his answer and choice would have been the same that Eleazar’s was, “It doth not become our age to dissemble, whereby many young persons might think,  that 1Osius an hundred years old and upward were now gone to another religion, and so through mine hypocrisy (for a little time of transitory life) they might be deceived by me, and I procure malediction and reproach to my old age. For though I were now delivered from the torments of men, yet could I not escape the hand of the Almighty, neither alive nor dead.” But such was the stream of those times, that all men gave place unto it, which we cannot but impute partly to their own oversight. For at the first the emperor was theirs, the determination of the council of Nice was for them, they had the Arians’ hands to that council. So great advantages are never changed so far to the contrary, but by great error.
It plainly appeareth that the first thing which weakened them was their security. Such as they knew were in heart still affected towards Arianism, they suffered by continual nearness to possess the minds of the greatest about the emperor, which themselves might have done with very good acceptation, and neglected it. In Constantine’s lifetime to have settled Constantius the same way had been a duty of good service towards God, a mean of peace and great quietness to the Church of Christ, a labour easy, and how likely we may conjecture, when after that so much pain was taken to instruct and strengthen him in the contrary course, after that so much was done by himself to the furtherance of heresy, yet being touched in the end voluntarily with remorse, nothing more grieved him than the memory of former proceedings in the cause of religion, and that which he now foresaw in Julian, the next physician into whose hands the body that was thus distempered must fall.
Howbeit this we may somewhat excuse, inasmuch as every man’s particular care to his own charge was such as gave them no leisure to heed what others practised in princes’ courts. But of the two synods of Arimine and Seleucia what should we think? Constantius by the Arians’ suggestion had devised to assemble all the bishops of the whole world about this controversy, but in two several places, the bishops of the  West at Arimine in Italy, the Eastern at Seleucia the same time. Amongst them of the East there was no stop, they agreed without any great ado, gave their sentence against heresy, excommunicated some chief maintainers thereof, and sent the emperor word what was done. They had at Arimine about four hundred which held the truth, scarce of the adverse part fourscore, but these obstinate, and the other weary of contending with them: whereupon by both it was resolved to send to the emperor such as might inform him of the cause, and declare what hindered their peaceable agreement. There are chosen for the Catholic side such men as had in them nothing to be noted but boldness, neither gravity nor learning nor wisdom. The Arians for the credit of their faction take the eldest, the best experienced, the most wary, and the longest practised veterans they had amongst them. The emperor conjecturing of the rest on either part by the quality of them whom he saw, sent them speedily away, and with them a certain confession of faith ambiguously and subtilly drawn by the Arians, whereunto unless they all subscribed, they should in no case be suffered to depart from the place where they were. At the length it was perceived, that there had not been in the Catholics either at Arimine or at Seleucia so much foresight, as to provide that true intelligence might pass between them what was done. Upon the advantage of which error, their adversaries, abusing each with persuasion that other had yielded, surprised both. The emperor the more desirous and glad of such events, for that, besides all other things wherein they hindered themselves, the gall and bitterness of certain men’s writings, who spared him little for honour’s sake, made him for their sakes the less inclinable to that truth, which he himself should have honoured and loved.
Only in Athanasius there was nothing observed throughout the course of that long tragedy, other than such as very well became a wise man to do and a righteous to suffer. So that  this was the plain condition of those times: the whole world against Athanasius, and Athanasius against it; half a hundred of years spent in doubtful trial which of the two in the end would prevail, the side which had all, or else the part which had no friend but God and death, the one a defender of his innocency, the other a finisher of all his troubles.
Now although these contentions were cause of much evil, yet some good the Church hath reaped by them, in that they occasioned the learned and sound in faith to explain such things as heresy went about to deprave. And in this respect the Creed of Athanasius first exhibited unto Julius bishop of Rome, and afterwards (as we may probably gather) sent to the emperor Jovian, for his more full information concerning that truth which Arianism so mightily did impugn, was both in the East and the West churches accepted as a treasure of inestimable price, by as many as had not given up even the very ghost of belief. Then was the Creed of Athanasius written, howbeit not then so expedient to be publicly used as now in the Church of God; because while the heat of division lasteth truth itself enduring opposition doth not so quietly and currently pass throughout all men’s hands, neither can be of that account which afterwards it hath, when the world once perceiveth the virtue thereof not only in itself, but also by the conquest which God hath given it over heresy.
That which heresy did by sinister interpretations go about to pervert in the first and most ancient Apostolic Creed, the same being by singular dexterity and plainness cleared from those heretical corruptions partly by this Creed of Athanasius, written about the year three hundred and forty, and partly by that other set down in the synod of Constantinople forty years after, comprehending together with the Nicene Creed an addition of other articles which the Nicene  Creed omitted, because the controversy then in hand needed no mention to be made of them; these catholic declarations of our belief delivered by them which were so much nearer than we are unto the first publication thereof, and continuing needful for all men at all times to know, these confessions as testimonies of our continuance in the same faith to this present day, we rather use than any other gloss or paraphrase devised by ourselves, which though it were to the same effect, notwithstanding could not be of the like authority and credit. For that of Hilary unto St. Augustine hath been ever and is likely to be always true: “Your most religious wisdom knoweth how great their number is in the Church of God, whom the very authority of men’s names doth keep in that opinion which they hold already, or draw unto that which they have not before held.”
Touching the Hymn of Glory, our usual conclusion to Psalms: the glory of all things is that wherein their highest perfection doth consist; and the glory of God that divine excellency whereby he is eminent above all things, his omnipotent, infinite, and eternal Being, which angels and glorified saints do intuitively behold, we on earth apprehend principally by faith, in part also by that kind of knowledge which groweth from experience of those effects, the greatness whereof exceedeth the powers and abilities of all creatures both in heaven and earth. God is glorified, when such his excellency above all things is with due admiration acknowledged. Which dutiful acknowledgment of God’s excellency by occasion of special effects, being the very proper subject and almost the only matter purposely treated of in all psalms, if that joyful Hymn of Glory have any use in the Church of God whose name we therewith extol and magnify, can we place it more fitly than where now it serveth as a close or conclusion to psalms?
Neither is the form thereof newly or unnecessarily  invented.  “We must (saith St. Basil) as we have received even so baptize, and as we baptize even so believe, and as we believe even so give glory.” Baptizing we use the name of the Father, of the Son, and of the Holy Ghost; confessing the Christian faith we declare our belief in the Father, and in the Son, and in the Holy Ghost; ascribing glory unto God we give it to the Father, and to the Son, and to the Holy Ghost. It is ἀπόδειξις του̑ ὀρθου̑ ϕρονήματος, “the token of a true and sound understanding” for matter of doctrine about the Trinity, when in ministering baptism, and making confession, and giving glory, there is a conjunction of all three, and no one of the three severed from the other two.
Against the Arians affirming the Father to be greater than the Son in honour, excellency, dignity, majesty, this form and manner of glorifying God was not at that time first begun, but received long before, and alleged at that time as an argument for the truth. “If (saith Phœbadius) there be that inequality which they affirm, then do we every day blaspheme God, when in thanksgivings and offerings of sacrifice we acknowledge those things common to the Father and the Son.” The Arians therefore, for that they perceived how this did prejudice their cause, altered the Hymn of Glory, whereupon ensued in the church of Antioch about the year 349 that jar which Theodoret and Sozomen mention. “In their quires while they praised  God together as the manner was, at the end of the psalms which they sung, it appeared what opinion every man held, forasmuch as they glorified some the Father, and the Son, and the Holy Ghost; some the Father by the Son in the Spirit; the one sort thereby declaring themselves to embrace the Son’s equality with the Father as the council of Nice had defined, the other sort against the council of Nice his inequality.” Leontius their bishop although an enemy to the better part, yet wary and subtile, as in a manner all the heads of the Arians’ faction were, could at no time be plainly heard to use either form, perhaps lest his open contradicting of them whom he favoured not might make them the more eager, and by that mean the less apt to be privately won; or peradventure for that though he joined in opinion with that sort of Arians who denied the Son to be equal with the Father, yet from them he dissented which thought the Father and the Son not only unequal but unlike, as Aëtius did upon a frivolous and false surmise, that because the Apostle hath said, “One God of whom, one Lord by whom, one Spirit in whom,” his different manner of speech doth argue a different nature and being in them of whom he speaketh: out of which blind collection it seemeth that this their new devised form did first spring.
But in truth even that very form which the Arians did then use (saving that they chose it to serve as their special mark of recognizance, and gave it secretly within themselves a sinister construction) hath not otherwise as much as the show of any thing which soundeth towards impiety. For albeit if we respect God’s glory within itself, it be the equal right and possession of all three, and that without any odds, any difference; yet touching his manifestation thereof unto us by continual effects, and our perpetual acknowledgment  thereof unto him likewise by virtuous offices, doth not every tongue both ways confess, that the brightness of his glory hath spread itself throughout the world by the ministry of his only-begotten Son, and is in the manifold graces of the Spirit every way marvellous; again, that whatsoever we do to his glory, it is done in the power of the Holy Ghost, and made acceptable by the merit and mediation of Jesus Christ? So that glory to the Father and the Son, or glory to the Father by the Son, saving only where evil minds do abuse and pervert most holy things, are not else the voices of error and schism, but of sound and sincere religion.
It hath been the custom of the Church of Christ to end sometimes prayers, and sermons always, with words of glory; wherein, as long as the blessed Trinity had due honour, and till Arianism had made it a matter of great sharpness and subtilty of wit to be a sound believing Christian, men were not curious what syllables or particles of speech they used. Upon which confidence and trust notwithstanding when St. Basil began to practise the like indifferency, and to conclude public prayers, glorifying sometime the Father with the Son and the Holy Ghost, sometime the Father by the Son in the Spirit, whereas long custom had inured them unto the former kind alone, by means whereof the later was new and strange in their ears; this needless experiment brought afterwards upon him a necessary labour of excusing himself to his friends, and maintaining his own act against them, who because the light of his candle too much drowned theirs, were glad to lay hold on so colourable matter, and exceeding forward to traduce him as an author of suspicious innovation.
How hath the world forsaken that course which it sometime held! How are the judgments, hearts, and affections of men altered! May we not wonder that a man of St. Basil’s authority and quality, an arch-prelate in the house of God, should have his name far and wide called in question, and be  driven to his painful apologies, to write in his own defence whole volumes, and yet hardly to obtain with all his endeavour a pardon, the crime laid against him being but only a change of some one or two syllables in their usual church liturgy? It was thought in him an unpardonable offence to alter any thing; in us as intolerable that we suffer any thing to remain unaltered. The very Creed of Athanasius and that sacred Hymn of Glory, than which nothing doth sound more heavenly in the ears of faithful men, are now reckoned as superfluities, which we must in any case pare away, lest we cloy God with too much service. Is there in that confession of faith any thing which doth not at all times edify and instruct the attentive hearer? Or is our faith in the blessed Trinity a matter needless to be so oftentimes mentioned and opened in the principal part of that duty which we owe to God, our public prayer? Hath the Church of Christ from the first beginning by a secret universal instinct of God’s good Spirit always tied itself to end neither sermon nor almost any speech of moment which hath concerned matters of God without some special words of honour and glory to that Trinity which we all adore; and is the like conclusion of psalms become now at the length an eyesore or a galling to their ears that hear it?
“Those flames of Arianism” they say “are quenched, which were the cause why the Church devised in such sort to confess and praise the glorious deity of the Son of God. Seeing therefore the sore is whole, why retain we as yet the plaister? When the cause why any thing was ordained doth once cease, the thing itself should cease with it, that the Church being eased of unprofitable labours, needful offices may the better be attended. For the doing of things unnecessary, is many times the cause why the most necessary are not done.” But in this case so to reason will not serve their turns.
For first, the ground whereupon they build is not certainly their own but with special limitations. Few things are so restrained to any one end or purpose that the same being extinct they should forthwith utterly become frustrate. Wisdom may have framed one and the same thing to serve commodiously for divers ends, and of those ends any one be sufficient cause for continuance though the rest have ceased; even as  the tongue, which nature hath given us for an instrument of speech, is not idle in dumb persons, because it also serveth for taste. Again, if time have worn out, or any other mean altogether taken away what was first intended, uses not thought upon before may afterwards spring up, and be reasonable causes of retaining that which other considerations did formerly procure to be instituted. And it cometh sometime to pass that a thing unnecessary in itself as touching the whole direct purpose whereto it was meant or can be applied, doth notwithstanding appear convenient to be still held even without use, lest by reason of that coherence which it hath with somewhat most necessary, the removal of the one should endamage the other; and therefore men which have clean lost the possibility of sight keep still their eyes nevertheless in the place where nature set them.
As for these two branches whereof our question groweth, Arianism was indeed some occasion of the one, but a cause of neither, much less the only entire cause of both. For albeit conflict with Arians brought forth the occasion of writing that Creed which long after was made a part of the church liturgy, as hymns and sentences of glory were a part thereof before; yet cause sufficient there is why both should remain in use, the one as a most divine explication of the chiefest articles of our Christian belief, the other as an heavenly acclamation of joyful applause to his praises in whom we believe; neither the one nor the other unworthy to be heard sounding as they are in the Church of Christ, whether Arianism live or die.
Against which poison likewise if we think that the Church at this day needeth not those ancient preservatives which ages before us were so glad to use, we deceive ourselves greatly. The weeds of heresy being grown unto such ripeness as that was, do even in the very cutting down scatter oftentimes those seeds which for a while lie unseen and buried in the earth, but afterward freshly spring up again no less pernicious than at the first. Which thing they very well know and I doubt not will easily confess, who live to their great both toil and grief, where the blasphemies of Arians, Samosatenians, Tritheites, Eutychians, and Macedonians are renewed;  renewed by them who to hatch their heresy have chosen those churches as fittest nests, where Athanasius’ Creed is not heard; by them I say renewed, who following the course of extreme reformation, were wont in the pride of their own proceedings to glory, that whereas Luther did but blow away  the roof, and Zwinglius batter but the walls of popish superstition, the last and hardest work of all remained, which was to raze up the very ground and foundation of popery, that doctrine concerning the deity of Christ which Satanasius (for so it pleased those impious forsaken miscreants to speak) hath in this memorable creed explained. So manifestly true is that which one of the ancient hath concerning Arianism, “Mortuis auctoribus hujus veneni, scelerata tamen eorum doctrina non moritur:” “The authors of this venom being dead and gone, their wicked doctrine notwithstanding continueth.”

\section*{Our want of particular thanksgiving.}
XLIII. Amongst the heaps of these excesses and superfluities, there is espied the want of a principal part of duty, “There are no thanksgivings for the benefits for which there are petitions in our book of prayer.” This they have thought a point material to be objected. Neither may we take it in evil part to be admonished what special duties of thankfulness we owe to that merciful God, for whose unspeakable graces the only requital which we are able to make is a true, hearty, and sincere acknowledgment how precious we esteem such benefits received, and how infinite in goodness the Author  from whom they come. But that to every petition we make for things needful there should be some answerable sentence of thanks provided particularly to follow such requests obtained, either it is not a matter so requisite as they pretend; or if it be, wherefore have they not then in such order framed their own Book of Common Prayer? Why hath our Lord and Saviour taught us a form of prayer containing so many petitions of those things which we want, and not delivered in like sort as many several forms of thanksgiving to serve when any thing we pray for is granted? What answer soever they can reasonably make unto these demands, the same shall discover unto them how causeless a censure it is that there are not in our book thanksgivings for all the benefits for which there are petitions.
For concerning the blessings of God, whether they tend unto this life or the life to come, there is great cause why we should delight more in giving thanks, than in making requests for them; inasmuch as the one hath pensiveness and fear, the other always joy annexed; the one belongeth unto them that seek, the other unto them that have found happiness; they that pray do but yet sow, they that give thanks declare they have reaped. Howbeit because there are so many graces whereof we stand in continual need, graces for which we may not cease daily and hourly to sue, graces which are in bestowing always, but never come to be fully had in this present life; and therefore when all things here have an end, endless thanks must have their beginning in a state which bringeth the full and final satisfaction of all such perpetual desires: again, because our common necessities, and the lack which we all have as well of ghostly as of earthly favours is in each kind so easily known, but the gifts of God according to those degrees and times which he in his secret wisdom seeth meet, are so diversely bestowed, that it seldom appeareth what all receive, what all stand in need of, it seldom lieth hid: we are not to marvel though the Church do oftener concur in suits than in thanks unto God for particular benefits.
Nevertheless lest God should be any way unglorified, the greatest part of our daily service they know consisteth, according to the blessed Apostle’s own precise rule, in much variety of Psalms and Hymns, for no other purpose, but only that out of so plentiful a treasure there might be for every man’s heart to choose out his own sacrifice, and to offer unto God by particular secret instinct what fitteth best the often occasions which any several either party or congregation may seem to have. They that would clean take from us therefore the daily use of the very best means we have to magnify and praise the name of Almighty God for his rich blessings, they that complain of our reading and singing so many psalms for so good an end, they I say that find fault with our store should of all men be least willing to reprove our scarcity of thanksgivings.
But because peradventure they see it is not either generally fit or possible that churches should frame thanksgivings answerable to each petition, they shorten somewhat the reins of their censure; “there are no forms of thanksgiving,” they say, “for release of those common calamities from which we have petitions to be delivered.” “There are prayers set forth to be said in the common calamities and universal scourges of the realm, as plague, famine, \&c. and indeed so it ought to be by the word of God. But as such prayers are needful, whereby we beg release from our distresses, so there ought to be as necessary prayers of thanksgiving, when we have received those things at the Lord’s hand which we asked in our prayers.” As oft therefore as any public or universal scourge is removed, as oft as we are delivered from those either imminent or present calamities, against the storm and tempest whereof we all instantly craved favour from above, let it be a question what we should render unto God for his blessings universally, sensibly and extraordinarily bestowed. A prayer of three or four lines inserted into some part of our church liturgy? No, we are not persuaded that when God doth in trouble enjoin us the duty of invocation, and promise us the benefit of deliverance, and profess that the thing he expecteth  after at our hands is to glorify him as our mighty and only Saviour, the Church can discharge in manner convenient a work of so great importance by fore-ordaining some short collect wherein briefly to mention thanks. Our custom therefore whensoever so great occasions are incident, is by public authority to appoint throughout all churches set and solemn forms as well of supplication as of thanksgiving, the preparations and intended complements whereof may stir up the minds of men in much more effectual sort, than if only there should be added to the Book of Prayer that which they require.
But we err in thinking that they require any such matter. For albeit their words to our understanding be very plain, that in our book “there are prayers set forth” to be said when “common calamities” are felt, as “plague, famine,” and such like; again that “indeed so it ought to be by the word of God;” that likewise “there ought to be as necessary prayers of thanksgiving when we have received those things;” finally that the want of such forms of thanksgiving for the release from those common calamities from which we have petitions to be delivered, is the “default of the Book of Common Prayer:” yet all this they mean but only by way of “supposition, if express prayers” against so many earthly miseries were convenient, that then indeed as many express and particular thanksgivings should be likewise necessary. Seeing therefore we know that they hold the one superfluous, they would not have it so understood as though their minds were that any such addition to the book is needful, whatsoever they say for argument’s sake concerning this pretended defect. The truth is, they wave in and out, no way sufficiently grounded, no way resolved what to think, speak, or write, more than only that because they have taken it upon them, they must (no remedy now) be opposite.

\section*{In some things the matter of our prayer, as they affirm, is unsound.}
XLIV. The last supposed fault concerneth some few things, the very matter whereof is thought to be much amiss. In a song of praise to our Lord Jesus Christ we have these words, “When thou hadst overcome the sharpness of death, thou didst open the kingdom of heaven to all believers.” Which maketh some show of giving countenance to their  error, who think that the faithful which departed this life before the coming of Christ, were never till then made partakers of joy, but remained all in that place which they term the “Lake of the Fathers.”
In our liturgy request is made that we may be preserved “from sudden death.” This seemeth frivolous, because the godly should be always prepared to die.
Request is made that God would give those things which we for our unworthiness dare not ask. “This,” they say, “carrieth with it the note of popish servile fear, and savoureth not of that confidence and reverent familiarity that the children of God have through Christ with their heavenly Father.”
Request is made that we may evermore be defended from all adversity. For this “there is no promise in Scripture,” and therefore “it is no prayer of faith, or of the which we can assure ourselves that we shall obtain it.”
Finally, request is made that God “would have mercy upon all men.” This is impossible, because some are the vessels of wrath to whom God will never extend his mercy.

\section*{“When thou hadst overcome the sharpness of death, thou didst open the Kingdom of Heaven unto all believers.”}
XLV. As Christ hath purchased that heavenly kingdom the last perfection whereof is glory in the life to come, grace in this life a preparation thereunto; so the same he hath “opened” to the world in such sort, that whereas none can possibly without him attain salvation, by him “all that believe” are saved. Now whatsoever he did or suffered, the end thereof was to open the doors of the kingdom of heaven which our iniquities had “shut up.” But because by ascending  after that the sharpness of death was overcome, he took the very local possession of glory, and that to the use of all that are his, even as himself before had witnessed, “I go to prepare a place for you;” and again, “Whom thou hast given me, O Father, I will that where I am they be also with me, that my glory which thou hast given me they may behold:” it appeareth that when Christ did ascend he then most liberally opened the kingdom of heaven, to the end that with him and by him all believers might reign.
In what estate the Fathers rested which were dead before, it is not hereby either one way or other determined. All we can rightly gather is, that as touching their souls what degree of joy or happiness soever it pleased God to bestow upon them, his ascension which succeeded procured theirs, and theirs concerning the body must needs be not only of but after his. As therefore Helvidius against whom St. Jerome writeth, abused greatly those words of Matthew concerning Joseph and the mother of our Saviour Christ, “He knew her not till she had brought forth her first-born,” thereby gathering against the honour of the blessed Virgin, that a thing denied with special circumstance doth import an opposite affirmation when once that circumstance is expired: after the selfsame manner it should be a weak collection, if whereas we say that when Christ had “overcome the sharpness of death, he then opened the kingdom of heaven to all believers;” a thing in such sort affirmed with circumstance were taken as insinuating an opposite denial before that circumstance be accomplished, and consequently that because when the sharpness of death was overcome he then opened heaven as well to believing Gentiles as Jews, heaven till then was no receptacle to the souls of either. Wherefore be the spirits of the just and righteous before Christ truly or falsely thought excluded out of heavenly joy; by that which we in the words alleged before do attribute to Christ’s ascension, there is to no such opinion nor to the favourers thereof any  countenance at all given. We cannot better interpret the meaning of these words than Pope Leo himself expoundeth them, whose speech concerning our Lord’s ascension may serve instead of a marginal gloss: “Christ’s exaltation is our promotion, and whither the glory of the head is already gone before, thither the hope of the body also is to follow. For as this day we have not only the possession of paradise assured unto us, but in Christ we have entered the highest of the heavens.” His “opening the kingdom of heaven” and his entrance thereinto was not only to his own use but for the benefit of “all believers.”

\section*{Touching prayer for deliverance from sudden death.}
XLVI. Our good or evil estate after death dependeth most upon the quality of our lives. Yet somewhat there is why a virtuous mind should rather wish to depart this world with a kind of treatable dissolution, than to be suddenly cut off in a moment; rather to be taken than snatched away from the face of the earth.
Death is that which all men suffer, but not all men with one mind, neither all men in one manner. For being of necessity a thing common, it is through the manifold persuasions, dispositions, and occasions of men, with equal desert both of praise and dispraise, shunned by some, by others desired. So that absolutely we cannot discommend, we cannot absolutely approve, either willingness to live or forwardness to die.
And concerning the ways of death, albeit the choice thereof be only in his hands who alone hath power over all flesh, and unto whose appointment we ought with patience meekly to submit ourselves (for to be agents voluntarily in our own destruction is against both God and nature); yet there is no doubt but in so great variety, our desires will and may lawfully  prefer one kind before another. Is there any man of worth and virtue, although not instructed in the school of Christ, or ever taught what the soundness of religion meaneth, that had not rather end the days of this transitory life as Cyrus in Xenophon, or in Plato Socrates are described, than to sink down with them of whom Elihu hath said, Momento moriuntur, “there is scarce an instant between their flourishing and their not being?” But let us which know what it is to die as Absalon or Ananias and Sapphira died, let us beg of God that when the hour of our rest is come, the patterns of our dissolution may be Jacob, Moses, Josua, David; who leisurably ending their lives in peace, prayed for the mercies of God to come upon their posterity; replenished the hearts of the nearest unto them with words of memorable consolation; strengthened men in the fear of God; gave them wholesome instructions of life, and confirmed them in true religion; in sum, taught the world no less virtuously how to die than they had done before how to live.
To such as judge things according to the sense of natural men and ascend no higher, suddenness because it shorteneth their grief should in reason be most acceptable. That which causeth bitterness in death is the languishing attendance and expectation thereof ere it come. And therefore tyrants use what art they can to increase the slowness of death. Quick riddance out of life is often both requested and bestowed as a benefit. Commonly therefore it is for virtuous considerations that wisdom so far prevaileth with men as to make them desirous of slow and deliberate death against the stream of their sensual inclination, content to endure the longer grief and bodily pain, that the soul may have time to call itself to a just account of all things past, by means whereof repentance is perfected, there is wherein to exercise patience, the joys of the kingdom of heaven have leisure to present themselves, the pleasures of sin and this world’s vanities are censured with uncorrupt judgment, charity is free to make advised choice of the soil wherein her last seed may most fruitfully be bestowed, the mind is at liberty to have due  regard of that disposition of worldly things which it can never afterwards alter, and because the nearer we draw unto God, the more we are oftentimes enlightened with the shining beams of his glorious presence as being then even almost in sight, a leisurable departure may in that case bring forth for the good of such as are present that which shall cause them for ever after from the bottom of their hearts to pray, “O let us die the death of the righteous, and let our last end be like theirs.” All which benefits and opportunities are by sudden death prevented.
And besides forasmuch as death howsoever is a general effect of the wrath of God against sin, and the suddenness thereof a thing which happeneth but to few; the world in this respect feareth it the more as being subject to doubtful constructions, which as no man willingly would incur, so they whose happy estate after life is of all men’s the most certain should especially wish that no such accident in their death may give uncharitable minds occasion of rash, sinister, and suspicious verdicts, whereunto they are over prone; so that whether evil men or good be respected, whether we regard ourselves or others, to be preserved from sudden death is a blessing of God. And our prayer against it importeth a twofold desire: first, that death when it cometh may give us some convenient respite; or secondly, if that be denied us of God, yet we may  have wisdom to provide always beforehand that those evils overtake us not which death unexpected doth use to bring upon careless men, and that although it be sudden in itself, nevertheless in regard of our prepared minds it may not be sudden.

\section*{Prayer that those things which we for our unworthiness dare not ask, God for the worthiness of his Son would vouchsafe to grant.}
XLVII. But is it credible that the very acknowledgment of our own unworthiness to obtain, and in that respect our professed fearfulness to ask any thing otherwise than only for his sake to whom God can deny nothing, that this should be noted for a popish error, that this should be termed baseness, abjection of mind, or “servility,” is it credible? That which we for our unworthiness are afraid to crave, our prayer is that God for the worthiness of his Son would notwithstanding vouchsafe to grant. May it please them to shew us which of these words it is that “carrieth the note of popish and servile fear?”
In reference to other creatures of this inferior world man’s worth and excellency is admired. Compared with God, the truest inscription wherewith we can circle so base a coin is that of David, Universa vanitas est omnis homo: “Whosoever  hath the name of a mortal man, there is in him whatsoever the name of vanity doth comprehend.” And therefore what we say of our own “unworthiness” there is no doubt but truth will ratify. Alleged in prayer it both becometh and behoveth saints. For as humility is in suitors a decent virtue, so the testification thereof by such effectual acknowledgments, not only argueth a sound apprehension of his supereminent glory and majesty before whom we stand, but putteth also into his hands a kind of pledge or bond for security against our unthankfulness, the very natural root whereof is always either ignorance, dissimulation, or pride: ignorance, when we know not the author from whom our good cometh; dissimulation, when our hands are more open than our eyes upon that we receive; pride, when we think ourselves worthy of that which mere grace and undeserved mercy bestoweth. In prayer therefore to abate so vain imaginations with the true conceit of unworthiness, is rather to prevent than commit a fault.
It being no error thus to think, no fault thus to speak of ourselves when we pray, is it a fault that the consideration of our unworthiness maketh us fearful to open our mouths by way of suit? While Job had prosperity and lived in honour, men feared him for his authority’s sake, and in token of their fear when they saw him they “hid themselves.” Between Elihu and the rest of Job’s familiars the greatest disparity was but in years. And he, though riper than they in judgment, doing them reverence in regard of age, stood long “doubtful,” and very loth to adventure upon speech in his elders’ hearing. If so small inequality between man and man make their modesty a commendable virtue, who respecting superiors as superiors, can neither speak nor stand before them without fear: that the publican approacheth not more boldly to God; that when Christ in mercy draweth near to Peter, he in humility and fear craveth distance; that being to stand, to speak, to sue in the presence of so great majesty, we are afraid, let no man blame us.
In which consideration notwithstanding, because to fly altogether from God, to despair that creatures unworthy shall be able to obtain any thing at his hands, and under that pretence to surcease from prayers as bootless or fruitless offices, were to him no less injurious than pernicious to our own souls; even that which we tremble to do we do, we ask those things which we dare not ask. The knowledge of our own unworthiness is not without belief in the merits of Christ. With that true fear which the one causeth there is coupled true boldness, and encouragement drawn from the other. The very silence which our unworthiness putteth us unto, doth itself make request for us, and that in the confidence of his grace. Looking inward we are stricken dumb, looking upward we speak and prevail. O happy mixture, wherein things contrary do so qualify and correct the one the danger of the other’s excess, that neither boldness can make us presume as long as we are kept under with the sense of our own wretchedness; nor, while we trust in the mercy of God through Christ Jesus, fear be able to tyrannize over us! As therefore our fear excludeth not that boldness which becometh saints; so if their familiarity with God do not savour of this fear, it draweth too near that irreverent confidence wherewith true humility can never stand.

\section*{Prayer to be evermore delivered from all adversity.}
XLVIII. Touching continual deliverance in the world from all adversity, their conceit is that we ought not to ask it of God by prayer, forasmuch as in Scripture there is no promise that we shall be evermore free from vexations, calamities, and troubles.
Minds religiously affected are wont in every thing of weight and moment which they do or see, to examine according unto rules of piety what dependency it hath on God, what reference to themselves, what coherence with any of those duties whereunto all things in the world should lead, and accordingly they frame the inward disposition of their minds sometime to admire God, sometime to bless him and give him thanks, sometime to exult in his love, sometime to implore his mercy. All which different elevations of spirit unto God are contained in the name of prayer. Every good and holy desire though it lack the form, hath notwithstanding in itself the substance and with him the force of a prayer, who regardeth the very moanings, groans, and sighs of the heart of man. Petitionary prayer belongeth only to such as are in themselves impotent, and stand in need of relief from others. We thereby declare unto God what our own desire is that he by his power should effect. It presupposeth therefore in us first the want of that which we pray for; secondly, a feeling of that want; thirdly, an earnest willingness of mind to be eased therein; fourthly, a declaration of this our desire in the sight of God, not as if he should be otherwise ignorant of our necessities, but because we this way shew that we honour him as our God, and are verily persuaded that no good thing can come to pass which he by his omnipotent power effecteth not.
Now because there is no man’s prayer acceptable whose person is odious, neither any man’s person gracious without faith, it is of necessity required that they which pray do believe. The prayers which our Lord and Saviour made were for his own worthiness accepted; ours God accepteth not but with this condition, if they be joined with belief in Christ.
The prayers of the just are accepted always, but not always those things granted for which they pray. For in prayer if faith and assurance to obtain were both one and the same thing, seeing that the effect of not obtaining is a plain testimony that they which prayed were not sure they should  obtain, it would follow that their prayer being without certainty of the event, was also made unto God without faith, and consequently that God abhorred it. Which to think of so many prayers of saints as we find have failed in particular requests, how absurd were it! His faithful people have this comfort, that whatsoever they rightly ask, the same no doubt but they shall receive, so far as may stand with the glory of God, and their own everlasting good, unto either of which two it is no virtuous man’s purpose to seek or desire to obtain any thing prejudicial, and therefore that clause which our Lord and Saviour in the prayer of his agony did express, we in petitions of like nature do always imply, Pater, si possibile est, “If it may stand with thy will and pleasure.” Or if not, but that there be secret impediments and causes in regard whereof the thing we pray for is denied us, yet the prayer itself which we make is a pleasing sacrifice to God, who both accepteth and rewardeth it some other way. So that sinners in very truth are denied when they seem to prevail in their supplications, because it is not for their sakes or to their good that their suits take place; the faithful contrariwise, because it is for their good oftentimes that their petitions do not take place, prevail even then when they most seem denied. “Our Lord God in anger hath granted some impatient men’s requests, as on the other side the Apostle’s suit he hath of favour and mercy not granted,” saith St. Augustine.
To think we may pray unto God for nothing but what he hath promised in Holy Scripture we shall obtain, is perhaps an error. For of prayer there are two uses. It serveth as a mean to procure those things which God hath promised to grant when we ask; and it serveth as a mean to express our lawful desires also towards that, which whether we shall have or no we know not till we see the event. Things in themselves unholy or unseemly we may not ask; we may whatsoever being not forbidden either nature or grace shall reasonably move us to wish as importing the good of men, albeit God himself have nowhere by promise  assured us of that particular which our prayer craveth. To pray for that which is in itself and of its own nature apparently a thing impossible, were not convenient. Wherefore though men do without offence wish daily that the affairs which with evil success are past might have fallen out much better, yet to pray that they may have been any other than they are, this being a manifest impossibility in itself, the rules of religion do not permit. Whereas contrariwise when things of their own nature contingent and mutable are by the secret determination of God appointed one way, though we the other way make our prayers, and consequently ask those things of God which are by this supposition impossible, we notwithstanding do not hereby in prayer transgress our lawful bounds.
That Christ, as the only begotten Son of God, having no superior, and therefore owing honour unto none, neither standing in any need, should either give thanks, or make petition unto God, were most absurd. As man what could beseem him better, whether we respect his affection to Godward, or his own necessity, or his charity and love towards men? Some things he knew should come to pass and notwithstanding prayed for them, because he also knew that the necessary means to effect them were his prayers. As in the Psalm it is said, “Ask of me and I will give thee the heathen for thine inheritance and the ends of the earth for thy possession.” Wherefore that which here God promiseth his Son, the same in the seventeenth of John he prayeth for: “Father, the hour is now come, glorify thy Son, that thy Son also may glorify thee according as thou hast given him power over all flesh.”
But had Christ the like promise concerning the effect of every particular for which he prayed? That which was not effected could not be promised. And we know in what sort he prayed for removal of that bitter cup, which cup he tasted, notwithstanding his prayer.
To shift off this example they answer first, “That  as other children of God, so Christ had a promise of deliverance as far as the glory of God in the accomplishment of his vocation would suffer.”
And if we ourselves have not also in that sort the promise of God to be evermore delivered from all adversity, what meaneth the sacred Scripture to speak in so large terms, “Be obedient, and the Lord thy God will make thee plenteous in every work of thy hand, in the fruit of thy body, and in the fruit of thy cattle, and in the fruit of the land for thy wealth.” Again, “Keep his laws, and thou shalt be blest above all people, the Lord shall take from thee all infirmities.” “The man whose delight is in the Law of God, whatsoever he doth it shall prosper.” “For the ungodly there are great plagues remaining; but whosoever putteth his trust in the Lord mercy embraceth him on every side.” Not only that mercy which keepeth from being overlaid or oppressed, but mercy which saveth from being touched with grievous miseries, mercy which turneth away the course of “the great water-floods,” and permitteth them not to “come near.”
Nevertheless, because the prayer of Christ did concern but one calamity, they are still bold to deny the lawfulness of our prayer for deliverance out of all, yea though we pray with the same exception that he did, “If such deliverance may stand with the pleasure of Almighty God and not otherwise.” For they have secondly found out a rule that prayer ought only to be made for deliverance “from this or that particular adversity, whereof we know not but upon the event what the pleasure of God is.” Which  quite overthroweth that other principle wherein they require unto every prayer which is of faith an assurance to obtain the thing we pray for. At the first to pray against all adversity was unlawful, because we cannot assure ourselves that this will be granted. Now we have license to pray against any particular adversity, and the reason given because we know not but upon the event what God will do. If we know not what God will do, it followeth that for any assurance we have he may do otherwise than we pray, and we may faithfully pray for that which we cannot assuredly presume that God will grant.
Seeing therefore neither of these two answers will serve the turn, they have a third, which is, that to pray in such sort is but idly mispent labour, because God already hath revealed his will touching this request, and we know that the suit we make is denied before we make it. Which neither is true, and if it were, was Christ ignorant what God had determined touching those things which himself should suffer? To say, “He knew not what weight of sufferances his heavenly Father had measured unto him.” is somewhat hard; harder that although “he knew them” notwithstanding for the present time they were “forgotten through the force of those unspeakable pangs which he then was in.” The one against the plain express words of the holy Evangelist, “he knew all things that should come upon him;” the other less credible, if any thing may be of less credit than what the Scripture itself gainsayeth. Doth any of them which wrote his sufferings make report that memory failed him? Is there in his words and speeches any sign of defect that way? Did not himself declare before whatsoever was to happen in the  course of that whole tragedy? Can we gather by any thing after taken from his own mouth either in the place of public judgment or upon the altar of the cross, that through the bruising of his body some part of the treasures of his soul were scattered and slipped from him? If that which was perfect both before and after did fail at this only middle instant, there must appear some manifest cause how it came to pass. True it is that the pangs of his heaviness and grief were unspeakable: and as true that because the minds of the afflicted do never think they have fully conceived the weight or measure of their own woe, they use their affection as a whetstone both to wit and memory, these as nurses to feed grief, so that the weaker his conceit had been touching that which he was to suffer, the more it must needs in that hour have helped to the mitigation of his anguish. But his anguish we see was then at the very highest whereunto it could possibly rise; which argueth his deep apprehension even to the last drop of the gall which that cup contained, and of every circumstance wherein there was any force to augment heaviness, but above all things the resolute determination of God and his own unchangeable purpose, which he at that time could not forget.
To what intent then was his prayer, which plainly testifieth so great willingness to avoid death? Will, whether it be in God or man, belongeth to the essence and nature of both. The Nature therefore of God being one, there are not in God divers wills although Godhead be in divers persons, because the power of willing is a natural not a personal propriety. Contrariwise, the Person of our Saviour Christ being but one there are in him two wills, because two natures, the nature of God and the nature of man, which both do imply this faculty and power. So that in Christ there is a divine and there is an human will, otherwise he were not both God and man. Hereupon the Church hath of old condemned Monothelites as heretics, for holding that Christ had but one will. The works and operations of our Saviour’s human will were all subject to the will of God, and framed according to his law, “I desired to do thy will O God, and thy law is within mine heart.”
Now as man’s will so the will of Christ hath two several kinds of operation, the one natural or necessary, whereby it desireth simply whatsoever is good in itself, and shunneth as generally all things which hurt; the other deliberate, when we therefore embrace things as good, because the eye of understanding judgeth them good to that end which we simply desire. Thus in itself we desire health, physic only for health’s sake. And in this sort special reason oftentimes causeth the will by choice to prefer one good thing before another, to leave one for another’s sake, to forego meaner for the attainment of higher desires, which our Saviour likewise did.
These different inclinations of the will considered, the reason is easy how in Christ there might grow desires seeming but being not indeed opposite, either the one of them unto the other, or either of them to the will of God. For let the manner of his speech be weighed, “My soul is now troubled, and what should I say? “Father, save me out of this hour. But yet for this very cause am I come into this hour.” His purpose herein was most effectually to propose to the view of the whole world two contrary objects, the like whereunto in force and efficacy were never presented in that manner to any but only to the soul of Christ. There was presented before his eyes in that fearful hour on the one side God’s heavy indignation and wrath towards mankind as yet unappeased, death as yet in full strength, hell as yet never mastered by any that came within the confines and bounds thereof, somewhat also peradventure more than is either possible or needful for the wit of man to find out, finally himself flesh and blood left alone to enter into conflict with all these; on the other side, a  world to be saved by one, a pacification of wrath through the dignity of that sacrifice which should be offered, a conquest over death through the power of that Deity which would not suffer the tabernacle thereof to see corruption, and an utter disappointment of all the forces of infernal powers, through the purity of that soul which they should have in their hands and not be able to touch. Let no man marvel that in this case the soul of Christ was much troubled. For what could such apprehensions breed but (as their nature is) inexplicable passions of mind, desires abhorring what they embrace, and embracing what they abhor? In which agony “how should the tongue go about to express” what the soul endured? When the griefs of Job were exceeding great, his words accordingly to open them were many; howbeit, still unto his seeming they were undiscovered: “Though my talk” (saith Job) “be this day in bitterness, yet my plague is greater than my groaning.” But here to what purpose should words serve, when nature hath more to declare than groans and strong cries, more than streams of bloody sweats, more than his doubled and tripled prayers can express, who thrice putting forth his hand to receive that cup, besides which there was no other cause of his coming into the world, he thrice pulleth it back again, and as often even with tears of blood craveth, “If it be possible, O Father: or if not, even what thine own good pleasure is,” for whose sake the passion that hath in it a bitter and a bloody conflict even with wrath and death and hell is most welcome.
Whereas therefore we find in God a will resolved that Christ shall suffer; and in the human will of Christ two actual desires, the one avoiding, and the other accepting death; is that desire which first declareth itself by prayer against that wherewith he concludeth prayer, or either of them against his mind to whom prayer in this case seeketh? We may judge of these diversities in the will, by the like in the understanding. For as the intellectual part doth not cross itself by conceiving man to be just and unjust when it meaneth not the same man, nor by imagining the same man learned and unlearned, if learned in one skill, and in another kind  of learning unskilful, because the parts of every true opposition do always both concern the same subject, and have reference to the same thing, sith otherwise they are but in show opposite and not in truth: so the will about one and the same thing may in contrary respects have contrary inclinations and that without contrariety. The minister of justice may for public example to others, virtuously will the execution of that party, whose pardon another for consanguinity’s sake as virtuously may desire. Consider death in itself, and nature teacheth Christ to shun it; consider death as a mean to procure the salvation of the world, and mercy worketh in Christ all willingness of mind towards it. Therefore in these two desires there can be no repugnant opposition. Again, compare them with the will of God, and if any opposition be, it must be only between his appointment of Christ’s death, and the former desire which wisheth deliverance from death. But neither is this desire opposite to the will of God. The will of God was that Christ should suffer the pains of death. Not so his will, as if the torment of innocency did in itself please and delight God, but such was his will in regard of the end whereunto it was necessary that Christ should suffer. The death of Christ in itself therefore God willeth not, which to the end we might thereby obtain life he both alloweth and appointeth. In like manner the Son of man endureth willingly to that purpose those grievous pains, which simply not to have shunned had been against nature, and by consequent against God.
I take it therefore to be an error that Christ either knew not what himself was to suffer, or else had forgotten the things he knew. The root of which error was an over-restrained consideration of prayer, as though it had no other lawful use but only to serve for a chosen mean, whereby the will resolveth to seek that which the understanding certainly knoweth it shall obtain: whereas prayers in truth both ours are and his were, as well sometime a presentation of mere desires, as a mean of procuring desired effects at the hands of God. We are therefore taught by his example, that the presence of dolorous and dreadful objects even in minds most perfect, may as clouds overcast all sensible joy; that no  assurance touching future victories can make present conflicts so sweet and easy but nature will shun and shrink from them, nature will desire ease and deliverance from oppressive burdens; that the contrary determination of God is oftentimes against the effect of this desire, yet not against the affection itself, because it is naturally in us; that in such case our prayers cannot serve us as means to obtain the thing we desire; that notwithstanding they are unto God most acceptable sacrifices, because they testify we desire nothing but at his hands, and our desires we submit with contentment to be overruled by his will, and in general they are not repugnant unto the natural will of God which wisheth to the works of his own hands in that they are his own handy work all happiness, although perhaps for some special cause in our own particular a contrary determination have seemed more convenient; finally, that thus to propose our desires which cannot take such effect as we specify, shall notwithstanding otherwise procure us His heavenly grace, even as this very prayer of Christ obtained Angels to be sent him as comforters in his agony. And according to this example we are not afraid to present unto God our prayers for those things which that he will perform unto us we have no sure nor certain knowledge.
St. Paul’s prayer for the church of Corinth was that they might not do any evil, although he knew that no man liveth which sinneth not, although he knew that in this life we always must pray, “Forgive us our sins.” It is our frailty that in many things we all do amiss, but a virtue that we would do amiss in nothing, and a testimony of that virtue when we pray that what occasion of sin soever do offer itself we may be strengthened from above to withstand it. They pray in vain to have sin pardoned which seek not also to prevent sin by prayer, even every particular sin by prayer against all sin; except men can name some transgression wherewith we ought to have truce. For in very deed although we cannot be free from all sin collectively in such sort that no part thereof shall be found inherent in us, yet distributively at the least all great and grievous actual offences as they offer themselves  one by one both may and ought to be by all means avoided. So that in this sense to be preserved from all sin is not impossible.
Finally, concerning deliverance itself from all adversity, we use not to say men are in adversity whensoever they feel any small hinderance of their welfare in this world, but when some notable affliction or cross, some great calamity or trouble befalleth them. Tribulation hath in it divers circumstances, the mind sundry faculties to apprehend them: it offereth sometime itself to the lower powers of the soul as a most unpleasant spectacle, to the higher sometimes as drawing after it a train of dangerous inconveniences, sometime as bringing with it remedies for the curing of sundry evils, as God’s instrument of revenge and fury sometime, sometime as a rod of his just yet moderate ire and displeasure, sometime as matter for them that spitefully hate us to exercise their poisoned malice, sometime as a furnace of trial for virtue to shew itself, and through conflict to obtain glory. Which different contemplations of adversity do work  for the most part their answerable effects. Adversity either apprehended by sense as a thing offensive and grievous to nature; or by reason conceived as a snare, an occasion of many men’s falling from God, a sequel of God’s indignation and wrath, a thing which Satan desireth and would be glad to behold; tribulation thus considered being present causeth sorrow, and being imminent breedeth fear. For moderation of which two affections growing from the very natural bitterness and gall of adversity, the Scripture much allegeth contrary fruits which affliction likewise hath, whensoever it falleth on them that are tractable, the grace of God’s Holy Spirit concurring therewith.
But when the Apostle St. Paul teacheth, “That every one which will live godly in Christ Jesus must suffer persecution,” and “by many tribulations we must enter into the kingdom of heaven,” because in a forest of many wolves sheep cannot choose but feed in continual danger of life; or when St. James exhorteth to “account it a matter of exceeding joy when we fall into divers temptations,” because “by the trial of faith patience is brought forth;” was it, suppose we, their meaning to frustrate our Lord’s admonition, “Pray that ye enter not into temptation?” When himself pronounceth them blessed that should for his name’s sake be subject to all kinds of ignominy and opprobrious malediction, was it his purpose that no man should ever pray with David, “Lord, remove from me shame and contempt?”
“In those tribulations” (saith St. Augustine) “which may hurt as well as profit, we must say with the Apostle, What we should ask as we ought we know not; yet because they are tough, because they are grievous, because the sense of our weakness flieth them, we pray according to the  general desire of the will of man that God would turn them away from us, owing in the meanwhile this devotion to the Lord our God, that if he remove them not, yet we do not therefore imagine ourselves in his sight despised, but rather with godly sufferance of evils expect greater good at his merciful hands. For thus is virtue in weakness perfected.”
To the flesh (as the Apostle himself granteth) all affliction is naturally grievous. Therefore nature which causeth to fear teacheth to pray against all adversity. Prosperity in regard of our corrupt inclination to abuse the blessings of Almighty God, doth prove for the most part a thing dangerous to the souls of men. Very ease itself is death to the wicked, “and the prosperity of fools slayeth them;” their table is a snare, and their felicity their utter overthrow. Few men there are which long prosper and sin not. Howbeit even as these ill effects although they be very usual and common are no bar to the hearty prayers whereby most virtuous minds wish peace and prosperity always where they love, because they consider that this in itself is a thing naturally desired: so because all adversity is in itself against nature, what should hinder to pray against it, although the providence of God turn it often unto the great good of many men? Such prayers of the Church to be delivered from all adversity are no more repugnant to any reasonable disposition of men’s minds towards death, much less to that blessed patience and meek contentment which saints by heavenly inspiration have to endure what cross or calamity soever it pleaseth God to lay upon them, than our Lord and Saviour’s own prayer before his passion was repugnant unto his most gracious resolution to die for the sins of the whole world.

\section*{Prayer that all men may find mercy.}
XLIX. In praying for deliverance from all adversity we seek that which nature doth wish to itself; but by entreating for mercy towards all, we declare that affection wherewith Christian charity thirsteth after the good of the whole world, we discharge that duty which the Apostle himself doth impose on the Church of Christ as a commendable office, a sacrifice acceptable in God’s sight, a service according to his heart whose desire is “to have all men saved,” a work most suitable with his purpose who gave himself to be the price of  redemption for all, and a forcible mean to procure the conversion of all such as are not yet acquainted with the mysteries of that truth which must save their souls. Against it there is but the bare show of this one impediment, that all men’s salvation and many men’s eternal condemnation or death are things the one repugnant to the other, that both cannot be brought to pass; that we know there are vessels of wrath to whom God will never extend mercy, and therefore that wittingly we ask an impossible thing to be had.
The truth is that as life and death, mercy and wrath are matters of mere understanding or knowledge, all men’s salvation and some men’s endless perdition are things so opposite that whosoever doth affirm the one must necessarily deny the other, God himself cannot effect both or determine that both shall be. There is in the knowledge both of God and man this certainty, that life and death have divided between them the whole body of mankind. What portion either of the two hath, God himself knoweth; for us he hath left no sufficient means to comprehend, and for that cause neither given any leave to search in particular who are infallibly the heirs of the kingdom of God, who castaways. Howbeit concerning the state of all men with whom we live (for only of them our prayers are meant) we may till the world’s end, for the present, always presume, that as far as in us there is power to discern what others are, and as far as any duty of ours dependeth upon the notice of their condition in respect of God, the safest axioms for charity to rest itself upon are these: “He which believeth already is;” and “he which believeth not as yet may be the child of God.” It becometh not us “during life altogether to condemn any man, seeing that” (for any thing we know) “there is hope of every man’s forgiveness, the possibility of whose repentance is  not yet cut off by death.” And therefore Charity which “hopeth all things,” prayeth also for all men.
Wherefore to let go personal knowledge touching vessels of wrath and mercy, what they are inwardly in the sight of God it skilleth not, for us there is cause sufficient in all men whereupon to ground our prayers unto God in their behalf. For whatsoever the mind of man apprehendeth as good, the will of charity and love is to have it enlarged in the very uttermost extent, that all may enjoy it to whom it can any way add perfection. Because therefore the farther a good thing doth reach the nobler and worthier we reckon it, our prayers for all men’s good no less than for our own the Apostle with very fit terms commendeth as being καλὸν, a work commendable for the largeness of the affection from whence it springeth, even as theirs, which have requested at God’s hands the salvation of many with the loss of their own souls, drowning as it were and overwhelming themselves in the abundance of their love towards others, is proposed as being in regard of the rareness of such affections ὑπέρκαλον, more than excellent. But this extraordinary height of desire after other men’s salvation is no common mark. The other is a duty which belongeth unto all and prevaileth with God daily. For as it is in itself good, so God accepteth and taketh it in very good part at the hands of faithful men. Our prayers for all men do include both them that shall find mercy, and them also that shall find none. For them that shall, no man will doubt but our prayers are both accepted and granted. Touching them for whom we crave that mercy which is not to be obtained, let us not think that our Saviour did misinstruct his disciples, willing them to pray for the peace even of such as should be uncapable of so great a blessing; or that the prayers of the Prophet Jeremy offended God because the answer of God was a resolute denial of favour to them for whom supplication was made. And if any man doubt how God should accept such prayers in case they be opposite to his will, or not grant them if they be according unto that which himself willeth, our answer is that such suits God accepteth in that they are conformable unto his general  inclination which is that all men might be saved, yet always he granteth them not, forasmuch as there is in God sometimes a more private occasioned will which determineth the contrary. So that the other being the rule of our actions and not this, our requests for things opposite to this will of God are not therefore the less gracious in his sight.
There is no doubt but we ought in all things to frame our wills to the will of God, and that otherwise in whatsoever we do we sin. For of ourselves being so apt to err, the only way which we have to straighten our paths is by following the rule of his will whose footsteps naturally are right. If the eye, the hand, or the foot do that which the will commandeth, though they serve as instruments to sin, yet is sin the commander’s fault and not theirs, because nature hath absolutely and without exception made them subjects to the will of man which is Lord over them. As the body is subject to the will of man, so man’s will to the will of God; for so it behoveth that the better should guide and command the worse. But because the subjection of the body to the will is by natural  necessity, the subjection of the will unto God voluntary; we therefore stand in need of direction after what sort our wills and desires may be rightly conformed to his. Which is not done by willing always the selfsame thing that God intendeth. For it may chance that his purpose is sometime the speedy death of them whose long continuance in life if we should not wish we were unnatural.
When the object or matter therefore of our desires is (as in this case) a thing both good of itself and not forbidden of God; when the end for which we desire it is virtuous and apparently most holy; when the root from which our affection towards it proceedeth is charity, piety that which we do in declaring our desire by prayer; yea over and besides all this, sith we know that to pray for all men living is but to shew the same affection which towards every of them our Lord Jesus Christ hath borne, who knowing only as God who are his did as man taste death for the good of all men: surely to that will of God which ought to be and is the known rule of all our actions, we do not herein oppose ourselves, although his secret determination haply be against us, which if we did understand as we do not, yet to rest contented with that which God will have done is as much as he requireth at the hands of men. And concerning ourselves, what we earnestly crave in this case, the same, as all things else that are of like condition, we meekly submit unto his most gracious will and pleasure.
Finally, as we have cause sufficient why to think the practice of our church allowable in this behalf, so neither is ours the first which hath been of that mind. For to end with the words of Prosper, “This law of supplication for all men,” (saith he,) “the devout zeal of all priests and of all faithful men doth hold with such full agreement, that there is not any part of all the world where Christian people do not use to pray in the same manner. The Church every where maketh prayers unto God not only for saints and such as  already in Christ are regenerate, but for all infidels and enemies of the Cross of Jesus Christ, for all idolaters, for all that persecute Christ in his followers, for Jews to whose blindness the light of the Gospel doth not yet shine, for heretics and schismatics, who from the unity of faith and charity are estranged. And for such what doth the Church ask of God but this, that leaving their errors they may be converted unto him, that faith and charity may be given them, and that out of the darkness of ignorance they may come to the knowledge of his truth? which because they cannot themselves do in their own behalf as long as the sway of evil custom overbeareth them, and the chains of Satan detain them bound, neither are they able to break through those errors wherein they are so determinately settled, that they pay unto falsity the whole sum of whatsoever love is owing unto God’s truth; our Lord merciful and just requireth to have all men prayed for; that when we behold innumerable multitudes drawn up from the depth of so bottomless evils, we may not doubt but” (in part) “God hath done the thing we requested, nor despair but that being thankful for them towards whom already he hath shewed mercy, the rest which are not as yet enlightened, shall before they pass out of life be made partakers of the like grace. Or if the grace of him which saveth (for so we see it falleth out) overpass some, so that the prayer of the Church for them be not received, this we may leave to the hidden judgments of God’s righteousness, and acknowledge that in this secret there is a gulf, which while we live we shall never sound.”

\section*{Of the name, the author, and the force of Sacraments, which force consisteth in this, that God hath ordained them as means to make us partakers of him in Christ, and of life through Christ.}
L. Instruction and Prayer whereof we have hitherto spoken, are duties which serve as elements, parts, or principles, to the rest that follow, in which number the Sacraments of the Church are chief. The Church is to us that very mother of our new birth, in whose bowels we are all bred, at whose breasts we receive nourishment.
Of the name, the author, and the force of sacraments; which force consisteth in this, that God hath ordained them as means to make us partakers of him in Christ, and of life through Christ.
 As many therefore as are apparently to our judgment born of God, they have the seed of their regeneration by the ministry of the Church which useth to that end and purpose not only the Word, but the Sacraments, both having generative force and virtue.

[2.]As oft as we mention a Sacrament properly understood, (for in the writings of the ancient Fathers all articles which are peculiar to Christian faith, all duties of religion containing that which sense or natural reason cannot of itself discern, are most commonly named Sacraments,) our restraint of the word to some few principal divine ceremonies importeth in every such ceremony two things, the substance of the ceremony itself which is visible, and besides that somewhat else more secret in reference whereunto we conceive that ceremony to be a Sacrament. For we all admire and honour the holy Sacraments, not respecting so much the service which we do unto God in receiving them, as the dignity of that sacred and secret gift which we thereby receive from God. Seeing that Sacraments therefore consist altogether in relation to some such gift or grace supernatural as only God can bestow, how should any but the Church administer those ceremonies as Sacraments which are not thought to be Sacraments by any but by the Church?

[3.]There is in Sacraments to be observed their force and their form of administration. Upon their force their necessity dependeth. So that how they are necessary we cannot discern till we see how effectual they are. When Sacraments are said to be visible signs of invisible grace, we thereby conceive  how grace is indeed the very end for which these heavenly mysteries were instituted, and besides sundry other properties observed in them,
 the matter whereof they consist is such as signifieth, figureth, and representeth their end. But still their efficacy resteth obscure to our understanding, except we search somewhat more distinctly what grace in particular that is whereunto they are referred, and what manner of operation they have towards it.

The use of Sacraments is but only in this life, yet so that here they concern a far better life than this, and are for that cause accompanied with “grace which worketh Salvation.” Sacraments are the powerful instruments of God to eternal life. For as our natural life consisteth in the union of the body with the soul; so our life supernatural in the union of the soul with God. And forasmuch as there is no union of God with man without that mean between both which is both, it seemeth requisite that we first consider how God is in Christ, then how Christ is in us, and how the Sacraments do serve to make us partakers of Christ. In other things we may be more brief, but the weight of these requireth largeness.


\section*{That God is in Christ by the personal incarnation of the Son, who is very God.}
LI. “The Lord our God is but one God.” In which indivisible unity notwithstanding we adore the Father as being altogether of himself, we glorify that consubstantial Word which is the Son, we bless and magnify that co-essential Spirit eternally proceeding from both which is the Holy Ghost. Seeing therefore the Father is of none, the Son is of the Father and the Spirit is of both, they are by these their several properties really distinguishable each from other. For the substance of God with this property to be of none doth make the Person of the Father; the very selfsame substance in number with this property to be of the Father maketh the Person of the Son; the same substance having added unto it the property of proceeding from the other two maketh the Person of the Holy Ghost. So that in every Person there is implied both the substance of God which is one, and  also that property which causeth the same person really and truly to differ from the other two.
 Every person hath his own subsistence which no other besides hath, although there be others besides that are of the same substance. As no man but Peter can be the person which Peter is, yet Paul hath the selfsame nature which Peter hath. Again, angels have every of them the nature of pure and invisible spirits, but every angel is not that angel which appeared in a dream to Joseph.

[2.]Now when God became man, lest we should err in applying this to the Person of the Father, or of the Spirit, St. Peter’s confession unto Christ was, “Thou art the Son of the living God,” and St. John’s exposition thereof was plain, that it is the Word which was made Flesh. “The Father and the Holy Ghost (saith Damascen) have no communion with the incarnation of the Word otherwise than only by approbation and assent.”

Notwithstanding, forasmuch as the Word and Deity are one subject, we must beware we exclude not the nature of God from incarnation, and so make the Son of God incarnate not to be very God. For undoubtedly even the nature of God itself in the only person of the Son is incarnate, and hath taken to itself flesh. Wherefore incarnation may neither be granted to any person but only one, nor yet denied to that nature which is common unto all three.

[3.]Concerning the cause of which incomprehensible mystery, forasmuch as it seemeth a thing unconsonant that the world should honour any other as the Saviour but him whom it honoureth as the Creator of the world, and in the wisdom of God it hath not been thought convenient to admit any way of  saving man but by man himself,
 though nothing should be spoken of the love and mercy of God towards man, which this way are become such a spectacle as neither men nor angels can behold without a kind of heavenly astonishment, we may hereby perceive there is cause sufficient why divine nature should assume human, that so God might be in Christ reconciling to himself the world. And if some cause be likewise required why rather to this end and purpose the Son than either the Father or the Holy Ghost should be made man, could we which are born the children of wrath be adopted the sons of God through grace, any other than the natural Son of God being Mediator between God and us? It became therefore him by whom all things are to be the way of salvation to all, that the institution and restitution of the world might be both wrought by one hand. The world’s salvation was without the incarnation of the Son of God a thing impossible, not simply impossible, but impossible it being presupposed that the will of God was no otherwise to have it saved than by the death of his own Son. Wherefore taking to himself our flesh, and by his incarnation making it his own flesh, he had now of his own although from us what to offer unto God for us.

And as Christ took manhood that by it he might be capable of death whereunto he humbled himself, so because manhood is the proper subject of compassion and feeling pity, which maketh the sceptre of Christ’s regency even in the kingdom of heaven amiable, he which without our nature could not on earth suffer for the sins of the world, doth now also by means thereof both make intercession to God for sinners and exercise dominion over all men with a true, a natural, and a sensible touch of mercy.


\section*{The misinterpretations which heresy hath made of the manner how God and man are united in one Christ.}
LII. It is not in man’s ability either to express perfectly or conceive the manner how this was brought to pass. But the strength of our faith is tried by those things wherein our wits and capacities are not strong. Howbeit because this divine mystery is more true than plain, divers having framed the same to their own conceits and fancies are found in their expositions thereof more plain than true. Insomuch that by the space of five hundred years after Christ, the Church was almost  troubled with nothing else saving only with care and travail to preserve this article from the sinister construction of heretics.
 Whose first mists when the light of the Nicene council had dispelled, it was not long ere Macedonius transferred unto God’s most Holy Spirit the same blasphemy wherewith Arius had already dishonoured his co-eternally begotten Son; not long ere Apollinarius began to pare away from Christ’s humanity. In refutation of which impieties when the Fathers of the Church, Athanasius, Basil, and the two Gregories, had by their painful travails sufficiently cleared the truth, no less for the Deity of the Holy Ghost than for the complete humanity of Christ, there followed hereupon a final conclusion, whereby those controversies, as also the rest which Paulus Samosatenus, Sabellius, Photinus, Aëtius, Eunomius, together with the whole swarm of pestilent Demi-Arians had from time to time stirred up sithence the council of Nice, were both privately first at Rome in a smaller synod, and then at Constantinople, in a general famous assembly brought to a peaceable and quiet end, seven-score bishops and ten agreeing in that confession which by them set down remaineth at this present hour a part of our church liturgy, a memorial of their fidelity and zeal, a sovereign preservative of God’s people from the venomous infection of heresy.

[2.]Thus in Christ the verity of God and the complete substance of man were with full agreement established throughout the world, till such time as the heresy of Nestorius broached itself, “dividing Christ into two persons the Son of God and  the Son of man,
 the one a person begotten of God before all worlds, the other also a person born of the Virgin Mary, and in special favour chosen to be made entire to the Son of God above all men, so that whosoever will honour God must together honour Christ, with whose person God hath vouchsafed to join himself in so high a degree of gracious respect and favour.” But that the selfsame person which verily is man should properly be God also, and that, by reason not of two persons linked in amity but of two natures human and divine conjoined in one and the same person, the God of glory may be said as well to have suffered death as to have raised the dead from their graves, the Son of man as well to have made as to have redeemed the world, Nestorius in no case would admit.

[3.]That which deceived him was want of heed to the first beginning of that admirable combination of God with man. “The Word (saith St. John) was made flesh and dwelt in us.” The Evangelist useth the plural number, men for manhood, us for the nature whereof we consist, even as the Apostle denying the assumption of angelical nature, saith likewise in the plural number, “He took not Angels but the seed of Abraham.” It pleased not the Word or Wisdom of God to take to itself some one person amongst men, for then should that one have been advanced which was assumed and no more, but Wisdom to the end she might save many built her house of that nature which is common unto all, she made not this or that man her habitation, but dwelt in us. The seeds of herbs and plants at the first are not in act but in possibility that which they afterwards grow to be. If the Son of God had taken to himself a man now made and already perfected, it would of necessity follow that there are in Christ two persons, the one assuming and the other assumed; whereas the Son of God did not assume a man’s person unto his own, but a man’s nature to his own Person, and therefore took semen, the seed of Abraham, the very first original element of our nature, before it was come to have any personal human subsistence. The flesh and the conjunction of the flesh with God began both at one  instant; his making and taking to himself our flesh, was but one act, so that in Christ there is no personal subsistence but one, and that from everlasting. By taking only the nature of man he still continueth one person, and changeth but the manner of his subsisting, which was before in the mere glory of the Son of God, and is now in the habit of our flesh.

Forasmuch therefore as Christ hath no personal subsistence but one whereby we acknowledge him to have been eternally the Son of God, we must of necessity apply to the person of the Son of God even that which is spoken of Christ according to his human nature. For example, according to the flesh he was born of the Virgin Mary, baptized of John in the river Jordan, by Pilate adjudged to die, and executed by the Jews. We cannot say properly that the Virgin bore, or John did baptize, or Pilate condemn, or the Jews crucify the Nature of man, because these all are personal attributes; his Person is the subject which receiveth them, his Nature that which maketh his person capable or apt to receive. If we should say that the person of a man in our Saviour Christ was the subject of these things, this were plainly to entrap ourselves in the very snare of the Nestorians’ heresy, between whom and the Church of God there was no difference, saving only that Nestorius imagined in Christ as well a personal human subsistence as a divine, the Church acknowledging a substance both divine and human, but no other personal subsistence than divine, because the Son of God took not to himself a man’s person, but the nature only of a man.

Christ is a Person both divine and human, howbeit not therefore two persons in one, neither both these in one sense, but a person divine, because he is personally the Son of God, human, because he hath really the nature of the children of men. In Christ therefore God and man “There is (saith Paschasius) a twofold substance, not a twofold person, because one person extinguisheth another, whereas one nature cannot in another become extinct.” For the personal being which the Son of God already had, suffered not the substance  to be personal which he took,
 although together with the nature which he had the nature also which he took continueth. Whereupon it followeth against Nestorius, that no person was born of the Virgin but the Son of God, no person but the Son of God baptized, the Son of God condemned, the Son of God and no other person crucified; which one only point of Christian belief, the infinite worth of the Son of God, is the very ground of all things believed concerning life and salvation by that which Christ either did or suffered as man in our behalf.

[4.]But forasmuch as St. Cyril, the chiefest of those two hundred bishops assembled in the council of Ephesus, where the heresy of Nestorius was condemned, had in his writings against the Arians avouched that the Word or Wisdom of God hath but one nature which is eternal, and whereunto he assumed flesh (for the Arians were of opinion that besides God’s own eternal wisdom, there is a wisdom which God created before all things, to the end he might thereby create all things else, and that this created wisdom was the Word which took flesh:) again, forasmuch as the same Cyril had given instance in the body and the soul of man no farther than only to enforce by example against Nestorius, that a visible and an invisible, a mortal and an immortal substance may united make one person: the words of Cyril were in process of time so taken as though it had been his drift to teach, that even as in us the body and the soul, so in Christ God and man make but one nature. Of which error, six hundred and thirty fathers in the council of Chalcedon condemned Eutyches. For as Nestorius teaching rightly that God and man are distinct natures, did thereupon misinfer that in Christ those natures can by no conjunction make one person; so Eutyches, of sound belief as touching their true personal copulation, became unsound by denying the difference which still continueth between the one and the other Nature. We must therefore keep warily a middle course, shunning both that distraction of Persons wherein Nestorius went awry, and also this later confusion of Natures which deceived Eutyches.



These natures from the moment of their first combination have been and are for ever inseparable.
 For even when his soul forsook the tabernacle of his body, his Deity forsook neither body nor soul. If it had, then could we not truly hold either that the person of Christ was buried, or that the person of Christ did raise up itself from the dead. For the body separated from the Word can in no true sense be termed the person of Christ; nor is it true to say that the Son of God in raising up that body did raise up himself, if the body were not both with him and of him even during the time it lay in the sepulchre. The like is also to be said of the soul, otherwise we are plainly and inevitably Nestorians. The very person of Christ therefore for ever one and the selfsame was only touching bodily substance concluded within the grave, his soul only from thence severed, but by personal union his Deity still unseparably joined with both.


\section*{That by the union of the one with the other nature in Christ, there groweth neither gain nor loss of essential properties to either.}
LIII. The sequel of which conjunction of natures in the person of Christ is no abolishment of natural properties appertaining to either substance, no transition or transmigration thereof out of one substance into another, finally no such mutual infusion as really causeth the same natural operations or properties to be made common unto both substances; but whatsoever is natural to Deity the same remaineth in Christ uncommunicated unto his manhood, and whatsoever natural to manhood his Deity thereof is uncapable. The true properties and operations of his Deity are to know that which is not possible for created natures to comprehend; to be simply the highest cause of all things, the wellspring of immortality and life; to have neither end nor beginning of days; to be every where present, and enclosed no where; to be subject to no alteration nor passion; to produce of itself those effects which cannot proceed but from infinite majesty and power. The true properties and operations of his manhood are such as Irenæus reckoneth up: “If Christ,” saith he, “had not taken flesh  from the very earth,
 he would not have coveted those earthly nourishments, wherewith bodies which be taken from thence are fed. This was the nature which felt hunger after long fasting, was desirous of rest after travail, testified compassion and love by tears, groaned in heaviness, and with extremity of grief even melted away itself into bloody sweats.” To Christ we ascribe both working of wonders and suffering of pains, we use concerning him speeches as well of humility as of divine glory, but the one we apply unto that nature which he took of the Virgin Mary, the other to that which was in the beginning.

[2.]We may not therefore imagine that the properties of the weaker nature have vanished with the presence of the more glorious, and have been therein swallowed up as in a gulf. We dare not in this point give ear to them who over boldly affirm that “the nature which Christ took weak and feeble from us by being mingled with Deity became the same which Deity is, that the assumption of our substance unto his was like the blending of a drop of vinegar with the huge ocean, wherein although it continue still, yet not with those properties which severed it hath, because sithence the instant of their conjunction, all distinction and difference of the one from the other is extinct, and whatsoever we can now conceive of the Son of God, is nothing else but mere Deity,” which words are so plain and direct for Eutyches, that I stand in doubt they are not his whose name they carry. Sure I am they are far from truth, and must of necessity give place to  the better-advised sentences of other men.
 “He which in himself was appointed,” saith Hilary, “a Mediator to save his Church, and for performance of that mystery of mediation between God and man, is become God and man, doth now being but one consist of both those natures united, neither hath he through the union of both incurred the damage or loss of either, lest by being born a man we should think he hath given over to be God, or that because he continueth God, therefore he cannot be man also, whereas the true belief which maketh a man happy proclaimeth jointly God and man, confesseth the Word and flesh together.” Cyril more plainly; “His two natures have knit themselves the one to the other, and are in that nearness as uncapable of confusion as of distraction. Their coherence hath not taken away the difference between them. Flesh is not become God, but doth still continue flesh, although it be now the flesh of God.” Yea, “of each substance,” saith Leo, “the properties are all preserved and kept safe.”

[3.]These two natures are as causes and original grounds of all things which Christ hath done. Wherefore some things he doth as God, because his Deity alone is the wellspring from which they flow; some things as man, because they issue from his mere human nature; some things jointly as both God and man, because both natures concur as principles thereunto. For albeit the properties of each nature do cleave only to that nature whereof they are properties, and therefore Christ cannot naturally be as God the same which he naturally is as man; yet both natures may very well concur unto one effect, and Christ in that respect be truly said to work  both as God and as man one and the selfsame thing.
 Let us therefore set it down for a rule or principle so necessary as nothing more to the plain deciding of all doubts and questions about the union of natures in Christ, that of both natures there is a co-operation often, an association always, but never any mutual participation, whereby the properties of the one are infused into the other.

[4.]Which rule must serve for the better understanding of that which Damascene hath touching cross and circulatory speeches, wherein there are attributed to God such things as belong to manhood, and to man such as properly concern the Deity of Christ Jesus, the cause whereof is the association of natures in one subject. A kind of mutual commutation there is whereby those concrete names, God and Man, when we speak of Christ, do take interchangeably one another’s room, so that for truth of speech it skilleth not whether we say that the Son of God hath created the world, and the Son of Man by his death hath saved it, or else that the Son of Man did create, and the Son of God die to save the world. Howbeit, as oft as we attribute to God what the manhood of Christ claimeth, or to man what his Deity hath right unto, we understand by the name of God and the name of Man neither the one nor the other nature, but the whole person of Christ, in whom both natures are. When the Apostle saith of the Jews that they crucified the Lord of Glory, and when the Son of Man being on earth affirmeth that the Son of Man was in heaven at the same instant, there is in these two speeches that mutual circulation before-mentioned. In the one, there is attributed to God or the Lord of Glory death, whereof divine nature is not capable; in the other ubiquity unto man, which human nature admitteth not. Therefore by the Lord of Glory we must needs understand the  whole person of Christ, who being Lord of Glory, was indeed crucified, but not in that nature for which he is termed the Lord of Glory.
 In like manner by the Son of Man the whole person of Christ must necessarily be meant, who being man upon earth, filled heaven with his glorious presence, but not according to that nature for which the title of Man is given him.

Without this caution the Fathers whose belief was sincere and their meaning most sound, shall seem in their writings one to deny what another constantly doth affirm. Theodoret disputeth with great earnestness that God cannot be said to suffer. But he thereby meaneth Christ’s divine nature against Apollinarius, which held even Deity itself passible, Cyril on the other side against Nestorius as much contendeth, that whosoever will deny very God to have suffered death, doth forsake the faith. Which notwithstanding to hold were heresy, if the name of God in this assertion did not import as it doth the person of Christ, who being verily God suffered death, but in the flesh, and not in that substance for which the name of God is given him.


\section*{What Christ hath obtained according to the flesh, by the union of his flesh with Deity.}
LIV. If then both natures do remain with their properties in Christ thus distinct as hath been shewed, we are for our better understanding what either nature receiveth from other, to note, that Christ is by three degrees a receiver: first, in that  he is the Son of God; secondly, in that his human nature hath had the honour of union with Deity bestowed upon it; thirdly, in that by means thereof sundry eminent graces have flowed as effects from Deity into that nature which is coupled with it. On Christ therefore there is bestowed the gift of eternal generation,
 the gift of union, and the gift of unction.

[2.]By the gift of eternal generation Christ hath received of the Father one and in number the selfsame substance, which the Father hath of himself unreceived from any other. For every beginning is a Father unto that which cometh of it; and every offspring is a Son unto that out of which it groweth. Seeing therefore the Father alone is originally that Deity which Christ originally is not, (for Christ is God by being of God, light by issuing out of light,) it followeth hereupon that whatsoever Christ hath common unto him with his heavenly Father, the same of necessity must be given him,  but naturally and eternally given, not bestowed by way of benevolence and favour, as the other gifts both are.
 And therefore where the Fathers give it out for a rule, that whatsoever Christ is said in Scripture to have received, the same we ought to apply only to the manhood of Christ; their assertion is true of all things which Christ hath received by grace, but to that which he hath received of the Father by eternal nativity or birth it reacheth not.

[3.]Touching union of Deity with manhood, it is by grace, because there can be no greater grace shewed towards man, than that God should vouchsafe to unite to man’s nature the person of his only begotten Son. Because “the Father loveth the Son” as man, he hath by uniting Deity with manhood, “given all things into his hands.” It hath pleased the Father, that in him “all fulness should dwell.” The “name” which he hath “above all names” is given him. “As the Father hath life in himself,” the “Son in himself hath life also” by the gift of the Father. The gift whereby God hath made Christ a fountain of life is that “conjunction of the nature of God with the nature of man” in the person of Christ, “which gift,” (saith Christ to the woman of Samaria,) “if thou didst know and in that respect understand who it is which asketh water of thee, thou wouldest ask of him that he might give thee living water.” The union therefore of the flesh with Deity is to that flesh a gift of principal grace and favour. For by virtue of this grace, man is really made God, a creature is exalted above the dignity of all creatures, and hath all creatures else under it.




[4.]This admirable union of God with man can enforce in that higher nature no alteration, because unto God there is nothing more natural than not to be subject to any change. Neither is it a thing impossible that the Word being made flesh should be that which it was not before as touching the manner of subsistence, and yet continue in all qualities or properties of nature the same it was, because the incarnation of the Son of God consisteth merely in the union of natures, which union doth add perfection to the weaker, to the nobler no alteration at all. If therefore it be demanded what the person of the Son of God hath attained by assuming manhood, surely, the whole sum of all is this, to be as we are truly, really, and naturally man, by means whereof he is made capable of meaner offices than otherwise his person could have admitted, the only gain he thereby purchased for himself was to be capable of loss and detriment for the good of others.

[5.]But may it rightly be said concerning the incarnation of Jesus Christ, that as our nature hath in no respect changed his, so from his to ours as little alteration hath ensued? The very cause of his taking upon him our nature was to change it, to better the quality, and to advance the condition thereof, although in no sort to abolish the substance which he took, nor to infuse into it the natural forces and properties of his Deity. As therefore we have shewed how the Son of God by his incarnation hath changed the manner of that personal subsistence which before was solitary, and is now in the association of flesh, no alteration thereby accruing to the nature of God; so neither are the properties of man’s nature in the person of Christ by force and virtue of the same conjunction so much altered, as not to stay within those limits which our substance is bordered withal; nor the state and quality of our substance so unaltered, but that there are in it many glorious effects  proceeding from so near copulation with Deity.
 God from us can receive nothing, we by him have obtained much. For albeit the natural properties of Deity be not communicable to man’s nature, the supernatural gifts graces and effects thereof are.

The honour which our flesh hath by being the flesh of the Son of God is in many respects great. If we respect but that which is common unto us with him, the glory provided for him and his in the kingdom of heaven, his right and title thereunto even in that he is man differeth from other men’s, because he is that man of whom God is himself a part. We have right to the same inheritance with Christ, but not the same right which he hath, his being such as we cannot reach, and ours such as he cannot stoop unto.

Furthermore, to be the Way, the Truth, and the Life; to be the Wisdom, Righteousness, Sanctification, Resurrection; to be the Peace of the whole world, the Hope of the righteous, the Heir of all things; to be that supreme Head whereunto all power both in heaven and in earth is given: these are not honours common unto Christ with other men, they are titles above the dignity and worth of any which were but a mere man, yet true of Christ even in that he is man, but man with whom Deity is personally joined, and unto whom it hath added those excellencies which make him more than worthy thereof.

Finally, sith God hath deified our nature, though not by turning it into himself, yet by making it his own inseparable habitation, we cannot now conceive how God should without man either exercise divine power, or receive the glory of divine praise. For man is in both an associate of Deity.

[6.]But to come to the grace of unction: did the parts of our nature, the soul and body of Christ, receive by the influence  of Deity wherewith they were matched no ability of operation, no virtue or quality above nature?
 Surely as the sword which is made fiery doth not only cut by reason of the sharpness which simply it hath, but also burn by means of that heat which it hath from fire, so there is no doubt but the Deity of Christ hath enabled that nature which it took of man to do more than man in this world hath power to comprehend; forasmuch as (the bare essential properties of Deity excepted) he hath imparted unto it all things, he hath replenished it with all such perfections as the same is any way apt to receive, at the least according to the exigence of that economy or service for which it pleased him in love and mercy to be made man. For as the parts, degrees, and offices of that mystical administration did require which he voluntarily undertook, the beams of Deity did in operation always accordingly either restrain or enlarge themselves.

[7.]From hence we may somewhat conjecture how the powers of that soul are illuminated, which being so inward unto God cannot choose but be privy unto all things which God worketh, and must therefore of necessity be endued with knowledge so far forth universal, though not with infinite knowledge peculiar to Deity itself. The soul of Christ that saw in this life the face of God was here through so visible presence of Deity filled with all manner graces and virtues in that unmatchable degree of perfection, for which of him we read it written, “That God with the oil of gladness anointed him above his fellows.”

[8.]And as God hath in Christ unspeakably glorified the nobler, so likewise the meaner part of our nature, the very bodily substance of man. Where also that must again be remembered which we noted before concerning degrees of the  influence of Deity proportionable unto his own purposes, intents, and counsels.
 For in this respect his body which by natural condition was corruptible wanted the gift of everlasting immunity from death, passion, and dissolution, till God which gave it to be slain for sin had for righteousness’ sake restored it to life with certainty of endless continuance. Yea in this respect the very glorified body of Christ retained in it the scars and marks of former mortality.

[9.]But shall we say that in heaven his glorious body by virtue of the same cause hath now power to present itself in all places and to be every where at once present? We nothing doubt but God hath many ways above the reach of our capacities exalted that body which it hath pleased him to make his own, that body wherewith he hath saved the world, that body which hath been and is the root of eternal life, the instrument wherewith Deity worketh, the sacrifice which taketh away sin, the price which hath ransomed souls from death, the leader of the whole army of bodies that shall rise again. For though it had a beginning from us, yet God hath given it vital efficacy, heaven hath endowed it with celestial power, that virtue it hath from above, in regard whereof all the angels of heaven adore it. Notwithstanding a body still it continueth, a body consubstantial with our bodies, a body of the same both nature and measure which it had on earth.

[10.]To gather therefore into one sum all that hitherto hath been spoken touching this point, there are but four things which concur to make complete the whole state of our Lord Jesus Christ: his Deity, his manhood, the conjunction of both, and the distinction of the one from the other being joined in one. Four principal heresies there are which have in those things withstood the truth: Arians by bending themselves against the Deity of Christ; Apollinarians by maiming and misinterpreting that which belongeth to his human nature; Nestorians by rending Christ asunder, and  dividing him into two persons; the followers of Eutyches by confounding in his person those natures which they should distinguish.
 Against these there have been four most famous ancient general councils: the council of Nice to define against Arians, against Apollinarians the council of Constantinople, the council of Ephesus against Nestorians, against Eutychians the Chalcedon council. In four words, ἀληθω̑ς, τελέως, ἀδιαιρέτως, ἀσυγχύτως, truly, perfectly, indivisibly, distinctly, the first applied to his being God, and the second to his being Man, the third to his being of both One, and the fourth to his still continuing in that one Both, we may fully by way of abridgment comprise whatsoever antiquity hath at large handled either in declaration of Christian belief, or in refutation of the foresaid heresies. Within the compass of which four heads, I may truly affirm, that all heresies which touch but the person of Jesus Christ, whether they have risen in these later days, or in any age heretofore, may be with great facility brought to confine themselves.

We conclude therefore that to save the world it was of necessity the Son of God should be thus incarnate, and that God should so be in Christ as hath been declared.


\section*{Of the personal presence of Christ every where, and in what sense it may be granted he is every where present according to the flesh.}
LV. Having thus far proceeded in speech concerning the person of Jesus Christ, his two natures, their conjunction, that which he either is or doth in respect of both, and that which the one receiveth from the other; sith God in Christ is generally the medicine which doth cure the world, and Christ in us is that receipt of the same medicine, whereby we are every one particularly cured, inasmuch as Christ’s incarnation and passion can be available to no man’s good which is not made partaker of Christ, neither can we participate him without his presence, we are briefly to consider how Christ is present, to the end it may thereby better appear how we are made partakers of Christ both otherwise and in the Sacraments themselves.

[2.]All things are in such sort divided into finite and infinite, that no one substance, nature, or quality, can be possibly capable of both. The world and all things in the world are stinted, all effects that proceed from them, all the powers and abilities whereby they work, whatsoever they do, whatsoever they may, and whatsoever they are, is limited. Which limitation  of each creature is both the perfection and also the preservation thereof.
 Measure is that which perfecteth all things, because every thing is for some end, neither can that thing be available to any end which is not proportionable thereunto, and to proportion as well excesses as defects are opposite. Again, forasmuch as nothing doth perish but only through excess or defect of that, the due proportioned measure whereof doth give perfection, it followeth that measure is likewise the preservation of all things. Out of which premises we may conclude not only that nothing created can possibly be unlimited, or can receive any such accident, quality, or property, as may really make it infinite, (for then should it cease to be a creature,) but also that every creature’s limitation is according to his own kind, and, therefore as oft as we note in them any thing above their kind, it argueth that the same is not properly theirs, but groweth in them from a cause more powerful than they are.

[3.]Such as the substance of each thing is, such is also the presence thereof. Impossible it is that God should withdraw his presence from any thing, because the very substance of God is infinite. He filleth heaven and earth, although he take up no room in either, because his substance is immaterial, pure, and of us in this world so incomprehensible, that albeit no part of us be ever absent from him who is present whole unto every particular thing, yet his presence with us we no way discern farther than only that God is present, which partly by reason and more perfectly by faith we know to be firm and certain.

[4.]Seeing therefore that presence every where is the sequel of an infinite and incomprehensible substance, (for what can be every where but that which can no where be comprehended?) to inquire whether Christ be every where is to inquire of a natural property, a property that cleaveth to the Deity of Christ. Which Deity being common unto him with none but only the Father and the Holy Ghost, it followeth that nothing  of Christ which is limited,
 that nothing created, that neither the soul nor the body of Christ, and consequently not Christ as man or Christ according to his human nature can possibly be every where present, because those phrases of limitation and restraint do either point out the principal subject whereunto every such attribute adhereth, or else they intimate the radical cause out of which it groweth. For example, when we say that Christ as man or according to his human nature suffered death, we shew what nature was the proper subject of mortality; when we say that as God or according to his Deity he conquered death, we declare his Deity to have been the cause, by force and virtue whereof he raised himself from the grave. But neither is the manhood of Christ that subject whereunto universal presence agreeth, neither is it the cause original by force whereof his Person is enabled to be every where present. Wherefore Christ is essentially present with all things, in that he is very God, but not present with all things as man, because manhood and the parts thereof can neither be the cause nor the true subject of such presence.

[5.]Notwithstanding, somewhat more plainly to shew a true immediate reason wherefore the manhood of Christ can neither be every where present, nor cause the person of Christ so to be; we acknowledge that of St. Augustine concerning Christ most true, “In that he is personally the Word he created all things, in that he is naturally man he himself is created of God,” and it doth not appear that any one creature hath power to be present with all creatures. Whereupon, nevertheless it will not follow that Christ cannot therefore be thus present, because he is himself a creature, forasmuch as only infinite presence is that which cannot possibly stand with the essence or being of any creature: as for presence with all things that are, sith the whole race, mass, and body  of them is finite, Christ by being a creature is not in that respect excluded from possibility of presence with them.
 That which excludeth him therefore as man from so great largeness of presence, is only his being man, a creature of this particular kind, whereunto the God of nature hath set those bounds of restraint and limitation, beyond which to attribute unto it any thing more than a creature of that sort can admit, were to give it another nature, to make it a creature of some other kind than in truth it is.

[6.]Furthermore if Christ in that he is man be every where present, seeing this cometh not by the nature of manhood itself, there is no other way how it should grow but either by the grace of union with Deity, or by the grace of unction received from Deity. It hath been already sufficiently proved that by force of union the properties of both natures are imparted to the person only in whom they are, and not what belongeth to the one nature really conveyed or translated into the other; it hath been likewise proved that natures united in Christ continue the very same which they are where they are not united. And concerning the grace of unction, wherein are contained the gifts and virtues which Christ as man hath above men, they make him really and habitually a man more excellent than we are, they take not from him the nature and substance that we have, they cause not his soul nor body to be of another kind than ours is. Supernatural endowments are an advancement, they are no extinguishment of that nature whereto they are given.

The substance of the body of Christ hath no presence, neither can have, but only local. It was not therefore every where seen, nor did it every where suffer death, every where it could not be entombed, it is not every where now being exalted into heaven. There is no proof in the world strong enough to enforce that Christ had a true body but by the true and natural properties of his body. Amongst which properties, definite or local presence is chief. “How is it true of Christ (saith Tertullian) that he died, was buried, and rose again, if Christ had not that very flesh the nature whereof is capable of these things, flesh mingled with blood, supported with bones, woven with sinews, embroidered with  veins?”
 If his majestical body have now any such new property, by force whereof it may every where really even in substance present itself, or may at once be in many places, then hath the majesty of his estate extinguished the verity of his nature. “Make thou no doubt or question of it” (saith St. Augustine) “but that the man Christ Jesus is now in that very place from whence he shall come in the same form and substance of flesh which he carried thither, and from which he hath not taken nature, but given thereunto immortality. According to this form he spreadeth not out himself into all places. For it behoveth us to take great heed, lest while we go about to maintain the glorious Deity of him which is man, we leave him not the true bodily substance of a man.” According to St. Augustine’s opinion therefore that majestical body which we make to be every where present, doth thereby cease to have the substance of a true body.

[7.]To conclude, we hold it in regard of the fore-alleged proofs a most infallible truth that Christ as man is not every where present. There are which think it as infallibly true, that Christ is every where present as man, which peradventure in some sense may be well enough granted. His human substance in itself is naturally absent from the earth, his soul and body not on earth but in heaven only. Yet because the substance is inseparably joined to that personal Word which by his very divine essence is present with all things, the nature which cannot have in itself universal presence hath it after a  sort by being nowhere severed from that which every where is present.
 For inasmuch as that infinite Word is not divisible into parts, it could not in part but must needs be wholly incarnate, and consequently, wheresoever the Word is, it hath with it manhood, else should the Word be in part or somewhere God only and not Man, which is impossible. For the Person of Christ is whole, perfect God and perfect Man wheresoever, although the parts of his Manhood being finite and his Deity infinite, we cannot say that the whole of Christ is simply every where, as we may say that his Deity is, and that his Person is by force of Deity. For somewhat of the Person of Christ is not every where in that sort, namely his manhood, the only conjunction whereof with Deity is extended as far as Deity, the actual position restrained and tied to a certain place; yet presence by way of conjunction is in some sort presence.

[8.]Again, as the manhood of Christ may after a sort be every where said to be present, because that Person is every where present, from whose divine substance manhood nowhere is severed: so the same universality of presence may likewise seem in another respect appliable thereunto, namely by co-operation with Deity, and that in all things. The light created of God in the beginning did first by itself illuminate the world; but after that the Sun and Moon were created, the world sithence hath by them always enjoyed the same. And that Deity of Christ which before our Lord’s incarnation wrought all things without man, doth now work nothing wherein the nature which it hath assumed is either absent from it or idle. Christ as man hath all power both in heaven and earth given him. He hath as Man, not as God only, supreme dominion over quick and dead, for so much his ascension into heaven, and his session at the right hand of God do import. The Son of God which did first humble himself by taking our flesh upon him, descended afterwards much lower, and became according to the flesh obedient so far as to suffer death, even the death of the cross, for all men, because such was his Father’s will. The former was an humiliation of Deity, the later an humiliation of manhood, for which cause there  followed upon the later an exaltation of that which was humbled; for with power he created the world, but restored it by obedience. In which obedience as according to his manhood he had glorified God on earth, so God hath glorified in heaven that nature which yielded him obedience, and hath given unto Christ even in that he is man such fulness of power over the whole world, that he which before fulfilled in the state of humility and patience whatsoever God did require, doth now reign in glory till the time that all things be restored. He which came down from heaven and descended into the lowest parts of the earth is ascended far above all heavens, that sitting at the right hand of God he might from thence fill all things with the gracious and happy fruits of his saving presence. Ascension into heaven is a plain local translation of Christ according to his manhood from the lower to the higher parts of the world. Session at the right hand of God is the actual exercise of that regency and dominion wherein the manhood of Christ is joined and matched with the Deity of the Son of God. Not that his manhood was before without the possession of the same power, but because the full use thereof was suspended till that humility, which had been before as a veil to hide and conceal majesty, were laid aside. After his rising again from the dead, then did God set him at his right hand in heavenly places far above all principality, and power, and might, and domination, and every name that is named not in this world only but also in that which is to come, and hath put all things under his feet, and hath appointed him over all the Head to the Church which is his body, the fulness of him that filleth all in all. The sceptre of which spiritual regiment over us in this present world is at the length to be yielded up into the hands of the Father which gave it; that is to say the use and exercise thereof shall  cease, there being no longer on earth any militant Church to govern.
 This government therefore he exerciseth both as God and as Man, as God by essential presence with all things, as Man by co-operation with that which essentially is present. Touching the manner how he worketh as man in all things; the principal powers of the soul of man are the will and understanding, the one of which two in Christ assenteth unto all things, and from the other nothing which Deity doth work is hid; so that by knowledge and assent the soul of Christ is present with all things which the Deity of Christ worketh.

[9.]And even the body of Christ itself, although the definite limitation thereof be most sensible, doth notwithstanding admit in some sort a kind of infinite and unlimited presence likewise. For his body being a part of that nature which whole nature is presently joined unto Deity wheresoever Deity is, it followeth that his bodily substance hath every where a presence of true conjunction with Deity. And forasmuch as it is by virtue of that conjunction made the body of the Son of God, by whom also it was made a sacrifice for the sins of the whole world, this giveth it a presence of force and efficacy throughout all generations of men. Albeit therefore nothing be actually infinite in substance but God only in that he is God, nevertheless as every number is infinite by possibility of addition, and every line by possibility of extension infinite, so there is no stint which can be set to the value or merit of the sacrificed body of Christ, it hath no measured certainty of limits, bounds of efficacy unto life it knoweth none, but is also itself infinite in possibility of application.

Which things indifferently every way considered, that gracious promise of our Lord and Saviour Jesus Christ concerning presence with his to the very end of the world, I see no cause but that we may well and safely interpret he doth perform both as God by essential presence of Deity, and as Man in that order, sense, and meaning, which hath been shewed.


\section*{The union or mutual participation which is between Christ and the Church of Christ in this present world.}
LVI. We have hitherto spoken of the Person and of the presence of Christ. Participation is that mutual inward hold which Christ hath of us and we of him, in such sort that each possesseth other by way of special interest, property, and inherent copulation. For plainer explication whereof we may from that which hath been before sufficiently proved assume  to our purpose these two principles, “That every original cause imparteth itself unto those things which come of it;” and “whatsoever taketh being from any other, the same is after a sort in that which giveth it being.”


[2.]It followeth hereupon that the Son of God being light of light, must needs be also light in light. The Persons of the Godhead, by reason of the unity of their substance, do as necessarily remain one within another, as they are of necessity to be distinguished one from another, because two are the issue of one, and one the offspring of the other two, only of three one not growing out of any other. And sith they all are but one God in number, one indivisible essence or substance, their distinction cannot possibly admit separation. For how should that subsist solitarily by itself which hath no substance but individually the very same whereby others subsist with it; seeing that the multiplication of substances in particular is necessarily required to make those things subsist apart which have the selfsame general nature, and the Persons of that Trinity are not three particular substances to whom one general nature is common, but three that subsist by one substance which itself is particular, yet they all three have it, and their several ways of having it are that which maketh their personal distinction? The Father therefore is in the Son, and the Son in him, they both in the Spirit, and the Spirit in both them. So that the Father’s first offspring, which is the Son, remaineth eternally in the Father; the Father eternally also in the Son, no way severed or divided by reason of the sole and single unity of their substance. The Son in the Father as light in that light out of which it floweth without separation; the Father in the Son as light in that light which it causeth and leaveth not. And because in this respect his eternal being is of the Father, which eternal being is his life, therefore he by the Father liveth.

[3.]Again, sith all things do accordingly love their offspring as themselves are more or less contained in it, he which  is thus the only-begotten, must needs be in this degree the only-beloved of the Father.
 He therefore which is in the Father by eternal derivation of being and life from him, must needs be in him through an eternal affection of love.

[4.]His Incarnation causeth him also as man to be now in the Father, and the Father to be in him. For in that he is man, he receiveth life from the Father as from the fountain of that ever living Deity, which in the person of the Word hath combined itself with manhood, and doth thereunto impart such life as to no other creature besides him is communicated. In which consideration likewise the love of the Father towards him is more than it can be towards any other, neither can any attain unto that perfection of love which he beareth towards his heavenly Father. Wherefore God is not so in any, nor any so in God as Christ, whether we consider him as the personal Word of God, or as the natural Son of man.

[5.]All other things that are of God have God in them and he them in himself likewise. Yet because their substance and his wholly differeth, their coherence and communion either with him or amongst themselves is in no sort like unto that before-mentioned.

God hath his influence into the very essence of all things, without which influence of Deity supporting them their utter annihilation could not choose but follow. Of him all things have both received their first being and their continuance to be that which they are. All things are therefore partakers of God, they are his offspring, his influence is in them, and the personal wisdom of God is for that very cause said to excel in nimbleness or agility, to pierce into all intellectual, pure, and subtile spirits, to go through all, and to reach unto every thing which is. Otherwise, how should the same wisdom be that which supporteth, beareth up, and sustaineth all?

Whatsoever God doth work, the hands of all three Persons are jointly and equally in it according to the order of that connexion whereby they each depend upon other. And therefore albeit in that respect the Father be first, the Son next, the Spirit last, and consequently nearest unto every effect  which groweth from all three, nevertheless, they all being of one essence, are likewise all of one efficacy.
 Dare any man unless he be ignorant altogether how inseparable the Persons of the Trinity are, persuade himself that every of them may have their sole and several possessions, or that we being not partakers of all, can have fellowship with any one? The Father as Goodness, the Son as Wisdom, the Holy Ghost as Power do all concur in every particular outwardly issuing from that one only glorious Deity which they all are. For that which moveth God to work is goodness, and that which ordereth his work is Wisdom, and that which perfecteth his work is Power. All things which God in their times and seasons hath brought forth were eternally and before all times in God, as a work unbegun is in the artificer which afterward bringeth it unto effect. Therefore whatsoever we do behold now in this present world, it was enwrapped within the bowels of divine Mercy, written in the book of eternal Wisdom, and held in the hands of omnipotent Power, the first foundations of the world being as yet unlaid.

So that all things which God hath made are in that respect the offspring of God, they are in him as effects in their highest cause, he likewise actually is in them, the assistance and influence of his Deity is their life.

[6.]Let hereunto saving efficacy be added, and it bringeth forth a special offspring amongst men, containing them to whom God hath himself given the gracious and amiable name of sons. We are by nature the sons of Adam. When God created Adam he created us, and as many as are descended from Adam have in themselves the root out of which they spring. The sons of God we neither are all nor any one of us otherwise than only by grace and favour. The sons of God have God’s own natural Son as a second Adam from heaven, whose race and progeny they are by spiritual and heavenly birth. God therefore loving eternally his Son, he must needs eternally in him have loved and preferred before all others them which are spiritually sithence descended and sprung out of him. These were in God as in their Saviour,  and not as in their Creator only.
 It was the purpose of his saving Goodness, his saving Wisdom, and his saving Power which inclined itself towards them.

[7.]They which thus were in God eternally by their intended admission to life, have by vocation or adoption God actually now in them, as the artificer is in the work which his hand doth presently frame. Life as all other gifts and benefits groweth originally from the Father, and cometh not to us but by the Son, nor by the Son to any of us in particular but through the Spirit. For this cause the Apostle wisheth to the church of Corinth, “The grace of our Lord Jesus Christ, and the love of God, and the fellowship of the Holy Ghost.” Which three St. Peter comprehendeth in one, “The participation of divine Nature.” We are therefore in God through Christ eternally according to that intent and purpose whereby we were chosen to be made his in this present world before the world itself was made, we are in God through the knowledge which is had of us, and the love which is borne towards us from everlasting. But in God we actually are no longer than only from the time of our actual adoption into the body of his true Church, into the fellowship of his children. For his Church he knoweth and loveth, so that they which are in the Church are thereby known to be in him. Our being in Christ by eternal foreknowledge saveth us not without our actual and real adoption into the fellowship of his saints in this present world. For in him we actually are by our actual incorporation into that society which hath him for their Head, and doth make together with him one Body, (he and they in that respect having one name,) for which cause, by virtue of this mystical conjunction, we are of him and in him even as though our very flesh and bones should be made continuate with his. We are in Christ because he knoweth and loveth us even as parts of himself. No man actually is in him but they in whom he actually is. For he which hath not the Son of God hath not life.” “I am the vine and you are the branches: he which abideth in me and I in him the same bringeth forth much fruit;” but  the branch severed from the vine withereth. We are therefore adopted sons of God to eternal life by participation of the only-begotten Son of God, whose life is the well-spring and cause of ours.

It is too cold an interpretation, whereby some men expound our being in Christ to import nothing else, but only that the selfsame nature which maketh us to be men, is in him, and maketh him man as we are. For what man in the world is there which hath not so far forth communion with Jesus Christ? It is not this that can sustain the weight of such sentences as speak of the mystery of our coherence with Jesus Christ. The Church is in Christ as Eve was in Adam. Yea by grace we are every of us in Christ and in his Church, as by nature we are in those our first parents. God made Eve of the rib of Adam. And his Church he frameth out of the very flesh, the very wounded and bleeding side of the Son of man. His body crucified and his blood shed for the life of the world, are the true elements of that heavenly being, which maketh us such as himself is of whom we come. For which cause the words of Adam may be fitly the words of Christ concerning his Church, “flesh of my flesh, and bone of my bones,” a true native extract out of mine own body. So that in him even according to his manhood we according to our heavenly being are as branches in that root out of which they grow.

To all things he is life, and to men light, as the Son of God; to the Church both life and light eternal by being made the Son of Man for us, and by being in us a Saviour, whether we respect him as God, or as man. Adam is in us as an original cause of our nature, and of that corruption of nature which causeth death, Christ as the cause original of restoration to life; the person of Adam is not in us, but his nature, and the corruption of his nature derived into all men by propagation; Christ having Adam’s nature as we have, but incorrupt, deriveth not nature but incorruption and that immediately from his own person into all that belong unto him. As therefore we are really partakers of the body of sin and death  received from Adam, so except we be truly partakers of Christ, and as really possessed of his Spirit, all we speak of eternal life is but a dream.

[8.]That which quickeneth us is the Spirit of the second Adam, and his flesh that wherewith he quickeneth. That which in him made our nature uncorrupt, was the union of his Deity with our nature. And in that respect the sentence of death and condemnation which only taketh hold upon sinful flesh, could no way possibly extend unto him. This caused his voluntary death for others to prevail with God, and to have the force of an expiatory sacrifice. The blood of Christ, as the Apostle witnesseth, doth therefore take away sin, because “through the eternal Spirit he offered himself unto God without spot.” That which sanctified our nature in Christ, that which made it a sacrifice available to take away sin, is the same which quickeneth it, raised it out of the grave after death, and exalted it unto glory. Seeing therefore that Christ is in us as a quickening Spirit, the first degree of communion with Christ must needs consist in the participation of his Spirit, which Cyprian in that respect well termeth germanissimam societatem, the highest and truest society that can be between man and him which is both God and man in one.

[9.]These things St. Cyril duly considering, reproveth their speeches which taught that only the deity of Christ is the vine whereupon we by faith do depend as branches, and that neither his flesh nor our bodies are comprised in this resemblance. For doth any man doubt but that even from the flesh of Christ our very bodies do receive that life which shall make them glorious at the later day, and  for which they are already accounted parts of his blessed body?
 Our corruptible bodies could never live the life they shall live, were it not that here they are joined with his body which is incorruptible, and that his is in ours as a cause of immortality, a cause by removing through the death and merit of his own flesh that which hindered the life of ours. Christ is therefore both as God and as man that true vine whereof we both spiritually and corporally are branches. The mixture of his bodily substance with ours is a thing which the ancient Fathers disclaim. Yet the mixture of his flesh with ours they speak of, to signify what our very bodies through mystical conjunction receive from that vital efficacy which we know to be in his: and from bodily mixtures they borrow divers similitudes rather to declare the truth, than the manner of coherence between his sacred and the sanctified bodies of saints.

[10.]Thus much no Christian man will deny, that when Christ sanctified his own flesh, giving as God and taking as man the Holy Ghost, he did not this for himself only but for our sakes, that the grace of sanctification and life which was first received in him might pass from him to his whole race, as malediction came from Adam unto all mankind. Howbeit, because the work of his Spirit to those effects is in us prevented by sin and death possessing us before, it is of necessity that as well our present sanctification unto newness of life, as the future restoration of our bodies should presuppose a participation of the grace, efficacy, merit or virtue of  his body and blood, without which foundation first laid there is no place for those other operations of the Spirit of Christ to ensue. So that Christ imparteth plainly himself by degrees.

It pleaseth him in mercy to account himself incomplete and maimed without us. But most assured we are that we all receive of his fulness, because he is in us as a moving and working cause; from which many blessed effects are really found to ensue, and that in sundry both kinds and degrees, all tending to eternal happiness. It must be confessed that of Christ, working as a Creator, and a Governor of the world by providence, all are partakers; not all partakers of that grace whereby he inhabiteth whom he saveth.

Again, as he dwelleth not by grace in all, so neither doth he equally work in all them in whom he dwelleth. “Whence is it (saith St. Augustine) that some be holier than others are, but because God doth dwell in some more plentifully than in others?”

And because the divine substance of Christ is equally in all, his human substance equally distant from all, it appeareth that the participation of Christ wherein there are many degrees and differences, must needs consist in such effects as being derived from both natures of Christ really into us, are made our own, and we by having them in us are truly said to have him from whom they come, Christ also more or less to inhabit and impart himself as the graces are fewer or more, greater or smaller, which really flow into us from Christ.

Christ is whole with the whole Church, and whole with every part of the Church, as touching his Person, which can no way divide itself, or be possessed by degrees and portions. But the participation of Christ importeth, besides the presence of Christ’s Person, and besides the mystical copulation thereof with the parts and members of his whole Church, a true actual influence of grace whereby the life which we live according to godliness is his, and from him we receive those perfections wherein our eternal happiness consisteth.




[11.]Thus we participate Christ partly by imputation, as when those things which he did and suffered for us are imputed unto us for righteousness; partly by habitual and real infusion, as when grace is inwardly bestowed while we are on earth, and afterwards more fully both our souls and bodies made like unto his in glory. The first thing of his so infused into our hearts in this life is the Spirit of Christ, whereupon because the rest of what kind soever do all both necessarily depend and infallibly also ensue, therefore the Apostles term it sometime the seed of God, sometime the pledge of our heavenly inheritance, sometime the handsel or earnest of that which is to come. From hence it is that they which belong to the mystical body of our Saviour Christ, and be in number as the stars of heaven, divided successively by reason of their mortal condition into many generations, are notwithstanding coupled every one to Christ their Head, and all unto every particular person amongst themselves, inasmuch as the same Spirit, which anointed the blessed soul of our Saviour Christ, doth so formalize, unite and actuate his whole race, as if both he and they were so many limbs compacted into one body, by being quickened all with one and the same soul.

[12.]That wherein we are partakers of Jesus Christ by imputation, agreeth equally unto all that have it. For it consisteth in such acts and deeds of his as could not have longer continuance than while they were in doing, nor at that very time belong unto any other but to him from whom they came, and therefore how men either then or before or sithence should be made partakers of them, there can be no way imagined but only by imputation. Again, a deed must either not be imputed to any, but rest altogether in him whose it is, or if at all it be imputed, they which have it by imputation must have it such as it is whole. So that degrees being neither in the personal presence of Christ, nor in the participation of those effects which are ours by imputation only, it resteth that we wholly apply them to the participation of Christ’s infused grace, although even in this kind also the first  beginning of life, the seed of God, the first-fruits of Christ’s Spirit be without latitude.
 For we have hereby only the being of the Sons of God, in which number how far soever one may seem to excel another, yet touching this that all are sons, they are all equals, some haply better sons than the rest are, but none any more a son than another.

[13.]Thus therefore we see how the Father is in the Son, and the Son in the Father; how they both are in all things, and all things in them; what communion Christ hath with his Church, how his Church and every member thereof is in him by original derivation, and he personally in them by way of mystical association wrought through the gift of the Holy Ghost, which they that are his receive from him, and together with the same what benefit soever the vital force of his body and blood may yield, yea by steps and degrees they receive the complete measure of all such divine grace, as doth sanctify and save throughout, till the day of their final exaltation to a state of fellowship in glory, with him whose partakers they are now in those things that tend to glory. As for any mixture of the substance of his flesh with ours, the participation which we have of Christ includeth no such kind of gross surmise.


\section*{The necessity of Sacraments unto the participation of Christ.}
LVII. It greatly offendeth, that some, when they labour to shew the use of the holy Sacraments, assign unto them no end but only to teach the mind, by other senses, that which the Word doth teach by hearing. Whereupon, how easily neglect and careless regard of so heavenly mysteries may follow, we see in part by some experience had of those men with whom that opinion is most strong. For where the word of God may be heard, which teacheth with much more expedition and more full explication any thing we have to learn, if all the benefit we reap by sacraments be instruction, they which at all times have opportunity of using the better mean to that purpose, will surely hold the worse in less estimation. And unto infants which are not capable of instruction, who would not think it a mere superfluity that any sacrament is administered, if to administer the sacraments be but to teach receivers what God doth for them? There is of sacraments therefore undoubtedly some other more excellent and heavenly use.




[2.]Sacraments, by reason of their mixed nature, are more diversely interpreted and disputed of than any other part of religion besides, for that in so great store of properties belonging to the selfsame thing, as every man’s wit hath taken hold of some especial consideration above the rest, so they have accordingly seemed one to cross another as touching their several opinions about the necessity of sacraments, whereas in truth their disagreement is not great. For let respect be had to the duty which every communicant doth undertake, and we may well determine concerning the use of sacraments, that they serve as bonds of obedience to God, strict obligations to the mutual exercise of Christian charity, provocations to godliness, preservations from sin, memorials of the principal benefits of Christ; respect the time of their institution, and it thereby appeareth that God hath annexed them for ever unto the New Testament, as other rites were before with the Old; regard the weakness which is in us, and they are warrants for the more security of our belief; compare the receivers of them with such as receive them not, and sacraments are marks of distinction to separate God’s own from strangers: so that in all these respects, they are found to be most necessary.

[3.]But their chiefest force and virtue consisteth not herein so much as in that they are heavenly ceremonies, which God hath sanctified and ordained to be administered in his Church, first, as marks whereby to know when God doth impart the vital or saving grace of Christ unto all that are capable thereof, and secondly as means conditional which  God requireth in them unto whom he imparteth grace.
 For sith God in himself is invisible, and cannot by us be discerned working, therefore when it seemeth good in the eyes of his heavenly wisdom, that men for some special intent and purpose should take notice of his glorious presence, he giveth them some plain and sensible token whereby to know what they cannot see. For Moses to see God and live was impossible, yet Moses by fire knew where the glory of God extraordinarily was present. The angel, by whom God endued the waters of the pool called Bethesda with supernatural virtue to heal, was not seen of any, yet the time of the angel’s presence known by the troubled motions of the waters themselves. The Apostles by fiery tongues which they saw, were admonished when the Spirit, which they could not behold, was upon them. In like manner it is with us. Christ and his Holy Spirit with all their blessed effects, though entering into the soul of man we are not able to apprehend or express how, do notwithstanding give notice of the times when they use to make their access, because it pleaseth Almighty God to communicate by sensible means those blessings which are incomprehensible.

[4.]Seeing therefore that grace is a consequent of sacraments, a thing which accompanieth them as their end, a benefit which he that hath receiveth from God himself the author of sacraments, and not from any other natural or supernatural quality in them, it may be hereby both understood that sacraments are necessary, and that the manner of their necessity to life supernatural is not in all respects as food unto natural life, because they contain in themselves no vital force or efficacy, they are not physical but moral instruments of salvation, duties of service and worship, which unless we perform as the Author of grace requireth, they are unprofitable. For  all receive not the grace of God which receive the sacraments of his grace.
 Neither is it ordinarily his will to bestow the grace of sacraments on any, but by the sacraments; which grace also they that receive by sacraments or with sacraments, receive it from him and not from them. For of sacraments the very same is true which Salomon’s wisdom observeth in the brazen serpent, “He that turned towards it was not healed by the thing he saw, but by thee, O Saviour of all.”

[5.]This is therefore the necessity of sacraments. That saving grace which Christ originally is or hath for the general good of his whole Church, by sacraments he severally deriveth into every member thereof. Sacraments serve as the instruments of God to that end and purpose, moral instruments, the use whereof is in our hands, the effect in his; for the use we have his express commandment, for the effect his conditional promise: so that without our obedience to the one, there is of the other no apparent assurance, as contrariwise where the signs and sacraments of his grace are not either through contempt unreceived, or received with contempt, we are not to doubt but that they really give what they promise, and are what they signify. For we take not baptism nor the eucharist for bare resemblances or memorials of things absent, neither for naked signs and testimonies assuring us of grace received before, but (as they are indeed and in verity) for means effectual whereby God when we take the sacraments delivereth into our hands that grace available unto eternal life, which grace the sacraments represent or signify.

[6.]There have grown in the doctrine concerning sacraments many difficulties for want of distinct explication what kind or degree of grace doth belong unto each sacrament. For by  this it hath come to pass, that the true immediate cause why Baptism, and why the Supper of our Lord is necessary, few do rightly and distinctly consider.
 It cannot be denied but sundry the same effects and benefits which grow unto men by the one sacrament may rightly be attributed unto the other. Yet then doth baptism challenge to itself but the inchoation of those graces, the consummation whereof dependeth on mysteries ensuing. We receive Christ Jesus in baptism once as the first beginner, in the eucharist often as being by continual degrees the finisher of our life. By baptism therefore we receive Christ Jesus, and from him that saving grace which is proper unto baptism. By the other sacrament we receive him also, imparting therein himself and that grace which the eucharist properly bestoweth. So that each sacrament having both that which is general or common, and that also which is peculiar unto itself, we may hereby gather that the participation of Christ which properly belongeth to any one sacrament, is not otherwise to be obtained but by the sacrament whereunto it is proper.


\section*{The substance of Baptism, the rites or solemnities thereunto belonging, and that the substance thereof being kept, other things in Baptism may give place to necessity.}
LVIII. Now even as the soul doth organize the body, and give unto every member thereof that substance, quantity, and shape, which nature seeth most expedient, so the inward grace of sacraments may teach what serveth best for their outward form, a thing in no part of Christian religion, much less here, to be neglected. Grace intended by sacraments was a cause of the choice, and is a reason of the fitness of the elements themselves. Furthermore, seeing that the grace which here we receive doth no way depend upon the natural force of that which we presently behold, it was of necessity that words of express declaration taken from the very mouth of our Lord himself should be added unto visible elements, that the one might infallibly teach what the other do most assuredly bring to pass.

[2.]In writing and speaking of the blessed sacraments we use for the most part under the name of their substance not only to comprise that whereof they outwardly and sensibly consist, but also the secret grace which they signify and  exhibit.
 This is the reason wherefore commonly in definitions, whether they be framed larger to augment, or stricter to abridge the number of sacraments, we find grace expressly mentioned as their true essential form, elements as the matter whereunto that form doth adjoin itself. But if that be separated which is secret, and that considered alone which is seen, as of necessity it must in all those speeches that make distinction of sacraments from sacramental grace, the name of a sacrament in such speeches can imply no more than what the outward substance thereof doth comprehend. And to make complete the outward substance of a sacrament, there is required an outward form, which form sacramental elements receive from sacramental words. Hereupon it groweth, that many times there are three things said to make up the substance of a sacrament, namely, the grace which is thereby offered, the element which shadoweth or signifieth grace, and the word which expresseth what is done by the element. So that whether we consider the outward by itself alone, or both the outward and inward substance of any sacrament; there are in the one respect but two essential parts, and in the other but three that concur to give sacraments their full being.

[3.]Furthermore, because definitions are to express but the  most immediate and nearest parts of nature,
 whereas other principles farther off although not specified in defining, are notwithstanding in nature implied and presupposed, we must note that inasmuch as sacraments are actions religious and mystical, which nature they have not unless they proceed from a serious meaning, and what every man’s private mind is, as we cannot know, so neither are we bound to examine, therefore always in these cases the known intent of the Church generally doth suffice, and where the contrary is not manifest, we may presume that he which outwardly doth the work, hath inwardly the purpose of the Church of God.

[4.]Concerning all other orders, rites, prayers, lessons, sermons, actions, and their circumstances whatsoever, they are to the outward substance of baptism but things accessory, which the wisdom of the Church of Christ is to order according to  the exigence of that which is principal.
 Again, considering that such ordinances have been made to adorn the sacrament, not the sacrament to depend on them; seeing also that they are not of the substance of baptism, and that baptism is far more necessary than any such incident rite or solemnity ordained for the better administration thereof; if the case be such as permitteth not baptism to have the decent complements of baptism, better it were to enjoy the body without his furniture, than to wait for this till the opportunity of that for which we desire it be lost. Which premisses standing, it seemeth to have been no absurd collection, that in cases of necessity which will not suffer delay till baptism be administered with usual solemnities, (to speak the least,) it may be tolerably given without them, rather than any man without it should be suffered to depart this life.


\section*{The ground in Scripture whereupon a necessity of outward Baptism hath been built.}
LIX. They which deny that any such case of necessity can fall, in regard whereof the Church should tolerate baptism, without the decent rites and solemnities thereunto belonging, pretend that such tolerations have risen from a false interpretation which “certain men” have made of the Scripture, grounding a necessity of external baptism upon the words of our Saviour Christ: “Unless a man be born again of water and of the Spirit, he cannot enter into the kingdom of heaven.” For by “water and the Spirit,” we are in that  place to understand (as they imagine) no more than if the Spirit alone had been mentioned and water not spoken of.
 Which they think is plain, because elsewhere it is not improbable that “the Holy Ghost and fire” do but signify the Holy Ghost in operation resembling fire. Whereupon they conclude, that seeing fire in one place may be, therefore water in another place is but a metaphor, Spirit the interpretation thereof, and so the words do only mean, “That unless a man be born again of the Spirit, he cannot enter into the kingdom of heaven.”

[2.]I hold it for a most infallible rule in expositions of sacred Scripture, that where a literal construction will stand, the farthest from the letter is commonly the worst. There is nothing more dangerous than this licentious and deluding art, which changeth the meaning of words, as alchymy doth or would do the substance of metals, maketh of any thing what it listeth, and bringeth in the end all truth to nothing. Or howsoever such voluntary exercise of wit might be borne with otherwise, yet in places which usually serve, as this doth concerning regeneration by water and the Holy Ghost, to be alleged for grounds and principles, less is permitted.

[3.]To hide the general consent of antiquity agreeing in the literal interpretation, they cunningly affirm that “certain” have taken those words as meant of material water, when they know that of all the ancient there is not one to be named that ever did otherwise either expound or allege the place than as implying external baptism. Shall that which hath always received this and no other construction be now disguised with the toy of novelty? Must we needs at the only shew of a critical conceit without any more deliberation, utterly condemn them of error, which will not admit that fire in the words of John is quenched with the name of the Holy Ghost, or with the name of the Spirit, water dried up in the words of Christ?

[4.]When the letter of the law hath two things plainly and expressly specified, Water, and the Spirit; Water as a duty  required on our parts, the Spirit as a gift which God bestoweth; there is danger in presuming so to interpret it, as if the clause which concerneth ourselves were more than needeth.
 We may by such rare expositions attain perhaps in the end to be thought witty, but with ill advice.

[5.]Finally if at the time when that Baptism which was meant by John came to be really and truly performed by Christ himself, we find the Apostles that had been, as we are, before baptized, new baptized with the Holy Ghost, and in this their later baptism as well a visible descent of fire, as a secret miraculous infusion of the Spirit; if on us he accomplish likewise the heavenly work of our new birth not with the Spirit alone but with water thereunto adjoined, sith the faithfullest expounders of his words are his own deeds, let that which his hand hath manifestly wrought declare what his speech did doubtfully utter.


\section*{What kind of necessity in outward Baptism hath been gathered by the words of our Saviour Christ; and what the true necessity thereof indeed is.}
LX. To this they add, that as we err by following a wrong construction of the place before alleged, so our second oversight is, that we thereupon infer a necessity over rigorous and extreme.

The true necessity of baptism a few propositions considered will soon decide. All things which either are known causes or set means, whereby any great good is usually procured, or men delivered from grievous evil, the same we must needs confess necessary. And, if regeneration were not in this very  sense a thing necessary to eternal life, would Christ himself have taught Nicodemus that to see the kingdom of God is impossible, saving only for those men which are born from above?

His words following in the next sentence are a proof sufficient, that to our regeneration his Spirit is no less necessary than regeneration itself necessary unto life.

Thirdly, unless as the Spirit is a necessary inward cause, so Water were a necessary outward mean to our regeneration, what construction should we give unto those words wherein we are said to be new-born, and that ἐξ ὕδατος, even of water? Why are we taught that with water God doth purify and cleanse his Church? Wherefore do the Apostles of Christ term baptism a bath of regeneration? What purpose had they in giving men advice to receive outward baptism, and in persuading them it did avail to remission of sins?

[2.]If outward baptism were a cause in itself possessed of that power either natural or supernatural, without the present operation whereof no such effect could possibly grow, it must then follow, that seeing effects do never prevent the necessary causes out of which they spring, no man could ever receive grace before baptism: which being apparently both known and also confessed to be otherwise in many particulars, although in the rest we make not baptism a cause of grace, yet the grace which is given them with their baptism doth so far forth depend on the very outward sacrament, that God will have it embraced not only as a sign or token what we receive, but also as an instrument or mean whereby we receive grace, because baptism is a sacrament which God hath instituted in his Church, to the end that they which receive the same might thereby be incorporated into Christ, and so through his most precious merit obtain as well that saving grace of imputation which taketh away all former guiltiness,  as also that infused divine virtue of the Holy Ghost, which giveth to the powers of the soul their first disposition towards future newness of life.

[3.]There are that elevate too much the ordinary and immediate means of life, relying wholly upon the bare conceit of that eternal election, which notwithstanding includeth a subordination of means without which we are not actually brought to enjoy what God secretly did intend; and therefore to build upon God’s election if we keep not ourselves to the ways which he hath appointed for men to walk in, is but a self-deceiving vanity. When the Apostle saw men called to the participation of Jesus Christ, after the Gospel of God embraced and the sacrament of life received, he feareth not then to put them in the number of elect saints, he then accounteth them delivered from death, and clean purged from all sin. Till then notwithstanding their pre-ordination unto life which none could know of saving God, what were they in the Apostle’s own account but children of wrath as well as others, plain aliens altogether without hope, strangers utterly without God in this present world? So that by sacraments and other sensible tokens of grace we may boldly gather that he, whose mercy vouchsafeth now to bestow the means, hath also long sithence intended us that whereunto they lead. But let us never think it safe to presume of our own last end by bare conjectural collections of his first intent  and purpose, the means failing that should come between. Predestination bringeth not to life, without the grace of external vocation, wherein our baptism is implied.
 For as we are not naturally men without birth, so neither are we Christian men in the eye of the Church of God but by new birth, nor according to the manifest ordinary course of divine dispensation new-born, but by that baptism which both declareth and maketh us Christians. In which respect we justly hold it to be the door of our actual entrance into God’s house, the first apparent beginning of life, a seal perhaps to the grace of Election, before received, but to our sanctification here a step that hath not any before it.

[4.]There were of the old Valentinian heretics, some which had knowledge in such admiration, that to it they ascribed all, and so despised the sacraments of Christ, pretending that as ignorance had made us subject to all misery, so the full redemption of the inward man, and the work of our restoration, must needs belong unto knowledge only. They draw very near unto this error, who fixing wholly their minds on the known necessity of faith imagine that nothing but faith is necessary for the attainment of all grace. Yet is it a branch of belief that sacraments are in their place no less required than belief itself. For when our Lord and Saviour promiseth eternal life, is it any otherwise than as he promised restitution of health unto Naaman the Syrian, namely with  this condition, “Wash, and be clean?”
 or, as to them which were stung of serpents, health by beholding the brazen serpent? If Christ himself which giveth salvation do require baptism, it is not for us that look for salvation to sound and examine him, whether unbaptized men may be saved, but seriously to do that which is required, and religiously to fear the danger which may grow by the want thereof. Had Christ only declared his will to have all men baptized, and not acquainted us with any cause why baptism is necessary, our ignorance in the reason of that he enjoineth might perhaps have hindered somewhat the forwardness of our obedience thereunto; whereas now being taught that baptism is necessary to take away sin, how have we the fear of God in our hearts if care of delivering men’s souls from sin do not move us to use all means for their baptism? Pelagius which denied utterly the guilt of original sin, and in that respect the necessity of baptism, did notwithstanding both baptize infants, and acknowledge their baptism necessary for “entrance into the kingdom of God.”

[5.]Now the law of Christ which in these considerations maketh baptism necessary, must be construed and understood according to rules of natural equity. Which rules if they themselves did not follow in expounding the law of God, would they ever be able to prove that the Scripture in saying, “Whoso believeth not the Gospel of Christ is condemned already,” “meaneth this sentence of those which can hear the Gospel, and have discretion when they hear to understand it, neither ought it to be applied unto infants, deaf men and fools?” That which teacheth them  thus to interpret the law of Christ is natural equity. And (because equity so teacheth) it is on all parts gladly confessed, that there may be in divers cases life by virtue of inward baptism, even where outward is not found. So that if any question be made, it is but about the bounds and limits of this possibility.

For example, to think that a man whose baptism the crown of martyrdom preventeth, doth lose in that case the happiness which so many thousands enjoy, that only have had the grace to believe, and not the honour to seal the testimony thereof with death, were almost barbarous.

Again, when some certain opinative men in St. Bernard’s time began privately to hold that, because our Lord hath said, “Unless a man be born again of water,” therefore life, without either actual baptism or martyrdom instead of baptism, cannot possibly be obtained at the hands of God: Bernard considering that the same equity which had moved them to think the necessity of baptism no bar against the happy estate of unbaptized martyrs is as forcible for the warrant of their salvation, in whom, although there be not the sufferings of holy martyrs, there are the virtues which sanctified those sufferings and made them precious in God’s sight, professed himself an enemy to that severity and strictness which admitteth no exception but of martyrs only. “For,” saith he,  “if a man desirous of baptism be suddenly cut off by death,
 in whom there wanted neither sound faith, devout hope, nor sincere charity, (God be merciful unto me and pardon me if I err,) but verily of such a one’s salvation in whom there is no other defect besides his faultless lack of baptism, despair I cannot, nor induce my mind to think his faith void, his hope confounded, and his charity fallen to nothing, only because he hath not that which not contempt but impossibility withholdeth.”

“Tell me I beseech you,” saith Ambrose, “what there is in any of us more than to will, and to seek for our own good. Thy servant Valentinian, O Lord, did both.” (For Valentinian the emperor died before his purpose to receive baptism could take effect.) “And is it possible that he which had purposely thy Spirit given him to desire grace, should not receive thy grace which that Spirit did desire? Doth it move you that the outward accustomed solemnities were not done? As though converts that suffer martyrdom before baptism did thereby forfeit their right to the crown of eternal glory in the kingdom of heaven. If the blood of martyrs in that case be their baptism, surely his religious desire of baptism standeth him in the same stead.”

It hath been therefore constantly held as well touching other believers as martyrs, that baptism taken away by necessity, is supplied by desire of baptism, because with equity this opinion doth best stand.

[6.]Touching infants which die unbaptized, sith they  neither have the sacrament itself, nor any sense or conceit thereof, the judgment of many hath gone hard against them. But yet seeing grace is not absolutely tied unto sacraments, and besides such is the lenity of God that unto things altogether impossible he bindeth no man, but where we cannot do what is enjoined us accepteth our will to do instead of the deed itself; again, forasmuch as there is in their Christian parents and in the Church of God a presumed desire that the sacrament of baptism might be given them, yea a purpose also that it shall be given; remorse of equity hath moved divers of the school divines in these considerations ingenuously to grant, that God all-merciful to such as are not in themselves able to desire baptism imputeth the secret desire that others have in their behalf, and accepteth the same as theirs rather than casteth away their souls for that which no man is able to help.

And of the will of God to impart his grace unto infants without baptism, in that case the very circumstance of their natural birth may serve as a just argument, whereupon it is not to be misliked that men in charitable presumption do  gather a great likelihood of their salvation, to whom the benefit of Christian parentage being given, the rest that should follow is prevented by some such casualty as man hath himself no power to avoid.
 For we are plainly taught of God, that the seed of faithful parentage is holy from the very birth. Which albeit we may not so understand, as if the children of believing parents were without sin, or grace from baptized parents derived by propagation, or God by covenant and promise tied to save any in mere regard of their parents’ belief: yet seeing that to all professors of the name of Christ this pre-eminence above infidels is freely given, the fruit of their bodies bringeth into the world with it a present interest and right to those means wherewith the ordinance of Christ is that his Church shall be sanctified, it is not to be thought that he which as it were from heaven hath nominated and designed them unto holiness by special privilege of their very birth, will himself deprive them of regeneration and inward grace, only because necessity depriveth them of outward sacraments. In which case it were the part of charity to hope, and to make men rather partial than cruel judges, if we had not those fair apparencies which here we have.

[7.]Wherefore a necessity there is of receiving, and a necessity of administering, the sacrament of baptism; the one peradventure not so absolute as some have thought, but out of all peradventure the other more strait and narrow, than that the Church which is by office a mother unto such as crave at her hands the sacred mystery of their new birth, should repel them and see them die unsatisfied of these their ghostly desires, rather than give them their soul’s rights with omission of those things that serve but only for the more convenient and orderly administration thereof. For as on the one side we grant that those sentences of holy Scripture which make sacraments most necessary to eternal life are no prejudice to their salvation that want them by some inevitable necessity, and without any fault of their own; so it ought in reason to be likewise acknowledged, that forasmuch as our Lord himself maketh  baptism necessary,
 necessary whether we respect the good received by baptism, or the testimony thereby yielded unto God of that humility and meek obedience, which reposing wholly itself on the absolute authority of his commandment, and on the truth of his heavenly promise, doubteth not but from creatures despicable in their own condition and substance to obtain grace of inestimable value, or rather not from them but from him, yet by them as by his appointed means; howsoever he by the secret ways of his own incomprehensible mercy may be thought to save without baptism, this cleareth not the Church from guiltiness of blood, if through her superfluous scrupulosity lets and impediments of less regard should cause a grace of so great moment to be withheld, wherein our merciless strictness may be our own harm, though not theirs towards whom we shew it; and we for the hardness of our hearts may perish, albeit they through God’s unspeakable mercy do live. God which did not afflict that innocent, whose circumcision Moses had over long deferred, took revenge upon Moses himself for the injury which was done through so great neglect, giving us thereby to understand that they whom God’s own mercy saveth without us are on our parts notwithstanding and as much as in us lieth even destroyed, when under unsufficient pretences we defraud them of such ordinary outward helps as we should exhibit. We have for baptism no day set as the Jews had for circumcision; neither have we by the law of God but only by the Church’s discretion a place thereunto appointed. Baptism therefore even in the meaning of the law of Christ belongeth unto infants capable thereof from the very instant of their birth. Which if they have not howsoever, rather than lose it by being put off because the time, the place, or some such like circumstance doth not solemnly enough concur, the Church as much as in her lieth, wilfully casteth away their souls.


\section*{What things in Baptism have been dispensed with by the fathers respecting necessity.}
LXI. The ancient it may be were too severe, and made the necessity of baptism more absolute than reason would, as touching  infants. But will any man say that they, notwithstanding their too much rigour herein, did not in that respect sustain and tolerate defects of local or of personal solemnities belonging to the sacrament of baptism? The Apostles themselves did neither use nor appoint for baptism any certain time. The Church for general baptism heretofore made choice of two chief days in the year, the feast of Easter, and the feast of Pentecost. Which custom when certain churches in Sicily began to violate without cause, they were by Leo Bishop of Rome advised rather to conform themselves to the rest of the world in things so reasonable, than to offend men’s minds through needless singularity: howbeit always providing that  nevertheless in apparent peril of death, danger of siege, straits of persecution, fear of shipwreck, and the like exigents, no respect of times should cause this singular defence of true safety to be denied unto any.
 This of Leo did but confirm that sentence which Victor had many years before given, extending the same exception as well unto places as times.

[2.]That which St. Augustine speaketh of women hasting to bring their children to the church when they saw danger, is a weak proof that when necessity did not leave them so much time, it was not then permitted them neither to make a church of their own home.

Which answer dischargeth likewise their example of a sick Jew carried in bed to the place of baptism, and not baptized at home in private.

The cause why such kind of baptism barred men afterwards from entering into holy orders, the reason wherefore it was objected against Novatian, in what respect and how far forth it did disable, may be gathered by the twelfth canon set down in the council of Neocæsarea after this manner. “A man which hath been baptized in sickness, is not after to be ordained priest.” For it may be thought, “that such do rather at that time, because they see no other remedy, than of a voluntary mind lay hold on the Christian faith, unless their true and sincere meaning be made afterwards the more manifest, or else the scarcity of others enforce the Church to admit them.”



They bring in Justinian’s imperial constitution, but to what purpose,
 seeing it only forbiddeth men to have the mysteries of God administered in their private chapels, lest under that pretence heretics should do secretly those things which were unlawful? In which consideration he therefore commandeth that if they would use those private oratories otherwise than only for their private prayers, the Bishop should appoint them a clerk whom they might entertain for that purpose. This is plain by later constitutions made in the time of Leo: “It was thought good,” saith the emperor, “in their judgment which have gone before, that in private chapels none should celebrate the holy communion but priests belonging unto greater churches. Which order they took as it seemeth for the custody of religion, lest men should secretly receive from heretics, instead of the food the bane of their souls, pollution in place of expiation.” Again, “Whereas a sacred canon of the sixth reverend synod requireth baptism, as others have likewise the holy sacrifices and mysteries, to be celebrated only in temples hallowed for public use, and not in private oratories; which strict decrees appear to have been made heretofore in regard of heretics, which entered closely into such men’s houses as favoured their opinions, whom under colour of performing with them such religious offices they drew from the soundness of true religion: now that perverse opinions through the grace of Almighty God  are extinct and gone, the cause of former restraints being taken away, we see no reason but that private oratories may henceforward enjoy that liberty which to have granted them heretofore had not been safe.”

In sum, all these things alleged are nothing, nor will it ever be proved while the world doth continue, but that the practice of the Church in cases of extreme necessity hath made for private baptism always more than against it.

[3.]Yea, “Baptism by any man in case of necessity,” was the voice of the whole world heretofore. Neither is Tertullian, Epiphanius, Augustine, or any other of the ancient against it.

The boldness of such as pretending Tecla’s example, took openly upon them both baptism and all other public functions of priesthood, Tertullian severely controlleth, saying, “To give baptism is in truth the bishop’s right. After him it belongeth unto priests and deacons, but not to them without authority from him received. For so the honour of the Church requireth, which being kept, preserveth peace. Were it not in this respect the laity might do the same, all sorts might give even as all sorts receive. But because emulation is the mother of schisms, let it content thee” (which  art of the order of laymen) “to do it in necessity when the state of time or place or person thereunto compelleth. For then is their boldness privileged that help when the circumstance of other men’s dangers craveth it.” What he granteth generally to lay persons of the house of God, the same we cannot suppose he denieth to any sort or sex contained under that name, unless himself did restrain the limits of his own speech, especially seeing that Tertullian’s rule of interpretation is elsewhere, “Specialties are signified under that which is general, because they are therein comprehended.” All which Tertullian doth deny is that women may be called to bear, or publicly take upon them to execute offices of ecclesiastical order, whereof none but men are capable.

As for Epiphanius, he striketh on the very self-same anvil with Tertullian.

And in necessity if St. Augustine allow as much unto laymen as Tertullian doth, his “not mentioning” of women is but a slender proof that his meaning was to exclude women.

Finally, the council of Carthage likewise, although it  make no express submission, may be very well presumed willing to stoop as other positive ordinances do to the countermands of necessity.

[4.]Judge therefore what the ancient would have thought if in their days it had been heard which is published in ours, that because “the substance of the sacrament doth chiefly depend on the institution of God, which is the form and as it were the life of the sacrament,” therefore first, “if the whole institution be not kept, it is no sacrament;” and secondly, if baptism be private his institution is broken, inasmuch as, “according to the orders which he hath set for baptism it should be done in the congregation,” from whose ordinance in this point “we ought not to swerve, although we know that infants should be assuredly damned without baptism.” O sir, you that would spurn thus at such as in case of so dreadful extremity should lie prostrate before your feet, you that would turn away your face from them at the hour of their most need, you that would dam up your ears and harden your heart as iron against the unresistible cries of supplicants calling upon you for mercy with terms of such invocation as that most dreadful perplexity might minister if God by miracle did open the mouths of infants to express their supposed necessity, should first imagine yourself in their case and them in yours. This done, let their supplications proceed out of your mouth, and your answer out of theirs. Would you then contentedly hear, “My son, the rites and solemnities of baptism must be kept, we may not do ill that good may come of it, neither are souls to be delivered from eternal death and condemnation, by breaking orders which Christ hath set;” would you in their case yourself be shaken off with these answers, and not rather embrace enclosed  with both your arms a sentence which now is no Gospel unto you,
 “I will have mercy and not sacrifice?”

[5.]To acknowledge Christ’s institution the ground of both sacraments, I suppose no Christian man will refuse: for it giveth them their very nature, it appointeth the matter whereof they consist, the form of their administration it teacheth, and it blesseth them with that grace whereby to us they are both pledges and instruments of life. Nevertheless seeing Christ’s institution containeth, besides that which maketh complete the essence or nature, other things that only are parts as it were of the furniture of sacraments, the difference between these two must unfold that which the general terms of indefinite speech would confound. If the place appointed for baptism be a part of Christ’s institution, it is but his institution as Sacrifice, baptism his institution as Mercy, in this case. He which requireth both mercy and sacrifice rejecteth his own institution of sacrifice, where the offering of sacrifice would hinder mercy from being shewed. External circumstances even in the holiest and highest actions are but the “lesser things of the law,” whereunto those actions themselves being compared are “the greater;” and therefore as the greater are of such importance that they must be done, so in that extremity before supposed if our account of the lesser which are not to be omitted, should cause omission of that which is more to be accounted of, were not this our strict obedience to Christ’s institution touching “mint and cummin,” a disobedience to his institution concerning love? But sith no institution of Christ hath so strictly tied baptism to public assemblies as it hath done all men unto baptism, away with these merciless and bloody sentences, let them never be found standing in the books and writings of a Christian man, they savour not of Christ nor of his most gracious and meek spirit, but under colour of exact obedience they nourish cruelty and hardness of heart.


\section*{Whether baptism by Women be true Baptism, good and effectual to them that receive it.}
LXII. To leave private baptism therefore and to come unto baptism by women, which they say is no more a  sacrament, than any other ordinary washing or bathing of man’s body; the reason whereupon they ground their opinion herein is such, as making baptism by women void, because women are no ministers in the Church of God, must needs generally annihilate the baptism of all unto whom their conceit shall apply this exception, whether it be in regard of sex, of quality, of insufficiency, or whatsoever.
 For if want of calling do frustrate baptism, they that baptize without calling do nothing, be they women or men.

[2.]To make women teachers in the house of God were a gross absurdity, seeing the Apostle hath said, “I permit not a woman to teach;” and again, “Let your women in churches be silent.” Those extraordinary gifts of speaking with tongues and prophesying, which God at that time did not only bestow upon men, but on women also, made it the harder to hold them confined with private bounds. Whereupon the Apostle’s ordinance was necessary against women’s public admission to teach. And because when law hath begun some one thing or other well, it giveth good occasion either to draw by judicious exposition out of the very law itself, or to annex to the law by authority and jurisdiction things of like conveniency, therefore Clement extendeth this apostolic constitution to baptism. “For,” saith he, “if we have denied  them leave to teach, how should any man dispense with nature and make them ministers of holy things, seeing this unskilfulness is a part of the Grecians’ impiety, which for the service of women goddesses have women priests?”

I somewhat marvel that men which would not willingly be thought to speak or write but with good conscience, dare hereupon openly avouch Clement for a witness, “That as when the Church began not only to decline but to fall away from the sincerity of religion it borrowed a number of other profanations of the heathens, so it borrowed this, and would needs have women priests as the heathens had, and that this was one occasion of bringing baptism by women into the Church of God.” Is it not plain in their own eyes that first by an evidence which forbiddeth women to be ministers of baptism, they endeavour to shew how women were admitted unto that function in the wane and declination of Christian piety; secondly, that by an evidence rejecting the heathens, and condemning them of impiety, they would prove such affection towards heathens as ordereth the affairs of the Church by the pattern of their example; and thirdly, that out of an evidence which nameth the heathens as being in some part a reason why the Church had no women priests, they gather the heathens to have been one of the first occasions why it had? So that throughout every branch of this testimony their issue is yea, and their evidence directly no.

[3.]But to women’s baptism in private by occasion of urgent necessity, the reasons that only concern ordinary baptism in public are no just prejudice, neither can we by force thereof disprove the practice of those churches which (necessity requiring) allow baptism in private to be administered by women. We may not from laws that prohibit any thing with restraint conclude absolute and unlimited prohibitions. Although we deny not but they which utterly forbid such baptism may have perhaps wherewith to justify their orders against it. For even things lawful are well prohibited,  when there is fear lest they make the way to unlawful more easy.
 And it may be the liberty of baptism by women at such times doth sometimes embolden the rasher sort to do it where no such necessity is.

[4.]But whether of permission besides law, or in presumption against law they do it, is it thereby altogether frustrate, void, and as though it were never given?

They which have not at the first their right baptism must of necessity be rebaptized, because the law of Christ tieth all men to receive baptism. Iteration of baptism once given hath been always thought a manifest contempt of that ancient apostolic aphorism, “One Lord, one Faith, one Baptism,” baptism not only one inasmuch as it hath every where the same substance and offereth unto all men the same grace, but one also for that it ought not to be received by any one man above once. We serve that Lord which is but one, because no other can be joined with him: we embrace that Faith which is but one, because it admitteth no innovation: that Baptism we receive which is but one, because it cannot be  received often.
 For how should we practise iteration of baptism, and yet teach that we are by baptism born anew, that by baptism we are admitted into the heavenly society of saints, that those things be really and effectually done by baptism which are no more possible to be often done than a man can naturally be often born, or civilly be often adopted into any one’s stock and family? This also is the cause why they that present us unto baptism are entitled for ever after our parents in God, and the reason why there we receive new names in token that by baptism we are made new creatures. As Christ hath therefore died and risen from the dead but once, so the sacrament which both extinguisheth in him our former sin and beginneth in us a new condition of life, is by one only actual administration for ever available, according to that in the Nicene Creed, “I believe one baptism for remission of sins.”

[5.]And because second baptism was ever abhorred in the Church of God as a kind of incestuous birth, they that iterate baptism are driven under some pretence or other to make the former baptism void. Tertullian the first that proposed to the Church, Agrippinus the first in the Church  that accepted,
 and against the use of the Church Novatian the first that publicly began to practise rebaptization, did it therefore upon these two grounds, a true persuasion that baptism is necessary, and a false that the baptism which others administered was no baptism. Novatianus his conceit was that none can administer true baptism but the true Church of Jesus Christ, that he and his followers alone were the Church, and for the rest he accounted them wicked and profane persons, such as by baptism could cleanse no man, unless they first did purify themselves, and reform the faults wherewith he charged them. At which time St. Cyprian with the greatest part of African bishops, because they likewise thought that none but only the true Church of God can baptize, and were of nothing more certainly persuaded than that heretics are as rotten branches cut off from the life and body of the true Church, gathered hereby that the Church of God both may with good consideration and ought to reverse that baptism which is given by heretics. These held and practised their own opinion, yet with great protestations often made that they neither loved a whit the less, nor thought in any respect the worse of them that were of a contrary mind. In requital of which ingenuous moderation the rest that withstood them did it in peaceable sort with very good regard had of them as of men in error but not in heresy.

[6.]The bishop of Rome against their novelties upheld as beseemed him the ancient and true apostolic customs, till they which unadvisedly before had erred became in a manner all reconciled friends unto truth, and saw that heresy in the ministers of baptism could no way evacuate the force thereof; such heresy alone excepted, as by reason of  unsoundness in the highest articles of Christian faith, presumed to change, and by changing to maim the substance, the form of baptism. In which respect the Church did neither simply disannul, nor absolutely ratify baptism by heretics. For the baptism which Novatianists gave stood firm, whereas they whom Samosatenians had baptized were rebaptized. It was likewise ordered in the council of Arles, that if any Arian did reconcile himself to the Church, they should admit him without new baptism, unless by examination they found him not baptized in the name of the Trinity.

Dionysius bishop of Alexandria maketh report how there lived under him a man of good reputation and of very ancient continuance in that church, who being present at the rites of baptism, and observing with better consideration than ever before what was there done, came and with weeping submission craved of his bishop not to deny him baptism, the  due of all which profess Christ,
 seeing it had been so long sithence his evil hap to be deceived by the fraud of heretics, and at their hands (which till now he never throughly and duly weighed) to take a baptism full fraught with blasphemous impieties, a baptism in nothing like unto that which the true Church of Christ useth. The bishop greatly moved thereat, yet durst not adventure to rebaptize, but did the best he could to put him in good comfort, using much persuasion with him not to trouble himself with things which were past and gone, nor after so long continuance in the fellowship of God’s people to call now in question his first entrance. The poor man that saw himself in this sort answered but not satisfied, spent afterwards his life in continual perplexity, whereof the bishop remained fearful to give release: perhaps too fearful, if the baptism were such as his own declaration importeth. For that, the substance whereof was rotten at the very first, is never by tract of time able to recover soundness. And where true baptism was not before given, the case of rebaptization is clear.

[7.]But by this it appeareth that baptism is not void in regard of heresy, and therefore much less through any other moral defect in the minister thereof. Under which second pretence Donatists notwithstanding took upon them to make frustrate the Church’s baptism, and themselves to rebaptize their own fry. For whereas some forty years after the martyrdom of blessed Cyprian the emperor Diocletian began to persecute the Church of Christ, and for the speedier abolishment of their religion to burn up their sacred books, there were in the Church itself Traditors content to deliver up the books of God by composition, to the end their own lives might be spared. Which men growing thereby odious to the rest whose constancy was greater, it fortuned that after, when one Cæcilian was ordained bishop in the church of Carthage, whom others endeavoured in vain to defeat by excepting against him as a Traditor, they whose accusations could not prevail, desperately joined themselves in one, and made a bishop of their own crew, accounting from that day forward their faction the only true and sincere Church. The  first bishop on that part was Majorinus, whose successor Donatus being the first that wrote in defence of their schism, the birds that were hatched before by others have their names from him.

[8.]Arians and Donatists began both about one time. Which heresies according to the different strength of their own sinews, wrought as hope of success led them, the one with the choicest wits, the other with the multitude so far, that after long and troublesome experience the perfectest view men could take of both was hardly able to induce any certain determinate resolution, whether error may do more by the curious subtlety of sharp discourse, or else by the mere appearance of zeal and devout affection, the later of which two aids gave Donatists beyond all men’s expectation as great a sway as ever any schism or heresy had within that reach of the Christian world where it bred and grew: the rather perhaps because the Church which neither greatly feared them, and besides had necessary cause to bend itself against others that aimed directly at a far higher mark, the Deity of Christ, was contented to let Donatists have their forth by the space of threescore years and above, even from ten years before Constantine till the time that Optatus bishop of Milevis published his books against Parmenian.

During which term and the space of that schism’s continuance afterwards, they had, besides many other secular and worldly means to help them forward, these special advantages. First, the very occasion of their breach with the Church of God, a just hatred and dislike of Traditors, seemed plausible; they easily persuaded their hearers that such men could not be holy as held communion and fellowship with them that betray religion. Again, when to dazzle the eyes of the simple, and to prove that it can be no church which is not holy, they had in show and sound of words the glorious pretence of the creed apostolic, “I believe the Holy Catholic Church,” we need not think it any strange thing that with the multitude they gained credit. And avouching that such as are not of the true Church can administer no true baptism, they had for this point whole volumes of St.  Cyprian’s own writing, together with the judgment of divers African synods whose sentence was the same with his.
 Whereupon the Fathers were likewise in defence of their just cause very greatly prejudiced, both for that they could not enforce the duty of men’s communion with a church confessed to be in many things blameworthy, unless they should oftentimes seem to speak as half-defenders of the faults themselves, or at the least not so vehement accusers thereof as their adversaries; and to withstand iteration of baptism, the other branch of the Donatists’ heresy, was impossible without manifest and professed rejection of Cyprian, whom the world universally did in his lifetime admire as the greatest amongst prelates, and now honour as not the lowest in the kingdom of heaven. So true we find it by experience of all ages in the Church of God, that the teacher’s error is the people’s trial, harder and heavier by so much to bear, as he is in worth and regard greater that mispersuadeth them. Although there was odds between Cyprian’s cause and theirs, he differing from others of sounder understanding in that point, but not dividing himself from the body of the Church by schism as did the Donatists. For which cause, saith Vincentius, “Of one and the same opinion we judge (which may seem strange) the authors catholic, and the followers heretical; we acquit the masters, and condemn the scholars; they are heirs of heaven which have written those books, the defenders whereof are trodden down to the pit of hell.”

[10.]The invectives of catholic writers therefore against them are sharp; the words of imperial edicts by Honorius and Theodosius made to bridle them very bitter, the punishments  severe in revenge of their folly.
 Howbeit for fear (as we may conjecture) lest much should be derogated from the baptism of the Church, and baptism by Donatists be more esteemed of than was meet, if on the one side that which heretics had done ill should stand as good, on the other side that be reversed which the Catholic Church had well and religiously done, divers better minded than advised men thought it fittest to meet with this inconvenience by rebaptizing Donatists as well as they rebaptized Catholics. For stay whereof the same emperors saw it meet to give their law a double edge, whereby it might equally on both sides cut off not only heretics which rebaptized whom they could pervert, but also Catholic and Christian priests which did the like unto such as before had taken baptism at the hands of heretics, and were afterwards reconciled to the Church of God. Donatists were therefore in process of time, though with much ado, wearied and at the length worn out by the constancy of that truth which teacheth, that evil ministers of good things are as torches, a light to others, a waste to none but themselves only, and that the foulness of their hands can neither any whit impair the virtue nor stain the glory of the mysteries of Christ.

[11.]Now that which was done amiss by virtuous and good men, as Cyprian carried aside with hatred against heresy, and was secondly followed by Donatists, whom envy and rancour covered with show of godliness made obstinate to cancel whatsoever the Church did in the sacrament of baptism, hath of later days in another respect far different from both the former, been brought freshly again into practice. For the Anabaptist rebaptizeth, because in his estimation the baptism of the Church is frustrate, for that we give it unto infants which have not faith, whereas according unto Christ’s institution, as they conceive it, true baptism should always presuppose actual belief in receivers, and is otherwise no baptism.

[12.]Of these three errors there is not any but hath been  able at the least to allege in defence of itself many fair probabilities.
 Notwithstanding, sith the Church of God hath hitherto always constantly maintained, that to rebaptize them which are known to have received true baptism is unlawful; that if baptism seriously be administered in the same element and with the same form of words which Christ’s institution teacheth, there is no other defect in the world that can make it frustrate, or deprive it of the nature of a true sacrament; and lastly, that baptism is only then to be readministered, when the first delivery thereof is void in regard of the fore-alleged imperfections and no other; shall we now in the case of baptism, which having both for matter and form the substance of Christ’s institution, is by a fourth sort of men voided for the only defect of ecclesiastical authority in the minister, think it enough that they blow away the force thereof with the bare strength of their very breath by saying, “We take such baptism to be no more the Sacrament of Baptism, than any other ordinary bathing to be a sacrament?”

[13.]It behoveth generally all sorts of men to keep themselves within the limits of their own vocation. And seeing God from whom men’s several degrees and pre-eminences do proceed, hath appointed them in his Church, at whose hands his pleasure is that we should receive both baptism and all other public medicinable helps of soul, perhaps thereby the more to settle our hearts in the love of our ghostly superiors, they have small cause to hope that with him their voluntary services will be accepted who thrust themselves into functions either above their capacity or besides their place, and over-boldly intermeddle with duties whereof no charge was ever given them. They that in any thing exceed the compass of their own order do as much as in them lieth to dissolve that order which is the harmony of God’s Church.

Suppose therefore that in these and the like considerations the law did utterly prohibit baptism to be administered by any other than persons thereunto solemnly consecrated, what necessity soever happen. Are not many things firm being  done, although in part done otherwise than positive rigour and strictness did require? Nature as much as is possible inclineth unto validities and preservations. Dissolutions and nullities of things done, are not only not favoured, but hated when either urged without cause, or extended beyond their reach.

If therefore at any time it come to pass, that in teaching publicly, or privately in delivering this blessed Sacrament of regeneration, some unsanctified hand contrary to Christ’s supposed ordinance do intrude itself, to execute that whereunto the laws of God and his Church have deputed others, which of these two opinions seemeth more agreeable with equity, ours that disallow what is done amiss, yet make not the force of the word and sacraments, much less their nature and very substance to depend on the minister’s authority and calling, or else theirs which defeat, disannul, and annihilate both, in respect of that one only personal defect, there being not any law of God which saith that if the minister be incompetent his word shall be no word, his baptism no baptism? He which teacheth and is not sent loseth the reward, but yet retaineth the name of a teacher; his usurped actions have in him the same nature which they have in others, although they yield him not the same comfort. And if these two cases be peers, the case of doctrine and the case of baptism both alike, sith no defect in their vocation that teach the truth is able to take away the benefit thereof from  him which heareth, wherefore should the want of a lawful calling in them that baptize make baptism to me vain?

[14.]They grant that the matter and the form in sacraments are the only parts of substance, and that if these two be retained, albeit other things besides be used which are inconvenient, the sacrament notwithstanding is administered but not sincerely. Why persist they not in this opinion? When by these fair speeches they have put us in hope of agreement, wherefore sup they up their words again, interlacing such frivolous interpretations and glosses as disgrace their sentence? What should move them, having named the matter and the form of the sacrament, to give us presently warning, that they mean by the form of the sacrament the institution, which exposition darkeneth whatsoever was before plain? For whereas in common understanding that form, which added to the element doth make a sacrament, and is of the outward substance thereof, containeth only the words of usual application, they set it down (lest common dictionaries should deceive us) that the form doth signify in their language the institution, which institution in truth comprehendeth both form and matter. Such are the fumbling shifts to enclose the minister’s vocation within the compass of some essential part of the sacrament.

A thing that can never stand with sound and sincere construction. For what if the minister be “no circumstance but a subordinate efficient cause” in the work of baptism? What if the minister’s vocation be a matter “of perpetual necessity and not a ceremony variable as times and occasions require?” What if his calling be “a principal part of the institution of Christ?” Doth it therefore follow that the minister’s authority is “of the substance of the sacrament,” and as incident into the nature thereof as  the matter and the form itself, yea more incident?
 For whereas in case of necessity the greatest amongst them professeth the change of the element of water, lawful, and others which like not so well this opinion could be better content that voluntarily the words of Christ’s institution were altered, and men baptized in the name of Christ without either mention made of the Father or of the Holy Ghost, nevertheless in denying that baptism administered by private persons ought to be reckoned of as a sacrament they both agree.

[15.]It may therefore please them both to consider that Baptism is an action in part moral, in part ecclesiastical, and in part mystical: moral, as being a duty which men perform towards God; ecclesiastical, in that it belongeth unto God’s Church as a public duty; finally mystical, if we respect what God doth thereby intend to work.

The greatest moral perfection of baptism consisteth in men’s devout obedience to the law of God, which law requireth both the outward act or thing done, and also that religious affection which God doth so much regard, that without it whatsoever we do is hateful in his sight, who therefore is said to respect adverbs more than verbs, because the end of his  law in appointing what we shall do is our own perfection, which perfection consisteth chiefly in the virtuous disposition of the mind, and approveth itself to him not by doing but by doing well. Wherein appeareth also the difference between human and divine laws, the one of which two are content with opus operatum, the other require opus operantis, the one do but claim the deed, the other especially the mind. So that according to laws which principally respect the heart of men, works of religion being not religiously performed, cannot morally be perfect.

Baptism as an ecclesiastical work is for the manner of performance ordered by divers ecclesiastical laws, providing that as the sacrament itself is a gift of no mean worth, so the ministry thereof might in all circumstances appear to be a function of no small regard.

All that belongeth to the mystical perfection of baptism outwardly, is the element, the word, and the serious application of both unto him which receiveth both; whereunto if we add that secret reference which this action hath to life and remission of sins by virtue of Christ’s own compact solemnly made with his Church, to accomplish fully the Sacrament of Baptism, there is not any thing more required.

Now put the question whether baptism administered to infants without any spiritual calling be unto them both a true sacrament and an effectual instrument of grace, or else an act of no more account than the ordinary washings are? The sum of all that can be said to defeat such baptism is, that those things which have no being can work nothing, and that baptism without the power of ordination is as judgment without sufficient jurisdiction, void, frustrate, and of no effect. But to this we answer, that the fruit of baptism dependeth  only upon the covenant which God hath made;
 that God by covenant requireth in the elder sort Faith and Baptism, in children the Sacrament of Baptism alone, whereunto he hath also given them right by special privilege of birth within the bosom of the holy Church; that infants therefore, which have received baptism complete as touching the mystical perfection thereof, are by virtue of his own covenant and promise cleansed from all sin, forasmuch as all other laws concerning that which in baptism is either moral or ecclesiastical do bind the Church which giveth baptism, and not the infant which receiveth it of the Church. So that if any thing be therein amiss, the harm which groweth by violation of holy ordinances must altogether rest where the bonds of such ordinances hold.

[16.]For that in actions of this nature it fareth not as in jurisdictions may somewhat appear by the very opinion which men have of them. The nullity of that which a judge doth by way of authority without authority, is known to all men, and agreed upon with full consent of the whole world, every man receiveth it as a general edict of nature; whereas the nullity of baptism in regard of the like defect is only a few men’s new, ungrounded, and as yet unapproved imagination. Which difference of generality in men’s persuasions on the one side, and their paucity whose conceit leadeth them the other way, hath risen from a difference easy to observe in the things themselves. The exercise of unauthorized jurisdiction is a grievance unto them that are under it, whereas they that without authority presume to baptize, offer nothing but that which to all men is good and acceptable. Sacraments are food, and the ministers thereof as parents or as nurses, at whose hands when there is necessity but no possibility of receiving it, if that which they are not present to do in right of their office be of pity and compassion done by others, shall this be thought to turn celestial bread into gravel, or the medicine of souls into poison? Jurisdiction is a yoke which law hath imposed on the necks of men in such sort that they must endure it for the good of others, how contrary soever it be to their own particular appetites and inclinations; jurisdiction bridleth men against their wills; that which a judge doth prevaileth by virtue of his very power, and therefore not without great reason, except the law have given him authority,  whatsoever he doth vanisheth.
 Baptism on the other side being a favour which it pleaseth God to bestow, a benefit of soul to us that receive it, and a grace which they that deliver are but as mere vessels either appointed by others or offered of their own accord to this service; of which two if they be the one it is but their own honour, their own offence to be the other; can it possibly stand with equity and right, that the faultiness of their presumption in giving baptism should be able to prejudice us, who by taking baptism have no way offended?

[17.]I know there are many sentences found in the books and writings of the ancient Fathers to prove both ecclesiastical and also moral defects in the minister of baptism a bar to the heavenly benefit thereof. Which sentences we always so understand, as Augustine understood in a case of like nature the words of Cyprian. When infants baptized were after their parents’ revolt carried by them in arms to the stews of idols, those wretched creatures as St. Cyprian thought were not only their own ruin but their children’s also; “Their children,” whom this their apostasy profaned, “did lose what Christian baptism had given them being newly born.” “They lost,” saith St. Augustine, “the grace of baptism, if we consider to what their parents’ impiety did tend; although the mercy of God preserved them, and will also in that dreadful day of account give them favourable audience pleading in their own behalf, ‘The harm of other men’s perfidiousness it lay not in us to avoid.’ ” After the same manner whatsoever we  read written if it sound to the prejudice of baptism through any either moral or ecclesiastical defect therein,
 we construe it, as equity and reason teacheth, with restraint to the offender only, which doth, as far as concerneth himself and them which wittingly concur with him, make the sacrament of God fruitless.

[18.]St. Augustine’s doubtfulness, whether baptism by a layman may stand or ought to be readministered, should not be mentioned by them which presume to define peremptorily of that wherein he was content to profess himself unresolved. Albeit in very truth his opinion is plain enough, but the manner of delivering his judgment being modest, they make of a virtue an imbecility, and impute his calmness of speech to an irresolution of mind. His disputation in that place is against Parmenian, which held, that a Bishop or a Priest if they fall into any heresy do thereby lose the power which they had before to baptize, and that therefore baptism by heretics is merely void. For answer whereof he first denieth that heresy can more deprive men of power to baptize others than it is of force to take from them their own baptism; and in the second place he farther addeth that if heretics did lose the power which before was given them by ordination, and did therefore unlawfully usurp as often as they took upon them to give the Sacrament of Baptism, it followeth not that baptism by them administered without authority is no baptism. For then what should we think of baptism by laymen to whom authority was never given? “I doubt,” saith St. Augustine, “whether  any man which carrieth a virtuous and godly mind will affirm that the baptism which laymen do in case of necessity administer should be iterated.
 For to do it unnecessarily is to execute another man’s office; necessity urging, to do it is then either no fault at all” (much less so grievous a crime that it should deserve to be termed by the name of sacrilege) “or if any, a very pardonable fault. But suppose it even of very purpose usurped and given unto any man by every man that listeth, yet that which is given cannot possibly be denied to have been given, how truly soever we may say it hath not been given lawfully. Unlawful usurpation a penitent affection must redress. If not, the thing that was given shall remain to the hurt and detriment of him which unlawfully either administered or received the same, yet so, that in this respect it ought not to be reputed as if it had not at all been given.” Whereby we may plainly perceive that St. Augustine was not himself uncertain what to think, but doubtful whether any well-minded man in the whole world could think otherwise than he did.

[19.]Their argument taken from a stolen seal may return to the place out of which they had it, for it helpeth their cause nothing. That which men give or grant to others must appear to have proceeded of their own accord. This being manifest, their gifts and grants are thereby made effectual both to bar themselves from revocation, and to assecure the right they  have given.
 Wherein for further prevention of mischiefs that otherwise might grow by the malice, treachery, and fraud of men, it is both equal and meet that the strength of men’s deeds and the instruments which declare the same should strictly depend upon divers solemnities, whereof there cannot be the like reason in things that pass between God and us; because sith we need not doubt lest the treasures of his heavenly grace should without his consent be passed by forged conveyances, nor lest he should deny at any time his own acts, and seek to revoke what hath been consented unto before, as there is no such fear of danger through deceit and falsehood in this case, so neither hath the circumstance of men’s persons that weight in baptism which for good and just considerations in the custody of seals of office it ought to have. The grace of baptism cometh by donation from God alone. That God hath committed the ministry of baptism unto special men, it is for order’s sake in his Church, and not to the end that their authority might give being, or add force to the sacrament itself. That infants have right to the sacrament of baptism we all acknowledge. Charge them we cannot as guileful and wrongful possessors of that whereunto they have right by the manifest will of the donor, and are not parties unto any defect or disorder in the manner of receiving the same. And if any such disorder be, we have sufficiently before declared that delictum cum capite semper ambulat, men’s own faults are their own harms.

[20.]Wherefore to countervail this and the like mischosen resemblances with that which more truly and plainly agreeth; the ordinance of God concerning their vocation that minister baptism wherein the mystery of our regeneration is wrought, hath thereunto the same analogy which laws of wedlock have to our first nativity and birth. So that if nature do effect procreation notwithstanding the wicked violation and breach even of nature’s law, made that the entrance of all mankind into this present world might be without blemish, may we not justly presume that grace doth accomplish the other, although there be faultiness in them that transgress the order which our Lord Jesus Christ hath established in his Church?




[21.]Some light may be borrowed from circumcision for explication what is true in this question of baptism. Seeing then that even they which condemn Sephora the wife of Moses for taking upon her to circumcise her son, a thing necessary at that time for her to do, and as I think very hard to reprove in her, considering how Moses, because himself had not done it sooner, was therefore stricken by the hand of God, neither could in that extremity perform the office; whereupon, for the stay of God’s indignation, there was no choice, but the action must needs fall into her hands; whose fact therein whether we interpret as some have done, that being a Midianite, and as yet not so throughly acquainted with the exercise of Jewish rites, it much discontented her, to see herself through her husband’s oversight, in a matter of his own religion, brought unto these perplexities and straits, that either she must now endure him perishing before her eyes, or else wound the flesh of her own child, which she could not do but with some indignation shewed, in that she fumingly both threw down the foreskin at his feet, and upbraided him with the cruelty of his religion: or if we better like to follow their more judicious  exposition which are not inclinable to think that Moses was matched like Socrates, nor that circumcision could now in Eleazar be strange unto her, having had Gersom her elder son before circumcised, nor that any occasion of choler could rise from a spectacle of such misery as doth naturally move compassion and not wrath, nor that Sephora was so impious as in the visible presence of God’s deserved anger to storm at the ordinance and law of God, nor that the words of the history itself can enforce any such affection, but do only declare how after the act performed she touched the feet of Moses saying, “Sponsus tu mihi es sanguinum,” “Thou art unto me an husband of blood,” which might be very well the one done and the other spoken even out of the flowing abundance of commiseration and love, to signify with hands laid under his feet that her tender affection towards him had caused her thus to forget womanhood, to lay all motherly affection aside, and to redeem her husband out of the hands of death with effusion of blood; the sequel thereof, take it which way you will, is a plain argument, that God was satisfied with that she did, as may appear by his own testimony declaring how there followed in the person of Moses present release of  his grievous punishment upon her speedy discharge of that duty which by him neglected had offended God,
 even as after execution of justice by the hands of Phinees the plague was immediately taken away, which former impunity of sin had caused; in which so manifest and plain cases not to make that a reason of the event which God himself hath set down as a reason, were falsely to accuse whom he doth justify, and without any cause to traduce what we should allow; yet seeing they which will have it a breach of the law of God for her to circumcise in that necessity, are not able to deny but circumcision being in that very manner performed was to the innocent child which received it true circumcision, why should that defect whereby circumcision was so little weakened be to baptism a deadly wound?

[22.]These premisses therefore remaining as hitherto they have been laid, because the commandment of our Saviour Christ, which committeth jointly to public ministers both doctrine and baptism, doth no more by linking them together import that the nature of the sacrament dependeth on the minister’s authority and power to preach the word than the force and virtue of the word doth on license to give the sacrament; and considering that the work of external ministry in baptism is only a preeminence of honour, which they that take to themselves and are not thereunto called as Aaron was, do but themselves in their own persons by means of such usurpation incur the just blame of disobedience to the law of God; farther also inasmuch as it standeth with no  reason that errors grounded on a wrong interpretation of other men’s deeds should make frustrate whatsoever is misconceived,
 and that baptism by women should cease to be baptism as oft as any man will thereby gather that children which die unbaptized are damned, which opinion if the act of baptism administered in such manner did enforce, it might be sufficient cause of disliking the same, but none of defeating or making it altogether void; last of all whereas general and full consent of the godly learned in all ages doth make for validity of baptism, yea albeit administered in private and even by women, which kind of baptism in case of necessity divers reformed churches do both allow and defend, some others which do not defend tolerate, few in comparison and they without any just cause do utterly disannul and annihilate; surely howsoever through defects on either side the sacrament may be without fruit, as well in some cases to him which receiveth as to him which giveth it, yet no disability of either part can so far make it frustrate and without effect as to deprive it of the very nature of true baptism, having all things else which the ordinance of Christ requireth. Whereupon we may consequently infer that the administration of this sacrament by private persons, be it lawful or unlawful, appeareth not as yet to be merely void.


\section*{Of Interrogatories in Baptism touching faith and the purpose of a Christian life.}
LXIII. All that are of the race of Christ, the Scripture nameth them “children of the promise” which God hath made. The promise of eternal life is the seed of the Church of God. And because there is no attainment of life but through the only begotten Son of God, nor by him otherwise than being such as the Creed apostolic describeth, it followeth that the articles thereof are principles necessary for all men to subscribe unto, whom by baptism the Church receiveth into Christ’s school.

All points of Christian doctrine are either demonstrable conclusions or demonstrative principles. Conclusions have strong and invincible proofs as well in the school of Jesus Christ as elsewhere. And principles be grounds which require no proof in any kind of science, because it sufficeth if either their certainty be evident in itself, or evident by the light of some higher knowledge, and in itself such as no  man’s knowledge is ever able to overthrow. Now the principles whereupon we do build our souls have their evidence where they had their original, and as received from thence we adore them, we hold them in reverent admiration, we neither argue nor dispute about them, we give unto them that assent which the oracles of God require.

We are not therefore ashamed of the Gospel of our Lord Jesus Christ because miscreants in scorn have upbraided us, that the highest point of our wisdom is Believe. That which is true and neither can be discerned by sense, nor concluded by mere natural principles, must have principles of revealed truth whereupon to build itself, and an habit of faith in us wherewith principles of that kind are apprehended. The mysteries of our religion are above the reach of our understanding, above discourse of man’s reason, above all that any creature can comprehend. Therefore the first thing required of him which standeth for admission into Christ’s family is belief. Which belief consisteth not so much in knowledge as in acknowledgment of all things that heavenly wisdom revealeth; the affection of faith is above her reach, her love to Godward above the comprehension which she hath of God.

And because only for believers all things may be done, he which is goodness itself loveth them above all. Deserve we then the love of God, because we believe in the Son of God? What more opposite than faith and pride? When God had created all things, he looked upon them and loved them, because they were all as himself had made them. So the true reason wherefore Christ doth love believers is because their belief is the gift of God, a gift than which flesh and blood in this world cannot possibly receive a greater. And as to love them of whom we receive good things is duty, because they satisfy our desires in that which else we should want; so to love them on whom we bestow is nature, because in them we behold the effects of our own virtue.

Seeing therefore no religion enjoyeth sacraments the signs of God’s love, unless it have also that faith whereupon the  sacraments are built;
 could there be any thing more convenient than that our first admittance to the actual receipt of his grace in the Sacrament of baptism should be consecrated with profession of belief, which is to the kingdom of God as a key, the want whereof excludeth infidels both from that and from all other saving grace.

[2.]We find by experience that although faith be an intellectual habit of the mind, and have her seat in the understanding, yet an evil moral disposition obstinately wedded to the love of darkness dampeth the very light of heavenly illumination, and permitteth not the mind to see what doth shine before it. Men are “lovers of pleasure more than lovers of God.” Their assent to his saving truth is many times withheld from it, not that the truth is too weak to persuade, but because the stream of corrupt affection carrieth them a clean contrary way. That the mind therefore may abide in the light of faith, there must abide in the will as constant a resolution to have no fellowship at all with the vanities and works of darkness.

[3.]“Two covenants there are which Christian men,” saith Isidore, “do make in baptism, the one concerning relinquishment of Satan, the other touching obedience to the faith of Christ.” In like sort St. Ambrose, “He which is baptized forsaketh the intellectual Pharao, the Prince of this world, saying, Abrenuncio, Thee O Satan and thy angels, thy works and thy mandates I forsake utterly.” Tertullian having speech of wicked spirits, “These,” saith he, “are the angels which we in baptism renounce.” The declaration of Justin the Martyr concerning baptism sheweth,  how such as the Church in those days did baptize made profession of Christian belief, and undertook to live accordingly.
 Neither do I think it a matter easy for any man to prove, that ever baptism did use to be administered without interrogatories of these two kinds. Whereunto St. Peter (as it may be thought) alluding, hath said, that the baptism “which saveth” us is not (as legal purifications were) a cleansing of the flesh from outward impurity, but ἐπερώτημα, “an interrogative trial of a good conscience towards God.”


\section*{Interrogatories proposed unto infants in Baptism, and answered as in their names by godfathers.}
LXIV. Now the fault which they find with us concerning interrogatories is, our moving of these questions unto infants which cannot answer them, and the answering of them by others as in their names.

The Anabaptist hath many pretences to scorn at the baptism of children, first because the Scriptures, he saith, do no where give commandment to baptize infants; secondly, for that as there is no commandment so neither any manifest example shewing it to have been done either by Christ or his Apostles; thirdly, inasmuch as the word preached and the sacraments must go together, they which are not capable of the one are no fit receivers of the other; last of all, sith the order of baptism continued from the first beginning hath in it those things which are unfit to be applied unto sucking children, it followeth in their conceit that the baptism of such is no baptism but plain mockery.

They with whom we contend are no enemies to the baptism of infants; it is not their desire that the church should hazard so many souls by letting them run on till they come to ripeness of understanding, that so they may be converted and then baptized as infidels heretofore have been; they bear not towards God so unthankful minds as not to acknowledge it even amongst the greatest of his endless mercies, that by making us his own possession so soon, many advantages which Satan otherwise might take are prevented, and (which should be esteemed a part of no small happiness) the first  thing whereof we have occasion to take notice is, how much hath been done already to our great good, though altogether without our knowledge;
 the baptism of infants they esteem as an ordinance which Christ hath instituted even in special love and favour to his own people; they deny not the practice thereof accordingly to have been kept as derived from the hands and continued from the days of the Apostles themselves unto this present. Only it pleaseth them not that to infants there should be interrogatories proposed in baptism. This they condemn as foolish, toyish, and profane mockery.

[2.]But are they able to shew that ever the Church of Christ had any public form of baptism without interrogatories; or that the Church did ever use at the solemn baptism of infants to omit those questions as needless in this case? Boniface a bishop in St. Augustine’s time knowing that the Church did universally use this custom of baptizing infants with interrogatories, was desirous to learn from St. Augustine the true cause and reason thereof. “If,” saith he, “I should set before thee a young infant, and should ask of  thee whether that infant when he cometh unto riper age will be honest and just or no, thou wouldst answer (I know) that to tell in these things what shall come to pass is not in the power of a mortal man. If I should ask what good or evil such an infant thinketh, thine answer hereunto must needs be again with the like uncertainty. If thou neither canst promise for the time to come nor for the present pronounce any thing in this case, how is it that when such are brought unto baptism, their parents there undertake what the child shall afterwards do, yea they are not doubtful to say it doth that which is impossible to be done by infants? at the least there is no man precisely able to affirm it done. Vouchsafe me hereunto some short answer, such as not only may press me with the bare authority of custom but also instruct me in the cause thereof.”

Touching which difficulty, whether it may truly be said for infants at the time of their baptism that they do believe, the effect of St. Augustine’s answer is yea, but with this distinction, a present actual habit of faith there is not in them;  there is delivered unto them that sacrament, a part of the due celebration whereof consisteth in answering to the articles of faith, because the habit of faith which afterwards doth come with years, is but a farther building up of the same edifice, the first foundation whereof was laid by the sacrament of baptism. For that which there we professed without any understanding, when we afterwards come to acknowledge, do we any thing else but only bring unto ripeness the very seed that was sown before? We are then believers, because then we begin to be that which process of time doth make perfect. And till we come to actual belief, the very sacrament of faith is a shield as strong as after this the faith of the sacrament against all contrary infernal powers. Which whosoever doth think impossible, is undoubtedly farther off from Christian belief though he be baptized than are these innocents, which at their baptism albeit they have no conceit or cogitation of faith, are notwithstanding pure and free from all opposite cogitations, whereas the other is not free. If therefore without any fear or scruple we may account them and term them believers only for their outward profession’s sake, which inwardly are farther from faith than infants, why not infants much more at the time of their solemn initiation by baptism the sacrament of faith, whereunto they not only conceive nothing opposite, but have also that grace given them  which is the first and most effectual cause out of which our belief groweth?

In sum, the whole Church is a multitude of believers, all honoured with that title, even hypocrites for their profession’s sake as well as saints because of their inward sincere persuasion, and infants as being in the first degree of their ghostly motion towards the actual habit of faith; the first sort are faithful in the eye of the world, the second faithful in the sight of God; the last in the ready direct way to become both if all things after be suitable to these their present beginnings. “This,” saith St. Augustine, “would not haply content such persons as are uncapable or unquiet, but to them which having knowledge are not troublesome it may suffice. Wherein I have not for ease of myself objected against you that custom only than which nothing is more firm, but of a custom most profitable I have done that little which I could to yield you a reasonable cause.”

[3.]Were St. Augustine now living there are which would tell him for his better instruction that to say of a child “it is elect” and to say it doth believe are all one, for which cause sith no man is able precisely to affirm the one of any infant in particular, it followeth that precisely and absolutely we ought not to say the other.

Which precise and absolute terms are needless in this case. We speak of infants as the rule of piety alloweth both to speak and think. They that can take to themselves in ordinary talk a charitable kind of liberty to name men of their own sort God’s dear children, (notwithstanding the large reign of hypocrisy,) should not methinks be so strict and rigorous against the Church for presuming as it doth of a Christian innocent. For when we know how Christ in general hath said that of such is the kingdom of heaven, which  kingdom is the inheritance of God’s elect,
 and do withal behold how his providence hath called them unto the first beginnings of eternal life, and presented them at the well-spring of new birth wherein original sin is purged, besides which sin there is no hinderance of their salvation known to us, as themselves will grant; hard it were that having so many fair inducements whereupon to ground, we should not be thought to utter at the least a truth as probable and allowable in terming any such particular infant an elect babe, as in presuming the like of others, whose safety nevertheless we are not absolutely able to warrant.

[4.]If any troubled with these scruples be only for instruction’s sake desirous to know yet some farther reason why interrogatories should be ministered to infants in baptism, and be answered unto by others as in their names, they may consider that baptism implieth a covenant or league between God and man, wherein as God doth bestow presently remission of sins and the Holy Ghost, binding also himself to add in process of time what grace soever shall be farther necessary for the attainment of everlasting life; so every baptized soul receiving the same grace at the hands of God tieth likewise itself for ever to the observation of his law, no less than the Jews by circumcision bound themselves to the law of Moses. The law of Christ requiring therefore faith and newness of life in all men by virtue of the covenant which they make in baptism, is it toyish that the Church in baptism exacteth at every man’s hands an express profession of faith and an irrevocable promise of obedience by way of solemn stipulation?



That infants may contract and covenant with God, the law is plain.
 Neither is the reason of the law obscure. For sith it tendeth we cannot sufficiently express how much to their own good, and doth no way hurt or endanger them to begin the race of their lives herewith, they are as equity requireth admitted hereunto, and in favour of their tender years, such formal complements of stipulation as being requisite are impossible by themselves in their own persons to be performed, leave is given that they may sufficiently discharge by others. Albeit therefore neither deaf nor dumb men, neither furious persons nor children can receive any civil stipulation, yet this kind of ghostly stipulation they may through his indulgence, who respecting the singular benefit thereof accepteth children brought unto him for that end, entereth into articles of covenant with them, and in tender commiseration granteth, that other men’s professions and promises in baptism made for them shall avail no less than if they had been themselves able to have made their own.

[5.]None more fit to undertake this office in their behalf than such as present them unto baptism. A wrong conceit, that none may receive the sacrament of baptism but they whose parents, at the least the one of them, are by the soundness of their religion and by their virtuous demeanour known to be men of God, hath caused some to repel children, whosoever bring them, if their parents be mispersuaded in religion, or for other misdeserts excommunicated; some likewise for that cause to withhold baptism, unless the father, albeit no such exception can justly be taken against him, do notwithstanding make profession of his faith, and avouch the child to be his own. Thus whereas God hath appointed  them ministers of holy things, they make themselves inquisitors of men’s persons a great deal farther than need is.

They should consider that God hath ordained baptism in favour of mankind. To restrain favours is an odious thing, to enlarge them acceptable both to God and man. Whereas therefore the civil law gave divers immunities to them which were fathers of three children and had them living, those immunities they held although their children were all dead, if war had consumed them, because it seemed in that case not against reason to repute them by a courteous construction of law as live men, in that the honour of their service done to the commonwealth would remain always. Can it hurt us in exhibiting the graces which God doth bestow on men, or can it prejudice his glory, if the selfsame equity guide and direct our hands?

When God made his covenant with such as had Abraham to their father, was only Abraham’s immediate issue, or only his lineal posterity according to the flesh included in that covenant? Were not proselytes as well as Jews always taken for the sons of Abraham? Yea because the very heads of families are fathers in some sort as touching providence and care for the meanest that belong unto them, the servants which Abraham had bought with money were as capable of  circumcision, being newly born, as any natural child that Abraham himself begat.

Be it then that baptism belongeth to none but such as either believe presently, or else being infants are the children of believing parents. In case the Church do bring children to the holy font whose natural parents are either unknown, or known to be such as the church accurseth, but yet forgetteth not in that severity to take compassion upon their offspring, (for it is the Church which doth offer them to baptism by the ministry of presentors,) were it not against both equity and duty to refuse the mother of believers herself, and not to take her in this case for a faithful parent? It is not the virtue of our fathers nor the faith of any other that can give us the true holiness which we have by virtue of our new birth. Yet even through the common faith and spirit of God’s Church, (a thing which no quality of parents can prejudice,) I say through the faith of the Church of God undertaking the motherly care of our souls, so far forth we may be and are in our infancy sanctified, as to be thereby made sufficiently capable of baptism, and to be interessed in the rites of our new birth for their piety’s sake that offer us thereunto.

“It cometh sometime to pass,” saith St. Augustine, “that the children of bond-slaves are brought to baptism by their  lord;
 sometime the parents being dead, the friends alive undertake that office; sometime strangers or virgins consecrated unto God which neither have nor can have children of their own take up infants in the open streets, and so offer them unto baptism, whom the cruelty of unnatural parents casteth out and leaveth to the adventure of uncertain pity.” As therefore he which did the part of a neighbour was a neighbour to that wounded man whom the parable of the Gospel describeth; so they are fathers, although strangers, that bring infants to him which maketh them the sons of God. In the phrase of some kind of men they use to be termed Witnesses, as if they came but to see and testify what is done. It savoureth more of piety to give them their old accustomed name of Fathers and Mothers in God, whereby they are well put in mind what affection they ought to bear towards those innocents, for whose religious education the Church accepteth them as pledges.

[6.]This therefore is their own duty. But because the answer which they make to the usual demands of stipulation proposed in baptism is not their own, the Church doth best to receive it of them in that form which best sheweth whose the act is. That which a guardian doth in the name of his guard or pupil standeth by natural equity forcible for his benefit, though it be done without his knowledge. And shall we judge it a thing unreasonable, or in any respect unfit, that infants by words which others utter should, though unwittingly yet truly and forcibly, bind themselves to that whereby their estate is so assuredly bettered? Herewith Nestorius the heretic was charged as having fallen from his  first profession, and broken the promise which he made to God in the arms of others.
 Of such as profaned themselves being Christians with irreligious delight in the ensigns of idolatry, heathenish spectacles, shows, and stage plays, Tertullian to strike them the more deep claimeth the promise which they made in baptism. Why were they dumb being thus challenged? Wherefore stood they not up to answer in their own defence, that such professions and promises made in their names were frivolous, that all which others undertook for them was but mockery and profanation? That which no heretic, no wicked liver, no impious despiser of God, no miscreant or malefactor, which had himself been baptized, was ever so desperate as to disgorge in contempt of so fruitfully received customs, is now their voice that restore as they say the ancient purity of religion.


\section*{Of the Cross in Baptism.}
LXV. In baptism many things of very ancient continuance are now quite and clean abolished, for that the virtue and grace of this sacrament had been therewith overshadowed, as fruit with too great abundance of leaves. Notwithstanding to them which think it always imperfect reformation that doth but shear and not flay, our retaining certain of those former rites, especially the dangerous sign of the cross, hath seemed almost an impardonable oversight. “The cross,” they say, “sith it is but a mere invention of man, should not therefore at all have been added to the sacrament of baptism. To sign children’s foreheads with a cross, in token that hereafter they shall not be ashamed to make profession of the faith of Christ, is to bring into the Church a new word, whereas there ought to be no Doctor heard in the Church but our Saviour Christ. That reason which moved the Fathers to use, should move us not to use, the sign of the cross. They lived with heathens which had the cross of Christ in contempt, we with such as adore the cross, and  therefore we ought to abandon it even as in like consideration Ezechias did of old the brazen serpent.”

[2.]These are the causes of displeasure conceived against the cross, a ceremony the use whereof hath been profitable although we observe it not as the ordinance of God but of man. For, saith Tertullian, “if of this and the like customs thou shouldest require some commandment to be shewed thee out of Scriptures, there is none found.” What reason there is to justify tradition, use or custom in this behalf, “either thou mayest of thyself perceive, or else learn of some other that doth.” Lest therefore the name of tradition should be offensive to any, considering how far by some it hath been and is abused, we mean by traditions, ordinances made in the prime of Christian religion, established with that authority which Christ hath left to his Church for matters indifferent, and in that consideration requisite to be observed, till like authority see just and reasonable cause to alter them. So that traditions ecclesiastical are not rudely and in gross to be shaken off, because the inventors of them were men.

[3.]Such as say they allow no invention of man to be  mingled with the outward administration of sacraments, and under that pretence condemn our using the sign of the cross, have belike some special dispensation themselves to violate their own rules.
 For neither can they indeed decently nor do they ever baptize any without manifest breach of this their profound axiom, that “men’s inventions should not be mingled with sacraments and institutions of God.” They seem to like very well in baptism the custom of godfathers, “because so generally all churches have received it.” Which custom being of God no more instituted than the other, (howsoever they pretend the other hurtful and this profitable,) it followeth that even in their own opinion, if their words do shew their minds, there is no necessity of stripping sacraments out of all such attire of ceremonies as man’s wisdom hath at any time clothed them withal, and consequently that either they must reform their speech as over general, or else condemn their own practice as unlawful.

[4.]Ceremonies have more in weight than in sight, they work by commonness of use much, although in the several acts of their usage we scarcely discern any good they do. And because the use which they have for the most part is not perfectly understood, superstition is apt to impute unto them greater virtue than indeed they have. For prevention whereof when we use this ceremony we always plainly express the end whereunto it serveth, namely, for a sign of remembrance to put us in mind of our duty.

But by this mean they say we make it a great deal worse.  For why?
 Seeing God hath no where commanded to draw two lines in token of the duty which we owe to Christ, our practice with this exposition publisheth a new gospel, and causeth another word to have place in the Church of Christ, where no voice ought to be heard but his.

By which good reason the authors of those grave Admonitions to the Parliament are well holpen up, which held that “sitting” at communions “betokeneth rest and full accomplishment of legal ceremonies in our Saviour Christ.” For although it be the word of God that such ceremonies are expired, yet seeing it is not the word of God that men to signify so much should sit at the table of our Lord, these have their doom as well as others, “Guilty of a new-devised gospel in the Church of Christ.”

[5.]Which strange imagination is begotten of a special dislike they have to hear that ceremonies now in use should be thought significant, whereas in truth such as are not significant must needs be vain. Ceremonies destitute of signification are no better than the idle gestures of men whose broken wits are not masters of that they do. For if we look but into secular and civil complements, what other cause can there possibly be given why to omit them where of course they are looked for, (for where they are not so due to use them, bringeth men’s secret intents oftentimes into great jealousy,) I would know I say what reason we are able to yield why things so light in their own nature should weigh in the opinions of men so much, saving only in regard of that which they use to signify or betoken?

Doth not our Lord Jesus Christ himself impute the omission  of some courteous ceremonies even in domestical entertainment to a colder degree of loving affection, and take the contrary in better part, not so much respecting what was less done as what was signified less by the one than by the other?
 For to that very end he referreth in part those gracious expostulations, “Simon, seest thou this woman? Since I entered into thine house thou gavest me no water for my feet, but she hath washed my feet with tears, and wiped them with the hairs of her head; thou gavest me no kiss, but this woman since the time I came in, hath not ceased to kiss my feet; mine head with oil thou didst not anoint, but this woman hath anointed my feet with ointment.”

Wherefore as the usual dumb ceremonies of common life are in request or dislike according to that they import, even so religion having likewise her silent rites, the chiefest rule whereby to judge of their quality is that which they mean or betoken. For if they signify good things, (as somewhat they must of necessity signify, because it is of their very nature to be signs of intimation, presenting both themselves unto outward sense and besides themselves some other thing to the understanding of beholders,) unless they be either greatly mischosen to signify the same, or else applied where that which they signify agreeth not, there is no cause of exception against them as against evil and unlawful ceremonies, much less of excepting against them only in that they are not without sense.

And if every religious ceremony which hath been invented of men to signify any thing that God himself alloweth were the publication of another gospel in the Church of Christ, seeing that no Christian church in the world is or can be without continual use of some ceremonies which men have instituted, and that to signify good things, (unless they be vain and frivolous ceremonies,) it would follow that the world hath no Christian church which doth not daily proclaim new gospels, a sequel the manifest absurdity whereof argueth the rawness of that supposal out of which it groweth.

[6.]Now the cause why antiquity did the more in actions  of common life honour the ceremony of the cross might be for that they lived with infidels. But that which they did in the sacrament of baptism was for the selfsame good of believers which is thereby intended still. The Cross is for us an admonition no less necessary than for them to glory in the service of Jesus Christ, and not to hang down our heads as men ashamed thereof, although it procure us reproach and obloquy at the hands of this wretched world.

Shame is a kind of fear to incur disgrace and ignominy. Now whereas some things are worthy of reproach, some things ignominious only through a false opinion which men have conceived of them, nature that generally feareth opprobrious reprehension must by reason and religion be taught what it should be ashamed of and what not. But be we never so well instructed what our duty is in this behalf, without some present admonition at the very instant of practice, what we know is many times not called to mind till that be done whereupon our just confusion ensueth. To supply the absence of such as that way might do us good when they see us in danger of sliding, there are judicious and wise men which think we may greatly relieve ourselves by a bare imagined presence of some, whose authority we fear and would be loth  to offend, if indeed they were present with us.
 “Witnesses at hand are a bridle unto many offences. Let the mind have always some whom it feareth, some whose authority may keep even secret thoughts under awe. Take Cato, or if he be too harsh and rugged, choose some other of a softer mettle, whose gravity of life and speech thou lovest, his mind and countenance carry with thee, set him always before thine eyes either as a watch or as a pattern. That which is crooked we cannot straighten but by some such level.”

If men of so good experience and insight in the maims of our weak flesh, have thought these fancied remembrances available to awaken shamefacedness, that so the boldness of sin may be stayed ere it look abroad, surely the wisdom of the Church of Christ which hath to that use converted the ceremony of the cross in baptism it is no Christian man’s part to despise, especially seeing that by this mean where nature doth earnestly implore aid, religion yieldeth her that ready assistance than which there can be no help more forcible serving only to relieve memory, and to bring to our cogitation that which should most make ashamed of sin.

[7.]The mind while we are in this present life, whether it contemplate, meditate, deliberate, or howsoever exercise itself, worketh nothing without continual recourse unto imagination, the only storehouse of wit and peculiar chair of memory. On this anvil it ceaseth not day and night to strike, by means whereof as the pulse declareth how the heart doth work, so the very thoughts and cogitations of man’s mind be they good or bad do no where sooner bewray themselves, than  through the crevices of that wall wherewith nature hath compassed the cells and closets of fancy.
 In the forehead nothing more plain to be seen than the fear of contumely and disgrace. For which cause the Scripture (as with great probability it may be thought) describeth them marked of God in the forehead, whom his mercy hath undertaken to keep from final confusion and shame. Not that God doth set any corporal mark on his chosen, but to note that he giveth his elect security of preservation from reproach, the fear whereof doth use to shew itself in that part. Shall I say, that the sign of the Cross (as we use it) is in some sort a mean to work our preservation from reproach? Surely the mind which as yet hath not hardened itself in sin is seldom provoked thereunto in any gross and grievous manner, but nature’s secret suggestion objecteth against it ignominy as a bar. Which conceit being entered into that palace of man’s fancy, the gates whereof hath imprinted in them that holy sign which bringeth forthwith to mind whatsoever Christ hath wrought and we vowed against sin, it cometh hereby to pass that Christian men never want a most effectual though a silent teacher to avoid whatsoever may deservedly procure shame. So that in things which we should be ashamed of, we are by the Cross admonished faithfully of our duty at the very moment when admonition doth most need.

[8.]Other things there are which deserve honour and yet do purchase many times our disgrace in this present world, as of old the very truth of religion itself, till God by his own outstretched arm made the glory thereof to shine over all the earth. Whereupon St. Cyprian exhorting to martyrdom in times of heathenish persecution and cruelty, thought it not vain to allege unto them with other arguments the very ceremony of that Cross whereof we speak. Never let that  hand offer sacrifice to idols which hath already received the Body of our Saviour Christ, and shall hereafter the crown of his glory;
 “Arm your foreheads” unto all boldness, that “the Sign of God” may be kept safe.

Again, when it pleased God that the fury of their enemies being bridled the Church had some little rest and quietness, (if so small a liberty but only to breathe between troubles may be termed quietness and rest,) to such as fell not away from Christ through former persecutions, he giveth due and deserved praise in the selfsame manner. “You that were ready to endure imprisonment, and were resolute to suffer death; you that have courageously withstood the world, ye have made yourselves both a glorious spectacle for God to behold, and a worthy example for the rest of your brethren to follow. Those mouths which had sanctified themselves with food coming down from heaven loathed after Christ’s own Body and Blood to taste the poisoned and contagious scraps of idols; those foreheads which the sign of God had purified kept themselves to be crowned by him, the touch of the garlands of Satan they abhorred.” Thus was the memory of that sign which they had in baptism a kind of bar or prevention to keep them even from apostasy, whereinto the frailty of flesh and blood overmuch fearing to endure shame, might peradventure the more easily otherwise have drawn them.

[9.]We have not now through the gracious goodness of Almighty God, those extreme conflicts which our fathers had with blasphemous contumelies every where offered to the name of Christ, by such as professed themselves infidels and unbelievers. Howbeit, unless we be strangers to the age wherein we live, or else in some partial respect dissemblers  of that we hourly both hear and see, there is not the simplest of us but knoweth with what disdain and scorn Christ is honoured far and wide.
 Is there any burden in the world more heavy to bear than contempt? Is there any contempt that grieveth as theirs doth, whose quality no way making them less worthy than others are of reputation, only the service which they do to Christ in the daily exercise of religion treadeth them down? Doth any contumely which we sustain for religion’s sake pierce so deeply as that which would seem even of mere conscience religiously spiteful? When they that honour God are despised; when the chiefest service of honour that man can do unto him, is the cause why they are despised; when they which pretend to honour him and that with greatest sincerity, do with more than heathenish petulancy trample under foot almost whatsoever either we or the whole Church of God by the space of so many ages have been accustomed unto, for the comelier and better exercise of our religion according to the soundest rules that wisdom directed by the word of God, and by long experience confirmed, hath been able with common advice, with much deliberation and exceeding great diligence, to comprehend; when no man fighting under Christ’s banner can be always exempted from seeing or sustaining those indignities, the sting whereof not to feel, or feeling, not to be moved thereat, is a thing impossible to flesh and blood; if this be any object for patience to work on, the strictest bond that thereunto tieth us is our vowed obedience to Christ; the solemnest vow that we ever made to obey Christ and to suffer willingly all reproaches for his sake was made in baptism; and amongst other memorials to keep us mindful of that vow we cannot think that the sign which our new baptized foreheads did there receive is either unfit or unforcible, the reasons hitherto alleged being weighed with indifferent balance.

[10.]It is not (you will say) the cross in our foreheads, but in our hearts the faith of Christ that armeth us with patience, constancy, and courage. Which as we grant to be most true, so neither dare we despise no not the meanest helps that serve though it be but in the very lowest degree of furtherance towards the highest services that God doth require  at our hands.
 And if any man deny that such ceremonies are available at the least as memorials of duty, or do think that himself hath no need to be so put in mind what our duties are, it is but reasonable that in the one the public experience of the world overweigh some few men’s persuasion, and in the other the rare perfection of a few condescend unto common imbecility.

[11.]Seeing therefore that to fear shame which doth worthily follow sin, and to bear undeserved reproach constantly is the general duty of all men professing Christianity; seeing also that our weakness while we are in this present world doth need towards spiritual duties the help even of corporal furtherances, and that by reason of natural intercourse between the highest and the lowest powers of man’s mind in all actions, his fancy or imagination carrying in it that special note of remembrance, than which there is nothing more forcible where either too weak or too strong a conceit of infamy and disgrace might do great harm, standeth always ready to put forth a kind of necessary helping hand; we are in that respect to acknowledge the good and profitable use of this ceremony, and not to think it superfluous that Christ hath his mark applied unto that part where bashfulness appeareth, in token that they which are Christians should be at no time ashamed of his ignominy.

But to prevent some inconveniences which might ensue if the over ordinary use thereof (as it fareth with such rites when they are too common) should cause it to be of less observation or regard where it most availeth, we neither omit it in that place, nor altogether make it so vulgar as the custom heretofore hath been: although to condemn the whole Church of God when it most flourished in zeal and piety, to mark that age with the brand of error and superstition only because they had this ceremony more in use than we now think needful, boldly to affirm that this their practice grew so soon through a fearful malediction of God upon the ceremony of  the cross,
 as if we knew that his purpose was thereby to make it manifest in all men’s eyes how execrable those things are in his sight which have proceeded from human invention, is as we take it a censure of greater zeal than knowledge. Men whose judgments in these cases are grown more moderate, although they retain not as we do the use of this ceremony, perceive notwithstanding very well such censures to be out of square, and do therefore not only acquit the Fathers from superstition therein, but also think it sufficient to answer in excuse of themselves, “This ceremony which was but a thing indifferent even of old we judge not at this day a matter necessary for all Christian men to observe.”

[12.]As for their last upshot of all towards this mark, they are of opinion that if the ancient Christians to deliver the Cross of Christ from contempt did well and with good consideration use often the sign of the cross, in testimony of their faith and profession before infidels which upbraided them with Christ’s sufferings, now that we live with such as contrariwise adore the sign of the cross, (because contrary diseases should always have contrary remedies,) we ought to take away all use thereof. In which conceit they both ways greatly seduce themselves, first for that they imagine the Fathers to have had no use of the cross but with reference unto infidels, which mispersuasion we have before discovered at large; and secondly by reason that they think there is not any other way besides universal extirpation to reform superstitious abuses of the cross. Wherein because there are that stand very much upon the example of Ezechias, as if his breaking to pieces that serpent of brass whereunto the children of Israel had burnt incense, did enforce the utter abolition of this ceremony, the fact of that virtuous prince is by so much the more attentively to be considered.

[13.]Our lives in this world are partly guided by rules,  and partly directed by examples.
 To conclude out of general rules and axioms by discourse of wit our duties in every particular action, is both troublesome and many times so full of difficulty that it maketh deliberations hard and tedious to the wisest men. Whereupon we naturally all incline to observe examples, to mark what others have done before us, and in favour of our own ease rather to follow them than to enter into new consultation, if in regard of their virtue and wisdom we may but probably think they have waded without error. So that the willingness of men to be led by example of others both discovereth and helpeth the imbecility of our judgment. Because it doth the one, therefore insolent and proud wits would always seem to be their own guides; and because it doth the other, we see how hardly the vulgar sort is drawn unto any thing for which there are not as well examples as reasons alleged. Reasons proving that which is more particular by things more general and farther from sense are with the simpler sort of men less trusted, for that they doubt of their own judgment in those things; but of examples which prove unto them one doubtful particular by another more familiarly and sensibly known, they easily perceive in themselves some better ability to judge. The force of examples therefore is great, when in matter of action being doubtful what to do we are informed what others have commendably done whose deliberations were like.

[14.]But whosoever doth persuade by example must as well respect the fitness as the goodness of that he allegeth. To Ezechias God himself in this fact giveth testimony of well doing. So that nothing is here questionable but only whether the example alleged be pertinent, pregnant, and strong.

The serpent spoken of was first erected for the extraordinary and miraculous cure of the Israelites in the desert. This use having presently an end when the cause for which God  ordained it was once removed, the thing itself they notwithstanding kept for a monument of God’s mercy, as in like consideration they did the pot of manna, the rod of Aaron, and the sword which David took from Goliah.
 In process of time they made of a monument of divine power a plain idol, they burnt incense before it contrary to the law of God, and did it the services of honour due unto God only. Which gross and grievous abuse continued till Ezechias restoring the purity of sound religion, destroyed utterly that which had been so long and so generally a snare unto them.

It is not amiss which the canon law hereupon concludeth, namely that “if our predecessors have done some things which at that time might be without fault, and afterward be turned to error and superstition, we are taught by Ezechias breaking the brazen serpent that posterity may destroy them without any delay and with great authority.” But may it be simply and without exception hereby gathered, that posterity “is bound to destroy” whatsoever hath been either at the first invented, or but afterwards turned to like superstition and error? No, it cannot be.

The serpent therefore and the sign of the cross, although seeming equal in this point, that superstition hath abused both, yet being herein also unequal, that neither they have been both subject to the like degree of abuse, nor were in hardness of redress alike, it may be that even as the one for abuse was religiously taken away, so now, when religion hath taken away abuse from the other, we should by utter abolition thereof deserve hardly his commendation whose example there is offered us no such necessary cause to follow.

[15.]For by the words of Ezechias in terming the serpent but “a lump of brass,” to shew that the best thing in it now was the metal or matter whereof it consisted, we may probably conjecture, that the people whose error is therein controlled had the selfsame opinion of it which the heathens had of  idols; they thought that the power of Deity was with it, and when they saw it dissolved haply they might to comfort themselves imagine as Olympius the sophister did beholding the dissipation of idols, “Shapes and counterfeits they were, fashioned of matter subject unto corruption, therefore to grind them to dust was easy, but those celestial powers which dwelt and resided in them are ascended into heaven.”

Some difference there is between these opinions of palpable idolatry and that which the schools in speculation have bolted out concerning the cross. Notwithstanding forasmuch as the church of Rome hath hitherto practised and doth profess the same adoration to the sign of the cross and neither less nor other than is due unto Christ himself, howsoever they varnish and qualify their sentence, pretending that the cross, which to outward sense presenteth visibly itself alone, is not by them apprehended alone, but hath in their secret surmise or conceit a reference to the person of our Lord Jesus Christ, so that the honour which they jointly do to both respecteth principally his person, and the cross but only for his person’s sake, the people not accustomed to trouble their wits with so nice and subtle differences in the exercise of religion are apparently no less ensnared by adoring the cross, than the Jews by burning incense to the brazen serpent.

It is by Thomas ingenuously granted, that because unto reasonable creatures a kind of reverence is due for the excellency which is in them and whereby they resemble God, therefore if reasonable creatures, angels or men, should receive at our hands holy and divine honour as the sign of the cross  doth at theirs,
 to pretend that we honour not them alone but we honour God with them would not serve the turn, neither would this be able to prevent the error of men, or cause them always to respect God in their adorations, and not to finish their intents in the object next before them. But unto this he addeth, that no such error can grow by adoring in that sort a dead image, which every man knoweth to be void of excellency in itself, and therefore will easily conceive that the honour done unto it hath an higher reference.

Howbeit, seeing that we have by over-true experience been taught how often, especially in these cases, the light even of common understanding faileth, surely their usual adoration of the cross is not hereby freed. For in actions of this kind we are more to respect what the greatest part of men is commonly prone to conceive, than what some few men’s wits may devise in construction of their own particular meanings. Plain it is, that a false opinion of some personal divine excellency to be in those things which either nature or art hath framed causeth always religious adoration. And as plain that the like adoration applied unto things sensible argueth to vulgar capacities, yea leaveth imprinted in them the very same opinion of Deity from whence all idolatrous worship groweth. Yea the meaner and baser a thing worshipped is in itself, the more they incline to think that every man which doth adore it, knoweth there is in it or with it a presence of divine power.

[16.]Be it therefore true that crosses purposely framed or used for receipt of divine honour be even as scandalous as the brazen serpent itself, where they are in such sort adored. Should we hereupon think ourselves in the sight of God and in conscience charged to abolish utterly the very ceremony of the cross, neither meant at the first, nor now converted unto any such offensive purpose? Did the Jews which could never be persuaded to admit in the city of Jerusalem that image of Cæsar which the Romans were accustomed to adore,  make any scruple of Cæsar’s image in the coin which they knew very well that men were not wont to worship?
 Between the cross which superstition honoureth as Christ, and that ceremony of the cross which serveth only for a sign of remembrance, there is as plain and as great a difference as between those brazen images which Salomon made to bear up the cistern of the temple, and (sith both were of like shape but of unlike use) that which the Israelites in the wilderness did adore; or between the altars which Josias destroyed because they were instruments of mere idolatry, and that which the tribe of Reuben with others erected near to the river Jordan, for which also they grew at the first into some dislike, and were by the rest of their brethren suspected yea hardly charged with open breach of the law of God, accused of backwardness in religion, upbraided bitterly with the fact of Peor, and the odious example of Achan, as if the building of their altar in that place had given manifest shew of no better than intended apostasy, till by a true declaration made in their own defence it appeared that such as misliked misunderstood their enterprise, inasmuch as they had no intent to build any altar for sacrifice, which God would have no where offered saving in Jerusalem only, but to a far other end and purpose, which being opened satisfied all parts, and so delivered them from causeless blame.

[17.]In this particular suppose the worst, imagine that the immaterial ceremony of the Cross had been the subject of as gross pollution as any heathenish or profane idol. If we think the example of Ezechias a proof that things which error and superstition hath abused may in no consideration be tolerated, although we presently find them not subject to so vile abuse, the plain example of Ezechias proveth the contrary. The temples and idols which under Salomon had been of very purpose framed for the honour of foreign gods Ezechias destroyed not, because they stood as forlorn things and did now  no harm, although formerly they had done harm.
 Josias for some inconvenience afterwards razed them up. Yet to both there is one commendation given even from God himself, that touching matter of religion they walked in the steps of David and did no way displease God.

[18.]Perhaps it seemeth that by force and virtue of this example although in bare detestation and hatred of idolatry all things which have been at any time worshipped are not necessarily to be taken out of the world, nevertheless for remedy and prevention of so great offences wisdom should judge it the safest course to remove altogether from the eyes of men that which may put them in mind of evil.

Some kinds of evil no doubt there are very quick in working on those affections that most easily take fire, which evils should in that respect no oftener than need requireth be brought in presence of weak minds. But neither is the Cross any such evil, nor yet the brazen serpent itself so strongly poisoned, that our eyes, ears, and thoughts ought to shun them both, for fear of some deadly harm to ensue the only representation thereof by gesture, shape, sound, or such like significant means. And for mine own part I most assuredly persuade myself, that had Ezechias (till the days of whose most virtuous reign they ceased not continually to burn incense to the brazen serpent) had he found the serpent, though sometimes adored, yet at that time recovered from the evil of so gross abuse, and reduced to the same that was before in the time of David, at which time they esteemed it only as a memorial, sign, or monument of God’s miraculous goodness towards them, even as we in no other sort esteem the ceremony of the Cross, the due consideration of an use so harmless common to both might no less have wrought their equal preservation, than different occasions have procured, notwithstanding the one’s extinguishment, the other’s lawful continuance.

[19.]In all persuasions which ground themselves upon example, we are not so much to respect what is done, as the causes and secret inducements leading thereunto. The question being therefore whether this ceremony supposed to have  been sometimes scandalous and offensive ought for that cause to be now removed;
 there is no reason we should forthwith yield ourselves to be carried away with examples, no not of them whose acts the highest judgment approveth for having reformed in that manner any public evil: but before we either attempt any thing or resolve, the state and condition as well of our own affairs as theirs whose example presseth us, is advisedly to be examined; because some things are of their own nature scandalous, and cannot choose but breed offence, as those sinks of execrable filth which Josias did overwhelm; some things albeit not by nature and of themselves, are notwithstanding so generally turned to evil by reason of an evil corrupt habit grown and through long continuance incurably settled in the minds of the greatest part, that no redress can be well hoped for without removal of that wherein they have ruined themselves, which plainly was the state of the Jewish people, and the cause why Ezechias did with such sudden indignation destroy what he saw worshipped; finally some things are as the sign of the Cross though subject either almost or altogether to as great abuse, yet curable with more facility and ease. And to speak as the truth is, our very nature doth hardly yield to destroy that which may be fruitfully kept, and without any great difficulty clean scoured from the rust of evil which by some accident hath grown into it. Wherefore to that which they build in this question upon the example of Ezechias let this suffice.

[20.]When heathens despised Christian religion, because of the sufferings of Jesus Christ, the Fathers to testify how little such contumelies and contempts prevailed with them chose rather the sign of the Cross than any other outward mark, whereby the world might most easily discern always what they were. On the contrary side now, whereas they which do all profess the Christian religion are divided amongst themselves, and the fault of the one part is that in zeal to the sufferings of Christ they admire too much and over-superstitiously adore the visible sign of his Cross, if you ask what we that mislike them should do, we are here advised to cure one contrary by another. Which art or method is not yet so current as they imagine.




For if, as their practice for the most part sheweth, it be their meaning that the scope and drift of reformation when things are faulty should be to settle the Church in the contrary, it standeth them upon to beware of this rule, because seeing vices have not only virtues but other vices also in nature opposite unto them, it may be dangerous in these cases to seek but that which we find contrary to present evils. For in sores and sicknesses of the mind we are not simply to measure good by distance from evil, because one vice may in some respect be more opposite to another than either of them to that virtue which holdeth the mean between them both. Liberality and covetousness, the one a virtue and the other a vice, are not so contrary as the vices of covetousness and prodigality; religion and superstition have more affiance, though the one be light and the other darkness, than superstition and profaneness which both are vicious extremities. By means whereof it cometh also to pass that the mean which is virtue seemeth in the eyes of each extreme an extremity; the liberal hearted man is by the opinion of the prodigal miserable, and by the judgment of the miserable lavish; impiety for the most part upbraideth religion as superstitious, which superstition often accuseth as impious, both so conceiving thereof because it doth seem more to participate each extreme, than one extreme doth another, and is by consequent less contrary to either of them, than they mutually between themselves. Now if he that seeketh to reform covetousness or superstition should but labour to induce the contrary, it were but to draw men out of lime into coaldust. So that their course which will remedy the superstitious abuse of things profitable in the Church is not still to abolish utterly the use thereof, because not using at all is most opposite to ill using, but rather if it may be to bring them back to a right perfect and religious usage, which albeit less contrary to the present sore is notwithstanding the better and by many degrees the sounder way of recovery.

[21.]And unto this effect that very precedent itself which they propose may be best followed. For as the Fathers when the Cross of Christ was in utter contempt did not superstitiously adore the same, but rather declare that they so esteemed it as was meet: in like manner where we find the Cross to have that honour which is due to Christ, is it not as lawful for us to  retain it in that estimation which it ought to have and in that use which it had of old without offence,
 as by taking it clean away to seem followers of their example which cure wilfully by abscission that which they might both preserve and heal?

Touching therefore the sign and ceremony of the Cross, we no way find ourselves bound to relinquish it, neither because the first inventors thereof were but mortal men, nor lest the sense and signification we give unto it should burden us as authors of a new gospel in the house of God, nor in respect of some cause which the Fathers had more than we have to use the same, nor finally for any such offence or scandal as heretofore it hath been subject unto by error now reformed in the minds of men.


\section*{Of Confirmation after Baptism.}
LXVI. The ancient custom of the Church was after they had baptized, to add thereunto imposition of hands with effectual prayer for the illumination of God’s most Holy Spirit to confirm and perfect that which the grace of the same Spirit had already begun in baptism.

For our means to obtain the graces which God doth bestow are our prayers. Our prayers to that intent are available as well for others as for ourselves. To pray for others is to bless them for whom we pray, because prayer procureth the blessing of God upon them, especially the prayer of such as God either most respecteth for their piety and zeal that way, or else regardeth for that their place and calling bindeth them above others unto this duty as it doth both natural and spiritual fathers.

With prayers of spiritual and personal benediction the manner hath been in all ages to use imposition of hands, as a ceremony betokening our restrained desires to the party, whom we present unto God by prayer. Thus when Israel blessed Ephraim and Manasses Joseph’s sons, he imposed upon them his hands and prayed, “God, in whose sight my fathers Abraham and Isaac did walk, God which hath fed me all my life long unto this day, and the Angel which hath delivered me from evil bless these children.” The prophets which healed diseases by prayer, used therein the selfsame ceremony. And therefore when Eliseus willed Naaman to  wash himself seven times in Jordan for cure of his foul disease it much offended him;
 “I thought,” saith he, “with myself, surely the man will come forth and stand and call upon the name of the Lord his God, and put his hand on the place to the end he may so heal the leprosy.” In consecrations and ordinations of men unto rooms of divine calling, the like was usually done from the time of Moses to Christ. Their suits that came unto Christ for help were also tendered oftentimes and are expressed in such forms or phrases of speech as shew that he was himself an observer of the same custom. He which with imposition of hands and prayer did so great works of mercy for restoration of bodily health, was worthily judged as able to effect the infusion of heavenly grace into them whose age was not yet depraved with that malice which might be supposed a bar to the goodness of God towards them. They brought him therefore young children to put his hands upon them and pray.

[2.]After the ascension of our Lord and Saviour Jesus Christ, that which he had begun continued in the daily practice of his Apostles, whose prayer and imposition of hands were a mean whereby thousands became partakers of the wonderful gifts of God. The Church had received from Christ a promise that such as have believed in him these signs and tokens should follow them. “To cast out devils, to speak with tongues, to drive away serpents, to be free from the harm which any deadly poison could work, and to cure diseases by imposition of hands.” Which power, common at the first in a manner unto all believers, all believers had not power to derive or communicate unto all other men, but whosoever was the instrument of God to instruct, convert and baptize them, the gift of miraculous operations by the power of the Holy Ghost they had not but only at the Apostles’ own hands. For which cause Simon Magus perceiving that power to be in none but them, and presuming that they which had it might sell it, sought to purchase it of them with money.




[3.]And as miraculous graces of the Spirit continued after the Apostles’ times; (“for,” saith Irenæus, “they which are truly his disciples do in his name and through grace received from him such works for the benefit of other men as every of them is by him enabled to work; some cast out devils, insomuch as they which are delivered from wicked spirits have been thereby won unto Christ, and do constantly persevere in the church and society of faithful men; some excel in the knowledge of things to come, in the grace of visions from God, and the gift of prophetical predictions; some by laying on their hands restore them to health which are grievously afflicted with sickness; yea there are that of dead have been made alive and have afterwards many years conversed with us. What should I say? The gifts are innumerable wherewith God hath enriched his Church throughout the world, and by virtue whereof in the name of Christ crucified under Pontius Pilate the Church every day doth many wonders for the good of nations, neither fraudulently nor in any respect of lucre and gain to herself, but as freely bestowing as God on her hath bestowed his divine graces;”) so it no where appeareth that ever any did by prayer and imposition of hands sithence the Apostles’ times make others partakers of the like miraculous gifts and graces, as long as it pleased God to continue the same in his Church, but only Bishops the Apostles’ successors for a time even in that power. St. Augustine acknowledgeth that such gifts were not permitted to last always, lest men should wax cold with the commonness of that the strangeness whereof at the first inflamed them. Which words of St. Augustine declaring how the  vulgar use of those miracles was then expired, are no prejudice to the like extraordinary graces more rarely observed in some either then or of later days.

[4.]Now whereas the successors of the Apostles had but only for a time such power as by prayer and imposition of hands to bestow the Holy Ghost, the reason wherefore confirmation nevertheless by prayer and laying on of hands hath hitherto always continued, is for other very special benefits which the Church thereby enjoyeth. The Fathers every where impute unto it that gift or grace of the Holy Ghost, not which maketh us first Christian men, but when we are made such, assisteth us in all virtue, armeth us against temptation and sin. For, after baptism administered, “there followeth,” saith Tertullian, “imposition of hands with invocation and invitation of the Holy Ghost, which willingly cometh down from the Father to rest upon the purified and blessed bodies, as it were acknowledging the waters of baptism a fit seat.” St. Cyprian in more particular manner alluding to that effect of the Spirit which here especially was respected, “How great,” saith he, “is that power and force wherewith the mind is here” (he meaneth in baptism) “enabled, being not only withdrawn from that pernicious hold which the world before had of it, nor only so purified and made clean that no stain or blemish of the enemy’s invasion doth remain, but over and besides” (namely through prayer and imposition of  hands) “becometh yet greater,
 yet mightier in strength, so far as to reign with a kind of imperial dominion over the whole band of that roaming and spoiling adversary.” As much is signified by Eusebius Emisenus saying, “The Holy Ghost which descendeth with saving influence upon the waters of baptism doth there give that fulness which sufficeth for innocency, and afterwards exhibiteth in confirmation an augmentation of further grace.” The Fathers therefore being thus persuaded held confirmation as an ordinance apostolic always profitable in God’s Church, although not always accompanied with equal largeness of those external effects which gave it countenance at the first.

[5.]The cause of severing confirmation from baptism (for most commonly they went together) was sometimes in the minister, which being of inferior degree might baptize but not confirm, as in their case it came to pass whom Peter and John did confirm, whereas Philip had before baptized them; and in theirs of whom St. Jerome hath said, “I deny not but the custom of the churches is that the Bishop should go abroad, and imposing his hands pray for the gift of the Holy Ghost on them whom presbyters and deacons far off in lesser cities have already baptized.” Which ancient custom of the Church St. Cyprian groundeth upon the example of Peter and John in the eighth of the Acts before  alleged.
 The faithful in Samaria, saith he, “had already obtained baptism: only that which was wanting Peter and John supplied, by prayer and imposition of hands to the end the Holy Ghost might be poured upon them. Which also is done amongst ourselves, when they which be already baptized are brought to the Prelates of the Church to obtain by our prayer and imposition of hands the Holy Ghost.” By this it appeareth that when the ministers of baptism were persons of inferior degree, the Bishops did after confirm whom such had before baptized.

[6.]Sometimes they which by force of their ecclesiastical calling might do as well the one as the other, were notwithstanding men whom heresy had disjoined from the fellowship of true believers. Whereupon when any man by them baptized and confirmed came afterwards to see and renounce their error, there grew in some churches very hot contention about the manner of admitting such into the bosom of the true Church, as hath been declared already in the question of rebaptization. But the general received custom was only to admit them with imposition of hands and prayer. Of which custom while some imagined the reason to be for that heretics might give remission of sins by baptism, but not the Spirit by imposition of hands because themselves had not God’s Spirit, and that therefore their baptism might stand, but confirmation must be given again: the imbecility of this ground gave Cyprian occasion to oppose himself against the practice of the Church herein, labouring many ways to prove that heretics could do neither, and, consequently, that their  baptism in all respects was as frustrate as their chrism; for the manner of those times was in confirming to use anointing. On the other side against Luciferians which ratified only the baptism of heretics but disannulled their confirmations and consecrations under pretence of the reason which hath been before specified, “heretics cannot give the Holy Ghost,” St. Jerome proveth at large, that if baptism by heretics be granted available to remission of sins, which no man receiveth without the Spirit, it must needs follow that the reason taken from disability of bestowing the Holy Ghost was no reason wherefore the Church should admit converts with any new imposition of hands. Notwithstanding because it might be objected, that if the gift of the Holy Ghost do always join itself with true baptism, the Church, which thinketh the bishop’s confirmation after other men’s baptism needful for the obtaining of the Holy Ghost, should hold an error, St. Jerome hereunto maketh answer, that the cause of this observation is not any absolute impossibility of receiving the Holy Ghost by the sacrament of baptism unless a bishop add after it the imposition of hands, but rather a certain congruity and fitness to honour prelacy with such preeminences, because the safety of the Church dependeth upon the dignity of her chief superiors, to whom if some eminent offices of power above others should not be given, there would be in the Church as many schisms as priests. By which answer it  appeareth his opinion was, that the Holy Ghost is received in baptism;
 that confirmation is only a sacramental complement; that the reason why bishops alone did ordinarily confirm, was not because the benefit, grace, and dignity thereof is greater than of baptism, but rather, for that by the Sacrament of Baptism men being admitted into God’s Church, it was both reasonable and convenient that if he baptize them not unto whom the chiefest authority and charge of their souls belongeth, yet for honour’s sake and in token of his spiritual superiority over them, because to bless is an act of authority, the performance of this annexed ceremony should be sought for at his hands. Now what effect their imposition of hands hath either after baptism administered by heretics or otherwise, St. Jerome in that place hath made no mention, because all men understood that in converts it tendeth to the fruits of repentance, and craveth in behalf of the penitent such grace as David after his fall desired at the hands of God; in others the fruit and benefit thereof is that which hath been before shewed.

[7.]Finally sometime the cause of severing confirmation from baptism was in the parties that received baptism being infants, at which age they might be very well admitted to live in the family; but because to fight in the army of God, to discharge the duties of a Christian man, to bring forth the fruits and to do the works of the Holy Ghost their time of ability was not yet come (so that baptism were not deferred) there could by stay of their confirmation no harm ensue but rather good. For by this mean it came to pass that children in expectation thereof were seasoned with the principles of true religion before malice and corrupt examples depraved  their minds,
 a good foundation was laid betimes for direction of the course of their whole lives, the seed of the Church of God was preserved sincere and sound, the prelates and fathers of God’s family to whom the cure of their souls belonged saw by trial and examination of them a part of their own heavy burden discharged, reaped comfort by beholding the first beginnings of true godliness in tender years, glorified Him whose praise they found in the mouths of infants, and neglected not so fit opportunity of giving every one fatherly encouragement and exhortation. Whereunto imposition of hands and prayer being added, our warrant for the great good effect thereof is the same which Patriarchs, Prophets, Priests, Apostles, Fathers and men of God have had for such their particular invocations and benedictions, as no man I suppose professing truth of religion will easily think to have been without fruit.

[8.]No, there is no cause we should doubt of the benefit, but surely great cause to make complaint of the deep neglect of this Christian duty almost with all them to whom by right of their place and calling the same belongeth. Let them not take it in evil part, the thing is true, their small regard hereunto hath done harm in the Church of God. That which error rashly uttereth in disgrace of good things  may peradventure be sponged out,
 when the print of those evils which are grown through neglect will remain behind.

[9.]Thus much therefore generally spoken may serve for answer unto their demands that require us to tell them “why there should be any such confirmation in the Church,” seeing we are not ignorant how earnestly they have protested against it; and how directly (although untruly, for so they are content to acknowledge) it hath by some of them been said to be “first brought in by the feigned decretal epistles of the Popes:” or why it should not be “utterly abolished, seeing that no one tittle thereof can be once found in the whole Scripture,” except the epistle to the Hebrews be Scripture: and again seeing that how free soever it be now from abuse, if we look back to the times past, which wise men do always more respect than the present, it hath been abused, and is found at the length no such profitable ceremony as the whole silly Church of Christ for the space of these sixteen hundred years hath through want of experience imagined: last of all “seeing” also besides the cruelty which is shewed towards poor country people, who are fain sometime to let their ploughs stand still, and with incredible wearisome toil of their feeble bodies to wander over mountains and through woods it may be now and then little less than a whole “half-score of miles” for a  bishop’s blessing, “which if it were needful might as well be done at home in their own parishes,” rather than they to purchase it with so great loss and so intolerable pain. There are they say in confirmation besides this, three terrible points.

The first is “laying on of hands with pretence that the same is done to the example of the Apostles,” which is not only as they suppose “a manifest untruth” (for all the world doth know that the Apostles did never after baptism lay hands on any, and therefore St. Luke which saith they did was much deceived) but farther also we thereby teach men to think imposition of hands a sacrament, belike because it is a principle engrafted by common light of nature in the minds of men that all things done by apostolic example must needs be sacraments.

The second high point of danger is, that by “tying confirmation to the bishop alone there is great cause of suspicion given to think that baptism is not so precious a thing as confirmation:” for will any man think that a velvet coat is of more price than a linen coif, knowing the one to be an ordinary garment, the other an ornament which only sergeants at law do wear?

Finally, to draw to an end of perils, the last and the weightiest hazard is where the book itself doth say that children by imposition of hands and prayer may receive strength against all temptation: which speech as a two-edged sword doth both ways dangerously wound; partly because it ascribeth grace to imposition of hands, whereby we are able no more to assure ourselves in the warrant of any promise from God that his heavenly grace shall be given, than the Apostle was that himself should obtain grace by the bowing of his knees to God; and partly because by using the very word strength in this matter, a word so apt to spread infection, we “maintain” with “popish” evangelists an old forlorn “distinction” of the Holy Ghost bestowed upon Christ’s Apostles before his ascension into heaven, and “augmented” upon them afterwards, a distinction of grace infused into Christian  men by degrees,
 planted in them at the first by baptism, after cherished, watered, and (be it spoken without offence) strengthened as by other virtuous offices which piety and true religion teacheth, even so by this very special benediction whereof we speak, the rite or ceremony of Confirmation.


\section*{Of the Sacrament of the body and blood of Christ.}
LXVII. The grace which we have by the holy Eucharist doth not begin but continue life. No man therefore receiveth this sacrament before Baptism, because no dead thing is capable of nourishment. That which groweth must of necessity first live. If our bodies did not daily waste, food to restore them were a thing superfluous. And it may be that the grace of baptism would serve to eternal life, were it not that the state of our spiritual being is daily so much hindered and impaired after baptism. In that life therefore where neither body nor soul can decay, our souls shall as little require this sacrament as our bodies corporal nourishment. But as long as the days of our warfare last, during the time that we are both subject to diminution and capable of augmentation in grace, the words of our Lord and Saviour Christ will remain forcible, “Except ye eat the flesh of the Son of man, and drink his blood, ye have no life in you.”

Life being therefore proposed unto all men as their end, they which by baptism have laid the foundation and attained the first beginning of a new life have here their nourishment and food prescribed for continuance of life in them. Such as will live the life of God must eat the flesh and drink the blood of the Son of man, because this is a part of that diet which if we want we cannot live. Whereas therefore in our infancy we are incorporated into Christ and by Baptism receive the grace of his Spirit without any sense or feeling of the gift which God bestoweth, in the Eucharist we so receive the gift of God, that we know by grace what the grace is which God giveth us, the degrees of our own increase in holiness and virtue we see and can judge of them, we understand that the strength of our life begun in Christ is Christ, that his flesh is meat and his blood drink, not by surmised imagination but truly, even so truly that through faith we perceive in the body and blood sacramentally presented the  very taste of eternal life, the grace of the sacrament is here as the food which we eat and drink.

[2.]This was it that some did exceedingly fear, lest Zwinglius and Œcolampadius would bring to pass, that men should account of this sacrament but only as of a shadow, destitute, empty and void of Christ. But seeing that by opening the several opinions which have been held, they are grown for aught I can see on all sides at the length to a general agreement concerning that which alone is material, namely the real participation of Christ and of life in his body and blood by means of this sacrament; wherefore should the world continue still distracted and rent with so manifold contentions, when there remaineth now no controversy saving only about the subject where Christ is? Yea even in this point no side denieth but that the soul of man is the receptacle of Christ’s presence. Whereby the question is yet driven to a narrower issue, nor doth any thing rest doubtful but this, whether when the sacrament is administered Christ be whole within man only, or else his body and blood be also externally seated in the very consecrated elements themselves; which opinion they that defend are driven either to consubstantiate and incorporate Christ with elements sacramental, or to transubstantiate and change their substance into his; and so the one to hold him really but invisibly moulded up with the substance of those elements, the other to hide him under the only visible show of bread and wine, the substance whereof  as they imagine is abolished and his succeeded in the same room.

[3.]All things considered and compared with that success which truth hath hitherto had by so bitter conflicts with errors in this point, shall I wish that men would more give themselves to meditate with silence what we have by the sacrament, and less to dispute of the manner how? If any man suppose that this were too great stupidity and dulness, let us see whether the Apostles of our Lord themselves have not done the like. It appeareth by many examples that they of their own disposition were very scrupulous and inquisitive, yea in other cases of less importance and less difficulty always apt to move questions. How cometh it to pass that so few words of so high a mystery being uttered, they receive with gladness the gift of Christ and make no show of doubt or scruple? The reason hereof is not dark to them which have any thing at all observed how the powers of the mind are wont to stir when that which we infinitely long for presenteth itself above and besides expectation. Curious and intricate speculations do hinder, they abate, they quench such inflamed motions of delight and joy as divine graces use to raise when extraordinarily they are present. The mind therefore feeling present joy is always marvellous unwilling to admit any other cogitation, and in that case casteth off those disputes whereunto the intellectual part at other times easily draweth.

A manifest effect whereof may be noted if we compare with our Lord’s disciples in the twentieth of John the people that are said in the sixth of John to have gone after him to Capernaum. These leaving him on the one side the sea of Tiberias, and finding him again as soon as themselves by ship were arrived on the contrary side, whither they knew that by ship he came not, and by land the journey was longer than according to the time he could have to travel, as they wondered so they asked also, “Rabbi, when camest thou hither?” The disciples when Christ appeared to them in far more strange and miraculous manner moved no question, but rejoiced greatly in that they saw. For why? The one sort beheld only that in Christ which they knew was more than natural, but  yet their affection was not rapt therewith through any great extraordinary gladness, the other when they looked on Christ were not ignorant that they saw the wellspring of their own everlasting felicity;
 the one because they enjoyed not disputed, the other disputed not because they enjoyed.

[4.]If then the presence of Christ with them did so much move, judge what their thoughts and affections were at the time of this new presentation of Christ not before their eyes but within their souls. They had learned before that his flesh and blood are the true cause of eternal life; that this they are not by the bare force of their own substance, but through the dignity and worth of his Person which offered them up by way of sacrifice for the life of the whole world, and doth make them still effectual thereunto; finally that to us they are life in particular, by being particularly received. Thus much they knew, although as yet they understood not perfectly to what effect or issue the same would come, till at the length being assembled for no other cause which they could imagine but to have eaten the Passover only that Moyses appointeth, when they saw their Lord and Master with hands and eyes lifted up to heaven first bless and consecrate for the endless good of all generations till the world’s end the chosen elements of bread and wine, which elements made for ever the instruments of life by virtue of his divine benediction they being the first that were commanded to receive from him, the first which were warranted by his promise that not only unto them at the present time but to whomsoever they and their successors after them did duly administer the same, those mysteries should serve as conducts of life and conveyances of his body and blood unto them, was it possible they should hear that voice, “Take, eat, this is my body; drink ye all of this, this is my blood;” possible that doing what was required and believing what was promised, the same should have present effect in them, and not fill them with a kind of fearful admiration at the heaven which they saw in themselves? They had at that time a sea of comfort and joy to wade in, and we by that which they did are taught that this heavenly food is given for the satisfying of our empty souls, and not for the exercising of our curious and subtle wits.




[5.]If we doubt what those admirable words may import, let him be our teacher for the meaning of Christ to whom Christ was himself a schoolmaster, let our Lord’s Apostle be his interpreter, content we ourselves with his explication, My body, the communion of my body, My blood, the communion of my blood. Is there any thing more expedite, clear, and easy, than that as Christ is termed our life because through him we obtain life, so the parts of this sacrament are his body and blood for that they are so to us who receiving them receive that by them which they are termed? The bread and cup are his body and blood because they are causes instrumental upon the receipt whereof the participation of his body and blood ensueth. For that which produceth any certain effect is not vainly nor improperly said to be that very effect whereunto it tendeth. Every cause is in the effect which groweth from it. Our souls and bodies quickened to eternal life are effects the cause whereof is the Person of Christ, his body and his blood are the true wellspring out of which this life floweth. So that his body and blood are in that very subject whereunto they minister life not only by effect or operation, even as the influence of the heavens is in plants, beasts, men, and in every thing which they quicken, but also by a far more divine and mystical kind of union, which maketh us one with him even as he and the Father are one.

[6.]The real presence of Christ’s most blessed body and blood is not therefore to be sought for in the sacrament, but in the worthy receiver of the sacrament.

And with this the very order of our Saviour’s words agreeth, first “take and eat;” then “this is my Body which was broken for you:” first “drink ye all of this;” then followeth “this is my Blood of the New Testament which is shed for many for the remission of sins.” I see not which way it should be gathered by the words of Christ, when and where the bread is His body or the cup His blood, but only in the very heart and soul of him which receiveth them. As for the sacraments, they really exhibit, but for aught we can gather out of that which is written of them, they are not really nor do really contain in themselves that grace which with them or by them it pleaseth God to bestow.




If on all sides it be confessed that the grace of Baptism is poured into the soul of man, that by water we receive it although it be neither seated in the water nor the water changed into it, what should induce men to think that the grace of the Eucharist must needs be in the Eucharist before it can be in us that receive it?

The fruit of the Eucharist is the participation of the body and blood of Christ. There is no sentence of Holy Scripture which saith that we cannot by this sacrament be made partakers of his body and blood except they be first contained in the sacrament, or the sacrament converted into them. “This is my body,” and “this is my blood,” being words of promise, sith we all agree that by the sacrament Christ doth really and truly in us perform his promise, why do we vainly trouble ourselves with so fierce contentions whether by consubstantiation, or else by transubstantiation the sacrament itself be first possessed with Christ, or no? A thing which no way can either further or hinder us howsoever it stand, because our participation of Christ in this sacrament dependeth on the co-operation of his omnipotent power which maketh it his body and blood to us, whether with change or without alteration of the element such as they imagine we need not greatly to care nor inquire.




[7.]Take therefore that wherein all agree, and then consider by itself what cause why the rest in question should not rather be left as superfluous than urged as necessary. It is on all sides plainly confessed, first that this sacrament is a true and a real participation of Christ, who thereby imparteth himself even his whole entire Person as a mystical Head unto every soul that receiveth him, and that every such receiver doth thereby incorporate or unite himself unto Christ as a mystical member of him, yea of them also whom he acknowledgeth to be his own; secondly that to whom the person of Christ is thus communicated, to them he giveth by the same sacrament his Holy Spirit to sanctify them as it sanctifieth him which is their head; thirdly that what merit, force or virtue soever there is in his sacrificed body and blood,  we freely, fully and wholly have it by this sacrament;
 fourthly that the effect thereof in us is a real transmutation of our souls and bodies from sin to righteousness, from death and corruption to immortality and life; fifthly that because the sacrament being of itself but a corruptible and earthly creature must needs be thought an unlikely instrument to work so admirable effects in man, we are therefore to rest ourselves altogether upon the strength of his glorious power who is able and will bring to pass that the bread and cup which he giveth us shall be truly the thing he promiseth.

[8.]It seemeth therefore much amiss that against them whom they term Sacramentaries so many invective discourses are made all running upon two points, that the Eucharist is not a bare sign or figure only, and that the efficacy of his body and blood is not all we receive in this sacrament. For no man having read their books and writings which are thus traduced can be ignorant that both these assertions they plainly confess to be most true. They do not so interpret the words of Christ as if the name of his body did import but the figure of his body, and to be were only to signify his blood. They grant that these holy mysteries received in due manner do instrumentally both make us partakers of the grace of that body and blood which were given for the life of the world, and besides also impart unto us even in true and real though mystical manner the very Person of our Lord himself, whole, perfect, and entire, as hath been shewed.

[9.]Now whereas all three opinions do thus far accord in one, that strong conceit which two of the three have embraced as touching a literal, corporal and oral manducation of the very substance of his flesh and blood is surely an opinion no where delivered in Holy Scripture, whereby they should think themselves bound to believe it, and (to speak with the softest terms we can use) greatly prejudiced in that when some others did so conceive of eating his flesh, our Saviour to abate that error in them gave them directly to understand how his flesh so eaten could profit them nothing, because the words which he spake were spirit, that is to say, they had a  reference to a mystical participation, which mystical participation giveth life.
 Wherein there is small appearance of likelihood that his meaning should be only to make them Marcionites by inversion, and to teach them that as Marcion did think Christ seemed to be a man but was not, so they contrariwise should believe that Christ in truth would so give them as they thought his flesh to eat, but yet lest the horror thereof should offend them, he would not seem to do that he did.

[10.]When they which have this opinion of Christ in that blessed sacrament go about to explain themselves, and to open after what manner things are brought to pass, the one sort lay the union of Christ’s deity with his manhood as their first foundation and ground; from thence they infer a power which the body of Christ hath thereby to present itself in all places; out of which ubiquity of his body they gather the presence thereof with that sanctified bread and wine of our Lord’s table; the conjunction of his body and blood with those elements they use as an argument to shew how the bread may as well in that respect be termed his body because his body is therewith joined, as the Son of God may be named man by reason that God and man in the person of Christ are united; to this they add how the words of Christ commanding us to eat must needs import that as he hath coupled the substance of his flesh and the substance of bread together, so we together should receive both. Which labyringh as the other sort doth justly shun, so the way which they take to the same inn is somewhat more short but no whit more certain. For through God’s omnipotent power they imagine that transubstantiation followeth upon the words of consecration, and upon transubstantiation the participation of Christ’s both body and blood in the only shape of sacramental elements.

So that they all three do plead God’s omnipotency: Sacramentaries to that alteration which the rest confess he accomplisheth; the patrons of transubstantiation over and besides that to the change of one substance into another; the followers of consubstantiation to the kneading up of both substances as it were into one lump.

[11.]Touching the sentence of antiquity in this cause, first forasmuch as they knew that the force of this sacrament doth necessarily presuppose the verity of Christ’s both body and  blood, they used oftentimes the same as an argument to prove that Christ hath as truly the substance of man as of God, because here we receive Christ and those graces which flow from him in that he is man.
 So that if he have no such being, neither can the sacrament have any such meaning as we all confess it hath. Thus Tertullian, thus Ireney, thus Theodoret disputeth.

Again as evident it is how they teach that Christ is personally there present, yea present whole, albeit a part of Christ be corporally absent from thence; that Christ assisting this heavenly banquet with his personal and true presence doth by his own divine power add to the natural substance thereof supernatural efficacy, which addition to the nature of those consecrated elements changeth them and maketh them that  unto us which otherwise they could not be; that to us they are thereby made such instruments as mystically yet truly, invisibly yet really work our communion or fellowship with the person of Jesus Christ as well in that he is man as God, our participation also in the fruit, grace and efficacy of his body and blood, whereupon there ensueth a kind of transubstantiation in us, a true change both of soul and body, an alteration from death to life. In a word it appeareth not that of all the ancient Fathers of the Church any one did ever conceive or imagine other than only a mystical participation of Christ’s both body and blood in the sacrament, neither are their speeches concerning the change of the elements themselves into the body and blood of Christ such, that a man can thereby in conscience assure himself it was their meaning to persuade the world either of a corporal consubstantiation of Christ with those sanctified and blessed elements before we receive them, or of the like transubstantiation of them into the body and blood of Christ. Which both to our mystical communion with Christ are so unnecessary, that the Fathers who  plainly hold but this mystical communion cannot easily be thought to have meant any other change of sacramental elements than that which the same spiritual communion did require them to hold.

[12.]These things considered, how should that mind which loving truth and seeking comfort out of holy mysteries hath not perhaps the leisure, perhaps not the wit nor capacity to tread out so endless mazes, as the intricate disputes of this cause have led men into, how should a virtuously disposed mind better resolve with itself than thus? “Variety of judgments and opinions argueth obscurity in those things whereabout they differ. But that which all parts receive for truth, that which every one having sifted is by no one denied or doubted of, must needs be matter of infallible certainty. Whereas therefore there are but three expositions made of ‘this is my body,’ the first, ‘this is in itself before participation really and truly the natural substance of my body by reason of the coexistence which my omnipotent body hath with the sanctified element of bread,’ which is the Lutherans’ interpretation; the second, ‘this is itself and before participation the very true and natural substance of my body, by force of that Deity which with the words of consecration abolisheth the substance of bread and substituteth in the place thereof my Body,’ which is the popish construction; the last, ‘this hallowed food, through concurrence of divine power, is in verity and truth, unto faithful receivers, instrumentally a cause of that mystical participation, whereby as I make myself wholly theirs, so I give them in hand an actual possession of all such saving grace as my sacrificed body can yield, and as their souls do presently need, this is to them and in them my body:’ of these three rehearsed interpretations the last hath in it nothing but what the rest do all approve and acknowledge to be most true, nothing but that which the words of Christ are on all sides confessed to enforce, nothing but that which the Church of God hath always thought necessary, nothing but that which alone is sufficient for every Christian man to believe concerning the use and force of this sacrament, finally nothing but that wherewith the writings of all antiquity are consonant and all Christian confessions agreeable. And as truth in what kind soever is  by no kind of truth gainsayed, so the mind which resteth itself on this is never troubled with those perplexities which the other do both find, by means of so great contradiction between their opinions and true principles of reason grounded upon experience, nature and sense. Which albeit with boisterous courage and breath they seem oftentimes to blow away, yet whoso observeth how again they labour and sweat by subtlety of wit to make some show of agreement between their peculiar conceits and the general edicts of nature, must needs percieve they struggle with that which they cannot fully master. Besides sith of that which is proper to themselves their discourses are hungry and unpleasant, full of tedious and irksome labour, heartless and hitherto without fruit, on the other side read we them or hear we others be they of our own or of ancienter times, to what part soever they be thought to incline touching that whereof there is controversy, yet in this where they all speak but one thing their discourses are heavenly, their words sweet as the honeycomb, their tongues melodiously tuned instruments, their sentences mere consolation and joy, are we not hereby almost even with voice from heaven, admonished which we may safeliest cleave unto?

“He which hath said of the one sacrament, ‘wash and be clean,’ hath said concerning the other likewise, ‘eat and live.’ If therefore without any such particular and solemn warrant as this is that poor distressed woman coming unto Christ for health could so constantly resolve herself, ‘may I but touch the skirt of his garment I shall be whole,’ what moveth us to argue of the manner how life should come by bread, our duty being here but to take what is offered, and most assuredly to rest persuaded of this, that can we but eat we are safe? When I behold with mine eyes some small and scarce discernible grain or seed whereof nature maketh promise that a tree shall come, and when afterwards of that tree any skilful artificer undertaketh to frame some exquisite and curious work, I look for the event, I move no question about performance either of the one or of the other. Shall I simply credit nature in things natural, shall  I in things artificial rely myself on art, never offering to make doubt, and in that which is above both art and nature refuse to believe the author of both, except he acquaint me with his ways, and lay the secret of his skill before me? Where God himself doth speak those things which either for height and sublimity of matter, or else for secresy of performance we are not able to reach unto, as we may be ignorant without danger, so it can be no disgrace to confess we are ignorant. Such as love piety will as much as in them lieth know all things that God commandeth, but especially the duties of service which they owe to God. As for his dark and hidden works, they prefer as becometh them in such cases simplicity of faith before that knowledge, which curiously sifting what it should adore, and disputing too boldly of that which the wit of man cannot search, chilleth for the most part all warmth of zeal, and bringeth soundness of belief many times into great hazard. Let it therefore be sufficient for me presenting myself at the Lord’s table to know what there I receive from him, without searching or inquiring of the manner how Christ performeth his promise; let disputes and questions, enemies to piety, abatements of true devotion, and hitherto in this cause but over patiently heard, let them take their rest; let curious and sharpwitted men beat their heads about what questions themselves will, the very letter of the word of Christ giveth plain security that these mysteries do as nails fasten us to his very Cross, that by them we draw out, as touching efficacy, force, and virtue, even the blood of his gored side, in the wounds of our Redeemer we there dip our tongues, we are dyed red both within and without, our hunger is satisfied and our thirst for ever quenched; they are things wonderful which he feeleth, great which he seeth and unheard of which he uttereth, whose soul is possessed of this Paschal Lamb and made joyful in the strength of this new wine, this bread hath in it more than the substance which our eyes behold, this cup hallowed with solemn benediction availeth to the  endless life and welfare both of soul and body,
 in that it serveth as well for a medicine to heal our infirmities and purge our sins as for a sacrifice of thanksgiving; with touching it sanctifieth, it enlighteneth with belief, it truly conformeth us unto the image of Jesus Christ; what these elements are in themselves it skilleth not, it is enough that to me which take them they are the body and blood of Christ, his promise in witness hereof sufficeth, his word he knoweth which way to accomplish; why should any cogitation possess the mind of a faithful communicant but this, O my God thou art true, O my Soul thou art happy!”

[13.]Thus therefore we see that howsoever men’s opinions do otherwise vary, nevertheless touching Baptism and the Supper of the Lord, we may with consent of the whole Christian world conclude they are necessary, the one to initiate or begin, the other to consummate or make perfect our life in Christ.


\section*{Of faults noted in the form of administering that holy Sacrament.}
LXVIII. In administering the Sacrament of the Body and Blood of Christ, the supposed faults of the Church of England are not greatly material, and therefore it shall suffice to touch them in few words. “The first is that we do not use in a generality once for all to say to communicants ‘take eat, and drink,’ but unto every particular person, ‘eat thou, drink thou,’ which is according to the popish manner and not the form that our Saviour did use. Our second oversight is by gesture. For in kneeling there hath been superstition; sitting agreeth better to the action of a supper; and our Saviour using that which was most fit did himself not kneel. A third accusation is for not examining all communicants, whose knowledge in the mystery of the Gospel should that way be made manifest, a thing every where they say used in the Apostles’ times, because all things necessary were used,  and this in their opinion is necessary,
 yea it is commanded inasmuch as the Levites are commanded to prepare the people for the Passover, and examination is a part of their preparation, our Lord’s Supper in place of the Passover. The fourth thing misliked is that against the Apostle’s prohibition to have any familiarity at all with notorious offenders, papists being not of the Church are admitted to our very communion before they have by their religious and gospel-like behaviour purged themselves of that suspicion of popery which their former life hath caused. They are dogs, swine, unclean beasts, foreigners and strangers from the Church of God, and therefore ought not to be admitted though they offer themselves. We are fifthly condemned, inasmuch as when there have been store of people to hear sermon and service in the church we suffer the communion to be ministered to a few. It is not enough that our book of common prayer hath godly exhortations to move all thereunto which are present. For it should not suffer a few to communicate, it should by ecclesiastical discipline and civil punishment provide that such as would withdraw themselves might be brought to communicate, according both to the law of God and the ancient church canons. In the sixth and last place cometh the enormity of imparting this sacrament privately unto the sick.”




[2.]Thus far accused we answer briefly to the first that seeing God by sacraments doth apply in particular unto every man’s person the grace which himself hath provided for the benefit of all mankind, there is no cause why administering the sacraments we should forbear to express that in our forms of speech, which he by his word and gospel teacheth all to believe. In the one sacrament “I baptize thee” displeaseth them not. If “eat thou” in the other offend them, their fancies are no rules for churches to follow.

Whether Christ at his last supper did speak generally once to all, or to every one in particular, is a thing uncertain. His words are recorded in that form which serveth best for the setting down with historical brevity what was spoken, they are no manifest proof that he spake but once unto all which did then communicate, much less that we in speaking unto every communicant severally do amiss, although it were clear that we herein do otherwise than Christ did. Our imitation of him consisteth not in tying scrupulously ourselves unto his syllables, but rather in speaking by the heavenly direction of that inspired divine wisdom which teacheth divers ways to one end, and doth therein control their boldness by whom any profitable way is censured as reprovable only under colour of some small difference from great examples going before. To do throughout every the like circumstance the same which Christ did in this action were by following his footsteps in that sort to err more from the purpose he aimed at than we now do by not following them with so nice and severe strictness.




They little weigh with themselves how dull, how heavy and almost how without sense the greatest part of the common multitude every where is, who think it either unmeet or unnecessary to put them even man by man especially at that time in mind whereabout they are. It is true that in sermons we do not use to repeat our sentences severally to every particular hearer, a strange madness it were if we should. The softness of wax may induce a wise man to set his stamp or image therein; it persuadeth no man that because wool hath the like quality it may therefore receive the like impression. So the reason taken from the use of sacraments in that they are instruments of grace unto every particular man may with good congruity lead the Church to frame accordingly her words in administration of sacraments, because they easily admit this form, which being in sermons a thing impossible without apparent ridiculous absurdity, agreement of sacraments with sermons in that which is alleged as a reasonable proof of conveniency for the one proveth not the same allegation impertinent because it doth not enforce the other to be administered in like sort. For equal principles do then avail unto equal conclusions when the matter whereunto we apply them is equal, and not else.

[3.]Our kneeling at Communions is the gesture of piety. If we did there present ourselves but to make some show or dumb resemblance of a spiritual feast, it may be that sitting  were the fitter ceremony;
 but coming as receivers of inestimable grace at the hands of God, what doth better beseem our bodies at that hour than to be sensible witnesses of minds unfeignedly humbled? Our Lord himself did that which custom and long usage had made fit; we that which fitness and great decency hath made usual.

[4.]The trial of ourselves before we eat of this bread and drink of this cup is by express commandment every man’s precise duty. As for necessity of calling others unto account besides ourselves, albeit we be not thereunto drawn by any great strength which is in their arguments, who first press us with it as a thing necessary by affirming that the Apostles did use it, and then prove the Apostles to have used it by affirming it to be necessary; again albeit we greatly muse  how they can avouch that God did command the Levites to prepare their brethren against the feast of the Passover,
 and that the examination of them was a part of their preparation, when the place alleged to this purpose doth but charge the Levites saying, “make ready Laahhechem for your brethren,” to the end they may do according to the word of the Lord by Moses:—wherefore in the selfsame place it followeth how lambs and kids and sheep and bullocks were delivered unto the Levites, and that thus “the service was made ready;” it followeth likewise how the Levites having in such sort provided for the people, they made provision for “themselves and for the priests the sons of Aaron;” so that confidently from hence to conclude the necessity of examination argueth their wonderful great forwardness in framing all things to serve their turn:—nevertheless the examination of communicants when need requireth, for the profitable use it may have in such cases, we reject not.

[5.]Our fault in admitting popish communicants, is it in that we are forbidden to eat and therefore much more to communicate with notorious malefactors? The name of a papist is not given unto any man for being a notorious male-factor. And the crime wherewith we are charged is suffering of papists to communicate, so that be their life and conversation whatsoever in the sight of men, their popish opinions are in this case laid as bars and exceptions against them, yea those opinions which they have held in former times although they now both profess by word and offer to shew by fact the contrary. All this doth not justify us, which ought not (they  say) to admit them in any wise, till their gospel-like behaviour have removed all suspicion of popery from them, because papists are “dogs, swine,
 beasts, foreigners and strangers” from the house of God; in a word, they are “not of the Church.”

[6.]What the terms of “gospel-like behaviour” may include is obscure and doubtful. But of the Visible Church of Christ in this present world, from which they separate all papists, we are thus persuaded: Church is a word which art hath devised thereby to sever and distinguish that society of men which professeth the true religion from the rest which profess it not. There have been in the world from the very first foundation thereof but three religions, Paganism which lived in the blindness of corrupt and depraved nature; Judaism embracing the Law which reformed heathenish impiety, and taught salvation to be looked for through one whom God in the last days would send and exalt to be Lord of all; finally Christian belief which yieldeth obedience to the Gospel of Jesus Christ, and acknowledgeth him the Saviour whom God did promise. Seeing then that the Church is a name which art hath given to professors of true religion, as they which will define a man are to pass by those qualities wherein one man doth excel another, and to take only those essential properties whereby a man doth differ from creatures of other kinds, so he that will teach what the Church is shall never rightly perform the work whereabout he goeth, till in matter of religion he touch that difference which severeth the Church’s Religion from theirs who are not the Church. Religion being therefore a matter partly of contemplation partly of action, we must define the Church which is a religious society by such differences as do properly explain the essence of such things, that is to say, by the object or matter whereabout the contemplations and actions of the Church are properly conversant. For so all knowledges and all virtues are defined. Whereupon because the only object which separateth ours from other religions is Jesus Christ, in whom none but the Church doth  believe and whom none but the Church doth worship, we find that accordingly the Apostles do every where distinguish hereby the Church from infidels and from Jews, accounting “them which call upon the name of our Lord Jesus Christ to be his Church.”

If we go lower, we shall but add unto this certain casual and variable accidents, which are not properly of the being, but make only for the happier and better being of the Church of God, either in deed, or in men’s opinions and conceits. This is the error of all popish definitions that hitherto have been brought. They define not the Church by that which the Church essentially is, but by that wherein they imagine their own more perfect than the rest are. Touching parts of eminency and perfection, parts likewise of imperfection and defect in the Church of God, they are infinite, their degrees and differences no way possible to be drawn unto any certain account. There is not the least contention and variance, but it blemisheth somewhat the unity that ought to be in the Church of Christ, which notwithstanding may have not only without offence or breach of concord her manifold varieties in rites and ceremonies of religion, but also her strifes and contentions many times and that about matters of no small importance, yea her schisms, factions and such other evils whereunto the body of the Church is subject, sound and sick remaining both of the same body, as long as both parts retain by outward profession that vital substance of truth which maketh Christian religion to differ from theirs which acknowledge not our Lord Jesus Christ the blessed Saviour of mankind, give no credit to his glorious gospel, and have his sacraments the seals of eternal life in derision.

Now the privilege of the visible Church of God (for of that we speak) is to be herein like the ark of Noah, that, for any thing we know to the contrary, all without it are lost sheep; yet in this was the ark of Noah privileged above the Church that whereas none of them which were in the one could perish, numbers in the other are cast away, because to eternal life our profession is not enough. Many things exclude from the kingdom of God although from the Church they separate not.

In the Church there arise sundry grievous storms, by means  whereof whole kingdoms and nations professing Christ both have been heretofore and are at this present day divided about Christ. During which divisions and contentions amongst men albeit each part do justify itself, yet the one of necessity must needs err if there be any contradiction between them be it great or little, and what side soever it be that hath the truth, the same we must also acknowledge alone to hold with the true Church in that point, and consequently reject the other as an enemy in that case fallen away from the true Church.

Wherefore of hypocrites and dissemblers whose profession at the first was but only from the teeth outward, when they afterwards took occasion to oppugn certain principal articles of faith, the Apostles which defended the truth against them pronounce them “gone out” from the fellowship of sound and sincere believers, when as yet the Christian religion they had not utterly cast off.

In like sense and meaning throughout all ages heretics have justly been hated as branches cut off from the body of the true Vine, yet only so far forth cut off as their heresies have extended. Both heresy and many other crimes which wholly sever from God do sever from the Church of God in part only. “The mystery of piety” saith the Apostle “is without peradventure great, God hath been manifested in the flesh, hath been justified in the Spirit, hath been seen of Angels, hath been preached to nations, hath been believed on in the world, hath been taken up into glory.” The Church a pillar and foundation of this truth, which no where is known or professed but only within the Church, and they all of the Church that profess it. In the meanwhile it cannot be denied that many profess this who are not therefore cleared simply from all either faults or errors which make separation between us and the wellspring of our happiness. Idolatry severed of old the Israelites, iniquity those scribes and Pharisees from God, who notwithstanding were a part of the seed of Abraham, a part of that very seed which God did himself acknowledge to be his Church. The Church of God may therefore contain both them which indeed are not his yet must be reputed his by us that know not their inward thoughts, and them whose  apparent wickedness testifieth even in the sight of the whole world that God abhorreth them.
 For to this and no other purpose are meant those parables which our Saviour in the Gospel hath concerning mixture of vice with virtue, light with darkness, truth with error, as well an openly known and seen as a cunningly cloked mixture.

That which separateth therefore utterly, that which cutteth off clean from the visible Church of Christ is plain Apostasy, direct denial, utter rejection of the whole Christian faith as far as the same is professedly different from infidelity. Heretics as touching those points of doctrine wherein they fail; schismatics as touching the quarrels for which or the duties wherein they divide themselves from their brethren; loose, licentious and wicked persons as touching their several offences or crimes, have all forsaken the true Church of God, the Church which is sound and sincere in the doctrine that they corrupt, the Church that keepeth the bond of unity which they violate, the Church that walketh in the laws of righteousness which they transgress, this very true Church of Christ they have left, howbeit not altogether left nor forsaken simply the Church upon the main foundations whereof they continue built, notwithstanding these breaches whereby they are rent at the top asunder.

[7.]Now because for redress of professed errors and open schisms it is and must be the Church’s care that all may in outward conformity be one, as the laudable polity of former ages even so our own to that end and purpose hath established divers laws, the moderate severity whereof is a mean both to stay the rest and to reclaim such as heretofore have been led awry. But seeing that the offices which laws require are  always definite, and when that they require is done they go no farther, whereupon sundry ill-affected persons to save themselves from danger of laws pretend obedience, albeit inwardly they carry still the same hearts which they did before, by means whereof it falleth out that receiving unworthily the blessed sacrament at our hands, they eat and drink their own damnation; it is for remedy of this mischief here determined,  that whom the law of the realm doth punish unless they communicate, such if they offer to obey law,
 the Church notwithstanding should not admit without probation before had of their gospel-like behaviour.

[8.]Wherein they first set no time how long this supposed probation must continue; again they nominate no certain judgment the verdict whereof shall approve men’s behaviour to be gospel-like; and that which is most material, whereas they seek to make it more hard for dissemblers to be received into the Church than law and polity as yet hath done, they make it in truth more easy for such kind of persons to wind themselves out of the law and to continue the same they were. The law requireth at their hands that duty which in conscience doth touch them nearest, because the greatest difference between us and them is the Sacrament of the Body and Blood of Christ, whose name in the service of our communion we celebrate with due honour, which they in the error of their mass profane. As therefore on our part to hear mass were an open departure from that sincere profession wherein we stand, so if they on the other side receive our communion, they give us the strongest pledge of fidelity that man can demand. What their hearts are God doth know. But if they which mind treachery to God and man shall once apprehend this  advantage given them,
 whereby they may satisfy law in pretending themselves conformable (for what can law with reason or justice require more?) and yet be sure the Church will accept no such offer, till their gospel-like behaviour be allowed; after that our own simplicity hath once thus fairly eased them from sting of law, it is to be thought they will learn the mystery of gospel-like behaviour when leisure serveth them. And so while without any cause we fear to profane sacraments, we shall not only defeat the purpose of most wholesome laws, but lose or wilfully hazard those souls from which the likeliest means of full and perfect recovery are by our indiscretion withheld.

For neither doth God thus bind us to dive into men’s consciences, nor can their fraud and deceit hurt any man but themselves. To him they seem such as they are, but to us they must be taken for such as they seem. In the eye of God they are against Christ that are not truly and sincerely with him, in our eyes they must be received as with Christ that are not to outward show against him.

The case of impenitent and notorious sinners is not like unto theirs whose only imperfection is error severed from pertinacy, error in appearance content to submit itself to better instruction, error so far already cured as to crave at our hands that sacrament the hatred and utter refusal whereof was the weightiest point wherein heretofore they swerved and went astray.

[9.]In this case therefore they cannot reasonably charge us with remiss dealing, or with carelessness to whom we impart the mysteries of Christ, but they have given us manifest occasion to think it requisite that we earnestly advise rather and exhort them to consider as they ought their sundry oversights, first in equalling undistinctly crimes with errors as touching force to make uncapable of this sacrament; secondly in suffering indignation at the faults of the church of Rome to blind and withhold their judgments from seeing that which withal they should acknowledge, concerning so much nevertheless still due to the same church, as to be held and reputed  a part of the house of God,
 a limb of the visible Church of Christ; thirdly in imposing upon the Church a burden to enter farther into men’s hearts and to make a deeper search of their consciences than any law of God or reason of man enforceth; fourthly and lastly in repelling under colour of longer trial such from the mysteries of heavenly grace, as are both capable thereof by the laws of God for any thing we hear to the contrary, and should in divers considerations be cherished according to the merciful examples and precepts whereby the gospel of Christ hath taught us towards such to shew compassion, to receive them with lenity and all meekness, if any thing be shaken in them to strengthen it, not to quench with delays and jealousies that feeble smoke of conformity which seemeth to breathe from them, but to build wheresoever there is any foundation, to add perfection unto slender beginnings, and that as by other offices of piety even so by this very food of life which Christ hath left in his Church not only for preservation of strength but also for relief of weakness.

[10.]But to return to our own selves in whom the next thing severely reproved is the paucity of communicants; if they require at communions frequency we wish the same, knowing how acceptable unto God such service is when multitudes cheerfully concur unto it; if they encourage men thereunto, we also (themselves acknowledge it) are not utterly forgetful to do the like; if they require some public coaction for remedy of that wherein by milder and softer means little good is done, they know our laws and statutes provided in that behalf, whereunto whatsoever convenient help may be added more by the wisdom of man, what cause  have we given the world to think that we are not ready to hearken to it, and to use any good mean of sweet compulsion to have this high and heavenly banquet largely furnished? Only we cannot so far yield as to judge it convenient that the holy desire of a competent number should be unsatisfied, because the greater part is careless and undisposed to join with them.

Men should not (they say) be permitted a few by themselves to communicate when so many are gone away, because this sacrament is a token of our conjunction with our brethren, and therefore by communicating apart from them we make an apparent show of distraction. I ask then on which side unity is broken, whether on theirs that depart or on theirs who being left behind do communicate? First in the one it is not denied but that they may have reasonable causes of departure, and that then even they are delivered from just blame. Of such kind of causes two are allowed, namely danger of impairing health and necessary business requiring our presence otherwhere. And may not a third cause, which is unfitness at the present time, detain us as lawfully back as either of these two? True it is that we cannot hereby altogether excuse ourselves, for that we ought to prevent this and do not. But  if we have committed a fault in not preparing our minds before, shall we therefore aggravate the same with a worse, the crime of unworthy participation?
 He that abstaineth doth want for the time that grace and comfort which religious communicants have, but he that eateth and drinketh unworthily receiveth death, that which is life to others turneth in him to poison.

Notwithstanding whatsoever be the cause for which men abstain, were it reason that the fault of one part should any way abridge their benefit that are not faulty? There is in all the Scripture of God no one syllable which doth condemn communicating amongst a few when the rest are departed from them.

[11.]As for the last thing which is our imparting this sacrament privately unto the sick, whereas there have been of old (they grant) two kinds of necessity wherein this sacrament might be privately administered, of which two  the one being erroneously imagined, and the other (they say) continuing no longer in use, there remaineth unto us no necessity at all, for which that custom should be retained. The falsely surmised necessity is that whereby some have thought all such excluded from possibility of salvation as did depart this life and never were made partakers of the holy Eucharist. The other case of necessity was, when men, which had fallen in time of persecution, and had afterwards repented them, but were not as yet received again unto the fellowship of this communion, did at the hour of their death request it, that so they might rest with greater quietness and comfort of mind, being thereby assured of departure in unity of Christ’s Church, which virtuous desire the Fathers did think it great impiety not to satisfy. This was Serapion’s case of necessity. Serapion a faithful aged person and always of very upright life till fear of persecution in the end caused him to shrink back, after long sorrow for his scandalous offence and suit oftentimes made to be pardoned of the Church, fell at length into grievous sickness, and being ready to yield up the ghost was then more instant than ever before to receive the sacrament. Which sacrament was necessary in this case, not that Serapion had been deprived of everlasting life without it, but that his end was thereby to him made the more comfortable. And do we think, that all cases of such  necessity are clean vanished?
 Suppose that some have by mis-persuasion lived in schism, withdrawn themselves from holy and public assemblies, hated the prayers, and loathed the sacraments of the Church, falsely presuming them to be fraught with impious and Antichristian corruptions, which error the God of mercy and truth opening at the length their eyes to see, they do not only repent them of the evil which they have done but also in token thereof desire to receive comfort by that whereunto they have offered disgrace, (which may be the case of many poor seduced souls even at this day), God forbid we should think that the Church doth sin in permitting the wounds of such to be suppled with that oil which this gracious Sacrament doth yield, and their bruised minds not only need but beg.

[12.]There is nothing which the soul of man doth desire in that last hour so much as comfort against the natural terrors of death and other scruples of conscience which commonly do then most trouble and perplex the weak, towards whom the very law of God doth exact at our hands all the helps that Christian lenity and indulgence can afford. Our general consolation departing this life is the hope of that glorious and blessed resurrection which the Apostle St. Paul nameth ἐξανάστασιν, to note that as all men shall  have their ἀνάστασιν and be raised again from the dead, so the just shall be taken up and exalted above the rest, whom the power of God doth but raise and not exalt. This life and this resurrection our Lord Jesus Christ is for all men as touching the sufficiency of that he hath done; but that which maketh us partakers thereof is our particular communion with Christ, and this sacrament a principal mean as well to strengthen the bond as to multiply in us the fruits of the same communion; for which cause St. Cyprian termeth it a joyful solemnity of expedite and speedy resurrection; Ignatius a medicine which procureth immortality and preventeth death; Irenæus the nourishment of our bodies to eternal life and their preservative from corruption. Now because that Sacrament which at all times we may receive unto this effect is then most acceptable and most fruitful, when any special extraordinary occasion nearly and presently urging kindleth our desires towards it, their severity, who cleave unto that alone which is generally fit to be done and so make all men’s condition alike, may add much affliction to divers troubled and grieved minds, of whose particular estate particular respect being had, according to the charitable order of the church wherein we live, there ensueth unto God that glory which his righteous saints comforted in their greatest distresses do yield, and unto them which have their reasonable petitions satisfied the same contentment, tranquillity, and joy, that others before them by means of like satisfaction have reaped, and wherein we all are or should be desirous finally to take our leave of the world whensoever our own uncertain time of most assured departure shall come.

Concerning therefore both prayers and sacraments together with our usual and received form of administering the same in the church of England, let thus much suffice.




\section*{Of Festival Days, and the natural causes of their convenient institution.}
LXIX. As the substance of God alone is infinite and hath no kind of limitation, so likewise his continuance is from everlasting to everlasting and knoweth neither beginning nor end. Which demonstrable conclusion being presupposed, it followeth necessarily that besides him all things are finite both in substance and in continuance.
Of festival days and the natural causes of their convenient institution.
 If in substance all things be finite, it cannot be but that there are bounds without the compass whereof their substance doth not extend; if in continuance also limited, they all have, it cannot be denied, their set and their certain terms before which they had no being at all. This is the reason why first we do most admire those things which are greatest, and secondly those things which are ancientest, because the one are least distant from the infinite substance, the other from the infinite continuance of God. Out of this we gather that only God hath true immortality or eternity, that is to say continuance wherein there groweth no difference by addition of hereafter unto now, whereas the noblest and perfectest of all things besides have continually through continuance the time of former continuance lengthened, so that they could not heretofore be said to have continued so long as now, neither now so long as hereafter.

[2.]God’s own eternity is the hand which leadeth Angels in the course of their perpetuity; their perpetuity the hand that draweth out celestial motion, the line of which motion and the thread of time are spun together. Now as nature  bringeth forth time with motion,
 so we by motion have learned how to divide time, and by the smaller parts of time both to measure the greater and to know how long all things else endure. For time considered in itself is but the flux of that very instant wherein the motion of the heaven began, being coupled with other things it is the quantity of their continuance measured by the distance of two instants. As the time of a man is a man’s continuance from the instant of his first breath till the instant of his last gasp.

Hereupon some have defined time to be the measure of the motion of heaven, because the first thing which time doth measure is that motion wherewith it began and by the help whereof it measureth other things, as when the Prophet David saith, that a man’s continuance doth not commonly exceed threescore and ten years, he useth the help both of motion and number to measure time. They which make time an effect of motion, and motion to be in nature before time, ought to have considered with themselves that albeit we should deny as Melissus did all motion, we might notwithstanding acknowledge time, because time doth but signify the quantity of continuance, which continuance may be in things that rest and are never moved. Besides we may also consider in rest both that which is past, and that which is present, and that which is future, yea farther even length and shortness in every of these, although we never had conceit of motion. But to define without motion how long or how short such continuance is were impossible. So that herein we must of necessity use the benefit of years, days, hours, minutes, which all grow from celestial motion.

Again forasmuch as that motion is circular whereby we make our divisions of time, and the compass of that circuit such, that the heavens which are therein continually moved and keep in their motions uniform celerity must needs touch often the same points, they cannot choose but bring unto us by equal distances frequent returns of the same times.

Furthermore whereas time is nothing but the mere quantity of that continuance which all things have that are not as God  is without beginning,
 that which is proper unto all quantities agreeth also to this kind, so that time doth but measure other things, and neither worketh in them any real effect nor is itself ever capable of any. And therefore when commonly we use to say that time doth eat or fret out all things, that time is the wisest thing in the world because it bringeth forth all knowledge, and that nothing is more foolish than time which never holdeth any thing long, but whatsoever one day learneth the same another day forgetteth again, that some men see prosperous and happy days, and that some men’s days are miserable, in all these and the like speeches that which is uttered of the time is not verified of time itself, but agreeth unto those things which are in time, and do by means of so near conjunction either lay their burden upon the back, or set their crown upon the head of time. Yea the very opportunities which we ascribe to time do in truth cleave to the things themselves wherewith time is joined; as for time it neither causeth things nor opportunities of things, although it comprise and contain both.

[3.]All things whatsoever having their time, the works of God have always that time which is seasonablest and fittest for them. His works are some ordinary, some more rare, all worthy of observation, but not all of like necessity to be often remembered, they all have their times, but they all do not add the same estimation and glory to the times wherein they are. For as God by being every where yet doth not give unto all places one and the same degree of holiness, so neither one and the same dignity to all times by working in all. For if all either places or times were in respect of God alike, wherefore was it said unto Moyses by particular designation, “This very place wherein thou standest is holy ground?” Why doth the Prophet David choose out of all the days of the year but one whereof he speaketh by way of principal admiration, “This is the day which the Lord hath made?” No doubt as God’s extraordinary presence hath hallowed and sanctified certain places, so they are his extraordinary works that have truly and worthily advanced certain times, for  which cause they ought to be with all men that honour God more holy than other days.

The wise man therefore compareth herein not unfitly the times of God with the persons of men. If any should ask how it cometh to pass that one day doth excel another seeing the light of all the days in the year proceedeth from one sun, to this he answereth, that “the knowledge of the Lord hath parted them asunder, he hath by them disposed the times and solemn feasts; some he hath chosen out and sanctified, some he hath put among the days to number:” even as Adam and all other men are of one substance, all created of the earth, “but the Lord hath divided them by great knowledge and made their ways divers, some he hath blessed and exalted, some he hath sanctified and appropriated unto himself, some he hath cursed, humbled and put them out of their dignity.”

So that the cause being natural and necessary for which there should be a difference in days, the solemn observation whereof declareth religious thankfulness towards him whose works of principal reckoning we thereby admire and honour, it cometh next to be considered what kinds of duties and services they are wherewith such times should be kept holy.


\section*{The manner of celebrating festival days.}
LXX. The sanctification of days and times is a token of that thankfulness and a part of that public honour which we owe to God for admirable benefits, whereof it doth not suffice that we keep a secret calendar, taking thereby our private occasions as we list ourselves to think how much God hath done for all men, but the days which are chosen out to serve as public memorials of such his mercies ought to be clothed with those outward robes of holiness whereby their difference from other days may be made sensible. But because time in itself as hath been already proved can receive no alteration, the hallowing of festival days must consist in the shape or countenance which we put upon the affairs that are incident into those days.

[2.]“This is the day which the Lord hath made,” saith the prophet David; “let us rejoice and be glad in it.” So  that generally offices and duties of religious joy are that wherein the hallowing of festival times consisteth.
 The most natural testimonies of our rejoicing in God are first His praises set forth with cheerful alacrity of mind, secondly our comfort and delight expressed by a charitable largeness of somewhat more than common bounty, thirdly sequestration from ordinary labours, the toils and cares whereof are not meet to be companions of such gladness. Festival solemnity therefore is nothing but the due mixture as it were of these three elements, Praise, and Bounty, and Rest.

Touching praise, forasmuch as the Jews, who alone knew the way how to magnify God aright, did commonly, as appeared by their wicked lives, more of custom and for fashion sake execute the services of their religion, than with hearty and true devotion (which God especially requireth) he therefore protesteth against their Sabboths and solemn days as being therewith much offended.

[3.]Plentiful and liberal expense is required in them that abound, partly as a sign of their own joy in the goodness of God towards them, and partly as a mean whereby to refresh those poor and needy, who being especially at these times made partakers of relaxation and joy with others do the more religiously bless God, whose great mercies were a cause thereof, and the more contentedly endure the burden of that hard estate wherein they continue.

[4.]Rest is the end of all motion, and the last perfection of all things that labour. Labours in us are journeys, and even in them which feel no weariness by any work, yet they are but  ways whereby to come unto that which bringeth not happiness till it do bring rest.
 For as long as any thing which we desire is unattained, we rest not.

Let us not here take rest for idleness. They are idle whom the painfulness of action causeth to avoid those labours, whereunto both God and nature bindeth them: they rest which either cease from their work when they have brought it unto perfection, or else give over a meaner labour because a worthier and better is to be undertaken. God hath created nothing to be idle or ill employed.

As therefore man doth consist of different and distinct parts, every part endued with manifold abilities which all have their several ends and actions thereunto referred; so there is in this great variety of duties which belong to men that dependency and order, by means whereof the lower sustaining always the more excellent, and the higher perfecting the more base, they are in their times and seasons continued with most exquisite correspondence; labours of bodily and daily toil purchase freedom for actions of religious joy, which benefit these actions requite with the gift of desired rest: a thing most natural and fit to accompany the solemn festival duties of honour which are done to God.

For if those principal works of God, the memory whereof we use to celebrate at such times, be but certain tastes and says as it were of that final benefit, wherein our perfect felicity and bliss lieth folded up, seeing that the presence of the one doth direct our cogitations, thoughts, and desires towards the other, it giveth surely a kind of life and addeth inwardly no small delight to those so comfortable expectations, when the very outward countenance of that we presently do representeth after a sort that also whereunto we tend, as festival rest doth that celestial estate whereof the very heathens themselves which had not the means whereby to apprehend much did notwithstanding imagine that it needs must consist in rest, and have therefore taught that above the highest moveable sphere there is nothing which feeleth  alteration, motion, or change,
 but all things immutable, unsubject to passion, blest with eternal continuance in a life of the highest perfection and of that complete abundant sufficiency within itself, which no possibility of want, maim, or defect can touch. Besides whereas ordinary labours are both in themselves painful, and base in comparison of festival services done to God, doth not the natural difference between them shew that the one as it were by way of submission and homage should surrender themselves to the other, wherewith they can neither easily concur, because painfulness and joy are opposite, nor decently, because while the mind hath just occasion to make her abode in the house of gladness, the weed of ordinary toil and travail becometh her not?

[5.]Wherefore even nature hath taught the heathens, and God the Jews, and Christ us, first that festival solemnities are a part of the public exercise of religion; secondly that praise, liberality and rest are as natural elements whereof solemnities consist. But these things the heathens converted to the honour of their false gods, and as they failed in the end itself, so neither could they discern rightly what form and measure religion therein should observe. Whereupon when the Israelites impiously followed so corrupt example, they are in every degree noted to have done amiss, their hymns or songs of praise were idolatry, their bounty excess, and their rest wantonness. Therefore the law of God which appointed them days of solemnity taught them likewise in what manner the same should be celebrated. According to the pattern of which institution, David establishing the state of religion ordained praise to be given unto God in the Sabboths, months, and appointed times, as their custom had been always before the Lord.

[6.]Now besides the times which God himself in the Law of Moyses particularly specifieth, there were through the wisdom of the Church certain other devised by occasion of like occurrents to those whereupon the former had risen, as namely that which Mardocheus and Hester did first celebrate in memory of the Lord’s most wonderful protection, when Haman had laid his inevitable plot to man’s thinking for the utter extirpation of the Jews even in one day. This they  call the feast of Lots, because Haman had cast their life and their death as it were upon the hazard of a Lot.
 To this may be added that other also of Dedication mentioned in the tenth of St. John’s Gospel, the institution whereof is declared in the history of the Maccabees.

[7.]But forasmuch as their law by the coming of Christ is changed, and we thereunto no way bound, St. Paul although it were not his purpose to favour invectives against the special sanctification of days and times to the service of God and to the honour of Jesus Christ, doth notwithstanding bend his forces against that opinion which imposed on the Gentiles the yoke of Jewish legal observations, as if the whole world ought for ever and that upon pain of condemnation to keep and observe the same. Such as in this persuasion hallowed those Jewish Sabboths, the Apostle sharply reproveth saying, “Ye observe days and months and times and years, I am in fear of you lest I have bestowed upon you labour in vain.” Howbeit so far off was Tertullian from imagining how any man could possibly hereupon call in question such days as the Church of Christ doth observe, that the observation of these days he useth for an argument whereby to prove it could not be the Apostle’s intent and meaning to condemn simply all observing of such times.

[8.]Generally therefore touching feasts in the Church of Christ, they have that profitable use whereof St. Augustine speaketh, “By festival solemnities and set days we dedicate and sanctify to God the memory of his benefits, lest unthankful forgetfulness thereof should creep upon us in course of time.”

And concerning particulars, their Sabboth the Church hath changed into our Lord’s day, that as the one did continually bring to mind the former world finished by creation, so the  other might keep us in perpetual remembrance of a far better world begun by him which came to restore all things,
 to make both heaven and earth new. For which cause they honoured the last day, we the first, in every seven throughout the year.

The rest of the days and times which we celebrate have relation all unto one head. We begin therefore our ecclesiastical year with the glorious Annunciation of his birth by angelical embassage. There being hereunto added his blessed Nativity itself, the mystery of his legal Circumcision the testification of his true incarnation by the Purification of her which brought him into the world, his Resurrection, his Ascension into heaven, the admirable sending down of his Spirit upon his chosen, and (which consequently ensued) the notice of that incomprehensible Trinity thereby given to the Church of God; again forasmuch as we know that Christ hath not only been manifested great in himself, but great in other his Saints also, the days of whose departure out of the world are to the Church of Christ as the birth and coronation days of kings or emperors, therefore especial choice being made of the very flower of all occasions in this kind, there are annual selected times to meditate of Christ glorified in them which had the honour to suffer for his sake, before they had age and ability to know him; glorified in them which knowing him as Stephen, had the sight of that before death whereinto so acceptable death did lead; glorified in those sages of the East that came from far to adore him and were conducted by strange light; glorified in the second Elias of the world sent before him to prepare his way; glorified in every of those Apostles whom it pleased him to use as founders of his kingdom here; glorified in the Angels as in Michael; glorified in all those happy souls that are already possessed of heaven. Over and besides which number not great, the rest be but four other days heretofore annexed to the feast of Easter and Pentecost by reason of general Baptism usual at those two feasts, which also is the cause why they had not as other days any proper name given them. Their first institution was therefore through necessity, and their  present continuance is now for the greater honour of the principals whereupon they still attend.

[9.]If it be then demanded whether we observe these times as being thereunto bound by force of divine law, or else by the only positive ordinances of the Church, I answer to this, that the very law of nature itself, which all men confess to be God’s law, requireth in general no less the sanctification of times, than of places, persons, and things unto God’s honour. For which cause it hath pleased him heretofore, as of the rest so of time likewise, to exact some parts by way of perpetual homage, never to be dispensed withal nor remitted; again to require some other parts of time with as strict exaction but for less continuance; and of the rest which were left arbitrary to accept what the Church shall in due consideration consecrate voluntarily unto like religious uses. Of the first kind amongst the Jews was the Sabboth day; of the second those feasts which are appointed by the law of Moyses; the feast of dedication invented by the Church standeth in the number of the last kind.

The moral law requiring therefore a seventh part throughout the age of the whole world to be that way employed, although with us the day be changed in regard of a new revolution begun by our Saviour Christ, yet the same proportion of time continueth which was before, because in reference to the benefit of creation and now much more of renovation thereunto added by him which was Prince of the world to come, we are bound to account the sanctification of one day in seven a duty which God’s immutable law doth exact for ever. The rest they say we ought to abolish, because the continuance of them doth nourish wicked superstition in the minds of men;  besides they are all abused by Papists the enemies of God, yea certain of them as Easter and Pentecost even by the Jews.


\section*{Exceptions against our keeping of other festival days besides the Sabbath.}
LXXI. Touching Jews, their Easter and Pentecost have with ours as much affinity, as Philip the Apostle with Philip the Macedonian king. As for “imitation of Papists” and the “breeding of superstition,” they are now become such common guests that no man can think it discourteous to let them go as they came. The next is a rare observation and strange. You shall find if you mark it (as it doth deserve to be noted well) that many thousands there are who if they have virtuously during those times behaved themselves, if their devotion and zeal in prayer have been fervent, their attention to the word of God such as all Christian men should yield, imagine that herein they have performed a good duty; which notwithstanding to think is a very dangerous error, inasmuch as the Apostle St. Paul hath taught that we ought  not to keep our Easter as the Jews did for certain days,
 but in the unleavened bread of sincerity and of truth to feast continually, whereas this restraint of Easter to a certain number of days causeth us to rest for a short space in that near consideration of our duties which should be extended throughout the course of our whole lives, and so pulleth out of our minds the doctrine of Christ’s gospel ere we be aware.

[2.]The doctrine of the gospel which here they mean or should mean is, that Christ having finished the law there is no Jewish paschal solemnity nor abstinence from sour bread now required at our hands, there is no leaven which we are bound to cast out but malice, sin, and wickedness, no bread but the food of sincere truth wherewith we are tied to celebrate our passover. And seeing no time of sin is granted us, neither any intermission of sound belief, it followeth that this kind of feasting ought to endure always. But how are standing festival solemnities against this?

That which the gospel of Christ requireth is the perpetuity of virtuous duties; not perpetuity of exercise or action, but disposition perpetual, and practice as oft as times and opportunities require. Just, valiant, liberal, temperate and holy men are they which can whensoever they will, and will whensoever  they ought, execute what their several perfections import. If virtues did always cease to be when they cease to work, there should be nothing more pernicious to virtue than sleep: neither were it possible that men as Zachary and Elizabeth should in all the commandments of God walk unreprovable, or that the chain of our conversation should contain so many links of divine virtues as the Apostles in divers places have reckoned up, if in the exercise of each virtue perpetual continuance were exacted at our hands. Seeing therefore all things are done in time, and many offices are not possible at one and the same time to be discharged, duties of all sorts must have necessarily their several successions and seasons, in which respect the schoolmen have well and soundly determined that God’s affirmative laws and precepts, the laws that enjoin any actual duty, as prayer, alms, and the like, do bind us ad semper velle, but not ad semper agere; we are tied to iterate and resume them when need is, howbeit not to continue them without any intermission. Feasts whether God himself hath ordained them, or the Church by that authority which God hath given, they are of religion such public services as neither can nor ought to be continued otherwise than only by iteration.

Which iteration is a most effectual mean to bring unto full maturity and growth those seeds of godliness that these very men themselves do grant to be sown in the hearts of many thousands, during the while that such feasts are present. The constant habit of well doing is not gotten without the custom of doing well, neither can virtue be made perfect but by the manifold works of virtue often practised. Before the powers of our minds be brought unto some perfection our first assays and offers towards virtue must needs be raw, yet commendable because they tend unto ripeness. For which cause the wisdom of God hath commended especially this circumstance amongst others in solemn feasts, that to children and novices in religion they minister the first  occasions to ask and inquire of God.
 Whereupon if there follow but so much piety as hath been mentioned, let the Church learn to further imbecility with prayer, “Preserve Lord these good and gracious beginnings that they suddenly dry not up like the morning dew, but may prosper and grow as the trees which rivers of waters keep always flourishing;” let all men’s acclamations be “Grace, grace unto it,” as to that first-laid corner-stone in Zerubbabel’s buildings. For who hath despised the day of those things which are small? Or how dare we take upon us to condemn that very thing which voluntarily we grant maketh us of nothing somewhat, seeing all we pretend against it is only that as yet this somewhat is not much? The days of solemnity which are but few cannot choose but soon finish that outward exercise of godliness which properly appertaineth to such times, howbeit men’s inward disposition to virtue they both augment for the present, and by their often returns bring also the same at the length unto that perfection which we most desire. So that although by their necessary short continuance they abridge the present exercise of piety in some kind, yet because by repetition they enlarge, strengthen and confirm the habits of all virtue, it remaineth that we honour, observe and keep them as ordinances many ways singularly profitable in God’s Church.

[3.]This exception being taken against holidays, for that they restrain the praises of God unto certain times, another followeth condemning restraint of men from their ordinary trades and labours at those times. It is not they say in the power of the Church to command rest, because God hath  left it to all men at liberty that if they think good to bestow six whole days in labour they may,
 neither is it more lawful for the Church to abridge any man of that liberty which God hath granted, than to take away the yoke which God hath laid upon them and to countermand what he doth expressly enjoin. They deny not but in times of public calamity, that men may the better assemble themselves to fast and pray, the Church “because it hath received commandment” from God to proclaim a prohibition from ordinary works, standeth bound to do it, as the Jews afflicted did in Babylon. But without some express commandment from God there is no power they say under heaven which may presume by any decree to restrain the liberty that God hath given.

[4.]Which opinion, albeit applied here no further than to this present cause, shaketh universally the fabric of government, tendeth to anarchy and mere confusion, dissolveth families, dissipatech colleges, corporations, armies, overthroweth  kingdoms, churches, and whatsoever is now through the providence of God by authority and power upheld. For whereas God hath foreprized things of the greatest weight, and hath therein precisely defined as well that which every man must perform, as that which no man may attempt, leaving all sorts of men in the rest either to be guided by their own good discretion if they be free from subjection to others, or else to be ordered by such commandments and laws as proceed from those superiors under whom they live; the patrons of liberty have here made solemn proclamation that all such laws and commandments are void, inasmuch as every man is left to the freedom of his own mind in such things as are not either exacted or prohibited by the Law of God; and because only in these things the positive precepts of men have place, which precepts cannot possibly be given without some abridgment of their liberty to whom they are given, therefore if the father command the son, or the husband the wife, or the lord the servant, or the leader the soldier, or the prince the subject to go or stand, sleep or wake at such times as God himself in particular commandeth neither, they are to stand in defence of the freedom which God hath granted and to do as themselves list, knowing that men may as lawfully command them things utterly forbidden by the law of God, as tie them to any thing which the law of God leaveth free. The plain contradictory whereunto is infallibly certain. Those  things which the law of God leaveth arbitrary and at liberty are all subject unto positive laws of men,
 which laws for the common benefit abridge particular men’s liberty in such things as far as the rules of equity will suffer. This we must either maintain, or else overturn the world and make every man his own commander. Seeing then that labour and rest upon any one day of the six throughout the year are granted free by the Law of God, how exempt we them from the force and power of ecclesiastical law, except we deprive the world of power to make any ordinance or law at all?

[5.]Besides is it probable that God should not only allow but command concurrency of rest with extraordinary occasions of doleful events befalling peradventure some one certain church, or not extending unto many, and not as much as permit or license the like, when piety triumphant with joy and gladness maketh solemn commemoration of God’s most rare and unwonted mercies, such especially as the whole race of mankind doth or might participate? Of vacation from labour in times of sorrow the only cause is for that the general public prayers of the whole Church and our own private businesses cannot both be followed at once: whereas of rest in the famous solemnities of public joy there is both this consideration the same, and also farther a kind of natural repugnancy, which maketh labours (as hath been proved) much more unfit to accompany festival praises of God than offices of humiliation and grief.

Again if we sift what they bring for proof and approbation of rest with fasting, doth it not in all respects as fully warrant and as strictly command rest, whensoever the Church hath equal reason by feasts and gladsome solemnities to testify public thankfulness towards God? I would know some cause, why those words of the prophet Joel, “Sanctify a fast, call a solemn assembly,” which words were uttered to the Jews in misery and great distress, should more bind the Church to do at all times after the like in their like perplexities, than the words of Moyses to the same people in a time of joyful deliverance from misery, “Remember this day,” may warrant any annual celebration of benefits no less importing  the good of men;
 and also justify, as touching the manner and form thereof, what circumstance soever we imitate only in respect of natural fitness or decency, without any Jewish regard to ceremonies such as were properly theirs and are not by us expedient to be continued.

According to the rule of which general directions, taken from the law of God no less in the one than the other, the practice of the Church commended unto us in holy Scripture doth not only make for the justification of black and dismal days (as one of the Fathers termeth them) but plainly offereth itself to be followed by such ordinances (if occasion require) as that which Mardocheus did sometime devise, Hester what lay in her power help forward, and the rest of the Jews establish for perpetuity, namely that the fourteenth and fifteenth days of the month Adar should be every year kept throughout all generations as days of feasting and joy, wherein they would rest from bodily labour, and what by gifts of charity bestowed upon the poor, what by other liberal signs of amity and love, all testify their thankful minds towards God, which almost beyond possibility had delivered them all when they all were as men dead.

[6.]But this decree they say was divine not ecclesiastical, as may appear in that there is another decree in another book of Scripture which decree is plain not to have proceeded from the Church’s authority but from the mouth of the prophet only; and as a poor simple man sometime was fully persuaded that if Pontius Pilate had not been a saint the Apostles would never have suffered his name to stand in the Creed, so  these men have a strong opinion that because the book of Hester is canonical the decree of Hester cannot be possibly ecclesiastical. If it were, they ask how the Jews could bind themselves always to keep it, seeing ecclesiastical laws are mutable? As though the purposes of men might never intend constancy in that the nature whereof is subject to alteration. Doth the Scripture itself make mention of any divine commandment? Is the Scripture witness of more than only that Mardocheus was the author of this custom, that by letters written to his brethren the Jews throughout all provinces under Darius the king of Persia he gave them charge to celebrate yearly those two days for perpetual remembrance of God’s miraculous deliverance and mercy, that the Jews hereupon undertook to do it, and made it with general consent an order for perpetuity, that Hester secondly by her letters confirmed the same which Mardocheus had before decreed, and that finally the ordinance was written to remain for ever upon record? Did not the Jews in provinces abroad observe at the first the fourteenth day, the Jews in Susis the fifteenth? Were they not all reduced to a uniform order by means of those two decrees, and so every where three days kept, the first with fasting in memory of danger, the rest in token of deliverance as festival and joyful days? Was not the first of these three afterwards, the day of sorrow and heaviness, abrogated, when the same Church saw it meet that a better day, a day in memory of like deliverance out of the bloody hands of Nicanor, should succeed in the room thereof?

[7.]But forasmuch as there is no end of answering fruitless oppositions, let it suffice men of sober minds to know that the law both of God and nature alloweth generally days of rest  and festival solemnity to be observed by way of thankful and joyful remembrance,
 if such miraculous favours be shewed towards mankind as require the same; that such graces God hath bestowed upon his Church as well in later as in former times; that in some particulars when they have fallen out himself hath demanded his own honour, and in the rest hath left it to the wisdom of the Church directed by those precedents and enlightened by other means always to judge when the like is requisite. About questions therefore concerning days and times our manner is not to stand at bay with the Church of God demanding wherefore the memory of Paul should be rather kept than the memory of Daniel, we are content to imagine it may be perhaps true that the least in the kingdom of Christ is greater than the greatest of all the prophets of God that have gone before; we never yet saw cause to despair but that the simplest of the people might  be taught the right construction of as great mysteries as the name of a saint’s day doth comprehend, although the times of the year go on in their wonted course; we had rather glorify and bless God for the fruit we daily behold reaped by such ordinances as his gracious Spirit maketh the ripe wisdom of this national church to bring forth, than vainly boast of our own peculiar and private inventions, as if the skill of profitable regiment had left her public habitation to dwell in retired manner with some few men of one livery; we make not our childish appeals sometimes from our own to foreign  churches,
 sometime from both unto churches ancienter than both are, in effect always from all others to our own selves, but as becometh them that follow with all humility the ways of peace, we honour, reverence, and obey in the very next degree unto God the voice of the church of God wherein we live. They whose wits are too glorious to fall to so low an ebb, they which have risen and swollen so high that the walls of ordinary rivers are unable to keep them in, they whose wanton contentions in the cause whereof we have spoken do make all where they go a sea, even they at their highest float are constrained both to see and grant, that what their fancy will not yield to like their judgment cannot with reason condemn. Such is evermore the final victory of all truth, that they which have not the hearts to love her acknowledge that to hate her they have no cause.

[8.]Touching those festival days therefore which we now observe, their number being no way felt discommodious to the commonwealth, and their grounds such as hitherto hath been shewed; what remaineth but to keep them throughout all generations holy, severed by manifest notes of difference from other times, adorned with that which most may betoken true virtuous and celestial joy? To which intent because surcease from labour is necessary, yet not so necessary no not on the Sabboth or seventh day itself, but that rarer occasions in men’s particular affairs, subject to manifest detriment unless they be presently followed, may with very good conscience draw them sometimes aside from the ordinary rule, considering the favourable dispensation which our Lord and Saviour  groundeth on this axiom,
 “Man was not made for the Sabboth but the Sabboth ordained for man,” so far forth as concerneth ceremonies annexed to the principal sanctification thereof, howsoever the rigour of the law of Moyses may be thought to import the contrary, if we regard with what severity the violation of Sabboths hath been sometime punished, a thing perhaps the more requisite at that instant, both because the Jews by reason of their long abode in a place of continual servile toil could not suddenly be weaned and drawn unto contrary offices without some strong impression of terror, and also for that there is nothing more needful than to punish with extremity the first transgressions of those laws that require a more exact observation for many ages to come; therefore as the Jews superstitiously addicted to their Sabboths’ rest for a long time, not without danger to themselves and obloquy to their very law, did afterwards perceive and amend wisely their former error, not doubting that bodily labours are made by necessity venial, though otherwise, especially on that day, rest be more convenient; so at all times the voluntary scandalous contempt of that rest from labour wherewith publicly God is served we cannot too severely correct and bridle.

[9.]The emperor Constantine having with overgreat facility licensed Sundays’ labours in country villages, under that pretence whereof there may justly no doubt sometime consideration be had, namely lest any thing which God by his providence hath bestowed should miscarry not being taken in due  time; Leo which afterwards saw that this ground would not bear so general and large indulgence as had been granted, doth by a contrary edict both reverse and severely censure his predecessor’s remissness, saying, “We ordain according to the true meaning of the Holy Ghost and of the Apostles thereby directed, that on the sacred day wherein our own integrity was restored all do rest and surcease labour, that neither husbandman nor other on that day put their hands to forbidden works. For if the Jews did so much reverence their Sabboth which was but a shadow of ours, are not we which inhabit the light and truth of grace bound to honour that day which the Lord himself hath honoured and hath therein delivered us both from dishonour and from death? are we not bound to keep it singular and inviolable, well contenting ourselves with so liberal a grant of the rest, and not encroaching upon that one which God hath chosen to his own honour? Were it not rechless neglect of religion to make that very day common and to think we may do with it as with the rest?”

Imperial laws which had such care of hallowing especially our Lord’s day did not omit to provide that other festival  times might be kept with vacation from labour,
 whether they were days appointed on the sudden as extraordinary occasions fell out, or days which were celebrated yearly for politic and civil considerations, or finally such days as Christian religion hath ordained in God’s Church.

[10.]The joy that setteth aside labour disperseth those things which labour gathereth. For gladness doth always rise from a kind of fruition and happiness, which happiness banisheth the cogitation of all want, it needeth nothing but only the bestowing of that it hath, inasmuch as the greatest felicity that felicity hath is to spread and enlarge itself; it cometh hereby to pass that the first effect of joyfulness is to rest, because it seeketh no more; the next, because it aboundeth, to give. The root of both is the glorious presence of that joy of mind which riseth from the manifold considerations of God’s unspeakable mercy, into which considerations we are led by occasion of sacred times.

[11.]For how could the Jewish congregations of old be put in mind by their weekly Sabboths what the world reaped through his goodness which did of nothing create the world; by their yearly Passover what farewell they took of the land of Egypt; by their Pentecost what ordinances, laws, and statutes their fathers received at the hands of God; by their feast of Tabernacles with what protection they journeyed from place to place through so many fears and hazards during the tedious time of forty years’ travail in the wilderness; by their annual solemnity of Lots, how near the whole seed of Israel was unto utter extirpation, when it pleased that great God which guideth all things in heaven and earth so to change the counsels and purposes of men, that the same hand which had signed a decree in the opinion both of them that granted and of them that procured it irrevocable, for the general massacre of man, woman, and child, became the buckler of their preservation that no one hair of their heads might be touched, the same days which had been set for the pouring out of so much innocent blood were made the days of their execution  whose malice had contrived the plot thereof, and the selfsame persons that should have endured whatsoever violence and rage could offer were employed in the just revenge of cruelty to give unto bloodthirsty men the taste of their own cup;
 or how can the Church of Christ now endure to be so much called on and preached unto by that which every dominical day throughout the year, that which year by year so many festival times, if not commanded by the Apostles themselves whose care at that time was of greater things, yet instituted either by such universal authority as no man, or at the least such as we with no reason may despise, do as sometime the holy angels did from heaven sing, “Glory be unto God on high, peace on earth, towards men good-will,” (for this in effect is the very song that all Christian feasts do apply as their several occasions require,) how should the days and times continually thus inculcate what God hath done, and we refuse to agnize the benefit of such remembrances, that very benefit which caused Moyses to acknowledge those guides of day and night, the sun and moon which enlighten the world, not more profitable to nature by giving all things life, than they are to the Church of God by occasion of the use they have in regard of the appointed festival times? That which the head of all philosophers hath said of women, “If they be good the half of the commonwealth is happy wherein they are,” the same we may fitly apply to times; well to celebrate these religious and sacred days is to spend the flower of our time happily. They are the splendour and outward dignity of our religion, forcible witnesses of ancient truth,  provocations to the exercise of all piety,
 shadows of our endless felicity in heaven, on earth everlasting records and memorials, wherein they which cannot be drawn to hearken unto that we teach, may only by looking upon that we do, in a manner read whatsoever we believe.


\section*{Of days appointed as well for ordinary as for extraordinary Fasts in the Church of God.}
LXXII. The matching of contrary things together is a kind of illustration to both. Having therefore spoken thus much of festival days, the next that offer themselves to hand are days of pensive humiliation and sorrow. Fastings are either of men’s own free and voluntary accord as their particular devotion doth move them thereunto; or else they are publicly enjoined in the Church and required at the hands of all men. There are which altogether disallow not the  former kind, and the latter they greatly commend, so that it be upon extraordinary occasions only,
 and after one certain manner exercised. But yearly or weekly fasts such as ours in the Church of England they allow no farther than as the temporal state of the land doth require the same for the maintenance of seafaring men and preservation of cattle, because the decay of the one and the waste of the other could not well be prevented but by a politic order appointing some such usual change of diet as ours is.

We are therefore the rather to make it manifest in all men’s eyes, that set times of fasting appointed in spiritual considerations to be kept by all sorts of men took not their beginning either from Montanus or any other whose heresies may prejudice the credit and due estimation thereof, but have their ground in the law of nature, are allowable in God’s sight, were in all ages heretofore, and may till the world’s end be observed not without singular use and benefit.

[2.]Much hurt hath grown to the Church of God through a false imagination that fasting standeth men in no stead for any spiritual respect, but only to take down the frankness of nature and to tame the wildness of flesh. Whereupon the world being bold to surfeit doth now blush to fast, supposing that men when they fast, do rather bewray a disease, than exercise a virtue. I much wonder what they who are thus persuaded do think, what conceit they have concerning the fasts of the Patriarchs, the Prophets, the Apostles, our Lord Jesus Christ himself.

The affections of Joy and Grief are so knit unto all the actions of man’s life, that whatsoever we can do or may be done unto us, the sequel thereof is continually the one or the other affection. Wherefore considering that they which  grieve and joy as they ought cannot possibly otherwise live than as they should, the Church of Christ, the most absolute and perfect school of all virtue, hath by the special direction of God’s good Spirit hitherto always inured men from their infancy partly with days of festival exercise for the framing of the one affection, and partly with times of a contrary sort for the perfecting of the other. Howbeit over and besides this, we must note that as resting so fasting likewise attendeth sometimes no less upon the actions of the higher, than upon the affections of the lower part of the mind. Fasting (saith Tertullian) is a work of reverence towards God. The end thereof sometimes elevation of mind; sometime the purpose thereof clean contrary. The cause why Moyses in the Mount did so long fast was mere divine speculation, the cause why David, humiliation. Our life is a mixture of good with evil. When we are partakers of good things we joy, neither can we but grieve at the contrary. If that befall us which maketh glad, our festival solemnities declare our rejoicing to be in him whose mere undeserved mercy is the author of all happiness; if any thing be either imminent or present which we shun, our watchings, fastings, cries and tears are unfeigned testimonies, that ourselves we condemn as the only causes of our own misery, and do all acknowledge him no less inclinable than able to save. And because as the memory of the one  though past reneweth gladness;
 so the other called again to mind doth make the wound of our just remorse to bleed anew, which wound needeth often touching the more, for that we are generally more apt to calendar saints’ than sinners’ days, therefore there is in the Church a care not to iterate the one alone but to have frequent repetition of the other.

Never to seek after God saving only when either the crib or the whip doth constrain were brutish servility: and a great derogation to the worth of that which is most predominant in man, if sometime it had not a kind of voluntary access to God and of conference as it were with God, all these inferior considerations laid aside. In which sequestration forasmuch as higher cogitations do naturally drown and bury all inferior cares, the mind may as well forget natural both food and sleep by being carried above itself with serious and heavenly meditation, as by being cast down with heaviness, drowned and swallowed up of sorrow.

[3.]Albeit therefore concerning Jewish abstinence from certain kinds of meats as being unclean the Apostle doth teach that “the kingdom of heaven is not meat nor drink,” that “food commendeth us not unto God” whether we take it or abstain from it, that if we eat we are not thereby the more acceptable in his sight, nor the less if we eat not; his purpose notwithstanding was far from any intent to derogate from that fasting, which is no such scrupulous abstinence as only refuseth some kinds of meats and drinks lest they make him unclean that tasteth them, but an abstinence whereby we either interrupt or otherwise abridge the care of our bodily sustenance, to show by this kind of outward exercise the serious intention of our minds fixed on heavenlier and better desires, the earnest hunger and thirst whereof depriveth the body of those usual contentments, which otherwise are not denied unto it.

[4.]These being in nature the first causes that induce fasting, the next thing which followeth to be considered is the ancient practice thereof amongst the Jews. Touching whose private voluntary fasts the precept which our Saviour gave them was, “When ye fast look not sour as hypocrites.  For they disfigure their faces that they might seem to men to fast.
 Verily I say unto you, they have their reward. When thou fastest, anoint thy head, and wash thy face, that thou seem not unto men to fast, but unto the Father which is in secret, and thy Father which seeth in secret will reward thee openly.” Our Lord and Saviour would not teach the manner of doing, much less propose a reward for doing, that which were not both holy and acceptable in God’s sight. The Pharisees weekly bound themselves unto double fasts, neither are they for this reproved. Often fasting which was a virtue in John’s disciples could not in them of itself be a vice, and therefore not the oftness of their fasting but their hypocrisy therein was blamed.

[5.]Of public enjoined fasts upon causes extraordinary the examples in Scripture are so frequent that they need no particular rehearsal. Public extraordinary fastings were sometimes for one only day, sometimes for three, sometimes for seven. Touching fasts not appointed for any such extraordinary causes, but either yearly or monthly or weekly observed and kept, first upon the ninth day of that month the tenth whereof was the feast of expiation, they were commanded of God that every soul year by year should afflict itself. Their yearly fasts every fourth month in regard of the city of Jerusalem entered by the enemy, every fifth in memory of the overthrow of their temple, every seventh for the treacherous destruction and death of Godolias the very last stay which they had to lean unto in their greatest misery, every tenth in remembrance of the time when siege began first to be laid against them; all these not commanded of God himself but ordained by a public constitution of their own, the Prophet Zachary expressly toucheth. That St. Jerome  following the tradition of the Hebrews doth make the first a memorial of the breaking of those two tables when Moyses descended from Mount Sina;
 the second a memorial as well of God’s indignation condemning them to forty years’ travail in the desert, as of his wrath in permitting Chaldeans to waste, burn and destroy their city; the last a memorial of heavy tidings brought out of Jewry to Ezechiel and the rest which lived as captives in foreign parts, the difference is not of any moment, considering that each time of sorrow is naturally evermore a register of all such grievous events is have happened either in or near about the same time. To these I might add sundry other fasts above twenty in number ordained amongst them by like occasions and observed in like manner, besides their weekly abstinence Mondays and Thursdays throughout the whole year.

[6.]When men fasted it was not always after one and the same sort, but either by depriving themselves wholly of all  food during the time that their fasts continued, or by abating both the quantity and kind of diet.
 We have of the one a plain example in the Ninevites’ fasting, and as plain a precedent for the other in the Prophet Daniel, “I was,” saith he, “in heaviness for three weeks of days; I ate no pleasant bread, neither tasted flesh nor wine.” Their tables when they gave themselves to fasting had not that usual furniture of such dishes as do cherish blood with blood, but for food they had bread, for suppage salt, and for sauce herbs. Whereunto the Apostle may be thought to allude saying, “One believeth he may eat all things, another which is weak” (and maketh a conscience of keeping those customs which the Jews observe) “eateth herbs.” This austere repast they took in the evening after abstinence the whole day. For to forfeit a noon’s meal and then to recompense themselves at night was not their use. Nor did they ever accustom themselves on Sabboths or festival days to fast.

[7.]And yet it may be a question whether in some sort they did not always fast the Sabboth. Their fastings were partly in token of penitency, humiliation, grief and sorrow, partly in sign of devotion and reverence towards God. Which second consideration (I dare not peremptorily and boldly affirm any thing) might induce to abstain till noon, as their manner was on fasting days to do till night. May it not very well be thought that hereunto the sacred Scripture doth give some secret kind of testimony? Josephus is plain,  that the sixth hour (the day they divided into twelve) was wont on the Sabboth always to call them home unto meat.
 Neither is it improbable but that the heathens did therefore so often upbraid them with fasting on that day. Besides they which found so great fault with our Lord’s disciples, for rubbing a few ears of corn in their hands on the Sabboth day, are not unlikely to have aimed also at the same mark. For neither was the bodily pain so great that it should offend them in that respect, and the very manner of defence which our Saviour there useth is more direct and literal to justify the breach of the Jewish custom in fasting than in working at that time. Finally the Apostles afterwards themselves when God first gave them the gift of tongues, whereas some in disdain and spite termed grace drunkenness, it being then the day of Pentecost and but only a fourth part of the day spent, they use this as an argument against the other cavil, “These men,” saith Peter, “are not drunk as you suppose, since as yet the third hour of the day is not overpast.”

[8.]Howbeit leaving this in suspense as a thing not altogether certainly known, and to come from Jews to Christians, we find that of private voluntary fastings the Apostle St. Paul speaketh more than once. And (saith Tertullian) they are sometime commanded throughout the Church “ex aliqua sollicitudinis ecclesiasticæ causa,” the care and fear of the Church so requiring. It doth not appear that the Apostles ordained any set and certain days to be generally kept of all. Notwithstanding, forasmuch as Christ had foresignified that when himself should be taken from them his absence would soon make them apt to fast, it seemeth that even as the first festival day appointed to be kept of the Church was the day of our Lord’s return from the dead, so the first sorrowful and mourning day was that which we now observe in memory of his departure out of this world. And because there could be  no abatement of grief, till they saw him raised whose death was the occasion of their heaviness, therefore the day he lay in the sepulchre hath been also kept and observed as a weeping day. The custom of fasting these two days before Easter is undoubtedly most ancient, insomuch that Ignatius not thinking him a Catholic Christian man which did not abhor and (as the state of the Church was then) avoid fasting on the Jews’ Sabboth, doth notwithstanding except for ever that one Sabboth or Saturday which falleth out to be the Easter-eve, as with us it always doth and did sometimes also with them which kept at that time their Easter the fourteenth day of March as the custom of the Jews was. It came afterwards to be an order that even as the day of Christ’s resurrection, so the other two in memory of his death and burial were weekly. But this when St. Ambrose lived had not as yet taken place throughout all churches, no not in Milan where himself was bishop. And for that cause he saith that although at Rome he observed the Saturday’s fast, because such was then the custom in Rome, nevertheless in his own church at home he did otherwise. The churches  which did not observe that day had another instead thereof, which was the Wednesday,
 for that when they judged it meet to have weekly a day of humiliation besides that whereon our Saviour suffered death, it seemed best to make their choice of that day especially whereon the Jews are thought to have first contrived their treason together with Judas against Christ. So that the instituting and ordaining both of these and of all other times of like exercise is as the Church shall judge expedient for men’s good.

[9.]And concerning every Christian man’s duty herein, surely that which Augustine and Ambrose are before alleged to have done, is such as all men favouring equity must needs allow, and follow if they affect peace. As for their specified errors, I will not in this place dispute whether voluntary fasting with a virtuous purpose of mind be any medicinable remedy of evil, or a duty acceptable unto God and in the world to come even rewardable as other offices are which proceed from Christian piety; whether wilfully to break and despise the wholesome laws of the Church herein be a thing which offendeth God; whether truly it may not be said that penitent both weeping and fasting are means to blot out sin, means whereby through God’s unspeakable and undeserved mercy we obtain or procure to ourselves pardon,  which attainment unto any gracious benefit by him bestowed the phrase of antiquity useth to express by the name of merit;
 but if either St. Augustine or St. Ambrose have taught any wrong opinion, seeing they which reprove them are not altogether free from error, I hope they will think it no error in us so to censure men’s smaller faults that their virtues be not thereby generally prejudiced. And if in churches abroad, where we are not subject to power or jurisdiction, discretion should teach us for peace and quietness’ sake to frame ourselves to other men’s example, is it meet that at home where our freedom is less our boldness should be more? Is it our duty to oppugn, in the churches whereof we are ministers, the rites and customs which in foreign churches piety and modesty did teach us as strangers not to oppugn, but to keep without shew of contradiction or dislike? Why oppose they the name of a minister in this case unto the state of a private man? Doth their order exempt them from obedience to laws? That which their office and place requireth is to show themselves patterns of reverend subjection, not authors and masters of contempt towards ordinances, the strength whereof when they seek to weaken they do but in truth discover to the world their own imbecilities, which a great deal wiselier they might conceal.

[10.]But the practice of the Church of Christ we shall by so much the better both understand and love, if to that which hitherto hath been spoken there be somewhat added for more particular declaration how heretics have partly abused fasts and partly bent themselves against the lawful use thereof in the Church of God. Whereas therefore Ignatius hath said, “if any keep Sundays’ or Saturdays’ fast (one only Saturday in the year excepted) that man is no better than a murderer of Christ,” the cause of such his earnestness at that time was the impiety of certain heretics, which thought  that this world being corruptible could not be made but by a very evil author.
 And therefore as the Jews did by the festival solemnity of their Sabboth rejoice in the God that created the world as in the author of all goodness, so those heretics in hatred of the Maker of the world sorrowed, wept, and fasted on that day as being the birthday of all evil.

And as Christian men of sound belief did solemnize the Sunday, in joyful memory of Christ’s resurrection, so likewise at the selfsame time such heretics as denied his resurrection did the contrary to them which held it, when the one sort rejoiced the other fasted.

Against those heretics which have urged perpetual abstinence from certain meats as being in their very nature unclean, the Church hath still bent herself as an enemy; St. Paul giving charge to take heed of them which under any such opinion should utterly forbid the use of meats or drinks. The Apostles themselves forbade some, as the order taken at Jerusalem declareth. But the cause of their so doing we all know.

[11.]Again when Tertullian together with such as were his followers began to Montanize, and pretending to perfect the severity of Christian discipline brought in sundry unaccustomed days of fasting, continued their fasts a great deal longer and made them more rigorous than the use of the Church had been, the minds of men being somewhat moved at so great and so sudden novelty, the cause was presently inquired into. After notice taken how the Montanists held  these additions to be supplements of the gospel, whereunto the Spirit of prophecy did now mean to put as it were the last hand, and was therefore newly descended upon Montanus, whose orders all Christian men were no less to obey than the laws of the apostles themselves, this abstinence the Church abhorred likewise and that justly. Whereupon Tertullian proclaiming even open war to the Church, maintained Montanism, wrote a book in defence of the new fast, and entitled the same, A Treatise of Fasting against the Opinion of the Carnal Sort. In which treatise nevertheless because so much is sound and good, as doth either generally concern the use, or in particular declare the custom of the Church’s fasting in those times, men are not to reject whatsoever is alleged out of that book for confirmation of the truth. His error discloseth itself in those places where he defendeth his fasts to be duties necessary for the whole Church of Christ to observe as commanded by the Holy Ghost, and that with the same authority from whence all other apostolical ordinances came, both being the laws of God himself, without any other distinction or difference, saving only that he which before had declared his will by Paul and Peter, did now farther reveal the same by Montanus also. “Against us ye pretend,” saith Tertullian, “that the public orders which Christianity is bound to keep were delivered at the first, and that no new thing is to be added thereunto. Stand if you can upon this point. For behold I challenge you for fasting more than at Easter yourselves. But in fine ye answer, that these things are to be done as established by the voluntary appointment of men, and not by virtue or force of any divine commandment. Well then,” he addeth, “ye have removed your first footing, and gone beyond that which was delivered by doing more than was at the first imposed upon you. You say you must do that which your own judgments have allowed,  we require your obedience to that which God himself doth institute.
 Is it not strange that men to their own will should yield that which to God’s commandment they will not grant? Shall the pleasure of men prevail more with you than the power of God himself?”

[12.]These places of Tertullian for fasting have worthily been put to silence. And as worthily Aërius condemned for opposition against fasting. The one endeavoured to bring in such fasts as the church ought not to receive, the other to overthrow such as already it had received and did observe: the one was plausible unto many by seeming to hate carnal looseness and riotous excess much more than the rest of the world did, the other drew hearers by pretending the maintenance of Christian liberty: the one thought his cause very strongly upheld by making invective declamations with a pale and a withered countenance against the Church, by filling the ears of his starved hearers with speech suitable to such men’s humours, and by telling them no doubt to their marvellous contentment and liking, “Our new prophecies are refused, they are despised. Is it because Montanus doth preach some other God, or dissolve the gospel of Jesus Christ, or overthrow any canon of faith and hope? No, our crime is, we teach that men ought to fast more often than marry, the best feast-maker is with them the perfectest saint, they are assuredly mere spirit, and therefore these our corporal devotions please them not:” thus the one for Montanus and his superstition. The other in a clean contrary tune against the religion of the church, “These set fasts away with them, for they are Jewish and bring men under the yoke of servitude; if I will fast let me choose my time, that Christian liberty be not abridged.” Hereupon their glory was to fast especially upon the Sunday, because the order of  the Church was on that day not to fast.
 “On Church fasting days and specially the week before Easter, when with us,” saith Epiphanius, “custom admitteth nothing but lying down upon the earth, abstinence from fleshly delights and pleasures, sorrowfulness, dry and unsavoury diet, prayer, watching, fasting, all the medicines which holy affections can minister, they are up betimes to take in of the strongest for the belly, and when their veins are well swollen they make themselves mirth with laughter at this our service wherein we are persuaded we please God.”

[13.]By this of Epiphanius it doth appear not only what fastings the Church of Christ in those times used, but also what other parts of discipline were together therewith in force, according to the ancient use and custom of bringing all men at certain times to a due consideration and an open humiliation of themselves. Two kinds there were of public penitency, the one belonging to notorious offenders whose open wickedness had been scandalous; the other appertaining to the whole Church and unto every several person whom the same containeth. It will be answered that touching this latter kind it may be exercised well enough by men in private. No doubt but penitency is as prayer a thing acceptable unto God, be it in public or in secret. Howbeit as in the one if men were wholly left to their own voluntary meditations in their closets, and not drawn by laws and orders unto the open assemblies of the Church that there they may join with others in prayer, it may be soon conjectured what Christian devotion that way would come unto in a short time: even so in the other we are by sufficient experience taught how little it booteth to tell men of washing away their sins with tears of repentance, and so to leave them altogether unto themselves. O Lord, what heaps of grievous transgressions have we committed, the best, the perfectest, the most righteous amongst  us all, and yet clean pass them over unsorrowed for and unrepented of,
 only because the Church hath forgotten utterly how to bestow her wonted times of discipline, wherein the public example of all was unto every particular person a most effectual mean to put them often in mind, and even in a manner to draw them to that which now we all quite and clean forget as if penitency were no part of a Christian man’s duty!

[14.]Again besides our private offences which ought not thus loosely to be overslipped, suppose we the body and corporation of the Church so just, that at no time it needeth to shew itself openly cast down in regard of those faults and transgressions, which though they do not properly belong unto any one, had notwithstanding a special sacrifice appointed for them in the law of Moyses, and being common to the whole society which containeth all, must needs so far concern every man in particular, as at some time in solemn manner to require acknowledgment with more than daily and ordinary testifications of grief. There could not hereunto a fitter preamble be devised than that memorable commination set down in the book of Common Prayer, if our practice in the rest were suitable. The head already so well drawn doth but wish a proportionable body. And by the preface to that very part of the English liturgy it may appear how at the first setting down thereof no less was intended. For so we are to interpret the meaning of those words wherein restitution of the primitive church discipline is greatly wished for, touching the manner of public penance in time of Lent. Wherewith some being not much acquainted, but having framed in their minds the conceit of a new discipline far unlike unto that of old, they make themselves believe it is undoubtedly this their discipline which at the first was so much desired. They have long pretended that the whole Scripture is plain for them. If now the communion book make for them too (I well think the one doth as much as the other) it may be hoped that being found such a well-willer unto their cause, they will more favour it than they have done.

[15.]Having therefore hitherto spoken both of festival days, and so much of solemn fasts as may reasonably serve  to shew the ground thereof in the law of nature,
 the practice partly appointed and partly allowed of God in the Jewish Church, the like continued in the Church of Christ, together with the sinister oppositions either of heretics erroneously abusing the same, or of others thereat quarrelling without cause, we will only collect the chiefest points as well of resemblance as of difference between them, and so end. First in this they agree, that because nature is the general root of both, therefore both have been always common to the Church with infidels and heathen men. Secondly they also herein accord, that as oft as joy is the cause of the one and grief the well-spring of the other, they are incompatible. A third degree of affinity between them is that neither being acceptable to God of itself, but both tokens of that which is acceptable, their approbation with him must necessarily depend on that which they ought to import and signify; so that if herein the mind dispose not itself aright, whether we rest or fast we offend. A fourth thing common unto them is, that the greatest part of the world hath always grossly and palpably offended in both; infidels because they did all in relation to false gods; godless, sensual, and careless minds, for that there is in them no constant true and sincere affection towards those things which are pretended by such exercise; yea certain flattering oversights there are, wherewith sundry, and they not of the worst sort, may be easily in these cases led awry, even through abundance of love and liking to that which must be embraced by all means, but with caution; inasmuch as the very admiration of saints, whether we celebrate their glory or follow them in humility, whether we laugh or weep, mourn or rejoice with them, is (as in all things the affection of love) apt to deceive, and doth therefore need the more to be directed by a watchful guide, seeing there is manifestly both ways, even in them whom we honour, that which we are to observe and shun. The best have not still been sufficiently mindful that God’s very angels in heaven  are but angels, and that bodily exercise considered in itself is no great matter.
 Finally seeing that both are ordinances well devised for the good of man, and yet not man created purposely for them as for other offices of virtue whereunto God’s immutable law for ever tieth; it is but equity to wish or admonish that where by uniform order they are not as yet received, the example of Victor’s extremity in the one, and of John’s disciples’ curiosity in the other be not followed; yea where they are appointed by law, that notwithstanding we avoid Judaism, and as in festival days men’s necessities for matter of labour, so in times of fasting regard be had to their imbecilities, lest they should suffer harm doing good.

[16.]Thus therefore we see how these two customs are in divers respects equal. But of fasting the use and exercise though less pleasant is by so much more requisite than the other, as grief of necessity is a more familiar guest than the contrary passion of mind, albeit gladness to all men be naturally more welcome. For first we ourselves do many more things amiss than well, and the fruit of our own ill-doing is remorse, because nature is conscious to itself that it should do the contrary. Again forasmuch as the world over-aboundeth with malice, and few are delighted in doing good unto other men, there is no man so seldom crossed as pleasured at the hands of others, whereupon it cannot be chosen but every man’s woes must double in that respect the number and measure of his delights. Besides concerning the very choice which oftentimes we are to make, our corrupt inclination well considered, there is cause why our Saviour should account them happiest that do most mourn, and why Salomon might judge it better to frequent mourning than feasting houses, not better simply and in itself (for then would nature that way incline) but in regard of us and our common weakness better. Job was not ignorant that his children’s banquets though tending to amity needed sacrifice. Neither doth  any of us all need to be taught that in things which delight we easily swerve from mediocrity, and are not easily led by a right direct line.
 On the other side the sores and diseases of mind which inordinate pleasure breedeth are by dolour and grief cured. For which cause as all offences use to seduce by pleasing, so all punishments endeavour by vexing to reform transgressions. We are of our own accord apt enough to give entertainment to things delectable, but patiently to lack what flesh and blood doth desire, and by virtue to forbear what by nature we covet, this no man attaineth unto but with labour and long practice.

[17.]From hence it riseth that in former ages abstinence and fasting more than ordinary was always a special branch of their praise in whom it could be observed and known, were they such as continually gave themselves to austere life; or men that took often occasions in private virtuous respects to lay Salomon’s counsel aside, “Eat thy bread with joy,” and to be followers of David’s example which saith “I humbled my soul with fasting;” or but they who otherwise worthy of no great commendation have made of hunger some their gain, some their physic, some their art, that by mastering sensual appetites without constraint, they might grow able to endure hardness whensoever need should require. For the body accustomed to emptiness pineth not away so soon as having still used to fill itself.

Many singular effects there are which should make fasting even in public considerations the rather to be accepted. For I presume we are not altogether without experience how great their advantage is in martial enterprises that lead armies of men trained in a school of abstinence. It is therefore noted at this day in some that patience of hunger and thirst hath given them many victories; in others that because if they want there is no man able to rule them, nor they in plenty to moderate themselves, he which can either bring them to hunger or overcharge them is sure to make them their own overthrow.  What nation soever doth feel these dangerous inconveniences may know that sloth and fulness in peaceable times at home is the cause thereof,
 and the remedy a strict observation of that part of Christian discipline which teacheth men in practice of ghostly warfare against themselves those things that afterwards may help them justly assaulting or standing in lawful defence of themselves against others.

[18.]The very purpose of the Church of God both in the number and in the order of her fasts, hath been not only to preserve thereby throughout all ages the remembrance of miseries heretofore sustained, and of the causes in ourselves out of which they have arisen, that men considering the one might fear the other the more, but farther also to temper the mind lest contrary affections coming in place should make it too profuse and dissolute, in which respect it seemeth that fasts have been set as ushers of festival days for prevention of those disorders as much as might be, wherein notwithstanding the world always will deserve, as it hath done, blame, because such evils being not possible to be rooted out, the most we can do is in keeping them low; and (which is chiefly the fruit we look for) to create in the minds of men a love towards frugal and severe life, to undermine the palaces of wantonness, to plant parsimony as nature where riotousness hath been study, to harden whom pleasure would melt, and to help the tumours which always fulness breedeth, that children as it were in the wool of their infancy dyed with hardness may never afterwards change colour; that the poor whose perpetual fasts are necessity, may with better contentment endure the hunger which virtue causeth others so often to choose and by advice of religion itself so far to esteem above the contrary; that they which for the most part do lead sensual and easy lives, they which as the prophet David describeth them, “are not plagued like other men,” may by the public spectacle of all be still put in mind what themselves are; finally that  every man may be every man’s daily guide and example as well by fasting to declare humility as by praise to express joy in the sight of God,
 although it have herein befallen the Church as sometimes David, so that the speech of the one may be truly the voice of the other, “My soul fasted, and even that was also turned to my reproof.”


\section*{The celebration of Matrimony.}
LXXIII. In this world there can be no society durable otherwise than only by propagation. Albeit therefore single life be a thing more angelical and divine, yet sith the replenishing, first of earth with blessed inhabitants, and then of heaven with saints everlastingly praising God did depend upon conjunction of man and woman, he which made all things complete and perfect saw it could not be good to leave man without an helper unto the fore-alleged end.

[2.]In things which some farther end doth cause to be desired choice seeketh rather proportion than absolute perfection of goodness. So that woman being created for man’s sake to be his helper in regard to the end before-mentioned, namely the having and the bringing up of children, whereunto it was not possible they could concur unless there were subalternation between them, which subalternation is naturally grounded upon inequality, because things equal in every respect are never willingly directed one by another: woman therefore was even in her first estate framed by nature not only after in time but inferior in excellency also unto man, howbeit in so due and sweet proportion as being presented before our eyes, might be sooner perceived than defined. And even herein doth lie the reason why that kind of love which is the perfectest ground of wedlock is seldom able to yield any reason of itself.

[3.]Now that which is born of man must be nourished with far more travail, as being of greater price in nature and of slower pace to perfection, than the offspring of any other creature besides. Man and woman being therefore to join themselves for such a purpose, they were of necessity to be linked with some strait and insoluble knot. The bond of wedlock hath been always more or less esteemed of as a thing religious and sacred. The title which the very heathens  themselves do thereunto oftentimes give is holy.
 Those rites and orders which were instituted in the solemnization of marriage, the Hebrews term by the name of conjugal Sanctifications.

[4.]Amongst ourselves because sundry things appertaining unto the public order of matrimony are called in question by such as know not from whence those customs did first grow, to shew briefly some true and sufficient reason of them shall not be superfluous, although we do not hereby intend to yield so far unto enemies of all church orders saving their own, as though every thing were unlawful the true cause and reason whereof at the first might hardly perhaps be now rendered.

Wherefore to begin with the times wherein the liberty of marriage is restrained. “There is,” saith Salomon, “a time for all things, a time to laugh and a time to mourn.” That duties belonging unto marriage and offices appertaining  to penance are things unsuitable and unfit to be matched together, the Prophets and Apostles themselves do witness.
 Upon which ground as we might right well think it marvellous absurd to see in a church a wedding on the day of a public fast, so likewise in the selfsame consideration our predecessors thought it not amiss to take away the common liberty of marriages during the time which was appointed for the preparation unto and for exercise of general humiliation by fasting and praying, weeping for sins.

[5.]As for the delivering up of the woman either by her father or by some other, we must note that in ancient times all women which had not husbands nor fathers to govern them had their tutors, without whose authority there was no act which they did warrantable. And for this cause they were in marriage delivered unto their husbands by others. Which custom retained hath still this use, that it putteth women in mind of a duty whereunto the very imbecility of their nature and sex doth bind them, namely to be always directed, guided and ordered by others, although our positive laws do not tie them now as pupils.

[6.]The custom of laying down money seemeth to have been derived from the Saxons, whose manner was to buy their  wives.
 But seeing there is not any great cause wherefore the memory of that custom should remain, it skilleth not much although we suffer it to lie dead, even as we see it in a manner already worn out.

The ring hath been always used as an especial pledge of faith and fidelity. Nothing more fit to serve as a token of our purposed endless continuance in that which we never ought to revoke. This is the cause wherefore the heathens themselves did in such cases use the ring, whereunto Tertullian alluding saith, that in ancient times “No woman was permitted to wear gold saving only upon one finger, which her husband had fastened unto himself with that ring which was usually given for assurance of future marriage.” The cause why the Christians use it, as some of the fathers think, is either to testify mutual love or rather to serve for a pledge of conjunction in heart and mind agreed upon between them. But what rite and custom is there so harmless wherein the wit of man bending itself to derision may not easily find out somewhat to scorn and jest at? He that should have beheld the Jews when they stood with a four-cornered garment  spread over the heads of espoused couples while their espousals were in making,
 he that should have beheld their praying over a cup and their delivering the same at the marriage feast with set forms of benediction as the order amongst them was, might being lewdly affected take thereat as just occasion of scornful cavil as at the use of the ring in wedlock among Christians.

[7.]But of all things the most hardly taken is the uttering those words, “With my body I thee worship,” in which  words when once they are understood there will appear as little cause as in the rest for any wise man to be offended. First therefore inasmuch as unlawful copulation doth pollute and dishonour both parties, this protestation that we do worship and honour another with our bodies may import a denial of all such lets and impediments to our knowledge as might cause any stain, blemish, or disgrace that way, which kind of construction being probable would easily approve that speech to a peaceable and quiet mind. Secondly in that the Apostle doth so expressly affirm that parties married have not any longer entire power over themselves, but each hath interest in other’s person, it cannot be thought an absurd construction to say that worshipping with the body is the imparting of that interest in the body unto another which none before had save only ourselves. But if this were the natural meaning the words should perhaps be as requisite to be used on the one side as on the other, and therefore a third sense there is which I rather rely upon. Apparent it is that the ancient difference between a lawful wife and a concubine was only in the different purpose of man betaking himself to the one or the other. If his purpose were only fellowship, there grew to the woman by this mean no worship at all but the contrary. In professing that his intent was to add by his person honour and worship unto hers, he took her plainly and clearly to wife. This is it which the Civil Law doth mean when it maketh a wife to differ from a concubine in dignity; a wife to be taken where conjugal honour and affection doth go before. The  worship that grew unto her being taken with declaration of this intent was that her children became by this mean legitimate and free;
 herself was made a mother over his family; last of all she received such advancement of state as things annexed unto his person might augment her with, yea a right of participation was thereby given her both in him and even in all things which were his. This doth somewhat the more plainly appear by adding also that other clause, “With all my worldly goods I thee endow.” The former branch having granted the principal, the latter granteth that which is annexed thereunto.

[8.]To end the public solemnity of marriage with receiving the blessed Sacrament is a custom so religious and so holy, that if the church of England be blameable in this respect it is not for suffering it to be so much but rather for not providing that it may be more put in ure. The laws of Romulus concerning marriage are therefore extolled above the rest  amongst the heathens which were before,
 in that they established the use of certain special solemnities, whereby the minds of men were drawn to make the greater conscience of wedlock, and to esteem the bond thereof a thing which could not be without impiety dissolved. If there be any thing in Christian religion strong and effectual to like purpose it is the Sacrament of the holy Eucharist, in regard of the force whereof Tertullian breaketh out into these words concerning matrimony therewith sealed; “Unde sufficiam ad enarrandam felicitatem ejus matrimonii quod Ecclesia conciliat et confirmat oblatio?”—‘I know not which way I should be able to shew the happiness of that wedlock the knot whereof the Church doth fasten and the Sacrament of the Church confirm.’ Touching marriage therefore let thus much be sufficient.


\section*{The Churching of Women.}
LXXIV. The fruit of marriage is birth, and the companion of birth travail, the grief whereof being so extreme, and the danger always so great, dare we open our mouths against the things that are holy and presume to censure it as a fault in the Church of Christ, that women after their deliverance do publicly show their thankful minds unto God? But behold what reason there is against it! Forsooth, “if there should be solemn and express giving of thanks in the Church for every benefit either equal or greater than this which any singular person in the Church doth receive, we should not only have no preaching of the word nor ministering of the sacraments, but we should not have so much leisure as to do any corporal or bodily work, but should be like those Massilian heretics which do nothing else but pray.” Surely better a great deal to be like unto those heretics which do nothing else but pray, than those which do nothing else but quarrel. Their heads it might haply trouble somewhat  more than as yet they are aware of to find out so many benefits greater than this or equivalent thereunto,
 for which if so be our laws did require solemn and express thanksgiving in the church the same were like to prove a thing so greatly cumbersome as is pretended. But if there be such store of mercies even inestimable poured every day upon thousands (as indeed the earth is full of the blessings of the Lord which are day by day renewed without number and above measure) shall it not be lawful to cause solemn thanks to be given unto God for any benefit, than which greater or whereunto equal are received, no law binding men in regard thereof to perform the like duty? Suppose that some bond there be which tieth us at certain times to mention publicly the names of sundry our benefactors. Some of them it may be are such that a day would scarcely serve to reckon up together with them the catalogue of so many men besides as we are either more or equally beholden unto. Because no law requireth this impossible labour at our hands, shall we therefore condemn that law whereby the other being possible and also dutiful is enjoined us? So much we owe to the Lord of Heaven that we can never sufficiently praise him nor give him thanks for half those benefits for which this sacrifice were most due. Howbeit God forbid we should cease performing this duty when public order doth draw us unto it, when it may be so easily done, when it hath been so long executed by devout and virtuous people; God forbid that being so many ways provoked in this case unto so good a duty, we should omit it, only because there are other cases of like nature wherein we cannot so conveniently or at leastwise do not perform the same most virtuous office of piety.

[2.]Wherein we trust that as the action itself pleaseth God so the order and manner thereof is not such as may justly offend any. It is but an overflowing of gall which causeth the woman’s absence from the church during the time of her lying-in to be traduced, and interpreted as though  she were so long judged unholy, and were thereby shut out or sequestered from the house of God according to the ancient Levitical Law. Whereas the very canon law itself doth not so hold, but directly professeth the contrary; she is not barred from thence in such sort as they interpret it, nor in respect of any unholiness forbidden entrance into the church, although her abstaining from public  assemblies,
 and her abode in separation for the time be most convenient.

[3.]To scoff at the manner of attire than which there could be nothing devised for such a time more grave and decent, to make it a token of some folly committed for which they are loth to shew their faces, argueth that great divines are sometime more merry than wise. As for the women themselves, God accepting the service which they faithfully offer unto him, it is no great disgrace though they suffer pleasant witted men a little to intermingle with zeal scorn.

[4.]The name of Oblations applied not only here to those small and petit payments which yet are a part of the minister’s  right,
 but also generally given unto all such allowances as serve for their needful maintenance, is both ancient and convenient. For as the life of the clergy is spent in the service of God, so it is sustained with his revenue. Nothing therefore more proper than to give the name of Oblations to such payments in token that we offer unto him whatsoever his ministers receive.


\section*{The Rites of Burial.}
LXXV. But to leave this, there is a duty which the Church doth owe to the faithful departed, wherein forasmuch as the church of England is said to do those things which are though “not unlawful” yet “inconvenient,” because it appointeth a prescript form of service at burials, suffereth mourning apparel to be worn, and permitteth funeral sermons, a word or two concerning this point will be necessary, although it be needless to dwell long upon it.

[2.]The end of funeral duties is first to shew that love towards the party deceased which nature requireth; then to do him that honour which is fit both generally for man and particularly for the quality of his person; last of all to testify the care which the Church hath to comfort the living, and the hope which we all have concerning the resurrection of the dead.

For signification of love towards them that are departed mourning is not denied to be a thing convenient. As in truth the Scripture every where doth approve lamentation made unto this end. The Jews by our Saviour’s tears therefore, gathered in this case that his love towards Lazarus was great.  And that as mourning at such times is fit,
 so likewise that there may be a kind of attire suitable to a sorrowful affection and convenient for mourners to wear, how plainly doth David’s example show, who being in heaviness went up to the mount with his head covered and all the people that were with him in like sort? White garments being fit to use at marriage feasts and such other times of joy, whereunto Salomon alluding when he requireth continual cheerfulness of mind speaketh in this sort, “Let thy garments be always white;” what doth hinder the contrary from being now as convenient in grief as this heretofore in gladness hath been? “If there be no sorrow” they say “it is hypocritical to pretend it, and if there be to provoke it” by wearing such attire “is dangerous.” Nay if there be, to show it is natural, and if there be not, yet the signs are meet to show what should be, especially sith it doth not come oftentimes to pass that men are fain to have their mourning gowns pulled off their backs for fear of killing themselves with sorrow that way nourished.



[3.]The honour generally due unto all men maketh a decent interring of them to be convenient even for very humanity’s sake.
 And therefore so much as is mentioned in the burial of the widow’s son, the carrying of him forth upon a bier and the accompanying of him to the earth, hath been used even amongst infidels, all men accounting it a very extreme destitution not to have at the least this honour done them. Some man’s estate may require a great deal more according as the fashion of the country where he dieth doth afford. And unto this appertained the ancient use of the Jews to embalm the corpse with sweet odours, and to adorn the sepulchres of certain.

In regard of the quality of men it hath been judged fit to commend them unto the world at their death, amongst the heathen in funeral orations, amongst the Jews in sacred poems; and why not in funeral sermons also amongst Christians? Us it sufficeth that the known benefit hereof doth countervail millions of such inconveniences as are therein surmised, although they were not surmised only but found therein. The life and the death of saints is precious in God’s sight. Let it not seem odious in our eyes if both the one and the other be spoken of then especially, when the present occasion doth make men’s minds the more capable of such speech. The care no doubt of the living both to live and to die well must needs be somewhat increased, when they know  that their departure shall not be folded up in silence but the ears of many be made acquainted with it.
 Besides when they hear how mercifully God hath dealt with their brethren in their last need, besides the praise which they give to God and the joy which they have or should have by reason of their fellowship and communion with saints, is not their hope also much confirmed against the day of their own dissolution? Again the sound of these things doth not so pass the ears of them that are most loose and dissolute in life but it causeth them one time or other to wish, “O that I might die the death of the righteous and that my end might be like his!” Thus much peculiar good there doth grow at those times by speech concerning the dead, besides the benefit of public instruction common unto funeral with other sermons.

For the comfort of them whose minds are through natural affection pensive in such cases no man can justly mislike the custom which the Jews had to end their burials with funeral banquets, in reference whereunto the prophet Jeremy spake concerning the people whom God had appointed unto a grievous manner of destruction, saying that men should not “give them the cup of consolation to drink for their father or for their mother,” because it should not be now with them as in peaceable times with others, who bringing their ancestors unto the grave with weeping eyes have notwithstanding means wherewith to be recomforted. “Give wine,” said Salomon, “unto them that have grief of heart.” Surely he that ministereth unto them comfortable speech doth much more than give them wine.

[4.]But the greatest thing of all other about this duty of Christian burial is an outward testification of the hope which we have touching the resurrection of the dead. For which purpose let any man of reasonable judgment examine, whether  it be more convenient for a company of men as it were in a dumb show to bring a corse to the place of burial, there to leave it covered with earth, and so end, or else to have the exequies devoutly performed with solemn recital of such lectures, psalms and prayers, as are purposely framed for the stirring up of men’s minds unto a careful consideration of their estate both here and hereafter.

Whereas therefore it is objected that neither the people of God under the Law, nor the Church in the Apostles’ times did use any form of service in burial of their dead, and therefore that this order is taken up without any good example or precedent followed therein: first while the world doth stand they shall never be able to prove that all things which either the one or the other did use at burial are set down in holy Scripture, which doth not any where of purpose deliver the whole manner and form thereof, but toucheth only sometime one thing and sometime another which was in use, as special occasions require any of them to be either mentioned or insinuated. Again if it might be proved that no such thing was usual amongst them, hath Christ so deprived his Church of judgment that what rites and orders soever the later ages thereof have devised the same must needs be inconvenient?

Furthermore, that the Jews before our Saviour’s coming had any such form of service although in scripture it be not affirmed, yet neither is it there denied; (for the forbidding  of priests to be present at burials letteth not but that others might discharge that duty, seeing all were not priests which had rooms of public function in their synagogues;) and if any man be of opinion that they had no such form of service, thus much there is to make the contrary more probable. The Jews at this day have, as appeareth in their form of funeral prayers and in certain of their funeral sermons published, neither are they so affected towards Christians, as to borrow that order from us, besides that the form thereof is such as hath in it sundry things which the very words of the Scripture itself do seem to allude unto, as namely after departure from the sepulchre unto the house whence the dead was brought it sheweth the manner of their burial feast, and a consolatory form of prayers appointed for the master of the synagogue thereat to utter, albeit I may not deny but it hath also some  things which are not perhaps so ancient as the Law and the Prophets.

But whatsoever the Jews’ custom was before the days of our Saviour Christ, hath it once at any time been heard of that either church or Christian man of sound belief did ever judge this a thing unmeet, undecent, unfit for Christianity, till these miserable days, wherein under the colour of removing superstitious abuses the most effectual means both to testify and to strengthen true religion are plucked at, and in some places even pulled up by the very roots? Take away this which was ordained to show at burials the peculiar hope of the Church of God concerning the dead, and in the manner of those dumb funerals what one thing is there whereby the world may perceive we are Christian men?


\section*{Of the nature of that Ministry which serveth for performance of divine duties in the Church of God, and how happiness not eternal only but also temporal doth depend upon it.}
LXXVI. I come now unto that function which undertaketh the public ministry of holy things according to the laws of Christian religion. And because the nature of things consisting, as this doth, in action is known by the object whereabout they are conversant, and by the end or scope whereunto they are referred, we must know that the object of this function is both God and men; God in that he is publicly worshipped of his Church, and men in that they are capable of happiness by means which Christian discipline appointeth. So that the sum of our whole labour in this kind is to honour God and to save men.

For whether we severally take and consider men one by one, or else gather them into one society and body, as it hath been before declared that every man’s religion is in him the well-spring of all other sound and sincere virtues, from whence both here in some sort and hereafter more abundantly their full joy and felicity ariseth, because while they live they are blessed of God and when they die their works follow them: so at this present we must again call to mind how the very worldly peace and prosperity, the secular happiness, the temporal  and natural good estate both of all men and of all dominions hangeth chiefly upon religion,
 and doth evermore give plain testimony that as well in this as in other considerations the priest is a pillar of that commonwealth wherein he faithfully serveth God. For if these assertions be true, first that nothing can be enjoyed in this present world against his will which hath made all things; secondly that albeit God doth sometime permit the impious to have, yet impiety permitteth them not to enjoy no not temporal blessings on earth; thirdly that God hath appointed those blessings to attend as handmaids upon religion; and fourthly that without the work of the ministry religion by no means can possibly continue, the use and benefit of that sacred function even towards all men’s worldly happiness must needs be granted.

[2.]Now the first being a theorem both understood and confessed of all, to labour in proof thereof were superfluous.

The second perhaps may be called in question except it be perfectly understood. By good things temporal therefore we mean length of days, health of body, store of friends and well-willers, quietness, prosperous success of those things we take in hand, riches with fit opportunities to use them during life, reputation following us both alive and dead, children or such as instead of children we wish to leave successors and partakers of our happiness. These things are naturally every man’s desire, because they are good. And on whom God bestoweth the same, them we confess he graciously blesseth.

Of earthly blessings the meanest is wealth, reputation the chiefest. For which cause we esteem the gain of honour an ample recompense for the loss of all other worldly benefits.

[3.]But forasmuch as in all this there is no certain perpetuity of goodness, nature hath taught to affect these things not for their own sake but with reference and relation to somewhat independently good, as is the exercise of virtue and  speculation of truth.
 None whose desires are rightly ordered would wish to live, to breathe and move, without performance of those actions which are beseeming man’s excellency. Wherefore having not how to employ it we wax weary even of life itself. Health is precious because sickness doth breed that pain which disableth action. Again why do men delight so much in the multitude of friends, but for that the actions of life being many do need many helping hands to further them? Between troublesome and quiet days we should make no difference if the one did not hinder and interrupt, the other uphold, our liberty of action. Furthermore if those things we do, succeed, it rejoiceth us not so much for the benefit we thereby reap as in that it probably argueth our actions to have been orderly and well guided. As for riches, to him which hath and doth nothing with them they are a contumely. Honour is commonly presumed a sign of more than ordinary virtue and merit, by means whereof when ambitious minds thirst after it, their endeavours are testimonies how much it is in the eye of nature to possess that body the very shadow whereof is set at so high a rate. Finally such is the pleasure and comfort which we take in doing, that when life forsaketh us, still our desires to continue action and to work though not by ourselves yet by them whom we leave behind us, causeth us providently to resign into other men’s hands the helps we have gathered for that purpose, devising also the best we can to make them perpetual. It appeareth therefore how all the parts of temporal felicity are only good in relation to that which useth them as instruments, and that they are no such good as wherein a right desire doth ever stay or rest itself.

[4.]Now temporal blessings are enjoyed of those which have them, know them, esteem them according to that they are in their own nature. Wherefore of the wicked whom God doth hate his usual and ordinary speeches are, that “blood-thirsty and deceitful men shall not live out half their days;” that God shall cause “a pestilence to cleave” unto the wicked, and shall strike them with consuming grief, with fevers, burning diseases, and sores which are past cure; that when the impious are fallen, all men shall tread them down  and none shew countenance of love towards them as much as by pitying them in their misery; that the sins of the ungodly shall bereave them of peace; that all counsels, complots, and practices against God shall come to nothing; that the lot and inheritance of the unjust is beggary; that the name of unrighteous persons shall putrefy, and the posterity of robbers starve. If any think that iniquity and peace, sin and prosperity can dwell together, they err, because they distinguish not aright between the matter, and that which giveth it the form of happiness, between possession and fruition, between the having and the enjoying of good things. The impious cannot enjoy that they have, partly because they receive it not as at God’s hands, which only consideration maketh temporal blessings comfortable, and partly because through error placing it above things of far more price and worth they turn that to poison which might be food, they make their prosperity their own snare, in the nest of their highest growth they lay foolishly those eggs out of which their woful overthrow is afterwards hatched. Hereby it cometh to pass that wise and judicious men observing the vain behaviours of such as are risen to unwonted greatness have thereby been able to prognosticate their ruin. So that in very truth no impious or wicked man doth prosper on earth but either sooner or later the world may perceive easily how at such time as others thought them most fortunate they had but only the good estate which fat oxen have above lean, when they appeared to grow their climbing was towards ruin.

The gross and bestial conceit of them which want understanding is only that the fullest bellies are happiest. Therefore  the greatest felicity they wish to the commonwealth wherein they live is that it may but abound and stand,
 that they which are riotous may have to pour out without stint, that the poor may sleep and the rich feed them, that nothing unpleasant may be commanded, nothing forbidden men which themselves have a lust to follow, that kings may provide for the ease of their subjects and not be too curious about their manners, that wantonness, excess, and lewdness of life may be left free, and that no fault may be capital besides dislike of things settled in so good terms. But be it far from the just to dwell either in or near to the tents of these so miserable felicities.

[5.]Now whereas we thirdly affirm that religion and the fear of God as well induceth secular prosperity as everlasting bliss in the world to come, this also is true. For otherwise godliness could not be said to have the promises of both lives, to be that ample revenue wherein there is always sufficiency, and to carry with it a general discharge of want, even so general that David himself should protest he “never saw the just forsaken.”

Howbeit to this we must add certain special limitations; as first that we do not forget how crazed and diseased minds (whereof our heavenly Physician must judge) receive oftentimes  most benefit by being deprived of those things which are to others beneficially given, as appeareth in that which the wise man hath noted concerning them whose lives God mercifully doth abridge lest wickedness should alter their understanding; again that the measure of our outward prosperity be taken by proportion with that which every man’s estate in this present life requireth. External abilities are instruments of action. It contenteth wise artificers to have their instruments proportionable to their work, rather fit for use than huge and goodly to please the eye. Seeing then the actions of a servant do not need that which may be necessary for men of calling and place in the world, neither men of inferior condition many things which greater personages can hardly want, surely they are blessed in worldly respects that have wherewith to perform sufficiently what their station and place asketh, though they have no more. For by reason of man’s imbecility and proneness to elation of mind, too high a flow of prosperity is dangerous; too low an ebb again as dangerous, for that the virtue of patience is rare, and the hand of necessity stronger than ordinary virtue is able to withstand. Salomon’s discreet and moderate desire we all know, “Give me O Lord neither riches nor penury.” Men over high exalted either in honour or in power or in nobility or in wealth; they likewise that are as much on the contrary hand sunk either with beggary or through dejection or by baseness do not easily give ear to reason, but the one exceeding apt unto outrages and the other unto petty mischiefs. For greatness delighteth to show itself by effects of power, and baseness to help itself with shifts of malice. For which cause a moderate indifferent temper between fulness of bread and emptiness hath been evermore thought and found (all  circumstances duly considered) the safest and happiest for all estates,
 even for kings and princes themselves.

Again we are not to look that these things should always concur, no not in them which are accounted happy, neither that the course of men’s lives or of public affairs should continually be drawn out as an even thread (for that the nature of things will not suffer) but a just survey being made, as those particular men are worthily reputed good whose virtues be great and their faults tolerable, so him we may register for a man fortunate, and that for a prosperous or happy state, which having flourished doth not afterwards feel any tragical alteration such as might cause them to be a spectacle of misery to others.

Besides whereas true felicity consisteth in the highest operations of that nobler part of man which showeth sometime greatest perfection not in using the benefits which delight nature but in suffering what nature can hardliest endure, there is no cause why either the loss of good if it tend to the purchase of better, or why any misery the issue whereof is their greater praise and honour that have sustained it, should be thought to impeach that temporal happiness wherewith religion we say is accompanied, but yet in such measure, as the several degrees of men may require by a competent estimation, and unless the contrary do more advance, as it hath done those most heroical saints whom afflictions have made glorious. In a word not to whom no calamity falleth, but whom neither misery nor prosperity is able to move from a right mind, them we may truly pronounce fortunate, and whatsoever doth outwardly happen without that precedent improbity for which it appeareth in the eyes of sound and unpartial judges to have proceeded from divine revenge, it passeth in the number of human casualties whereunto we are all alike subject. No misery is reckoned more than common or human, if God so dispose that we pass through it and come safe to shore, even as contrariwise men do not use to think those flourishing days happy which do end with tears.

[6.]It standeth therefore with these cautions firm and true, yea ratified by all men’s unfeigned confessions drawn from the very heart of experience, that whether we compare men of note in the world with others of like degree and state,  or else the same men with themselves;
 whether we confer one dominion with another or else the different times of one and the same dominion, the manifest odds between their very outward condition as long as they steadfastly were observed to honour God and their success being fallen from him, are remonstrances more than sufficient how all our welfare even on earth dependeth wholly upon our religion.

Heathens were ignorant of true religion. Yet such as that little was which they knew, it much impaired or bettered always their worldly affairs, as their love and zeal towards it did wane or grow.

Of the Jews did not even their most malicious and mortal adversaries all acknowledge, that to strive against them it was in vain as long as their amity with God continued, that nothing could weaken them but apostasy? In the whole course of their own proceedings did they ever find it otherwise, but that during their faith and fidelity towards God every man of them was in war as a thousand strong, and as much as a grand Senate for counsel in peaceable deliberations, contrariwise that if they swerved, as they often did, their wonted courage and magnanimity forsook them utterly, their soldiers and military men trembled at the sight of the naked sword; when they entered into mutual conference, and sate in council for their own good, that which children might have seen their gravest Senators could not discern, their Prophets saw darkness instead of visions, the wise and prudent were as men bewitched, even that which they knew (being such as might stand them in stead) they had not the grace to utter, or if any thing were well proposed it took no place, it entered not into the minds of the rest to approve and follow it, but as men confounded with strange and unusual amazements of spirit they attempted tumultuously they saw not what; and by the issues of all tempts they found no certain conclusion but this, “God and heaven are strong against us in all we do.” The cause whereof was secret fear which took heart and courage from them, and the cause of their fear an inward guiltiness that they all had offered God such apparent wrongs as were not pardonable.

[7.]But it may be the case is now altogether changed, and that in Christian religion there is not the like force towards  temporal felicity.
 Search the ancient records of time, look what hath happened by the space of these sixteen hundred years, see if all things to this effect be not luculent and clear, yea all things so manifest that for evidence and proof herein we need not by uncertain dark conjectures surmise any to have been plagued of God for contempt, or blest in the course of faithful obedience towards true religion, more than only them whom we find in that respect on the one side guilty by their own confessions, and happy on the other side by all men’s acknowledgment, who beholding the prosperous estate of such as are good and virtuous impute boldly the same to God’s most especial favour, but cannot in like manner pronounce that whom he afflicteth above others with them he hath cause to be more offended. For virtue is always plain to be seen, rareness causeth it to be observed, and goodness to be honoured with admiration. As for iniquity and sin it lieth many times hid, and because we be all offenders it becometh us not to incline towards hard and severe sentences touching others, unless their notorious wickedness did sensibly before proclaim that which afterwards came to pass.

[8.]Wherefore the sum of every Christian man’s duty is to labour by all means towards that which other men seeing in us may justify, and what we ourselves must accuse, if we fall into it, that by all means we can to avoid, considering especially that as hitherto upon the Church there never yet fell tempestuous storm the vapours whereof were not first noted to rise from coldness in affection and from backwardness in duties of service towards God, so if that which the tears of antiquity have uttered concerning this point should be here set down, it were assuredly enough to soften and to mollify an heart of steel. On the contrary part although we confess with St. Augustine most willingly, that the chiefest happiness  for which we have some Christian kings in so great admiration above the rest is not because of their long reign, their calm and quiet departure out of this present life, the settled establishment of their own flesh and blood succeeding them in royalty and power, the glorious overthrow of foreign enemies, or the wise prevention of inward dangers and of secret attempts at home; all which solaces and comforts of this our unquiet life it pleaseth God oftentimes to bestow on them which have no society or part in the joys of heaven, giving thereby to understand that these in comparison are toys and trifles far under the value and price of that which is to be looked for at his hands; but in truth the reason wherefore we mostly extol their felicity is if so be they have virtuously reigned, if honour have not filled their hearts with pride, if the exercise of their power have been service and attendance upon the majesty of the Most High, if they have feared him as their own inferiors and subjects have feared them, if they have loved neither pomp nor pleasure more than heaven, if revenge have slowly proceeded from them and mercy willingly offered itself, if so they have tempered rigour with lenity that neither extreme severity might utterly cut them off in whom there was manifest hope of amendment, nor yet the easiness of pardoning offences embolden offenders, if knowing that whatsoever they do their potency may bear it out, they have been so much the more careful not to do any thing but that which is commendable in the best rather than usual with greatest personages, if the true knowledge of themselves have  humbled them in God’s sight no less than God in the eyes of men hath raised them up;
 I say albeit we reckon such to be the happiest of them that are mightiest in the world, and albeit those things alone are happiness, nevertheless considering what force there is even in outward blessings to comfort the minds of the best disposed, and to give them the greater joy when religion and peace, heavenly and earthly happiness are wreathed in one crown, as to the worthiest of Christian princes it hath by the providence of the Almighty hitherto befallen: let it not seem to any man a needless and superfluous waste of labour that there hath been thus much spoken to declare how in them especially it hath been so observed, and withal universally noted even from the highest to the very meanest, how this peculiar benefit, this singular grace and preeminence religion hath, that either it guardeth as an heavenly shield from all calamities, or else conducteth us safe through them, and permitteth them not to be miseries; it either giveth honours, promotions, and wealth, or else more benefit by wanting them than if we had them at will; it either filleth our houses with plenty of all good things, or maketh a sallet of green herbs more sweet than all the sacrifices of the ungodly.

[9.]Our fourth proposition before set down was that religion without the help of spiritual ministry is unable to plant itself, the fruits thereof not possible to grow of their own accord. Which last assertion is herein as the first, that it needeth no farther confirmation. If it did I could easily declare how all things which are of God he hath by wonderful art and wisdom sodered as it were together with the glue of mutual assistance, appointing the lowest to receive from the nearest to themselves what the influence of the highest yieldeth. And therefore the Church being the most absolute of all his works was in reason to be also ordered with like harmony, that what he worketh might no less in grace than in nature be effected by hands and instruments duly subordinated unto the power of his own Spirit. A thing both needful for the humiliation of man which would not willingly be debtor to any but to himself, and of no small effect to nourish that divine love which now maketh each embrace other not as men but as angels of God.




[10.]Ministerial actions tending immediately unto God’s honour and man’s happiness are either as contemplation, which helpeth forward the principal work of the ministry; or else they are parts of that principal work of administration itself, which work consisteth in doing the service of God’ house and in applying unto men the sovereign medicines of grace, already spoken of the more largely, to the end it might thereby appear that we owe to the guides of our souls even as much as our souls are worth, although the debt of our temporal blessings should be stricken off.


\section*{Of power given unto men to execute that heavenly office, of the gift of the Holy Ghost in Ordination; and whether conveniently the power of order may be sought or sued for.}
LXXVII. The ministry of things divine is a function which as God did himself institute, so neither may men undertake the same but by authority and power given them in lawful manner. That God which is no way deficient or wanting unto man in necessaries, and hath therefore given us the light of his heavenly truth, because without that inestimable benefit we must needs have wandered in darkness to our endless perdition and woe, hath in the like abundance of mercies ordained certain to attend upon the due execution of requisite parts and offices therein prescribed for the good of the whole world, which men thereunto assigned do hold their authority from him, whether they be such as himself immediately or as the Church in his name investeth, it being neither possible for all nor for every man without distinction convenient to take upon him a charge of so great importance. They are therefore ministers of God, not only by way of subordination as princes and civil magistrates whose execution of judgment and justice the supreme hand of divine providence doth uphold, but ministers of God as from whom their authority is derived, and not from men. For in that they are Christ’s ambassadors and his labourers, who should give them their commission but he whose most inward affairs they manage? Is not God alone the Father of spirits? Are not souls the purchase of Jesus Christ? What angel in Heaven could have said to man as our Lord did unto Peter, “Feed my sheep: Preach: Baptize: Do this in remembrance of me: Whose sins ye retain they are  retained:
 and their offences in heaven pardoned whose faults you shall on earth forgive?” What think we? Are these terrestrial sounds, or else are they voices uttered out of the clouds above? The power of the ministry of God translateth out of darkness into glory, it raiseth men from the earth and bringeth God himself down from heaven, by blessing visible elements it maketh them invisible grace, it giveth daily the Holy Ghost, it hath to dispose of that flesh which was given for the life of the world and that blood which was poured out to redeem souls, when it poureth malediction upon the heads of the wicked they perish, when it revoketh the same they revive. O wretched blindness if we admire not so great power, more wretched if we consider it aright and notwithstanding imagine that any but God can bestow it!

[2.]To whom Christ hath imparted power both over that mystical body which is the society of souls, and over that natural which is himself for the knitting of both in one; (a work which antiquity doth call the making of Christ’s body;) the same power is in such not amiss both termed a kind of mark or character and acknowledged to be indelible. Ministerial power is a mark of separation, because it severeth them that have it from other men, and maketh them a special order consecrated unto the service of the Most High in things wherewith others may not meddle. Their difference therefore from other men is in that they are a distinct order. So Tertullian calleth them. And St. Paul himself dividing the body of the Church of Christ into two moieties nameth the one part ἰδιώτας, which is as much as to say the Order of the Laity, the opposite part whereunto we in like sort term the Order of God’s Clergy, and the spiritual power which he hath given them the power of their Order, so far forth as the same consisteth in the bare execution of holy things called properly the affairs of God. For of the power of their  jurisdiction over men’s persons we are to speak in the books following.

[3.]They which have once received this power may not think to put it off and on like a cloak as the weather serveth, to take it, reject and resume it as oft as themselves list, of which profane and impious contempt these later times have yielded as of all other kinds of iniquity and apostasy strange examples; but let them know which put their hands unto this plough, that once consecrated unto God they are made his peculiar inheritance for ever. Suspensions may stop, and degradations utterly cut off the use or exercise of power before given: but voluntarily it is not in the power of man to separate and pull asunder what God by his authority coupleth. So that although there may be through misdesert degradation, as there may be cause of just separation after matrimony, yet if (as sometime it doth) restitution to former dignity or reconciliation after breach do happen, neither doth the one nor the other ever iterate the first knot.

Much less is it necessary which some have urged, concerning the reordination of such as others in times more corrupt did consecrate heretofore. Which error already quelled by St. Jerome doth not now require any other refutation.

[4.]Examples I grant there are which make for restraint of those men from admittance again into rooms of spiritual function, whose fall by heresy or want of constancy in professing the Christian faith hath been once a disgrace to their calling. Nevertheless as there is no law which bindeth, so there is no cause that should always lead, to show one and  the same severity towards persons culpable.
 Goodness of nature itself more inclineth to clemency than rigour. And we in other men’s offences do behold the plain image of our own imbecility. Besides also, them that wander out of the way it cannot be unexpedient to win with all hopes of favour, lest strictness used towards such as reclaim themselves should make others more obstinate in error. Wherefore after that the church of Alexandria had somewhat recovered itself from the tempests and storms of Arianism, being in consultation about the reestablishment of that which by long disturbance had been greatly decayed and hindered, the ferventer sort gave quick sentence that touching them which were of the clergy and had stained themselves with heresy there should be none so received into the Church again as to continue in the order of the clergy. The rest which considered how many men’s cases it did concern thought it much more safe and consonant to bend somewhat down towards them which were fallen, to show severity upon a few of the chiefest leaders, and to offer to the rest a friendly reconciliation without any other demand saving only the abjuration of their error; as in the gospel that wasteful  young man which returned home to his father’s house was with joy both admitted and honoured,
 his elder brother hardly thought of for repining thereat, neither commended so much for his own fidelity and virtue as blamed for not embracing him freely whose unexpected recovery ought to have blotted out all remembrance of misdemeanours and faults past. But of this sufficient.

[5.]A thing much stumbled at in the manner of giving orders is our using those memorable words of our Lord and Saviour Christ, “Receive the Holy Ghost.” The Holy Ghost they say we cannot give, and therefore we “foolishly” bid men receive it. Wise men for their authority’s sake must have leave to befool them whom they are able to make wise by better instruction. Notwithstanding if it may please their wisdom as well to hear what fools can say as to control that which they do, thus we have heard some wise men teach, namely that the “Holy Ghost” may be used to signify  not the Person alone but the gifts of the Holy Ghost,
 and we know that spiritual gifts are not only abilities to do things miraculous, as to speak with tongues which were never taught us, to cure diseases without art, and such like, but also that the very authority and power which is given men in the Church to be ministers of holy things, this is contained within the number of those gifts whereof the Holy Ghost is author; and therefore he which giveth this power may say without absurdity or folly “Receive the Holy Ghost,” such power as the Spirit of Christ hath endued his Church withal, such power as neither prince nor potentate, king nor Cæsar on earth can give. So that if men alone had devised this form of speech thereby to express the heavenly wellspring of that power which ecclesiastical ordinations do bestow, it is not so foolish but that wise men might bear with it.

[6.]If then our Lord and Saviour himself have used the selfsame form of words and that in the selfsame kind of action, although there be but the least shew of probability, yea or any possibility that his meaning might be the same which ours is, it should teach sober and grave men not to be too venturous in condemning that of folly which is not impossible to have in it more profoundness of wisdom than flesh and blood should presume to control. Our Saviour after his resurrection from the dead gave his Apostles their commission saying, “All power is given me in Heaven and in earth: Go therefore and teach all nations, Baptizing them in the name of the Father and the Son and the Holy Ghost, teaching them to observe all things whatsoever I have commanded you.” In sum, “As my Father sent me, so send I you.” Whereunto St. John doth add farther that “having thus spoken he breathed on them and said, Receive the Holy Ghost.” By which words he must of likelihood understand some gift of the Spirit which was presently at that time bestowed upon them, as both the speech of actual delivery in saying Receive, and the visible sign thereof his  breathing did show.
 Absurd it were to imagine our Saviour did both to the ear and also to the very eye express a real donation, and they at that time receive nothing.

[7.]It resteth then that we search what especial grace they did at that time receive. Touching miraculous power of the Spirit, most apparent it is that as then they received it not, but the promise thereof was to be shortly after performed. The words of St. Luke concerning that power are therefore set down with signification of the time to come: “Behold I will send the promise of my Father upon you, but tarry you in the city of Jerusalem until ye be endued with power from on high.” Wherefore undoubtedly it was some other effect of the Spirit, the Holy Ghost in some other kind which our Saviour did then bestow. What other likelier than that which himself doth mention as it should seem of purpose to take away all ambiguous constructions, and to declare that the Holy Ghost which he then gave was an holy and a ghostly authority, authority over the souls of men, authority a part whereof consisteth in power to remit and retain sins? “Receive the Holy Ghost: whose sins soever ye remit they are remitted; whose sins ye retain they are retained.” Whereas therefore the other Evangelists had set down that Christ did before his suffering promise to give his Apostles the keys of the kingdom of heaven, and being risen from the dead promise moreover at that time a miraculous power of the Holy Ghost, St. John addeth that he also invested them even then with the power of the Holy Ghost for castigation and relaxation of sin, wherein was fully accomplished that which the promise of the keys did import.

Seeing therefore that the same power is now given, why should the same form of words expressing it be thought foolish? The cause why we breathe not as Christ did on them unto whom he imparted power is for that neither Spirit nor spiritual authority may be thought to proceed from us,  which are but delegates or assigns to give men possession of his graces.

[8.]Now, besides that the power and authority delivered with those words is itself χάρισμα, a gracious donation which the Spirit of God doth bestow, we may most assuredly persuade ourselves that the hand which imposeth upon us the function of our ministry doth under the same form of words so tie itself thereunto, that he which receiveth the burden is thereby for ever warranted to have the Spirit with him and in him for his assistance, aid, countenance and support in whatsoever he faithfully doth to discharge duty. Knowing therefore that when we take ordination we also receive the presence of the Holy Ghost, partly to guide, direct and strengthen us in all our ways, and partly to assume unto itself for the more authority those actions that appertain to our place and calling, can our ears admit such a speech uttered in the reverend performance of that solemnity, or can we at any time renew the memory and enter into serious cogitation thereof, but with much admiration and joy? Remove what these foolish words do imply, and what hath the ministry of God besides wherein to glory? Whereas now, forasmuch as the Holy Ghost which our Saviour in his first ordinations gave doth no less concur with spiritual vocations throughout all ages, than the Spirit which God derived from Moses to them that assisted him in his government did descend from them to their successors in like authority and place, we have for the least and meanest duties performed by virtue of ministerial power, that to dignify, grace and authorize them, which no other offices on earth can challenge. Whether we  preach,
 pray, baptize, communicate, condemn, give absolution, or whatsoever, as disposers of God’s mysteries, our words, judgments, acts and deeds, are not ours but the Holy Ghost’s. Enough, if unfeignedly and in heart we did believe it, enough to banish whatsoever may justly be thought corrupt, either in bestowing, or in using, or in esteeming the same otherwise than is meet. For profanely to bestow, or loosely to use, or vilely to esteem of the Holy Ghost we all in shew and profession abhor.

[9.]Now because the ministry is an office of dignity and honour, some are doubtful whether any man may seek for it without offence, or to speak more properly doubtful they are not, but rather bold to accuse our discipline in this respect, as not only permitting but requiring also ambitious suits and other oblique ways or means whereby to obtain it. Against this they plead that our Saviour did stay till his Father sent him, and the Apostles till he them; that the ancient Bishops in the Church of Christ were examples and patterns of the same modesty. Whereupon in the end they infer, “Let us therefore at the length amend that custom of repairing from all parts unto the bishop at the day of ordination, and of seeking to obtain orders; let the custom of bringing commendatory letters be removed; let men keep themselves at home, expecting there the voice of God and the authority of such as may call them to undertake charge.”

[10.]Thus severely they censure and control ambition, if it be ambition which they take upon them to reprehend. For  of that there is cause to doubt.
 Ambition as we understand it hath been accounted a vice which seeketh after honours inordinately. Ambitious minds esteeming it their greatest happiness to be admired, reverenced, and adored above others, use all means lawful and unlawful which may bring them to high rooms. But as for the power of order considered by itself and as in this case it must be considered, such reputation it hath in the eye of this present world, that they which affect it rather need encouragement to bear contempt than deserve blame as men that carry aspiring minds. The work whereunto this power serveth is commended, and the desire thereof allowed by the Apostle for good. Nevertheless because the burden thereof is heavy and the charge great, it cometh many times to pass that the minds even of virtuous men are drawn into clean contrary affections, some in humility declining that by reason of hardness which others in regard of goodness only do with fervent alacrity covet. So that there is not the least degree in this service but it may be both in reverence shunned, and of very devotion longed for.

If, then, the desire thereof may be holy religious and good, may not the profession of that desire be so likewise? We are not to think it so long good as it is dissembled and evil if once we begin to open it.

And allowing that it may be opened without ambition, what offence I beseech you is there in opening it there where it may be furthered and satisfied in case they to whom it appertaineth think meet? In vain are those desires allowed the accomplishment whereof it is not lawful for men to seek.

Power therefore of ecclesiastical order may be desired, the desire thereof may be professed, they which profess themselves that way inclined may endeavour to bring their desires to effect, and in all this no necessity of evil. Is it the bringing of testimonial letters wherein so great obliquity consisteth?  What more simple,
 more plain, more harmless, more agreeable with the law of common humanity than that men where they are not known use for their easier access the credit of such as can best give testimony of them? Letters of any other construction our church discipline alloweth not, and these to allow is neither to require ambitious suings nor to approve any indirect or unlawful act.

[11.]The prophet Esay receiving his message at the hands of God and his charge by heavenly vision heard the voice of the Lord saying, “Whom shall I send; who shall go for us?” Whereunto he recordeth his own answer, “Then I said, Here Lord I am, send me.” Which in effect is the rule and canon whereby touching this point the very order of the church is framed. The appointment of times for solemn ordination is but the public demand of the Church in the name of the Lord himself, “Whom shall I send, who shall go for us?” The confluence of men whose inclinations are bent that way is but the answer thereunto, whereby the labours of sundry being offered, the Church hath freedom to take whom her agents in such case think meet and requisite.

[12.]As for the example of our Saviour Christ who took not to himself this honour to be made our high priest, but received the same from him which said, “Thou art a Priest for ever after the order of Melchisedec,” his waiting and not attempting to execute the office till God saw convenient time may serve in reproof of usurped honours, forasmuch as we ought not of our own accord to assume dignities, whereunto we are not called as Christ was. But yet it should be withal considered that a proud usurpation without any orderly calling is one thing, and another the bare declaration of willingness to obtain admittance, which willingness of mind I suppose did not want in him whose answer was to the voice of his heavenly calling, “Behold I am come to do thy will.” And had it been for him as it is for us expedient to receive his commission signed with the hands of men, to seek it might better have beseemed his humility than it doth our boldness to reprehend them of pride and ambition that make no worse kind of suits than by letters of information.




[13.]Himself in calling his Apostles prevented all cogitations of theirs that way, to the end it might truly be said of them, “Ye chose not me, but I of my own voluntary motion made choice of you.” Which kind of undesired nomination to ecclesiastical places befell divers of the most famous amongst the ancient Fathers of the Church in a clean contrary consideration. For our Saviour’s election respected not any merit or worth, but took them which were farthest off from likelihood of fitness, that afterwards their supernatural ability and performance beyond hope might cause the greater admiration; whereas in the other, mere admiration of their singular and rare virtues was the reason why honours were enforced upon them, which they of meekness and modesty did what they could to avoid. But did they ever judge it a thing unlawful to wish or desire the office, the only charge and bare function of the ministry? Towards which labour what doth the blessed Apostle else but encourage saying, “He which desireth it is desirous of a good work?” What doth he else by such sentences but stir, kindle, and inflame ambition, if I may term that desire ambition, which coveteth more to testify love by painfulness in God’s service, than to reap any other benefit?

[14.]Although of the very honour itself, and of other emoluments annexed to such labours, for more encouragement of man’s industry, we are not so to conceive neither, as if no affection could be cast towards them without offence. Only as the wise man giveth counsel, “Seek not to be made a judge, lest thou be not able to take away iniquity, and lest thou fearing the person of the mighty shouldest commit an offence against thine uprightness;” so it always behoveth men to take good heed, lest affection to that which hath in it as well difficulty as goodness sophisticate the true and sincere judgment which beforehand they ought to have of their own ability, for want whereof many forward minds have found instead of contentment repentance. But forasmuch as hardness of things in themselves most excellent cooleth the fervency of men’s desires, unless there be somewhat naturally acceptable to incite labour, (for both the  method of speculative knowledge doth by things which we sensibly perceive conduct to that which is in nature more certain though less sensible,
 and the method of virtuous actions is also to train beginners at the first by things acceptable unto the taste of natural appetite, till our minds at the length be settled to embrace things precious in the eye of reason, merely and wholly for their own sakes,) howsoever inordinate desires do hereby take occasion to abuse the polity of God and nature, either affecting without worth, or procuring by unseemly means, that which was instituted and should be reserved for better minds to obtain by more approved courses; in which consideration the emperors Anthemius and Leo did worthily oppose against such ambitious practices that ancient famous constitution wherein they have these sentences: “Let not a prelate be ordained for reward or upon request, who should be so far sequestered from all ambition that they which advance him might be fain to search where he hideth himself, to entreat him drawing back, and to follow him till importunity have made him yield; let nothing promote him but his excuses to avoid the burden; they are unworthy of that vocation which are not thereunto brought unwillingly:” notwithstanding we ought not therefore with the odious name of ambition to traduce and draw into hatred every poor request or suit wherein men may seem to affect honour; seeing that ambition and modesty do not always so much differ in the mark they shoot at as in the manner of their prosecutions.

Yea even in this may be error also, if we still imagine them least ambitious which most forbear to stir either hand or foot towards their own preferments. For there are that make an idol of their great sufficiency, and because they surmise the place should be happy that might enjoy them, they walk every where like grave pageants observing whether men do not  wonder why so small account is made of so rare worthiness,
 and in case any other man’s advancement be mentioned they either smile or blush at the marvellous folly of the world which seeth not where dignities should offer themselves.

Seeing therefore that suits after spiritual functions may be as ambitiously forborne as prosecuted, it remaineth that the evenest line of moderation between both is neither to follow them without conscience, nor of pride to withdraw ourselves utterly from them.


\section*{Of Degrees whereby the power of Order is distinguished, and concerning the Attire of ministers.}
LXXVIII. It pleased Almighty God to choose to himself for discharge of the legal ministry one only tribe out of twelve others, the tribe of Levi, not all unto every divine service, but Aaron and his sons to one charge, the rest of that sanctified tribe to another. With what solemnities they were admitted into their functions, in what manner Aaron and his successors the high priests ascended every Sabboth and festival day, offered, and ministered in the temple; with what sin-offering once every year they reconciled first themselves and their own house, afterwards the people unto God; how they confessed all the iniquities of the children of Israel, laid all their trespasses upon the head of a sacred goat, and so carried them out of the city; how they purged the holy place from all uncleanness, with what reverence they entered within the veil, presented themselves before the mercy seat, and consulted with the oracle of God: what service the other priests did continually in the holy place, how they ministered about the lamps, morning and evening, how every Sabboth they placed on the table of the Lord those twelve loaves with pure incense in perpetual remembrance of that mercy which the fathers the twelve tribes had found by the providence of God for their food, when hunger caused them to leave their natural soil and to seek for sustenance in Egypt; how they employed themselves in sacrifice day by day; finally what offices the Levites discharged, and what duties the rest did execute, it were a  labour too long to enter into if I should collect that which Scriptures and other ancient records do mention.

Besides these there were indifferently out of all tribes from time to time, some called of God, as Prophets foreshowing them things to come, and giving them counsel in such particulars as they could not be directed in by the law; some chosen of men to read, study, and interpret the Law of God, as the sons or scholars of the old Prophets, in whose room afterwards Scribes and expounders of the law succeeded.

And because where so great variety is, if there should be equality, confusion would follow, the Levites were in all their service at the appointment and direction of the sons of Aaron or priests, they subject to the principal guides and leaders of their own order, and they all in obedience under the high priest. Which difference doth also manifest itself in the very titles that men for honour’s sake gave unto them, terming Aaron and his successors High or Great; the ancients over the companies of priests, arch-priests; prophets, fathers; scribes and interpreters of the Law, masters.

[2.]Touching the ministry of the Gospel of Jesus Christ: the whole body of the Church being divided into laity and clergy, the clergy are either presbyters or deacons.

I rather term the one sort Presbyters than Priests, because in a matter of so small moment I would not willingly offend their ears to whom the name of Priesthood is odious though  without cause. For as things are distinguished one from another by those true essential forms which being really and actually in them do not only give them the very last and highest degree of their natural perfection, but are also the knot, foundation and root whereupon all other inferior perfections depend, so if they that first do impose names did always understand exactly the nature of that which they nominate, it may be that then by hearing the terms of vulgar speech we should still be taught what the things themselves most properly are. But because words have so many artificers by whom they are made, and the things whereunto we apply them are fraught with so many varieties, it is not always apparent what the first inventors respected, much less what every man’s inward conceit is which useth their words. For any thing myself can discern herein, I suppose that they which have bent their study to  search more diligently such matters do for the most part find that names advisedly given had either regard unto that which is naturally most proper; or if perhaps to some other specialty, to that which is sensibly most eminent in the thing signified; and concerning popular use of words that which the wisdom of their inventors did intend thereby is not commonly thought of, but by the name the thing altogether conceived in gross, as may appear in that if you ask of the common sort what any certain word, for example, what a Priest doth signify, their manner is not to answer, a Priest is a clergyman which offereth sacrifice to God, but they show some particular person whom they use to call by that name. And, if we list to descend to grammar, we are told by masters in those schools that the word Priest hath his right place ἐπὶ του̑ ψιλω̑ς προεστω̑τος τη̑ς θεραπείας του̑ Θεου̑, “in him whose mere function or charge is the service of God.” Howbeit because the most eminent part both of Heathenish and Jewish service did consist in sacrifice, when learned men declare what the word Priest doth properly signify according to the mind of the first imposer of that name, their ordinary scholies do well expound it to imply sacrifice.

Seeing then that sacrifice is now no part of the church ministry, how should the name of Priesthood be thereunto rightly applied? Surely even as St. Paul applieth the name of Flesh unto that very substance of fishes which hath a proportionable correspondence to flesh, although it be in nature another thing. Whereupon when philosophers will speak warily, they make a difference between flesh in one sort of living creatures and that other substance in the rest which hath but a kind of analogy to flesh: the Apostle contrariwise having matter of greater importance whereof to speak nameth indifferently both flesh. The Fathers of the Church of Christ with like security of speech call usually the ministry of the Gospel Priesthood in regard of that which the Gospel hath  proportionable to ancient sacrifices,
 namely the Communion of the blessed Body and Blood of Christ, although it have properly now no sacrifice. As for the people when they hear the name it draweth no more their minds to any cogitation of sacrifice, than the name of a Senator or of an Alderman causeth them to think upon old age or to imagine that every one so termed must needs be ancient because years were respected in the first nomination of both.

[3.]Wherefore to pass by the name, let them use what dialect they will, whether we call it a Priesthood, a Presbytership, or a Ministry it skilleth not: although in truth the word Presbyter doth seem more fit, and in propriety of speech more agreeable than Priest with the drift of the whole Gospel of Jesus Christ. For what are they that embrace the Gospel but sons of God? What are churches but his families? Seeing therefore we receive the adoption and state of sons by their ministry whom God hath chosen out for that purpose, seeing also that when we are the sons of God, our continuance is still under their care which were our progenitors, what better title could there be given them than the reverend name of Presbyters or fatherly guides? The Holy Ghost throughout the body of the New Testament making so much mention of them doth not any where call them Priests. The prophet Esay I grant doth; but in such sort as the ancient fathers, by way of analogy. A Presbyter according to the proper meaning of the New Testament is “he unto whom our Saviour Christ hath communicated the power of spiritual procreation.” Out of twelve patriarchs issued the whole multitude  of Israel according to the flesh.
 And according to the mystery of heavenly birth our Lord’s Apostles we all acknowledge to be the patriarchs of his whole Church. St. John therefore beheld sitting about the throne of God in heaven four and twenty Presbyters, the one half fathers of the old, the other of the new Jerusalem. In which respect the Apostles likewise gave themselves the same title, albeit that name were not proper but common unto them with others.

[4.]For of Presbyters some were greater some less in power, and that by our Saviour’s own appointment; the greater they which received fulness of spiritual power, the less they to whom less was granted. The Apostles’ peculiar charge was to publish the Gospel of Christ unto all nations, and to deliver them his ordinances received by immediate revelation from himself. Which preeminence excepted, to all other offices and duties incident into their order it was in them to ordain and consecrate whomsoever they thought meet, even as our Saviour did himself assign seventy other of his own disciples inferior presbyters, whose commission to preach and baptize was the same which the Apostles had. Whereas therefore we find that the very first sermon which the Apostles did publicly make was the conversion of above three thousand souls, unto whom there were every day more and more added, they having no open place permitted them for the exercise of Christian religion, think we that twelve were sufficient to teach and administer sacraments in so many private places as so great a multitude of people did require? This harvest our Saviour no doubt foreseeing provided accordingly labourers for it beforehand. By which means it came to pass that the growth of that church being so great and so sudden, they had notwithstanding in a readiness presbyters enough to furnish it. And therefore the history doth make no mention by what occasion presbyters were instituted in Jerusalem, only we read of things which they did, and how the like were made afterwards elsewhere.

[5.]To these two degrees appointed of our Lord and  Saviour Christ his Apostles soon after annexed deacons.
 Deacons therefore must know, saith Cyprian, that our Lord himself did elect Apostles, but deacons after his ascension into heaven the Apostles ordained. Deacons were stewards of the Church, unto whom at the first was committed the distribution of church goods, the care of providing therewith for the poor, and the charge to see that all things of expense might be religiously and faithfully dealt in. A part also of their office was attendance upon their presbyters at the time of divine service. For which cause Ignatius to set forth the dignity of their calling saith, that they are in such case to the bishop as if angelical powers did serve him.

These only being the uses for which deacons were first made, if the church hath sithence extended their ministry farther than the circuit of their labour at the first was drawn, we are not herein to think the ordinance of Scripture violated except there appear some prohibition which hath abridged the Church of that liberty. Which I note chiefly in regard of them to whom it seemeth a thing so monstrous that deacons should sometime be licensed to preach, whose institution was at the first to another end. To charge them for this as  men not contented with their own vocations and as breakers into that which appertaineth unto others is very hard. For when they are thereunto once admitted, it is a part of their own vocation, it appertaineth now unto them as well as others, neither is it intrusion for them to do it being in such sort called, but rather in us it were temerity to blame them for doing it. Suppose we the office of teaching to be so repugnant unto the office of deaconship that they cannot concur in one and the same person? What was there done in the Church by deacons which the Apostles did not first discharge being teachers?

Yea but the Apostles found the burden of teaching so heavy that they judged it meet to cut off that other charge and to have deacons which might undertake it. Be it so. The multitude of Christians increasing in Jerusalem and waxing great, it was too much for the Apostles to teach and to minister unto tables also. The former was not to be slacked that this latter might be followed. Therefore unto this they appointed others. Whereupon we may rightly ground this axiom, that when the subject wherein one man’s labours of sundry kinds are employed doth wax so great that the same men are no  longer able to manage it sufficiently as before, the most natural way to help this is by dividing their charge into slips and ordaining of under officers, as our Saviour under twelve Apostles seventy Presbyters, and the Apostles by his example seven Deacons to be under both. Neither ought it to seem less reasonable, that when the same men are sufficient both to continue in that which they do and also to undertake somewhat more, a combination be admitted in this case, as well as division in the former. We may not therefore disallow it in the church of Geneva, that Calvin and Beza were made both pastors and readers of divinity, being men so able to discharge both. To say they did not content themselves with their pastoral vocations, but break into that which belonged to others; to allege against them, “He that exhorteth in exhortation,” as against us, “He that distributeth in simplicity” is alleged in great dislike of granting license for deacons to preach, were very hard.

The ancient custom of the Church was to yield the poor much relief especially widows. But as poor people are always querulous and apt to think themselves less respected than they should be, we see that when the Apostles did what they could without hinderance to their weightier business, yet there were which grudged that others had too much and they too little, the Grecian widows shorter commons than the Hebrews. By means whereof the Apostles saw it meet to ordain Deacons. Now tract of time having clean worn out those first occasions for which the deaconship was then most necessary, it might the better be afterwards extended to other services, and so remain as at this present day a degree in the clergy of God which the Apostles of Christ did institute.




That the first seven Deacons were chosen out of the seventy disciples is an error in Epiphanius. For to draw men from places of weightier unto rooms of meaner labour had not been fit. The Apostles to the end they might follow teaching with more freedom committed the ministry of tables unto deacons. And shall we think they judged it expedient to choose so many out of those seventy to be ministers unto tables, when Christ himself had before made them teachers?

It appeareth therefore how long these three degrees of ecclesiastical order have continued in the Church of Christ, the highest and largest that which the Apostles, the next that which Presbyters, and the lowest that which Deacons had.

[6.]Touching Prophets, they were such men as having otherwise learned the Gospel had from above bestowed upon them a special gift of expounding Scriptures and of foreshowing things to come. Of this sort Agabus was and besides him in Jerusalem sundry others, who notwithstanding are not therefore to be reckoned with the clergy, because no man’s gifts or qualities can make him a minister of holy things, unless ordination do give him power. And we no where find Prophets to have been made by ordination, but all whom the Church did ordain were either to serve as presbyters or as deacons.

[7.]Evangelists were presbyters of principal sufficiency whom the Apostles sent abroad and used as agents in ecclesiastical affairs wheresoever they saw need. They whom we find to have been named in Scripture Evangelists as Ananias, Apollos, Timothy and others were thus employed. And  concerning Evangelists afterwards in Trajan’s days,
 the history ecclesiastical noteth that many of the Apostles’ disciples and scholars which were then alive and did with singular love of wisdom affect the heavenly word of God, to show their willing minds in executing that which Christ first of all required at the hands of men, they sold their possessions, gave them to the poor, and betaking themselves to travail undertook the labour of Evangelists, that is they painfully preached Christ and delivered the Gospel to them who as yet had never heard the doctrine of faith.

Finally whom the Apostle nameth Pastors and Teachers what other were they than Presbyters also, howbeit settled in some certain charge and thereby differing from Evangelists?

[8.]I beseech them therefore which have hitherto troubled the Church with questions about degrees and offices of ecclesiastical calling, because they principally ground themselves upon two places, that all partiality laid aside they would sincerely weigh and examine whether they have not misinterpreted both places, and all by surmising incompatible offices where nothing is meant but sundry graces, gifts, and abilities which Christ bestowed. To them of Corinth his words are these: “God placed in the Church first of all some Apostles, secondly Prophets, thirdly teachers, after them powers, then gifts of cures, aids, governments, kinds  of languages.
 Are all Apostles? Are all Prophets? Are all Teachers? Is there power in all? Have all grace to cure? Do all speak with tongues? Can all interpret? But be you desirous of the better graces.” They which plainly discern first that some one general thing there is which the Apostle doth here divide into all these branches, and do secondly conceive that general to be church offices, besides a number of other difficulties, can by no means possibly deny but that many of these might concur in one man, and peradventure in some one all, which mixture notwithstanding their form of discipline doth most shun. On the other side admit that communicants of special infused grace, for the benefit of members knit into one body, the Church of Christ, are here spoken of, which was in truth the plain drift of that whole discourse, and see if every thing do not answer in due place with that fitness which showeth easily what is likeliest to have been meant. For why are Apostles the first but because unto them was granted the revelation of all truth from Christ immediately? Why Prophets the second, but because they had of some things knowledge in the same manner? Teachers the next, because whatsoever was known to them it came by hearing, yet God withal made them able to instruct, which every one could not do that was taught. After gifts of edification there follow general abilities to work things above nature, grace to cure men of bodily diseases, supplies against occurrent defects and impediments, dexterities to govern and direct by counsel, finally aptness to speak or interpret foreign tongues. Which graces not poured out equally but diversely sorted and given, were a cause why not only they all did furnish up the whole body but each benefit and help other.

[9.]Again the same Apostle otherwhere in like sort, “To every one of us is given grace according to the measure of the gift of Christ. Wherefore he saith, When he ascended up on high he led captivity captive and gave gifts unto men. He therefore gave some Apostles and some Prophets and some Evangelists and some Pastors and Teachers, for the gathering together of saints, for the work of the ministry, for the edification of the body of Christ.” In this place  none but gifts of instruction are expressed.
 And because of teachers some were Evangelists which neither had any part of their knowledge by revelation as the Prophets and yet in ability to teach were far beyond other Pastors, they are as having received one way less than Prophets and another way more than Teachers set accordingly between both. For the Apostle doth in neither place respect what any of them were by office or power given them through ordination, but what by grace they all had obtained through miraculous infusion of the Holy Ghost. For in Christian religion this being the ground of our whole belief, that the promises which God of old had made by his Prophets concerning the wonderful gifts and graces of the Holy Ghost, wherewith the reign of the true Messias should be made glorious, were immediately after our Lord’s ascension performed, there is no one thing whereof the Apostles did take more often occasion to speak. Out of men thus endued with gifts of the Spirit upon their conversion to Christian faith the church had her ministers chosen, unto whom was given ecclesiastical power by ordination. Now because the Apostle in reckoning degrees and varieties of grace doth mention Pastors and Teachers, although he mention them not in respect of their ordination to exercise the ministry, but as examples of men especially enriched with the gifts of the Holy Ghost, divers learned and skilful men have so taken it as if those places did intend to teach what orders of ecclesiastical persons there ought to be in the Church of Christ; which thing we are not to learn from thence but out of other parts of Holy Scripture, whereby it clearly appeareth that churches apostolic did know but three degrees in the power of ecclesiastical order, at the first Apostles, Presbyters, and Deacons, afterwards instead of Apostles Bishops, concerning whose order we are to speak in the seventh book.

[10.]There is an error which beguileth many who much entangle both themselves and others by not distinguishing Services, Offices, and Orders ecclesiastical, the first of which three and in part the second may be executed by the laity, whereas none have or can have the third but the clergy. Catechists, Exorcists, Readers, Singers, and the rest of like sort, if the nature only of their labours and pains be considered, 
 may in that respect seem clergymen, even as the Fathers for that cause term them usually Clerks; as also in regard of the end whereunto they were trained up, which was, to be ordered when years and experience should make them able. Notwithstanding inasmuch as they no way differed from others of the laity longer than during that work of service which at any time they might give over, being thereunto but admitted not tied by irrevocable ordination, we find them always exactly severed from that body whereof those three before rehearsed orders alone are natural parts.

[11.]Touching Widows, of whom some men are persuaded, that if such as St. Paul describeth may be gotten we ought to retain them in the Church for ever; certain mean services there were of attendance, as about women at the time of their baptism, about the bodies of the sick and dead, about the necessities of travellers, wayfaring men, and such like, wherein the Church did commonly use them when need required, because they lived of the alms of the Church and were fitted for such purposes. St. Paul doth therefore to avoid scandal require that none but women well experienced and virtuously given, neither any under threescore year of  age should be admitted of that number.
 Widows were never in the Church so highly esteemed as Virgins. But seeing neither of them did or could receive ordination, to make them ecclesiastical persons were absurd.

[12.]The ancientest therefore of the Fathers mention those three degrees of ecclesiastical order specified and no more. “When your captains,” saith Tertullian, “that is to say the Deacons, Presbyters and Bishops fly, who shall teach the laity that they must be constant?” Again, “What should I mention laymen,” saith Optatus, “yea or divers of the ministry itself? To what purpose Deacons which are in the third, or presbyters in the second degree of priesthood, when the very heads and princes of all, even certain of the Bishops themselves, were content to redeem life with the loss of heaven?” Heaps of allegations in a case so evident and plain are needless. I may securely therefore conclude that there are at this day in the church of England no other than the same degrees of ecclesiastical order, namely Bishops, Presbyters, and Deacons, which had their beginning from Christ and his blessed Apostles themselves.

As for Deans, Prebendaries, Parsons, Vicars, Curates, Archdeacons, Chancellors, Officials, Commissaries, and such other the like names, which being not found in Holy Scripture, we have been thereby through some men’s error thought to allow of ecclesiastical degrees not known nor ever heard of in the better ages of former times; all these are in truth but titles of office whereunto partly ecclesiastical persons, and partly others are in sundry forms and conditions admitted as the state of the Church doth need, degrees of order still continuing the same they were from the first beginning.

[13.]Now what habit or attire doth beseem each order to use in the course of common life both for the gravity of his  place and for example sake to other men is a matter frivolous to be disputed of.
 A small measure of wisdom may serve to teach them how they should cut their coats. But seeing all well-ordered polities have ever judged it meet and fit by certain special distinct ornaments to sever each sort of men from other when they are in public, to the end that all may receive such complements of civil honour as are due to their rooms and callings even where their persons are not known, it argueth a disproportioned mind in them whom so decent orders displease.


\section*{Of Oblations, Foundations, Endowments, Tithes, all intended for perpetuity of religion; which purpose being chiefly fulfilled by the clergy’s certain and sufficient maintenance, must needs by alienation of church livings be made frustrate.}
LXXIX. We might somewhat marvel what the Apostle St. Paul should mean to say that “covetousness is idolatry,” if the daily practice of men did not shew that whereas nature requireth God to be honoured with wealth, we honour for the most part wealth as God. Fain we would teach ourselves to believe that for worldly goods it sufficeth frugally and honestly to use them to our own benefit, without detriment and hurt of others; or if we go a degree farther, and perhaps convert some small contemptible portion thereof to charitable uses, the whole duty which we owe unto God herein is fully satisfied. But forasmuch as we cannot rightly honour God unless both our souls and bodies be sometime employed merely in his service; again sith we know that religion requireth at our hands the taking away of so great a part of the time of our lives quite and clean from our own business and the bestowing of the same in his, suppose we that nothing of our wealth and substance is immediately due to God, but all our own to bestow and spend as ourselves think  meet? Are not our riches as well his as the days of our life are his?
 Wherefore unless with part we acknowledge his supreme dominion by whose benevolence we have the whole, how give we honour to whom honour belongeth, or how hath God the things that are God’s? I would know what nation in the world did ever honour God and not think it a point of their duty to do him honour with their very goods. So that this we may boldly set down as a principle clear in nature, an axiom which ought not to be called in question, a truth manifest and infallible, that men are eternally bound to honour God with their substance in token of thankful acknowledgment that all they have is from him. To honour him with our worldly goods, not only by spending them in lawful manner, and by using them without offence, but also by alienating from ourselves some reasonable part or portion thereof and by offering up the same to him as a sign that we gladly confess his sole and sovereign dominion over all, is a duty which all men are bound unto and a part of that very worship of God which as the law of God and nature itself requireth, so we are the rather to think all men no less strictly bound thereunto than to any other natural duty, inasmuch as the hearts of men do so cleave to these earthly things, so much admire them for the sway they have in the world, impute them so generally either to nature or to chance and fortune, so little think upon the grace and providence from which they come, that unless by a kind of continual tribute we did acknowledge God’s dominion, it may be doubted that in short time men would learn to forget whose tenants they are, and imagine that the world is their own absolute free and independent inheritance.

[2.]Now concerning the kind or quality of gifts which God receiveth in that sort, we are to consider them partly as first they proceed from us, and partly as afterwards they are to serve for divine uses. In that they are testimonies of our affection towards God, there is no doubt but such they should be as beseemeth most his glory to whom we offer them. In this respect the fatness of Abel’s sacrifice is commended, the flower of all men’s increase assigned to God by Salomon,  the gifts and donations of the people rejected as oft as their cold affection to God-ward made their presents to be little worth.
 Somewhat the heathens saw touching that which was herein fit, and therefore they unto their gods did not think they might consecrate any thing which was impure or unsound, or already given, or else not truly their own to give.

[3.]Again in regard of use, forasmuch as we know that God hath himself no need of worldly commodities, but taketh them because it is our good to be so exercised, and with no other intent accepteth them but to have them used for the endless continuance of religion, there is no place left of doubt or controversy but that we in the choice of our gifts are to level at the same mark, and to frame ourselves to his known intents and purposes. Whether we give unto God therefore that which himself by commandment requireth; or that which the public consent of the Church thinketh good to allot; or that which every man’s private devotion doth best like, inasmuch as the gift which we offer proceedeth not only as a testimony of our affection towards God, but also as a mean to uphold religion, the exercise whereof cannot stand without the help of temporal commodities; if all men be taught of nature to wish and as much as in them lieth to procure the perpetuity of good things, if for that very cause we honour and admire their wisdom who having been founders of commonweals could devise how to make the benefit they left behind them durable, if especially in this respect we prefer Lycurgus before Solon and the Spartan before the Athenian polity, it must needs follow that as we do unto God very acceptable service in honouring him with our substance, so our service that way is then most acceptable when it tendeth to perpetuity.

[4.]The first permanent donations of honour in this kind are temples. Which works do so much set forward the exercise of religion, that while the world was in love with  religion it gave to no sort greater reverence than to whom it could point and say,
 “These are the men that have built us synagogues.” But of churches we have spoken sufficiently heretofore.

[5.]The next things to churches are the ornaments of churches, memorials which men’s devotion hath added to remain in the treasure of God’s house not only for uses wherein the exercise of religion presently needeth them, but also partly for supply of future casual necessities whereunto the Church is on earth subject, and partly to the end that while they are kept they may continually serve as testimonies giving all men to understand that God hath in every age and nation such as think it no burden to honour him with their substance. The riches first of the tabernacle of God and then of the temple of Jerusalem arising out of voluntary gifts and donations were as we commonly speak a nemo scit, the value of them above that which any man would imagine. After that the tabernacle was made, furnished with all necessaries and set up, although in the wilderness their ability could not possibly be great, the very metal of those vessels which the princes of the twelve tribes gave to God for their first presents amounted even then to two thousand and four hundred shekels of silver a hundred and twenty shekels of gold, every shekel weighing half an ounce. What was given to the temple which Salomon erected, we may partly conjecture, when over and besides wood, marble, iron, brass, vestments, precious stones, and money, the sum which David delivered into Salomon’s hands for that purpose was of gold in mass eight thousand and of silver seventeen thousand cichars, every cichar containing a thousand and eight hundred shekels which riseth to nine hundred ounces in every one cichar: whereas the whole charge of the tabernacle did not amount unto thirty cichars. After  their return out of Babylon they were not presently in case to make their second temple of equal magnificence and glory with that which the enemy had destroyed. Notwithstanding what they could they did. Insomuch that the building finished, there remained in the coffers of the Church to uphold the fabric thereof six hundred and fifty cichars of silver, one hundred of gold. Whereunto was added by Nehemias of his own gift a thousand drachms of gold, fifty vessels of silver, five hundred and thirty priests’ vestments, by other the princes of the fathers twenty thousand drachms of gold, two thousand and two hundred pieces of silver; by the rest of the people twenty thousand of gold, two thousand of silver, threescore and seven attires of priests. And they furthermore bound themselves towards other charges to give by the poll in what part of the world soever they should dwell the third of a shekel, that is to say the sixth part of an ounce, yearly. This out of foreign provinces they always sent in gold. Whereof Mithridates is said to have taken up by the way before it could pass to Jerusalem from Asia in one adventure eight hundred talents; Crassus after that to have borrowed of the temple itself eight thousand: at which time Eleazar having both many other rich ornaments and all the tapestry of the temple under his custody thought it the safest way to grow unto some composition, and so to redeem the residue by  parting with a certain beam of gold about seven hundred and a half in weight,
 a prey sufficient for one man as he thought who had never bargained with Crassus till then, and therefore upon the confidence of a solemn oath that no more should be looked for he simply delivered up a large morsel, whereby the value of that which remained was betrayed and the whole lost.

[6.]Such being the casualties whereunto moveable treasures are subject, the Law of Moses did both require eight and twenty cities together with their fields and whole territories in the land of Jewry to be reserved for God himself, and not only provide for the liberty of farther additions if men of their own accord should think good, but also for the safe preservation thereof unto all posterities, that no man’s avarice or fraud by defeating so virtuous intents might discourage from like purposes. God’s third endowment did therefore of old consist in lands.

[7.]Furthermore some cause no doubt there is why besides sundry other more rare donations of uncertain rate, the tenth should be thought a revenue so natural to be allotted out unto God. For of the spoils which Abraham had taken in war he delivered unto Melchisedec the tithes. The vow of Jacob at such time as he took his journey towards Haran was, “If God will be with me and will keep me in this voyage which I am to go, and will give me bread to eat and clothes to put on, so that I may return to my father’s house in safety, then shall the Lord be my God, and this stone which I have set up as a pillar the same shall be God’s house, and of all thou shalt give me I will give unto thee the tithe.” And as Abraham gave voluntarily, as Jacob vowed to give God tithes, so the Law of Moyses did require at the hands of all men the selfsame kind of tribute, the tenth of their corn, wine, oil, fruit, cattle and whatsoever increase his heavenly providence should send. Insomuch that Painims being herein followers of their steps paid tithes likewise.



Imagine we that this was for no cause done,
 or that there was not some special inducement to judge the tenth of our worldly profits the most convenient for God’s portion? Are not all things by him created in such sort that the forms which give them their distinction are number, their operations measure, and their matter weight? Three being the mystical number of God’s unsearchable perfection within himself; seven the number whereby our own perfections through grace are most ordered; and ten the number of nature’s perfections (for the beauty of nature is order, and the foundation of order number, and of number ten the highest we can rise unto without iteration of numbers under it) could nature better acknowledge the power of the God of nature than by assigning unto him that quantity which is the continent of all she possesseth? There are in Philo the Jew many arguments to shew the great congruity and fitness of this number in things consecrated unto God.

[8.]But because over-nice and curious speculations become not the earnestness of holy things, I omit what might be farther observed as well out of others as out of him touching the quantity of this general sacred tribute, whereby it cometh to  pass that the meanest and the very poorest amongst men yielding unto God as much in proportion as the greatest, and many times in affection more, have this as a sensible token always assuring their minds, that in his sight from whom all good is expected, they are concerning acceptation, protection, divine privileges and preeminences whatsoever, equals and peers with them unto whom they are otherwise in earthly respects inferiors; being furthermore well assured that the top as it were thus presented to God is neither lost nor unfruitfully bestowed, but doth sanctify to them again the whole mass, and that he by receiving a little undertaketh to bless all. In which consideration the Jews were accustomed to name their tithes the hedge of their riches. Albeit a hedge do only fence and preserve that which is contained, whereas their tithes and offerings did more, because they procured increase of the heap out of which they were taken. God demanded no such debt for his own need but for their only benefit that owe it. Wherefore detaining the same they hurt not him whom they wrong, and themselves whom they think they relieve they wound, except men will haply affirm that God did by fair speeches and large promises delude the world in saying, “Bring ye all the tithes into the storehouse that there may be meat in mine house,” (deal truly, defraud not God of his due, but bring all,) “and prove if I will not open unto you the windows of heaven and pour down upon you an immeasurable blessing.” That which St. James hath concerning the effect of our prayers unto God is for the most part of like moment in our gifts. We pray and obtain not, because he which knoweth our hearts doth see our desires are evil. In like manner we give and we are not the more accepted, because he beholdeth how unwisely we spill our gifts in the bringing. It is to him which needeth nothing all one whether any thing or nothing be given him. But for our own good it always behoveth that whatsoever we offer up into his hands  we bring it seasoned with this cogitation,
 “Thou Lord art worthy of all honour.”

[9.]With the Church of Christ touching these matters it standeth as it did with the whole world before Moses. Whereupon for many years men being desirous to honour God in the same manner as other virtuous and holy personages before had done, both during the time of their life and if farther ability did serve by such device as might cause their works of piety to remain always, it came by these means to pass that the Church from time to time had treasure proportionable unto the poorer or wealthier estate of Christian men. And as soon as the state of the Church could admit thereof, they easily condescended to think it most natural and most fit that God should receive as before of all men his ancient accustomed revenues of tithes.

[10.]Thus therefore both God and nature have taught to convert things temporal to eternal uses, and to provide for the perpetuity of religion even by that which is most transitory. For to the end that in worth and value there might be no abatement of any thing once assigned to such purposes, the law requireth precisely the best of that we possess, and to prevent all damages by way of commutation, where instead of natural commodities or other rights the price of them might be taken, the Law of Moses determined their rates, and the payments to be always made by the shekel of the sanctuary wherein there was great advantage of weight above the ordinary current shekel. The truest and surest way for God to have always his own is by making him payment in kind out of the very selfsame riches which through his gracious benediction the earth doth continually yield. This where it may be without inconvenience is for every man’s conscience safe. That which cometh from God to us by the natural course of his providence which we know to be innocent and pure is perhaps best accepted, because least spotted with the stain of unlawful or indirect procurement. Besides whereas prices daily change, nature which commonly is one must needs be the most indifferent and permanent standard between God and man.

[11.]But the main foundation of all, whereupon the security of these things dependeth, as far as any thing may be ascertained  amongst men,
 is that the title and right which man had in every of them before donation, doth by the act and from the time of any such donation, dedication or grant, remain the proper possession of God till the world’s end, unless himself renounce or relinquish it. For if equity have taught us that every one ought to enjoy his own; that what is ours no other can alienate from us but with our own deliberate consent; finally that no man having passed his consent or deed may change it to the prejudice of any other, should we presume to deal with God worse than God hath allowed any man to deal with us?

[12.]Albeit therefore we be now free from the Law of Moyses and consequently not thereby bound to the payment of tithes, yet because nature hath taught men to honour God with their substance, and Scripture hath left us an example of that particular proportion which for moral considerations hath been thought fittest by him whose wisdom could best judge, furthermore seeing that the Church of Christ hath long sithence entered into like obligation, it seemeth in these days a question altogether vain and superfluous whether tithes be a matter of divine right: because howsoever at the first it might have been thought doubtful, our case is clearly the same now with theirs unto whom St. Peter sometime spake saying, “While it was whole it was whole thine.” When our tithes might have probably seemed our own, we had colour of liberty to use them as we ourselves saw good. But having made them his whose they are, let us be warned by other men’s example what it is νοσϕίσασθαι, to wash or clip that coin which hath on it the mark of God.




[13.]For that all these are his possessions and that he doth himself so reckon them appeareth by the form of his own speeches. Touching gifts and oblations, “Thou shalt give them me;” touching oratories and churches, “My house shall be called the house of prayer;” touching tithes, “Will a man spoil God? yet behold even me your God ye have spoiled, notwithstanding ye ask wherein, as though ye were ignorant what injury there hath been offered in tithes, ye are heavily accursed because with a kind of public consent ye have joined yourselves in one to rob me, imagining the commonness of your offence to be every man’s particular justification;” touching lands, “Ye shall offer to the Lord a sacred portion of ground, and that sacred portion shall belong to the priests.”

[14.]Neither did God only thus ordain amongst the Jews, but the very purpose intent and meaning of all that have honoured him with their substance was to invest him with the property of those benefits the use whereof must needs be committed to the hands of men. In which respect the style of ancient grants and charters is “We have given unto God both for us and our heirs for ever:” yea “We know,” saith Charles the Great, “that the goods of the Church are the sacred endowments of God, to the Lord our God we offer and dedicate whatsoever we deliver unto his Church.” Whereupon the laws imperial do likewise divide all things in such sort that they make some to belong by right of nature indifferently unto every man, some to be the certain goods and possessions of commonweals, some to appertain unto several corporations or companies of men, some to be  privately men’s own in particular, and some to be separated quite from all men,
 which last branch compriseth things sacred and holy, because thereof God alone is owner. The sequel of which received opinion as well without as within the walls of the house of God touching such possessions hath been ever, that there is not an act more honourable than by all means to amplify and to defend the patrimony of religion, not any more impious and hateful than to impair those possessions which men in former times when they gave unto holy uses were wont at the altar of God and in the presence of their ghostly superiors to make as they thought inviolable by words of fearful execration, saying, “These things we offer to God; from whom if any take them away (which we hope no man will attempt to do) but if any shall, let his account be without favour in the last day, when he cometh to receive the doom which is due for sacrilege against that Lord and God unto whom we dedicate the same.”

The best and most renowned Prelates of the Church of Christ have in this consideration rather sustained the wrath than yielded to satisfy the hard desire of their greatest commanders on earth coveting with ill advice and counsel that which they willingly should have suffered God to enjoy. There are of Martyrs whom posterity doth much honour, for that having under their hands the custody of such treasures they could by virtuous delusion invent how to save them from prey, even when the safety of their own lives they gladly neglected; as one sometime an Archdeacon under Xystus the Bishop of Rome did, whom when his judge understood to be one of the church-stewards, thirst of blood began to slake and another humour to work, which first by a favourable countenance  and then by quiet speech did thus calmly disclose itself: “You that profess the Christian religion make great complaint of the wonderful cruelty we shew towards you. Neither peradventure altogether without cause. But for myself, I am far from any such bloody purpose. Ye are not so willing to live, as I unwilling that out of these lips should proceed any capital sentence against you. Your bishops are said to have rich vessels of gold and silver, which they use in the exercise of their religion, besides the fame is that numbers sell away their lands and livings, the huge prices whereof are brought to your church-coffers, by which means the devotion that maketh them and their whole posterity poor must needs mightily enrich you, whose God we know was no coiner of money, but left behind him many wholesome and good precepts, as namely that Cæsar should have of you the things that are fit for and due to Cæsar. His wars are costly and chargeable unto him. That which you suffer to rust in corners the affairs of the commonwealth do need. Your profession is not to make account of things transitory. And yet if ye  can be contented but to forego that which ye care not for,
 I dare undertake to warrant you both safety of life and freedom of using your conscience, a thing more acceptable to you than wealth.” Which fair parley the happy Martyr quietly hearing, and perceiving it necessary to make some shift for the safe concealment of that which being now desired was not unlikely to be more narrowly afterwards sought, he craved respite for three days to gather the riches of the Church together, in which space against the time the governor should come to the doors of the temple big with hope to receive his prey, a miserable rank of poor, lame, and impotent persons was provided, their names delivered him up in writing as a true inventory of the Church’s goods, and some few words used to signify how proud the Church was of these treasures.

[15.]If men did not naturally abhor sacrilege, to resist or defeat so impious attempts would deserve small praise. But such is the general detestation of rapine in this kind, that whereas nothing doth either in peace or war more uphold men’s reputation than prosperous success, because in common construction unless notorious improbity be joined with prosperity it seemeth to argue favour with God, they which once have stained their hands with these odious spoils do thereby fasten unto all their actions an eternal prejudice, in respect whereof, for that it passeth through the world as an undoubted rule and principle that sacrilege is open defiance to God, whatsoever they afterwards undertake if they prosper in it men reckon it but Dionysius his navigation; and if any thing befall them otherwise it is not, as commonly, so in them ascribed to the great uncertainty of casual events, wherein the providence of God doth control the purposes of men oftentimes much more for their good than if all things did answer fully their heart’s desire, but the censure of the world is ever directly against them both bitter and peremptory.




[16.]To make such actions therefore less odious, and to mitigate the envy of them, many colourable shifts and inventions have been used, as if the world did hate only Wolves and think the Fox a goodly creature. The time it may be will come when they that either violently have spoiled or thus smoothly defrauded God shall find they did but deceive themselves. In the meanwhile there will be always some skilful persons which can teach a way how to grind treatably the Church with jaws that shall scarce move, and yet devour in the end more than they that come ravening with open mouth as if they would worry the whole in an instant; others also who having wastefully eaten out their own patrimony would be glad to repair if they might their decayed estates with the ruin they care not of what nor of whom so the spoil were theirs, whereof in some part if they happen to speed, yet commonly they are men born under that constellation which maketh them I know not how as unapt to enrich themselves as they are ready to impoverish others, it is their lot to sustain during life both the misery of beggars and the infamy of robbers.

But though no other plague and revenge should follow sacrilegious violations of holy things, the natural secret disgrace and ignominy, the very turpitude of such actions in the eye of a wise understanding heart is itself a heavy punishment. Men of virtuous quality are by this sufficiently  moved to beware how they answer and requite the mercies of God with injuries whether openly or indirectly offered.

I will not absolutely say concerning the goods of the Church that they may in no case be seized on by men, or that no obligation, commerce and bargain made between man and man can ever be of force to alienate the property which God hath in them. Certain cases I grant there are wherein it is not so dark what God himself doth warrant, but that we may safely presume him as willing to forego for our benefit, as always to use and convert to our benefit whatsoever our religion hath honoured him withal. But surely under the name of that which may be, many things that should not be are often done. By means whereof the Church most commonly for gold hath flannel, and whereas the usual saw of old was “Glaucus his change,” the proverb is now, “A church bargain.”

[17.]And for fear lest covetousness alone should linger out the time too much and not be able to make havock of the house of God with that expedition which the mortal enemy thereof did vehemently wish, he hath by certain strong enchantments so deeply bewitched religion itself as to make it in the end an earnest solicitor and an eloquent persuader of sacrilege, urging confidently, that the very best service which men of power can do to Christ is without any more ceremony to sweep all and to leave the Church as bare as in the day it was first born, that fulness of bread having made the children of the household wanton, it is without any scruple to be taken away from them and thrown to dogs; that they which laid the prices of their lands as offerings at the Apostles’ feet did but sow the seeds of superstition; that they which endowed churches with lands poisoned religion; that tithes and oblations are now in the sight of God as the sacrificed blood of goats; that if we give him our hearts and affections our goods are better bestowed otherwise; that Irenæus Polycarp’s disciple should not have said, “We offer unto God our goods as tokens of thankfulness for that we receive,” neither Origen, “He which worshippeth God must by gifts and oblations acknowledge him the Lord of  all;”
 in a word that to give unto God is error, reformation of error to take from the Church that which the blindness of former ages did unwisely give. By these or the like suggestions received with all joy and with like sedulity practised in certain parts of the Christian world they have brought to pass, that as David doth say of man so it is in hazard to be verified concerning the whole religion and service of God: “The time thereof may peradventure fall out to be threescore and ten years, or if strength do serve unto fourscore, what followeth is likely to be small joy for them whosoever they be that behold it.” Thus have the best things been overthrown not so much by puissance and might of adversaries as through defect of counsel in them that should have upheld and defended the same.


\section*{Of Ordination lawful without Title, and without any popular Election precedent, but in no case without regard of due information what their quality is that enter into holy orders.}
LXXX. There are in a minister of God these four things to be considered, his ordination which giveth him power to meddle with things sacred, the charge or portion of the Church allotted unto him for exercise of his office, the performance of his duty according to the exigence of his charge, and lastly the maintenance which in that respect he receiveth. All ecclesiastical laws and canons which either concern the bestowing or the using of the power of ministerial order have relation to these four. Of the first we have spoken before at large.

[2.]Concerning the next, for more convenient discharge of ecclesiastical duties, as the body of the people must needs be severed by divers precincts, so the clergy likewise accordingly distributed. Whereas therefore religion did first take place in cities, and in that respect was a cause why the name of Pagans which properly signifieth country people came to be used in common speech for the same that infidels and unbelievers were, it followed thereupon that all such cities had their ecclesiastical colleges consisting of Deacons and of Presbyters, whom first the Apostles or their delegates the Evangelists did both ordain and govern. Such were the colleges of Jerusalem, Antioch, Ephesus, Rome, Corinth, and the rest where the Apostles are known to have planted  our faith and religion.
 Now because religion and the cure of souls was their general charge in common over all that were near about them, neither had any one presbyter his several cure apart till Evaristus Bishop in the see of Rome about the year 112, began to assign precincts unto every church or title which the Christians held, and to appoint unto each presbyter a certain compass whereof himself should take charge alone, the commodiousness of this invention caused all parts of Christendom to follow it, and at the length among the rest our own churches about the year 636 became divided in like manner. But other distinction of Churches there doth not appear any in the Apostles’ writings save only according to those cities wherein they planted the Gospel of Christ and erected ecclesiastical colleges. Wherefore to ordain κατὰ πόλιν throughout every city, and κατ’ ἐκκλησίαν throughout every church do in them signify the same thing. Churches then neither were nor could be in so convenient sort limited as now they are; first by the bounds of each state, and then within each state by more particular precincts, till at the length we descend unto several congregations termed parishes with far narrower restraint than this name at the first was used.

[3.]And from hence hath grown their error, who as oft as they read of the duty which ecclesiastical persons are now  to perform towards the Church,
 their manner is always to understand by that church some particular congregation or parish church. They suppose that there should now be no man of ecclesiastical order which is not tied to some certain parish. Because the names of all church-officers are words of relation, because a shepherd must have his flock, a teacher his scholars, a minister his company which he ministereth unto, therefore it seemeth a thing in their eyes absurd and unreasonable that any man should be ordained a minister otherwise than only for some particular congregation.

Perceive they not how by this mean they make it unlawful for the Church to employ men at all in converting nations? For if so be the Church may not lawfully admit to an ecclesiastical function unless it tie the party admitted unto some particular parish, then surely a thankless labour it is whereby men seek the conversion of infidels which know not Christ and therefore cannot be as yet divided into their special congregations and flocks.

[4.]But, to the end it may appear how much this one thing amongst many more hath been mistaken, there is first no precept requiring that presbyters and deacons be made in such sort and not otherwise. Albeit therefore the Apostles did make them in that order, yet is not their example such a law as without all exception bindeth to make them in no other order but that.

Again if we will consider that which the Apostles themselves did, surely no man can justly say that herein we practise any thing repugnant to their example. For by them  there was ordained only in each Christian city a college of presbyters and deacons to administer holy things.
 Evaristus did a hundred years after the birth of our Saviour Christ begin the distinction of the church into parishes. Presbyters and deacons having been ordained before to exercise ecclesiastical functions in the church of Rome promiscuously, he was the first that tied them each one to his own station. So that of the two indefinite ordination of Presbyters and Deacons doth come more near the Apostles’ example, and the tying of them to be made only for particular congregations may justlier ground itself upon the example of Evaristus than of any Apostle of Christ.

[5.]It hath been the opinion of wise men and good men heretofore that nothing was ever devised more singularly beneficial unto God’s Church than this which our honourable predecessors have to their endless praise found out, by the erecting of such houses of study as those two most famous universities do contain, and by providing that choice wits after reasonable time spent in contemplation may at the length either enter into that holy vocation for which they have been so long nourished and brought up, or else give place and suffer others to succeed in their rooms, that so the Church may be always furnished with a number of men whose ability being first known by public trial in church labours there where men can best judge of them, their calling afterwards unto particular charge abroad may be according. All this is frustrate, those worthy foundations we must dissolve, their whole device and religious purpose which did erect them is made void, their orders and statutes are to be cancelled and disannulled, in case the Church be forbidden to grant any power of order unless it be with restraint to the party ordained unto some particular parish or congregation.

[6.]Nay might we not rather affirm of presbyters and of deacons that the very nature of their ordination is unto necessary local restraint a thing opposite and repugnant? The emperor Justinian doth say of tutors, “Certæ rei vel causæ tutor dari non potest, quia personæ non causæ vel rei tutor datur.” He that should grant a tutorship restraining his grant to some one certain thing or cause should do  but idly,
 because tutors are given for personal defence generally and not for managing of a few particular things or causes. So he that ordaining a presbyter or a deacon should in the form of ordination restrain the one or the other to a certain place might with much more reason be thought to use a vain and a frivolous addition, than they reasonably to require such local restraint as a thing which must of necessity concur evermore with all lawful ordinations. Presbyters and deacons are not by ordination consecrated unto places but unto functions. In which respect and in no other it is, that sith they are by virtue thereof bequeathed unto God, severed and sanctified to be employed in his service, which is the highest advancement that mortal creatures on earth can be raised unto, the Church of Christ hath not been acquainted in former ages with any such profane and unnatural custom as doth hallow men with ecclesiastical functions of order only for a time and then dismiss them again to the common affairs of the world: whereas contrariwise from the place or charge where that power hath been exercised we may be by sundry good and lawful occasions translated, retaining nevertheless the selfsame power which was first given.

[7.]It is some grief to spend thus much labour in refuting a thing that hath so little ground to uphold it, especially sith they themselves that teach it do not seem to give thereunto any great credit, if we may judge their minds by their actions. There are amongst them that have done the work of ecclesiastical persons sometime in the families of noblemen, sometime in much more public and frequent congregations, there are that have successively gone through perhaps seven or eight particular churches after this sort, yea some that at one and the same time have been, some which at this present hour are in real obligation of ecclesiastical duty and possession of commodity thereto belonging even in sundry particular churches within the land, some there are amongst them which will not so much abridge their liberty as to be fastened or tied unto any place, some which have bound themselves to one place only for a time and that time being once expired have afterwards voluntarily given unto other places the like experience and trial of them. All this  I presume they would not do if their persuasion were as strict as their words pretend.

[8.]But for the avoiding of these and such other the like confusions as are incident into the cause and question whereof we presently treat, there is not any thing more material than first to separate exactly the nature of the ministry from the use and exercise thereof; secondly to know that the only true and proper act of ordination is to invest men with that power which doth make them ministers by consecrating their persons to God and his service in holy things during term of life whether they exercise that power or no; thirdly that to give them a title or charge where to use their ministry concerneth not the making but the placing of God’s ministers, and therefore the laws which concern only their election or admission unto place of charge are not appliable to infringe any way their ordination; fourthly that as oft as any ancient constitution, law, or canon is alleged concerning either ordinations or elections, we forget not to examine whether the present case be the same which the ancient was, or else do contain some just reason for which it cannot admit altogether the same rules which former affairs of the Church now altered did then require.

[9.]In the question of making ministers without a title, which to do they say is a thing unlawful, they should at the very first have considered what the name of title doth imply, and what affinity or coherence ordinations have with titles, which thing observed would plainly have showed them their own error. They are not ignorant that when they speak of a title they handle that which belongeth to the placing of a minister in some charge, that the place of charge wherein a minister doth execute his office requireth some house of God for the people to resort unto, some definite numbers of souls unto whom he there administereth holy things, and some certain allowance whereby to sustain life; that the Fathers at the first named oratories and houses of prayer titles, thereby  signifying how God was interessed in them and held them as his own possessions.
 But because they know that the Church had ministers before Christian temples and oratories were, therefore some of them understand by a title a definite congregation of people only, and so deny that any ordination is lawful which maketh ministers that have no certain flock to attend, forgetting how the Seventy whom Christ himself did ordain ministers had their calling in that manner, whereas yet no certain charge could be given them. Others referring the name of a title especially to the maintenance of the minister infringe all ordinations made, except they which receive orders be first entitled to a competent ecclesiastical benefice, and (which is most ridiculously strange) except besides their present title to some such benefice they have likewise “some other title of annual rent or pension, whereby” they may be “relieved in case through infirmity, sickness, or other lawful impediment” they grow unable “to execute” their “ecclesiastical function.” So that every man lawfully ordained must bring a bow which hath two strings, a title of present right and another to provide for future possibility or chance.

[10.]Into these absurdities and follies they slide by misconceiving the true purpose of certain canons, which indeed  have forbidden to ordain a minister without a title,
 not that simply it is unlawful so to ordain, but because it might grow to an inconvenience if the Church did not somewhat restrain that liberty. For seeing they which have once received ordination cannot again return into the world, it behoveth them which ordain to foresee how such shall be afterwards able to live, lest their poverty and destitution should redound to the disgrace and discredit of their calling. Which evil prevented, those very laws which in that respect forbid, do exp ressly admit ordinations to be made at large and without title, namely if the party so ordained have of his own for the sustenance of this life, or if the bishop which giveth him orders will find him competent allowance till some place of ministration from whence his maintenance may arise be provided for him, or if any other fit and sufficient means be had against the danger before mentioned.

[11.]Absolutely therefore it is not true that any ancient canon of the Church which is or ought to be with us in force doth make ordinations at large unlawful, and as the state of the Church doth stand they are most necessary. If there be any conscience in men touching that which they write or speak, let them consider as well what the present condition of all things doth now suffer, as what the ordinances of former ages did appoint; as well the weight of those causes for which our affairs have altered, as the reasons in regard whereof our fathers and predecessors did sometimes strictly and severely keep that which for us to observe now is neither meet nor always possible. In this our present cause and controversy whether any not having title of right to a benefice may be lawfully ordained a minister, is it not manifest in the eyes of all men, that whereas the name of a benefice doth signify some standing ecclesiastical revenue taken out of the treasure of God and allotted to a spiritual person, to the end he may use the same and enjoy it as his own for term of life unless  his default cause deprivation, the clergy for many years after Christ had no other benefices but only their canonical portions, or monthly dividends allowed them according to their several degrees and qualities out of the common stock of such gifts, oblations, and tithes as the fervour of Christian piety did then yield? Yea that even when ministers had their churches and flocks assigned unto them in several, yet for maintenance of life their former kind of allowance continued, till such time as bishops and churches cathedral being sufficiently endowed with lands, other presbyters enjoyed instead of their first benefices the tithes and profits of their own congregations whole to themselves? Is it not manifest that in this realm, and so in other the like dominions, where the tenure of lands is altogether grounded on military laws, and held as in fee under princes which are not made heads of the people by force of voluntary election, but born the sovereign lords of those whole and entire territories, which territories their famous progenitors obtaining by way of conquest retained what they would in their own hands and divided the rest to others with reservation of sovereignty and capital interest, the building of churches and consequently the assigning of either parishes or benefices was a thing impossible without consent of such as were principal owners of land; in which consideration for their more encouragement hereunto they which did so far benefit the Church had by common consent granted (as great equity and reason was) a right for them and their heirs till the world’s end to nominate in those benefices men whose  quality the bishop allowing might admit them thereunto?
 Is it not manifest that from hence inevitably such inequality of parishes hath grown, as causeth some through the multitude of people which have resort unto one church to be more than any one man can wield, and some to be of that nature by reason of chapels annexed, that they which are incumbents should wrong the church if so be they had not certain stipendaries under them, because where the corps of the profit or benefice is but one, the title can be but one man’s, and yet the charge may require more?

[12.]Not to mention therefore any other reason whereby it may clearly appear how expedient it is and profitable for this Church to admit ordinations without title, this little may suffice to declare how impertinent their allegations against it are out of ancient canons, how untrue their confident asseverations that only through negligence of popish prelates the custom of making such kind of ministers hath prevailed in the church of Rome against their canons, and that with us it is expressly against the laws of our own government when a minister doth serve as a stipendary curate, which kind of service nevertheless the greatest Rabbins of that part do altogether follow. For howsoever they are loth peradventure to be named curates, stipendaries they are and the labour they bestow is in other men’s cures, a thing not unlawful for them to do, yet unseemly for them to condemn which practise it.

[13.]I might here discover the like oversight throughout all their discourses made in behalf of the people’s pretended right to elect their ministers before the bishop may lawfully ordain. But because we have otherwhere at large disputed of popular elections, and of the right of patronage wherein is drowned whatsoever the people under any pretence or colour  may seem to challenge about admission and choice of the pastors that shall feed their souls,
 I cannot see what one duty there is which always ought to go before ordination, but only care of the party’s worthiness as well for integrity and virtue as knowledge, yea for virtue more, inasmuch as defect of knowledge may sundry ways be supplied, but the scandal of vicious and wicked life is a deadly evil.


\section*{Of the Learning that should be in ministers, their Residence, and the number of their Livings.}
LXXXI. The truth is that of all things hitherto mentioned the greatest is that threefold blot or blemish of notable ignorance, unconscionable absence from the cures whereof men have taken charge, and unsatiable hunting after spiritual preferments without either care or conscience of the public good. Whereof to the end that we may consider as in God’s own sight and presence with all uprightness, sincerity and truth, let us particularly weigh and examine in every of them first how far forth they are reprovable by reasons and maxims of common right; secondly whether that which our laws do permit be repugnant to those maxims, and with what equity we ought to judge of things practised in this case, neither on the one hand defending that which must be acknowledged out of square, nor on the other side condemning rashly whom we list for whatsoever we disallow.

[2.]Touching arguments therefore taken from the principles of common right to prove that ministers should be learned, that they ought to be resident upon their livings, and that more than one only benefice or spiritual living may not be granted unto one man; the first because St. Paul requireth in a minister ability to teach, to convince, to distribute the word rightly, because also the Lord himself hath protested they shall be no priests to him which have rejected knowledge, and because if the blind lead the blind they must both needs fall into the pit: the second because teachers are shepherds whose flocks can be at no time secure from danger, they are watchmen whom the enemy doth always besiege, their labours in the Word and Sacraments admit no intermission,  their duty requireth instruction and conference with men in private,
 they are the living oracles of God to whom the people must resort for counsel, they are commanded to be patterns of holiness, leaders, feeders, supervisors amongst their own, it should be their grief as it was the Apostle’s to be absent though necessarily from them over whom they have taken charge: finally the last because plurality and residence are opposite, because the placing of one clerk in two churches is a point of merchandise and filthy gain, because no man can serve two masters, because every one should remain in that vocation whereto he is called; what conclude they of all this? Against ignorance, against nonresidence, and against plurality of livings is there any man so raw and dull but that the volumes which have been written both of old and of late may make him in so plentiful a cause eloquent?

For if by that which is generally just and requisite we measure what knowledge there should be in a minister of the Gospel of Christ; the arguments which light of nature offereth, the laws and statutes which scripture hath, the canons that are taken out of ancient synods, the decrees and constitutions of sincerest times, the sentences of all antiquity, and in a word even every man’s full consent and conscience is against ignorance in them that have charge and cure of souls.

Again what availeth it if we be learned and not faithful? or what benefit hath the Church of Christ if there be in us sufficiency without endeavour or care to do that good which our place exacteth? Touching the pains and industry therefore wherewith men are in conscience bound to attend the work of their heavenly calling even as much as in them lieth bending thereunto their whole endeavour, without either fraud, sophistication, or guile; I see not what more effectual obligation or bond of duty there should be urged than their own only vow and promise made unto God himself at the time of their  ordination. The work which they have undertaken requireth both care and fear. Their sloth that negligently perform it maketh them subject to malediction. Besides we also know that the fruit of our pains in this function is life both to ourselves and others.

And do we yet need incitements to labour? Shall we stop our ears both against those conjuring exhortations which Apostles, and against the fearful comminations which Prophets have uttered out of the mouth of God, the one for prevention, the other for reformation, of our sluggishness in this behalf? St. Paul, “Attend to yourselves and to all the flock whereof the Holy Ghost hath made you overseers, to feed the Church of God which he hath purchased with his own blood.” Again, “I charge thee before God and the Lord Jesus Christ which shall judge the quick and the dead at his coming, preach the word; be instant.” Jeremy, “Wo unto the pastors that destroy and scatter the sheep of my pasture, I will visit you for the wickedness of your works, saith the Lord, the remnant of my sheep I will gather together out of all countries and will bring them again to their folds, they shall grow and increase, and I will set up shepherds over them which shall feed them.” Ezekiel, “Should not the shepherds, should they not feed the flocks? Ye eat the fat, and ye clothe yourselves with the wool, and the weak ye have not strengthened, the sick ye have not cured, neither have ye bound up the broken nor brought home again that which was driven away, ye have not inquired after that which was lost, but with cruelty and rigour ye have ruled. Wherefore, as I live, saith the Lord God, I will require my sheep at their hands, nor shall the shepherds feed themselves any more, for I will deliver my sheep from their mouths, they shall no more devour them.”

Nor let us think to excuse ourselves if haply we labour though it be at random, and sit not altogether idle abroad. For we are bound to attend that part of the flock of Christ whereof the Holy Ghost hath made us overseers. The residence of ministers upon their own peculiar charge is by so much the rather necessary, for that absenting themselves  from the place where they ought to labour they neither can do the good which is looked for at their hands,
 nor reap that comfort which sweeteneth life to them that spend it in these travails upon their own. For it is in this as in all things else, which are through private interest dearer than what concerneth either others wholly or us but in part and according to the rate of a general regard.

As for plurality it hath not only the same inconveniences which are observed to grow by absence, but over and besides, at the least in common construction, a show of that worldly humour which men do think should not reign so high.

[3.]Now from hence their collections are as followeth, first a repugnancy or contradiction between the principles of common right and that which our laws in special considerations have allowed; secondly a nullity or frustration of all such acts as are by them supposed opposite to those principles, an invalidity in all ordinations of men unable to preach, and in all dispensations which mitigate the law of common right for the other two. And why so? Forsooth because whatsoever we do in these three cases and not by virtue of common right, we must yield it of necessity done by warrant of peculiar right or privilege. Now “a privilege is said to be that, that for favour of certain persons cometh forth against common right; things prohibited are dispensed with because things permitted are despatched by common right, but things forbidden require dispensations. By which descriptions of a privilege and dispensation it is,” they say, “apparent,”  that a privilege must license and authorize the same which the law against ignorance,
 nonresidence and plurality doth infringe, and so be a law contrariant or repugnant to the law of nature and the law of God, because “all the reasons whereupon the positive law of man against these three was first established are taken and drawn from the law of nature, and the law of God.” For answer whereunto we will but lead them to answer themselves.

[4.]First therefore if they will grant (as they must) that all direct oppositions of speech require one and the selfsame subject to be meant on both parts where opposition is pretended, it will follow that either the maxims of common right do enforce the very same things not to be good which we say are good, grounding ourselves on the reasons by virtue whereof our privileges are established; or if the one do not reach unto that particular subject for which the other have provided, then is there no contradiction between them. In all contradictions if the one part be true the other eternally must be false. And therefore if the principles of common right do at any time truly enforce that particular not to be good which privileges make good, it argueth invincibly that such privileges have been grounded upon some error. But to say that every privilege is opposite unto the principles of common right, because it dispenseth with that which common right doth prohibit, hath gross absurdity. For the voice of equity and justice is that a general law doth never derogate from a special privilege, whereas if the one were contrariant to the other, a general law being in force should always dissolve a privilege.

The reason why many are deceived by imagining that so it should do, and why men of better insight conclude directly it should not, doth rest in the subject or matter itself, which matter indefinitely considered in laws of common right is in privileges considered as beset and limited with special circumstances, by means whereof to them which respect it but by way of generality it seemeth one and the same in both, although it be not the same if once we descend to particular consideration thereof. Precepts do always propose perfection, not such as none can attain unto, for then in vain should we ask or require it at the hands of men, but such perfection as all men must aim at to the end that as largely as human providence and  care can extend it, it may take place. Moral laws are the rules of politic, those politic, which are made to order the whole Church of God, rules unto all particular churches, and the laws of every particular church rules unto every particular man within the body of the same church. Now because the higher we ascend in these rules the further still we remove from those specialties; which being proper to the subject whereupon our actions must work are therefore chiefly considered by us, by them least thought upon that wade altogether in the two first kinds of general directions; their judgment cannot be exact and sound concerning either laws of churches or actions of men in particular, because they determine of effects by a part of the causes only out of which they grow, they judge conclusions by demipremises and half-principles, they lay them in the balance stripped from those necessary material circumstances, which should give them weight, and by show of falling uneven with the scale of most universal and abstracted rules, they pronounce that too light which is not, if they had the skill to weigh it. This is the reason why men altogether conversant in study do know how to teach but not how to govern; men experienced contrariwise govern well, yet know not which way to set down orderly the precepts and reasons of that they do.

He that will therefore judge rightly of things done must join with his forms and conceits of general speculation the matter wherein our actions are conversant. For by this shall appear what equity there is in those privileges and peculiar grants or favours which otherwise will seem repugnant to justice, and because in themselves considered they have a show of repugnancy, this deceiveth those great clerks which hearing a privilege defined to be “an especial right brought in by their power and authority that make it for some public benefit against the general course of reason,” are not able to comprehend how the word against doth import exception without any opposition at all. For inasmuch as the hand of justice must distribute to every particular what is due, and judge what is due with respect had no less of  particular circumstances than of general rules and axioms, it cannot fit all sorts with one measure, the wills, counsels, qualities and states of men being divers.

For example, the law of common right bindeth all men to keep their promises, perform their compacts, and answer the faith they have given either for themselves or others. Notwithstanding he which bargaineth with one under years can have no benefit by this allegation, because he bringeth it against a person which is exempt from the common rule. Shall we then conclude that thus to exempt certain men from the law of common right is against God, against nature, against whatsoever may avail to strengthen and justify that law before alleged; or else acknowledge (as the truth is) that special causes are to be ordered by special rules; that if men grown unto ripe age disadvantage themselves by bargaining, yet what they have wittingly done is strong and in force against them, because they are able to dispose and manage their own affairs, whereas youth for lack of experience and judgment being easily subject to circumvention is therefore justly exempt from the law of common right whereunto the rest are justly subject? This plain inequality between men of years and under years is a cause why equity and justice cannot apply equally the same general rule to both, but ordereth the one by common right and granteth to the other a special privilege.

Privileges are either transitory or permanent. Transitory such as serve only some one turn, or at the most extend no further than to this or that man with the end of whose natural life they expire; permanent such as the use whereof doth continue still, for that they belong unto certain kinds of men and causes which never die. Of this nature are all immunities and preeminences which for just considerations one sort of men enjoyeth above another both in the Church and commonwealth, no man suspecting them of contrariety to any  branch of those laws or reasons whereupon the general right is grounded.

[5.]Now there being general laws and rules whereby it cannot be denied but the Church of God standeth bound to provide that the ministry may be learned, that they which have charge may reside upon it, and that it may not be free for them in scandalous manner to multiply ecclesiastical livings; it remaineth in the next place to be examined, what the laws of the Church of England do admit which may be thought repugnant to any thing hitherto alleged, and in what special consideration they seem to admit the same.

Considering therefore that to furnish all places of cure in this realm it is not an army of twelve thousand learned men that would suffice, nor two universities that can always furnish as many as decay in so great a number, nor a fourth part of the livings with cure that when they fall are able to yield sufficient maintenance for learned men, is it not plain that unless the greatest part of the people should be left utterly without the public use and exercise of religion there is no remedy but to take into the ecclesiastical order a number of men meanly qualified in respect of learning? For whatsoever we may imagine in our private closets or talk for communication’s sake at our boards, yea or write in our books through a notional conceit of things needful for performance of each man’s duty, if once we come from the theory of learning to take out so many learned men, let them be diligently viewed out of whom the choice shall be made, and thereby an estimate made what degree of skill we must either admit or else leave numbers utterly destitute of guides, and I doubt not but that men endued with sense of common equity will soon discern that besides eminent and competent knowledge we are to descend to a lower step, receiving knowledge in that degree which is but tolerable.

When we commend any man for learning our speech importeth him to be more than meanly qualified that way; but when laws do require learning as a quality which maketh capable of any function, our measure to judge a learned man  by must be some certain degree of learning beneath which we can hold no man so qualified. And if every man that listeth may set that degree himself, how shall we ever know when laws are broken, when kept, seeing one man may think a lower degree sufficient, another may judge them unsufficient that are not qualified in some higher degree. Wherefore of necessity either we must have some judge in whose conscience they that are thought and pronounced sufficient are to be so accepted and taken, or else the law itself is to set down the very lowest degree of fitness that shall be allowable in this kind.

So that the question doth grow to this issue. St. Paul requireth learning in presbyters, yea such learning as doth enable them to exhort in doctrine which is sound, and to disprove them that gainsay it. What measure of ability in such things shall serve to make men capable of that kind of office he doth not himself precisely determine, but referreth it to the conscience of Titus and others which had to deal in ordaining presbyters. We must therefore of necessity make this demand, whether the Church lacking such as the Apostle would have chosen may with good conscience take out of such as it hath in a meaner degree of fitness them that may serve to perform the service of public prayer, to minister the sacraments unto the people, to solemnize marriage, to visit the sick and bury the dead, to instruct by reading although by preaching they be not as yet so able to benefit and feed Christ’s flock. We constantly hold that in this case the Apostle’s law is not broken. He requireth more in presbyters than there is found in many whom the Church of England alloweth. But no man being tied unto impossibilities, to do that we cannot we are not bound.

It is but a stratagem of theirs therefore and a very indirect practice, when they publish large declamations to prove that learning is required in the ministry, and to make the silly people believe that the contrary is maintained by the Bishops and upheld by the laws of the land; whereas the question in truth is not whether learning be required, but whether a church wherein there is not sufficient store of learned men to furnish all congregations should do better to let thousands  of souls grow savage,
 to let them live without any public service of God, to let their children die unbaptized, to withhold the benefit of the other sacrament from them, to let them depart this world like Pagans without any thing as much as read unto them concerning the way of life, than as it doth in this necessity, to make such presbyters as are so far forth sufficient although they want that ability of preaching which some others have.

[6.]In this point therefore we obey necessity, and of two evils we take the less; in the rest a public utility is sought and in regard thereof some certain inconveniences tolerated, because they are recompensed with greater good. The law giveth liberty of non-residence for a time to such as will live in universities, if they faithfully there labour to grow in knowledge that so they may afterwards the more edify and the better instruct their congregations. The Church in their absence is not destitute, the people’s salvation not neglected for the present time, the time of their absence is in the intendment of law bestowed to the Church’s great advantage and benefit, those necessary helps are procured by it which turn by many degrees more to the people’s comfort in time to come than if their pastors had continually abidden with them. So that the law doth hereby provide in some part to remedy and help that evil which the former necessity hath imposed upon the Church. For compare two men of equal meanness, the one perpetually resident, the other absent for a space in such sort as the law permitteth. Allot unto both some nine years’ continuance with cure of souls. And must not three years’ absence in all probability and likelihood make the one more profitable than the other unto God’s Church, by so much as the increase of his knowledge gotten in those three years may add unto six years’ travail following? For the greater ability there is added to the instrument wherewith it pleaseth God to save souls, the more facility and expedition it hath to work that which is otherwise hardlier effected.

As much may be said touching absence granted to them that attend in the families of bishops, which schools of gravity, discretion and wisdom, preparing men against the time that they come to reside abroad, are in my poor opinion even the fittest places that any ingenuous mind can wish to enter into between departure from private study and access to a more  public charge of souls, yea no less expedient for men of the best sufficiency and most maturity in knowledge, than the very universities themselves are for the ripening of such as be raw.

Employment in the families of noblemen or in princes’ courts hath another end for which the selfsame leave is given not without great respect to the good of the whole Church. For assuredly whosoever doth well observe how much all inferior things depend upon the orderly courses and motions of those greater orbs, will hardly judge it either meet or good that the Angels assisting them should be driven to betake themselves unto other stations, although by nature they were not tied where now they are, but had charge also elsewhere, as long as their absence from beneath might but tolerably be supplied, and by descending their rooms above should become vacant. For we are not to dream in this case of any platform which bringeth equally high and low unto parish churches, nor of any constraint to maintain at their own charge men sufficient for that purpose; the one so repugnant to the majesty and greatness of English nobility, the other so improbable and unlikely to take effect that they which mention either of both seem not indeed to have conceived what either is. But the eye of law is the eye of God; it looketh into the hearts and secret dispositions of men, it beholdeth how far one star differeth from another in glory, and as men’s several degrees require, accordingly it guideth them, granting unto principal personages privileges correspondent to their high estates, and that not only in civil but even in spiritual affairs, to the end they may love that religion the more which no way seeketh to make them vulgar, no way diminisheth their dignity and greatness, but to do them good doth them honour also, and by such extraordinary favours teacheth them to be in the Church of God the same which the Church of God esteemeth them, more worth than thousands.

It appeareth therefore in what respect the laws of this realm have given liberty of non-residence; to some that their knowledge may be increased and their labours by that mean be made afterwards the more profitable, to others lest the houses of great men should want that daily exercise of religion wherein their example availeth as much yea many times peradventure more than the laws themselves with the common sort.




[7.]A third thing respected both in permitting absence and also in granting to some that liberty of addition or plurality which necessarily enforceth their absence is a mere both just and conscionable regard, that as men are in quality and as their services are in weight for the public good, so likewise their rewards and encouragements by special privilege of law might somewhat declare how the state itself doth accept their pains, much abhorring from their bestial and savage rudeness which think that oxen should only labour and asses feed. Thus to readers in universities, whose very paper and book expenses their ancient allowances and stipends at this day do either not or hardly sustain; to governors of colleges, lest the great overplus of charges necessarily enforced upon them by reason of their place, and very slenderly supplied by means of that change in the present condition of things which their founders could not foresee; to men called away from their cures and employed in weightier business either of the church or commonwealth, because to impose upon them a burden which requireth their absence and not to release them from the duty of residence were a kind of cruel and barbarous injustice; to residents in cathedral churches or upon dignities ecclesiastical, forasmuch as these being rooms of greater hospitality, places of more respect and consequence than the rest, they are the rather to be furnished with men of best quality, and the men for their quality’s sake to be favoured above others; I say unto all these in regard of their worth and merit the law hath therefore given leave while themselves bear weightier burdens to supply inferior by deputation, and in like consideration partly, partly also by way of honour to learning, nobility, and authority, permitteth that men which have taken theological degrees in schools, the suffragans of bishops, the household chaplains of men of honour or in great office, the brethren and sons of lords temporal or of knights, if God shall move the hearts of such to enter at any time into holy orders, may obtain to themselves a faculty or license to hold two ecclesiastical livings though having cure, any spiritual person of the Queen’s council three such livings, her chaplains what number of promotions herself in her own princely wisdom thinketh good to bestow upon them.

[8.]But, as it fareth in such cases, the gap which for just  considerations we open unto some letteth in others through corrupt practices to whom such favours were neither meant nor should be communicated.
 The greatness of the harvest and the scarcity of able workmen hath made it necessary that law should yield to admit numbers of men but slenderly and meanly qualified. Hereupon because whom all other worldly hopes have forsaken they commonly reserve ministerial vocation as their last and surest refuge ever open to forlorn men, the Church that should nourish them whose service she needeth hath obtruded upon her their service that know not otherwise how to live and sustain themselves. These finding nothing more easy than means to procure the writing of a few lines to some one or other which hath authority, and nothing more usual than too much facility in condescending unto such requests, are often received into that vocation whereunto their unworthiness is no small disgrace.

Did any thing more aggravate the crime of Jeroboam’s profane apostasy than that he chose to have his clergy the scum and refuse of his whole land? Let no man spare to tell it them, they are not faithful towards God that burden wilfully his Church with such swarms of unworthy creatures. I will not say of all degrees in the ministry that which St. Chrysostom doth of the highest, “He that will undertake so weighty a charge had need to be a man of great understanding, rarely assisted with divine grace, for integrity of manners, purity of life, and for all other virtues, to have in him more than a man:” but surely this I will say with Chrysostom, “We need not doubt whether God be highly displeased with us, or what the cause of his anger is, if things of so great fear and holiness as are the least and lowest duties of his service be thrown wilfully on them whose not only mean but bad and scandalous quality doth defile whatsoever they  handle.”
 These eyesores and blemishes in continual attendants about the service of God’s sanctuary do make them every day fewer that willingly resort unto it, till at length all affection and zeal towards God be extinct in them, through a wearisome contempt of their persons which for a time only live by religion and are for recompense in fine the death of the nurse that feedeth them. It is not obscure how incommodious the Church hath found both this abuse of the liberty which law is enforced to grant, and not only this but the like abuse of that favour also which law in other considerations already mentioned affordeth touching residence and plurality of spiritual livings.

Now that which is practised corruptly to the detriment and hurt of the Church against the purpose of those very laws which notwithstanding are pretended in defence and justification thereof, we must needs acknowledge no less repugnant to the grounds and principles of common right than the fraudulent proceedings of tyrants to the principles of just sovereignty. Howbeit not so those special privileges which are but instruments wrested and forced to serve malice.

There is in the patriarch of heathen philosophers this precept, “Let no husbandman nor no handicraftsman be a priest.” The reason whereupon he groundeth is a maxim in the law of nature, “it importeth greatly the good of all men that God be reverenced,” with whose honour it standeth not that they which are publicly employed in his service should live of base and manuary trades. Now compare herewith the Apostle’s words. “Ye know that these hands have ministered to my necessities and to them that are with me.” What think we? Did the Apostle any thing opposite herein or repugnant to the rules and maxims of the law of nature? The selfsame reasons that accord his actions with the law of nature shall declare our privileges and his laws no less consonant.

[9.]Thus therefore we see that although they urge very colourably the Apostle’s own sentences, requiring that a minister should be able to divide rightly the word of God, that they who are placed in charge should attend unto it themselves  which in absence they cannot do,
 and that they which have divers cures must of necessity be absent from some, whereby the law apostolic seemeth apparently broken, which law requiring attendance cannot otherwise be understood than so as to charge them with perpetual residence; again though in every of these causes they infinitely heap up the sentences of Fathers, the decrees of popes, the ancient edicts of imperial authority, our own national laws and ordinances prohibiting the same and grounding evermore their prohibitions partly on the laws of God and partly on reasons drawn from the light of nature, yet hereby to gather and infer contradiction between those laws which forbid indefinitely and ours which in certain cases have allowed the ordaining of sundry ministers whose sufficiency for learning is but mean, again the licensing of some to be absent from their flocks, and of others to hold more than one only living which hath cure of souls, I say to conclude repugnancy between these especial permissions and the former general prohibitions which set not down their own limits is erroneous, and the manifest cause thereof ignorance in differences of matter which both sorts of law concern.

[10.]If then the considerations be reasonable, just and good, whereupon we ground whatsoever our laws have by special right permitted; if only the effects of abused privileges be repugnant to the maxims of common right, this main foundation of repugnancy being broken whatsoever they have built thereupon falleth necessarily to ground. Whereas therefore upon surmise or vain supposal of opposition between our special and the principles of common right they gather that such as are with us ordained ministers before they can preach be neither lawful, because the laws already mentioned forbid generally to create such, neither are they indeed ministers although we commonly so name them, but whatsoever they execute by virtue of such their pretended vocation is void; that all our grants and tolerations as well of this as the rest are frustrate and of no effect, the persons that enjoy them possess them wrongfully and are deprivable at all hours; finally that other just and sufficient remedy of evils there can be none besides the utter abrogation of these our mitigations and the strict establishment of former ordinances to be absolutely executed whatsoever follow; albeit the answer already made in  discovery of the weak and unsound foundation whereupon they have built these erroneous collections may be thought sufficient,
 yet because our desire is rather to satisfy if it be possible than to shake them off, we are with very good will contented to declare the causes of all particulars more formally and largely than the equity of our own defence doth require.

There is crept into the minds of men at this day a secret pernicious and pestilent conceit that the greatest perfection of a Christian man doth consist in discovery of other men’s faults, and in wit to discourse of our own profession. When the world most abounded with just, righteous, and perfect men, their chiefest study was the exercise of piety, wherein for their safest direction they reverently hearkened to the readings of the law of God, they kept in mind the oracles and aphorisms of wisdom which tended unto virtuous life, if any scruple of conscience did trouble them for matter of actions which they took in hand, nothing was attempted before counsel and advice were had, for fear lest rashly they might offend. We are now more confident, not that our knowledge and judgment is riper, but because our desires are another way. Their scope was obedience, ours is skill; their endeavour was reformation of life, our virtue nothing but to hear gladly the reproof of vice; they in the practice of their religion wearied chiefly their knees and hands, we especially our ears and tongues. We are grown as in many things else so in this to a kind of intemperancy which (only sermons excepted) hath almost brought all other duties of religion out of taste. At the least they are not in that account and reputation which they should be.

[11.]Now because men bring all religion in a manner to the only office of hearing sermons, if it chance that they who are thus conceited do embrace any special opinion different from other men, the sermons that relish not that opinion can in no wise please their appetite. Such therefore as preach unto them but hit not the string they look for are respected  as unprofitable, the rest as unlawful and indeed no ministers if the faculty of sermons want. For why? A minister of the word should they say be able “rightly to divide the word.” Which apostolic canon many think they do well observe, when in opening the sentences of holy Scripture they draw all things favourably spoken unto one side; but whatsoever is reprehensive, severe, and sharp, they have others on the contrary part whom that must always concern; by which their over partial and unindifferent proceeding while they thus labour amongst the people to divide the word, they make the word a mean to divide and distract the people.

Ὀρθοτομει̑ν “to divide aright” doth note in the Apostles’ writings soundness of doctrine only; and in meaning standeth opposite to καινοτομει̑ν “the broaching of new opinions against that which is received.” For questionless the first things delivered to the Church of Christ were pure and sincere truth. Which whosoever did afterwards oppugn could not choose but divide the Church into two moieties, in which division such as taught what was first believed held the truer part, the contrary side in that they were teachers of novelty erred.

For prevention of which evil there are in this church many singular and well-devised remedies, as namely the use of subscribing to the articles of religion before admission to degrees of learning or to any ecclesiastical living, the custom of reading the same articles and of approving them in public assemblies wheresoever men have benefices with cure of souls, the order of testifying under their hands allowance of the book of common prayer and the book of ordaining ministers, finally the discipline and moderate severity which is used either in otherwise correcting or silencing them that trouble and disturb the Church with doctrines which tend unto innovation, it being better that the Church should want altogether the benefit of such men’s labours than endure the mischief of their inconformity to good laws; in which case if any repine at the course and proceedings of justice, they must learn to content themselves with the answer of M. Curius,  which had sometime occasion to cut off one from the body of the commonwealth,
 in whose behalf because it might have been pleaded that the party was a man serviceable, he therefore began his judicial sentence with this preamble, “Non esse opus reip. eo cive qui parere nesciret: The commonwealth needeth men of quality, yet never those men which have not learned how to obey.”

[12.]But the ways which the church of England taketh to provide that they who are teachers of others may do it soundly, that the purity and unity as well of ancient discipline as doctrine may be upheld, that avoiding singularities we may all glorify God with one heart and one tongue, they of all men do least approve, that most urge the Apostle’s rule and canon. For which cause they allege it not so much to that purpose, as to prove that unpreaching ministers (for so they term them) can have no true nor lawful calling in the Church of God. St. Augustine hath said of the will of man that “simply to will proceedeth from nature, but our well-willing is from grace.” We say as much of the minister of God, “publicly to teach and instruct the Church is necessary in every ecclesiastical minister, but ability to teach by sermons is a grace which God doth bestow on them whom he maketh sufficient for the commendable discharge of their duty.” That therefore wherein a minister differeth from other Christian men is not as some have childishly imagined the “sound preaching of the word of God,” but as they are lawfully and truly governors to whom authority of regiment  is given in the commonwealth according to the order which polity hath set,
 so canonical ordination in the Church of Christ is that which maketh a lawful minister as touching the validity of any act which appertaineth to that vocation. The cause why St. Paul willed Timothy not to be over hasty in ordaining ministers was (as we very well may conjecture) because imposition of hands doth consecrate and make them ministers whether they have gifts and qualities fit for the laudable discharge of their duties or no. If want of learning and skill to preach did frustrate their vocation, ministers ordained before they be grown unto that maturity should receive new ordination whensoever it chanceth that study and industry doth make them afterwards more able to perform the office, than which what conceit can be more absurd? Was not St. Augustine himself contented to admit an assistant in his own church, a man of small erudition; considering that what he wanted in knowledge was supplied by those virtues which made his life a better orator than more learning could make others whose conversation was less holy? Were the priests sithence Moyses all able and sufficient men learnedly to interpret the law of God? or was it ever imagined that this defect should frustrate what they executed, and deprive them of right unto any thing they claimed by virtue of their priesthood? Surely as in magistrates the want of those gifts which their office needeth is cause of just imputation of blame in them that wittingly choose unsufficient and unfit men when they might do otherwise, and yet therefore is not their choice void, nor every action of magistracy frustrate in that respect: so whether it were of necessity or even of very carelessness that men unable to preach should be taken in pastors’ rooms, nevertheless it seemeth to be an error in them which think that the lack of any such perfection defeateth utterly their calling.

[13.]To wish that all men were so qualified as their places and dignities require, to hate all sinister and corrupt dealings which hereunto are any let; to covet speedy redress of those things whatsoever whereby the Church sustaineth detriment, these good and virtuous desires cannot offend any but ungodly  minds.
 Notwithstanding some in the true vehemency, and others under the fair pretence of these desires, have adventured that which is strange, that which is violent and unjust. There are, which in confidence of their general allegations concerning the knowledge, the residence, and the single livings of ministers, presume not only to annihilate the solemn ordinations of such as the Church must of force admit, but also to urge a kind of universal proscription against them, to set down articles, to draw commissions, and almost to name themselves of the Quorum for inquiry into men’s estates and dealings, whom at their pleasure they would deprive and make obnoxious to what punishment themselves list; and that not for any violation of laws either spiritual or civil, but because men have trusted the laws too far, because they have held and enjoyed the liberty which law granteth, because they had not the wit to conceive as these men do that laws were made to entrap the simple by permitting those things in show and appearance which indeed should never take effect, forasmuch as they were but granted with a secret condition to be put in practice “if they should be profitable and agreeable with the word of God;” which condition failing in all ministers that cannot preach, in all that are absent from their livings, and in all that have divers livings, (for so it must be presumed though never as yet proved,) therefore as men which have broken the law of God and nature they are deprivable at all hours. Is this the justice of that discipline whereunto all Christian churches must stoop and submit themselves? Is this the equity wherewith they labour to reform the world?

[14.]I will no way diminish the force of those arguments whereupon they ground. But if it please them to behold the visage of these collections in another glass, there are civil as well as ecclesiastical unsufficiencies, non-residences, and pluralities; yea the reasons which light of nature hath ministered against both are of such affinity that much less they cannot enforce in the one than in the other.

When they that bear great offices be persons of mean worth, the contempt whereinto their authority groweth  weakeneth the sinews of the whole state.
 Notwithstanding where many governors are needful and they not many whom their quality can commend, the penury of worthier must needs make the meaner sort of men capable.

Cities in the absence of their governors are as ships wanting pilots at sea. But were it therefore justice to punish whom superior authority pleaseth to call from home, or alloweth to be employed elsewhere?

In committing many offices to one man there are apparently  these inconveniences: the commonwealth doth lose the benefit of serviceable men which might be trained up in those rooms; it is not easy for one man to discharge many men’s duties well; in service of warfare and navigation were it not the overthrow of whatsoever is undertaken, if one or two should engross such offices as being now divided into many hands are discharged with admirable both perfection and expedition?

Nevertheless be it far from the mind of any reasonable man to imagine, that in these considerations princes either ought of duty to revoke all such kind of grants though made with very special respect to the extraordinary merit of certain men, or might in honour demand of them the resignation of their offices with speech to this or the like effect: “Forasmuch as you A.B. by the space of many years have done us that faithful service in most important affairs, for which we always judging you worthy of much honour have therefore committed unto you from time to time very great and weighty offices, which offices hitherto you quietly enjoy; we are now given to understand that certain grave and learned men have found in the books of ancient philosophers divers arguments drawn from the common light of nature, and declaring the wonderful discommodities which use to  grow by dignities thus heaped together in one:
 for which cause at this present moved in conscience and tender care for the public good we have summoned you hither, to dispossess you of those places and to depose you from those rooms, whereof indeed by virtue of our own grant, yet against reason, you are possessed. Neither ought you, or any other, to think us rash, light, or unconstant, in so doing. For we tell you plain that herein we will both say and do that thing which the noble and wise emperor sometimes both said and did in a matter of far less weight than this, ‘Quod inconsulto fecimus consulto revocamus,’ ‘That which we unadvisedly have done we advisedly will revoke and undo.’ ”

Now for mine own part the greatest harm I would wish them who think that this were consonant with equity and right, is that they might but live where all things are with such kind of justice ordered, till experience have taught them to see their error.

[15.]As for the last thing which is incident into the cause whereof we speak, namely what course were the best and safest whereby to remedy such evils as the Church of God may sustain where the present liberty of the law is turned to great abuse, some light we may receive from abroad not unprofitable for direction of God’s own sacred house and family. The Romans being a people full of generosity and by nature courteous did no way more show their gentle disposition than by easy condescending to set their bondmen at liberty. Which benefit in the happier and better times of the commonwealth was bestowed for the most part as an ordinary reward of virtue, some few now and then also purchasing freedom with that which their just labours could gain and their honest frugality save. But as the empire daily grew up so the manners and conditions of men decayed, wealth was honoured and virtue not cared for, neither did any thing seem opprobrious out of which there might rise commodity and profit, so that it could be no marvel in a state thus far degenerated, if when the more ingenuous sort were become base, the baser laying aside all shame and face of honesty did some by robberies, burglaries, and prostitutions of their bodies gather wherewith to redeem liberty; others obtain the same at the  hands of their lords by serving them as vile instruments in those attempts which had been worthy to be revenged with ten thousand deaths. A learned, judicious, and polite historian having mentioned so foul disorders giveth his judgment and censure of them in this sort: “Such eye-sores in the commonwealth have occasioned many virtuous minds to condemn altogether the custom of granting liberty to any bondslave, forasmuch as it seemed a thing absurd that a people which commanded all the world should consist of so vile refuse. But neither is this the only custom wherein the profitable inventions of former are depraved by later ages, and for myself I am not of their opinion that wish the abrogation of so grossly used customs, which abrogation might peradventure be cause of greater inconveniences ensuing, but as much as may be I would rather advise that redress were sought through the careful providence of chief rulers and overseers of the commonwealth, by whom a yearly survey being made of all that are manumised, they which seem worthy might be taken and divided into tribes with other citizens, the rest dispersed into colonies abroad or otherwise disposed of that the commonwealth might sustain neither harm nor disgrace by them.”

The ways to meet with disorders growing by abuse of laws are not so intricate and secret, especially in our case, that men should need either much advertisement or long time for the search thereof. And if counsel to that purpose may seem  needful, this Church (God be thanked) is not destitute of men endued with ripe judgment whensoever any such thing shall be thought necessary.
 For which end at this present to propose any special inventions of mine own might argue in a man of my place and calling more presumption perhaps than wit.

[16.]I will therefore leave it entire unto graver consideration, ending now with request only and most earnest suit: first that they which give ordination would as they tender the very honour of Jesus Christ, the safety of men and the endless good of their own souls, take heed lest unnecessarily and through their default the Church be found worse or less furnished than it might be:

Secondly that they which by right of patronage have power to present unto spiritual livings, and may in that respect much damnify the Church of God, would for the ease of their own account in the dreadful day somewhat consider what it is to betray for gain the souls which Christ hath redeemed with blood, what to violate the sacred bond of fidelity and solemn promise given at the first to God and his Church by them, from whose original interest together with the selfsame title of right the same obligation of duty likewise is descended:

Thirdly that they unto whom the granting of dispensations is committed, or which otherwise have any stroke in the disposition of such preferments as appertain unto learned men, would bethink themselves what it is to respect any thing either above or besides merit; considering how hardly the world taketh it when to men of commendable note and quality there is so little respect had, or so great unto them whose deserts are very mean, that nothing doth seem more strange than the one sort because they are not accounted of, and the other because they are; it being every man’s hope and expectation in the church of God especially that the only purchase of greater rewards should be always greater deserts, and that nothing should ever be able to plant a thorn where a vine ought to grow:

Fourthly that honourable personages, and they who by virtue of any principal office in the commonwealth are enabled to qualify a certain number and make them capable of favours  or faculties above others, suffer not their names to be abused contrary to the true intent and meaning of wholesome laws by men in whom there is nothing notable besides covetousness and ambition:

Fifthly that the graver and wiser sort in both universities, or whosoever they be with whose approbation the marks and recognizances of all learning are bestowed, would think the Apostle’s caution against unadvised ordinations not impertinent or unnecessary to be borne in mind even when they grant those degrees of schools, which degrees are not gratiæ gratis datæ, kindnesses bestowed by way of humanity, but they are gratiæ gratum facientes, favours which always imply a testimony given to the Church and commonwealth concerning men’s sufficiency for manners and knowledge, a testimony upon the credit whereof sundry statutes of the realm are built, a testimony so far available that nothing is more respected for the warrant of divers men’s abilities to serve in the affairs of the realm, a testimony wherein if they violate that religion wherewith it ought to be always given, and do thereby induce into error such as deem it a thing uncivil to call the credit thereof in question, let them look that God shall return back upon their heads and cause them in the state of their own corporations to feel either one way or other the punishment of those harms which the Church through their negligence doth sustain in that behalf:

Finally and to conclude, that they who enjoy the benefit of any special indulgence or favour which the laws permit would as well remember what in duty towards the Church and in conscience towards God they ought to do, as what they may do by using to their own advantage whatsoever they see tolerated; no man being ignorant that the cause why absence in some cases hath been yielded unto and in equity thought sufferable is the hope of greater fruit through industry elsewhere; the reason likewise wherefore pluralities are allowed unto men of note, a very sovereign and special care that as fathers in the ancient world did declare the preeminence of priority in birth by doubling the worldly portions of their first-born, so the Church by a course not unlike in assigning men’s rewards might testify an estimation had proportionably of their virtues, according to the  ancient rule apostolic, “They which excel in labour ought to excel in honour;”
 and therefore unless they answer faithfully the expectation of the Church herein, unless sincerely they bend their wits day and night both to sow because they reap, and to sow as much more abundantly as they reap more abundantly than other men, whereunto by their very acceptance of such benignities they formally bind themselves, let them be well assured that the honey which they eat with fraud shall turn in the end into true gall, forasmuch as laws are the sacred image of his wisdom who most severely punisheth those colourable and subtle crimes that seldom are taken within the walk of human justice.

[17.]I therefore conclude that the grounds and maxims of common right, whereupon ordinations of ministers unable to preach, tolerations of absence from their cures, and the multiplications of their spiritual livings are disproved, do but indefinitely enforce them unlawful, not unlawful universally and without exception; that the laws which indefinitely are against all these things, and the privileges which make for them in certain cases are not the one repugnant to the other; that the laws of God and nature are violated through the effects of abused privileges; that neither our ordinations of men unable to make sermons nor our dispensations for the rest, can be justly proved frustrate by virtue of any such surmised opposition between the special laws of this Church which have permitted and those general which are alleged to disprove the same; that when privileges by abuse are grown incommodious there must be redress: that for remedy of such evils there is no necessity the Church should abrogate either in whole or in part the specialties before-mentioned; and that the most to be desired were a voluntary reformation thereof on all hands which may give passage unto any abuse.






