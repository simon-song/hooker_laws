\chapter*[The Sixth Book]{THE SIXTH BOOK. 
CONTAINING THEIR FIFTH ASSERTION, WHICH IS, THAT OUR LAWS ARE CORRUPT AND REPUGNANT TO THE LAWS OF GOD, IN MATTER BELONGING TO THE POWER OF ECCLESIASTICAL JURISDICTION, IN THAT WE HAVE NOT THROUGHOUT ALL CHURCHES CERTAIN LAY-ELDERS ESTABLISHED FOR THE EXERCISE OF THAT POWER.}
\label{chap:book6}
\addcontentsline{toc}{chapter}{THE SIXTH BOOK}

%\PRLsep

\section*{The question between us, whether all congregations or parishes ought to have lay-elders invested with power of jurisdiction in spiritual causes.}

I. THE same men which in heat of contention do hardly either speak or give ear to reason, being after sharp and bitter conflict retired to a calm remembrance of all their former proceedings; the causes that brought them into quarrel, the course which their stirring affections have followed, and the issue whereunto they are come;
may per-adventure, as troubled waters, in small time, of their own accord, by certain easy degrees settle themselves again, and so recover that clearness of well-advised judgment, whereby they shall stand at the length indifferent, both to yield and admit any reasonable satisfaction, where before they could not endure with patience to be gainsayed. Neither will I despair of the like success in these unpleasant controversies touching ecclesiastical policy; the time of silence which both parts have willingly taken to breathe, seeming now as it were a pledge of all men’s quiet contentment to hear with more indifferency the weightiest and last remains of that cause, Jurisdiction, Dignity, Dominion Ecclesiastical. For, let not any man imagine, that the bare and naked difference of a few ceremonies could either have kindled so much fire, or have caused it to flame so long; but that the parties which herein laboured mightily for change, and (as they say) for Reformation, had somewhat more than this mark only whereat to aim.

[2]Having therefore drawn out a complete form, as they supposed, of public service to be done to God, and set down their plot for the office of the ministry in that behalf; they very well knew how little their labours so far forth bestowed would avail them in the end, without a claim of jurisdiction to uphold the fabric which they had erected; and this neither likely to be obtained but by the strong hand of the people, nor the people unlikely to favour it; the more, if overture were made of their own interest, right, and title thereunto. Whereupon there are many which have conjectured this to be the cause, why in all the projects of their discipline (it being manifest that their drift is to wrest the key of spiritual authority out of the hands of former governors, and equally to possess therewith the pastors of all several congregations) the people, first for surer accomplishment, and then for better defence thereof, are pretended necessary actors in those things, whereunto their ability for the most part is as slender, as their title and challenge unjust.

[3]Notwithstanding whether they saw it necessary for them so to persuade the people,  without whose help they could do nothing; or else, (which I rather think,) the affection which they bare towards this new form of government made them to imagine it God’s own ordinance, their doctrine is, “that by the law of God, there must be for ever in all congregations certain lay-elders, ministers of ecclesiastical jurisdiction,” inasmuch as our Lord and Saviour by testament (for so they presume) hath left all ministers or pastors in the Church executors equally to the whole power of spiritual jurisdiction, and with them hath joined the people as colleagues. By maintenance of which assertion there is unto that part apparently gained a twofold advantage; both because the people in this respect are much more easily drawn to favour it, as a matter of their own interest; and for that, if they chance to be crossed by such as oppose against them, the colour of divine authority, assumed for the grace and countenance of that power in the vulgar sort, furnisheth their leaders with great abundance of matter, behoveful for their encouragement to proceed always with hope of fortunate success in the end, considering their cause to be as David’s was, a just defence of power given them from above, and consequently, their adversaries’ quarrel the same with Saul’s by whom the ordinance of God was withstood.

[4]Now on the contrary side, if this their surmise prove false; if such, as in justification whereof no evidence sufficient either hath been or can be alleged (as I hope it shall clearly appear after due examination and trial), let them then consider whether those words of Corah, Dathan and Abiram against Moses and against Aaron, “It is too much that ye take upon you, seeing all the congregation is holy,” be not the very true abstract and abridgment of all their published Admonitions, Demonstrations, Supplications, and Treatises whatsoever, whereby they have laboured to void the rooms of their spiritual superiors before authorized, and to advance the new fancied sceptre by lay presbyterial power.

\section*{The nature of spiritual jurisdiction.}

II. But before there can be any settled determination, whether truth do rest on their part, or on ours, touching lay-elders; we are to prepare the way thereunto, by explication of some things requisite and very needful to be considered; as first, how besides that spiritual power which is of Order, and was instituted for performance of those duties whereof there hath been speech sufficient already had, there is in the Church no less necessary a second kind, which we call the power of Jurisdiction. When the Apostle doth speak of ruling the Church of God, and of receiving accusations, his words have evident reference to the power of jurisdiction. Our Saviour’s words to the power of order, when he giveth his disciples charge, saying, “Preach; baptize; do this in remembrance of me.” “A Bishop” (saith Ignatius) “doth bear the image of God and of Christ; of God in ruling, of Christ in administering, holy things4.” By this therefore we see a manifest difference acknowledged between the power of Ecclesiastical Order, and the power of Jurisdiction ecclesiastical.

[2]The spiritual power of the Church being such as neither can be challenged by right of nature, nor could by human authority be instituted, because the forces and effects thereof are supernatural and divine; we are to make no doubt or question, but that from him which is the Head it hath descended unto us that are the body now invested therewith. He gave it for the benefit and good of souls, as a mean to keep them in the path which leadeth unto endless felicity, a bridle to hold them within their due and convenient bounds, and if they do go astray, a forcible help to reclaim them. Now although there be no kind of spiritual power, for which our Lord Jesus Christ did not give both commission to exercise, and direction how to use the same, although his laws in that behalf recorded by the holy evangelists be the only ground and foundation, whereupon the practice of the Church must sustain itself: yet, as all multitudes, once grown to the form of societies, are even thereby naturally warranted to enforce upon their own subjects particularly those things which public wisdom shall judge expedient for the common good: so it were absurd to imagine the Church itself, the most glorious amongst them, abridged of this liberty; or to think that no law, constitution, or canon, can be further made either for limitation or amplification in the practice of our Saviour’s ordinances, whatsoever occasion be offered through variety of times and things, during the state of this unconstant world, which bringing forth daily such new evils as must of necessity by new remedies be redrest, did both of old enforce our venerable predecessors, and will always constrain others, sometime to make, sometime to abrogate, sometime to augment, and again to abridge sometime; in sum, often to vary, alter, and change customs incident into the manner of exercising that power which doth itself continue always one and the same. I therefore conclude, that spiritual authority is a power which Christ hath given to be used over them which are subject unto it for the eternal good of their souls, according to his own most sacred laws and the wholesome positive constitutions of his Church.

In doctrines referred unto action and practice, as this is which concerneth spiritual jurisdiction, the first step towards sound and perfect understanding is the knowledge of the end, because thereby both use doth frame, and contemplation judge all things.

\section*{Of penitence, the chiefest end propounded by spiritual Jurisdiction. Two kinds of Penitency; the one a private duty towards God, the other a duty of external discipline. Of the Virtue of Repentance, from which the former duty proceedeth; and of Contrition, the first part of that duty.}

III. Seeing then that the chiefest cause of spiritual jurisdiction is to provide for the health and safety of men’s souls, by bringing them to see and repent their grievous offences committed against God, as also to reform all injuries offered with the breach of Christian love and charity, towards their brethren, in matters of ecclesiastical cognizance; the use of this power shall by so much the plainlier appear, if first the nature of repentance itself be known.

We are by repentance to appease whom we offend by sin. For which cause, whereas all sins deprive us of the favour of Almighty God, our way of reconciliation with him is the inward secret repentance of the heart; which inward repentance alone sufficeth, unless some special thing, in the quality of sin committed, or in the party that hath done amiss, require more. For besides our submission in God’s sight, repentance must not only proceed to the private contentation of men, if the sin be a crime injurious; but also further, where the wholesome discipline of God’s Church exacteth a more exemplary and open satisfaction1. Now the Church being satisfied with outward repentance, as God is with inward, it shall not be amiss, for more perspicuity, to term this latter always the Virtue, that former the Discipline of Repentance: which discipline hath two sorts of penitents to work upon, inasmuch as it hath been accustomed to lay the offices of repentance on some seeking, others shunning them; on some at their own voluntary request, on others altogether against their wills; as shall hereafter appear by store of ancient examples. Repentance being therefore either in the sight of God alone, or else with the notice also of men: without the one, sometimes throughly performed, but always practised more or less, in our daily devotions and prayers, we have no remedy for any fault; whereas the other is only required in sins of a certain degree and quality: the one necessary for ever, the other so far forth as the laws and orders of God’s Church shall make it requisite: the nature, parts, and effects of the one always the same; the other limited, extended, varied by infinite occasions1.

[2] The virtue of repentance in the heart of man is God’s handy work, a fruit or effect of divine grace. Which grace continually offereth itself, even unto them that have forsaken it, as may appear by the words of Christ in St. John’s Revelation, “I stand at the door and knock:” nor doth he only knock without, but also within assist to open, whereby access and entrance is given to the heavenly presence of that saving power, which maketh man a repaired Temple for God’s good Spirit again to inhabit. And albeit the whole train of virtues which are implied in the name of grace be infused at one instant; yet because when they meet and concur unto any effect in man, they have their distinct operations rising orderly one from another; it is no unnecessary thing that we note the way or method of the Holy Ghost in framing man’s sinful heart to repentance.

A work, the first foundation whereof is laid by opening and illuminating the eye of faith, because by faith are discovered the principles of this action, whereunto unless the understanding do first assent, there can follow in the will towards penitency no inclination at all. Contrariwise, the resurrection of the dead, the judgment of the world to come, and the endless misery of sinners being apprehended, this worketh fear; such as theirs was, who feeling their own distress and perplexity, in that passion besought our Lord’s Apostles earnestly to give them counsel what they should do5. For fear is impotent and unable to advise itself; yet this good it hath, that men are thereby made desirous to prevent, if possibly they may, whatsoever evil they dread. The first thing that wrought the Ninivites’ repentance, was fear of destruction within forty days: signs and miraculous works of God, being extraordinary representations of divine power, are commonly wont to stir any the most wicked with terror, lest the same power should bend itself against them. And because tractable minds, though guilty of much sin, are hereby moved to forsake those evil ways which make his power in such sort their astonishment and fear; therefore our Saviour denounced his curse against Corazin and Bethsaida, saying, that if Tyre and Sidon had seen that which they did, those signs which prevailed little with the one would have brought the other’s repentance. As the like thereunto did in the men given to curious arts, of whom the apostolic history saith, that “fear came upon them, and many which had followed vain sciences, burnt openly the very books out of which they had learned the same.” As fear of contumely and disgrace amongst men, together with other civil punishments, are a bridle to restrain from many heinous acts whereinto men’s outrage would otherwise break; so the fear of divine revenge and punishment, where it taketh place, doth make men desirous to be rid likewise from that inward guiltiness of sin, wherein they would else securely continue.

[3]Howbeit, when faith hath wrought a fear of the event of sin, yet repentance hereupon ensueth not, unless our belief conceive both the possibility and means to avert evil: the possibility, inasmuch as God is merciful, and most willing to have sin cured; the means, because he hath plainly taught what is requisite and shall suffice unto that purpose. The nature of all wicked men is, for fear of revenge to hate whom they most wrong; the nature of hatred, to wish that destroyed which it cannot brook; and from hence ariseth the furious endeavour of godless and obdurate sinners to extinguish in themselves the opinion of God, because they would not have him to be, whom execution of endless woe doth not suffer them to love. Every sin against God abateth, and continuance in sin extinguisheth our love towards him. It was therefore said to the angel of Ephesus having sinned, “Thou art fallen away from thy first love;” so that, as we never decay in love till we sin, in like sort neither can we possibly forsake sin, unless we first begin again to love. What is love towards God, but a desire of union with God? And shall we imagine a sinner converting himself to God, in whom there is no desire of union with God presupposed? I therefore conclude, that fear worketh no man’s inclination to repentance, till somewhat else have wrought in us love also. Our love and desire of union with God ariseth from the strong conceit which we have of his admirable goodness. The goodness of God which particularly moveth unto repentance, is his mercy towards mankind, notwithstanding sin: for let it once sink deeply into the mind of man, that howsoever we have injuried God, his very nature is averse from revenge, except unto sin we add obstinacy; otherwise always ready to accept our submission as a full discharge or recompense for all wrongs; and can we choose but begin to love him whom we have offended? or can we but begin to grieve that we have offended him whom we now love? Repentance considereth sin as a breach of the law of God, an act obnoxious to that revenge, which notwithstanding may be prevented, if we pacify God in time.

The root and beginning of penitency therefore is the consideration of our own sin, as a cause which hath procured the wrath, and a subject which doth need the mercy of God. For unto man’s understanding there being presented, on the one side, tribulation and anguish upon every soul that doth evil; on the other, eternal life unto them which by continuance in well-doing seek glory, and honour, and immortality: on the one hand, a curse to the children of disobedience; on the other, to lovers of righteousness all grace and benediction: yet between these extremes, that eternal God, from whose unspotted justice and undeserved mercy the lot of each inheritance proceedeth, is so inclinable rather to shew compassion than to take revenge, that all his speeches in Holy Scripture are almost nothing else but entreaties of men to prevent destruction by amendment of their wicked lives; all the works of his providence little other than mere allurements of the just to continue steadfast, and of the unrighteous to change their course; all his dealings and proceedings such towards true converts, as have even filled the grave writings of holy men with these and the like most sweet sentences: “Repentance (if I may so speak1) stoppeth God in his way, when being provoked by crimes past he cometh to revenge them with most just punishments; yea, it tieth as it were the hands of the avenger, and doth not suffer him to have his will.” Again, “The merciful eye of God towards men hath no power to withstand penitency, at what time soever it comes in presence.” And again, “God doth not take it so in evil part, though we wound that which he hath required us to keep whole, as that after we have taken hurt there should be in us no desire to receive his help.” Finally, lest I be carried too far in so large a sea, “There was never any man condemned of God but for neglect, nor justified except he had care, of repentance.”

[4]From these considerations, setting before our eyes our inexcusable both unthankfulness in disobeying so merciful, and foolishness in provoking so powerful a God, there ariseth necessarily a pensive and corrosive desire that we had done otherwise; a desire which suffereth us to foreslow no time, to feel no quietness within ourselves, to take neither sleep nor food with contentment, never to give over supplications, confessions, and other penitent duties, till the light of God’s reconciled favour shine in our darkened soul.

Fulgentius asking the question, why David’s confession should be held for effectual penitence, and not Saul’s; answereth, that the one hated sin, the other feared only punishment in this world: Saul’s acknowledgment of sin was fear, David’s both fear and also love. This was the fountain of Peter’s tears, this the life and spirit of David’s eloquence, in those most admirable hymns entitled Penitential, where the words of sorrow for sin do melt the very bowels of God remitting it, and the comforts of grace in remitting sin carry him which sorrowed rapt as it were into heaven with ecstasies of joy and gladness. The first motive of the Ninivites unto repentance was their belief in a sermon of fear, but the next and most immediate, an axiom of love; “Who can tell whether God will turn away his fierce wrath, that we perish not?” No conclusion such as theirs, “Let every man turn from his evil way,” but out of premises such as theirs were, fear and love. Wherefore the well-spring of repentance is faith, first breeding fear, and then love; which love causeth hope, hope resolution of attempt; “I will go to my Father, and say, I have sinned against heaven and against thee;” that is to say, I will do what the duty of a convert requireth.

[5] Now in a penitent’s or convert’s duty, there are included, first, the aversion of the will from sin; secondly, the submission of ourselves to God by supplication and prayer; thirdly, the purpose of a new life, testified with present works of amendment: which three things do very well seem to be comprised in one definition, by them which handle repentance, as a virtue that hateth, bewaileth, and sheweth a purpose to amend sin. We offend God in thought, word, and deed. To the first of which three, they make contrition; to the second, confession; and to the last, our works of satisfaction, answerable.

Contrition doth not here import those sudden pangs and convulsions of the mind which cause sometimes the most forsaken of God to retract their own doings; it is no natural passion or anguish, which riseth in us against our wills, but a deliberate aversion of the will of man from sin; which being always accompanied with grief, and grief oftentimes partly with tears, partly with other external signs, it hath been thought, that in these things contrition doth chiefly consist:  whereas the chiefest thing in contrition is that alteration whereby the will, which was before delighted with sin, doth now abhor and shun nothing more. But forasmuch as we cannot hate sin in ourselves without heaviness and grief, that there should be in us a thing of such hateful quality, the will averted from sin must needs make the affection suitable; yea, great reason why it should so do: for sith the will by conceiving sin hath deprived the soul of life; and of life there is no recovery without repentance, the death of sin; repentance not able to kill sin, but by withdrawing the will from it; the will unpossible to be withdrawn, unless it concur with a contrary affection to that which accompanied it before in evil: is it not clear that as an inordinate delight did first begin sin, so repentance must begin with a just sorrow, a sorrow of heart, and such a sorrow as renteth the heart; neither a feigned nor a slight sorrow; not feigned, lest it increase sin; nor slight, lest the pleasures of sin overmatch it.

[6] Wherefore of Grace, the highest cause from which man’s penitency doth proceed; of faith, fear, love, hope, what force and efficiency they have in repentance; of parts and duties thereunto belonging, comprehended in the schoolmen’s definitions; finally, of the first among those duties, contrition, which disliketh and bewaileth iniquity, let this suffice.

And because God will have offences by repentance not only abhorred within ourselves, but also with humble supplication displayed before him, and a testimony of amendment to be given, even by present works, worthy repentance, in that they are contrary to those we renounce and disclaim: although the virtue of repentance do require that her other two parts, confession and satisfaction, should here follow; yet seeing they belong as well to the discipline as to the virtue of repentance, and only differ for that in the one they are performed to man, in the other to God alone; I had rather distinguish them in joint handling, than handle them apart, because in quality and manner of practice they are distinct.

\section*{Of the Discipline of Repentance instituted by Christ, practised by the Fathers, converted by the Schoolmen into a Sacrament: and of Confession; that which belongeth to the virtue of repentance, that which was used among the Jews, that which the Papacy imagineth a Sacrament, and that which ancient discipline practised.}

IV. Our Lord and Saviour in the sixteenth of St. Matthew’s Gospel giveth his Apostles regiment in general over God’s Church1. For they that have the keys of the kingdom of heaven are thereby signified to be stewards of the house of God, under whom they guide, command, judge, and correct his family. The souls of men are God’s treasure, committed to the trust and fidelity of such as must render a strict account for the very least which is under their custody. God hath not invested them with power to make a revenue thereof, but to use it for the good of them whom Jesus Christ hath most dearly bought.

And because their office herein consisteth of sundry functions, some belonging to doctrine, some to discipline, all contained in the name of the Keys; they have for matters of discipline, as well litigious as criminal, their courts and consistories erected by the heavenly authority of his most sacred voice, who hath said, Dic Ecclesiæ, Tell the Church: against rebellious and contumacious persons which refuse to obey their sentence, armed they are with power to eject such out of the Church, to deprive them of the honours, rights, and privileges of Christian men, to make them as heathen and publicans, with whom society was hateful.

Furthermore, lest their acts should be slenderly accounted of, or had in contempt, whether they admit to the fellowship of saints or seclude from it, whether they bind offenders or set them again at liberty, whether they remit or retain sins, whatsoever is done by way of orderly and lawful proceeding, the Lord himself hath promised to ratify. This is that grand original warrant, by force whereof the guides and prelates in God’s Church, first his Apostles, and afterwards others following them successively, did both use and uphold that discipline, the end whereof is to heal men’s consciences, to cure their sins, to reclaim offenders from iniquity, and to make them by repentance just.

Neither hath it of ancient time for any other respect been accustomed to bind by ecclesiastical censures, to retain so bound till tokens of manifest repentance appeared, and upon apparent repentance to release, saving only because this was received as a most expedient method for the cure of sin.

[2] The course of discipline in former ages reformed open transgressors by putting them unto offices of open penitence; especially confession, whereby they declared their own crimes in the hearing of the whole Church, and were not from the time of their first convention capable of the holy mysteries of Christ, till they had solemnly discharged thist duty.

Offenders in secret, knowing themselves altogether as unworthy to be admitted to the Lord’s table, as the others which were withheld, being also persuaded, that if the Church did direct them in the offices of their penitency, and assist them with public prayer, they should more easily obtain that they sought, than by trusting wholly to their own endeavours; finally, having no impediment to stay them from it but bashfulness, which countervailed not the former inducements, and besides was greatly eased by the good construction which the charity of those times gave to such actions, wherein men’s piety and voluntary care to be reconciled to God, did purchase them much more love, than their faults (the testimonies of common frailty) were able to procure disgrace; they made it not nice to use some one of the ministers of God, by whom the rest might take notice of their faults, prescribe them convenient remedies, and in the end after public confession, all join in prayer unto God for them.

[3]The first beginners of this custom had the more followers, by means of that special favour which always was with good consideration shewed towards voluntary penitents above the rest. But as professors of Christian belief grew more in number, so they waxed worse, when kings and princes had submitted their dominions unto the sceptre of Jesus Christ, by means whereof persecution ceasing, the Church immediately became subject to those evils which peace and security bringeth forth; there was not now that love which before kept all things in tune, but every where schisms, discords, dissensions amongst men, conventicles of heretics, bent more vehemently against the sounder and better sort than very infidels and heathens themselves; faults not corrected in charity, but noted with delight, and kept for malice to use when deadliest opportunities should be offered. Whereupon, forasmuch as public confessions became dangerous and prejudicial to the safety of well-minded men, and in divers respects advantageous to the enemies of God’s Church, it seemed first unto some, and afterwards generally, requisite, that voluntary penitents should surcease from open confession.

Instead whereof, when once private and secret confession had taken place with the Latins, it continued as a profitable ordinance, till the Lateran council had decreed, that all men once in a year at the least should confess themselves to the priest. So that being thus made a thingy both general and also necessary, the next degree of estimation whereunto it grew, was to be honoured and lifted up to the nature of a sacrament; that as Christ did institute Baptism to give life, and the Eucharist to nourish life, so Penitency might be thought a sacrament ordained to recover life, and Confession a part of the sacrament.

They define therefore their private penitency to be “a sacrament of remitting sins after baptism:” the virtue of repentance, “a detestation of wickedness, with full purpose to amend the same, and with hope to obtain pardon at God’s hands.” Wheresoever the Prophets cry Repent, and in the Gospel Saint Peter maketh the same exhortation to the Jews as yet unbaptized, they will have the virtue of repentance only to be understood; the sacrament, where he adviseth Simon Magus to repent, because the sin of Simon Magus was after baptism.

Now although they have only external repentance for a sacrament, internal for a virtue, yet make they sacramental repentance nevertheless to be composed of three parts, contrition, confession, and satisfaction: which is absurd; because contrition, being an inward thing, belongeth to the virtue and not to the sacrament of repentance, which must consist of external parts, if the nature thereof be external. Besides, which is more absurd, they leave out absolution; whereas some of their school-divines, handling penance in the nature of a sacrament, and being not able to espy the least resemblance of a sacrament save only in absolution (for a sacrament by their doctrine must both signify and also confer or bestow some special divine grace), resolved themselves, that the duties of the penitent could be but mere preparations to the sacrament, and that the sacrament itself was wholly in absolution. And albeit Thomas with his followers have thought it safer, to maintain as well the services of the penitent, as the words of the minister, necessary unto the essence of their sacrament; the services of the penitent, as a cause material; the words of absolution, as a formal; for that by them all things else are perfected to the taking away of sin; which opinion now reigneth in all their schools, sithence the time that the council of Trent gave it solemn approbation; seeing they all make absolution, if not the whole essence, yet the very form whereunto they ascribe chiefly the whole force and operation of their sacrament; surely to admit the matter as a part, and not to admit the form, hath small congruity with reason.

Again, forasmuch as a sacrament is complete, having the matter and form which it ought, what should lead them to set down any other part of sacramental repentance, than confession and absolution, as Durandus hath done? For touching satisfaction, the end thereof, as they understand it, is a further matter, which resteth after the sacrament administered, and therefore can be no part of the sacrament. Will they draw in contrition with satisfaction, which are no parts, and exclude absolution, a principal part, yea, the very complement, form, and perfection of the rest, as themselves account it?

[4] But for their breach of precepts in art, it skilleth not, if their doctrine otherwise concerning penitency, and in penitency, touching confession, might be found true. We say, let no man look for pardon, which doth smother and conceal sin, where in duty it should be revealed. The cause why God requireth confession to be made to him is, that thereby testifying a deep hatred of our own iniquitiesd, the only cause of his hatred and wrath towards us, we might, because we are humble, be so much the more capable of that compassion and tender mercy, which knoweth not how to condemn sinners that condemn themselves. If it be our Saviour’s own principle, that the conceit we have of our debt forgiven, proportioneth our thankfulness and love to him at whose hands we receive pardon, doth not God foresee, that they which with ill-advised modesty seek to hide their sin like Adam, that they which rake it up under ashes, and confess it not, are very unlikely to requite with offices of love afterwards the grace which they shew themselves unwilling to prize at the very time when they sue for it; inasmuch as their not confessing what crimes they have committed, is a plain signification, how loth they are that the benefit of God’s most gracious pardon should seem great? Nothing more true than that of Tertullian, “Confession doth as much abate the weight of men’s offences, as concealment doth make them heavier. For he which confesseth hath a purpose to appease God; he, a determination to persist and continue obstinate, which keepeth them secret to himself.” St. Chrysostom almost in the same words, “Wickedness is by being acknowledged lessened, and doth grow by being hid. If men having done amiss let it slip, as though they knew no such matter, what is there to stay them from falling often into one and the same evil? To call ourselves sinners availeth nothing, except we lay our faults in the balance, and take the weight of them one by one. Confess thy crimes to God, disclose thy transgressions before the Judge, by way of humble supplication and suit, if not with tongue, at the least with heart, and in this sort seek mercy. A general persuasion that thou art a sinner will neither so humble nor bridle thy soul, as if the catalogue of thy sins examined severally be continually kept in mind. This shall make thee lowly in thine own eyes, this shall preserve thy feet from falling, and sharpen thy desire towards all good things. The mind I know doth hardly admit such unpleasant remembrances, but we must force it, we must constrain it thereunto. It is safer now to be bitten with the memory, than hereafter with the torment of sin.”

The Jews, with whom no repentance for sin is held available without confession, either conceived in mind or uttered; which latter kind they call usually וִדּוּי, confession delivered by word of mouth; had first that general confession which once every year was made, both severally by each of the people for himself upon the day of expiation, and by the priest for them all, acknowledging unto God the manifold transgressions of the whole nation, his own personal offences likewise, together with the sins, as well of his family, as of the rest of his rank and order.

They had again their voluntary confessions, at all times and seasons, when men, bethinking themselves of their wicked conversation past, were resolved to change their course, the beginning of which alteration was still confession of sins.

Thirdly, over and besides these, the law imposed upon them also that special confession which they in their books call וִרּוִי עַל עָוׁן מְיוּחָר, confession of that particular fault for which we namely seek pardon at God’s hands. The words of the law concerning confession in this kind are as followeth: “When a man or woman shall commit any sin that men commit, and transgress against the Lord, their sin which they have done” (that is to say, the very deed itself in particular) “they shall acknowledge.” In Leviticus, after certain transgressions there mentioned, we read the like: “When a man hath sinned in any one of these things, he shall then confess, how in that thing he hath offended.” For such kind of special sins they had also special sacrifices, wherein the manner was, that the offender should lay his hands on the head of the sacrifice which he brought, and should there make confession to God, saying, “Now, O Lord, that I have offended, committed sin and done wickedly in thy sight, this or this being my fault; behold I repent me, and am utterly ashamed of my doings; my purpose is, never to return more to the same crime.”

Finally, there was no man amongst them at any time, either condemned to suffer death, or corrected, or chastised with stripes, none ever sick and near his end, but they called upon him to repent and confess his sins.

Of malefactors convict by witnesses, and thereupon either adjudged to die, or otherwise chastised, their custom was to exact, as Joshua did of Achan, open confession: “My son, now give glory to the Lord God of Israel; confess unto him, and declare unto me what thou hast committed; conceal it not from me.”

Concerning injuries and trespasses which happen between men, they highly commend such as will acknowledge before many. “2It is in him which repenteth accepted as an high sacrifice, if he will confess before many, make them acquainted with his oversights, and reveal the transgressions which have passed between him and any of his brethren; saying, I have verily offended this man, thus and thus I have done unto him; but behold I do now repent and am sorry. Contrariwise, whosoever is proud, and will not be known of his faults, but cloaketh them, is not yet come to perfect repentance; for so it is written, ‘He that hideth his sins shall not prosper:’ ” which words of Salomon they do not further extend, than only to sins committed against men, which are in that respect meet before men to be acknowledged particularly. “But in sins between man and God, there is no necessity that man should himself make any such open and particular recital of them:” to God they are known, and of us it is required, that we cast not the memory of them carelessly and loosely behind our backs, but keep in mind, as near as we can, both our own debt and his grace which remitteth the same.

[5] Wherefore, to let pass Jewish confession, and to come unto them which hold confession in the ear of the priest commanded, yea, commanded in the nature of a sacrament, and thereby so necessary that sin without it cannot be pardoned; let them find such a commandment in holy Scripture, and we ask no more. John the Baptist was an extraordinary person; his birth, his actions of life, his office extraordinary. It is therefore recorded for the strangeness of the act, but not set down as an everlasting law for the world, “that to him Jerusalem and all Judæa made confession of their sins;” besides, at the time of this confession, their pretended sacrament of repentance, as they grant, was not yet instituted; neither was it sin after baptism which penitents did there confess. When that which befell the seven sons of Sceva, for using the name of our Lord Jesus Christ in their conjurations, was notified to Jews and Grecians in Ephesus, it brought an universal fear upon them, insomuch that divers of them which had believed before, but not obeyed the laws of Christ as they should have done, being terrified by this example, came to the Apostle, and confessed their wicked deeds. Which good and virtuous act no wise man, (as I suppose,) will disallow, but commend highly in them, whom God’s good Spirit shall move to do the like when need requireth. Yet neither hath this example the force of any general commandment or law, to make it necessary for every man to pour into the ears of the priest whatsoever hath been done amiss, or else to remain everlastingly culpable and guilty of sin; in a word, it proveth confession practised as a virtuous act, but not commanded as a sacrament.

Now concerning St. James his exhortation, whether the former branch be considered, which saith, “Is any sick amongst you? let him call for the ancients of the Church, and let them make their prayers for him;” or the latter, which stirreth up all Christian men unto mutual acknowledgment of faults among themselves, “Lay open your minds, make your confessions one to another;” is it not plain, that the one hath relation to that gift of healing, which our Saviour promised his Church, saying, “They shall lay their hands on the sick, and the sick shall recover health;” relation to that gift of healing, whereby the Apostle imposed his hands on the father of Publius, and made him miraculously a sound man; relation, finally, to that gift of healing, which so long continued in practice after the Apostles’ times, that whereas the Novatianists denied the power of the Church of God in curing sin after baptism, St. Ambrose asked them again, “Why it might not as well prevail with God for spiritual as for corporal and bodily health; yea, wherefore,” saith he, “do ye yourselves lay hands on the diseased, and believe it to be a work of benediction or prayer, if happily the sick person be restored to his former safety?” And of the other member, which toucheth mutual confession, do not some of themselves, as namely Cajetan, deny that any other confession is meant, than only that, “which seeketh either association of prayers, or reconciliation, and pardon of wrongs?” Is it not confessed by the greatest part of their own retinue, that we cannot certainly affirm sacramental confession to have been meant or spoken of in this place? Howbeit Bellarmine, delighted to run a course by himself where colourable shifts of wit will but make the way passable, standeth as formally for this place, and no less for that in St. John, than for this.

St. John saith, “If we confess our sins, God is faithful and just to forgive our sins, and to cleanse us from all unrighteousness;” doth St. John say, If we confess to the priest, God is righteous to forgive; and if not, that our sins are unpardonable? No, but the titles of God, just and righteous, do import that he pardoneth sin only for his promise sake; “And there is not” (they say) “any promise of forgiveness upon confession made to God without the priest.” Not any promise, but with this condition, and yet this condition no where exprest? Is it not strange, that the Scripture speaking so much of repentance, and of the several duties which appertain thereunto, should ever mean, and no where mention, that one condition, without which all the rest is utterly of none effect? or will they say, because our Saviour hath said to his ministers, “Whose sins ye retain,” \&c. and because they can remit no more than what the offenders have confest, that therefore, by virtue of this promise, it standeth with the righteousness of God to take away no man’s sins, until by auricular confession they be opened unto the priest?

[6] They are men that would seem to honour antiquity, and none more to depend upon the reverend judgment thereof. I dare boldly affirm, that for many hundred years after Christ the Fathers held no such opinion; they did not gather by our Saviour’s words any such necessity of seeking the priest’s absolution from sin, by secret and (as they now term it) sacramental confession: public confession they thought necessary by way of discipline, not private confession, as in the nature of a sacrament, necessary.

For to begin with the purest times, it is unto them which read and judge without partiality a thing most clear, that the ancient ἐξομολόγησις or Confession, defined by Tertullian to be a discipline of humiliation and submission, framing men’s behaviour in such sort as may be fittest to move pity, the confession which they use to speak of in the exercise of repentance, was made openly in the hearing of the whole both ecclesiastical consistory and assembly. This is the reason [24] wherefore he perceiving that divers were better content their sores should secretly fester and eat inward, than be laid so open to the eyes of many, blameth greatly their unwise bashfulness, and to reform the same, persuadeth with them, saying, “Amongst thy brethren and fellow-servants, which are partakers with thee of one and the same nature, fear, joy, grief, sufferings, (for of one common Lord and Father we all have received one spirit,) why shouldst thou not think with thyself, that they are but thine ownself? wherefore dost thou avoid them, as likely to insult over thee, whom thou knowest subject to the same haps? At that which grieveth any one part, the whole body cannot rejoice, it must needs be that the whole will labour and strive to help that wherewith a part of itself is molested.”

St. Cyprian, being grieved with the dealings of them, who in time of persecution had through fear betrayed their faith, and notwithstanding thought by shift to avoid in that case the necessary discipline of the church, wrote for their better instruction the book intituled De Lapsis; a treatise concerning such as had openly forsaken their religion, and yet were loth openly to confess their fault in such manner as they should have done: in which book he compareth with this sort of men, certain others which had but a purpose only to have departed from the faith; and yet could not quiet their minds, till this very secret and hidden fault was confest: “How much both greater in faith,” saith St. Cyprian, “and also as touching their fear better are those men, who although neither sacrifice nor libel could be objected against them, yet because they thought to have done that which they should not, even this their intent they dolefully open unto God’s priests; they confess that whereof their conscience accuseth them, the burden that presseth their minds they discover, they foreslow not of smaller and slighter evils to seek remedy.” He saith, they declared their fault, not to one only man in private, but they revealed it to God’s priests; they confest it before the whole consistory of God’s ministers.

Salvianus, (for I willingly embrace their conjecture, who ascribe those homilies to him, which have hitherto by common error past under the counterfeit name of Eusebius Emesenus,) I say, Salvianus, though coming long after Cyprian in time, giveth nevertheless the same evidence for this truth, in a case very little different from that before alleged; his words are these: “2Whereas, most dearly beloved, we see that penance oftentimes is sought and sued for by holy souls which even from their youth have bequeathed themselves a precious treasure unto God, let us know that the inspiration of God’s good spirit moveth them so to do for the benefit of his Church, and let such as are wounded learn to inquire for that remedy, whereunto the very soundest do thus offer and obtrude as it were themselves, that if the virtuous do bewail small offences, the other cease not to lament great. And surely, when a man that hath less need, performeth sub oculis Ecclesiæ, in the view, sight, and beholding of the whole Church, an office worthy of his faith and compunction for sin, the good which others thereby reap is his own harvest, the heap of his rewards groweth by that which another gaineth, and through a kind of spiritual usury, from that amendment of life which others learn by him, there returneth lucre into his coffers.”

The same Salvianus, in another of his Homilies, “If faults happily be not great and grievous, (for example, if a man have offended in word, or in desire, worthy of reproof, if in the wantonness of his eye, or the vanity of his heart,) the stains of words and thoughts are by daily prayer to be cleansed, and by private compunction to be scoured out: but if any man examining inwardly his own conscience, have committed some high and capital offence, as, if by bearing false witness he have quelled and betrayed his faith, and by rashness of perjury have violated the sacred name of truth; if with the mire of lustful uncleanness he have sullied the veil of baptism, and the gorgeous robe of virginity; if by being the cause of any man’s death, he have been the death of the new man within himself; if by conference with soothsayers, wizards, and charmers, he hath enthralled himself to Satan: these and such like committed crimes cannot throughly be taken away with ordinary, moderate, and secret satisfaction; but greater causes do require greater and sharper remedies: they need such remedies as are not only sharp, but solemn, open, and public.” Again, “Let that soul,” saith he, “answer me, which through pernicious shamefastness is now so abasht to acknowledge his sin in conspectu fratrum, before his brethren, as he should have been before abasht to commit the same, what he will do in the presence of that Divine tribunal, where he is to stand arraigned in the assembly of a glorious and celestial host?”

I will hereunto add but St. Ambrose’s testimony; for the places which I might allege are more than the cause itself needeth. “There are many,” saith he, “who fearing the judgment that is to come, and feeling inward remorse of conscience, when they have offered themselves unto penitency and are enjoined what they shall do, give back for the only scar which they think that public supplication will put them unto.” He speaketh of them which sought voluntarily to be penanced, and yet withdrew themselves from open confession, which they that were penitents for public crimes could not possibly have done, and therefore it cannot be said he meaneth any other than secret sinners in that place.

Gennadius, a Presbyter of Marsiles, in his book touching Ecclesiastical Assertions, maketh but two kinds of confession necessary: the one in private to God alone for smaller offences; the other open, when crimes committed are heinous and great: “Although,” saith he, “a man be bitten with the conscience of sin, let his will be from thenceforward to sin no more; let him, before he communicate, satisfy with tears and prayers, and then putting his trust in the mercy of Almighty God (whose wont is to yield unto godly confessions) let him boldly receive the sacrament. But I speak this of such as have not burthened themselves with capital sins: them I exhort to satisfy first by public penance,  that so being reconciled by the sentence of the priest, they may communicate safely with others.”

Thus still we hear of public confessions, although the crimes themselves discovered were not public; we hear that the cause of such confessions was not the openness, but the greatness, of men’s offences; finally, we hear that the same being now not held by the church of Rome to be sacramental, were the only penitential confessions used in the Church for a long time, and esteemed as necessary remedies against sin.

They which will find auricular confessions in St. Cyprian, therefore, must seek out some other passage than that which Bellarmine allegeth; “Whereas in smaller faults which are not committed against the Lord himself, there is a competent time assigned unto penitency, and that confession is made, after that observation and trial had been had of the penitent’s behaviour, neither may any communicate till the Bishop and clergy have laid their hands upon him; how much more ought all things to be warily and stayedly observed, according to the discipline of the Lord, in those most grievous and extreme crimes.” St. Cyprian’s speech is against rashness in admitting idolaters to the holy Communion, before they had shewed sufficient repentance, considering that other offenders were forced to stay out their time, and that they made not their public confession, which was the last act of penitency, till their life and conversation had been seen into, not with the eye of auricular scrutiny, but of pastoral observation, according to that in the council of Nice, where, thirteen years being set for the penitency of certain offenders, the severity of this decree is mitigated with special caution: “That in all such cases, the mind of the penitent and the manner of his repentance is to be noted, that as many as with fear and tears and meekness, and the exercise of good works, declared themselves to be converts indeed, and not in outward appearance only, towards them the bishop at his discretion might use more lenity.” If the council of Nice suffice not, let Gratian, the founder of the Canon Law, expound Cyprian, who sheweth that the stint of time in penitency is either to be abridged or enlarged, as the penitent’s faith and behaviour shall give occasion. “I have easilier found out men,” saith St. Ambrose, “able to keep themselves free from crimes, than conformable to the rules which in penitency they should observe.” St. Gregory Bishop of Nyser complaineth and inveigheth bitterly against them, who in the time of their penitency lived even as they had done always before: “Their countenance as cheerful, their attire as neat, their diet as costly, and their sleep as secure as ever, their worldly business purposely followed, to exile pensive thoughts from their minds, repentance pretended, but indeed nothing less exprest:” these were the inspections of life whereunto St. Cyprian alludeth; as for auricular examinations he knew them not.

[7] Were the Fathers then without use of private confession as long as public was in use? I affirm no such thing.  The first and ancientest that mentioneth this confession is Origen, by whom it may seem that men, being loth to present rashly themselves and their faults unto the view of the whole Church, thought it best to unfold first their minds to some one special man of the clergy, which might either help them himself, or refer them to an higher court, if need were. “Be therefore circumspect,” saith Origen, “in making choice of the party to whom thou meanest to confess thy sin; know thy physician before thou use him: if he find thy malady such as needeth to be made public, that others may be the better by it, and thyself sooner help, his counsel must be obeyed and followed.”

That which moved sinners thus voluntarily to detect themselves both in private and in public, was fear to receive with other Christian men the mysteries of heavenly grace, till God’s appointed stewards and ministers did judge them worthy. It is in this respect that St. Ambrose findeth fault with certain men which sought imposition of penance, and were not willing to wait their time, but would be presently admitted communicants. “Such people,” saith he, “do seek, by so rash and preposterous desires, rather to bring the priest into bonds than to loose themselves.” In this respect it is that St. Augustine hath likewise said, “When  the wound of sin is so wide, and the disease so far gone, that the medicinable body and blood of our Lord may not be touched, men are by the Bishop’s authority to sequester themselves from the altar, till such time as they have repented, and be after reconciled by the same authority.”

Furthermore, because the knowledge how to handle our own sores is no vulgar and common art, but we either carry towards ourselves for the most part an over-soft and gentle hand, fearful of touching too near the quick; or else, endeavouring not to be partial, we fall into timorous scrupulosities, and sometimes into those extreme discomforts of mind, from which we hardly do ever lift up our heads again; men thought it the safest way to disclose their secret faults, and to crave imposition of penance from them whom our Lord Jesus Christ hath left in his Church to be spiritual and ghostly physicians, the guides and pastors of redeemed souls, whose office doth not only consist in general persuasions unto amendment of life, but also in the private particular cure of diseased minds.

Howsoever the Novatianists presume to plead against the Church, saith Salvianus, that “every man ought to be his own penitentiary, and that it is a part of our duty to exercise, but not of the Church’s authority to impose or prescribe repentance;” the truth is otherwise, the best and strongest of us may need in such cases direction: “What doth the Church in giving penance, but shew the remedies which sin requireth? or what do we in receiving the same, but fulfil her precepts? what else but sue unto God with tears and fasts, that his merciful ears may be opened?”

St. Augustine’s exhortation is directly to the same purpose; “2Let every man while he hath time judge himself, and  change his life of his own accord; and when this is resolved upon, let him from the disposers of the holy sacraments learn in what manner he is to pacify God’s displeasure.”

But the greatest thing which made men forward and willing upon their knees to confess whatsoever they had committed against God, and in no wise to be withheld from the same with any fear of disgrace, contempt, or obloquy, which might ensue, was their fervent desire to be helped and assisted with the prayers of God’s saints. Wherein as St. James doth exhort unto mutual confession, alleging this only for a reason, that just men’s devout prayers are of great avail with God; so it hath been heretofore the use of penitents for that intent to unburthen their minds, even to private persons, and to crave their prayers. Whereunto Cassianus alluding, counselleth, “That if men possest with dulness of spirit be themselves unapt to do that which is required, they should in meek affection seek health at the least by good and virtuous men’s prayers unto God for them.” And to the same effect Gregory, Bishop of Nyss: “Humble thyself, and take unto thee such of thy brethren as are of one mind, and do bear kind affection towards thee, that they may together mourn and labour for thy deliverance. Shew me thy bitter and abundant tears, that I may blend mine own with them.” But because of all men there is or should be none in that respect more fit for troubled and distressed minds to repair unto than God’s ministers, he proceedeth further: “Make the priest, as a father, partaker of thy affliction and grief; be bold to impart unto him the things that are most secret, he will have care both of thy safety and of thy credit.”


“Confession,” saith Leo, “is first to be offered to God, and then to the priest, as to one which maketh supplication for the sins of penitent offenders.” Suppose we, that men would ever have been easily drawn, much less of their own accord have come unto public confession, whereby they knew they should sound the trumpet of their own disgrace; would they willingly have done this, which naturally all men are loth to do, but for the singular trust and confidence which they had in the public prayers of God’s Church? “Let thy mother the Church weep for thee,” saith St. Ambrose, “let her wash and bathe thy faults with her tears: our Lord doth love that many should become suppliants for one.” In like sort, long before him, Tertullian, “Some few assembled make a Church, and the Church is as Christ himself; when thou dost therefore put forth thy hands to the knees of thy brethren, thou touchest Christ; it is Christ unto whom thou art a suppliant; so when they pour out their tears over them, it is even Christ that taketh compassion; Christ which prayeth when they pray: neither can that be easily denied, for which the Son is himself contented to become a suitor.”

[8] Whereas in these considerations therefore, voluntary penitents had been long accustomed, for great and grievous crimes, though secret, yet openly both to repent and confess, as the canons of ancient discipline required; the Greek church first, and in process of time the Latin altered this order, judging it sufficient and more convenient that such offenders should do penance and make confession in private only. The cause why the Latins did, Leo declareth, saying,  “1Although that ripeness of faith be commendable, which for the fear of God doth not fear to incur shame before all men; yet because every one’s crimes are not such, that it can be free and safe for them to make publication of all things wherein repentance is necessary; let a custom so unfit to be kept be abrogated, lest many forbear to use the remedies of penitency, whilst they either blush or are afraid to acquaint their enemies with those acts for which the laws may take hold upon them. Besides, it shall win the more to repentance, if the consciences of sinners be not emptied into the people’s ears.” And to this only cause doth Sozomen impute the change which the Grecians made, by ordaining throughout all churches certain penitentiaries to take the confessions, and appoint the penances of secret offenders. Socrates (for this also may be true, that moe inducements than one did set forward an alteration so generally made) affirmeth the Grecians (and not unlikely) to have especially respected therein the occasion, which the Novatianists took at the multitude of public penitents, to insult over the discipline of the Church, against which they still cried out wheresoever they had time and place, “He that sheweth sinners favour, doth but teach the innocent to  sin.” And therefore they themselves admitted no man to their communion upon any repentance, which once was known to have offended after baptism, making sinners thereby not the fewer, but the closer and the more obdurate, how fair soever their pretence might seem.

[9] The Grecians’ canon for some one presbyter in every Church to undertake the charge of penitency, and to receive their voluntary confessions which had sinned after baptism, continued in force for the space of about some hundred years, till Nectarius, and the bishops of churches under him, began a second alteration, abolishing even that confession which their penitentiaries took in private. There came to the penitentiary of the Church of Constantinople a certain gentlewoman, and to him she made particular confession of her faults committed after baptism, whom thereupon he advised to continue in fasting and prayer, that as with tongue she had acknowledged her sins, so there might appear in her likewise some work worthy of repentance. But the gentlewoman goeth forward, and detecteth herself of a crime, whereby they were forced to disrobe an ecclesiastical person, that is, to degrade a deacon of the same Church. When the matter by this mean came to public notice, the people were in a kind of tumult offended, not only at that which was done,  but much more, because the Church should thereby endure open infamy and scorn. The clergy perplexed and altogether doubtful what way to take, till one Eudæmon, born in Alexandria, but at that time a priest in the church of Constantinople, considering that the cause of voluntary confession, whether public or private, was especially to seek the Church’s aid, as hath been before declared, lest men should either not communicate with others, or wittingly hazard their souls, if so be they did communicate, and that the inconvenience which grew to the whole Church was otherwise exceeding great, but especially grievous by means of so manifold offensive detections, which must needs be continually more, as the world did itself wax continually worse (for antiquity together with the gravity and severity thereof (saith Sozomen) had already begun by little and little to degenerate into loose and careless living, whereas before offences were less, partly through bashfulness in them which opened their own faults, and partly by means of their great austerity which sate as judges in this business): these things Eudæmon having weighed with himself, resolved easily the mind of Nectarius, that the penitentiaries’ office must be taken away, and for participation in God’s holy mysteries every man be left to his own conscience; which was, as he thought, the only mean to free the Church from danger of obloquy and disgrace. “Thus much,” saith Socrates, “I am the bolder to relate, because I received it from Eudæmon’s own mouth, to whom my answer was at that time; Whether your counsel, sir, have been for the Church’s good, or otherwise, God knoweth: but I see  you have given occasion, whereby we shall not now any more reprehend one another’s faults, nor observe that apostolic precept, which saith, Have no fellowship with the unfruitful works of darkness, but rather be ye also reprovers of them.” With Socrates, Sozomen both agreeth in the occasion of abolishing penitentiaries; and moreover testifieth also, that in his time, living with the younger Theodosius, the same abolition did still continue, and that the bishops had in a manner every where followed the example given them by Nectarius.

[10] Wherefore to implead the truth of this history, Cardinal Baronius allegeth that Socrates, Sozomen and Eudæmon were all Novatianists; and that they falsify in saying (for so they report), that as many as held the consubstantial being of Christ, gave their assent to the abrogation of the forerehearsed canon. The sum is, he would have it taken for a fable, and the world to be persuaded that Nectarius did never any such thing. Why then should Socrates first and afterwards Sozomen publish it? To please their pew-fellows, the disciples of Novatian. A poor gratification, and they very silly friends, that would take lies for good turns. For the more acceptable the matter was, being deemed true, the less they must needs (when they found the contrary) either credit or affect him, which had deceived them. Notwithstanding we know that joy and gladness rising from false information, do not only make men forward to believe that which they  first hear, but also apt to scholie upon it, and to report as true whatsoever they wish were true. But so far is Socrates from any such purpose, that the fact of Nectarius, which others did both like and follow, he doth both disallow and reprove. His speech to Eudæmon, before set down, is proof sufficient that he writeth nothing but what was famously known to all, and what himself did wish had been otherwise. As for Sozomen’s correspondence with heretics, having shewed to what end the Church did first ordain penitentiaries, he addeth immediately, that Novatianists, which had no care of repentance, could have no need of this office1. Are these the words of a friend or an enemy? Besides, in the entrance of that whole narration, “Not to sin,” saith he, “at all, would require a nature more divine than ours is: but God hath commanded to pardon sinners: yea, although they transgress and offend often.” Could there be any thing spoken more directly opposite to the doctrine of Novatian?

Eudæmon was presbyter under Nectarius. To Novatianists the Emperor gave liberty of using their religion quietly by themselves, under a bishop of their own, even within the city, for that they stood with the Church in defence of the Catholic faith against all other heretics besides3. Had therefore Eudæmon favoured their heresy, their camps were not pitched so far off, but he might at all times have found easy access unto them. Is there any man that lived with him, and hath touched him that way? if not, why suspect we him more than Nectarius?

Their report touching Grecian catholic bishops, who gave approbation to that which was done, and did also the like themselves in their own churches, we have no reason to discredit, without some manifest and clear evidence brought against it. For of Catholic bishops, no likelihood but that their greatest respect to Nectarius, a man honoured in those  parts no less than the Bishop of Rome himself in the western churches, brought them both easily and speedily unto conformity with him; Arians, Eunomians, Apollinarians, and the rest that stood divided from the Church, held their penitentiaries as before. Novatianists from the beginning had never any, because their opinion touching penitency was against the practice of the Church therein, and a cause why they severed themselves from the Church: so that the very state of things as they then stood, giveth great show of probability to his speech, who hath affirmed, “That they only which held the Son consubstantial with the Father, and Novatianists which joined with them in the same opinion, had no penitentiaries in their churches, the rest retained them.”

By this it appeareth therefore how Baronius, finding the relation plain, that Nectarius did abolish even those private secret confessions, which the people had before been accustomed to make to him that was penitentiary, laboureth what he may to discredit the authors of the report, and to leave it imprinted in men’s minds, that whereas Nectarius did but abrogate public confession, Novatianists have maliciously forged the abolition of private. As if the odds between these two were so great in the balance of their judgment, which equally hated and contemned both; or, as if it were not more clear than light, that the first alteration which established penitentiaries took away the burthen of public confession in that kind of penitents, and therefore the second must either abrogate private, or nothing.

[11] Cardinal Bellarmine therefore finding that against the writers of the history it is but in vain to stand upon so doubtful terms and exceptions, endeavoureth mightily to prove, even by their report, no other confession taken away than public, which penitentiaries used in private to impose upon public offenders2. “For why? It is,” saith he, “very  certain, that the name of penitents in the Fathers’ writings signifieth only public penitents; certain, that to hear the confessions of the rest was more than one could possibly have done; certain, that Sozomen, to shew how the Latin Church retained in his time what the Greek had clean cast off, declareth the whole order of public penitency used in the Church of Rome, but of private he maketh no mention.” And, in these considerations, Bellarmine will have it the meaning both of Socrates and of Sozomen, that the former episcopal constitution, which first did erect penitentiaries, could not concern any other offenders, than such as publicly had sinned after baptism; that only they were prohibited to come to the holy communion, except they did first in secret confess all their sins to the penitentiary, by his appointment openly acknowledge their open crimes, and do public penance for them; that whereas, before Novatian’s uprising, no man was constrainable to confess publicly any sin, this canon enforced public offenders thereunto, till such time as Nectarius thought good to extinguish the practice thereof.

Let us examine therefore these subtile and fine conjectures, whether they be able to hold the touch. “It seemed good,” saith Socrates, “to put down the office of these priests which had charge of penitency;” what charge that was, the  kinds of penitency then usual must make manifest. There is often speech in the Fathers’ writings, in their books frequent mention of penitency, exercised within the chambers of our own heart, and seen of God, and not communicated to any other, the whole charge of which penitency is imposed of God, and doth rest upon the sinner himself. But if penitents in secret being guilty of crimes whereby they knew they had made themselves unfit guests for the table of our Lord, did seek direction for their better performance of that which should set them clear; it was in this case the Penitentiary’s office to take their confessions, to advise them the best way he could for their soul’s good, to admonish them, to counsel them, but not to lay upon them more than private penance. As for notorious wicked persons, whose crimes were known, to convent, judge, and punish them, was the office of the ecclesiastical consistory; Penitentiaries had their institution to another end. Now unless we imagine that the ancient time knew no other repentance than public, or that they had little occasion to speak of any other repentance, or else that in speaking thereof they used continually some other name, and not the name of repentance, whereby to express private penitency; how standeth it with reason, that wheresoever they write of penitents, it should be thought they meant only public penitents? The truth is, they handle all three kinds, but private and voluntary repentance much oftener, as being of far more general use; whereas public was but incident unto few, and not oftener than once incident unto any. Howbeit, because they do not distinguish one kind of penitency from another by difference of names, our safest way for construction is to follow circumstance of matter, which in this narration will not yield itself appliable only unto public penance, do what they can that would so expound it.

They boldly and confidently affirm, that no man being compellable to confess publicly any sin before Novatian’s time, the end of instituting penitentiaries afterward in the Church was, that by them men might be constrained unto public confession. Is there any record in the world which doth testify this to be true? There is that testifieth the plain contrary. For Sozomen  declaring purposely the cause of their institution, saith, “That whereas men openly craving pardon at God’s hands (for public confession, the last act of penitency, was always made in the form of a contrite prayer unto God), it could not be avoided but they must withal confess what their offences were; this in the opinion of their prelates seemed from the first beginning (as we may probably think) to be somewhat burthensome;” not burthensome, I thinks, to notorious offenders; for what more just than in such sort to discipline them? but burthensome, that men whose crimes were unknown should blaze their own faults as it were on a stage, acquainting all the people with whatsoever they had done amiss. And therefore to remedy this inconvenience, they laid the charge upon one only priest, chosen out of such as were of best conversation, a silent and a discreet man, to whom they which had offended might resort and lay open their lives. He according to the quality of every one’s transgressions appointed what they should do or suffer, and left them to execute it upon themselves. Can we wish a more direct and evident testimony, that the office here spoken of was to ease voluntary penitents from the burthen of public confessions, and not to constrain notorious offenders thereunto? That such offenders were not compellable to open confession till Novatian’s time, that is to say, till after the days of persecution under Decius the emperor, they of all men should not so peremptorily avouch; with whom if Fabian bishop of Rome, who suffered martyrdom uthe first year of Decius, be of any authority and credit, it must enforce them to reverse their sentence, his words are so plain and clear against them. “For such as commit those crimes, whereof the Apostle hath said, They that do them shall never inherit the kingdom of heaven,  must,” saith he, “be forced unto amendment, because they slip down to hell, if ecclesiastical authority stay them not.” Their conceit of impossibility, that one man should suffice to take the general charge of penitency in such a church as Constantinople, hath arisen from a mere erroneous supposal, that the ancient manner of private confession was like the shrift at this day usual in the Church of Rome, which tieth all men at one certain time to make confession; whereas confession was then neither looked for till men did offer it, nor offered for the most part by any other than such as were guilty of heinous transgressions, nor to them any time appointed for that purpose. Finally, the drift which Sozomen had in relating the discipline of Rome, and the form of public penitency there retained even till his time, is not to signify that only public confession was abrogated by Nectarius, but that the West or Latin Church held still one and the same order from the very beginning, and had not, as the Greek, first cut off public voluntary confession by ordaining, and then private by removing Penitentiaries.

Wherefore to conclude, it standeth, I hope, very plain and clear, first against the one Cardinal, that Nectarius did truly abrogate confession in such sort as the ecclesiastical history hath reported; and secondly, as clear against them both, that it was not public confession only which Nectarius did abolish.

[12] The paradox in maintenance whereof Hassels wrote purposely a book touching this argument, to shew that Nectarius did but put the penitentiary from his office, and not take away the office itself, is repugnant to the whole advice which Eudæmon gave, of leaving the people from that time forward to their own consciences; repugnant to the conference between Socrates and Eudæmon, wherein complaint is made of some inconvenience which the want of the office would breed; finally, repugnant to that which the history declareth concerning other churches, which did as Nectarius had done before them, not in deposing the same man (for that was impossible) but in removing the same office out of their churches, which Nectarius had banished from his. For which cause  Bellarmine doth well reject the opinion of Hessels, howsoever it please Pamelius to admire it as a wonderful happy invention. But in sum, they are all gravelled, no one of them able to go smoothly away, and to satisfy either others or himself with his own conceit concerning Nectarius.

[13] Only in this they are stiff, that auricular confession Nectarius did not abrogate, lest if so much should be acknowledged, it might enforce them to grant that the Greek church at that time held not confession, as the Latin now doth, to be the part of a sacrament instituted by our Saviour Jesus Christ, which therefore the Church till the world’s end hath no power to alter. Yet seeing that as long as public voluntary confession of private crimes did continue in either church (as in the one it remained not much above two hundred years, in the other about four hundred) the only acts of such repentance were; first, the offender’s intimation of those crimes to some one presbyter, for which imposition of penance was sought; secondly, the undertaking of penance imposed by the Bishop; thirdly, after the same performed and ended, open confession to God in the hearing of the whole church; whereupon ensued the prayers of the Church; then the Bishop’s imposition of hands; and so the party’s reconciliation or restitution to his former right in the holy sacrament: I would gladly know of them which make only private confession a part of their sacrament of penance, how it could be so in those times. For where the sacrament of penance is ministered, they hold that confession to be sacramental which he receiveth who must absolve; whereas during the fore-rehearsed manner of penance, it can no where be shewed, that the priest to whom secret information was given did reconcile or absolve any; for how could he, when public confession was to go before reconciliation, and reconciliation likewise in public thereupon to  ensue? So that if they did account any confession sacramental, it was surely public, which is now abolisht in the Church of Rome; and as for that which the Church of Rome doth so esteem, the ancient neither had it in such estimation, nor thought it to be of so absolute necessity for the taking away of sin.

But (for any thing that I could ever observe out of them) although not only in crimes open and notorious, which made men unworthy and uncapable of holy mysteries, their discipline required first public penance, and then granted that which St. Hierom mentioneth, saying, “The priest layeth his hand upon the penitent, and by invocation entreateth that the Holy Ghost may return to him again, and so after having enjoined solemnly all the people to pray for him, reconcileth to the altar him who was delivered to Satan for the destruction of his flesh, that his spirit might be safe in the day of the Lord:”—Although I say not only in such offences being famously known to the world, but also if the same were committed secretly, it was the custom of those times, both that private intimation should be given, and public confession made thereof; in which respect, whereas all men did willingly the one, but would as willingly have withdrawn themselves from the other, had they known how; “Is it tolerable,” saith St. Ambrose, “that to sue to God thou shouldst be ashamed, which blushest not to seek and sue unto man? Should it grieve thee to be a suppliant to him from whom thou canst not possibly hide thyself; when to open thy sins to him, from whom, if thou wouldst, thou  mightest conceal them, it doth not any thing at all trouble thee? This thou art loth to do in the Church, where, all being sinners, nothing is more opprobrious indeed than concealment of sin, the most humble the best thought of, and the lowliest accounted the justest:”—All this notwithstanding, we should do them very great wrong, to father any such opinion upon them, as if they did teach it a thing impossible for any sinner to reconcile himself unto God, without confession unto the priest. 1Would Chrysostom thus persuaded have said, “Let the inquiry and presentmente of thy offences be made in thine own thoughts; let the tribunal whereat thou arraignest thyself be without witness: let God and only God see thee and thy confession?” Would Cassianus, so believing, have given counsel, “That if any were withheld by bashfulness from discovering their faults to men, they should be so much the more instant and constant in opening them by supplication to God himself, whose wont is to help without publication of men’s shame, and not to upbraid them when he pardoneth?” Finally, would Prosper, settled in this opinion, have made it, as touching reconciliation to God, a matter indifferent, “Whether men of ecclesiastical order did detect their crimes by confession, or leaving the world ignorant thereof, would separate voluntarily themselves for a time from the altar, though not in affection, yet in execution of their ministry, and so bewail their corrupt life?” Would he have willed them as he doth “to make bold of it, that the favour of God being either way recovered by fruits  of forcible repentance, they should not only receive whatsoever they had lost by sin, but also after this their new enfranchisement, aspire to the endless joys of that supernal city?”

To conclude, we every where find the use of confession, especially public, allowed of and commended by the Fathers; but that extreme and rigorous necessity of auricular and private confession, which is at this day so mightily upheld by the church of Rome, we find not. It was not then the faith and doctrine of God’s Church, as of the papacy at this present, 1. That the only remedy for sin after baptism is sacramental penitency. 2. That confession in secret is an essential part thereof. 3. That God himself cannot now forgive sins without the priest. 4. That because forgiveness at the hands of the priest must arise from confession in the offender, therefore to confess unto him is a matter of such necessity, as being not either in deed, or at the least in desire performed, excludeth utterly from all pardon, and must consequently in Scripture be commanded, wheresoever any promise of forgiveness is made. No, no; these opinions have youth in their countenance; antiquity know them not, it never thought nor dreamed of them.

[14] But to let pass the papacy. Forasmuch as repentance doth import alteration within the mind of a sinful man, whereby through the power of God’s most gracious and blessed Spirit, he seeth and with unfeigned sorrow acknowledgeth former offences committed against God, hath them in utter detestation, seeketh pardon for them in such sort as a Christian should do, and with a resolute purpose settleth himself to avoid them, leading as near as God shall assist him, for ever after, an unspotted life; and in the order (which Christian religion hath taught for procurement of God’s mercy towards sinners) confession is acknowledged a principal duty; yea, in some cases, confession to man, not to God only; it is not in the reformed churches denied by the learneder sort of divines, but that even this confession, cleared from all errors, is both lawful and behoveful for God’s people.


Confession by man to man 1being either private or public, private confession to the minister alone touching secret crimes, or absolution thereupon ensuing, as the one, so the other is neither practised by the French discipline, nor used in any of those churches which have been cast by the French mould. Open confession to be made in the face of the whole congregation by notorious malefactors they hold necessary; howbeit not necessary towards the remission of sins, “but only in some sort to content the Church, and that one man’s repentance may seem to strengthen many, which before have been weakened by one man’s fall.”

Saxonians and Bohemians in their discipline constrain no man to open confession. Their doctrine is, that whose faults have been public, and thereby scandalous unto the world, such, when God giveth them the spirit of repentance, ought as solemnly to return, as they have openly gone astray: first, for the better testimony of their own unfeigned conversion unto God; secondly, the more to notify their reconcilement unto the church; and lastly, that others may make benefit of their ensample.

But concerning confession in private, the churches of Germany, as well the rest as Lutherans, agree all, that all  men should at certain times confess their offences to God in the hearing of God’s ministers, thereby to shew how their sins displease them; to receive instruction for the warier carriage of themselves hereafter; to be soundly resolved, if any scruple or snare of conscience do entangle their minds; and, which is most material, to the end that men may at God’s hands seek every one his own particular pardon, through the power of those keys, which the minister of God using according to our blessed Saviour’s institution in that case, it is their part to accept the benefit thereof as God’s most merciful ordinance for their good, and, without any distrust or doubt, to embrace joyfully his grace so given them, according to the word of our Lord, which hath said, “Whose sins ye remit they are remitted.” So that grounding upon this assured belief, they are to rest with minds encouraged and persuaded concerning the forgiveness of all their sins, as out of Christ’s own word and power, by the ministry of the keys.

[15] It standeth with us in the Church of England, as touching public confession, thus:

First, seeing day by day we in our Church begin our public prayers to Almighty God with public acknowledgment of our sins, in which confession every man prostrate as it were before his glorious Majesty crieth guilty against himself; and the minister with one sentence pronounceth universally all clear, whose acknowledgment so made hath proceeded from a true penitent mind; what reason is there every man should not under the general terms of confession represent to himself  his own particulars whatsoever, and adjoining thereunto that affection which a contrite spirit worketh, embrace to as full effect the words of divine Grace, as if the same were severally and particularly uttered with addition of prayers, imposition of hands, or all the ceremonies and solemnities that might be used for the strengthening of men’s affiance in God’s peculiar mercy towards them? Such complements are helps to support our weakness, and not causes that serve to procure or produce his gifts. If with us there be “truth in the inward parts,” as David speaketh, the difference of general and particular forms in confession and absolution is not so material, that any man’s safety or ghostly good should depend upon it.

And for private confession and absolution it standeth thus with us:

The minister’s power to absolve is publicly taught and professed, the Church not denied to have authority either of abridging or enlarging the use and exercise of that power, upon the people no such necessity imposed of opening their transgressions unto men, as if remission of sins otherwise were impossible; neither any such opinion had of the thing itself, as though it were either unlawful or unprofitable, saving only for these inconveniences, which the world hath by experience observed in it heretofore. And in regard thereof, the Church of England hitherto hath thought it the safer way to refer men’s hidden crimes unto God and themselves only; howbeit, not without special caution for the admonition of such as come to the holy Sacrament, and for the comfort of such as are ready to depart the world.

First, because there are but few that consider how much that part of divine service which consisteth in partaking the holy Eucharist doth import their souls; what they lose by neglect thereof, and what by devout practice they might attain unto: therefore, lest carelessness of general confession  should, as commonly it doth, extinguish all remorse of men’s particular enormous crimes; our custom (whensoever men present themselves at the Lord’s Table) is, solemnly to give them very fearful admonitions what woes are perpendicularly hanging over the heads of such as dare adventure to put forth their unworthy hands to those admirable mysteries of life, which have by rare examples been proved conduits of irremediable death to impenitent receivers; whom therefore as we repel being known, so being not known we can but terrify. Yet with us, the ministers of God’s most holy word and sacraments, being all put in trust with the custody and dispensation of those mysteries, wherein our communion is and hath been ever accounted the highest grace that men on earth are admitted unto, have therefore all equally the same power to withhold that sacred mystical food from notorious evil livers, from such as have any way wronged their neighbours, and from parties between whom there doth open hatred and malice appear, till the first sort have reformed their wicked life, the second recompensed them unto whom they were injurious, and the last condescended unto some course of Christian reconciliation, whereupon their mutual accord may ensue. In which cases, for the first branch of wicked life, and the last which is open enmity, there can arise no great difficulty about the exercise of his power: in the second, concerning wrongs, there may, if men shall presume to define or measure injuries according to their own conceits, depraved oftentimes as well by error as partiality, and that no less in the minister himself, than in any other of the people under him. The knowledge therefore which he taketh of wrongs must rise as it doth in the other two, not from his own opinion or conscience, but from the evidence of the fact which is committed; yea, from such evidence as neither doth admit denial nor defence. For if the offender having either colour of law to uphold, or any other pretence to excuse his own uncharitable and wrongful dealings, shall wilfully stand in defence thereof, it serveth as a bar to the power of the minister in this kind. 1Because (as it is observed by men  of very good judgment in these affairs) “although in this sort our separating of them be not to strike them with the mortal wound of excommunication, but to stay them rather from running desperately headlong into their own harm; yet in us it is not to sever from the holy communion but such as are either found culpable by their own confession, or have been convicted in some public secular, or ecclesiastical court. For who is he that dare take upon him to be any man’s both accuser and judge? Evil persons are not rashly, and as we list, to be thrust from communion with the Church; insomuch that, if we cannot proceed against them by any orderly course of judgment, they are rather to be suffered for the time than molested. Many there are reclaimed, as Peter; many, as Judas, known well enough, and yet tolerated; many, which must remain undescried till the day of His appearance, by whom the secret corners of darkness shall be brought into open light.”

Leaving therefore unto his judgment them whom we cannot stay from casting their own souls into so great hazard, we have in the other part of penitential jurisdiction, in our power and authority to release sin, joy on all sides, without trouble or molestation unto any. And if to give be a thing more blessed than to receive, are we not infinitely happier in being authorized to bestow the treasure of God, than when necessity doth constrain to withdraw the same?

They which, during life and health, are never destitute of ways to delude repentance, do notwithstanding oftentimes, when their last hour draweth on, both feel that sting which before lay dead in them, and also thirst after such helps as  have been always till then unsavoury. St. Ambrose’s words touching late repentance are somewhat hard, “If a man be penitent and receive absolution (which cannot in that case be denied him) even at the very point of death, and so depart, I dare not affirm he goeth out of the world well; I will counsel no man to trust to this, because I am loth to deceive any man, seeing I know not what to think of it. Shall I judge such a one a castaway? Neither will I avouch him safe. All I am able to say, is, Let his estate be left to the will and pleasure of Almighty God. Wilt thou be therefore clearly delivered of all doubt? Repent while yet thou art healthy and strong. If thou defer it till time give no longer possibility of sinning, thou canst not be thought to have left sin, but rather sin to have forsaken thee.” Such admonitions may in their time and place be necessary, but in no wise prejudicial to the generality of God’s own high and heavenly promise, “Whensoever a sinner doth repent from the bottom of his heart, I will put out all his iniquity.” And of this, although it hath pleased God not to leave to the world any multitude of examples, lest the careless should too far presume; yet one he hath given, and that most memorable, to withhold from despair in the mercies of God, at what instant soever man’s unfeigned conversion be wrought. Yea, because to countervail the fault of delay, there are in the latest repentance oftentimes the surest tokens of sincere dealing; therefore upon special confession made to the minister of God, he presently absolveth in this case the sick party from all his sins by that authority which Jesus Christ hath committed unto him, knowing that God respecteth  not so much what time is spent, as what truth is shewed in repentance.

[16] In sum, when the offence doth stand only between God and man’s conscience, the counsel is good which St. Chrysostom giveth: “I wish thee not to bewray thyself publicly, nor to accuse thyself before others. I wish thee to obey the Prophet, who saith, Disclose thy way unto the Lord, confess thy sin before him, tell thy sins to him that he may blot them out. If thou be abasht to tell unto any other wherein thou hast offended, rehearse them every day between thee and thy soul. I wish thee not to confess them to thy fellow-servant, who may upbraid thee with them; tell them to God, who will cure them; there is no need for thee in the presence of witnesses to acknowledge them; let God alone see thee at thy confession. I pray and beseech you, that you would more often than you do confess to God eternal, and reckoning your trespasses desire his pardon2. I carry you not into a theatre or open court of many your fellow-servants, I seek not to detect your crimes before men; disclose your conscience before God, unfold yourselves to him, lay forth your wounds before him, the best physician that is, and desire of him salve for them.” If hereupon it follow, as it did with David, “I thought, I will confess against myself my wickedness unto thee, O Lord, and thou forgavest me the plague of my sin,” we have then our desire, and there  remaineth only thankfulness, accompanied with perpetuity of care to avoid that, which being not avoided we know we cannot remedy without new perplexity and grief. Contrariwise, if peace with God do not follow the pains we have taken in seeking after it, if we continue disquieted, and not delivered from anguish, mistrusting whether that we do be sufficient; it argueth that our sore doth exceed the power of our own skill, and that the wisdom of the pastor must bind up those parts, which being bruised are not able to be recured of themselves.

\section*{Of Satisfaction.}

V. There resteth now Satisfaction only to be considered; a point which the Fathers do often touch, albeit they never aspire to such mysteries, as the papacy hath found enwrapped within the folds and plaits thereof. And it is happy for the Church of God, that we have the writings of the Fathers, to shew what their meaning was. The name of Satisfaction, as the ancient Fathers meant it, containeth whatsoever a penitent should do in the humbling himself unto God, and testifying by deeds of contribution the same which confession in words pretendeth. “He which by repentance for sins” (saith Tertullian, speaking of fickle-minded men) “had a purpose to satisfy the Lord, will now by repenting his repentance make Satan satisfaction; and be so much more hateful to God, as he is unto God’s enemy more acceptable.” Is it not plain, that satisfaction doth here include the whole work of penitency, and that God is satisfied when men are restored through sin into favour by repentance? “How canst thou,” saith Chrysostom, “move God to pity thee, when thou wilt not seem as much as to know that thou hast offended?” By appeasing, pacifying, and moving God to pity, St. Chrysostom meaneth the very same with the Latin Fathers, when they speak of satisfying God: “We feel,” saith St.f Cyprian,  “the bitter smart of his rod and scourge, because there is in us neither care to please him with our good deeds, nor to satisfy him for our evil.” Again, “Let the eyes which have looked on idols, sponge out their unlawful acts with those sorrowful tears, which have power to satisfy God.” The Master of Sentences allegeth out of St. Augustine that which is plain enough to this purpose: “Three things there are in perfect penitency, compunction, confession, and satisfaction; that as we three ways offend God, namely in heart, word, and deed, so by three duties we may satisfy God.”

Satisfaction, as a part, comprehendeth only that which the Baptist meant by works worthy of repentance; and if we speak of the whole work of repentance itself, we may in the phrase of antiquity term it very well satisfaction.

[2] Satisfaction is a work which justice requireth to be done for contentment of persons injured: neither is it in the eye of justice a sufficient satisfaction, unless it fully equal the injury for which we satisfy. Seeing then that sin against God eternal and infinite must needs be an infinite wrong; justice in regard thereof doth necessarily exact an infinite recompense, or else inflict upon the offender infinite punishment. Now because God was thus to be satisfied, and man not able to make satisfaction in such sort, his unspeakable love and inclination to save mankind from eternal death ordained in our behalf a Mediator, to do that which had been for any other impossible. Wherefore all sin is remitted in  the only faith of Christ’s passion, and no man without belief thereof justified1. Faith alone maketh Christ’s satisfaction ours; howbeit that faith alone which after sin maketh us by conversion his. kFor inasmuch as God will have the benefit of Christ’s satisfaction both thankfully acknowledged and duly esteemed of all such as enjoy the same, he therefore imparteth so high a treasure unto no man, whose faith hath not made him willing by repentance to do even that, which of itself how unavailable soever, yet being required and accepted with God, we are in Christ made thereby capable and fit vessels to receive the fruitm of his satisfaction: yea, we so far please and content God, that because when we have offended he looketh but for repentance at our hands, our repentance and the works thereof are therefore termed satisfactory, not for that so much is thereby done as the justice of God can exact, but because such actions of grief and humility in man after sin are illices divinæ misericordiæ (as Tertullian speaketh of them), they draw that pity of God towards us, wherein he is for Christ’s sake contented upon our submission to pardon our rebellion against him; and when that little which his law appointeth is faithfully executed, it pleaseth him in tender compassion and mercy to require no more.

[3] Repentance is a name which noteth the habit and operation of a certain grace or virtue in us: Satisfaction, the effect which it hath, either with God or man. And it is not in this respect said amiss, that satisfaction importeth acceptation, reconciliation, and amity; because that through satisfaction, on the one part made, and allowed on the other, they which before did reject are now content to receive, they to be won again which were lost, and they to love unto whom just cause of hatred was given. We satisfy therefore in doing  that which is sufficient to this effect; and they towards whom we do it are satisfied, if they accept it as sufficient, and require no more: otherwise we satisfy not, although we do satisfy: for so between man and man it oftentimes falleth out, but between man and God, never. It is therefore true, that our Lord Jesus Christ by one most precious and propitiatory sacrifice, which was his body, a gift of infinite worth, offered for the sins of the whole world, hath thereby once reconciled us to God, purchased his general free pardon, and turned away divine indignation from mankind. But we are not for that cause to think any office of penitence either needless or fruitless on our own behalf: for then would not God require any such duties at our hands. Christ doth remain everlastingly a gracious intercessor, even for every particular penitent. Let this assure us, that God, how highly soever displeased and incensed with our sins, is notwithstanding for his sake by our tears pacified, taking that for satisfaction which is due [done?] by us, because Christ hath by his satisfaction made it acceptable. For, as he is the High-priest of our salvation, so he hath made us priests likewise under him, to the end we might offer unto God praise and thankfulness, while we continue in the way of life, and when we sin, the satisfactory or propitiatory sacrifice of a broken and contrite heart2. There is not any thing that we do that could pacify God, and clear us in his sight from sin, if the goodness and mercy of our Lord Jesus Christ were not; whereas now beholding the poor offer of our religious endeavour meekly to submit ourselves as often as we have offended, he regardeth with infinite mercy those services which are as nothing, and with words of comfort reviveth our afflicted minds, saying, “It is I, even I, that take away thine iniquities for mine own sake.” Thus doth repentance satisfy God, changing his wrath and indignation unto mercy.


[4] Anger and mercy are in us passions; but in him not so. “God,” saith St. Basil, “is no ways passionate, but because the punishments which his judgments do inflict are, like effects of indignation, severe and grievous to such as suffer them, therefore we term the revenge which he taketh upon sinners, anger; and the withdrawing of his plagues, mercy.” “His wrath,” saith St. Augustine, is not as ours, the trouble of a mind disturbed and disquieted with things amiss, but a calm, unpassionate, and just assignation of dreadful punishment to be their portion which have disobeyed; his mercy a free determination of all felicity and happiness unto men, except their sins remain as a bar between it and them.” So that when God doth cease to be angry with sinful men, when he receiveth them into favour, when he pardoneth their offences, and remembereth their iniquities no more (for all these signify but one thing), it must needs follow, that all punishments before due in revenge of sin, whether they be temporal or eternal, are remitted. For how should God’s indignation import only man’s punishment, and yet some punishment remain unto them, towards whom there is now in God no indignation remaining? “God,” saith Tertullian, “taketh penitency at men’s hands, and men at his in lieu thereof receive impunity;” which notwithstanding doth not prejudice the chastisements that God after pardon hath laid upon some offenders, as on the people of Israel, on Moses, on Miriam, on David, either for their own more sound amendment, or  for example unto others in this present world (for in the world to come punishments have unto these intents no use, the dead being not in case to be bettered by correction, nor to take warning by executions of God’s justice there seen); but assuredly to whomsoever he remitteth sin, their very pardon is in itself a full absolute and perfect discharge for revengeful punishments; which God doth nowheres threaten, but with purpose of revocation if men repent, nowhere inflict but on them whom impenitency maketh obdurate.

Of the one therefore it is said, “Though I tell the wicked, Thou shalt die the death, yet if he turn from his sin, and do that which is lawful and right, he shall surely live and not die.” Of the other, “Thou according to thine hardness, and heart that will not repent, treasurest up to thyself wrath against the day of wrath, and evident appearance of the just judgment of God.” If God be satisfied and do pardon sin, our justification restored is as perfect as it was at the first bestowed. For so the Prophet Isaiah witnesseth, “Though your sins were as crimson, they shall be made as white as snow; though they were all scarlet, they shall be as white as wool.” And can we doubt concerning the punishment of revenge, which was due to sin, but that if God be satisfied and have forgotten his wrath, it must be even as St. Augustin reasoneth, “What God hath covered he will not observe, and what he observeth not he will not punish.” The truth of which doctrine is not to be shifted off by restraining it unto eternal punishment alone. For then would not David have said, “They are blessed to whom God imputeth no sin;” blessedness having no part or fellowship at all with malediction: whereas to be subject to revenge for sin, although the punishment be but temporal, is to be under the curse of the law: wherefore, as one and the same fire consumeth  stubble and refineth gold, so if it please God to lay punishment on them whose sins he hath forgiven, yet is not this done for any destructive end of wasting and eating them out, as in plagues inflicted upon the impenitent, neither is the punishment of the one as of the other proportioned by the greatness of sin past, but according to that future purpose whereunto the goodness of God referreth it, and wherein there is nothing meant to the sufferer but furtherance of all happiness, now in grace, and hereafter in glory. St. Augustine, to stop the mouths of Pelagians arguing, “That if God had imposed death upon Adam and Adam’s posterity, as a punishment of sin, death should have ceased when Christy had procured sinners their pardon;” answereth first, “It is no marvel, either that bodily death should not have happened to the first man, unless he had first sinned (death as a punishment following his sin), or that after sin is forgiven, death notwithstanding befalleth the faithful; to the end that the strength of righteousness might be exercised by overcoming the fear thereof1. So that justly God did inflict bodily death on man for committing sin, and yet after sin forgiven took it not away, that his righteousness might still have whereby to be exercised.” He fortifieth this with David’s example, whose sin he forgave, and yet afflicted him for exercise and trial of his humility. Briefly, a general axiom he hath for all such chastisements, “Before forgiveness, they are the punishment of sinners; and after forgiveness, they are exercises and trials of righteous men.” Which kind of proceeding is so agreeable with God’s nature and man’s comfort, that it sheweth even injurious to both, if we should admit those surmised reservations of temporal wrath in God appeased towards  reconciled sinners. “As a Father he delights in his children’s conversion, neither doth he threaten the penitent with wrath, or them with punishment which already mourn; but by promise assureth such of indulgence and mercy;” yea, even of plenary pardon, which taketh away all both faults and penalties: there being no reason why we should think him the less just because he sheweth him thus merciful; when they which before were obstinate labour to appease his wrath with the pensive meditations of contrition, the meek humility which confession expresseth, and the deeds wherewith repentance declareth itself to be an amendment as well of the rotten fruits, as the dried leaves and withered root of the tree. For with these duties by us performed, and presented unto God in heaven by Jesus Christ, whose blood is a continual sacrifice of propitiation for us, we content, please, and satisfy God.

[5]Repentance therefore, even the sole virtue of repentance, without either purpose of shrift, or desire of absolution from the priest; repentance, the secret conversion of the heart, in that it consisteth of these three, and doth by these three pacify God, may be without hyperbolical terms most truly magnified, as a recovery of the soul of man from deadly sickness, a restitution of glorious light to his darkened mind, a comfortable reconciliation with God, a spiritual nativity, a rising from the dead, a day-spring from out the depth of obscurity, a redemption from more than the Egyptian thraldom, a grinding of the old Adam even into dust and powder, a deliverance out of the prisons of hell, a full restoration of the seat of grace and throne of glory, a triumph over sin, and a saving victory.

[6] Amongst the works of satisfaction, the most respected have been always these three, Prayers, Fasts, and Almsdeeds: by prayer, we lift up our souls to him from whom sin and iniquity hath withdrawn them; by fasting, we reduce the body from thraldom under vain delights, and make it serviceable for parts of virtuous conversation; by alms,  we dedicate to charity these worldly goods and possessions, which unrighteousness doth neither get nor bestow well: the first, a token of piety intended towards God; the second, a pledge of moderation and sobriety in the carriage of our own persons; the last, a testimony of our meaning to do good to all men. In which three, the Apostle by way of abridgment comprehendeth whatsoever may appertain to sanctimony, holiness, and good life: as contrariwise the very mass of general corruption throughout the world, what is it but only forgetfulness of God, carnal pleasure, immoderate desire after worldly things; profaneness, licentiousness, covetousness?

All offices of repentance have these two properties; there is in performance of them painfulness, and in their nature a contrariety unto sin. The one consideration causeth them both in holy Scripture and elsewhere to be termed judgments or revenges taken voluntarily on ourselves, and to be furthermore also preservatives from future evils, inasmuch as we commonly use to keep with the greater care that which with pain we have recovered2. And they are in the other respect contrary to sin committed; contrition, contrary to the pleasure; confession, to the error, which is mother of sin; and to the deeds of sin, the works of satisfaction contrary; therefore they all the more effectual to cure the evil habit thereof. Hereunto it was that St. Cyprian referred his earnest and vehement exhortations, “That they which had fallen should be instant in prayer, reject bodily ornaments when once they have stripped themselves out of Christ’s attire, abhor all food after Satan’s morsels tasted, follow works of righteousness which wash away sin, and be plentiful in alms-deeds wherewith souls are delivered from death.” Not, as if God did, according to the manner of corrupt  judges, take so much money to abate so much in the punishment of malefactors. “These duties must be offered,” saith Salvianus, “not in confidence to redeem or buy out sin, but as tokens of meek submission; neither are they with God accepted, because of their value, but for the kaffection’s sake, which doth thereby shew itself.”

Wherefore concerning Satisfaction made to God by Christ only, and of the manner how repentance generally, particularly also, how certain special works of penitency, both are by the Fathers in their ordinary phrase of speech called satisfactory, and may be by us very well so acknowledged; enough hath been spoken.

[7] Our offences sometimes are of such nature, as requireth that particular men be satisfied, or else repentance to be utterly void, and of none effect. For, if either through open rapine or cloaked fraud, if through injurious or unconscionable dealings, a man have wittingly wronged others to enrich himself; the first thing evermore in this case required (ability serving) is restitution. For let no man deceive himself: from such offences we are not discharged, neither can be, till recompense and restitution to man accompany the penitent confession we have made to Almighty God. In which case the law of Moses was direct and plain2. “If any sin and commit a trespass against the Lord, and deny unto his neighbour that which was given him to keep, or that which was put unto him of trust; or doth by robbery or by violence oppress his neighbour; or hath found that which was lost, and denieth it, and sweareth falsely: for any of these things that a man doth wherein he sinneth, he that doth thus offend and trespass, shall restore the robbery that he hath taken, or the thing he hath gotten by violence, or that which was delivered him to keep, or the lost thing which he found; and for whatsoever he hath sworn falsely, adding perjury to injury, he shall both restore the whole sum, and shall add thereunto a fifth part more, and deliver  it unto him, to whom it belongeth, the same day wherein he offereth for his trespass.” Now because men are commonly overslack to perform this duty, and do therefore defer it sometimes, till God hath taken the party wronged out of the world; the law providing that trespassers might not under any such pretence gain the restitution which they ought to make, appointeth the kindred surviving to receive what the dead should, if they had continued. “But,” saith Moses, “if the party wronged have no kinsman to whom this damage may be restored, it shall then be rendered to the Lord himself for the priests’ use.” The whole order of proceeding herein is in sundry traditional writings set down by their great interpreters and scribes, which taught them that a trespass between a man and his neighbour can never be forgiven, till the offender have by restitution made recompense for wrongs done; yea, they hold it necessary that he appease the party grieved by submitting himself unto him, or, if that will not serve, by using the help and mediation of others: “In this case (say they) for any man to shew himself unappeasable and cruel, were a sin most grievous, considering that the people of God should be easy to relent, as Joseph was towards his brethren.” Finally, if so it fall out, that the death of him which was injured prevent his submission which did offend, let him then (for so they determine that he ought) go accompanied with ten others unto the sepulchre of the dead, and there make confession of the fault, saying, “I have sinned against the Lord God of Israel, and against this man, to whom I have done such or such injury; and if money be due, let it be restored to his heirs, or in case he have none known, leave it with the house of judgment:” that is to say, with the senators, ancients, and guidersy of Israel. We hold not Christian people tied unto Jewish orders for the manner of restitution; but surely restitution we must hold necessary, as well in our own repentance as theirs, for sins of wilful oppression and wrong.


[8] Now although it suffice, that the offices wherewith we pacify God or private men be secretly done; yet in cases where the Church must be also satisfied, it was not to this end and purpose unnecessary, that the ancient discipline did further require outward signs of contrition to be shewed, confession of sins to be made openly, and those works to be apparent, which served as testimonies of conversion before men. Wherein, if either hypocrisy did at any time delude their judgment, they knew that God is he whom masks and mockeries cannot blind, that he which seeth men’s hearts would judge them according unto his own evidence, and, as Lord, correct the sentence of his servants concerning matters beyond their reach: or if such as ought to have kept the rules of canonical satisfaction would by sinister means and practices undermine the same, obtruding presumptuously themselves to the participation of Christ’s most sacred mysteries before they were orderly readmitted thereunto, the Church for contempt of holy things held them uncapable of that grace, which God in the Sacrament doth impart to devout communicants; and no doubt but he himself did retain bound, whom the Church in those cases refused to loose.

The Fathers, as may appear by sundry decrees and canons of the primitive Church, were (in matter specially of public scandal) provident that too much facility of pardoning might not be shewed. “He that casteth off his lawful wife,” saith St. Basil, “and doth take another, is adjudged an adulterer by the verdict of our Lord himself; and by our fathers it is canonically ordained, that such for the space of a year shall mourn, for two years’ space hear, three years be prostrate,  the seventh year assemble with the faithful in prayer, and after that be admitted to communicate, if with tears they bewail their fault.”

Of them which had fallen from their faith in the time of the Emperor Licinius, and were not thereunto forced by any extreme usage, the Nicene synod under Constantine ordained, “That earnestly repenting, they should continue three years hearers, seven years be prostrate, and two years communicate with the people in prayer, before they came to receive the oblation.” Which rigour sometimes they tempered nevertheless with lenity, the selfsame synod having likewise defined, “That whatsoever the cause were, any man desirous at the time of departure out of this life to receive the Eucharist might (with examination and trial) have it granted him by the bishop.” Yea, besides this case of special commiseration, there is a canon more large, which giveth always liberty to abridge or extend out the time, as the party’s meek or sturdy disposition should require.

By means of which discipline, the Church having power to hold them many years in suspense, there was bred in the minds of the penitents, through long and daily practice of such submission, a contrary habit unto that which before had been their ruin, and for ever afterwards wariness not to fall into those snares out of which they knew they could not easily wind themselves. Notwithstanding, because there was likewise hope and possibility of shortening the time, this made them in all the parts and offices of their repentance the more fervent. In the first station, while they only beheld others, passing towards the temple of God, whereunto for themselves  to approach it was not lawful; they stood as miserable forlorn men, the very patterns of perplexity and woe. In the second, when they had the favour to wait at the doors of God, where the sound of his comfortable word might be heard; none received it with attention like to theirs. Being taken and admitted to the next degree of prostrates, at the feet yet behind the back of that angel representing God, whom the rest saw face to face; their tears, and entreaties both of Pastor and people, were such as no man could resist. After the fourth step, which gave them liberty to hear and pray with the rest of the people; being so near the haven, no diligence was then slacked which might hasten admission to the heavenly table of Christ, their last desire. It is not therefore a thing to be marvelled at, though St. Cyprian took it in very evil part, when open backsliders from the faith and sacred religion of Christ laboured by sinister practice to procure from imprisoned saints those requests for present absolution, which the Church could neither yield unto with safety of discipline, nor in honour of martyrdom easily deny. For, what would thereby ensue they needed not to conjecture, when they saw how every man which came so commended to the Church by letters thought that now he needed not to crave, but might challenge of duty, his peace; taking the matter very highly, if but any little forbearance or small delay were used. “He which is overthrown,” saith St.g Cyprian, “menaceth them that stand, the wounded them that were never toucht; and because presently he hath not the body of our Lord in his foul imbrued hands, nor the blood within his polluted lips, the miscreant fumeth at God’s priests: such is thy madness, O thou furious man; thou art angry with him which laboureth to turn away God’s anger from thee: him thou threatenest, which sueth unto God for grace and mercy on thy behalf.”


Touching Martyrs he answereth, “That it ought not in this case to seem offensive, though they were denied, seeing God himself did refuse to yield to the piety of his own righteous saints, making suit for obdurate Jews.”

As for the parties, in whose behalf such shifts were used; to have their desire was, in very truth, a way to make them the more guilty: such peace granted contrary to the vigour of the Gospel, contrary to the law of our Lord and God, doth but under colour of merciful relaxation deceive sinners, and by soft handling destroy them; a grace dangerous for the giver, and to him which receiveth it nothing at all available. “The patient expectation that bringeth health is by this means not regarded; recovery of soundness not sought for  by the only medicine available, which is satisfaction; penitency thrown out of men’s hearts; the remembrance of that heaviest and last judgment clean banisht; the wounds of dying men, which should be healed, are covered; the stroke of death, which hath gone as deep as any bowels are to receive it, is overcast with the slight show of a cloudy look. From the altars of Satan to the holy of the Lord men are not afraid to come even belching in a manner the sacrificed morsels they have eaten; yea, their jaws yet breathing out the irksome savour of their former contagious wickedness, they seize upon the blessed body of our Lord, nothing terrified with that dreadful commination, which saith, ‘Whosoever eateth and drinketh unworthily, is guilty of the body and blood of Christ.’ They vainly think it to be peace, which is gotten before they be purged of their faults, before their crime be solemnly confest, before their conscience be cleared by the sacrifice, and imposition of the priests’ hands, and before they have pacified the indignation of God. Why term they that a favour, which is an injury? Wherefore cloak they impiety with the name of charitable indulgence? Such facility giveth not, but rather taketh away peace; and is itself another fresh persecution or trial, whereby that fraudulent enemy maketh a secret havoc of such as before he had overthrown; and now to the end he may clean swallow them, he casteth sorrow in a dead sleep, putteth grief to silence, wipeth out the memory of faults newly done, smothereth the sighs that should arise from a contrite spirit, drieth up eyes which ought to send forth rivers of tears, and permitteth not God to be pacified with full repentance, whom heinous and enormous crimes have displeased.”

By this then we see, that in St. Cyprian’s judgment, all absolutions are void, frustrate, and of no effect, without sufficient repentance first shewed; whereas contrariwise, if true and full satisfaction have gone before, the sentence of man here given is ratified of God in heaven, according to our Saviour’s own sacred testimony, “Whose sins ye remit, they are remitted.”


[9] By what works in the Virtue, and by what in the Discipline of Repentance, we are said to satisfy either God or men, cannot now be thought obscure.
As for the inventors of sacramental satisfaction, they have both altered the natural order heretofore kept in the Church, by bringing in a strange preposterous course, to absolve before satisfaction be made, and moreover by this their misordered practice are grown into sundry errors concerning the end whereunto it is referred.
%The end of satisfaction. 

They imagine, beyond all conceit of antiquity, that when God doth remit sin and the punishment eternal thereunto belonging, he reserveth the torments of hell-fire, to be nevertheless endured for a time, either shorter or longer, according to the quality of men’s crimes. Yet so that there is between God and man a certain composition (as it were) or contract, by virtue whereof works assigned by the priest to be done after absolution shall satisfy God, as touching the punishment which he otherwise would inflict for sin pardoned and forgiven.

%The way of satisfying by others.
Now because they cannot assure any man, that if he perform what the priest appointeth it shall suffice; this (I say) because they cannot do, inasmuch as the priest hath no power to determine or define of equivalency between sins and satisfactions; (and yet if a penitent depart this life, the debt of satisfaction being either in whole or in part undischarged, they steadfastly hold that the soul must remain in unspeakable torment till all be paid:) therefore for help and mitigation in this case, they advise men to set certain copesmates on work, whose prayers and sacrifices may satisfy God for such souls  as depart in debt. Hence have arisen the infinite pensions of their priests, the building of so many altars and tombs, the enriching of Churches with so many glorious and costly gifts, the bequeathing of lands and ample possessions to religious companies, even with utter forgetfulness of friends, parents, wife, childreno all natural affection giving place unto that desire, which men doubtful of their own estate have to deliver their souls from torment after death.

%The ground of satisfying by the Pope’s indulgencesr.
Yet behold, even this being also done, how far forth it shall avail they are not sure; and therefore the last upshot unto all their former inventions is, that as every action of Christ did both merit for himself, and satisfy partly for the eternal, and partly for the temporal punishment due unto men for sin; so his saints have obtained the like privilege of grace, making every good work they do, not only meritorious in their own behalf, but satisfactory too for the benefit of others. Or if, having at any time grievously sinned, they do more to satisfy God than he in justice can exact or look for at their hands; the surplusage runneth to a common stock, out of which treasury, containing whatsoever Christ did by way of satisfaction for temporal punishment, together with the satisfactory force which resideth in all the virtuous works of saints, and in their satisfactions whatsoever doth abound, (I say,) “From hence they hold God satisfied for such arrearages as men behind in accompt discharge not by other means; and for disposition hereof, as it is their doctrine that Christ remitteth not eternal death without the priest’s absolution, so without the grant of the Pope they cannot but teach it alike unpossibles that souls in hell should receive any temporal release of pain; the sacrament of pardon from him being to this effect no less necessary, than the priest’s absolution to the other.” So that by this postern-gate cometh in the whole mart of papal indulgences; a gain inestimable unto him, to others a spoil; a scorn both to God and man. So many works of satisfaction pretended to be done by Christ, by saints, and martyrs; so many virtuous acts possessed with satisfactory force and virtue; so many  supererogations in satisfying beyond the exigence of their own necessity; and this that the Pope might make a monopoly of all, turning all to his own gain, or at the least to the gain of them which are his own. Such facility they have to convert a pretended sacrament into a true revenue.

\section*{Of Absolution of Penitents.}

VI. Sin is not helped but by being assecured of pardon. It resteth therefore to be considered what warrant we have concerning forgiveness, when the sentence of man absolveth us from sin committed against God. At the words of our Saviour, saying to the sick of the palsy, “Son, thy sins are forgiven thee,” exception was taken by the Scribes, who secretly reasoned against him, “Is any able to forgive sins, but only God?” Whereupon they condemned his speech as blasphemy; the rest, which believed him to be a Prophet sent from God, saw no cause wherefore he might not as lawfully say, and as truly, to whomsoever amongst them, “God hath taken away thy sins,” as Nathan (they all knew) had used the very like speech; to whom David did not therefore impute blasphemy, but embraced, as became him, the words of truth with joy and reverence.

Now there is no controversy but as God in that special case did authorize Nathan, so Christ more generally his Apostles and the ministers of his word in his name to absolve sinners. Their power being equal, all the difference between them can be but only in this, that whereas the one had prophetical evidence, the other have the certainty partly of faith, and partly of human experience, whereupon to ground their sentence: faith, to assure them of God’s most gracious pardon in Heaven unto all penitents; and touching the sincerity of each particular party’s repentance, as much as outward sensible tokens or signs can warrant.

[2] It is not to be marvelled that so great a difference appeareth between the doctrine of Rome and ours, when we teach repentance. They imply in the name of repentance much more than we do. We stand chiefly upon the true inward conversion of the heart; they more upon works of external show. We teach, above all things, that repentance which is one and the same from the beginning to the world’s  end; they a sacramental penance of their own devising and shaping. We labour to instruct men in such sort, that every soul which is wounded with sin may learn the way how to cure itself; they, clean contrary, would make all sores seem incurable, unless the priest have a hand in them.

Touching the force of whose absolution they strangely hold, that whatsoever the penitent doth, his contrition, confession, and satisfaction have no place of right to stand as material parts in this sacrament, nor consequently any such force as to make them available for the taking away of sin, in that they proceed from the penitent himself without the privity of the minister, but only, as they are enjoined by the minister’s authority and power. So that no contrition or grief of heart, till the priest exact it; no acknowledgment of sins, but that which he doth demand; no praying, no fasting, no alms, no recompense or restitution for whatsoever we have done, can help, except by him it be first imposed. It is the chain of their own doctrine, no remedy for mortal sin committed after baptism but the sacrament of penance only; no sacrament of penance, if either matter or form be wanting; no ways to make those duties a material part of the sacrament, unless we consider them as required and exacted by the priest. Our Lord and Saviour, they say, hath ordained his priests judges in such sort, that no man which sinneth after baptism can be reconciled unto God but by their sentence2. For why? If there were any other way of reconciliation, the very promise of Christ should be false, in saying, “Whatsoever ye bind on earth, shall be bound in heaven; and whose sins soever ye retain, are retained4.” Except therefore the priest be willing, God hath by promise so hampered himself, that it is not now in his own power to pardon any man. Let him which hath offended crave as the publican did; “Lord, be thou  merciful to me a sinner;” let him, as David, make a thousand times his supplication, “Have mercy upon me, O God, according to thy loving-kindness; according to the multitude of thy compassionsd put away mine iniquities:” all this doth not help, till such time as the pleasure of the priest be known; till he have signed us a pardon, and given us our quietus est, God himself hath no answer to make but such as that of his angel unto Lot, “I can do nothing.”

[3] It is true, that our Saviour by those words, “Whose sins ye remit, they are remitted,” did ordain judges over sinful souls, give them authority to absolve from sin, and promise to ratify in heaven whatsoever they should do on earth in execution of this their office; to the end that hereby, as well his ministers might take encouragement to do their duty with all faithfulness, as also his people admonition, gladly with all reverence to be ordered by them; both parts knowing that the functions of the one towards the other have his perpetual assistance and approbation. Howbeit all this with two restraints, which every jurisdiction in the world hath; the one, that the practice thereof proceed in due order; the other, that it do not extend itself beyond due bounds; which bounds or limits have so confined penitential jurisdiction, that although there be given unto it power of remitting sin, yet not such sovereignty of power, that no sin should be pardonable in man without it. Thus to enforce our Saviour’s words, is as though we should gather, that because whatsoever Joseph did command in the land of Egypt, Pharaoh’s grant was, it should be done; therefore he granted that nothing should be done in the land of Egypt but what Joseph did command, and so consequently, by enabling his servant Joseph to command under him, disableth himself to command any thing without Joseph.

But by this we see how the papacy maketh all sin unpardonable, which hath not the priest’s absolution; except peradventure in some extraordinary case, where albeit absolution be not had, yet it must be desired.


[4] What is then the force of absolution? What is it which the act of absolution worketh in a sinful man? Doth it by any operation derived from itself alter the state of the soul? Doth it really take away sin, or but ascertain us of God’s most gracious and merciful pardon? The latter of which two is our assertion, the former theirs.

At the words of our Lord and Saviour Jesus Christ, saying unto the sick of the palsy, “Son, thy sins are forgiven thee,” the Pharisees, which knew him not to be the “Son of the living God,” took secret exception, and fell to reasoning with themselves against him; “Is any able to forgive sins but God only?” “The sins,” saith St. Cyprian, “that are committed against him, he alone hath power to forgive, which took upon him our sins, he which sorrowed and suffered for us, he whom the Father delivered unto death for our offences.” Whereunto may be added that which Clemens Alexandrinus hath, “Our Lord is profitable every way, every way beneficial, whether we respect him as mani, or as God; as God forgiving, as man instructing and learning how to avoid sin.” For it is “I, even I, that putteth away thine iniquities for mine own sake, and will not remember thy sins,” saith the Lord.

Now albeit we willingly confess with St. Cyprian, “The sins that are committed against him, he only hath power to forgive, who hath taken upon him our sins, he which hath sorrowed and suffered for us, he whom God hath given for our offences:” yet neither did St. Cyprian intend to deny  the power of the minister, otherwise than if he presume beyond his commission to remit sin, where God’s own will is it should be retained; for against such absolutions he speaketh (which being granted to whom they ought to have been denied, are of no validity;) and, if rightly it be considered how higher causes in operation use to concur with inferior means, his grace with our ministry, God really performing the same which man is authorized to act as in his name, there shall need for decision of this point no great labour.

[5] To remission of sins there are two things necessary; grace, as the only cause which taketh away iniquity; and repentance, as a duty or condition required in us. To make repentance such as it should be, what doth God demand but inward sincerity joined with fit and convenient offices for that purpose? the one referred wholly to our own consciences, the other best discerned by them whom God hath appointed judges in this court. So that having first the promises of God for pardon generally unto all offenders penitent; and particularly for our own unfeigned meaning, the unfallible testimony of a good conscience; the sentence of God’s appointed officer and vicegerent to approve with unpartial judgment the quality of that we have done, and as from his tribunal, in that respect to assoil us of any crime: I see no cause but that by the rules of our faith and religion we may rest ourselves very well assured touching God’s most merciful pardon and grace; who, especially for the strengthening of weak, timorous, and fearful minds, hath so far endued his church with power to absolve sinners. It pleaseth God that men sometimes should, by missing this help, perceive how much they stand bound to him for so precious a benefit enjoyed. And surely, so long as the world lived in any awe or fear of falling away from God, so dear were his ministers to the people, chiefly in this respect, that being through tyranny and persecution deprived of pastors, the doleful rehearsal of  their lost felicities hath not any one thing more eminent, than that sinners distrest should not now know how or where to unlade their burthen. Strange it were unto me, that the Fathers, who so much every where extol the grace of Jesus Christ in leaving unto his Church this heavenly and divine power, should as men whose simplicity had generally been abused, agree all to admire and magnify a needless office.

The sentence therefore of ministerial absolution hath two effects: touching sin, it only declareth us free from the guiltiness thereof, and restored into God’s favour; but concerning right in sacred and divine mysteries, whereof through sin we were made unworthy, as the power of the Church did before effectually bind and retain us from access unto them, so upon our apparent repentance it truly restoreth our liberty, looseth the chains wherewith we were tied, remitteth all whatsoever is past, accepteth us no less, returned, than if we never had gone astray.

For inasmuch as the power which our Saviour gave to his Church is of two kinds, the one to be exercised over voluntary penitents only, the other over such as are to be brought to amendment by ecclesiastical censure; the words wherein he hath given this authority must be so understood, as the subject or matter whereupon it worketh will permit. It doth not permit that in the former kind, (that is to say, in the use of power over voluntary converts,) to bind or loose, remit or retain, should signify any other than only to pronounce of sinners according to that which may be gathered by outward signs; because really to effect the removal or continuance of sin in the soul of any offender, is no priestly act, but a work which far exceedeth their ability. Contrariwise, in the latter  kind of spiritual jurisdiction, which by censures constraineth men to amend their lives; it is true, that the minister of God doth more than declare and signify what God hath wrought. And this power, true it is, that the Church of Christ hath invested in it.

[6] Howbeit, as other truths, so this hath both by error been oppugned, and depraved through abuse. The first of name, that openly in writing withstood the Church’s authority and power to remit sin, was Tertullian, after he had combined himself with Montanists drawn to the liking of their heresy through the very sourness of his own nature, which neither his incredible skill and knowledge otherwise, nor the very doctrine of the gospel itself, could but so much alter, as to make him savour any thing which carried with it the taste of lenity. A sponge steeped in wormwood and gall, a man through too much severity merciless, and neither able to endure nor to be endured of any. His book entitled Concerning Chastity, and written professedly against the discipline of the Church, hath many fretful and angry sentences, declaring a mind very much offended with such as would not persuade themselves, that of sins, some be pardonable by the keys of the Church, some uncapable of forgiveness; that middle and moderate offences having received chastisement, may by spiritual authority afterwards be remitted, but greater transgressions must (as touching indulgence) be left to the only pleasure of Almighty God in the world to come; that as idolatry and bloodshed, so likewise fornication and sinful lust  are of this nature; that they which so far are fallen from God, ought to continue for ever after barred from access unto his sanctuary, condemned to perpetual profusion of tears, deprived of all expectation and hope to receive any thing at the Church’s hands, but publication of their shame2. “For,” saith he, “who will fear to waste out that which he hopeth he may recover? Who will be careful for ever to hold that, which he knoweth cannot for ever be withheld from him? He which slackeneth the bridle to sin, doth thereby give it even the spur also3. Take away fear, and that which presently succeedeth instead thereof is licentious desire. Greater offences therefore are punishable, but not pardonable, by the Church. If any Prophet or Apostle be found to have remitted such transgressions, they did it not by the ordinary course of discipline, but by extraordinary power. For they also raised the dead, which none but God is able to do; they restored impotentx and lame men, a work peculiar to Jesus Christ; yea, that which Christ would not do, because executions of such severity beseemed not him who came to save and redeem the world by his sufferings, they by their power struck Elymas and Ananias, the one blind, and the other dead. Approve first yourselves to be as they were  Apostles or Prophets, and then take upon you to pardon all men. But if the authority you have be only ministerial, and no way sovereign, over-reach not the limits which God hath set you; know that to pardon capital sin is beyond your commission.”

Howbeit, as oftentimes the vices of wicked men do cause other their commendable qualities to be abhorred, so the honour of great men’s virtues is easily a cloak to their errors. In which respect Tertullian hath past with much less obloquy and reprehension than Novatian; who, broaching afterwards the same opinion, had not otherwise wherewith to countervail the offence he gave, and to procure it the like toleration. Novatian, at the first a stoical philosopher, (which kind of men hath always accounted stupidity the highest top of wisdom, and commiseration the deadliest sin,) became by institution and study the very same which the other had been before through a secret natural distemper, upon his conversion to the Christian faith and recovery from sickness, which moved him to receive the sacrament of Baptism in his bed. The bishop contrary to the canons of the Church would needs in special love towards him ordain him presbyter, which favour satisfied not him who thought himself worthy of greater place and dignity. He closed therefore with a number of well-minded men, and not suspicious what his secret purposes were, and having made them sure unto him by fraud, procureth his own consecration to be their bishop. His prelacy now was able as he thought to countenance what he intended to publish, and therefore his letters went presently abroad to sundry churches, advising them never to admit to the fellowship of holy mysteries such as had after baptism offered sacrifice to idols.

There was present at the council of Nice, together with other bishops, one Acesius a Novatianist, touching whose diversity in opinion from the Church the emperor desirous to hear some reason, asked of him certain questions; for answer whereunto Acesius weaveth out a long history of things that  happened in the persecution under Decius, and of men, which to save life forsook faith. But the end was a certain bitter canon framed in their own school, “That men which fall into deadly sin after holy baptism, ought never to be again admitted to the communion of divine mysteries; that they are to be exhorted unto repentance, howbeit not to be put in hope that pardon can be had at the priest’s hands; but with God, which hath sovereign power and authority in himself to remit sins, it may be in the end they shall find mercy.”

Those followers of Novatian, which gave themselves the title of καθαροὶ, clean, pure, and unspotted men, had one point of Montanism more than their master did profess; for amongst sins unpardonable they reckoned second marriages, of which opinion Tertullian making (as his usual manner was) a salt apology, “Such is,” saith he, “our stony hardness, that defaming our Comforter with a kind of enormity in discipline, we dam up the doors of the church no less against twice-married men than against adulterers and fornicators.” Of this sort therefore it was ordained by the Nicene Synod, that if any such did return to the catholic and apostolic unity, they should in writing bind themselves to observe the orders of the Church, and communicate as well with them which had been often married, or had fallen in time of persecution, as with other sorts of Christian people. But further to relate, or at all to refel the errors of misbelieving men concerning this point, is not now to our present purpose greatly necessary.

[7] The Church may receive no small detriment by corrupt practice, even there where doctrine concerning the substance of things practised is free from any great or dangerous  corruption. If therefore that which the papacy doth in matter of confessions and absolutions be offensive; if it palpably swerve in the use of the keys; howsoever that which it teacheth in general concerning the Church’s power to retain and forgive sins be admitted true, have they not on the one side as much whereat to be abasht, as on the other wherein to rejoice?

They bind all men, upon pain of everlasting condemnation and death, to make confession to their ghostly fathers of every great offence they know, and can remember that they have committed against God. Hath Christ in his Gospel so delivered the doctrine of repentance unto the world? Did his Apostles so preach it to nations? Have the Fathers so believed or so taught? Surely Novatian was not so merciless in depriving the Church of power to absolve some certain offenders, as they in imposing upon all a necessity thus to confess. Novatian would never deny but God might remit that which the Church could not; whereas in the papacy it is maintained, that what we conceal from men, God himself shall never pardon. By which oversight, as they have surcharged the world with multitude, but much abated the weight of confession, so the careless manner of their absolution hath made discipline for the most part amongst them a bare formality; yea, rather a mean of emboldening unto vicious and wicked life, than either any help to prevent future, or medicine to remedy present evils in the soul of man. The Fathers were slow and always fearful to absolve any before very manifest tokens given of a true penitent and contrite spirit. It was not their custom to remit sin first, and then to impose works of satisfaction, as the fashion of Rome is now; insomuch that this their preposterous course, and misordered practice, hath bred in them also an error concerning the end and purpose of these works. For against the guiltiness of sin, and the danger of everlasting condemnation thereby incurred, confession and absolution succeeding the same, are, as they take it, a remedy sufficient; and therefore what their penitentiaries do think good to enjoin farther,  whether it be a number of Ave-Maries daily to be scored up, a journey of pilgrimage to be undertaken, some few dishes of ordinary diet to be exchanged, offerings to be made at the shrines of saints, or a little to be scraped off from men’s superfluity for relief of poor people, all is in lieu or exchange with God, whose justice, notwithstanding our pardon, yet oweth us still some temporal punishment, either in this or in the life to come, except we quite it ourselves here with works of the former kind, and continued till the balance of God’s most strict severity shall find the pains we have taken equivalent with the plagues we should endure, or else that the mercy of the pope relieve us. And at this postern gate cometh in the whole mart of papal indulgences, so infinitely strewed, that the pardon of sin, which heretofore was obtained hardly and by much suit, is with them become now almost impossible to be escaped.

[8] To set down then the force of this sentence in absolving penitents; there are in sin these three things: the act which passeth away and vanisheth; the pollution wherewith it leaveth the soul defiled; and the punishment whereunto they are made subject that have committed it. The act of sin, is every deed, word, and thought against the law of God. “For sin is the transgression of the law;” and although the deed itself do not continue, yet is that bad quality permanent, whereby it maketh the soul unrighteous and deformed in God’s sight. “From the heart come evil cogitations, murders, adulteries, fornications, thefts, false testimonies, slanders; these are things which defile a man.” They do not only, as effects of impurity, argue the nest to be unclean, out of which they came, but as causes they strengthen that disposition unto wickedness which brought them forth; they are both fruits and seeds of uncleanness, they nourish the root out of which they grow, they breed that iniquity which bred them. The blot therefore of sin abideth, though the act be transitory. And out of both ariseth a present debt, to endure what punishment soever the evil which we have done deserveth;  an obligation, in the chains whereof sinners by the justice of Almighty God continue bound till repentance loose them. “Repent this thy wickedness,” saith Peter unto Simon Magus, “beseech God, that if it be possible the thought of thine heart may be pardoned; for I see that thou art in the gall of bitterness, and in the bond of iniquity.” In like manner Salomon: “The wicked shall be held fast in the cords of his own sin.”

Nor doth God only bind sinners hands and foot by the dreadful determination of his own unsearchable judgment against them; but sometime also the Church bindeth by the censures of her discipline: so that when offenders upon their repentance are by the same discipline absolved, the Church looseth but her own bands, the chains wherein she had tied them before.

The act of sin God alone remitteth, in that his purpose is never to call it to account, or to lay it unto men’s charge; the stain he washeth out by the sanctifying grace of his Spirit; and concerning the punishment of sin, as none else hath power to cast body and soul into hell-fire, so none power to deliver either besides him6. As for the ministerial sentence of private absolution, it can be no more than a declaration what God hath done; it hath but the force of the Prophet Nathan’s absolution, “God hath taken away thy sin:” than which construction, especially of words judicial, there is not any thing more vulgar. For example, the publicans are said in the Gospel to have justified God; the Jews in Malachi to have blessed proud men, which sin and prosper; not that the one did make God righteous, or the other the wicked happy: but to “bless,” to “justify,” and to “absolve,” are as commonly used for words of judgment or declaration, as of true and real efficacy. Yea even by the  opinion of the Master of Sentences, “it may be soundly affirmed and thought that God alone doth remit and retain sins, although he have given the Church power to do both: but he one way, and the Church another. He only by himself forgiveth sin, who cleanseth the soul from inward blemish, and looseth the debt of eternal death. So great a privilege he hath not given unto his priests, who notwithstanding are authorized to loose and bind, that is to say, ton declare who are bound, and who are loosed. For albeit a man be already cleared before God, yet he is not in the face of the Church so taken, but by virtue of the priest’s sentence; who likewise may be said to bind by imposing satisfactions, and to loose by admitting to the holy communion.”

Saint Hierome also, whom the Master of the Sentences allegeth for more countenance of his own opinion, doth no less plainly and directly affirm: “That as the priests of the law could only discern, and neither cause nor remove leprosies; so the ministers of the Gospel, when they retain or remit sin, do but in the one judge how long we continue guilty, and in the other declare when we are clear or free.” For there is nothing more apparent, than that the discipline  of repentance both public and private was ordained as an outward mean to bring men to the virtue of inward conversion; so that when this by manifest tokens did seem effected, absolution ensuing (which could not make) served only to declare men innocent.

[9] But the cause wherefore they are so stiff, and have forsaken their own master in this point, is for that they hold the private discipline of penitency to be a sacrament, absolution an external sign in this sacrament, the signs external of all sacraments in the New Testament to be both causes of that which they signify, and signs of that which they truly cause.

To this opinion concerning sacraments they are now tied by expounding a canon in the Florentine council according to a former scholastical invention received from Thomas. For his device it was, that the mercy of God, which useth sacraments as instruments whereby to work, endueth them at the time of their administration with supernatural force and ability to induce grace into the souls of men; even as the axe and saw do serves to bring timber into that fashion which the mind of the artificer intendeth. His conceitt  Scotus, Occam, Petrus Alliacensis, with sundry others, do most earnestly and strongly impugn, shewing very good reason, wherefore no sacrament of the new law can either by virtue which itself hath, or by force supernaturally given it, be properly a cause to work grace; but sacraments are therefore said to work or confer grace, because the will of Almighty God is, although not to give them such efficacy, yet himself to be present in the ministry of the working that effect, which proceedeth wholly from him without any real operation of theirs, such as can enter into men’s souls.

[10] In which construction, seeing that our books and writings have made it known to the world how we join with them, it seemeth very hard and injurious dealing, that Bellarmine throughout the whole course of his second book De Sacramentis in Genere, should so boldly face down his adversaries, as if their opinion were, that sacraments are naked, empty, and uneffectual signs; wherein there is no other force than only such as in pictures to stir up the mind, that so by theory and speculation of things represented, faith may grow: finally, that all the operation which sacraments  have, is a sensible and divine instruction. But had it pleased him not to hoodwink his own knowledge, I nothing doubt but he fully saw how to answer himself; it being a matter very strange and incredible, that one which with so great diligence had winnowed his adversaries’ writings, should be ignorant of their minds. For, even as in the person of our Lord Jesus Christ both God and man, when his human nature is by itself considered, we may not attribute that unto him, which we do and must ascribe as oft as respect is had unto both natures combined; so because in sacraments there are two things distinctly to be considered, the outward sign, and the secret concurrence of God’s most blessed Spirit, in which respect our Saviour hath taught that water and the Holy Ghost are combined to work the mystery of new birth; sacraments therefore as signs have only those effects before mentioned; but of sacraments, in that by God’s own will and ordinance they are signs assisted always with the power of the Holy Ghost, we acknowledge whatsoever either the places of Scripture, or the authorities of councils and fathers, or the proofs and arguments of reason which he allegeth, can shew to be wrought by them. The elements and words have power of infallible signification, for which they are called seals of God’s truth; the spirit affixed unto those elements and words, power of operation within the soul, most admirable, divine, and impossible to be exprest. For so God hath instituted and ordained, that, together with due administration and receipt of sacramental signs, there shall proceed from himself grace effectual to sanctify, to cure, to comfort, and whatsoever is else for the good of the souls of men.

Howbeit this opinion Thomas rejecteth, under pretence that it maketh sacramental words and elements to be in themselves no more than signs, whereas they ought to be held as causes of that they signify. He therefore reformeth  it with this addition, that the very sensible parts of the Sacraments do instrumentally effect and produce, not grace (for the schoolmen both of these times and long after did for the most part maintain it untrue, and some of them unpossible, that sanctifying grace should efficiently proceed but from God alone, and that by immediate creation as the substance of the soul doth;) but the phantasy which Thomas had was, that sensible things through Christ and the priest’s benediction receive a certain supernatural transitory force, which leaveth behind it a kind of preparative quality or beauty within the soul, whereupon immediately from God doth ensue the grace that justifieth.

Now they which pretend to follow Thomas, differ from him in two points. For first, they make grace an immediate effect of the outward sign, which he for the dignity and excellency thereof was afraid to do. Secondly, whereas he to produce but a preparative quality in the soul, did imagine God to create in the instrument a supernatural gift or ability; they confess, that nothing is created, infused, or any way inherent, either in the word or in the elements; nothing that giveth them instrumental efficacy, but God’s mere motion or application. Are they able to explain unto us, or themselves  to conceive, what they mean when they thus speak? For example, let them teach us, in the sacrament of Baptism, what it is for water to be moved till it bring forth grace. The application thereof by the minister is plain to sense; the force which it hath in the mind, as a moral instrument of information or instruction, we know by reason; and by faith we understand how God doth assist it with his Spirit: whereupon ensueth the grace which Saint Cyprian did in himself observe, saying, “After the bath of regeneration having scoured out the stained foulness of former life, supernatural light had entrance into the breast which was purified and cleansed for it: after that a second nativity had made me another man, by inward receipt of the Spirit from heaven; things doubtful began in marvellous manner to appear certain, that to be open which lay hid, darkness to shine like the clear light, former hardness to be made facility, impossibility easiness: insomuch as it might be discerned how that was earthly, which before had been carnally bred, and lived, given over unto sins; that now God’s own, which the Holy Ghost did quicken.”

[11] Our opinion is therefore plain unto every man’s understanding. We take it for a very good speech which Bonaventure hath uttered in saying, “Heed must be taken, that while we ascribe too much to the bodily signs in way of their commendation, we withdraw not the honour which is due to the cause which worketh in them, and the soul which receiveth them:” whereunto we conformably teach, that the outward sign applied hath of itself no natural efficacy towards grace, neither doth God put into it any supernatural  inherent virtue. And, as I think, we thus far avouch no more than they themselves confess to be very true.

If any thing displease them, it is because we add to these premisses another assertion; that with the outward sign God joineth his Holy Spirit, and so the whole instrument of God bringeth that to pass, whereunto the baser and meaner part could not extend. As for operations through the motions of signs, they are dark, intricate, and obscure; perhaps possible; howbeit, not proved either true or likely, by alleging that the touch of our Saviour’s garment restored health, clay sight, when he applied it. Although ten thousand such examples should be brought, they overthrow not this one principle; that, where the instrument is without inherent virtue, the effect must necessarily proceed from the only agent’s adherent power.

It passeth a man’s conceit how water should be carried into the soul with any force of divine motion, or grace proceed but merely from the influence of God’s Spirit. Notwithstanding if God did himself teach his Church in this case to believe that which he hath not given us capacity to comprehend, how incredible soever it may seem, yet our wits should submit themselves, and reason give place unto faith therein. But they yield it to be no question of faith, how grace doth proceed from sacraments; if in general they be acknowledged true instrumental causes, by the ministry whereof men receive divine grace; and that they which impute grace to the only operation of God himself, concurring with the external sign, do no less acknowledge the true efficacy of the sacrament, than they that ascribe the same to the quality of the sign  applied, or to the motion of God applying, and so far carrying it, till grace be thereby not created, but extracted out of the natural possibility of the soul. Nevertheless this last philosophical imagination (if I may call it philosophical,) which useth the terms, but overthroweth the rules of philosophy, and hath no article of faith to support it, but whatsoever it be, they follow it in a manner all; they cast off the first opinion, wherein is most perspicuity and strongest evidence of certain truth.

The Council of Florence and Trent, defining that sacraments contain and confer grace, the sense whereof (if it liked them) might so easily conform itself with the same opinion, which they drew without any just cause quite and clean the other way, making grace the issue of bare words in such sacraments as they have framed destitute of any visible element, and holding it the offspring as well of elements as of words in those sacraments where both are, but in no sacrament acknowledging grace to be the fruit of the Holy Ghost working with the outward sign and not by it; in such sort as Thomas himself teacheth; that the Apostle’s imposition of hands caused not the coming of the Holy Ghost, which notwithstanding was bestowed together with the exercise of that ceremony; yea, by it, (saith the Evangelist,) to wit, as by a mean which came between the true agent and the effect, but not otherwise.

Many of the ancient Fathers, presupposing that the faithful before Christ had not till the time of his coming that perfect life and salvation which they looked for and we possess, thought likewise their sacraments to be but prefigurations of that which ours in present do exhibit. For which cause the Florentine council comparing the one with the other, saith, “That the old did only shadow grace, which was afterward to be given through the passion of Jesus Christ.” But the after-wit of later days hath found out another more exquisite distinction, that evangelical sacraments are causes to effect grace, through motion of signs legal, according to the same signification and sense wherein evangelical sacraments are held by us to be God’s instruments for that purpose. For howsoever Bellarmine hath shrunk up the Lutherans’ sinews, and cut off our doctrine by the skirts; Allen, although he term us heretics, according to the usual bitter venom of his proud style, doth yet ingenuously confess, that the old schoolmen’s doctrine and ours is one concerning sacramental efficacy, derived from God himself assisting by promise those outward signs of elements and words, out of which their schoolmen of the newer mint are so desirous to hatch grace. Where God doth work and use these outward means, wherein he neither findeth nor planteth force and aptness towards his intended purpose, such means are but signs to bring men to  the consideration of his own omnipotent power, which without the use of things sensible would not be marked. At the time therefore when he giveth his heavenly grace, he applieth by the hands of his ministers that which betokeneth the same; nor only betokeneth, but, being also accompanied for ever with such power as doth truly work, is in that respect termed God’s instrument, a true efficient cause of grace; a cause not in itself, but only by connexion of that which is in itself a cause, namely God’s own strength and power. Sacraments, that is to say, the outward signs in sacraments, work nothing till they be blessed and sanctified of God. But what is God’s heavenly benediction and sanctification, saving only the association of his Spirit? Shall we say that sacraments are like magical signs, if thus they have their effect? Is it magic for God to manifest by things sensible what he doth, and to do by his own most glorious Spirit really what he manifesteth in his sacraments? the delivery and administration whereof remaineth in the hands of mortal men, by whom, as by personal instruments, God doth apply signs, and with signs inseparably join his Spirit, and through the power of his Spirit work grace. The first is by way of concomitance and consequence to deliver the rest also that either accompany or ensue.

It is not here, as in cases of mutual commerce, where diverse persons have divers acts to be performed in their own behalf; a creditor to shew his bill, and a debtor to pay his money. But God and man do here meet in one action upon a third, in whom, as it is the work of God to create grace, so it is his work by the hand of the minister to apply a sign which should betoken, and his work to annex, that Spirit, which shall effect it. The action therefore is but one, God the author thereof, and man a cooperator by him assigned to work for, with, and under him. God the giver of grace by the outward ministry of man, so far forth as he authorizeth man to apply the sacraments of grace in the soul, which he alone worketh, without either instrument or co-agent.

[12] Whereas therefore with us the remission of sin is ascribed unto God, as a thing which proceedeth from him only, and presently followeth upon the virtue of true repentance appearing in man; that which we attribute to the virtue,  they do not only impute to the sacrament of repentance, but having made repentance a sacrament, and thinking of sacraments as they do, they are enforced to make the ministry of his priests and their absolution a cause of that which the sole omnipotency of God worketh.

And yet, for mine own part, I am not able well to conceive how their doctrine, that human absolution is really a cause out of which our deliverance from sin doth ensue, can cleave with the council of Trent, defining, “That contrition perfected with charity doth at all times itself reconcile offenders to God, before they come to receive actually the sacrament of penance:” how it can stand with those discourses of the learnedest rabbins, which grant, “That whosoever turneth unto God with his whole heart, hath immediately his sins taken away; that if a man be truly converted, his pardon can neither be denied nor delayed:” it doth not stay for the priest’s absolution, but presently followeth. Surely, if every contrite sinner, in whom there is charity and a sincere conversion of heart, have remission of sins given him before he seek it at the priest’s hands; if reconciliation to God be a present and immediate sequel upon every such conversion or change: it must of necessity follow, seeing no man can be a true penitent or contrite which doth not both love God and sincerely abhor sin, that therefore they all before absolution attain forgiveness; whereunto notwithstanding absolution is pretended a cause so necessary, that sin without it, except in some rare extraordinary case, cannot possibly be remitted. Shall absolution be a cause producing and working that effect which is always brought forth without it, and had before absolution be sought? But when they which are thus beforehand pardoned of God shall come to be also assoiled by the  priest, I would know what force his absolution hath in this case? Are they able to say here that the priest doth remit any thing? Yet when any of ours ascribeth the work of remission to God, and interpreteth the priest’s sentence to be but a solemn declaration of that which God himself hath already performed, they scorn at it; they urge against it, that if this were true, our Saviour Christ should rather have said, “What is loosed in heaven, ye shall loose on earth,” than as he doth, “Whatsoever ye loose on earth, shall in heaven be loosed.” As if he were to learn of us how to place his words, and not we to crave rather of him a sound and right understanding, lest to his dishonour and our own hurt we misexpound them. It sufficeth, I think, both against their constructions to have proved that they ground an untruth on his speech, and in behalf of our own, that his words without any such transposition do very well admit the sense we give them; which is, that he taketh to himself the lawful proceedings of authority in his name, and that the act of spiritual authority in this case, is by sentence to acquit or pronounce them free from sin whom they judge to be sincerely and truly penitent; which interpretation they themselves do acknowledge, though not sufficient, yet very true. Absolution, they say, declareth indeed, but this is not all, for it likewise maketh innocent; which addition being an untruth proved, our truth granted hath we hope sufficiency without it, and consequently our opinion therein neither to be challenged as untrue, nor as unsufficient.

[13] To rid themselves out of these briers, and to make remission of sins an effect of absolution, notwithstanding that which hitherto hath been said, they have two shifts. As first, that in many penitents there is but attrition of heart, which attrition they define to be grief proceeding from fear without love; and to these they say absolution doth give that contrition  whereby men are really purged from sin. Secondly, that even where contrition or inward repentance doth cleanse without absolution, the reason why it cometh so to pass is, because such contrites intend and desire absolution, though they have it not. Which two things granted; the one, that absolution given maketh them contrite that are not, the other, that even in them which are contrite, the cause why God remitteth sin is the purpose or desire they have to receive absolution; we are not to stand against a sequel so clear and manifest as this, that always remission of sin proceedeth from absolution either had or desired.

But should a reasonable man give credit to their bare conceit, and because their positions have driven them to imagine absolving of unsufficiently-disposed penitents to be a real creating of further virtue in them, must all other men think it true? Let them cancel henceforward and blot out of all their books those old cautions touching necessity of wisdom, lest priests should inconsiderately absolve any man in whom there were not apparent tokens of true repentance; which to do was, in Cyprian’s judgment, “pestilent deceit and flattery, not only not available, but hurtful to them that had transgrest; a frivolous, frustrate and false peace, such as caused the unrighteous to trust to a lie, and destroyed them unto whom it promised safety.” What needeth observation whether penitents have worthiness and bring contrition, if the words of absolution do infuse contrition? Have they borne us all this while in hand that contrition is a part of the  matter of their sacraments, a condition or preparation of the mind towards grace to be received by absolution in the form of their sacrament? and must we now believe that the form doth give the matter? that absolution bestoweth contrition, and that the words do make presently of Saul, David; of Judas, Peter? For what was the penitency of Saul and Judas, but plain attrition; horror of sin through fear of punishment, without any loving sense, or taste of God’s mercy?

Their other fiction, imputing remission of sin to desire of absolution from the priest, even in them which are truly contrite, is an evasion somewhat more witty, but no whit more possible for them to prove. Belief of the world and judgment to come, faith in the promises and sufferings of Christ for mankind, fear of his majesty, love of his mercy, grief for sin, hope for pardon, suit for grace; these we know to be the elements of true contrition: suppose that besides all this, God did also command that every penitent should seek his absolution at the priest’s hands; where so many causes are concurring unto one effect, have they any reason to impute the whole effect unto one? any reason in the choice of that one, to pass by faith, fear, love, humility, hope, prayer, whatsoever else, and to enthronize above them all a desire of absolution from the priest, as if, in the whole work of man’s repentance, God did regard and accept nothing, but for and in consideration of this? Why doth the Tridentine council impute it to charity, “that contrites are reconciled in God’s sight before they receive the sacrament of penance,” if desired absolution be the true cause?

But let this pass how it will; seeing the question is not, what virtues God may accept in penitent sinners, but what grace absolution actually given doth really bestow upon them. If it were, as they will have it, that God, regarding the humiliation of a contrite spirit, because there is joined therewith a lowly desire of the sacrament of priestly absolution, pardoneth immediately and forgiveth all offences; doth this any thing help to prove that absolution received afterward  from the priest, can more than declare him already pardoned which did desire it? To desire absolution, presupposing it commanded, is obedience; and obedience in that case is a branch of the virtue of repentance; which virtue being thereby made effectual to the taking away of sins without the sacrament of repentance, is it not an argument that the sacrament of absolution hath here no efficacy, but the virtue of contrition worketh all? For how should any effect ensue from causes which actually are not? The sacrament must be applied wheresoever any grace doth proceed from it. So that where it is but desired only, whatsoever may follow upon God’s acceptation of this desire, the sacrament afterwards received can be no cause thereof. Wherefore the further we wade, the better we see it still appear, that the priest doth never in absolution, no not so much as by way of service and ministry, really either forgive the act, take away the uncleanness, or remove the punishment of sin: but if the party penitent come contrite, he hath by their own grant absolution before absolution; if not contrite, although the priest should ten thousand times absolvey him, all were in vain. For which cause, the ancienter and better sort of their school-divines, Abulensis, Alexander Hales, Bonaventure, ascribe the real abolition of sin and eternal punishment to the mere pardon of Almighty God without dependency upon the priest’s absolution as a cause to effect the same. His absolution hath in their doctrine certain other effects specified but this denied.


Wherefore, having hitherto spoken of the virtue of repentance required; of the discipline of repentance which Christ did establish; and of the sacrament of repentance invented sithence, against the pretended force of human absolution in sacramental penitency: let it suffice thus far to have shewed how God alone doth truly give, the virtue of repentance alone procure, and private ministerial absolution but declare remission of sins.

[14] Now the last and sometimes hardest to be satisfied by repentance, are our minds; and our minds we have then satisfied, when the conscience is of guilty become clear. For as long as we are in ourselves privy to our own most heinous crimes, but without sense of God’s mercy and grace towards us, unless the heart be either brutish for want of knowledge, or altogether hardened by wilful atheism, the remorse of sin is in it as the deadly sting of a serpent. Which point sith very infidels and heathens have observed in the nature of sin (for the disease they felt, though they knew no remedy to help it) we are not rashly to despise those sentences which are the testimonies of their experience touching this point. They knew that the eye of a man’s own conscience is more to be feared by evil doers than the presence of a thousand witnesses, inasmuch as the mouths of other accusers are many ways stopt, the ears of the accused not always subject to glowing with contumely and exprobration; whereas a guilty mind being forced to be still both a martyr and a tyrant itself, must of necessity endure perpetual anguish and grief. For, as the body is rent with stripes, so the mind with guiltiness of cruelty, lust, and wicked resolutions. Which furies brought the Emperor Tiberius sometimes into such perplexity, that writing to the senate, his wonted art of dissimulation failed him utterly in this case; and whereas it had been ever his peculiar delight so to speak that no man might be able to sound his meaning, he had not  the power to conceal what he felt through the secret scourge of an evil conscience, though no necessity did now enforce to disclose the same. “What to write, or how to write, at this present, if I know,” saith Tiberius, “let those gods and goddesses, who thus continually eat me, only be worse to me than they are.” It was not his imperial dignity and power that could provide a way to protect him against himself, the fears and suspicions which improbity had bred being strengthened by every occasion, and those virtues clean banished which are the only foundation of sound tranquillity of mind. For which cause it hath been truly said, and agreeably with all men’s experience, that if the righteous did excel in no other privilege, yet far happier they are than the contrary sort of men, for that their hopes be always better.

Neither are we to marvel that these things, known unto all, do stay so few from being authors of their own woe. For we see by the ancient example of Joseph’s unkind brethren, how it cometh to remembrance easily when crimes are once past, what the difference is of good from evil, and of right from wrong: but such considerations when they should have prevented sin, were overmatcht by unordinate desires.

Are we not bound then with all thankfulness to acknowledge his infinite goodness and mercy, which hath revealed unto us the way how to rid ourselves of these mazes; the way how to shake off that yoke, which no flesh is able to bear; the way how to change most grisly horror into a comfortable apprehension of heavenly joy?

[15] Whereunto there are many which labour with so much the greater difficulty, because imbecility of mind doth not suffer them to censure rightly their own doings: some fearful lest the enormity of their crimes be so impardonable that no repentance can do them good; some lest the imperfection of their repentance make it uneffectual to the taking away of sin. The one drive all things to this issue, whether  they be not men which have sinned against the Holy Ghost; the other to this, what repentance is sufficient to clear sinners, and to assure them that they are delivered.

Such as by error charge themselves of unpardonable sin, must think, it may be they deem that impardonable which is not. Our Saviour speaketh indeed of ah blasphemy which shall never be forgiven. But have they any sure and infallible knowledge what that blasphemy is? If not, why are they unjust and cruel to their own souls, imagining certainty of guiltiness in a crime concerning the very nature whereof they are uncertain? For mine own part, although where this blasphemy is mentioned, the cause why our Saviour spake thereof was the Pharisees’ blasphemy, which were not afraid to say, “He had an unclean spirit, and did cast out spirits by the power of Beelzebub;” nevertheless I dare not precisely deny but that even the very Pharisees themselves might have repented and been forgiven, and that our Lord Jesus Christ peradventure might but take occasion at their blasphemy, which as yet was pardonable, to tell them further of an unpardonable blasphemy, whereinto he foresaw that the Jews would fall. For it is plain that many thousands, at the first professing Christian religion, became afterwards wilful apostatas, moved with no other cause of revolt, but mere indignation that the Gentiles should enjoy the benefit of the Gospel as much as they, and yet not be burthened with the yoke of Moses’ law. The Apostles by preaching had won them to Christ, in whose name they embraced with great alacrity the full remission of their former sins and iniquities; they received by imposition of the Apostles’ hands that grace and power of the Holy Ghost whereby they cured diseases, prophesied, spake with tongues: and yet in the end after all this they fell utterly away, renounced the mysteries of Christian faith, blasphemed in their formal abjurations that most glorious and blessed Spirit, the gifts whereof themselves had possest, and by this means sunk their souls in the gulf of that unpardonable sin, whereof as our Lord Jesus  Christ had told them beforehand, so the Apostle at the first appearance of such their revolt putteth them in mind again, that falling now to their former blasphemies, their salvation was irrecoverably gone. It was for them in this case impossible to be renewed by any repentance: because they were now in the state of Satan and his angels, the Judge of quick and dead had passed his irrevocable sentence against them. So great difference there is between infidels unconverted, and backsliders in this manner fallen away, that always we have hope to reclaim the one, which only hate whom they never knew; but to the other, which know and blaspheme, to them that with more than infernal malice accurse both the seen brightness of glory which is in him, and in themselves the tasted goodness of divine grace, as those execrable miscreants did, who first received in extraordinary miraculous manner, and then in outrageous sort blasphemed, the Holy Ghost, abjuring both it and the whole religion, which God by it did confirm and magnify; to such as wilfully thus sin, after so great light of the truth and gifts of the Spirit, there remaineth justly no fruit or benefit to be expected by Christ’s sacrifice.

For all other offenders, without exception or stint, whether they be strangers that seek access, or followers that will make return unto God; upon the tender of their repentance, the grant of his grace standeth everlastingly signed with his blood in the book of eternal life. That which in this case over-terrifieth fearful souls, is a misconceit whereby they imagine every act which we do knowing that we do amiss, and every wilful breach or transgression of God’s law, to be mere sin against the Holy Ghost; forgetting that the Law of Moses itself ordained sacrifices of expiation as well for faults presumptuously committed, as things wherein men offend by error.

[17] Now there are on the contrary side others, who doubting not of God’s mercy toward all that perfectly repent, remain notwitstanding scrupulous and troubled with continual fear, lest defects in their own repentance be a bar  against them. These cast themselves first into very great, and peradventure needless agonies, through misconstruction of things spoken about proportioning our griefs to our sins, for which they never think they have wept and mourned enough; yea, if they have not always a stream of tears at commandment, they take it for a sign of a heart congealed and hardened in sin; when to keep the wound of contrition bleeding, they unfold the circumstances of their transgressions, and endeavour to leave out nothing which may be heavy against themselves. Yet do what they can, they are still fearful, lest herein also they do not that which they ought and might. Come to prayer, their coldness taketh all heart and courage from them; with fasting albeit their flesh should be withered and their blood clean dried up, would they ever the less object, What is this to David’s humiliation? wherein notwithstanding there was not any thing more than necessary. In works of charity and alms-deeds, it is not all the world can persuade them they did ever reach the poor bounty of the widow’s two mites, or by many millions of leagues come near the marks which Cornelius touched: so far they are off from the proud surmise of any penitential supererogation in miserable wretched worms of the earth.

Notwithstanding, forasmuch as they wrong themselves with over rigorous and extreme exactions, by means whereof they fall sometimes into such perplexities as can hardly be allayed; it hath therefore pleased Almighty God, in tender commiseration over these imbecillities of men, to ordain for their spiritual and ghostly comfort consecrated persons, which by sentence of power and authority given from above, may as it were out of his very mouth ascertain timorous and doubtful  minds in their own particular, ease them of all their scrupulosities, leave them settled in peace and satisfied touching the mercy of God towards them. To use the benefit of thist help for our better satisfaction in such cases is so natural, that it can be forbidden no man; but yet not so necessary, that all men should be in case to need it.

[18] They are of the two the happier therefore that can content and satisfy themselves by judging discreetly what they perform, and soundly what God doth require of them. For having that which is most material, the substance of penitency rightly bred; touching signs and tokens thereof, we may boldly affirm that they err, which imagine for every offence a certain proportionable degree in the passions and griefs of mind, whereunto whosoever aspireth not, repenteth in vain: that to frustrate men’s confessions and considerations of sin, except every circumstance which may aggravate the same be unript and laid in the balance, is a merciless extremity, although it be true, that as near as we can such wounds must be searched to the very bottom: last of all, that to set down the like stint, and to shut up the doors of mercy against penitents which come short thereof in the devotion of their prayers, in the continuance of their fasts, in the largeness and bounty of their alms, or in the course of any other such like duties, is more than God hath himself thought meet, and consequently more than mortal men should presume to do. That which God doth chiefly respect in men’s penitency, is their hearts. The heart is it which maketh repentance sincere, sincerity that which findeth favour in God’s sight, and the favour of God that which supplieth by gracious acceptation whatsoever may seem defective in the faithful, hearty, and true offices of his servants. “Take it” (saith Chrysostom) “upon my credit, Such is God’s merciful  inclination towards men, that repentance offered with a single and sincere mind he never refuseth; no not although we be come to the very top of iniquity. If there be a will and desire to return, he receiveth, embraceth, omitteth nothing which may restore us to former happiness; yea, that which is yet above all the rest, albeit we cannot in the duty of satisfying him attain what we ought and would, but come far behind our mark, he taketh nevertheless in good worth that little which we do; be it never so mean, we lose not our labour therein.” The least and lowest step of repentance in Saint Chrysostom’s judgment severeth and setteth us above them that perish in their sin. I will therefore end with St. Augustin’s conclusion, “Lord, in thy book and volume of life all shall be written, as well the least of thy saints, as the chiefest. Let not therefore the unperfect fear; let them only proceed and go forward.”

