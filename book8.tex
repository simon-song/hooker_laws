\chapter*[The Eighth Book]{THE EIGHTH BOOK. 
THEIR SEVENTH ASSERTION, THAT UNTOb NO CIVIL PRINCE OR GOVERNOR THERE MAY BE GIVEN SUCH POWER OF ECCLESIASTICAL DOMINION AS BY THE LAWS OF THIS LAND BELONGETH UNTO THE SUPREME REGENT THEREOF.}
\label{chap:book8}
\addcontentsline{toc}{chapter}{THE EIGHTH BOOK}



[THE MATTER CONTAINED IN THIS EIGHTH BOOK.

I. State of the Question between the Church of England and its Opponents regarding the King’s Supremacy.

II. Principles on which the King’s modified Supremacy is grounded.

III. Warrant for it in the Jewish Dispensation.

IV. Vindication of the Title, Supreme Head of the Church within his own Dominions.

V. Vindication of the Prerogative regarding Church Assemblies.

VI. Vindication of the Prerogative regarding Church Legislation.

VII. Vindication of the Prerogative regarding Nomination of Bishops.

VIII. Vindication of the Prerogative regarding Ecclesiastical Courts.

IX. Vindication of the Prerogative regarding Exemption from Excommunication.]


I. WE come now to the last thing whereof there is controversy moved, namely the power of supreme jurisdiction, which for distinction’s sake we call the power of ecclesiastical dominion.

It was not thought fit in the Jews’ commonwealth, that the exercise of supremacy ecclesiastical should be denied unto him, to whom the exercise of chiefty civil did appertain; and therefore their kings were invested with both. This power they gave unto Simon when they consented that he should be “their prince,” not only “to set men over the works, and over the country, and over the weapons, and over the fortresses,” but also “to provide for the holy things;” “and that he should be obeyed of every man, and that fall the writings in the country should be made in his name, and that it should not be lawful for any of the people or priests to withstand his words, or to call any congregation in the country without him.”

And if it be haply surmised, that thus much was given unto Simon, as being both prince and high priest; which otherwise, being only their civil governor, he could not lawfully have enjoyed: we must note, that all this is no more than the ancient kings of that people had, being kings and not priests. By this power David, Asa, Jehosaphat, Ezekiask, Josias, and the rest, made those laws and orders which the Sacred History speaketh of, concerning matter of mere religion, the affairs of the temple, and service of God. Finally, had it not been by the virtue of this power, how should it possibly have come to pass, that the piety or impiety of the king did always accordingly change the public face of religion, which thing the priests by themselves never did, neither could at any times hinder from being done? Had the priests alone been possessed oft all power in spiritual affairs, how should any law concerning matter of religion have been made but only by them? In them it had been, and  not in the king, to change the face of religion at any time. The altering of religion, the making of ecclesiastical laws, with other the like actions belonging unto the power of dominion, are still termed the deeds of the king; to shew that in him was placed supremacy of power even in this kind over all, and that unto their high priests the same was never committed, saving only at such times as their priests were also kings orb princes over them.

[2]According to the pattern of which example, the like power in causes ecclesiastical is by the laws of this realm annexed unto the crown. And there are which imagine, that kings, being mere lay persons, do by this means exceed the lawful bounds of their calling. Which thing to the end that they may persuade, they first make a necessary separation perpetual and personal between the Church and thee commonwealth. Secondly they so tie all kind of power ecclesiastical unto the Church, as if it were in every degree their only right which are by proper spiritual function termed Church-governors, and might not unto Christian princes in any wise appertain.

To lurk under shifting ambiguities and equivocations of words in matters of principal weight is childish. A church and a commonwealth we grant are things in nature the one distinguished from the other. A commonwealth is one way, and a church another way, defined. In their opinion the church and the commonwealth are corporations, not distinguished  only in nature and definition, but in subsistence perpetually severed; so that they which are of the one can neither appoint nor execute, in whole nor in part, the duties which belong unto them which are of the other, without open breach of the law of God, which hath divided them, and doth require that being son divided they should distinctly and severally work, as depending both upon God, and not hanging one upon the other’s approbation for that which either hath to do.

We say that the care of religion being common unto all societies politic, such societies as do embrace the true religion have the name of the Church given unto every of them for distinction from the rest; so that every body politic hath some religion, but the Church that religion which is only true. Truth of religion is that proper difference whereby a church is distinguished from other politic societies of men. We here mean true religion in gross, and not according to every particular: for they which in some particular points of religion do swerve from the truth, may nevertheless most truly, if we compare them to men of an heathenish religion, be said to hold and profess that religion which is true. For which cause, there being of old so many politic societies established throughout the world, only the commonwealth of Israel, which had the truth of religion, was in that respect the Church of God: and the Church of Jesus Christ is every such politic society of men, as doth in religion hold that truth which is proper to Christianity. As a politic society it doth maintain religion; as a church, that religion which God hath revealed by Jesus Christ.

With us therefore the name of a church importeth only a society of men, first united into some public form of regiment, and secondly distinguished from other societies by the exercise of Christiant religion. With them on the other side the name of the Church in this present question importeth not only a multitude of men so united and so distinguished, but also further the same divided necessarily and perpetually from the body of the commonwealth: so that even in such a politic society as consisteth of none but Christians, yet the Church of  Christ and the commonwealth are two corporations, independently each subsisting by itself.

We hold, that seeing there is not any man of the Church of England but the same man is also a member of the commonwealth; nor any man ax member of the commonwealth, which is not also of the Church of England; therefore as in a figure triangulary the base doth differ from the sides thereof, and yet one and the selfsame line is both a base and also a side; a side simply, a base if it chance to be the bottom and underlie the rest: so, albeit properties and actions of one kind do cause the name of a commonwealth, qualities and functions of another sort the name of a Church to be given unto a multitude, yet one and the selfsame multitude may in such sort be both, can is so with us, that no person appertaining to the one can be denied to be also of the other. Contrariwise, unless they against us should hold, that the Church and the commonwealth are two, both distinct and separate societies, of which two, thee one comprehendeth always persons not belonging to the other; that which they doe they could not conclude out of the difference between the Church and the commonwealth; namely, that bishops may not meddle with the affairs of the commonwealth, because they are governors of another corporation, which is the Church; nor kings with making laws for the Church, because they have government not of this corporation, but of another divided from it, the commonwealth; and the walls of separation between these two must for ever be upheld. They hold the necessity of personal separation, which clean excludeth the power of one man’s dealing in both; we of natural, which doth not hinder but that one and the same person may in both bear a principal sway.


[3]The causes of common received error in this point seem to have been especially two: one, that they who embrace true religion living in such commonwealths as are opposite thereunto, and in other public affairs retaining civil communion with such, are constrained, for the exercise of their religion, to have a several communion with those who are of the same religion with them. This was the state of the Jewish Church both in Egypt and in mBabylon, the state of Christian Churches a long time after Christ. And in this case, because the proper affairs and actions of the Church, as it is the Church, haven no dependence upon the laws, or upon the governors of the civil state, an opinion hath thereby grown, that even so it should be always.This was it which deceived Allen in the writing of his Apology: “The Apostles,” saith he “did govern the church in Rome when  Nero bare rule, even as at this day in all the Turk’s dominions, the Church hath a spiritual regiment without dependence, and so ought she to have, live shes amongst heathens, or with Christians.”

[4]Another occasion of which misconceit is, that things appertaining unto religion are both distinguished from other affairs, and have always had in the Church special persons chosen to be exercised about them. By which distinction of spiritual affairs and persons therein employed from temporal, the error of personal separation always necessary between the Church and the commonwealth hath strengthened itself. For of every politic society that being true which Aristotle hath namely, “that the scope thereof is not simply to live, nor the duty so much to provide for life, as for means of living well:” and that even as the soul is the worthier part of man, so human societies are much more to care for that which tendeth properly unto the soul’s estate, than for such temporal things as this life doth stand in need of: other proof there needs none to shew that as by all men the kingdom of God is first to be sought for, so in all commonwealths things spiritual ought above temporal to be provided for. And of things spiritual, the chiefest is religion.For this cause, persons and things employed peculiarly about the affairs of religion, are by an excellency termed spiritual. The heathen themselves had their spiritual laws, causes, and offices, always severed from their temporal; neither did this make two independent estates among them. God by revealing true religion doth make them that receive it his  Church. Unto the Jews he so revealed the truth of religion, that he gave them in special consideration laws, not only for the administration of things spiritual, but also temporal. The Lord himself appointing both the one and the other in that commonwealth, did not thereby distract it into several independent communities, but institute several functions of one and the same community. Some reason therefore must be alleged why it should be otherwise in the Church of Christ.

Three kinds of proofs for confirmation of the foresaid separation between the Church and commonwealth, the first taken from difference of affairs and offices in each.I shall not need to spend any great store of words in answering that which is brought out of holy Scripture to shew that secular and ecclesiastical affairs and offices are distinguished; neither that which hath been borrowed from antiquity, using by phrase of speech to oppose the commonwealth to the Church of Christ; nor yet them reasons which are wont to be brought forth as witnesses, that the Church and commonwealth are always distinct. For whether a church and a commonwealth do differ, is not the question we strive for; but our controversy is concerning the kind of distinction, whereby they are severed the one from the other; whether as under heathen kings the Church did deal with her own affairs within herself, without depending at all upon any in civil authority, and the commonwealth in hers, altogether without the privity of the Church; so it ought to continue still, even in such commonwealths as have now publicly embraced the truth of Christian religion; whether they ought to be evermore two societies, in such sort, several and distinct.

I ask therefore, what society that was, that was in Rome, whereunto the Apostle did give the name of the Church of Rome in his time? If they answer, as needs they must, that the Church of Rome in those days was that whole society of men which in Rome professed the name of Christ, and not that religion which the laws of the commonwealth did then authorize; we say as much, and therefore grant that the commonwealth of Rome was one society, and the Church of  Rome another, in such sort as there was between them no mutual dependency. But when whole Rome became Christian, when they all embraced the gospel, and made laws in the defence thereof, if it be held that the church and the commonwealth of Rome did then remain as before; there is no way how this could be possible, save only one, and that is, they must restrain the name of the Church in a Christian commonwealth to the clergy, excluding all the residue of believers, both prince and people. For if all that believe be contained in the name of the Church, how should the Church remain by personal subsistence divided from the commonwealth, when the whole commonwealth doth believe?

The Church and the commonwealth therefore are in this case personally one society, which society being termed a commonwealth as it liveth under whatsoever form of secular law and regiment, a church as it hath the spiritual law of Jesus Christ; forasmuch as these two laws contain so many and so different offices, there must of necessity be appointed in it some to one charge, and some to another, yet without dividing the whole, and making it two several impaled societies.

The difference therefore either of affairs or offices ecclesiastical from secular is no argument that the Church and the commonwealth are always separate and independent the one on the other: which thing even Allen himself considering somewhat better, doth in this point a little correct his former judgment before mentioned and confesseth in his  Defence of English Catholics, that “the power political hath her princes, laws, tribunals; the spiritual, her prelates, canons, councils, judgments; and those (when the princes are pagans) wholly separate, but in Christian commonwealths joined though not confounded.” Howbeit afterwards his former sting appeareth again; for in a Christian commonwealth he holdeth, that the Church ought not to depend at all upon the authority of any civil person whatsoever, as in England he saith it doth.

%.Proofs of separation between the Church and commonwealthk, taken from the speeches of the Fathers opposing the one to the other.
[5]It will be objected, that “the Fathers do oftentimes mention the commonwealth and the Church of God by way of opposition.Can the same thing be opposite unto itself? If one and the same society be both, what sense can there be in that speech which saith, that ‘they suffer and flourish together’ What sense in that which maketh one thing adjudged to the Church, another to the commonwealth Finally, in that which putteth a difference between the causes of the province and of the Church? Doth it not hereby appear that the Church and the commonwealth are things evermore personally separate”

No, it doth not hereby appear that there is perpetually  any such separation; we mays speak of them as two, we may sever the rights and causes of the one well enough from the other, in regard of that difference which we grant there is between them, albeit we make no personal difference. For the truth is, that the Church and the commonwealth are names which import things really different; but those things are accidents, and such accidents as may and should always dwell lovingly together in one subject. Wherefore the real difference between the accidents signified by those names, doth not prove different subjects for them always to reside in. For albeit the subjects wherein they are resident be sometime different, as when the people of God have their being among infidels; yet the nature of them is not such but that their subject may be one, and therefore it is but a changeable accident, in those accidents, when the subjects they are in be diverse.

There can be no error in our conceit concerning this point, if we remember still what accident that is, for which a society hath the name of a commonwealth, and what accident that which doth cause it to be termed a Church. A commonwealth we name it simply in regard of some regiment or policy under which men live; a church for the truth of that religion which they profess. Now names betokening accidents unabstracted, do betoken not only those accidents, but also together with them the subjects whereunto they cleave. As when we name a schoolmaster and a physician, these names do not only betoken two accidents, teaching and curing, but also some person or persons in whom these accidents are. For there is no impediment but both may be one man, as well as they are for the most part diverse. The commonwealth and the Church therefore being such names, they do not only betoken those accidents of civil government and Christian religion which we have mentioned, but also together with them such multitudes as are the subjects of those accidents. Again, their nature being such that they may well enough dwell together in one subject, it followeth  that their names, though always implying that difference of accidents which hath been set down, yet do not always imply different subjects also. When we oppose the Church therefore and the commonwealth in am Christian society, we mean by the commonwealth that society with relation unto all the public affairs thereof, only the matter of true religion excepted; by the Church, the same society with only reference unto the matter of true religion, without any other affairs besides: when that society which is both a church and a commonwealth doth flourish in those things which belong unto it as a commonwealth, we then say, “the commonwealth doth flourish;” when in those things which concern it as a church, “the Church doth flourish;” when in both, then “the Church and commonwealth flourish together.”

The Prophet Esay, to note corruptions in the commonwealth, complaineth, “1That where judgment and justice had lodged now were murderers; princes were become companions of thieves; every one loved gifts and rewards; but the fatherless was not judged, neither did the widow’s cause come before them.” To shew abuses in the Church, Malachy doth make his complaint “Ye offer unclean bread upon mine altar: if ye offer the blind for sacrifice, it is not amiss as yes think; if the lame and the sick, nothing is amiss.” The treasures which David did bestow upon the temple do argue the love which he bare to the Church: the pains that Nehemias took for building the walls of the city are tokens of his care for the commonwealth. Causes of the commonwealth, or province, are still as Gallio was content to be judge of “If it were a matter of wrong, or an evil deed, O ye Jews, I would according to reason maintain you.” Causes of the Church are such as Gallio thererejecteth: “If it be a question of your law, look you unto it, I will be no judge of those things.” In respect of these  differences therefore the Church and the commonwealth may in speech be compared or opposed aptly enough the one to the other; yet this is no argument that they are two independent societies.

%.Proofs of perpetual separation and independency between the commonwealth and the Churchc, taken from the effects of punishments inflicted and releasedd by the one or the other.
[6]Some other reasons there are, which seem a little more nearly to make for the purpose, as long as they are but heard and not sifted. For what though a man being severed by excommunication from the Church, be not thereby deprived of freedom in the city; nor being there discommoned, is thereby forthwith excommunicated and excluded from the Church what though the Church be bound to receive them upon repentance, whom the commonwealth may refuse again to admit if it chance the same men to be shut out of both? That division of the church and commonwealth, which they contend for, will very hardly hereupon follow.

For we must note that members of a Christian commonwealth have a triple state; a natural, a civil, and a spiritual. No man’s natural estate is cut off otherwise than by that capital execution, after which he that is gone from the body of the commonwealth doth not, I think, remain still in the body of the visible Church.

And concerning am man’s civil estate, the same is subject partly to inferior abatements of liberty, and partly unto diminution in the very highest degree, such as banishment is; which, sith it casteth out quite and clean from the body of the commonwealth, must needs also consequently cast the banished party even out of the very Church he was of before, because that Church and the commonwealth he was of were  both one and the same society: so that whatsoever doth separate utterly a man’s person from the one, it separateth also from the others. As for such abatements of civil state as take away only some privilege, dignity, or other benefit which a man enjoyeth in the commonwealth, they reach only unto our dealing with public affairs, from which what should let but that men may be excluded and thereunto restored again, without diminishing or augmenting the number of persons in whom either church or commonwealth consisteth? He that by way of punishment loseth his voice in a public election of magistrates, ceaseth not thereby to be a citizen. A man disfranchised may notwithstanding enjoy as a subject the common benefit of protection under laws and magistrates. So that these inferior diminutions which touch men civilly, but neither do clean extinguish their estate as they belong to the commonwealth, nor impair a whit their condition as they are of the Church of God: these I say clearly do prove a difference of the affairs ofx the one from the other, but such a difference as maketh nothing for their surmise of distracted societies.

And concerning excommunication, it cutteth off indeed from the Church, and yet not from the commonwealth; howbeit so, that the party excommunicate is not thereby severed from one body which subsisteth in itself, and retained of another in like sort subsisting; but he that before had fellowship with that society whereof he was a member, as well touching things spiritual as civil, is now by force of excommunication, although not severed from the same body in civil affairs, nevertheless for the time cut off from it as touching communion in those things which belong to the said body, as it is the Church.

A man which hath both been excommunicated by the Church, and deprived of civil dignity in the commonwealth, is upon his repentance necessarily readunited into the one, but not of necessity into the other. What then? that which he is adunited unto is a communion in things divine, whereof saints are partakers; that from which he is withheld  is the benefit of some human privilege or right which other citizens haply enjoy. But are not those Saints and Citizens one and the same people? are they not one and the same society? doth it hereby appear that the Church which receiveth an excommunicate man, can have no dependency of any person which is of chief authority and power, in those things of the commonwealth whereunto the same party is not admitted?

[7]Wherefore to end this point, I conclude: First, that under dominions of infidels, the Church of Christ, and their commonwealth, were two societies independent. Secondly, that in those commonwealths where the bishop of Rome beareth sway, one society is both the Church and the commonwealth; but the bishop of Rome doth divide the body into two diverse bodies, and doth not suffer the Church to depend upon the power of any civil prince or potentate. Thirdly, that within this realm of England the case is neither as in the one, nor as in the other of the former two: but from the state of pagans we differ, in that with us one society is both the Church and commonwealth, which with them it was not; as also from the state of those nations which subject themselves to the bishop of Rome, in that our Church hath dependency upon the chief in our commonwealth, which it hath not under him. In a word, our estate is according to the pattern of God’s own ancient elect people, which people was not part of them the commonwealth, and part of them the Church of God, but the selfsame people whole and entire were both under one chief Governor, on whose supreme authority they did alls depend.

II.[1.] Now the drift of all that hath been alleged to prove perpetual separation and independency between the Church and the commonwealth is, that this being held necessary, it might consequently be thought, that in a Christian kingdom he whose power is greatest over the commonwealth may not lawfully have supremacy of power also over the  Church, as it is a church; that is to say, so far as to order and dispose of spiritual affairs, as the highest uncommanded commander in them. Whereupon it is grown a question, whether power ecclesiastical over the Church, power of dominion in such degreed as the laws of this land do grant unto the sovereign governor thereof, may by the said supreme Head and Governor lawfully be enjoyed and held? For resolution wherein, we are, first, to define what the power of dominion is: then to shew by what right: after what sort: in what measure: with what conveniency: according unto whose example Christian kings may have it. And when these generalities are opened, to examine afterwards how lawful that is which we in regard of dominion do attribute unto our own: namely, the title of headship over the Church, so far as the bounds of this kingdom do reach: the prerogative of calling and dissolving greater assemblies, about spiritual affairs public: the right of assenting unto all those orders concerning religion, which must after be in force as laws: the advancement of principal church-governors to their rooms of prelacy: judicial authority higher than others are capable of: and exemption from being punishable with such kind of censures as the platform of reformation doth teach that they ought to be subject unto.

%What the power of dominion is.
[2]Without order there is no living in public society, because the want thereof is the mother of confusion, whereupon division of necessity followeth, and out of division, inevitable destruction.The Apostle therefore giving instruction to public societies, requireth that all things be orderly done. Order can have no place in things, unless it be settled amongst the persons that shall by office be conversant about them. And if things or persons be ordered,  this doth imply that they are distinguished by degrees. For order is a gradual disposition.

The whole world consisting of parts so many, so different, is by this only thing upheld; he which framed them hath set them in order. Yea, the very Deity itself both keepeth and requireth for ever this to be kept as a law, that wheresoever there is a coagmentation of many, the lowest be knit to the highest by that which being interjacent may cause each to cleave unto other, and so all to continue one.

This order of things and persons in public societies is the work of polity, and the proper instrument thereof in every degree is power; power being that ability which we have of ourselves, or receive from others, for performance of any action. If the action which we ares to perform be conversant about matter of mere religion, the power of performing it is then spiritual; and if that power be such as hath not any other to overrule it, we term it dominion, or power supreme, so far as the bounds thereof do extend.

[3]When therefore Christian kings are said to have spiritual dominion or supreme power in ecclesiastical affairs and causes, the meaning is, that within their own precincts and territories they have authority and power to command even in matters of Christian religion, and that there is no higher nor greater that can in those causes over-command them, where they are placed to reign as kings. But withal we must likewise note that their power is termed supremacy, as being the highest, not simply without exception of any thing. For what man is there so brain-sick, as not to except in such speeches God himself, the King of all the kings of the earth?a Besides, where the law doth give him dominion, who doubteth but that the king who receiveth it must hold it of and under the law? according to that axiom, “Attribuat rex legi, quod lex attribuit ei, potestatem et dominium:” and again, “Rex non debet esse sub homine, sed sub Deo et lege.” Thirdly, whereas it is note altogether  without reason, “that kings are judged to have by virtue of their dominion, although greater power than any, yet not than all the states of those societies conjoined, wherein such sovereign rule is given them;” there is not hereunto any thing contrary by us affirmed, no, not when we grant supreme authority unto kings, because supremacy is no otherwise intended or meant than to exclude partly foreign powers, and partly the power which belongeth in several unto others, contained as parts within that politic body over which those kings have supremacy. “Where the king hath power of dominion, or supreme power, there no foreign state or potentate, no state or potentate domestical, whether it consist of one or of many, can possibly have in the same affairs and causes authority higher than the king.”

Power of spiritual dominion therefore is in causes ecclesiastical that ruling authority, which neither any foreign state, nor yet any part of that politic body at home, wherein the same is established, can lawfully overrule.

%By what right, namely, such as though men do give, God doth ratify.
[4]Unto which supreme power in kings two kinds of adversaries there are that have opposed themselves: one sort defending, “that supreme power in causes ecclesiastical throughout the world appertaineth of divine right to the bishop of Rome:” another sort, “that the said power belongeth in every national church unto the clergy thereof assembled.” We which defend as well against the one as against the other, “that kings within their own precincts may have it,” must shew by what right it may come unto them.

[5]First, unto me it seemeth almost out of doubt and controversy, that every independent multitude, before any certain form of regiment established, hath, under God’s supreme authority, full dominion over itself, even as a man  not tied with the bond of subjection as yet unto any other, hath over himself the like power. God creating mankind did endue it naturally with full power to guide itself in what kind of societies soever it should choose to live. A man which is born lord of himself may be made another’s servant: and that power which naturally whole societies have, may be derived into many, few, or one, under whom the rest shall then live in subjection.

Some multitudes are brought into subjection by force, as they who being subdued are fain to submit their necks unto what yoke it pleaseth their conquerors to lay upon them; which conquerors by just and lawful wars do hold their power over such multitudes as a thing descending unto them, divine providence itself so disposing. For it is God who giveth victory in the day of war. And unto whom dominion in this sort is derived, the same they enjoy according unto that law of nations, which law authorizeth conquerors to reign as absolute lords over them whom they vanquish.

Sometimes it pleaseth God himself by special appointment to choose out and nominate such as to whom dominion shall be given, which thing he did often in the commonwealth of Israel. They who in this sort receive power have it immediately from God, by mere divine right; they by human, on whom the same is bestowed according unto men’s discretion, when they are left free by God to make choice of their own governor. By which of these means soever it happen that kings or governors be advanced unto their states, we must acknowledge both their lawful choice to be approved of God, and themselves to be God’s lieutenants and confess their power his.


As for supreme power in ecclesiastical affairs, the word of God doth no where appoint that all kings should have it, neither that any should not have it; for which cause it seemeth to stand altogether by human right, that unto Christian kings there is such dominion given.

[6]Again, on whom the same is bestowed even at men’s discretion, they likewise do hold it by divine right. If God in his own revealed word haven appointed such power to be, although himself extraordinarily bestow it not, but leave the appointment of the persons unto men; yea, albeit God do neither appoint the thing nor assign the person; nevertheless when men have established both, who doth doubt but that sundry duties and offices depending thereupon are prescribed ins the word of God, and consequently by that very right to be exacted?

For example’s sake, the power which the Roman emperors had over foreign provinces was not a thing which the law of God did ever institute, neither was Tiberius Cæsar by special commission from heaven therewith invested; and yet the payment of tribute unto Cæsar being made emperor is the plain law of Jesus Christ. Unto kings by human right, honour by very divine right, is due; man’s ordinances are  many times presupposed as grounds in the statutes of God. And therefore of what kind soever the means be whereby governors are lawfully advanced unto their seats, as we by the law of God stand bound meekly to acknowledge them for God’s lieutenants, and to confess their power his, so they by the same law are both authorized and required to use that power as far as it may be in any sort available to his honour. The law appointeth no man to be an husband, but if a man have betaken himself unto that condition, it giveth him then authority over his own wife. That the Christian world should be ordered by kingly regiment, the law of God doth not any where command; and yet the law of God doth give them right, which once are exalted to that estate, to exact at the hands of their subjects general obediences in whatsoever affairs their power may serve to command. So God doth ratify the works of that sovereign authority which kings have received by men.

%After what sort.
[7]This is therefore the right whereby kings do hold their power; but yet in what sort the same doth rest and abide in them it somewhat further behoveth to search. Wherein, that we be not enforced to make over-large discourses about the different conditions of sovereign or supreme power, that which we speak of kings shall be with respect too the state and according to the nature of this kingdom, where the people are in no subjection, but such as willingly themselves have condescended unto, for their own most behoof and security. In kingdoms therefore of this quality the highest governor hath indeed universal dominion, but with dependence upon that whole entire body, over the several parts whereof he hath dominion; so that it standeth for an axiom in this case, The king is “major singulis, universis minor.”


[8]The king’s dependency we do not construe as some have done, who are of opinion that no man’s birth can make him a king, but every particular person advanced unto such authority hath at his entrance into his reign the same bestowed upon him, as an estate in condition, by the voluntary deed of the people, in whom it doth lie to put by any one, and to prefer some other before him, better liked of, or judged fitter for the place, and that the party so rejected hath herein no injury, no not although this be done in a place where the crown doth go κατὰt γένος, by succession, and to a person which being capable hath apparently, if blood be respected, the nearest right. They plainly affirm that “in all well-appointed  kingdoms, the custom evermore hath been, and is, that children succeed not their deceased parents till the people after a sort have created them anew, neither that they grow to their fathers as natural and proper heirs, but are then to be reckoned for kings, when at the hands of such as represent the people’s majesty they have by a sceptre and diadem received as it were the investiture of kingly power.” Their very words are “That where such power is settled into a family or kindred, the stock itself is thereby chosen, but not the twig that springeth of it. The next of the stock unto him which reigneth are not through nearness of blood made kings, but rather set forth to stand for the kingdom. Where regal dominion is hereditary, it is notwithstanding if ye look to the persons themselves which have it altogether elective.” To this purpose are alleged heaps of Scriptures concerning the solemn coronation or inauguration of Saul, of David, of Solomon, off others, by the nobles, ancients, and people of the commonwealth of Israel; as if these solemnities were a kind of deed, whereby the right of dominion is given.Which strange, untrue, and unnatural conceits, set abroad by seedsmen of rebellion, only to animate unquiet spirits, and to feed them with a possibility of aspiring  unto thrones and sceptres, if they can win the hearts of the people, what hereditary title soever any other before them may have, I say, these unjust and insolent positions I would not mention, were it not thereby to make the countenance of truth more orient: for unless we will openly proclaim defiance unto all law, equity, and reason, we must (there is no remedy) acknowledge, that in kingdoms hereditary birth giveth right unto sovereign dominion; and the death of the predecessor putteth the successor by blood in seisin. Those public solemnities before mentioned do but either serve for an open testification of the inheritor’s right, or belong to the form of inducting him into possession of that thing he hath right unto. And therefore in case it do happen that without right of blood a man in such wise be possessed, all those things are utterly void, they make him no indefeasible estate, the inheritor by blood may dispossess him as an usurper.

[9]The case thus standing, albeit we judge it a thing most true, that kings, even inheritors, do hold their right ton the power of dominion, with dependency upon the whole entire body politic over which they rule as kings; yet so it may not be understood, as if such dependency did grow, for that every supreme governor doth personally take from thence his power by way of gift, bestowed of their own free accord upon him at the time of his entrance into his said place of sovereign government. But the cause of dependency is ins that first original conveyance, when power was derived by the whole into one; to pass from him unto them, whom out of him nature by lawful birth should produce, and no natural or legal inability make uncapable. Neither can any man with reason think, but that the first institution of kings is a sufficient consideration wherefore their power should always depend on that from which it did then flow.Original influence of power from the body into the king, is cause of the king’s dependency in power upon the body.


[10]By dependency we mean subordination and subjection. A manifest token of which dependency may be this: as there is no more certain argument that lands are held under any as lord, than if we see that such lands in defect of heirs do fall by escheat unto him; in like manner it doth rightly follow, that seeing dominion, when there is none to inherit it, returneth unto the body, therefore they which before were inheritors thereof did hold it with dependency upon the body. So that by comparing the body with the head, as touching power, it seemeth always to reside in both; fundamentally or radically in the one, in the other derivatively; in the one the habit, in the other the act of power.

May then a body politic at all times withdraw in whole or in part that influence of dominion which passeth from it, if inconvenience doth grow thereby? It must be presumed, that supreme governors will not in such case oppose themselves, and be stiff in detaining that, the use whereof is with public detriment: but surely without their consent I see not how the body should be able by any just means to help itself, saving when dominion doth escheat. Such things therefore must be thought upon beforehand, that power may be limited ere it be granted; which is the next thing we are to consider.

%In what measure.
[11]In power of dominion, all kings have not an equal latitude. Kings by conquest make their own charter: so that how large their power, either civil or spiritual, is, we cannot with any certainty define, further than only to set them in general the law of God and nature for bounds. Kings by God’s own special appointment have also that largeness of power, which he doth assign or permit with approbation. Touching kings which were first instituted by agreement and composition made with them over whom they reign, how far their power may lawfully extend, the articles of compact between them must shew: not the articles only of compact at the first beginning, which for the most part are either clean worn out  of knowledge, or else known unto very few, but whatsoever hath been after in free and voluntary manner condescended unto, whether by express consent, whereof positive laws are witnesses, or else by silent allowance famously notified through custom reaching beyond the memory of man. By which means of after-agreement, it cometh many times to pass in kingdoms, that they whose ancient predecessors were by violence and force made subject, do grow even by little and little into that most sweet form of kingly government which philosophers define to bes “regency willingly sustained and endured, with chiefty of power in the greatest things.”

[12]Many of the ancients in their writings do speak of kings with such high and ample terms, as if universality of power, even in regard of things and not of persons only, did appertain to the very being of a king.The reason is, because their speech concerning kings they frame according to the state of those monarchs to whom unlimited authority was given: which some not observing, imagine that all kings, even in that they are kings, ought to have whatsoever power they find any sovereign ruler lawfully to have enjoyed. But theu most judicious philosopher, whose eye scarce any thing did escape which was to be found in the bosom of nature, he considering how far the power of one sovereign ruler may be different from another’s regal authority, noteth in Spartan kings “that of all others lawfully reigning they had the  most restrained power.” A king which hath not supreme power in the greatest things, is rather entitled a king, than invested with real sovereignty. We cannot properly term him a king, of whom it may not be said, at the leastwise, as touching certain the very chiefest affairs of state, αὐτῳ̑ μὲνd ἄρχειν, ἄρχεσθαιδὲ ὑπ’ οὐδενὸς, “his right in them is to have rule, not subject to any other predominante.” I am not of opinion that simply always in kings the most, but the best limited power is best: the most limited is, that which may deal in fewest things; the best, that which in dealing is tied unto the soundest, perfectest, and most indifferent rule; which rule is the law; I mean not only the law of nature and of God, but very national or municipal law consonant thereuntoh. Happier that people whose law is their king in the greatest things, than that whose king is himself their law. Where the king doth guide the state, and the law the king, that commonwealth is like an harp or melodious instrument, the strings whereof are tuned and handled all by one, following as laws the rules and canons of musical science. Most divinely therefore Archytas maketh unto public felicity these four steps, every later whereof doth spring from the former, as from a mother cause; ὁ μὲνl βασιλεὺς νόμιμος, ὁ δὲ ἄρχων ἀκόλουθος, ὁ δὲ ἀρχόμενος ἐλεύθεροςm, ἁ δ’ ὅλαn κοινωνία εὐδαίμων adding on the contrary side, that “where this order is not, it cometh by transgression thereof to pass that the king groweth a tyrant; he that ruleth under him abhorreth  to be guided and commanded by him; the people subject under both, have freedom under neither; and the whole community is wretched.”

[13]In which respect, I cannot choose but commend highly their wisdom, by whom the foundations of this commonwealth have been laid; wherein though no manner person or cause be unsubject to the king’s power, yet so is the power of the king over all and in all limited, that unto all his proceedings the law itself is a rule. The axioms of our regal government are these: “Lex facit regem:” the king’s grant of any favour made contrary to the law is void; “Rex nihil potest nisi quod jure potest.” Our kings therefore, when they take possession of the roomy they are called unto, have it painted out before their eyes, even by the very solemnities and rites of their inauguration, to what affairs by the said law their supreme authority and power reacheth. Crowned we see they are, andcenthronized, and anointed: the crown a sign of military; the throne, of sedentary ore judicial; the oil, of religious or sacred power.

[14]It is not on any side denied, that kings may have such authority in secular affairs. The question then is, “What power they lawfully may have, and exercise in causes of God.” “A prince, a magistrate, or community,” saith D. Stapleton “may have power to lay corporal punishment on them which are teachers of perverse things; power to make laws for the peace of the Church; power to proclaim, to defend, and even by revenge to preserve from violationi dogmata, very articles of religion themselves.” Others in affection no less devoted unto the papacy, do likewise yield, that “the civil magistrate may by his edicts and laws keep all ecclesiastical persons within the bounds of their duties, and constrain them to observe the canons of the Church, to follow the rules of ancient discipline.” That “if Joas were commended for his care and provision concerning so small a part of religion as the church-treasury; it must needs be both unto Christian kings themselves greater honour, and to Christianity a larger benefit, when the custody of religion whole and of the worship of God in general is their charge.” If therefore all these things mentioned be most properly the affairs of God, and ecclesiastical causes; if the actions specified be works of power; and if that power be such as kings may use of themselves, without the leave of any other power superior in the same things: it followeth necessarily, that kings may have supreme power, not only in civil, but also in ecclesiastical affairs; and consequently, that they may withstand what bishop or pope soever shall, under the pretended claim of higher spiritual authority, oppose himself against their proceedings. But they which have made us the former grant, will hereunto never condescend. What they yield that princes may do, it is with secret exception always understood, if the bishop of Rome give leave, if he interpose no prohibition: wherefore somewhat it is in shew, in truth nothing, which they grant.

Our own reformers do the very like. When they make their discourses in general concerning the authority which magistrates may have, a man would think them far from withdrawing  any jot of that which with reason may be thought due. “The prince and civil magistrate” saith one of them, “hath to see that the laws of God touching his worship, and touching all matters and orders of the Church be executed, and duly observed; and to see that every ecclesiastical person do that office whereunto he is appointed, and to punish those which fail in their office accordingly.” Another acknowledgeth that “the magistrate may lawfully uphold all truth by his sword, punish all persons, enforce all to doc their duties unto God and men; maintain by his laws every point of God’s word, punish all vice in all men; see into all causes, visit the ecclesiastical estate, and correct the abuses thereof; finally, to look to his subjects, that under him they may lead their lives in all godliness and honesty.” A third more frankly professeth that in case their church-discipline were established, so little it shorteneth the arms of sovereign dominion in causes ecclesiastical, that her gracious Majesty, for any thing which they teach or hold to the contrary, may no less than now “remain still over all persons, in all things supreme governess, even with that full and royal authority, superiority, preeminence, supremacy, and prerogative, which the laws already established do give her, and her Majesty’s injunctions, and the articles of the Convocation-house, and other writings apologetical of her royal authority and supreme dignity, do declare and explain.”

[15]Posidonius was wont to say of the Epicure, “That he thought there were no gods, but that those things which he spake concerning the gods were only given out for fear of growing odious amongst men; and therefore that in words he left gods remaining, but in very deed overthrew them, inasmuch as he gave them no kind of motion, no kind of action.” After the very selfsame manner, when we come  unto those particular effects and prerogatives of dominion which the laws of this land do grant unto the kings thereof, it will appear how these men, notwithstanding their large and liberal speeches, abate such parcels out of the fore-alleged grand and flourishing sum, that a man comparing the one with the other may half stand in doubt, lest their opinions in very truth be against that authority which by their speeches they seem mightily to uphold, partly for the avoiding of public obloquy, envy, and hatred; partly to the intent they may both in the end, by establishment of their discipline, extinguish the force of supreme power which princes have, and yet in the meanwhile by giving forth these smooth discourses, obtain that their favourers may have somewhat to allege for them by way of apology, and that in such words as sound towards all kind of fulness in power. But for myself, I had rather construe such their contradictions in the better part, and impute their general acknowledgment of the lawfulness of kingly power unto the force of truth, presenting itself before them sometimes alone; their particular contrarieties, oppositions, denials, unto that error which having so fully possessed their minds, casteth things inconvenient upon them; of which things in their due place.

[16]Touching that which is now in hand, we are on all sides fully agreed; first, that there is not any restraint or limitation of matter for regal authority and power to be conversant in, but of religion wholes, and of whatsoever cause thereto appertaineth, kings may lawfully have charge, they lawfully may therein exercise dominion, and use the temporal sword: secondly, that some kindsu of actions conversant about such affairs are denied unto kings; as, namely, actions of the power  of order, and of that power of jurisdiction, which is with it unseparably joined; power to administer the word and sacraments, power to ordain, to judge as an ordinary, to bind and loose, to excommunicate, and such like: thirdly, that even in these very actions which are proper unto dominion, there must be some certain rule, whereunto kings in all their proceedings ought to be strictly tied; which rule for proceedings in ecclesiastical affairs and causes by regal power, hath not hitherto been agreed upon with soc uniform consent and certainty as might be wished. The different sentences of men herein I will note now go about to examine, but it shall be enough to propose what rule doth seem in this case most reasonable.

%By what rulef.
[17]It hath been declared already in general, how “the best established dominion is where the law doth most rule the king:” the true effect whereof particularly is found as well in ecclesiastical as in civil affairs. In these the king, through his supreme power, may do great things and sundry himself, both appertaining unto peace and war, both at home, by commandment and by commerce with states abroad, because so much the law doth permit. Some things on the other side, the king alone hath no power to do without consent of the lords and commons assembled in parliament: the king of himself cannot change the nature of pleas, nor courts, no not so much as restore blood; because the law is a bar unto him; not any law divine or natural, for against neither it were though kings of themselves might do both, buts the positive laws of the realm have abridged therein and restrained the king’s power; which positive laws, whether by custom or otherwise established without repugnancy unto the law of God and nature, ought no less to be of force even in the spiritual  affairs of the Church. Wherefore in regard of ecclesiastical laws, we willingly embrace that of Ambrose, “Imperator bonusintra ecclesiam, non supra ecclesiam, est; kings have dominion to exercise in ecclesiastical causes, but according to the laws of the Church.” Whether it be therefore the nature of courts, or the form of pleas, or the kind of governors, or the order of proceedings in whatsoever spiritual businesses; for the received laws and liberties of the Church the king hath supreme authority and power, but against them, none.

What such positive laws have appointed to be done by others than the king, or by others with the king, and in what form they have appointed the doing of it, the same of necessity must be kept, neither is the king’s sole authority to alter it.

Yea even as it were a thing unreasonable, if in civil affairs the king (albeit the whole universal body did join with him) should do any thing by their absolute supreme power for the ordering of their state at home, in prejudice of any of those ancient laws of nations which are of force throughout the world, because the necessary commerce of kingdoms dependeth on them; so in principal matters belonging to Christian religion, a thing very scandalous and offensive it must needs be thought, if either kings or laws should dispose of the affairs of God, without any respect had to that which of old hath been reverently thought of throughout the world, and wherein there is no law of God which forceth us to swerve from the way wherein so many and so holy ages have gone.

Wherefore not without good consideration the very law itself hath provided, “That judges ecclesiastical appointed under the king’s commission shall not adjudge for heresy any thing but that which heretofore hath been sop adjudged  by the authority of the canonical scriptures, or by the first four general councils, or by some other general council wherein the same hath been declared heresy by the express words of the said canonical scriptures, or such as hereafter shall be termed heresy by the high court of parliament of this realm, with the assent of the clergy in the convocation.” By which words of the law who doth not plainly see, how in that one branch of proceeding by virtue of the king’s supreme authority, the credit which those fours general councils have throughout all churches evermore had, was judged by the makers of the foresaid act a just cause wherefore they should be mentioned in that case, as a requisite part of they rule wherewith dominion was to be limited.But of this we shall further consider, when we come unto that which sovereign power may do in making ecclesiastical laws.

%With what conveniency.
[18]The cause of deriving supreme power from a whole entire multitude unto some special part thereof, is partly the necessity of expedition in public affairs; partly the inconveniency of confusion and trouble, where a multitude of equals dealeth; and partly the dissipation which must needs ensue in companies, where every man wholly seeketh his own particular (as we all would do, even with other men’s hurt) and haply the very overthrow of ourselves in the end also, if for procurement of the common good of all men, by keeping every several man in order, some were not armed with authority over all, and encouraged with prerogatives of honour to sustain the weighty burden of that charge. The good which is proper unto each man belongeth to the common good of all, as a part of the whole’s perfection; but yeti these two  are things different; for men by that which is proper are severed, united they are by that which is common. Wherefore, besides that which moveth each man in particular to seek his private, there must of necessity in all public societies be also a general mover, directing unto the common good, and framing every man’s particular to it. The end whereunto all government was instituted, was bonum publicum, the universal or common good. Our question is of dominion, for that end and purpose derived into one.Such as in one public state have agreed that the supreme charge of all things should be committed unto one, they I say, considering what inconveniences may grow where states are subject unto sundry supreme authorities, were for fear of those inconveniences withdrawn from liking to establish many; οὐκ ἀγαθὸν πολυκοιρανίη, the multitude of supreme commanders is troublesome. “No man,” saith our Saviour, “can serve two masters:” surely two supreme masters would make any one man’s service somewhat uneasy in such cases as might fall out. Suppose that to-morrow the power which hath dominion in justice require thee at the court; that which in war, at the field; that which in religion, at the temple: all have equal authority over thee, and impossible it is, that thou shouldest be in such case obedient to all: by choosing any one whom thou wilt obey, certain thou art for thy disobedience to incur the displeasure of the other two.

%According unto what example or patterns.
[III.] But there is nothing for which some colourable reason or other may not be found. Are we able to shew any commendable state of government, which by experience and practice hath felt the benefit of being in all causes subject unto the supreme authority of one? Against the polity of  Israel, I hope there will no man except, where Moses deriving so great a part of his burden in government unto others, did notwithstanding retain to himself universal supremacy. Jehosaphat appointing one to be chiefy in the affairs of God, and another in the king’s affairs, did this as having himself dominion over them in both. If therefore, with approbation from heaven, the kings of God’s own chosen people had in the affairs of Jewish religion supreme power, why not Christian kings the like power also in Christian religion? Unless men will answer, as some have done “that touching the Jews, first their religion was of far less perfection and dignity than ours is, ours being that truth whereof theirs was but a shadowish prefigurative resemblance.” Secondly “That all parts of their religion, their laws, their sacrifices, their rites and ceremonies, being fully set down to their hands, and needing no more but only to be put in execution, the kings might well have highest authority to see that done: whereas with us there are a number of mysteries even in belief, which were not so generally for them, as for us, necessary to be with sound express acknowledgment understood; a number of things belonging unto external regiment, and one manner of serving God, not set down by particular ordinances, and  delivered unto us in writing; for which cause the state of the Church doth now require that the spiritual authority of ecclesiastical persons be large, absolute, and not subordinate to regal power.” Thirdly “that whereas God armeth religion Jewish, with temporal, Christian, with am sword but of spiritual punishment; the one with power to imprison, to scourge, and to put to death, the other with bare authority to censure and excommunicate; there is no reason that the Church, which now hath no visible sword, should in regiment be subject unto any other power, than only unto theirs which have authority to bind and loose.” Fourthly “that albeit while the Church was restrained unto one people, it seemed not incommodious to grant their kings the general chiefty of power; yet now, the Church having spread itself over all nations, great inconveniency might thereby grow, if every Christian king in his several territory should have the like power.” Of all these differences, there is not one which doth prove it a thing repugnant unto the law either of God or nature, that all supremacy of external power be in Christian kingdoms granted unto the kings thereof, for preservation of quietness, unity, order, and peace, in such manner as hath been shewed.


[2]The service which we do unto the true God who made heaven and earth is far different from that which heathens have done unto their supposed gods, though nothing else were respected but only the odds between their hope and ours. The offices of piety or true religion sincerely performed have the promises both of this life and of the life to come: the practices of superstition have neither. If notwithstanding the heathens, reckoning upon no other reward for all which they did but only protection and favour in the temporal estate and condition of this present life, and perceiving how great good did hereby publicly grow, as long as fear to displease (they knew not what) divine power was some kind of bridle unto them, did therefore provide that the highest degree of care for their religion should be the principal charge of such as having otherwise also the greatest and chiefest power were by so much the more fit to have custody thereof: shall the like kind of provision be in us thought blameworthy?

%[Ad primum.]
A gross error it is, to think that regal power ought to serve for the good of the body, and not of the soul; for men’s temporal peace, and not for their eternal safety: and if God had ordained kings for no other end and purpose but only to fat up men like hogs, and to see that they have their mast Indeed, to lead men unto salvation by the hand of secret, invisible, and ghostly regiment, or by the external administration of things belonging unto priestly order, (such as the word and sacraments are,) this is denied unto Christian kings: no cause in the world to think them uncapable of supreme authority in the outward government which disposeth the affairs of religion so far forth as the same are disposable by human authority, and to think them uncapable thereof, only for that the said religion is everlastingly beneficial to them that faithfully continue in it. And even as little cause there is, that being admitted thereunto amongst the Jews, they should amongst the Christians of necessity be delivered from ever exercising any such power, for the  dignity and perfection which is in our religion more than in theirs.

%Ad secundum.
[3]It may be a question, whether the affairs of Christianity require more wit, more study, more knowledge of divine things in him which shall order them, than the Jewish religion did. For although we deny not the form of external government, together with all other rites and ceremonies, to have been in more particular manner set down: yet withal it must be considered also, that even this very thing did in some respects make the burthen of their spiritual regiment the harder to be borne; by reason of infinite doubts and difficulties which the very obscurity and darkness of their law did breed, and which being not first decided, the law could not possibly have due execution.

Besides, inasmuch as their law did also dispose even of all kind of civil affairs; their clergy, being the interpreters of the whole law, sustained not only the same labour which divines do amongst us, but even the burthen of our lawyers too. Nevertheless, be it granted that more things do now require to be publicly deliberated and resolved upon with exacter judgment in matters divine than kings for the most part have: their personal inability to judge, in such sort as professors do, letteth not but that their regal authority may have the selfsame degree or sway which the kings of Israel had in the affairs of their religion, to rule and command according to the manner of supreme governors.

%Ad tertium.
[4]As for the sword, wherewith God armed his Church of old, if that were a reasonable cause why kings might then have dominion, I see not but that it ministreth still as forcible an argument for the lawfulness and expediency of their continuance therein now. As we degrade and excommunicate, even so did the Church of the Jews both separate offenders from the temple, and depose the clergy also from their rooms, when cause required. The other sword of corporal punishment is not by Christ’s own appointment in the hands of the Church of Christ, as God did place it himself in the hands of the Jewish Church. For why? He knew that they whom  he sent abroad to gather a people unto him only by persuasive means, were to build up his Church even within the bosom of kingdoms, the chiefest governors whereof would be open enemies unto it every where for the space of many years. Wherefore such commission for discipline he gave them, as they might any where exercise in quiet and peaceable manner; the subjects of no commonwealth being touched in goods or person, by virtue of that spiritual regiment whereunto Christian religion embraced did make them subject.

Now when afterwards it came to pass, that whole kingdoms were made Christian, I demand whether that authority, which served before for the furtherance of religion, may not as effectually serve to the maintenance of Christian religion. Christian religion hath the sword of spiritual discipline. But doth that suffice? The Jewish which had it also, did nevertheless stand in need to be aided with the power of the civil sword. The help whereof, although when Christian religion cannot have, it must without it sustain itself as far as the other which it hath will serve; notwithstanding, where both may be had, what forbiddeth the Church to enjoy the benefit of both? Will any man deny that the Church doth need the rod of corporal punishment to keep her children in obedience withal? Such a law as Macabeus made amongst the Scots, that he which continued an excommunicate two years together, and reconciled not himself to the church, should forfeit all his goods and possessions.

Again, the custom which many Christian churches have to fly to the civil magistrate for coercion of those that will not otherwise be reformed,—these things are proof sufficient that even in Christian religion, the power wherewith ecclesiastical persons were endued at the first is unable to do of itself so much as when secular power doth strengthen it; and that,  not by way of ministry or service, but of predominancy, such as the kings of Israel in their time exercised over the Church of God.

%Ad quartum.
[5]Yea, but the Church of God was then restrained more narrowly to one people and one king, which now being spread throughout all kingdoms, it would be a cause of great dissimilitude in the exercise of Christian religion if every king should be over the affairs of the church where he reigneth supreme ruler.

Dissimilitude in great things is such a thing which draweth great inconvenience after it, a thing which Christian religion must always carefully prevent. And the way to prevent it is, not as some do imagine, the yielding up of supreme power over all churches into one only pastor’s hands; but the framing of their government, especially for matter of substance, every where according to the rule of one only Law, to stand in no less force than the law of nations doth, to be received in all kingdoms, all sovereign rulers to be sworn no otherwise unto it than some are to maintain the liberties, laws, and received customs of the country where they reign. This shall cause uniformity even under several dominions, without those woeful inconveniences whereunto the state of Christendom was subject heretofore, through the tyranny and oppression of that one universal Nimrod who alone did all.

And, till the Christian world be driven to enter into the peaceable and true consultation about some such kind of general law concerning those things of weight and moment wherein now we differ, if one church hath not the same order which another hath: let every church keep as near as may be the order it should have, and commend the just defence thereof unto God, even as Juda did, when it differed in the exercise of religion from that form which Israel followed.

[6]Concerning therefore the matter whereof we have hitherto spoken, let it stand for our final conclusion, that in a free Christian state or kingdom, where one and the selfsame people are the Church and the commonwealth, God through  Christ directing that people to see it for good and weighty considerations expedient that their sovereign lord and governor in causes civil have also in ecclesiastical affairs a supreme power; forasmuch as the light of reason doth lead them unto it, and against it God’s own revealed law hath nothing: surely they do not in submitting themselves thereunto any other than that which a wise and religious people ought to do.

It was but a little overflowing of wit in Thomas Aquinas so to play upon the words of Mosesin the Old, and of Peterin the New Testament, as though because the one did term the Jews “a priestly kingdom,” the other us “a kingly priesthood,” those two substantives “kingdom” and “priesthood” should import, that Judaism did stand through the kings’ superiority over priests, Christianity through the priests’ supreme authority over kings. Is it probable, that Moses and Peter had herein so nice and curious conceits? Or else more likely that both meant one and the same thing; namely that God doth glorify and sanctify his, even with full perfection in both; which thing St. John doth in plainer sort express, saying that “Christ hath made us both kings and priests.”


[IV. 1.] These things being thus first considered, it will be the easier to judge concerning our own estate, whether by force of ecclesiastical dominion with us kings have any other kind of prerogative than they may lawfully hold and enjoy. It is as some do imagine too much, that kings of England should be termed Heads, in relation to the Church.
That which we understand by headship, is their only supreme power in ecclesiastical affairs or causes. That which lawfully princes are, what should make it unlawful for men by special styles or titles to signify? If the having of supreme power be allowed, why is the expressing thereof by the title of head condemned? They seem in words, at them leastwise some of them, now at the length to acknowledge that kings may have supreme government even over all, both persons and causes. We in terming our princes heads of the Church, do but testify that we acknowledge them such governors.
%To be entitled, Heads of the Church under Christ within their own dominions [from D]. 

[2]Against this peradventure it will be replied that  howsoever we interpret ourselves, it is not fit for a mortal man, and therefore not fit for a civil magistrate, to be entitled head of the Church. Why so? First “this title, Head of the Church, was given unto our Saviour Christ to lift him above all powers, rules, and dominions, either in heaven or in earth. Where if this title belong also to the civil magistrate, then it is manifest that there is a power in earth whereunto our Saviour Christ is not in this point superior. Again, if the civil magistrate may have this title, he may be also termed the first-begotten of all creatures, the first-begotten of the dead, yea the Redeemer of his people. For these are alike given him as dignities whereby he is lifted up above all creatures. Besides this, the whole argument of the Apostle in both places doth lead to shew that this title, Head of the Church, cannot be said of any creature. And further, the very demonstrative article, among the Hebrews especially, whom S. Paul doth follow, serveth to tie that which is verified of one, unto himself alone: so that when the apostle doth say that Christ is  ἡ κεϕαλὴ, the Head; it is as much as if he should say, Christ, and noa other, is the Head of the Church.”

[3]Thus have we against the entitling of the highest magistrates, Head, with relation unto the Church, four several arguments, gathered by strong surmise out of words marvellous unlikely to have been written forc any such purpose as that whereunto they are now urged. To the Ephesians, the apostle writeth “That Christ, God hath seated on his own right hand in the heavenly places, above all regency, and authority, and power, and dominion, and whatsoever name is named, not in this world only, but in that which shall be also: and hath under his feet set all things, and hath given him head above all things unto the Church, which is his body, even the complement of him which accomplished all in all.” To the Colossians in like manner “That He is the head of the body of the Church, who is a first-born regency out of the dead, to the end he might be made amongst them all such an one as hath the chiefty:” he meaneth, amongst all them whom before he mentioned, saying “In him all things that are, were made; the things in the heavens, and the things in the earth, the things that are visible, and the things that are invisible, whether they be thrones, or dominations, or regencies,” \&c.

Unto the fore-alleged arguments therefore we answer: first, that it is not simply the title of Head, which lifteth our Saviour above all powers, but the title of Head in such sort understood, as the apostle himself meant it: so that the same being imparted in another sense unto others, doth not any way make those others therein his equals; inasmuch as diversity of things is usual to be understood, even when of words there is no diversity; and it is only the adding of one and the selfsame thing unto diverse persons, which doth argue equality in them. If I term Christ and Cæsar lords, yet this is no equalling of Cæsar with Christ, because it is not  thereby intended. “To term the emperor Lord,” saith Tertullian “I for mine own part will not refuse, so that I be not required to term him Lord in the same sense that God is so termed.”

Neither doth it follow, which is objected in the second place, that if the civil magistrate may be entitled an Head, he may also as well bes termed, “the first-begotten of all creatures,” “the first-begotten of the dead,” and “the Redeemer of his people.” For albeit the former dignity dot lift him up no less than these, yet these terms are not appliable and apt to signify any other inferior dignity, as the former term of Head was.

The argument or matter which the Apostle followeth hath small evidence for proof, that his meaning was to appropriate unto Christ the foresaid title, otherwise than only in such sense as doth make it, being so understood, too high to be given to any creature.

As for the force of the article, whereby our Lord and Saviour is named the Head, it serveth to tie that unto him by way of excellency, which in a meaner degree is common to others; it doth not exclude any other utterly from being termed Head, but from being entitled as Christ is, the Head, by way of the very highest degree of excellency. Not in the communication of names, but in the confusion of things, is error.

[4]Howbeit, if Head were a name which well could not bed, ore never had been used to signify that which a magistrate may be in relation unto some church, but were by continual use of speech appropriated unto that only thing which it signifieth, being applied unto Jesus Christ; then, although we might carry in ourselves a right understanding, yet ought we otherwise rather to speak, unless we interpret our own meaning by some clause of plainer speech; because  we are else in manifest danger to be understood according to that construction and sense wherein such words are usually taken. But here the rarest construction, and most removed from common sense, is that which the word doth import being applied unto Christ; that which we signify by it in giving it unto the magistrate, is a great deal more familiar in the common conceit of men. The word is so fit to signify all kinds of superiority, preeminence, and chiefty, that no one thing is more ordinary than so to use it even in vulgar speech, and in common understanding so to take it. If therefore a Christian king may have any preeminence or chiefty above all other in the Church, (albeit it were less than Theodore Beza giveth, who placeth kings amongst the principal members whereunto public function in the Church belongeth, and denieth not, but that of them which have public function, the civil magistrate’s power hath all the rest at commandment, in regard of that part of his office, which is to procure that peace and good order be especially kept in things concerning the first Table;) evens hereupon tot term him the Head of that Church which is his kingdom, should not seem so unfit a thing: which title surely we would not communicate to any other, no not although it should at our hands be exacted with torments, but that our meaning herein is made known to the whole world, so that no man which will understand can easily be ignorant, that we do not impart to kings, when we term them Heads, the honour which properly is given to our Lord and Saviour  Christ, when the blessed Apostles in Scripture do term him the Head of the Church.

%Differences between Christ’s Headship and that which we give to kings.
[5]The power which we signify by that name, differeth in three things plainly from that which Christ doth challenge.

It differeth in order, measure, and kind. In order, because God hath given him to his Church for the Head, ὑπὲρ πάντα, above all, ὑπεράνω πάσης τη̑ςf ἀρχη̑ς, “far above all principality, and power, and might, and dominion, and every name that is named, not in this world only, but also in that which is to come:” whereas the power which others have is subordinate unto his.

Again, as he differeth in order, so in measure of power also; because God hath given unto him the ends of the earth for his possession; unto him, dominion from sea to sea; unto him, all power in heaven and in earth; unto him, such sovereignty, as doth not only reach over all places, persons, and things, but doth rest in his one only person, and is not by any succession continued: He reigneth as Head and King for ever, nor is there any kind of law which tieth him, but his own proper will and wisdom: his power is absolute, the same jointly over all which it is severally over each; not so the power of any other’s headship. How kings are restrained, and in what sort their authority is limited, we have shewed before. So that unto him is given by the title of Headship over the Church, that largeness of power, wherein neither man nor angel can be matched or compared with him.

The last and the weightiest difference between him and them, is in the very kind of their power. The head being of all other parts of man’s body the most divine hath dominion over all the rest: it is the fountain of sense, of motion; the throne where the guide of the soul doth reign; the court from whence direction of all things human proceedeth.  Why Christ is called Head of his Church, these causes they themselves do yield. As the head is the highests part of a man, above which there is none, always joined with the body: so Christ it the highest in his Church, inseparably knit with it. Again, as the head giveth sense and moving to all the body, so he quickeneth, and together with understanding of heavenly things, giveth strength to walk therein. Seeing therefore, that they cannot affirm Christ sensibly present, or always visibly joined unto his body the Church which is on earth, inasmuch as his corporal residence is in heaven; again, seeing they do not affirm (it were intolerable if they should) that Christ doth personally administer the external regiment of outward actions in the Church, but by the secret inward influence of his grace, giveth spiritual life and the strength of ghostly motions thereunto: impossible it is, that they should so close up their eyes, as not to discern what odds there is between that kind of operation which we imply in the headship of princes, and that which agreeth to our Saviour’s dominion over the Church. The headship which we give unto kings is altogether visibly exercised, and ordereth only the external frame of the Church’s affairs here amongst us; so that it plainly differeth from Christ’s, even in very nature and kind. To be in such sort united unto the Church as he is; to work as he worketh, either on the whole Church, or on any particular assembly, or in any one man; doth neither agree, nor hath possibility of agreeing, unto any besides him.

%Opposition against the first difference, whereby, Christ being Head simply, princes are said to be Heads under Christ.
[6]Against the first distinction or difference it is objected that to entitle a magistrate Head of the Church, although it be under Christ, is most absurd. For Christ hath a twofold superiority; a superiority over his Church, and a superiority over kingdoms: according to the one, he “hath a superior, which is his Father; according to the other, none but immediate authority with his Father:” that is to  say, of the Church he is Head and Governor only as the Son of man; Head and Governor over kingdoms only as the Son of God. In the Church, as man, he hath officers under him, which officers are ecclesiastical persons: as for the civil magistrate, his office belongeth unto kingdoms, and commonwealths, neither is he therein an under or subordinate head of Christi; “considering that his authority cometh from God, simply and immediately, even as our Saviour Christ’s doth.”

Whereunto the sum of our answer is, first, that as Christ being Lord or Head over all, doth by virtue of that sovereignty rule all; so he hath no more a superior in governing his Church, than in exercising sovereign dominion upon the rest of the world besides. Secondly, that all authority, as well civil as ecclesiastical, is subordinate unto his. And thirdly, that the civil magistrate being termed Head, by reason of that authority in ecclesiastical affairs which it hath been already declared that themselves do in word acknowledge to be lawful; it followeth that he is an Head even subordinated of, and to Christ.

For more plain explication whereof, first, unto God we acknowledge daily that kingdom, power, and glory, are his; that he is the immortal and invisible King of ages, as well the future which shall be, as the present which now is. That which the Father doth work as Lord and king over all, he worketh not without, but by the Son, who through coeternal generation receiveth of the Father that power which the Father hath of himself. And for that cause our Saviour’s words concerning his own dominion are, “To me all power both in heaven and in earth is given.” The Father by the Son both did create, and doth guide all; wherefore Christ hath supreme dominion over the whole universal world.

Christ is God, Christ is Λόγος, the consubstantial Word of God, Christ is also that consubstantial Word made man.  As God, he saith of himself “I am Alpha and Omega, the beginning and the end: he which was, which is, and which is to come; even the very Omnipotent.” As the consubstantial Word of God, he had with God before the beginning of the world, that glory which as man he requesteth to have “Father, glorify thy Son now with that glory which with thee Iy enjoyed before the world was.” For there is no necessity that all things spoken of Christ should agree unto him either as God, or else as man; but some things as he is the consubstantial Word of God, some things as he is that Word incarnate. The works of supreme dominion which have been since the first beginning wrought by the power of the Son of God, are now most truly and properly the works of the Son of man: the Word made flesh doth sit for ever, and reign as sovereign Lord over all. Dominion belongeth unto the kingly office of Christ, as propitiation and mediation unto his priestly; instruction, unto his pastoral orb prophetical office. His works of dominion are in sundry degrees or kinds, according to the different condition of them which are subject unto it: he presently doth govern, and hereafter shall judge the world, entire and whole, therefore his regal power cannot be with truth restrained unto a portion of the world only. Notwithstanding forasmuch as all do not shew and acknowledge with dutiful submission that obedience which they owe unto him; therefore such as do, their Lord he is termed by way of excellency, no otherwise than the Apostle doth term God the Saviour generally of all, but especially of the faithful: these being brought to the obedience of faith, are every where spoken of as men translated into that kingdom, wherein whosoever is comprehended, Christ is the author of eternal salvation unto them; they have a high kind of ghostly fellowship with God, and Christ, and saints; or as the Apostle in more ample manner speaketh “Aggregated they are unto  Mount Sion, and to the city of the living God, the celestial Jerusalem, and to the company of innumerable angels, and to the congregation of the first-born, which are written in heaven, and to God the judge of all, and to the spirits of just and perfect men, and to Jesus the Mediator of the New Testament.” In a word, they are of that mystical body, which we term the Church of Christ. As for the rest, we find them accounted “aliens from the commonwealth of Israel, men that lay in the kingdom of darkness, and that are in this present world without God.” Our Saviour’s dominion is therefore over these, as over rebels; over them as dutiful subjects.

Which things being in holy Scriptures so plain, I somewhat muse at those strange positions, that Christin the government of the Church, and superiority over the officers of it, hath himself a superior, which is his Father; but in the government of kingdoms and commonwealths, and in the superiority which he hath over kings, no superior. Again “that the civil magistrate cometh from God immediately, as Christ doth, and is not subordinate unto Christ.” In what evangelist, apostle, or prophet, is it found, that Christ, supreme governor of the Church, should be so unequal to himself, as he is supreme governor of kingdoms? The works of his providence for preservation of mankind by upholding of kingdoms, not only obedient unto, but even rebellious and obstinate against him, are such as proceed from divine power; and are not the works of his providence for safety of God’s elect, by gathering, inspiring, comforting, and every way preserving his Church, such as proceed from the same power likewise? Surely, if Christ“as God and man have ordained certain means for the gathering and keeping of his Church,” seeing this doth belong to the government of his Church; it must in reason follow, I think, that as God and man he worketh in church regiment, and consequently hath  no more therein any superior, than in the government of commonwealths.

Again, to “be in the midst of his, wheresoever they are assembled in his name,” and to be “with them till the world’s end,” are comforts which Christ doth perform to his Church as Lord and Governor; yea, such as he cannot perform but by that very power wherein he hath no superior.

Wherefore, unless it can be proved, that all the works of our Saviour’s government in the Church are done by the mere and only force of his human nature, there is no remedy but to acknowledge it a manifest error, that Christ in the government of the world is equal unto the Father, but not in the government of the Church. Indeed, to the honour of this dominion it cannot be said that God did exalt him otherwise than only according to that human nature wherein he was made low: for as the Son of God, there could no advancement or exaltation grow unto him: and yet the dominion, whereunto he was in his human nature lifted up, is not without divine power exercised. It is by divine power, that the Son of man who sitteth in heaven, doth work as king and lord upon us which are on earth.

The exercise of his dominion over the Church militant cannot choose but cease, when there is no longer any militant Church in the world. And therefore as generals of armies when they have finished their work, are wont to yield up such commissions as were given them for that purpose, and to remain in the state of subjects and not of lords, as concerning their former authority; even so, when the end of all things is come, the Son of man, who till then reigneth, shall do the like, as touching regiment over the militant Church on earth. So that between the Son of man and his brethren, over whom he now reigneth in this their warfare, there shall be then, as touching the exercise of that regiment, no such difference; they not warfaring under him any longer, but he together with them under God receiving the joys of everlasting triumph, that so God may be all in all; all misery in all the wicked through his justice; in all the righteous, through his  love, all felicity and bliss. In the meanwhile he reigneth over this world as king, and doth those things wherein none is superior unto him, whether we respect the works of his providence over kingdoms, or of his regiment over the Church.

The cause of error in this point doth seem to have been a misconceit, that Christ, as Mediator, being inferior unto his Father, doth, as Mediator, all works of regiment over the Church when in truth, governments doth belong to his kingly office, mediatorship, to his priestly. For, as the high priest both offered sacrifice for expiation of the people’s sins, and entered into the holy place, there to make intercession for them: so Christ having finished upon the cross that part of his priestly office which wrought the propitiation for our sins, did afterwards enter into very heaven, and doth there as mediator of the New Testament appear in the sight of God for us. A like slip of judgment it is, when they hold that civil authority is from God, but not mediately through Christ, nor with any subordination unto Christ. For “there is no power,” saith the Apostle, “but from God” nor doth any thing come from God but by the hands of our Lord Jesus Christ.

They deny it not to be said of Christ in the Old Testament “By me kings reign, and princes decree justice; by me princes rule, and the nobles, and all the judges of the earth.” In the New as much is taught “That Christ is the Prince of the kings of the earth.” Wherefore to the end it may more plainly appear how all authority of man is derived from God through Christ, and must by Christian men be acknowledged to be no otherwise held than of and under him;  we are to note, that because whatsoever hath necessary being, the Son of God doth cause it to be, and those things without which the world cannot well continue, have necessary being in the world; a thing of so great use as government amongst men, and human dominion in governmena, cannot choose but be originally from him, and have reference also of subordination unto him. Touching that authority which civil magistrates have in ecclesiastical affairs, it being from God by Christ, as all other good things are, cannot choose but be held as a thing received at his hands; and because such power asc is of necessary used for the ordering of religion, wherein the essence and very being of the Church consisteth, can no otherwise flow from him, than according to that special care which he hath to guide and govern his own people: it followeth that the said authority is of and under him after a more peculiar manner, namely, in that he is Head of the Church, and not in respect of his general regency over the world. “All things,” (saith the Apostle speaking unto the Church) “are yours, and ye are Christ’s, and Christ is God’s.” Kings are Christ’s, as saints; and kings are Christ’s, as kings: as saints, because they are of the Church; as kings, because they are in authority over the Church, if not collectively, yet divisively understood; that is over each particular person within that Church where they are kings. Such authority, reaching both unto all men’s persons, and unto all kinds of causes also, it is not denied but that they lawfully may have and exercise; such authority it is, for which, and for no other in the world, we term them heads; such authority they have under Christ, because he in all things is Lord over all. And even of Christ it is that they have received such authority, inasmuch as of him all lawful powers are: therefore the civil magistrate is, in regard of this power, an under and subordinate head of Christ’s people.


[7]It is but idle when they plead “that although for several companies of men there may be several heads or governors, differing in the measure of their authority from the chiefest who is head of all; yet so it cannot be in the Church, for that the reason why head-magistrates appoint others for such several places is,
because they cannot be present every where to perform the office of a head. But Christ is never from his body, nor from any part of it, and therefore needeth not to substitute any, which may be heads, some over one church and some over another.” Indeed the consideration of man’s imbecillity, which maketh many hands necessary where the burden is too great for one, moved Jethro to be a persuader of Moses, that as number of heads or rulers might be instituted for discharge of that duty by parts, which in whole he saw was troublesome. Now although there be not in Christ any such defect or weakness, yet other causes there may be diverse, more than we are able to search into, wherefore it might seem to him expedient to divide his kingdom into many portions, and tow place many heads over it, that the power which each of them hath in particular with restraint, might illustrate the greatness of his unlimited authority. Besides, howsoever Christ be spiritually always united unto every part of his body, which is the Church; nevertheless we do all know, and they themselves who allege this will, I doubt not, confess also, that from every church here visible, Christ, touching visible and corporal presence, is removed as far as heaven from earth is distant. Visible government is a thing necessary for the Church; and it doth not appear how the exercise of visible government over such multitudes every where dispersed throughout the world should consist without sundry visible governors; whose power being the greatest in that kind so far as it reacheth, they are in consideration thereof termed so far heads. Wherefore, notwithstanding that perpetual conjunction, by virtue whereof our Saviour remaineth always spiritually united unto the parts of his mystical body; Heads endued with supreme power, extending unto a certain compass, are for the exercise of visible regiment not unnecessary.
%Against the second difference, whereby Christ is said to be universally head, the king no further than within his own dominions. 

Some other reasons there are belonging unto this branch, which seem to have been objected, rather for the exercise of men’s wits in dissolving sophisms, than that the authors of them could think in likelihood thereby to strengthen their cause. For example “If the magistrate be head of the Church within his own dominion, then is he none of the Church; for all that Church maketh the body of Christ, and every one of the Church fulfilleth the place of one member of the body. By making the magistrate therefore head we do exclude him from being a member subject to the head, and so leave him no place in the Church.” By which reason, the name of a body politic is supposed to be always taken of the inferior sort alone, excluding the principal guides and governors; contrary to all men’s custom of speech. The error riseth by misconstruing of some scripture sentences, where Christ as the head, and the Church as the body, are compared or opposed the one to the other: and because in such comparisons and oppositions, the body is taken for those only parts which are subject to the head, they imagine that whoso is head of any church, he is even thereby excluded from being a part of that church: that the magistrate can be none of the Church, if so be we make him the head of the church in his own dominions. A chief and principal part of the Church, therefore no part; this is surely a strange conclusion. A church doth indeed make the body of Christ, being wholly taken together; and every one in the same church fulfilleth the place of a member in the body, but not the place of an inferior member, he which hath supreme authority and power over all the rest. Wherefore, by making the magistrate head in his own dominions, we exclude him from being a member subject unto any other person which may visibly there rule in place of an head or governor over him; but so far are we off from leaving him by this means no place in the Church, that we grant him the chiefest place. Indeed the heads of those visible bodies, which are many, can be but parts inferior  in that spiritual body which is but one; yea, they may from this be excluded clean, who notwithstanding ought to be honoured, as possessing in the others the highest rooms: but for the magistrate to be termed, one way, within his own dominions, an head, doth not bar him from being either way a part or member of the Church of God.

As little to the purpose are those other cavils: “A Church which hath the magistrate for head, is ax perfect man without Christ. So that the knitting of our Saviour thereunto should be an addition of that which is too much.” Again, “If the Church be the body of Christ, and of the civil magistrate, it shall have two heads, which being monstrous, is to the great dishonour of Christ and his Church.” Thirdly, “If the Church be planted in a popular estate, then, forasmuch as all govern in common, and all have authority, all shall be heady there, and no body at all; which is another monster.” It might be feared what this birth of so many monsters might portend, but that we know how things natural enough in themselves may seem monstrous through misconceit; which error of mind is indeed a monster, and so the skilful in nature’s mysteries have used to term it. The womb of monsters, if any be, is that troubled understanding, wherein, because things lie confusedly mixed together, what they are it appeareth not.

A Church perfect without Christ, I know not which way a man should imagine; unless there may be either Christianity without Christ, or else a Church without Christianity. If magistrates be heads of the Church, they are of necessity Christians; if Christians, then is their Head Christ.

The adding of Christ the universal Head over all unto the magistrate’s particular headship, is no more superfluous in any church than in other societies it is to be both severally each subject unto some head, and to have also a head general for them all to be subject unto. For so in armies and in  civil corporations we see it fareth. A body politic in such respects is not like to a natural body; in this, more heads than one are superfluous; in that, not.

It is neither monstrous nor as much as uncomely for a church to have different heads: for if Christian churches be in number many, and every of them a body perfect by itself, Christ being Lord and Head over all; why should we judge it a thing more monstrous for one body to have two heads, than one head so many bodies? Him God hath made the supreme Head of the whole Church; the Head, not only of that mystical body which the eye of man is not able to discern, but even of every Christian politic society, of every visible Church in the world.

And whereas, lastly, it is thought so strange, that in popular states a multitude should to itself be both body and head, all this wonderment doth grow from a little oversight, in deeming that the subject wherein headship is to reside, should be evermore some one person; which thing is not necessary. For in at collective body that hath not derived as yet the principality of power into some one or few, the whole of necessity must be head over each part; otherwise it could not possibly have power to make any one certain person head; inasmuch as the very power of making a head belongeth unto headship. These supposed monsters therefore we see are no such giants, that there should need any Hercules to tame them.

[8] For the title or style itself, although the laws of this land have annexed it to the crown, yet so far we would not strive, if so be men were nice and scrupulous in this behalf only, because they do wish that for reverence unto Christ Jesus, the civil magistrate did rather use some other form of speech wherewith to express that sovereign authority which  he lawfully hath over all, both persons and causes of the Church. But I see that hitherto they which condemn utterly the name so applied, do it because they mislike that any such power should be given unto civil governors. The greatest exception that Sir Thomas More took against that title, who suffered death for denial of it was “for that it maketh a lay, or secular person, the head of the state spiritual or ecclesiastical;” as though God himself did not name even Saul the head of all the tribes of Israel; and consequently of that tribe also among the rest, whereunto the state spiritual or ecclesiastical belonged. When the authors of the Centuries reprove it in kings and civil governors, the reason is “istis non competit iste primatus;” “such kind of power is too high for them, they fit it not.” In excuse of Mr. Calvin by whom this realm is condemned of blasphemy for entitling Henry the Eighth Supreme Head of this Church under Christ, a charitable conjecture is made, that he spake by misinformation, and thought we had meant thereby far otherwise than we dog; howbeit, as he professeth utter dislike of that name, so whether the name be used or no, the very power itself which we give unto civil magistrates he much complaineth of,  and testifieth, “That their power over all things was it which had ever wounded him deeply; that unadvised persons had made them too spiritual; that throughout Germany this fault did reign; that in those very parts where Calvin himself was, it prevailed more than were to be wished; that rulers, by imagining themselves so spiritual, have taken away ecclesiastical regiment; that they think they cannot reign unless they abolish all authority of the Church, and be themselves the chief judges, as well in doctrine, as in the whole spiritual regency.” So that in truth the question is, whether the magistrate, by being head in such sense as we term him, do use or exercise any part of that authority, not which belongeth unto Christ, but which other men ought to have.

%Opposition against the difference in kind.
[9]The last difference which we have made between the title of head when we gave it unto Christ, and when we gave it to other governors, is, that the kind of dominion which it importeth is not the same in both. Christ is head as being the fountain of life and ghostly nutriment, the well-spring of spiritual blessings poured into the body of the Church; they heads, as being his principal instruments for the Church’s outward government: He head, as founder of the house; they, as his chiefest overseers.Against this theres is exception especially taken, and our purveyors are herein said to have their provision from the popish shambles: for by  Pighius and Harding, to prove that Christ alone is not head of the Church, this distinction they say is brought, that according to the inward influence of grace, Christ only is head; but according to outward government the being head is a thing common with him toy others.

To raise up falsehoods of old condemned, and to bring that for confirmation of any thing doubtful, which hath already been sufficiently proved an error, and is worthily so taken, this would justly deserve censuring. But shall manifest truth be therefore reproached, because men in some things convicted of manifest untruth have at any time taught or alleged it? If too much eagerness against their adversaries had not made them forget themselves, they might remember where being charged as maintainers of those very things, for which others before them have been condemned of heresy, yet lest the name of any such heretic holding the same which they do should make them odious, they stick not frankly to profess, “that they are not afraid to consent in some points with Jews and Turks.” Which defence, for all that, were a very weak buckler for such as should consent with Jews and Turks, in that which they have been abhorred and hated for of the Church.

But as for this distinction of headship, spiritual and mystical in Jesus Christ, ministerial and outward in others besides Christ; what cause is to dislike either Harding, or Pighius, or any other besides for it? That which they have been reproved for is, not because they did herein utter an untruth, but such a truth as was not sufficient to bear up the cause which they did thereby seek to maintain. By this distinction they have both truly and sufficiently proved that the name of head, importing power of dominion over the Church, might be given unto others besides Christ, without prejudice unto any part of his honour. That which they should have made  manifest was, that the name of Head, importing the power of universal dominion over the whole Church of Christ militant, doth, and that by divine right, appertain unto the Pope of Rome. They did prove it lawful to grant unto others besides Christ the power of headship in a different kind from his; but they should have proved it lawful to challenge, as they did to the bishop of Rome, a power universal in that different kind. Their fault was therefore in exacting wrongfully so great power as they challenged in that kind, and not in making two kinds of power, unless some reason can be shewed for which this distinction of power should be thought erroneous and false.

[10]A little they stir, although in vain, to prove that we cannot with truth make any such distinction of power, whereof the one kind should agree unto Christ only, and the other be further communicated. Thus therefore they argue “If there be no head but Christ, in respect of the spiritual government, there is no head but he in respect of the word, sacraments, and discipline, administered by those whom he hath appointed, forasmuch as that is also his spiritual government.” Their meaning is, that whereas we make two kinds of power, of which two, the one being spiritual is proper unto Christ; the other men are capable of, because it is visible and external: we do amiss altogether, they think, in so distinguishing, forasmuch as the visible and external power of regiment over the Church, is only in relation unto the word, the sacraments, and discipline, administered by such as Christ hath appointed thereunto, and the exercise of this power is also his spiritual government: therefore we do but vainly imagine a visible and external power in the Church differing from his spiritual power.

Such disputes as this do somewhat resemble the wonted practising of well-willers upon their friends in the pangs of death, whose manner is even then to put smoke in their nostrils, and so to fetch them again, although they know it a matter impossible to keep them living. The kind affection which the favourers of this labouring cause bear towards it will  not suffer them to see it die, although by what means they should be able to make it live, they do not see. But they may see that these wrestlings will not help. Can they be ignorant how little it booteth to overcast so clear a light with some mist of ambiguity in the name of spiritual regiment?

To make things therefore so plain that henceforth a child’s capacity may serve rightly to conceive our meaning: we make the spiritual regiment of Christ to be generally that whereby his Church is ruled and governed in things spiritual. Of this general we make two distinct kinds; the one invisibly exercised by Christ himself in his own person; the other outwardly administered by them whom Christ doth allow to be the rulers and guiders of his Church. Touching the former of these two kinds, we teach that Christ in regard thereof is peculiarly termed the Head of the Church of God; neither can any other creature in that sense and meaning be termed head besides him, because it importeth the conduct and government of our souls by the hand of that blessed Spirit wherewith we are sealed and marked, as being peculiarly his. Him only therefore we do acknowledge to be that Lord, which dwelleth, liveth and reigneth in our hearts; him only to be that Head, which giveth salvation and life unto his body; him only to be that fountain, from whence the influence of heavenly grace distilleth, and is derived into all parts, whether the word, or sacraments, or discipline, or whatsoever be the mean whereby it floweth. As for the power of administering these things in the Church of Christ, which power we call the power of order, it is indeed both Spiritual and His; Spiritual, because such duties properly concern the Spirit; His, because by him it was instituted. Howbeit neither spiritual, as that which is inwardly and invisibly exercised; nor his, as that which he himself in person doth exercise.

Again, that power of dominion which is indeed the point of this controversy, and doth also belong to the second kind of spiritual government namely unto that regiment which is  external and visible; this likewise being spiritual in regard of the matter about which it dealeth, and being his, inasmuch as he approveth whatsoever is done by it, must notwithstanding be distinguished also from that power whereby he himself in person administereth the former kind of his own spiritual regiment, because he himself in person doth not administer this. We do not, therefore, vainly imagine, but truly and rightly discern a power external and visible in the Church, exercised by men, and severed in nature from that spiritual power of Christ’s own regiment, which power is termed spiritual, because it worketh secretly, inwardly, and invisibly; his, because none doth or can it personally exercise, either besides or together with him. Sop that him only we may name our Head, in regard of this, and yet, in regard of that other power differing from this, term others also besides him heads, without any contradiction at all.

[11]Which thing may very well serve for answer unto that also which they further allege against the foresaid distinction, namely “that even ins the outward society and assemblies of the Church, where one or two are gathered in his name, either for hearing of the word, or for prayer, or any other church-exercise, our Saviour Christ being in the midst of them as Mediator, must needs be there as head: and if he be there not idle, but doing the office of the head fully, it followeth that even in the outward society and meetings of the Church, no mere man can be called the head of it, seeing that our Saviour Christ doing the whole office of the head himself alone, leaveth nothing to men by doing whereof they may obtain that title.”

Which objection I take as being made for nothing but only to maintain argument. For they are not so far gone as to argue thus in sooth and right good earnest. “God standeth,” saith the Psalmist, “in the midst of gods;” if God be there present, he must undoubtedly be present as a God; if he be there not idle, but doing the office of a God fully, it followeth,  that God himself alone doing the whole office of a God, leaveth nothing in such assemblies unto any other, by doing whereof they may obtain so high a name. The Psalmist therefore hath spoken amiss, and doth ill to call judges gods. Not so; for as God hath his office differing from theirs, and doth fully discharge it even in the very midst of them, so they are not thereby excluded from all kind of duty for which that name should be given unto them also, but in that duty for which it was given them they are encouraged religiously and carefully to order themselves. After the selfsame manner, our Lord and Saviour being in the midst of his Church as Head, is our comfort, without the abridgment of any one duty, for performance whereof others are termed heads in another kind than he is.

[12]If there be of the ancient Fathers which say, “There is but one Head of the Church, Christ; and that the minister which baptizeth cannot be the head of him which is baptized, because Christ is the head of the whole Church: and that Paul could not be the head of the Churches which he planted, because Christ is Head of the whole body” they understand the name of head in such sort as we grant that it is not appliable to any other, no not in relation to the  least part of the whole Church: he which baptizeth, baptizeth into Christ: he which converteth, converteth unto Christ; he which ruleth, ruleth for Christ. The whole Church can have but one to be head as lord and owner of all: wherefore if Christ be Head in that kind, it followeth, that no other can be sop else either to the whole or to any part.

* * * * * *

%To call and dissolve all solemn assemblies about the public affairs of the Church.
V.[1.] The consuls of Rome Polybius affirmeth to have had a kind of regal authority, in that they might call together the senate and people whensoever it pleased them. Seeing therefore the affairs of the Church and Christian religion are public affairs, for the ordering whereof more solemn assemblies sometimes are of as great importance and use, as they are for secular affairs; it seemeth no less an act of supreme authority to call the one than the other. Wherefore amongst sundry others prerogatives of Simon’s dominion over the Jews, this is reckoned as not the least, “that no man might gather any great assembly in the land without him.” For so the manner of Jewish regiment had always been, that whether the cause for which men assembled themselves in peaceable, good, and orderly course, were ecclesiastical or civil, supreme authority should assemble them. David gathered all Israel together unto Jerusalem, when the ark was to be removed; he assembled the sons of Aaron and the Levites. Solomon did the like at such time as the temple was to be dedicated when the Church was to be reformed, Asa in his time did the same: the same upon like occasions done afterwards by Joas, Ezekias, Josias, and others.

[2]The ancient imperial law forbiddeth such assemblies  as the emperor’s authority did not cause to be made. Before emperors became Christiany, the Church had never any synod general; their greatest meetings consisted of bishops and others the gravest in each province. As for the civil governor’s authority, it suffered them only as things not regarded or accounted of, at such times as it did suffer them. So that what right a Christian king hath as touching assemblies of that kind we are not able to judge, till we come unto later times, when religion had won the hearts of the highest powers. Constantine (as Pighiusdoth grant) was not only the first that ever did call any general council together, but even the first that devised the calling of them for consultation about the business of God. After he had once given the example, his successors long time followed the same; insomuch that S. Jerome, to disprove the authority of a synod which was pretended to be general, useth this as a forcible argument “Dic quis imperator hanc synodum jusserit convocari.” Their answer hereunto is no answer, which say, “that emperors did not this without conference had with  bishops:” for to our purpose it is enough, if the clergy alone did it not otherwise than by the leave or appointment of their sovereign lords and kings.

Whereas therefore it is on the contrary side alleged, that Valentinian the elder being requested by Catholic bishops to grant that there might be a synod for the ordering of matters called in question by the Arians, answered, that he being one of the laity might not meddle with such affairs, and thereupon wished, that the priests and bishops, to whom the care of those things belonged, should meet and consult thereof by themselves wheresoever they thought good: we must together with the emperor’s speech weigh the occasion and the drift thereof. Valentinian and Valens, the one a Catholic, the other an Arian, were emperors together: Valens the governor of the east, Valentinian of the west empire. Valentinian therefore taking his journey from the east part into the west, and passing for that intent through Thracia, the bishops theres which held the soundness of Christian belief, because they knew that Valens was their professed enemy, and therefore it the other were once departed out of those quarters, the Catholic cause was like to find small favour, moved presently Valentinian about a council to be  assembled under the countenance of his authority; who by likelihood considering what inconvenience might thereby grow, inasmuch as it could not be but a mean to incense Valens the more against them, refused himself to be author of, or present at any such assembly; and of this his denial gave them a colourable reason, to wit, that he was although an emperor, yet a secular person, and therefore not able in matters of so great obscurity to sit as a competent judge; but, if they which were bishops and learned men did think good to consult thereof together, they might. Whereupon when they could not obtain that which they most desired, yet that which was granted them they took, and forthwith had a council. Valentinian went on towards Rome, they remaining in consultation till Valens which accompanied him returned back; so that now there was no remedy, but either to incur a manifest contempt, or else at the hands even of Valens himself to seek approbation of that they had done. To him, therefore, they became suitors: his answer was short, “Either Arianism, or else exile, which they would;” whereupon their banishment ensued. Let reasonable men therefore now be judges, how much this example of Valentinian doth make against the authority, which we say that sovereign rulers may lawfully have as concerning synods and meetings ecclesiastical.

The clergy, in such wise gathered together, is an ecclesiastical senate, which with us, as in former times the chiefest prelate at his discretion did use to assemble, so afterwards in such considerations as have been before specified, it seemed more meet to annex the said prerogative unto the crown. The plot of reformed discipline not liking hereof so well, taketh order that every former assembly before it break up should itself appoint both the time and place of their after meeting again. But because I find not any thing on that side particularly alleged against us herein, a longer disputation about so plain a cause shall not need.

%Their power in making ecclesiastical laws. 
VI.[1.] The natural subject of power civil all men confess to be the body of the commonwealth: the good or evil estate whereof dependeth so much upon the power of making laws, that in all well settled states, yea though they be monarchies, yet diligent care is evermore had that the commonwealth do not clean resign up herself and make over this power wholly into the hands of any one.
For this cause William, whom we call the Conqueror, making war against England in right of his title to the crown, and knowing that as inheritor thereof he could not lawfully change the laws of the land by himself, for that the English commonwealth had not invested their kings before with the fulness of so great power; therefore he took the style and title of a conqueror. Wherefore, as they themselves cannot choose but grant that the natural subject of power to make laws civil is the commonwealth; so we affirm that in like congruity the true original subject of power also to make church-laws is the whole entire body of that church for which they are made. Equals cannot impose laws and statutes upon their equals. Therefore neither may any one man indifferently impose canons ecclesiastical upon another, nor yet one church upon another. If they go about at any time to do it, they must either shew some commission sufficient for their warrant, or else be justly condemned of presumption in the sight both of God and men. But nature itself doth abundantly authorize the Church to make laws and orders for her children that are within her. For every whole thing, being naturally of greater power than is any part thereof, that which a whole church will appoint may be with reason exacted indifferently of any within the compass of the same church, and so bind all unto strict obedience.

[2]The greatest agents of the bishop of Rome’s inordinate sovereignty strive against no one point with such earnestness as against this, that jurisdiction (and in the name of jurisdiction they also comprehend the power of dominion spiritual)  should be thought originally to be the right of the whole Church; and that no person hath or can have the same, otherwise than derived from the body of the Church.

The reason wherefore they can in no wise brook this opinion is, as friar Soto confesseth because they which make councils above popes do all build upon this ground, and therefore even with teeth and all they that favour the papal throne must hold the contrary. Which thing they do. For, as many as draw the chariot of the pope’s preeminence, the first conclusion which they contend for is The power of jurisdiction ecclesiastical doth not rest derived from Christ immediately into the whole body of the Church, but into the prelacy. Unto the prelacy alone it belongeth; as ours also do imagine, unto the governors of the Church alone it was first given, and doth appertain, even of very divine right, in every church established to make such laws concerning orders and ceremonies as occasion doth require.

[3]Wherein they err, for want of observing as they should, in what manner the power whereof we speak was instituted. One thing it is to ordain a power, and another thing to bestow the same being ordained: or, to appoint the special subject of it, or the person in whom it shall rest. Nature hath appointed that there should be in a civil society power to make laws; but the consent of the people (which are that society) hath instituted the prince’s person to be the subject wherein supremacy of that power shall reside. The act of instituting  such power may and sometimes doth go in time before the act of conferring or bestowing it. And for bestowing it there may be order two ways taken: namely, either by appointing thereunto some certain person, one or many; or else, without any personal determination, and with appointment only of some determinate condition touching the quality of their persons (whosoever they be that shall receive the same), and for the form or manner of taking it.

Now God himself preventeth sometimes these communities, himself nominateth and appointeth sometimes the subject wherein their power shall rest, and by whom either in whole or in part it shall be exercised; which thing he did often in the commonwealth of Israel. Even so Christ having given unto his Church the power whereof we speak, what she doth by her appointed agents, that duty though they discharge, yet is it not theirs peculiarly, but hers; her power it is which they do exercise. But Christ hath sometimes prevented his Church, conferring that power and appointing it unto certain persons himself, which otherwise the Church might have done. Those persons excepted which Christ himself did immediately bestow such power upon, the rest succeeding have not received power as they did, Christ bestowing it upon their persons; but the power which Christ did institute in the Church they from the Church do receive, according to such laws and canons as Christ hath prescribed, and the light of nature or Scripture taught men to institute.

But in truth the whole body of the Church being the first original subject of all mandatory and coercive power within itself, in case a monarch of the world together with his whole kingdom under him receive Christianity, the question is whether the monarch of that commonwealth may without offence or breach of the law of God have and exercise power of dominion ecclesiastical within the compass of his own territories, in such ample sort as the kings of this land may do by the laws thereof.

* * * * * *

1[4.] The case is not like when such assemblies are  gathered together by supreme authority concerning other affairs of the Church, and when they meet about the making of ecclesiastical laws or statutes. For in the one they are only to advise, in the other they ares to decree. The persons which are of the one, the King doth voluntarily assemble, as being in respect of gravity fit to consult withal; them which are of the other he calleth by prescript of law, as having right to be thereunto called. Finally, the one are but themselves, and their sentence hath but the weight of their own judgment; the other represent the whole clergy, and their voices are as much as if all did give personal verdict. Now the question is, Whether the clergy alone so assembled ought to have the whole power of making ecclesiastical laws, or else consent of the laity may thereunto be made necessary, and the King’s assent so necessary, that his sole denial may be of force to stay them from being laws.

%What laws may be made for the affairs of the Church, and to whom the power of making them appertaineth.
[5]If they with whom we dispute were uniform, strong and constant in that which they say, we should not need to trouble ourselves about their persons to whom the power of making laws for the Church belongeth. For they are sometimes very vehement in contention, that from the greatest thing unto the least about the Church, all must needs be immediately from God. And to this they applyth pattern of the ancient tabernacle which God delivered unto Moses, and was therein so exact, that there was not left so much as the least pin for the wit of man to devise in the framing of it.  To this they often apply that strict and severe charge which God so often gave concerning his own law, “Whatsoever I command you, take heed yea do it; thou shalt put nothing thereunto, thou shalt take nothing from it;” nothing, whether it be great or small. Yet sometime bethinking themselves better, they speak as acknowledging that it doth suffice to have received in such sort the principal things from God, and that for other matters the Church hath sufficient authority to make laws. Whereupon they now have made it a question, what persons they are whose right it is to take order for the Church’s affairs, when the institution of any new thing therein is requisite.

Laws may be requisite to be made either concerning things that are only to be known and believed in, or else touching that which is to be done by the Church of God. The law of nature and the law of God are sufficient for declaration in both what belongeth unto each man separately, as his soul is the spouse of Christ, yea so sufficient, that they plainly and fully shew whatsoever God doth require by way of necessary introduction unto the state of everlasting bliss. But as a man liveth joined with others in common society, and belongeth unto the outward politic body of the Church, albeit the same law of nature and scripture have in this respect also made manifest the things that are of greatest necessity; nevertheless, by reason of new occasions still arising which the Church having care of souls must take order for as need requireth, hereby it cometh to pass, that there is and ever will beg great use even of human laws and ordinances, deducted by way of discourse as conclusions from the former divine and natural, serving fori principles thereunto.

No man doubteth, but that for matters of action and practice in the affairs of God, for the manner of divine  service, for order in ecclesiastical proceedings about the regiment of the Church, there may be oftentimes cause very urgent to have laws made: but the reason is not so plain wherefore human laws should appoint men what to believe. Wherefore in this we must note two things: First, That in matter of opinion, the law doth not make that to be truth which before was not, as in matter of action it causeth that to be duty which was not before, but it manifesteth only and giveth men notice of that to be truth, the contrary whereunto they ought not before to have believed. Secondly, That as opinions do cleave to the understanding, and are in heart assented unto, it is not in the power of any human law to command them, because to prescribe what men shall think belongeth only unto God. “Corde creditur, ore fit confessio,” saith the Apostle.As opinions are either fit or inconvenient to be professed, so man’s law hath to determine of them. It may for public unity’s sake require men’s professed assent, or prohibit contradiction to special articles, wherein, as there haply hath been controversy what is true, so the same were like to continue still, not without grievous detriment to a number of souls, except law to remedy that evil should set down a certainty which no man is to gainsay. Wherefore as in regard of divine laws, which the Church receiveth from God, we may unto every man apply those words of Wisdomr in Solomon Conserva, fili mi, præcepta patris tuis: “My son, keep thou thy father’s precepts;” even so concerning the statutes and ordinances which the Church itself maketh, we may add thereunto the words that follow, Et ne dimittas legem matris tuæ, “And forsake not thou thy mother’s law.”

[6]It is undoubtedly a thing even natural, that all free and independent societies should themselves make their own laws, and that this power should belong to the whole, not to any certain part of a politic body, though haply some one part may have greater sway in that action than the rest: which thing being generally fit and expedient in the making  of all laws, we see no cause why to think otherwise in laws concerning the service of God; which in all well-ordered states and commonwealths is the first thing that law hath care to provide for.When we speak of the right which naturally belongeth to a commonwealth, we speak of that which needs must belong to the Church of God. For if the commonwealth be Christian, if the people which are of it do publicly embrace the true religion, this very thing doth make it the Church, as hath been shewed. So that unless the verity and purity of religion do take from them which embrace it, that power wherewith otherwise they are possessed; look, what authority, as touching laws for religion, a commonwealth hath simply, it must of necessity being Christian, have the same as touching laws for Christian religion.

[7]It will be therefore perhaps alleged, that a part of the verity of Christian religion is to hold the power of making ecclesiastical laws a thing appropriated unto the clergy in their synods; and that whatsoever is by their only voices agreed upon, it needeth no further approbation to give unto it the strength of a law: as may plainly appear by the canons of that first most venerable assembly where those things which the Apostles and James had concluded, were afterward published and imposed upon the churches of the Gentiles abroad as laws, the records thereof remaining still in the book of God for a testimony, that the power of making ecclesiastical laws belongeth to the successors of the Apostles the bishops and prelates of the Church of God.


To this we answer, that the council of Jerusalem is no argument for the power of the clergy alone to make laws. For first, there hath not been sithence any council of like authority to that in Jerusalem: secondly, the cause why that was of such authority came by a special accident: thirdly, the reason why other councils being not like unto that in nature, the clergy in them should have no power to make laws by themselves alone, is in truth so forcible, that except some commandment of God to the contrary can be shewed, it ought notwithstanding the foresaid example to prevail.

The decrees of the council of Jerusalem were not as the canons of other ecclesiastical assemblies, human, but very divine ordinances: for which cause the churches were far and wide commanded every where to see them kept, no otherwise than if Christ himself had personally on earth been the author of them.

The cause why that council was of so great authority and credit above all others which have been sithence, is expressed in those words of principal observation “Unto the Holy Ghost and to us it hath seemed good:” which form of speech, though other councils have likewise used, yet neither could they themselves mean, nor may we so understand them, as if both were in equal sort assisted with the power of the Holy Ghost; but the later had the favour of that general assistance and presence which Christ doth promise unto all his, according to the quality of their several estates and callings; the former, that grace of special, miraculous, rare, and extraordinary illumination, in relation whereunto the Apostle, comparing the Old Testament and the New together, termeth the one a Testament of the letter, for that God delivered it written in stone, the other a Testament of the Spirit, because God imprinted it in the hearts and declared it by the tongues of his chosen Apostles through the power of the Holy Ghost, framing both their conceits and speeches in most divine and incomprehensible manner. Wherefore inasmuch as the council  of Jerusalem did chance to consist of men so enlightened, it had authority greater than were meet for any other council besides to challenge, wherein non such kind of persons are.

[8]As now the state of the Church doth stand, kings being not then that which now they are, and the clergy not now that which then they were: till it be proved that some special law of Christ hath for ever annexed unto the clergy alone the power to make ecclesiastical laws, we are to hold it a thing most consonant with equity and reason, that no ecclesiastical law be made in a Christian commonwealth, without consent as well of the laity as of the clergy, but least of all without consent of the highest power.

For of this thing no man doubteth, namely, that in all societies, companies, and corporations, what severally each shall be bound unto, it must be with all their assents ratified. Against all equity it were that a man should suffer detriment at the hands of men, for not observing that which he never did either by himself or by others, mediately or immediately, agree unto; much more that a king should constrain all others unto the strict observation of any such human ordinance as passeth without his own approbation. In this case therefore especially that vulgar axiom is of force “Quod omnes tangit ab omnibus tractari et approbari debet.” Whereupon Pope Nicholas, although otherwise not admitting lay-persons, no not emperors themselves to be present at synods, doth notwithstanding seem to allow of their presence when matters of faith are determined, whereunto all men must  stand bound “Ubinam legistis imperatores, antecessores vestros, synodalibus conventibus interfuisse; nisi forsitan in quibus de fide tractatum est, quæ universalis est, quæ omnibus communis estq, quæ non solum ad clericos, verum etiam ad laicos et omnes pertinet Christianos?” A law, be it civil or ecclesiastical, is as a public obligation, wherein seeing that the whole standeth charged, no reason it should pass without his privity and will, whom principally the whole doth depend upon. “Sicut laici jurisdictionem clericorum perturbare, ita clerici jurisdictionem laicorum non debent imminueres;” saith Innocent “As the laity should not hinder the clergy’s jurisdiction, so neither is it reason that the laity’s right should be abridged by the clergy.” But were it so that the clergy alone might give laws unto all the rest, forasmuch as every estate doth desire to enlarge the bounds of their own liberties, is it not easy to see how injurious this might prove unto men of other condition? Peace and justice are maintained by preserving unto every order their rights, and by keeping all estates as it were in an even balance. Which thing is no way better done, than if the king, their common parent, whose care is presumed to extend most indifferently over all, do bear the chiefest sway in the making of laws which all must be ordered by.


[9]Wherefore, of them which in this point attribute most to the clergy, I would demand what evidence there is, which way it may clearly be shewed, that, in ancient kingdoms Christian, any canon devised by the clergy alone in their synods, whether provincial, national, or general, hath by mere force of their agreement taken place as a law, making all men constrainable to be obedient thereunto, without any other approbation from the king before or afterwards required in that behalf? But what speak we of ancient kingdoms, when at this day, even in the papacy itself, the very Tridentineb council hath not every where as yet obtained to have in all points the strength of ecclesiastical law. Did not Philip, king of Spain, publishing that council in the Low Countries, add thereunto an express clause of special provision, that the same should in no wise prejudice, hurt, or diminish any kind of privilege which the king or his vassals aforetime had enjoyed, either touching possessory judgments of ecclesiastical livings, or concerning nominations thereunto, or belonging to whatsoever rights they had else in such affairs? If therefore the king’s exception taken against some part of the canons contained in that council, were a sufficient  bar to make them of none effect within his territories; it followeth that the like exception against any other part had been also of like efficacy, and so consequently that no part thereof had obtained the strength of a law, if he which excepted against a part had so done against the whole: as, what reason was there but that the same authority which limited might quite and clean have refused that council? Whoso alloweth the said act of the Catholic King for good and lawful, must grant that the canons even of general councils have but the force of wise men’s opinions concerning that whereof they treat, till they be publicly assented unto, where they are to take place as laws; and that, in giving such public assent, as maketh a Christian kingdom subject unto those laws, the king’s authority is the chiefest. That which an University of men, a Company or Corporation doth without consent of their Rector, is as nothing. Except therefore we make the king’s authority over the clergy less in the greatest things, than the power of the meanest governor is in all things over the college or society which is under him; how should we think it a matter decent, that the clergy should impose laws, the supreme governor’s assent not asked?

[10]There are which wonder that we should count any statute a law, which the high court of parliament in England hath established about the matter of church regiment; the prince and court of parliament having, as they suppose, no more lawful means to give order to the Church and clergy in these things, than they have to make laws for the hierarchies of angels in heaven that the parliament being a mere temporal court, can neither by the law of nature, nor of God,  have competent power to define of such matters that supremacy of power in this kind cannot belong unto kings, as kings, because pagan emperors, whose princely power was notwithstanding true sovereignty, never challenged thus much over theChurch: that power, in this kind, cannot be the right of any earthly crown, prince, or state, in that they be Christian, forasmuch as if they be Christians, they all owe subjection unto the pastors of their souls that the prince therefore not having it himself cannot communicate it unto the parliament, and consequently cannot make laws, hear, or determine of the Church’s regiment by himself, parliament, or any other court in such sorts subjected unto him.

[11]The parliament of England together with the convocation annexed thereunto, is that whereupon the very essence of all government within this kingdom doth depend; it is even the body of the whole realm; it consisteth of the king, and of all that within the land are subject unto him: for they all are there present, either in person or by such as they  voluntarily have derived their very personal right unto. The parliament is a court not so merely temporal as if it might meddle with nothing but only leather and wool.Those days of Queen Mary are not yet forgotten, wherein the realm did submit itself unto the legate of Pope Julius: at which time had they been persuaded as this man seemeth now to be, had they thought that there is no more force in laws made by parliament concerning the Church affairs, than if men shall take upon them to make orders for the hierarchies of angels in heaven, they might have taken all former statutes in that kind as cancelled, and by reason of nullity abrogated in themselves. What need was there that they should bargain with the cardinal, and purchase their pardon by promise made beforehand, that what laws they had made, assented unto, or executed against the bishop of Rome’s supremacy, the same they would in that present parliament effectually abrogate and repeal? Had they power to repeal laws made, and none to make laws concerning the regiment of the Church?

Again, when they had by suit obtained his confirmation for such foundations of bishoprics, cathedral churches, hospitals, colleges, and schools; for such marriages before made, for such institutions unto livings ecclesiastical, and for all such judicial processes, as having been ordered according to laws before in force, but contrary to the canons and orders of the church of Rome, were in that respect thought defective; although the cardinal in his letters of dispensation did give validity unto those acts, even apostolicæ firmitatis robur, “the very strength of apostolical solidity;” what had all this been without those  graved authentical words “Be it enacted by the authority of this present parliament, that all and singular articles and clauses contained in the said dispensation, shall remain and be reputed and taken to all intents and constructions in the laws of this realm, lawful, good and effectual to be alleged and pleaded in all courts ecclesiastical and temporal, for good and sufficient matter either for the plaintiff or defendant, without any allegation or objection to be made against the validity of them by pretence of any general council, canon, or decree to the contrary.” Somewhat belike they thought there was in this mere temporal court, without which the pope’s own mere ecclesiastical legate’s dispensation had taken small effect in the Church of England; neither did they or the cardinal himself, as then, imagine any thing committed against the law of nature or of God, because they took order for the Church’s affairs, and that even in the court of parliament.

The most natural and religious course in making of laws is, that the matter of them be taken from the judgment of the wisest in those things which they are to concern. In matters of God, to set down a form of public prayer, a solemn confession of the articles of Christian faith, rites and ceremonies meet for the exercise of religion; it were unnatural not to think the pastors and bishops of our souls a great deal more fit, than men of secular trades and callings: howbeit, when all which the wisdom of all sorts can do is done for devising of laws in the Church, it is the general consent of all that giveth them the form and vigour of laws, without which they could be no more unto us than the counsels of physicians to the sick: well might they seem as wholesome admonitions and instructions, but laws could they never be without consent of the whole Church, which is the only thing that bindeth each member of the Church, to be guided by them. Whereunto both nature and the practice of the Church of God set down in Scripture, is found every way so fully consonant, that God himself would not impose, no not his own laws upon his people  by the hand of Moses, without their free and open consent. Wherefore to define and determine even of the church’s affairs by way of assent and approbation, as laws are defined of in that right of power, which doth give them the force of laws; thus to define of our own church’s regiment, the parliament of England hath competent authority.

Touching the supremacy of power which our kings have in this case of making laws, it resteth principally in the strength of a negative voice; which not to give them, were to deny them that without which they were but kings by mere title, and not in exercise of dominion. Be it in states of regiment popular, aristocratical, or regal, principality resteth in that person, or those persons, unto whom is given the right of excluding any kind of law whatsoever it be before establishment. This doth belong unto kings, as kings; pagan emperors even Nero himself had not less, but much more than this in the laws of his own empire. That he challenged not any interest in giving voice in the laws of the church, I hope no man will so construe, as if the cause were conscience, and fear to encroach upon the Apostles’ right.

If then it be demanded by what right from Constantine downward, the Christian emperors did so far intermeddle with the church’s affairs, either we must herein condemn them utterly, as being over presumptuously bold, or else judge that by a law which is termed Regia, that is to say royals, the people having derived into the emperor their whole power for making of laws, and by that mean his edicts being made laws what matter soever they did concern, as imperial dignity endowed them with competent authority and power to make laws for religion, so they were taught by Christianity to use their power, being Christians, unto the benefit of the Church of Christ. Was there any Christian bishop in the world which did then judge this repugnant unto the dutiful subjection which Christians do owe to the pastors of their souls? to whom,  in respect of their sacred order, it is not by us, neither may be denied, that kings and princes are as much as the very meanest that liveth under them, bound in conscience to shew themselves gladly and willingly obedient, receiving the seals of salvation, the blessed sacraments, at their hands, as at the hands of our Lord Jesus Christ, with all reverence, not disdaining to be taught and admonished by them, not withholding from them as much as the least part of their due and decent honour. All which, for any thing that hath been alleged, may stand very well without resignation of supremacy of power in making laws, even laws concerning the most spiritual affairs of the Church.

Which laws being made amongst us, are not by any of us so taken or interpreted, as if they did receive their force from power which the prince doth communicate unto the parliament, or to any other court under him, but from power which the whole body of this realm being naturally possessed with, hath by free and deliberate assent derived unto him that ruleth over them, so far forth as hath been declared. So that our laws made concerning religion, do take originally their essence from the power of the whole realm and church of England, than which nothing can be more consonant unto the law of nature and the will of our Lord Jesus Christ.

[12]To let these go, and to return to our own men; “Ecclesiastical governors,” they say “may not meddle with the making of civil laws, and of laws for the commonwealth; nor the civil magistrate, high or low, with making of orders for the Church.” It seemeth unto me very strange, that those men which are in no cause more vehement and fierce, than where they plead that ecclesiastical persons may not κυριεύειν, be lords, should hold that the power of making ecclesiastical laws, which thing is of all other most proper unto  dominion, belongeth to none but persons ecclesiastical only. Their oversight groweth herein for want of exact observation, what it is to make a law. Tully, speaking of the law of nature, saith, “That thereof God himself was inventor, disceptator, lator, the deviser, the discusser, the deliverer” wherein he plainly alludeth unto the chiefest parts which then did appertain to this public action. For when laws were made, the first thing was to have them devised; the second, to sift them with as much exactness of judgment as any way might be used; the next, by solemn voice of sovereign authority to pass them, and give them the force of laws. It cannot in any reason seem otherwise than most fit, that unto ecclesiastical persons the care of devising ecclesiastical laws be committed, even as the care of civil unto them which are in those affairs most skilful. This taketh not away from ecclesiastical persons all right of giving voice with others, when civil laws are proposed for regiment of that commonwealth, whereof themselves, (howsoever now the world would have them annihilated,) are notwithstanding as yet a part: much less doth it cut off that part of the power of princes, whereby, as they claim, so we know no reasonable cause wherefore we may not grant them, without offence to Almighty God, so much authority in making of all manner of laws within their own dominions, that neither civil nor ecclesiastical do pass without their royal assent. In devising and discussing of laws, wisdom is specially required: but that which establisheth and maketh them, is power, even power of dominion; the chiefty whereof, amongst us, resteth in the person of the king. Is there any law of Christ’s which forbiddeth kings and rulers of the earth to have such sovereign and supreme power in the making of laws, either civil or ecclesiastical? If there be, our controversy hath an end.

[13]Christ in his church hath not appointed any such law concerning temporal power, as God did of old deliver unto the commonwealth of Israel; but leaving that to be at the  world’s free choice, his chiefest care was that the spiritual law of the Gospel might be published far and wide.

They that received the law of Christ, were for a long time people scattered in sundry kingdoms, Christianity not exempting them from the laws which they had been subject unto, saving only in such cases as those laws did enjoin that which the religion of Christ forbade. Hereupon grew their manifold persecutions throughout all places where they lived: as oft as it thus came to pass, there was no possibility that the emperors and kings under whom they lived, should meddle any whit at all with making laws for the Church. From Christ therefore having received power, who doubteth, but as they did, so they might bind themselves to such orders as seemed fittest for the maintenance of their religion, without the leave of high or low in the commonwealth; forasmuch as in religion it was divided utterly from them, and they from it?

But when the mightiest began to like of the Christian faith; by their means whole free states and kingdoms became obedient unto Christ. Now the question is, whether kings by embracing Christianity do therein receive any such law, as taketh from them the weightiest part of that sovereignty which they had even when they were heathens: whether being infidels they might do more in causes of religion, than now they can by the law of God, being true believers. For whereas in regal states, the king or supreme head of the commonwealth, had before Christianity a supreme stroke in the making of laws for religion: he must by embracing Christian religion utterly thereof deprive himself, and in such causes become ac subject to his own subjects, having even within his own dominions them whose commandment he must obey; unless this power be placed in the hand of some foreign spiritual potentate: so that either a foreign or domestical commander upon earth he must needs admit, more now than before he had, and that in the chiefest things whereupon commonwealths do stand. But apparent it is unto all men which are not strangers in the doctrine of Jesus Christ, that no state in the world receiving Christianity is by any law therein contained  bound to resign the power which they lawfully held before: but over what persons and in what causes soever the same hath been in force, it may so remain and continue still. That which as kings they might do in matter of religion, and did in matter of false religion, being idolaters or superstitious kings, the same they are now even in every respect as fully authorized to do in all affairs pertinent unto the state of true Christiano religion.

[14]And concerning their supreme power of making laws for all persons in all causes to be guided by, it is not to be let pass, that the head enemies of this headship are constrained to acknowledgeth king endowed even with this very power, so that he may and ought to exercise the same, taking order for the Church and her affairs of what nature or kind soever, in case of necessity: as when there is no lawful ministry, which they interpret then to be (and this surely is a point very markable), whensoever the ministry is wicked. A wicked ministry no lawful ministry; and in such sort no lawful ministry, that what doth belong to them as ministers by right of their calling, the same to be annihilated in respect of their bad qualities; their wickedness in itself a deprivation of right to deal in the affairs of the Church, and a warrant for others to deal in them which are held to be of a clean other society, the members whereof have been before so peremptorily for ever excluded from power of dealing with the affairs of the Church.


They which have once throughly learned this lesson, will quickly be capable perhaps of another equivalent unto it. For if the wickedness of the ministry transfer their right unto the king; in case the king be as wicked as they, to whom then shall the right descend? There is no remedy, all must come by devolution at the length, even as the family of Brown will have it unto the godly among the people; for confusion unto the wise and thee great, the poor and the simple, some Knipperdoling with his retinue, must take the work of the Lord in hand; and the making of church laws and orders  must prove to be their right in the end. If not for love of the truth, yet for very shame of so gross absurdities, let these contentions and shifting  fancies be abandoned.

The cause which moved them for a time to hold a wicked ministry no lawful ministry; and in this defect of a lawful ministry, kings authorized to make laws and orders for the affairs of the Church, till the Church be well established, is surely this: First, they see that whereas the continual dealing of the kings of Israel in the affairs of the Church doth make  now very strongly against them, the burden thereof they shall in time well enough shake off, if it may be obtained that it is for kings lawful indeed to follow those holy examples, howbeit no longer than during the foresaid case of necessity, while the wickedness, and in respect thereof the unlawfulness of the ministry doth continue. Secondly, they perceive right well, that unless they should yield authority unto kings in case of such supposed necessity, the discipline they urge were clean excluded, as long as the clergy of England doth thereunto remain opposite. To open therefore a door for her entrance, there is no remedy but the tenet must be this: that now when the ministry of England is universally wicked, and, in that respect, hath lost all authority, and is become no lawful ministry, no such ministry as hath the right which otherwise should belong unto them, if they were virtuous and godly as their adversaries are; in this necessity the king may do somewhat for the church: that which we do imply in the name of headship, he may both have and exercise till they be entered which will disburde and ease him of it; till they come, the king is licensed to hold that power which we call headship. But what afterwards? In a church [well?] ordered, that which the supreme magistrate hath is “to see that the laws of God touching his worship, and touching all matters and orders of the Church, be executed and duly observed; to see that every ecclesiastical person do that office whereunto he is appointed; to punish those that fail in their office.” In a word, (that which Allen himself acknowledgeth) unto the earthly power which God hath given him it doth belong to defend the laws of the Church, to cause them to be executed, and to punish the transgressors of the same.

On all sides therefore it is confessed, that to the king belongeth power of maintaining laws made for church regiment,  and of causing them to be observed; but principality of power in making them, which is the thing that we attribute unto kings, this both the one sort and the other do withstand although not both in such sort but that still it is granted by the one that albeit ecclesiastical councils consisting of church officers did frame the laws whereby the church affairs were ordered in ancient times, yet no canon, no not of any council, had the force of a law in the Church, unless it were ratified and confirmed by the emperor being Christian. Seeing therefore it is acknowledged that it was then the manner of the emperor to confirm the ordinances which were made by the ministers, which is as much in effect to say that the emperor  had in church ordinances a voice negative;—and that without his confirmation they had not the strength of public ordinances;—why are we condemned as giving more unto kings than the Church did in those times, we giving them no more but that supreme power which the emperor did then exercise with much larger scope than at this day any Christian king either doth or possibly can use it over the Church?

%The Prince’s power in the advancement of Bishops unto the rooms of prelacy.
VII. Touching the advancement of prelates unto their rooms by the king; whereas it seemeth in the eyes of many a thing very strange that prelates, the officers of God’s own sanctuary, than which nothing is more sacred, should be made by persons secular; there are that will not have kings be altogether of the laity, but to participate that sanctified power which God hath endued his clergy with, and that in such respect they are anointed with oil. A shift vain and needless. For as much as, if we speak properly, we cannot say kings do make, but that they only do place, bishops. For in a bishop there are these three things to be considered; the power whereby he is distinguished from other pastors; the special portion of the clergy and peopled over whom he is to exercise that bishoply power; and the place of his seat or throne, together with the profits, preeminences, honours thereunto belonging. The first every bishop hath by consecration; the second his election investeth him with; the third he receiveth of the king alone.

[2]With consecration the king intermeddleth not further than only by his letters to present such an elect bishop as shall be consecrated. Seeing therefore that none but bishops do consecrate, it followeth that none but they only do give unto every bishop his being. The manner of uniting bishops as heads, unto the flock and clergy under them, hath often  altered. For, if some be not deceived, this thing was sometime done even without any election at all. At the first (saith he to whom the name of Ambrose is given1) the first created in the college of presbyters was still the bishop. He dying, the next senior did succeed him. “Sed quia cœperunt sequentes presbyteri indigni inveniri ad primatus tenendos immutata est ratio, prospiciente concilio; ut non ordo sed meritum crearet episcopum multorum sacerdotum judicioh constitutum, ne indignus temere usurparet et esset multis scandalum.”

In elections at the beginning the clergy and the people both had to do, although not both after one sort. The people gave their testimony, and shewed their affection, either of desire or dislike, concerning the party which was to be chosen. But the choice was wholly in the sacred college of presbyters. Hereunto it is that those usual speeches of the ancient do commonly allude: as when Pontius concerning S. Cyprian’s election saith, he was chosen “judicio Dei et populi favore,” “by the judgment of God and favour of the people” the one branch alluding to the voices of the ecclesiastical senate which with religious sincerity choose him, the other to the people’s affection, who earnestly desired to have him chosen their bishop.

Again, Leo “Nulla ratio sinit, ut inter episcopos habeantur qui nec a clericis sunt electi nec a plebibus expetiti.” “No reason doth grant that they should be reckoned amongst bishops, whom neither clergy hath elected nor laity coveted.” In like sort Honorius “Let him only be established bishop  in the see of Rome whom Divine judgment and universal consent hath chosen.”

[3]That difference, which is between the form of electing bishops at this day with us, and that which was usual in former ages, riseth from the ground of that right which the kings of this land do claim in furnishing the place where bishops, elected and consecrated, are to reside as bishops. For considering the huge charges which the ancient famous princes of this land have been at, as well in erecting episcopal sees, as also in endowing them with ample possessions; sure of their religious magnificence and bounty we cannot think but to have been most deservedly honoured with those royal prerogatives, [of] taking the benefit which groweth out of them in their vacancy, and of advancing alone unto such dignities what persons they judge most fit for the same. A thing over and besides even therefore the more reasonable; for that, as the king most justly hath preeminence to make lords temporal which are not such by right of birth, so the like preeminence of bestowing where pleaseth him the honour of spiritual nobility also, cannot seem hard, bishops being peers of the realm, and by law itself so reckoned.

Now, whether we grant so much unto kings in this respect, or in the former consideration whereupon the laws have annexed it unto the crown it must of necessity being  granted, both make void whatsoever interest the people aforetime hath had towards the choice of their own bishop, and also restrain the very act of canonical election usually made by the dean and chapter; as with us in such sort it doth, that they neither can proceed unto any election till leave be granted nor elect any person but that is named unto them. If they might do the one, it would be in them to defeat the king of his profits; if the other, then were the king’s preeminences of granting those dignities nothing. And therefore, were it not for certain canons requiring canonical election to be before consecration I see no cause but that the king’s letters patents alone might suffice well enough  toq that purpose, as by law they do in case those electors should happen not to satisfy the king’s pleasure. Their election is now but a matter of form: it is the king’s mere grant which placeth, and the bishop’s consecration which maketh, bishops.

[4]Neither do the kings of this land use herein any other than such prerogatives as foreign nations have been accustomed unto.

About the year of our Lord 425 pope Boniface solicited most earnestly the emperor Honorius to take some order that the bishops of Rome might be created without ambitious seeking of the place. A needless petition, if so be the emperor had no right at all in the placing of bishops there. But from the days of Justinian the emperor, about the year 55 Onuphrius himself doth grant that no man was bishop in the see of Rome whom first the emperor by his letters patents did not license to be consecrated. Till in Benedict’s time it pleased the emperor to forego that right; which afterwards  was restored to Charles with augmentation and continued in his successors till such time as Hildebrand took it from Henry IV and ever since the cardinals have held it as at this day.

Had not the right of giving them belonged to the emperors of Rome within the compass of their dominions, what needed pope Leo the fourth to trouble Lotharius and Lodowick with those his letters whereby, having done them to understand that the church called Reatina was without a bishop, he maketh suit that one Colonus might have the room, or, if that were otherwise disposed of, his next request was, “Tusculanam ecclesiam, quæ viduata existit, illi vestra serenitas dignetur concedere, ut consecratus a nostro præsulatu Deo Omnipotenti vestroque imperio grates  peragere valeat.” “May it please your clemencies to grant unto him the church of Tusculum now likewise void; that by our episcopal authority he being after consecrated may be to Almighty God and your highness therefore thankful.”

[5]Touching other bishopricks, extant there is a very short but a plain discourse written almost 500 years since, by occasion of that miserable contention raised between the emperor Henry IVth and pope Hildebrand, named otherwise Gregory the VIIth, not, as Platina would bear men in hand, for that the bishop of Rome would not brook the emperor’s simoniacal dealings, but because the right, which Christian kings and emperors had to invest bishops, hindered so much his ambitious designments, that nothing could detain him from attempting to wrest it violently out of their hands.

This treatise I mention, for that it shortly comprehendeth not only the fore-alleged right of the emperor of Rome acknowledged by six several popes even with bitter execration against whomsoever of their successors that should by word or deed at any time go about to infringe the same, but also further these other specialties appertaining thereunto: First that the bishops likewise of Spain, England, Scotland, Hungary, had by ancient institution always been invested by their kings, without opposition or disturbance. Secondly, that such  was their royal interest, partly for that they were founders of bishopricks, partly because they undertook the defence of them against all ravenous oppressions and wrongs, partly in as much that it was not safe that rooms of so great power and consequence in their estate should without their appointment be held by any under them. And therefore that bishops even then did homage and took their oaths of fealty unto the kings which invested them. Thirdly that what solemnity or ceremony kings do use in this action it skilleth not, as namely whether they do it by word, or by precept set down in writing, or by delivery of a staff and a ring, or by any other means whatsoever, only that use and custom would, to avoid all offence, be kept. Some base canonists there are, which contend that neither kings nor emperors had ever any right hereunto, saving only by the pope’s either grant or toleration. Whereupon not to spend any further labour, we leave their folly to be controlled by men of more ingenuity and judgment even among themselves,  Duarenus Papon Choppinus Ægidius Magister Arnulphus Rusæus Costlius Philippus Probus and the rest, by whom the right of Christian kings and princes herein is maintained to be such as the bishop of Rome cannot lawfully either withdraw or abridge or hinder.

But of this thing there is with us no question, although with them there be. The laws and customs of the realm approving such regalities, in case no reason thereof did appear, yet are they hereby abundantly warranted unto us, except some law of God or nature to the contrary could be shewed. How much more, when they have been every where thought so reasonable that Christian kings throughout the world use and exercise, if not altogether, yet surely with very little odds the same. So far that Gregory the Tenth forbidding such regalities to be newly begun where they were not in former  times, if any do claim those rights from the first foundation of churches, or by ancient custom, of them he only requireth that neither they nor their agents damnify the Church of God by using the said prerogatives.

[6]Now as there is no doubt but the church of England by this means is much eased of some inconveniences, so likewise a special care there is requisite to be had, that other evils no less dangerous may not grow. By the history of former times it doth appear, that when the freedom of elections was most large, men’s dealings and proceedings therein were not the least faulty.

Of the people S. Jerome complaineth that their judgments many times went much awry, and that in allowing of their bishops every man favoured his own quality; every one’s desire was, not so much to be under the regiment of good and virtuous men, as of them which were like himself. What man is there whom it doth not exceedingly grieve to read the tumults, tragedies, and schisms, which were raised by occasion of the clergy at such time as, diverse of them standing for some one place, there was not any kind of practice, though never so unhonest or vile, left unassayed whereby men might supplant their competitors and the one side foil the other. Sidonius, speaking of a bishoprick void in his time “The decease of the former bishop,” saith he, “was an alarum to such as would labour for the room: whereupon the people, forthwith betaking themselves unto parts, storm on each side: few there are that make suit for the advancement of any other man; many who not only offer, but enforce themselves. All things light, variable, counterfeit: what should I say? I see not any thing plain and open but impudence only.”


In the church of Constantinople about the election of S. Chrysostom by reason that some strove mightily for him and some for Nectarius, the troubles growing had not been small, but that Arcadius the emperor interposed himself: even as at Rome the emperor Valentinian, whose forces were hardly able to establish Damasus bishop, and to compose the strife between him and his competitor Ursicinus, about whose election the blood of a hundred and thirty-seven was already shed. Where things did not break out into so manifest and open flames, yet between them which obtained the place and such as before withstood their promotion, that secret heart burning often grew which could not afterwards be easily slaked. Insomuch that Pontius doth note it as a rare point of virtue in Cyprian, that whereas some were against his election, he notwithstanding dealt ever after in most friendly manner with them, all men wondering that so good a memory was so easily able to forget.

[7]These and other the like hurts accustomed to grow from ancient elections we do not feel. Howbeit, lest the Church in more hidden sort should sustain even as grievous detriment by that order which is now of force, we are most humbly to crave at the hands of ourq sovereign kings and governors, the highest patrons which this church of Christ hath on earth, that it would please them to be advertised thus much.


Albeit these things which have been sometimes done by any sort may afterwards appertain unto others, and so the kind of agents vary as occasions daily growing shall require; yet sundry unremovable and unchangeable burthens of duty there are annexed unto every kind of public action, which burthens in this case princes must know themselves to stand now charged with in God’s sight no less than the people and the clergy, when the power of electing their prelates did rest fully and wholly in them. A fault it had been if they should in choice have preferred anywhom desert of most holy life and the gift of divine wisdom did not commend; a fault, if they had permitted longthe rooms of the principal pastors of God to continue void; not to preserve the church patrimony as good to each successor as any predecessor did enjoy the same, had been in them a most odious and grievous fault. Simply good and evil do not lose their nature: that which was, is the one or the other, whatsoever the subject of either be. The faults mentioned are in kings by so much greater, for that in what churches they exercise those regalities whereof we do now entreat, the same churches they have received into their special care and custody, with no less effectual obligation of conscience than the tutor standeth bound in for the person and state of that pupil whom he hath solemnly taken upon him to protect and keep. All power is given unto edification, none to the overthrow and destruction of the Church.

Concerning therefore the first branch of spiritual dominion  thus much may suffice; seeing that they with whom we contend do not directly oppose themselves against regalities, but only so far forth as generally they hold that no church-dignity should be granted without consent of the common people, and that there ought not to be in the Church of Christ any episcopal rooms for princes to use their regalities in. Of both which questions we have sufficiently spoken before.

%Their power to command all persons, and to be over all causes ecclesiastical, whatsoever.
VIII. Touching the king’s supereminent authority in commanding, and in judging of causes ecclesiastical; First, to explain therein our meaning, It hath been taken as if we did hold, that kings may prescribe what themselves think good to be done in the service of God; how the word shall be taught, how sacraments administered: that kings may personally sit in the consistory where bishopsy do, hearing and determining what causes soever do appertain unto those courts: that kings and queens in their own proper persons are by judicial sentence to decide the questions which rise about matters of faith and Christian religion: that kings may excommunicate: finally, that kings may do whatsoever is incident unto the office and duty of an ecclesiastical judge. Which opinion because we count as absurd as they who have fathered the same upon us, we do them to wit that thus our meaning is, and no otherwise: There is not within this realm any ecclesiastical officer, that may by the authority of his own place command universally throughout the king’s dominions; but they of his people whom one may command, are to another’s commandment unsubject: only the king’s royal power is of so large compass, that no man commanded by him according to order of law, can plead himself to be without the bounds and limits of that authority; I say, according to order of law, because with us the highest have thereunto so tied themselves, that otherwise than so they take not upon them to command any.

[2]And, that kings should be in such sort supreme commanders over all men, we hold it requisite, as well for the  ordering of spiritual as of civil affairs; inasmuch as without universal authority in this kind, they should not be able when need is, to do as virtuous kings have done. Joas purposing to renew the “house of the Lord, assembled the Priests and Levites, and when they were together, gave them their charge, saying, Go out unto the cities of Judah, and gather of all Israel money to repair the house of your God from year to year, and haste the things: but the Levites hasted not. Therefore the king called Jehoiada, the chief, and said unto him, Why hast thou not required of the Levites to bring in out of Judah and Jerusalem, the tax of Moses, the servant of the Lord, and of the congregation of Israel, for the tabernacle of the testimony? For wicked Athaliah and her children brake up the house of God, and all the things that were dedicated for the house of the Lord did they bestow upon Baalimoo. Therefore the king commanded, and they made a chest, and set it at the gate of the house of the Lord without; and they made a proclamation through Judah and Jerusalem, to bring unto the Lord the tax of Moses the servant of God, laid upon Israel in the wilderness.” Could either he have done this, or after himEzechias the like concerning the celebration of the passover, but that all sorts of men in all things did owe unto those their sovereign rulers the same obedience which sometimr Josua had them by solemn vow and promise bound unto “Whosoever shall rebel against thy commandments, and will not obey thy words in all that thou commandest him, let him be put to death; only be strong and of a good courage.”

[3]Furthermore, judgment ecclesiastical we say is necessary for decision of controversies rising between man and man, and for correction of faults committed in the affairs of God; unto the due execution whereof there are three things necessary, laws, judges, and a supreme governor of judgments.


What courts there shall be, and what causes shall belong to each court, and what judges shall determine of every cause, and what order in all judgments shall be kept; of these things the laws have sufficiently disposed: so that his duty which sitteth in every such court is to judge, not of, but after, the said lawsb: “Imprimisillud observare debet judex, ne aliter judicet quam legibus, autc constitutionibus, aut moribus proditum estd.” Which laws (for we mean the positive laws of our own realm concerning ecclesiastical affairs) if they otherwise dispose of any such thing than according to the law of reason and of God, we must both acknowledge them to be amiss, and endeavour to have them reformed: but touching that point what may be objected shall after appear.

Our judges in causes ecclesiastical are either ordinary or commissionary: ordinary, those whom we term Ordinaries; and such by the laws of this land are none but prelates only, whose power to do that which they do is in themselves, and belongeth unto the nature of their ecclesiastical calling. In spiritual causes, a lay person may be no ordinary; a commissionary judge there is no let but that he may be: and that our laws do evermore refer the ordinary judgment of spiritual causes unto spiritual persons, such as are termed Ordinaries, no man which knoweth any thing in the practice of this realm can easily be ignorant.

[4]Now, besides them which are authorized to judge in several territories, therei is required an universal power which reacheth over all, importing supreme authority of government over all courts, all judges, all causes; the operation of which power is as well to strengthen, maintain and uphold particular jurisdictions, which haply might else be of small effect; as also to remedy that which they are not able to help, and to redress that wherein they at any time do otherwise than they ought to do. This power being sometime in the bishop of Rome, who by sinister practices had drawn it into  his hands, was for just considerations by public consent annexed unto the king’s royal seat and crown. From thence the authors of reformation would translate it into their national assemblies orl synods; which synods are the only help which they think lawful to use against such evils in the Church as particular jurisdictions are not sufficient to redress. In which case our laws have provided that the king’s supereminent authority and power shall serve. As namely, when the whole ecclesiastical state, or the principal persons therein, do need visitation and reformation; when, in any part of the Church, errors, heresies, schisms, abuses, offences, contempts, enormities, are grown, which men in their several jurisdictions either do not or cannot help: whatsoever any spiritual authority or power (such as legates from the see of Rome did sometimes exercise) hath done or might heretofore have done for the remedy of those evils in lawful sort (that is to say, without violation of the law of God or nature in the deed done), as much in every degree our laws have fully granted that the king for ever may do, not only by setting ecclesiastical synods on work, that the thing may be their act and the king their motioner unto it, (for so much perhaps the masters of reformation will grant;) but by commissionariest few or many, who having the king’s letters patents, may in the virtue thereof execute the premises as agents in the right, not of their own peculiar and ordinary but of his superemiment power.

[5]When men are wronged by inferior judges, or have any just cause to take exception against them, their way for redress is to make their appeal. An appeal is a present delivery of him which maketh it out of the hands of their power and jurisdiction from whence it is made. Pope Alexander having sometime the king of England at the advantage, caused him, amongst other things, to agree, that as many of his subjects as would, might appeal to the court of Rome.  “And thus,” saith one “that whereunto a mean person at this day would scorn to submit himself, so great a king was content to be subject. Notwithstanding even when the pope,” saith he, “had so great authority amongst princes which were far off, the Romans he could not frame to obedience, nor was able to obtain that himself might abide at Rome, though promising not to meddle with other than ecclesiastical affairs.” So much are things that terrify more feared by such as behold them aloof off than at hand.

Reformers I doubt not in some cases will admit appeals, made unto their synods; even as the church of Rome doth allow of them so they be made to the bishop of Rome. As for that kind of appeal which the English lawsd approve, from the judge of anyc particular court unto the king, as the only supreme governor on earth, who by his delegates may give a final definitive sentence, from which no further appeal can be made; will their platform allow of this? Surely, forasmuch as in that estate which they all dream of, the whole Church must be divided into parishes, of which none can have greater or less authority and power than another; again, the king himself must be but as a common member in the body of his own parish, and the causes of that only parish must be by the officers thereof determinable; in case the king had so much preferment, as to be made one of those officers (for otherwise by their positions he were not to meddle any more than the meanest amongst his subjects with the judgment of any ecclesiastical cause), how is it possible they should allow of appeals to be made from any other abroad to the king?

[6]To receive appeals from all other judges, belongeth unto the highest in power over all; and to be in power over all, as touching the judgment of ecclesiastical causes, this  as they think belongeth only unto synods. Whereas therefore with us, kings do exercise over all kinds of persons and causes, power both of voluntary and litigious jurisdiction; so that according to the one they visit, reform, and command; according to the other, they judge universally, doing both in far other sort than such as have ordinary spiritual power: oppugned herein we are by some colourable shew of argument, as if to grant thus much unto any secular person it were unreasonable. “For sith it is,” say they “apparent out of the Chronicles, that judgment in church matters pertaineth unto God; seeing likewise it is evident out of the Apostles, that the high priest is set over those matters in God’s behalf; it must needs follow that the principality or direction of the judgment of them is by God’s ordinance appertaining unto the hight priest, and consequently to the ministry of the Church: and if it be by God’s ordinance appertaining unto them, how can it be translated from them unto the civil magistrate?” Which argument, briefly drawn into form, lieth thus: That which belongeth unto God, may not be translated unto any other than whom he hath appointed to have it in his behalf: but principality of judgment in church matters appertaineth unto God, which hath appointed the high priest, and consequently the ministry of the Church alone, to have it in this behalf; therefore, it may not from them be translated to the civil magistrate. The first of which three propositions we grant; as also in the second that branch which ascribeth unto God principality in church matters. But that either he did appoint none but only the high priest to exercise the said principality for him; or that the ministry of the Church may in reason from thence be concluded to have alone the same principality by his appointment: these two points we deny utterly.

For concerning the high priest, there is first no such ordinance of God to be found. “Every high priest,” saith the Apostle “is taken from among men, and is ordained for  men in things pertaining to God:” whereupon it may well be gathered, that the priest was indeed ordained of God to have power in things pertaining unto God. For the Apostle doth there mention the power of offering gifts and sacrifices for sins; which kind of power was not only given of God unto priests, but restrained unto priests only. The power of jurisdiction and ruling authority, this also God gave them, but not them alone. For it is held, as all men know, that others of the laity were herein joined by the law with them. But concerning principality in church affairs (for of this our question is, and of no other) the priests neither had it alone, nor at all; but (as hath been already shewed) principality in spiritual affairs was the royal prerogative of kings.

Again, though it were so, that God had appointed the high priest to have the said principality of government in those matters; yet how can they who allege this, enforce thereby that consequently the ministry of the Church, and no other, ought to have the same, when they are so far off from allowing as much to the ministry of the Gospel, as the priesthood of the Law had by God’s appointment, that we but collecting thereout a difference in authority and jurisdiction amongst the Clergy, to be for the policy of the Church not inconvenient, they forthwith think to close up our mouths by answering, “That the Jewish high priests had authority above the rest, only in that they prefigured the sovereignty of Jesus Christ; as for the ministers of the Gospel, it is,” they say, “altogether unlawful to give them as much as the least title, any syllable that any way may sound towards principality?” And of the regency which may be granted, they hold others even of the laity no less capable than pastors themselves. How shall these things cleave together?

[7]The truth is, that they have some reason to think it not all of the fittest for kings to sit as ordinary judges in matters of faith and religion. An ordinary judge must be of that quality which in a supreme judge is not necessary:  because the person of the one is charged with that which the other’s authority dischargeth, without employing personally himself herein. It is an error to think that the king’s authority can have no force or power in the doing of that which himself may not personally do. For first, impossible it is, that at one and the same time the king in person should order so many and so different affairs, as by his power every where present are wont to be ordered both in peace and in war, at home and abroad. Again, the king, in regard of his nonage or minority, may be unable to perform that thing wherein years of discretion are requisite for personal action; and yet his authority even then be of force. For which cause we say, that the king’s authority dieth not, but is, and worketh, always alike. Sundry considerations there may be, effectual to withhold the king’s person from being a doer of that which his power must notwithstanding give force unto. Even in civil affairs, where nothing doth either more concern the duty, or better beseem the majesty of kings, than personally to administer justice unto their people, as most famous princes have done: yet, if it be in case of felony or treason, the learned ins the laws of this realm do plainly affirm that well may the king commit his authority unto another to judge between him and the offender; but the king being himself here a party, he cannot personally sit to give judgment.

As therefore the person of the king may, for just considerations, even where the cause is civil, be notwithstanding withdrawn from occupying the seat of judgment, and others under his authority be fit, he unfit himself to judge; so the considerations for which it were haply not convenient for kings to sit and give sentence in spiritual courts, where causes ecclesiastical are usually debated, can be no bar to that force and efficacy which their sovereign power hath over those very consistories, and for which, we hold without any exception that all courts are the king’s. All men are not for all things  sufficient; and therefore public affairs being divided, such persons must be authorized judges in each kind, as common reason may presume to be most fit: which cannot of kings and princes ordinarily be presumed in causes merely ecclesiastical; so that even common sense doth rather adjudge this burden unto other men. We see it hereby a thing necessary, to put a difference, as well between that ordinary jurisdiction which belongeth to the clergy alone, and that commissionary wherein others are for just considerations appointed to join with them; as also between both these jurisdictions, and a third, whereby the king hath a transcendent authority, and that in all causes, over both. Why this may not lawfully be granted unto him, there is no reason.

[8]A time there was when kings were not capable of any such power, as namely, while they professed themselves open adversaries unto Christ and Christianity. A time there followed, when they, being capable, took sometimes more sometimes less to themselves, as seemed best in their own eyes, because no certainty touching their right was as yet determined. The bishops, who alone were before accustomed to have the ordering of such affairs, saw very just cause of grief, when the highest, favouring heresy, withstood by the strength of sovereign authority religious proceedings. Whereupon they oftentimes, against this new unresistible power, pleaded that use and custom which had been to the contrary; namely, that the affairs of the Church should be dealt in by the clergy, and by no other: unto which purpose, the sentences that then were uttered in defence of unabolished orders and laws, against such as did of their own heads contrary thereunto, are now altogether impertinently brought in opposition against them who use but the power which laws have given them, unless men can shew that there is in those laws some manifest iniquity or injustice.

Whereas therefore against the force judicial and imperial, which supreme authority hath, it is alleged, how Constantine termeth church-officers, “Overseers of things within the  Church” himself, “of those without the Church:” how Augustine witnesseth, that the emperor not daring to judge of the bishops’ cause, committed it unto the bishops; and was to crave pardon of the bishops, for that by the Donatists’ importunity, which made no end of appealing unto him, he was, being weary of them, drawn to give sentence in a matter of theirs how Hilary beseecheth the emperor Constance to provide that the governors of his provinces should not presume to take upon them the judgment of ecclesiastical causes, to whom commonwealth matters onlyr belonged: how Ambrose affirmeth, that palaces belong unto the emperor,  churches to the minister; that the emperor hath authority over the common walls of the city, and not in holy things for which cause he never would yield to have “the causes of the Church debated in the prince’s consistory,” but “excused himself to the emperor Valentinian, for that being convented to answer concerning church matters in a civil court, he came not:” we may by these testimonies drawn from antiquity, if we list to consider them, discern how requisite it is that authority should always follow received laws in the manner of proceeding. For inasmuch as there was at the first no certain law, determining what force the principal civil magistrate’s authority should be of, how far it should reach, and what order it should observe; but Christian emperors from time to time did what themselves thought most reasonable in those affairs; by this mean it cometh to pass that they in their practice vary, and are not uniform.

Virtuous emperors, such as Constantine the Great was, made conscience to swerve unnecessarily from the customs which had been used in the Church, even when it lived under infidels. Constantine, of reverence to bishops and their spiritual authority, rather abstained from that which himself might lawfully do, than was willing to claim a power not fit or decent for him to exercise. The order which had been before, he ratified, exhorting bishops to look to the Church, and promising that he would do the office of a bishop over the commonwealth: which very Constantine notwithstanding, did  not thereby so renounce all authority in judging of spiritual causes, but that sometime he took, as St. Augustine witnesseth even personal cognition of them; howbeit whether as purposing to give therein judicially any sentence, I stand in doubt. For if the other, of whom St. Augustine elsewhere speaketh, did in such sort judge, surely there was cause why he should excuse it as a thing not usually done. Otherwise there is no let, but that any such great person may hear those causes to and fro debated, and deliver in the end his own opinion of them, declaring on which side himself doth judge that the truth is. But this kind of sentence bindeth no side to stand thereunto; it is a sentence of private persuasion, and not of solemn jurisdiction, albeit a king or an emperor pronounce it.

Again, on the contrary part, when governors infected with heresy were possessed of the highest power, they thought they might use it as pleased themselves, to further by all means therewith that opinion which they desired should prevail; they not respecting at all what was meet, presumed to command and judge all men in all causes, without either care of orderly proceeding, or regard to such laws and customs as the Church had been wont to observe. So that the one sort feared to do even that which they might; and that which the other ought not they boldly presumed upon; the one sort of modesty, excused themselves where they scarce needed; the other, though doing that which wash inexcusable, bare it out with main power, not enduring to be told by any man how far they roved beyond their bounds. So great odds between them whom before we mentioned, and such as the younger Valentinian, by whom St. Ambrose being commanded to yield up one of the churches under him unto the Arians, whereas they which were sent on the message alleged, that the emperor did but use his own right, forasmuch as all things were in his power: the answer which the holy bishop gave them was  “That the Church is the house of God, and that those things which be God’s are not to be yielded up, and disposed of at the emperor’s will and pleasure; his palaces he might grant unto whomsoever, but God’s own habitations not so.” A cause why many times emperors dido more by their absolute authority than could very well stand with reason, was the over great importunity of heretics, who being enemies to peace and quietness, cannot otherwise than by violent means be supported.

[9]In this respect therefore we must needs think the state of our own church much better settled than theirs was; because our laws have with far more certainty prescribed bounds unto each kind of power. All decisions of things doubtful, and corrections of things amiss, are proceeded in by order of law, what person soever he be unto whom the administration of judgment belongeth. It is neither permitted unto prelate nor prince to judge ands determine at their own discretion, but law hath prescribed what both shall do. What power the king hath he hath it by law, the bounds and limits of it are known; the entire community giveth general order by law how all things publicly are to be done, and the king as head thereof, the highest in authority over all, causeth according to the same law every particular to be framed and ordered thereby. The whole body politic maketh laws, which laws give power unto the king, and the king having bound himself to use according unto law that power, it so falleth out, that the execution of the one is accomplished by the other in most religious and peaceable sort. There is no cause given unto any to make supplication, as Hilary did, that civil governors, to whom commonwealth-matters only belong, might not presume to take upon them the judgment of ecclesiastical causes. If the cause be spiritual, secular courts do not meddle with it: we need not excuse ourselves with Ambrose, but boldly and lawfully we may refuse to answer before any civil  judge in a matter which is not civil, so that we do not mistake the nature either of the cause or of the court, as we easily may do both, without some better direction than can be had by the rules of this new-found discipline. But of this most certain we are, that our laws do neither suffer a spiritual court to entertain those causes which by law are civil, nor yet if the matter be indeed spiritual, a mere civil court to give judgment of it.

Touching supreme power therefore to command all men, in all manner of causes of judgment to be highest, let thus much suffice as well for declaration of our own meaning, as for defence of the truth therein.

%The king’s exemption from censure and other judicial power.
IX. The last thing of all which concerns the king’s supremacy is, whether thereby he may be exempted from being subject to that judicial power which ecclesiastical consistories have over men. It seemeth, first, in most men’s judgments to be requisite that on earth there should not be any alive altogether without standing in awe of some by whom they may be controlled and bridled.

The good estate of a commonwealth within itself is thought on nothing to depend more than upon these two special affections, fear and love: fear in the highest governor himself; and love, in the subjects that live under him. The subject’s love for the most part continueth as long as the righteousness of  kings doth last; in whom virtue decayeth not as long as they fear to do that which may alienate the loving hearts of their subjects from them. Fear to do evil groweth from the harm which evildoers are to suffer. If therefore private men, which know the danger they are subject unto, being malefactors, do notwithstanding so boldly adventure upon heinous crimes, only because they know it is possible for some transgressor sometimes to escape the danger of law: in the mighty upon earth, (which are not always so virtuous and holy that their own good minds will bridle them,) what may we look for, considering the frailty of man’s nature, if the world do once hold it for a maxim that kings ought to live in no subjection: that, how grievous disorders soever they fall into, none may have coercive power over them? Yet so it is that this we must necessarily admit, as a number of right well learned men are persuaded.

[2]Let us therefore set down first, what there is which may induce men so to think; and then consider their several inventions or ways, who judge it a thing necessary, even for kings themselves, to be punishable, and that by men. The question itself we will not determine. The reasons of each opinion being opened, it shall be best for the wise to judge which of them is likeliest to be true. Our purpose being not to oppugn any save only that which reformers hold; and of the rest, rather to inquire than to give sentence. Inducements leading men to think the highest magistrate should not be judged of any, saving God alone, are specially these. .First, as there could be in natural bodies no motion of any thing, unless there were some which moveth all things and continueth unmoveable; even so in politic societies there must be some unpunishable, or else no man shall suffer punishment. For sith punishments proceed always from superiors, to whom the administration of justice belongeth, which administration must have necessarily a fountain that deriveth it to all others, and receiveth it not from any; because otherwise the course of justice should go infinitely in a circle, every superior having his superior without end, which cannot be: therefore a well-spring it followeth there is, and a supreme head of justice,  whereunto all are subject, but itself in subjection to none. Which kind of preeminence if some ought to have in a kingdom, who but the king should have it? Kings therefore no man can have lawfully power and authority to judge. If private men offend, there is the magistrate over them, which judgeth. If magistrates, they have their prince. If princes, there is Heaven, a tribunal, before which they shall appear: on earth they are not accountable to any.

.Which thing likewise the very original of kingdoms doth shew.


* * * * * *

[3]“His second point, whereby he would make us odious, is, that we think the prince may be subject to excommunication; that is, that he is a brother that he is not without but within the Church.If this be dangerous, why is it printed and allowed in the famous writings of bishop Jewel ‘In that the high priest doth his office when he excommunicates and cuts off a dead member from the body, so far forth the prince, be he never so mighty, is inferior to him. Yea not only to a bishop, but to a simple priest?’ Why is it suffered which Mr. Nowell hath written ‘The prince ought patiently to abide excommunication at the bishop’s hands?’ Why are not the worthy examples of emperors rased out of the histories, seeing they have been subject to his [this] censure”


The Jews were forbidden to choose an alien king over them; inasmuch as there is not any thing more natural than that the head and the body subject thereunto should always, if it were possible, be linked in that bond of nearness also which birth and breeding as it were in the bowels of one common mother usually causeth. Which being true did not greatly need to be alleged for proof that kings are in the Church of God of the same spiritual fraternity with their subjects: a thing not denied nor doubted of.

Indeed the king is a brother; but such a brother as unto whom all the rest of the brethren are subject. He is a sheaf of the Lord’s field as the rest are; howbeit, a sheaf which is so far raised up above the rest that they all owe reverence unto it. The king is a brother which hath dominion over all his brethren. A strange conclusion to gather hereby, that therefore some of his brethren ought to have the authority of correcting him. We read that God did say unto David, “If Solomon thy son forget my laws, I will punish his transgressions with a rod.” But that he gave dominion unto any of Solomon’s brethren to chastise Solomon, we do not read.

It is a thing very much alleged, that the church of the Jews had the sword of excommunication. Is any man able to allege where the same was ever drawn forth against the king? Yet how many of their kings how notoriously spotted?

Our Saviour’s words are, “If thy brother offend thee.” And St. Paul’s, “Do ye not judge them that are within?” Both which speeches are but indefinite. So that neither the one nor the other is any let but some brother there may be  whose person is exempt from being subject to any such kind of proceeding: some within, yet not therefore under, the jurisdiction of any other. Sentences, indefinitely uttered, must sometimes universally be understood: but not where the subject or matter spoken of doth in particulars admit that difference which may in reason seclude any part from society with the residue of that whole, whereunto one common thing is attributed. As in this case it clearly fareth where the difference between kings and others of the Church is a reason sufficient to separate the one from the other in that which is spoken of brethren, albeit the name of brethren itself do agree to both. Neither doth our Saviour nor the Apostle speak in more general sort of ecclesiastical punishments than Moses in his law doth of civil: “If there be found men or the man “amongst you that hath served other gods.” Again, “The man that committeth adultery.” The punishment of both which transgressions being death, what man soever did offend therein, why was not Manasses for the one, for the other why not David accordingly executed? “Rex judicat, non judicatur,” saith one. The king is appointed a judge of all men that live under him; but not any of them his judge.

The king is not subject unto laws; that is to say, the punishment which breach of laws doth bring upon inferiors  taketh not hold on the king’s person; although the general laws which all mankind is bound unto do tie no less the king than others, but rather more. For the grievousness of sin is aggravated by the greatness of him that committeth it: for which cause it also maketh him by so much the more obnoxious unto Divine revenge, by how much the less he feareth human.

[4]Touching Bishop Jewel’s opinion hereof there is not in the place alleged any one word or syllable against the king’s prerogative royal to be free from the coercive power of all spiritual, both persons and courts, within the compass of his own dominions. “In that,” saith he, “the priest doeth his office, in that he openeth God’s word, or declareth his threats, or rebuketh sin, or excommunicateth and cutteth off a dead member from the body; so far forth the prince, be he never so mighty, is inferior unto him. But in this respect the prince is inferior not only to the pope or bishop, but also to any other simple priest.” He disputeth earnestly against that supremacy which the bishop of Rome did challenge over his sovereign lord the emperor: and by many allegations he laboureth to shew that popes have been always subject unto his supreme dominion, not he to theirs; he supreme judge over them, not they over him. Now whereas it was objected, that within the Church, when the priest doth execute his office, the very prince is inferior to him; so much being granted by Mr. Jewel, he addeth that this doth no more prove the pope than the simplest priest in the Church to be lord and head over kings. For although it doth hereby appear that in those things which belong to his priestly office the pope may do that which kings are not licensed to meddle with; in which respect it cannot be denied but that the emperor himself hath not only less power than the chiefest bishop, but even less than the meanest priest within his empire, and is consequently every priest’s inferior that way: nevertheless, sith this appertaineth nothing at all to judicial authority and power, how doth this prove kings and emperors to be by way of subjection inferior to the pope as to their ecclesiastical judge? Impertinently therefore is the answer,  which to such effect that admirable prelate maketh, brought by way of evidence to shew that in his opinion the king may not be exempted from the coercive authority and power of his own Clergy, but ought for his faults to be as punishable in their courts as any other subject under him.

[5]The excommunication, which good Mr. Nowell thinketh that princes ought patiently to suffer at the bishop’s hands, is no other than that which we also grant may be exercised on such occasions and in such manner as those two alleged examples out of antiquity do enforce.

“It is reported,” saith Eusebius “that one of the Philips which succeeded Gordian, came, being a Christian, to join with the rest of the people in prayer, the last festival day of Easter. At which time he which governed the Church there whither the emperor did resort, would in no case admit him, unless he first made confession, and were contented afterwards to stay his time in the place appointed for penitents,” (according to the manner of Church discipline in those days, whereof we have spoken in the fifth [sixth?] book sufficiently); “because he was known to be many ways faulty. To this he readily condescended, making manifest by his deeds his true and religious affection to Godwards.”

Another example there is, of the emperor Theodosius, who understanding that violence in the city of Thessalonica had been offered unto certain magistrates, sent in great rage a band of men; and, without any examination had to know where the fault was, slew mel-pell both guilty and innocent, to the number of 7000.It chanced afterwards, that the  emperor coming to Milan, and intending to go to the Church as his accustomed manner was, St. Ambrose the bishop of that city, who before had heard of the emperor’s so cruel and bloody an act, met him before the gate of the church, and in this wise forbade him to enter: “Emperor, it seemeth that how great the slaughter is which thyself hast made thou weighest not; nor, as I think, when wrath was settled did reason ever call to account what thou hadst committed. Peradventure thine imperial royalty hindreth the acknowledgment of thy sin; and thy power is a let to reason. Notwithstanding know thou shouldst what our nature is, how frail a thing and how fading; and that the first original from whence we have all sprung was the very dust whereunto we must slide again. Neither is it meet that being inveigled with the show of thy glistering robes thou shouldst  forget the imbecility of that flesh which is covered therewith. Thy subjects (O emperor) are in nature thy colleagues: yea even in her vice [service?] thou art also joined as a fellow with them. For there is one Lord and Emperor, the Maker of this whole assembly of all things. With what eyes therefore wilt thou look upon the habitation of that common Lord? With what feet wilt thou tread upon that sacred floor? How wilt thou stretch forth those hands from which the blood as yet of unrighteous slaughter doth distil? The body of our Lord all holy how wilt thou take into such hands? How wilt thou put his honourable blood unto that mouth, the wrathful word whereof hath caused against all order of law the pouring out of so much blood? Depart therefore, and go not about by after deeds to add to thy former iniquity. Receive that bond wherewith from heaven the Lord of all doth give consent that thou shouldst be tied; a bond which is medicinable, and procureth health.” Hereunto the king submitted himself; (for being brought up in religion he knew very well what belonged unto priests, what unto kings;) and with sobbing tears returned to the court again. Some eight months after, came the feast of our Saviour’s Nativity; but yet the king sat still at home, mourning and emptying the lake of tears: which when Rufinus beheld, being at that time commander over the king’s house, and by reason of usual access the bolder to speak; he came and asked the cause of those tears. To whom the king, with bitter grief and tears more abundantly gushing out, answered; “Thou, O Ruffin, dalliest, for mine evils thou feelest not: I mourn and bewail mine own wretchedness, considering that servants and beggars go freely to the house of God, and there present themselves before their Lord: whereas both from thence and from heaven also I am excluded. For in my mind I carry that voice of our Lord which saith with express terms, ‘Whomsoever ye shall bind on earth, he in heaven shall be bound also.’ ” The rest of the history, which concerneth the manner of the emperor’s admission after so earnest repentance, needeth not to be here set down.

It now remaineth to be examined whether these alleged examples prove that which they should do, yea or no. The  thing which they ought to confirm is, that no less Christian kings than other persons under them ought to be subject to the selfsame coercive authority of Church-governors, and for the same kinds of transgressions, to receive at their hands the same spiritual censure of excommunication judicially inflicted by way of punishment. But in the aforesaid examples, whether we consider the offence itself of the excommunicate, or the persons excommunicating, or the manner of their proceeding; which three comprehend the whole substance of that which was done; it doth not by any of these appear that kings in suchwise should be subject. For, concerning the offences of men, there is no breach of Christian charity, whether it be by deed or by word; no excess, no lightness of speech or behaviour; no fault for which a man in the course of his life is openly noted as blameable; but the same being unamended through admonition ought, (as they say,) with the spiritual censure of excommunication to be punished. Wherefore unless they can shew, that in some such ordinary transgression, kings and princes, upon contempt of the Church’s more mild censure, have been like other men in ancient times excommunicated, what should hinder any man to think but that the rare and unwonted crimes of those two emperors did cause their bishops to try what unusual remedy would work in so desperate diseases? Which opinion is also made more probable, inasmuch as the very histories, which have recorded them, propose them for strange and admirable patterns; the bishops, of boldness; the emperors, of meekness and humility. The [they?] wonder at the one, for adventuring to do it unto emperors; at the other, for taking it in so good part at the hands of bishops. What greater argument that all which was herein done proceeded from extraordinary zeal on both sides, and not from a settled judicial authority which the one was known to have over the other by a common received order in the Church. For at such things who would wonder?

Furthermore, if ye consider their persons, whose acts these excommunications were; he which is said to have excommunicated Philip emperor of Rome was Babylas the bishop of Antioch: and he which Theodosius emperor of Constantinople, Ambrose the bishop of Milan. Neither of which two bishops  (as I suppose) was ordinary unto either of the two emperors.  And therefore they both were incompetent judges, and such as had no authority to punish whom they excommunicated: except we will grant the emperor to have been so much the more subject than his subjects, that whereas the meanest of them was under but some one diocesan, any that would might be judge over him. But the manner of proceeding doth as yet more plainly evict that these examples make less than nothing for proof that ecclesiastical governors had at that time judicial authority to excommunicate emperors and kings. For what form of judgment was there observed, when neither judges nor parties judged did once dream of any such matter; till the one by chance repaired unto the place where the others were, and at that very instant suffered a sudden repulse; not only besides their own expectation, but also without any purpose beforehand in them who gave it? Judicial punishment hath at the leastwise sentence going always before execution, whereas all which we read of here is, that the guilty being met in the way were presently turned back, and not admitted to be partakers of those holy things whereof they were famously known unworthy.

[6]I therefore conclude, that these excommunications have neither the nature of judicial punishments, nor the force of sufficient arguments to prove that ecclesiastical judges should have authority to call their own sovereign to appear before them into their consistories, there to examine, to judge, and by excommunication to punish them, if so be they be found culpable.

But concerning excommunication, such as is only a dutiful, religious, and holy refusal to admit notorious transgressors in so extreme degree unto the blessed communion of saints, especially the mysteries of the Body and Blood of Christ, till their humbled penitent minds be made manifest: this we grant every king bound to abide at the hands of any minister of God wheresoever through the world. As for judicial authority to punish malefactors, if the king be as the kings of Israel were, and as every of ours is, a supreme Lord, than whom none under God is by way of ruling authority and power higher, where he reigneth, how should any man there have the high place of a judge over him? He must be more  than thine equal that hath a chastising power over thee: so far is it off that any under thee should be thy judge. Wherefore, sith the kings of England are within their own dominions the most high, and can have no peer, how is it possible that any, either civil or ecclesiastical person under them should have over them coercive power, when such power would make that person so far forth his superior’s superior, ruler, and judge? It cannot therefore stand with the nature of such sovereign regiment that any subject should have power to exercise on kings so highly authorized the greatest censure of excommunication, according to the platform of Reformed Discipline: but if this ought to take place, the other is necessarily to give place. For which cause, till better reason be brought, to prove that kings cannot lawfully be exempted from subjection unto ecclesiastical courts, we must and do affirm their said exemption lawful.


