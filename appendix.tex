
[456]
APPENDIX, No. I. 
[Supposed Fragment of a Sermon on Civil Obedience, hitherto printed as part of the Eighth Book.]
* * * * * * * * *

BOOK VIII. Appendix, No. 1.Yea that1 which is more, the laws thus made, God himself doth in such sort authorize, that to despise them, is to despise in them him. It is a loose and licentious opinion, which the Anabaptists have embraced, holding that a Christian man’s liberty is lost, and the soul which Christ hath redeemed unto himself injuriously drawn into servitude under the yoke of human power, if any law be now imposed besides the Gospel of Christ, in obedience whereunto the Spirit of God, and not the constraint of men, is to lead us, according to that of the blessed Apostle2, “Such as are led by the Spirit of God, they are the sons of God,” and not such as live in thraldom unto men. Their judgment is therefore that the Church of Christ should admita no lawmakers but the evangelists, no courts but presbyteries, no punishments but ecclesiastical censures.

As against this sort, we are to maintain the use of human laws, and the continual necessity of making them from time to time, as long as this present world doth last; so likewise the authority of laws so made doth need much more by us to be strengthened against another sort, who, although they do not utterly condemn the making of laws in the Church, yet make they a greatb deal less account of them than they should do. There are which think simply of human laws, that they can in no sort touch the conscience; that to break and transgress them cannot make men in the sight of God culpable as sin doth; only when we violate such laws, we do [457] thereby make ourselves obnoxious unto external punishment in this world, so that the magistrate may in regard of such offence committed justly correct the offender, and cause him without injury to endure such pain as the lawc doth appoint; but further it reacheth not. For first, the conscience is the proper court of God, the guiltiness thereof is sin, and the punishment eternal death: men are not able to make any law that shall command the heart, it is not in them to make thed inward conceit a crime, or to appoint for any crime other punishment than corporal: their laws therefore can have no power over the soul, neither can the heart of man be polluted by transgressing them. St. Austine1 rightly defineth sin to be that which is spoken, done or desired, not against any lawe, but against the law of the living God. The law of God is proposed unto men, as a glass wherein to behold the stains and spotsf of their sinful souls. By it they are to judge themselves, and when they findg themselves to have transgressed against it, then to bewail their offences with David2, “Against thee only, O Lord, have I sinned, and done wickedly in thy sight;” that so our present tears may extinguish the flames, which otherwise we are to feel, and which God in that day shall condemn the wicked unto, when they shall render account of the evil which they have done, not by violating statute laws and canons, but by disobedience unto his law and wordh.

For our better instruction therefore concerningi this point, first we must note, that the law of God himselfk doth require at our hands subjection. “Be ye subject3,” saith St. Peter; and St. Paul, “4Let every soul be subject; subject all unto such powers as are set over us.” For if such as are not set over us require our subjection, we by denying it are not disobedient to the law of God, or undutiful unto higher powers; because though they be such in regard of them over whom they have lawful dominion, yet having not so over us, unto us they are not such5.

[458]
Subjection therefore we owe, and that by the law of God; we are in conscience bound to yield it even unto every of them that hold the seats of authority and power in relation unto us. Howbeit, not all kindl of subjection unto every such kind of power. Concerning Scribes and Pharisees, our Saviour’s precept was1, “Whatsoever they shall tell youm, do it;” was it his meaning, that if they should at any time enjoin the people to levy an army, or to sell their lands and goods for the furtherance of so great an enterprize; and in a word, that simply whatsoevern it were which they did command, they ought without any exception forthwith to be obeyed? No, but “whatsoever they shall tell you,” must be understood in pertinentibus ad Cathedram, it must be construed with limitation, and restrained unto things of that kind which did belong to their place and power. For they had not power general, absolutely given them to command ino all things.

The reason why we are bound in conscience to be subject unto all such powerp is, because all “powers are of God2.” They are of God either instituting or permitting them. Power is then of divine institution, when either God himself doth deliver, or men by light of nature find out the kind thereof. So that the power of parents over children, and of husbands over their wives, the power of all sorts of superiors, made by consent of commonwealths within themselves, or grown from agreement amongst nations, such power is of God’s own institution in respect of the kind thereof. Again, if respect be had unto those particular persons to whom the same is derived, if they either receive it immediately from God, as Moses and Aaron did; or from nature, as parents do; or from men by a natural and orderly course, as every governor appointed in any commonwealth, by the orderq thereof, doth: then is not the kind of their power only of God’s institutionr, but the derivation thereof also into their persons, is from him. He hath placed them in their rooms, and doth term them his ministers; subjection therefore is due unto all such powers, inasmuch as they are of God’s own institution, [459] even then when they are of man’s creation, omni humanæ creaturæs: which things the heathens themselves do acknowledge:

Σκηπτου̑χος βασιλεὺς, ᾠ̑τε Ζεὶς κυ̑δος ἔδωκεν1s.

As for them that exercise power altogether against order, although the kind of power which they have may be of God, yet is their exercise thereof against God, and therefore not of God, otherwise than by permission, as all injustice is.

Touching such acts as are done by that power which is according to his institution, that God in like sort doth authorize them, and account them to be his; though it were not confessed, it might be proved undeniablet. For if that be accounted our deed, which others do, whom we have appointed to be our agents, how should God but approve those deeds, even as his own, which are done by virtue of that commission and power which he hath given. “Take heed,” saith Jehoshaphat unto his judges2, “be careful and circumspect what ye do; ye do not execute the judgments of men, but of the Lord.” The authority of Cæsar over the Jews, from whence was it? Had it any other ground than the law of nations, which maketh kingdoms, subdued by just war, to be subject unto their conquerors? By this power Cæsar exacting tribute, our Saviour confesseth it to be his right, a right which could not be withheld without injury; yea disobedience herein unto him had beenu rebellion against God. Usurpers of power, whereby we do not mean them that by violence have aspired unto places of highest authority, but them that use more authority than they did ever receive in form and manner beforementioned: (for so they may do, whose title unto the rooms of authority which they possess, no man can deny to be just and lawful: even as contrariwise some men’s proceedings in government have been very orderly, who notwithstanding did not attain to be made governors without great violence and disorder;) such usurpers thereforex, as in the exercise of their power do more than they have been authorized to do, cannot in conscience bind any man unto obedience.

That subjection which we owe unto lawful powers, doth not only import that we should be under them by order of our state, but that we shew all submission towards them both by honour and obedience. [460] He that resisteth them, resisteth God:BOOK VIII. Appendix, No. 2. and resisted they arey, if either the authority itself which they exercise be denied, as by Anabaptists all secular jurisdiction isz; or if resistance be made but only so far forth as doth touch their persons which are invested with power (for they which said, Nolumus hunc regnare, did not utterly exclude regiment; nor did they wish all kind of government cleana removed, which would not at the first have David governb): or if that which they do by virtue of their power, namely, their laws, edicts, sentencesc, or other acts of jurisdiction, be not suffered to take effect, contrary to the blessed Apostle’s most holy preceptd, “Obey them that have the oversight of you1.” Or if they do take effect, yet is not the will of God thereby satisfied neither, as long as that which we do is contemptuously or repiningly done, because we can do no otherwise. In such sort the Israelites in the desert obeyed Moses, and were notwithstanding deservedly plagued for disobedience. The Apostle’s precept therefore is, “Be subject even for God’s cause; be subject, not for fear, but fore mere conscience, knowing, that he which resisteth them, purchaseth unto himself condemnation.” Disobedience therefore unto laws which are made by menf is not a thing of so small account as some would make it.

Howbeit, too rigorous it were, that the breach of every human law should be held a deadly sin: a mean there is between those extremities, if so be we can find it out.

* * * * * * * * *

APPENDIX, No. II. 
A Discovery of the Causes of the Continuance of these Contentions concerning Church Government, out of the Fragments of Richard Hooker2.
Contention ariseth, either through error in men’s judgments, or else disorder in their affections.

When contention doth grow by error in judgment, it ceaseth not [461] till men by instruction come to see wherein they err, and what it is that did deceive them. Without this, there is neither policy nor punishment that can establish peace in the Church.

The Moscovian emperor1, being weary of the infinite strifes and [462] contentions amongst preachers, and by their occasion amongst others, forbad preaching utterly throughout all his dominions; and instead thereof commanded certain sermons of the Greek and Latin Fathers to be translated, and them to be read in public assemblies, without adding a word of their own thereunto upon pain of death. He thought by this politic devise to bring them to agreement, or at least to cover their disagreement. But so bad a policy was no fit salve for so great a sore.

We may think perhaps, that punishment would have been more effectual to that purpose. For neither did Solomon speak without book in saying1, that when “folly is bound up in the heart of a child, the rod of correction must drive it out;” and experience doth shew, that when error hath once disquieted the minds of men and made them restless, if they do not fear they will terrify. Neither hath it repented the Church at any time to have used the rod in moderate severity for the speedier reclaiming of men from error, and the reunitingu such as by schism have sundered themselves. But we find by trial, that as being taught and not terrified, they shut their ears against the word of truth, and soothe themselves in that wherewith custom or sinister persuasion hath inured them: so contrariwise, if they be terrified and not taught, their punishment doth not commonly work their amendment.

As Moses therefore, so likewise Aaron; as Zerubabel, so Jehoshua; as the prince which hath laboured by the sceptre of righteousness and sword of justice to end strife, so the prophets which with the book and doctrine of salvation have soundly and wisely endeavoured to instruct the ignorant in those litigious points wherewith the Church is now troubled: whether by preaching, as Apollos among the Jews; or by disputing, as Paul at Athens, or by writing, as the learned in their several times and ages heretofore, or by conferring in synods and councils, as Peter, James, and others at Jerusalem, or by any the like allowable and laudable means; their praise is worthily in the gospel2, and their portion in that promise which God hath made by his prophet3, “They that turn many unto “righteousness shall shine as the stars for ever and [463] ever;” I say, whosoever have soundly and wisely endeavoured by those means to reclaim the ignorant from their error, and to make peace.

Want of sound proceeding in church controversies hath made many more stiff in error now than before.

Want of wise and discreet dealing, hath much hindered the peace of the Church. It may be thought, and is, that Arius had never raised those tempestuous storms which we read he did; if Alexander, the first that withstood the Arians’ heresy, had borne himself with greater moderation, and been less eager1 in so good a cause. Sulpitius Severus doth note as much in the dealings of Idacius2 against the favourers of Priscillian, when that heresy was but green and new sprung up. For by overmuch vehemency against Jactantiusy and his mates, a spark was made a flame: insomuch that thereby the seditious waxed rather more fierce than less troublesome. In matters of so great moment, whereupon the peace or disturbance of the Church is known to depend, if there were in us that reverend care which should be; it is not possible we should either speak at any time without fear, or ever write but with a trembling hand. Do they consider whereabout they go, or what it is they have in hand, who taking upon them the causes of God, deal only or chiefly against the persons of men?

We cannot altogether excuse ourselves in this respect, whose home controversies and debates at this day, although I trust they be as the strife of Paul with Barnabas and not with Elymas, yet because there is a truth, which on the one side being unknown hath caused contention, I do wish it had pleased Almighty God, that in sifting it out, those offences had not grown, which I had rather bewail with secret tears than public speech.

Nevertheless as some sort of people is reported to have bred a detestation of drunkenness in their children by presenting the deformity thereof in servants, so it may come to pass (I wish it might) that we beholding more foul deformityz in the face and countenance of a common adversary, shall be induced to correct some smaller blemishes in our own. Ye are not ignorant of the Demands3, [464] Motives1, Censures2, Apologies3, Defences, and other writings, which our great enemies have published under colour of seeking peace; promising to bring nothing but reason and evident remonstrance of truth. But who seeth not how full gorged they are with virulent, slanderous, and immodest speeches, tending much to the disgrace, to the disproof nothing, of that cause which they endeavour to overthrow? “Will you speak wickedly for God’s defence4?” saith Job. Will you dip your tongues in gall and your pens in blood, when youa write and speak in his cause? Is the truth confirmed, are men convicted of their error when they are upbraided with the miseries of their condition and estate? When their understanding, wit, and knowledge is depressed? When suspicions and rumours, without respect how true or how false, are objected to diminish their credit and estimation in the world? Is it likely that Invectives, Epigrams, Dialogues, Epistles, Libels, laden with contumelies and criminations, should be the means to procure peace? Surely they which do take this course, “the way of peace they have not known5.” If they did but once enter into a stayed consideration with themselves what they do, no doubt they would give over and resolve with Job6, “Behold I am vile, what shall I answer? I will lay my hand upon my mouth. If I have spoken once amiss, I will speak no more; or if twice, I will proceed no further.”

II. But how sober and how sound soever our proceeding be in these causes; all is in vain which we do to abate the errors of men, except their unruly affections be bridled. Self-love, vainglory, impatience, pride, pertinacy, these are the bane of our peace. And these are not conquered or cast out, but by prayer. Pray for Jerusalem, and your prayer shall cause “the hills to bring forth peace7:” peace shall distil and “come down like the rain upon the mown grass, and as the showers that water the earth.” We have used all other means, and behold we are frustrate, we have laboured in vain. In disputations, whether it be because men are ashamed to acknowledge their errors before many witnesses, or because extemporality doth exclude mature and ripe advice without which the truth cannot soundly and thoroughly be demonstrated, or because [465] the fervour of contention doth so disturb men’s understanding, that they cannot sincerely and effectually judge:BOOK VIII. Appendix, No. 3. in books and sermons, whether it be because we do speak and write with too little advice, or because you do hear and read with too much prejudice: in all human means which have hitherto been used to procure peace; whether it be because our dealings have been too feeble, or the minds of men with whom we have dealt too too implacable, or whatsoever the cause or causes have been: forasmuch as we see that as yet we fail in our desires, yea the ways which we take to be most likely to make peace, do but move strife; O that we would now hold our tongues, leave contending with men, and have our talk and treaty of peace with God. We have spoken and written enough of peaceb: there is no wayc left but this one1, “Pray for the peace of Jerusalem.”

APPENDIX, No. III.
A Table, shewing how the several portions of the Eighth Book in Dobson’s edition, 1825, Vol. II. are distributed in the present.
I.	“We come now,” p. 379, to “lawfully overrule,” p. 391.
See above, c. i. 1-ii. 3.
II.	“It hath been declared,” p. 391, to “ecclesiastical laws,” p. 393.
See above, c. ii. 17.
III.	“Unto which supreme,” p. 393, to “most reasonable,” p. 402.
See above, c. ii. 4-16.
IV.	“The cause of deriving,” p. 402, to “hath been shewed,” p. 404.
See above, c. ii. 18. iii. 1.
V.	“For the title or style,” p. 404, to “ought to have,” p. 405.
See above, c. iv. 8.
VI.	“These things being first,” p. 405, to “Hercules to tame them,” p. 418.
See above, c. iv. 1-7.
VII.	“The last difference,” p. 418, to “or to any part,” p. 422.
See above, c. iv. 9-12.
VIII.	“Among sundry prerogatives,” p. 422, to “and others,” p. 423.
See above, c. v. 1. latter part.
IX.	“The consuls of Rome,” p. 423, to “than the other,” ibid.
See above, c. v. 1. former part.
X.	“Wherefore the clergy,” ibid. to “shall not need,” ibid.
See above, c. v. 2. last paragraph.
XI.	“The ancient imperial,” ibid. to “meetings ecclesiastical,” p. 425.
See above, c. v. 2. former part.
XII.	“There are which wonder,” p. 425, to “do withstand,” p. 432.
See above, c. vi. 10-14. former part.
XIII.	“Touching the king’s,” p. 432, to “of the truth therein,” p. 443.
See above, c. viii. 1-9.
XIV.	“The case is not like,” p. 443, to “assent not asked,” p. 449.
See above, c. vi. 4-9.
XV.	“Yea, that which is more,” p. 449, to “can find it out,” p. 453.
See above, Appendix to B. VIII. No. I.
A Table, shewing the arrangement of the fragments in Bernard’s Clavi Trabales, as compared with the present Edition.
P. 65.	“The service which we do,” to “kings and priests,” p. 71.
See above, c. iii. 2-6.
P. 71.	“Wherein it is,” to “unto kings,” p. 72.
See above, c. vi. 14. note 1, p. 418.
P. 72.	“Although not both,” to “over the Church,” ibid.
See above, c. vi. 14. latter part.
P. 73.	“The case is not like,” to “commonwealth hath simply,” p. 76.
See above, c. vi. 4-6.
P. 77.	“Touching the advancement,” to “sufficiently spoken before,” p. 86.
See above, c. vii. 1-7.
P. 86.	“As therefore the person” to “he came not,” p. 87.
See above, c. viii. 7, 8.
P. 88.	“Besides these testimonies,” to “bear rule,” ibid.
See above, c. viii. 8. note 3, p. 440.
P. 88.	“We may by these testimonies,” to “the truth therein,” p. 92.
See above, c. viii. 9.
P. 92.	“The last thing,” to “accountable to any,” p. 94.
See above, c. ix. 1, 2.[467]
APPENDIX, No. IV.
The following are detached notes in the Dublin MS. which occur,BOOK VIII. Appendix, No. 4. with an interval of one blank page, immediately after the dissertation on the making of laws, p. 419. The words “one man,” at the top, probably refer to some passage intended to be produced for refutation.

“One man. Then could not any of them be under another’s authority so far as thereby to be either licensed or hindered in those things which he doth by the said power, but God alone should himself on earth authorize and disauthorize all that bare rule in the Church. Wherefore, to set down briefly that which we hold for truth. Power ecclesiastical itself is originally God’s ordinance: he hath appointed it to be; and therefore in that respect on him only they all which have it are most rightly said to depend. The derivation of that power into the several persons which have it is the proper deed of the Church, and of those high ministers which are in that case appointed to ordain and consecrate such as from time to time shall exercise and use the same.

“Furthermore, sith when they have that power, it resteth nevertheless unexercised, except some part of the people of God be permitted them to work upon; they must of necessity for the peaceable and quiet practice of their authority upon the persons of men, where all are subject to a Christian king, depend in that respect on him also. By holding therefore this dependency whereof we speak, it is not meant that either the king did first institute, or that he doth confer and give, the grace of ecclesiastical presidency; but only add unto it exercise by the furtherance of his supereminent authority and power, without the predominant concurrency whereof spiritual jurisdiction could take no effect, men’s persons could not in open and orderly sort be subject thereunto. A bishop, whose calling is authorized wholly from God, and received by imposition of sacred hands, can execute safely no act of episcopal authority on any one of the king’s liege people, otherwise than under him who hath sovereignty over them all.”

The election of Bishops.
At the first, the first created in the College of Presbyters was still the Bishop1: he dying, the next senior did succeed him. “Sed [468] quia cœperunt sequentes Presbyteri indigni inveniri ad primatus tenendos, immutata est ratio; prospiciente Concilio ut non ordo sed meritum crearet Episcopum, multorum sacerdotum judicio constitutum, ne indignus temere usurparet, et esset multis scandalum.” Ambr. in 4. ad Eph.

“Apud nos Apostolorum locum episcopi tenent. Bishops, the Apostles’ successors. Hieron. Epist. 54.” (al. 41. tom. i. 187. ed. Vallars.) “ad Marcell.” “Absit ut de his quicquam sinistrum loquar, qui Apostolico gradui succedentes Christi Corpus sacro ore conficiunt.” “Speech against the clergy of God irreligious. Hieron. Ep. 1. ad Heliodor.” (al. 14. § 8. t. i. 33.)

“Privileges granted unto the Clergy. A law in general, to make good all such privileges as by way of honour had been granted to the clergy before, the Roman emperor thought himself bound in conscience to ratify.” L. xii. c. De Sacr. Eccl. [Cod. i. tit. ii. lex 12. ad 454. “Privilegia, quæ generalibus constitutionibus universis sacrosanctis ecclesiis orthodoxæ religionis retro Principes præstiterunt, firma et illibata in perpetuum decernimus custodiri.”] “Again, whereas Church lands did before stand charged with ordinary burdens even of the meanest kind, this the law imperial taketh away as a thing contumelious unto religion, and giveth for the time to come a privilege of immunity from such burdens.” “Prima illius usurpationis contumelia depellenda est, ne prædia usibus cœlestium secretorum dedicata, sordidorum munerum fæce vexentur.” L. v. c. De Sacr. Eccles. [ad 412.] “Imprimis concessimus Deo, et hac præsenti charta nostra confirmavimus, pro nobis et hæredibus nostris in perpetuum, quod Ecclesia Anglicana libera sit, et habeat omnia jura sua integra, et libertates suas illæsas.” Magn. Chart. cap. 1.

[469]
A LEARNED AND COMFORTABLE SERMON OF THE CERTAINTY AND PERPETUITY OF FAITH IN THE ELECT. 
ESPECIALLY OF THE PROPHET HABAKKUK’S FAITH1.
Habak. i. 4.
[“Therefore the law is slacked, and judgment doth never go forth.”]

Whether the Prophet Habakkuk2, by admitting this cogitation into his mind, “The law doth fail,” did thereby shew himself an unbeliever.

SERM. 1.WE have seen in the opening of this clause which concerneth the weakness of the prophet’s faith, first what things they are whereunto the faith of sound believers doth assent: secondly, wherefore all men assent not thereunto: and thirdly, why they that do, do it many times with small assurance. Now because nothing can be so truly spoken, but through misunderstanding it may be depraved; therefore to prevent, if it be possible, all misconstruction in this cause, where a small error cannot rise but with great danger, it is perhaps needful, ere we come to the fourth point, that something be added to that which hath been already spoken concerning the third.

[470]
That mere natural men do neither know nor acknowledge the things of God, we do not marvel, because they are spiritually to be discerned; but they in whose hearts the light of grace doth shine, they that are taught of God, why are they so weak in faith? Why is their assenting to the law so scrupulous, so much mingled with fear and wavering? It seemeth strange that ever they should imagine the law to fail. It cannot seem strange if we weigh the reason. If the things which we believe be considered in themselves, it may truly be said that faith is more certain than any science. That which we know either by sense, or by infallible demonstration, is not so certain as the principles, articles, and conclusions of Christian faith. Concerning which we must note, that there is a Certainty of Evidence and a Certainty of Adherence. Certainty of Evidence we call that, when the mind doth assent unto this or that, not because it is true in itself, but because the truth is clear, because it is manifest to us. Of things in themselves most certain, except they be also most evident, our persuasion is not so assured as it is of things more evident, although in themselves they be less certain. It is as sure, if not surer, that there be spirits, as that there be men; but we be more assured of these than of them, because these are more evident. The truth of some things is so evident, that no man which heareth them can doubt of them: as when we hear that “a part of any thing is less than the whole,” the mind is constrained to say, this is true. If it were so in matters of faith, then, as all men have equal certainty of this, so no believer should be more scrupulous and doubtful than another. But we find the contrary. The angels and spirits of the righteous in heaven have certainty most evident of things spiritual: but this they have by the light of glory. That which we see by the light of grace, though it be indeed more certain; yet is it not to us so evidently certain, as that which sense or the light of nature will not suffer a man to doubt of. Proofs are vain and frivolous except they be more certain than is the thing proved: and do we not see how the Spirit every where in the Scripture proveth matters of faith, laboureth to confirm us in the things which we believe, by things whereof we have sensible knowledge? I conclude therefore that we have less certainty of evidence concerning things believed, than concerning [471] sensible or naturally perceived. Of these who doth doubt at any time? Of them at some time who doubteth not? I will not here allege the sundry confessions of the perfectest that have lived upon earth concerning their great imperfections this way; which if I did, I should dwell too long upon a matter sufficiently known by every faithful man that doth know himself.

The other, which we call the Certainty of Adherence, is when the heart doth cleave and stick unto that which it doth believe. This certainty is greater in us than the other. The reason is this: the faith of a Christian doth apprehend the words of the law, the promises of God, not only as true, but also as good; and therefore even then when the evidence which he hath of the truth is so small that it grieveth him to feel his weakness in assenting thereto, yet is there in him such a sure adherence unto that which he doth but faintly and fearfully believe, that his spirit having once truly tasted the heavenly sweetness thereof, all the world is not able quite and clean to remove him from it; but he striveth with himself to hope against all reason of believing, being settled with Job upon this unmoveable resolution, “Though God kill me, I will not give over trusting in him1.” For why? this lesson remaineth for ever imprinted in him, “It is good for me to cleave unto God2.”

Now the minds of all men being so darkened as they are with the foggy damp of original corruption, it cannot be that any man’s heart living should be either so enlightened in the knowledge, or so established in the love of that wherein his salvation standeth, as to be perfect, neither doubting nor shrinking at all. If any such were, what doth let why that man should not be justified by his own inherent righteousness? For righteousness inherent, being perfect, will justify. And perfect faith is a part of perfect righteousness inherent; yea a principal part, the root and the mother of all the rest: so that if the fruit of every tree be such as the root is, faith being perfect, as it is if it be not at all mingled with distrust and fear, what is there to exclude other Christian virtues from the like perfections? And then what need we the righteousness of Christ? His garment is superfluous: we may be honourably clothed with our own robes, if it be thus. But let [472] them beware who challenge to themselves a strength which they have not, lest they lose the comfortable support of that weakness which indeed they have.

Some shew, although no soundness of ground, there is, which may be alleged for defence of this supposed perfection in certainty touching matters of our faith; as, first, that Abraham did believe and doubted not: secondly, that the Spirit which God hath given us to no other end, but only to assure us that we are the sons of God, to embolden us to call upon him as our Father, to open our eyes, and to make the truth of things believed evident unto our minds, is much mightier in operation than the common light of nature, whereby we discern sensible things: wherefore we must needs be more sure of that we believe, than of that we see; we must needs be more certain of the mercies of God in Christ Jesus, than we are of the light of the sun when it shineth upon our faces.

To that of Abraham, “He did not doubt1;” I answer, that this negation doth not exclude all fear, all doubting; but only that which cannot stand with true faith. It freeth Abraham from doubting through infidelity, not from doubting through infirmity; from the doubting of Unbelievers, not of weak Believers; from such a doubting as that whereof the prince of Samaria is attainted, who hearing the promise of sudden plenty in the midst of extreme dearth, answered, “Though the Lord would make windows in heaven, were it possible so to come to pass2?” But that Abraham was not void of all doubtings, what need we any other proof than the plain evidence of his own words3?

The reason which is taken from the power of the Spirit were effectual, if God did work like a natural agent, as the fire doth inflame, and the sun enlighten, according to the uttermost ability which they have to bring forth their effects. But the incomprehensible wisdom of God doth limit the effects of his power to such a measure as seemeth best unto himself. Wherefore he worketh that certainty in all, which sufficeth abundantly to their salvation in the life to come; but in none so great as attaineth in this life unto perfection. Even so, O Lord, it hath pleased thee; even so it is best and fittest for us, [473] that feeling still our own infirmities, we may no longer breathe than pray, Adjuva, Domine; “Help, Lord, our incredulity1.” Of the third question, this I hope will suffice, being added unto that which hath been thereof already spoken. The fourth question resteth, and so an end of this point.

That which cometh last of all in this first branch to be considered concerning the weakness of the Prophet’s faith, “Whether he did by this very thought, The law doth fail, quench the Spirit, fall from faith, and shew himself an unbeliever or no?” The question is of moment; the repose and tranquillity of infinite souls doth depend upon it. The Prophet’s case is the case of many; which way soever we cast for him, the same way it passeth for all others. If in him this cogitation did extinguish grace, why the like thoughts in us should not take the like effect, there is no cause. Forasmuch therefore as the matter is weighty, dear, and precious, which we have in hand, it behoveth us with so much the greater chariness to wade through it, taking special heed both what we build, and whereon we build: that if our building be pearl, our foundation be not stubble; if the doctrine we teach be full of comfort and consolation, the ground whereupon we gather it be sure; otherwise we shall not save but deceive both ourselves and others. In this we know we are not deceived, neither can we deceive you, when we teach that the faith whereby ye are sanctified cannot fail; it did not in the Prophet, it shall not in you. If it be so, let the difference be shewed between the condition of unbelievers and his, in this or in the like imbecility and weakness. There was in Habakkuk that which St. John doth call “the seed of God2,” meaning thereby the first grace which God poureth into the hearts of them that are incorporated into Christ; which having received, if because it is an adversary unto sin, we do therefore think we sin not, both otherwise, and also by distrustful and doubtful apprehending of that which we ought steadfastly to believe, surely we do but deceive ourselves. Yet they which are of God do not sin either in this, or in any thing, any such sin as doth quite extinguish grace, clean cut them off from Christ Jesus; because the “seed of God” abideth in them, and doth shield them from receiving any irremediable wound. Their faith, when it is at the strongest, [474] is but weak; yet even then when it is at the weakest, so strong, that utterly it never faileth, it never perisheth altogether, no not in them who think it extinguished in themselves. There are for whose sakes I dare not deal slightly in this cause, sparing that labour which must be bestowed to make it plain. Men in like agonies unto this of the Prophet Habakkuk’s are through the extremity of grief many times in judgment so confounded, that they find not themselves in themselves. For that which dwelleth in their hearts they seek, they make diligent search and inquiry. It abideth, it worketh in them, yet still they ask where? Still they lament as for a thing which is past finding: they mourn as Rachel, and refuse to be comforted, as if that were not, which indeed is, and as if that which is not, were; as if they did not believe when they do, and as if they did despair when they do not. Which in some I grant is but a melancholy passion, proceeding only from that dejection of mind, the cause whereof is the body, and by bodily means can be taken away. But where there is no such bodily cause, the mind is not lightly in this mood, but by some of these three occasions. One, that judging by comparison either with other men, or with themselves at some other time more strong, they think imperfection to be a plain deprivation, weakness to be utter want of faith.

Another cause is, they often mistake one thing for another. St. Paul wishing well to the Church of Rome prayeth for them after this sort: “The God of hope fill you with all joy of believing1.” Hence an error groweth, when men in heaviness of spirit suppose they lack faith, because they find not the sugared joy and delight which indeed doth accompany faith, but so as a separable accident, as a thing that may be removed from it; yea, there is a cause why it should be removed. The light would never be so acceptable, were it not for that usual intercourse of darkness. Too much honey doth turn to gall; and too much joy even spiritually would make us wantons. Happier a great deal is that man’s case, whose soul by inward desolation is humbled, than he whose heart is through abundance of spiritual delight lifted up and exalted above measure. Better it is sometimes to go down into the pit with him, who, beholding darkness, and bewailing the loss [475] of inward joy and consolation, crieth from the bottom of the lowest hell, “My God, my God, why hast thou forsaken me1?” than continually to walk arm in arm with angels, to sit as it were in Abraham’s bosom, and to have no thought, no cogitation, but “I thank my God it is not with me as it is with other men2.” No, God will have them that shall walk in light to feel now and then what it is to sit in the shadow of death. A grieved spirit therefore is no argument of a faithless mind.

A third occasion of men’s misjudging themselves, as if they were faithless when they are not, is, they fasten their cogitations upon the distrustful suggestions of the flesh, whereof finding great abundance in themselves, they gather thereby, Surely unbelief hath full dominion, it hath taken plenary possession of me; if I were faithful, it could not be thus: not marking the motions of the Spirit and of faith, because they lie buried and overwhelmed with the contrary: when notwithstanding as the blessed Apostle doth acknowledge3, that “the spirit groaneth,” and that God heareth when we do not; so there is no doubt, but that our faith may have and hath her privy operations secret to us in whom, yet known to him by whom, they are.

Tell this to a man that hath a mind deceived by too hard an opinion of himself, and it doth but augment his grief: he hath his answer ready, Will you make me think otherwise than I find, than I feel in myself? I have thoroughly considered and exquisitely sifted all the corners of my heart, and I see what there is; never seek to persuade me against my knowledge; “I do not, I know I do not believe.”

Well, to favour them a little in their weakness; let that be granted which they do imagine; be it that they are faithless and without belief. But are they not grieved for their unbelief? They are. Do they not wish it might, and also strive that it may, be otherwise? We know they do. Whence cometh this, but from a secret love and liking which they have of those things that are believed? No man can love things which in his own opinion are not. And if they think those things to be, which they shew that they love when they desire to believe them; then must it needs be, that by desiring to believe they [476] prove themselves true believers. For without faith, no man thinketh that things believed are. Which argument all the subtilty of infernal powers will never be able to dissolve.

The faith therefore of true believers, though it have many and grievous downfalls, yet doth it still continue invincible; it conquereth and recovereth itself in the end. The dangerous conflicts whereunto it is subject are not able to prevail against it. The Prophet Habakkuk remained faithful in weakness, though weak in faith.

It is true, such is our weak and wavering nature, we have no sooner received grace, but we are ready to fall from it: we have no sooner given our assent to the law, that it cannot fail, but the next conceit which we are ready to embrace is, that it may, and that it doth fail. Though we find in ourselves a most willing heart to cleave unseparably unto God, even so far as to think unfeignedly with Peter, “Lord, I am ready to go with thee into prison and to death1;” yet how soon and how easily, upon how small occasions are we changed, if we be but a while let alone and left unto ourselves? The Galatians2 to-day, for their sakes which teach them the truth in Christ, content, if need were, to pluck out their own eyes3, and the next day ready to pluck out theirs which taught them. The love of the Angel of4 the Church of Ephesus, how greatly inflamed, and how quickly slacked5.

The higher we flow, the nearer we are unto an ebb, if men be respected as mere men, according to the wonted course of their alterable inclination, without the heavenly support of the Spirit.

Again, the desire of our ghostly enemy is so uncredible, and his means so forcible to overthrow our faith, that whom the blessed Apostle knew betrothed and made hand-fast unto Christ, to them he could not write but with great trembling: “I am jealous over you with a godly jealousy, for I have prepared you to one husband to present you a pure virgin unto Christ: but I fear, lest as the Serpent beguiled Eve through his subtilty, so your minds should be corrupted from the simplicity which is in Christ6.” The simplicity [477] of faith which is in Christ taketh the naked promise of God, his bare word, and on that it resteth. This simplicity the serpent laboureth continually to pervert, corrupting the mind with many imaginations of repugnancy and contrariety between the promise of God, and those things which sense or experience or some other fore-conceived persuasion hath imprinted.

The word of the promise of God unto his people is, “I will not leave thee nor forsake thee1:” upon this the simplicity of faith resteth, and it is not afraid of famine. But mark how the subtilty of Satan2 did corrupt the minds of that rebellious generation, whose spirits were not faithful unto God. They beheld the desolate state of the desert in which they were, and by the wisdom of their sense concluded the promise of God to be but folly: “Can God prepare a table in the wilderness3?”

The word of the promise to Sara was, “Thou shalt bear a son.” Faith is simple, and doubteth not of it: but Satan, to corrupt the simplicity of faith, entangleth the mind of the woman with an argument drawn from common experience to the contrary: “A woman that is old! Sara now to be acquainted again with forgotten passions of youth4!”

The word of the promise of God by Moses and the prophets made the Saviour of the world so apparent unto Philip, that his simplicity could conceive no other Messias than Jesus of Nazareth the son of Joseph. But to stay Nathanael, lest being invited to come and see, he should also believe, and so be saved, the subtilty of Satan casteth a mist before his eyes, putteth in his head against this the common-conceived persuasion of all men concerning Nazareth: “Is it possible that a good thing should come from thence5?”

This stratagem he doth use with so great dexterity, the minds of all men are so strangely ensorceled6 with it, that it bereaveth them for the time of all perceivance of that which should relieve them and be their comfort; yea it taketh all remembrance from them, even of things wherewith they are most familiarly acquainted. The people of Israel could not be ignorant, that he which led them through the sea was able to feed them in the desert: but this was obliterated and put [478] out by the sense of their present want. Feeling the hand of God against them in their food, they remembered not his hand in the day that he delivered them from the hand of the oppressor. Sara was not then to learn, that “with God all things were possible1.” Had Nathanael never noted how “God doth choose the base things of this world to disgrace them that are most honourably esteemed2?”

The prophet Habakkuk knew that the promises of grace, protection, and favour, which God in the law doth make unto his people, do not grant them any such immunity as can free and exempt them from all chastisements: he knew that as God said, “I will continue my mercy for ever towards them;” so he likewise said, “Their transgressions I will punish with a rod3:” he knew that it cannot stand with any reason we should set the measure of our own punishments, and prescribe unto God how great or how long our sufferings shall be: he knew that we were blind, and altogether ignorant what is best for us; that we sue for many things very unwisely against ourselves, thinking we ask fish when indeed we crave a serpent: he knew that when the thing we ask is good, and yet God seemeth slow to grant it, he doth not deny but defer our petitions, to the end we might learn to desire great things greatly: all this he knew. But, beholding the land which God had severed for his own people, and seeing it abandoned unto heathen nations; viewing how reproachfully they did tread it down, and wholly make havock of it at their pleasure; beholding the Lord’s own royal seat made a heap of stones, his temple defiled, the carcasses of his servants cast out for the fowls of the air to devour, and the flesh of his meek ones for the beasts of the field to feed upon; being conscious to himself how long and how earnestly he had cried, “Succour us, O God of our welfare, for the glory of thine own name4;” and feeling that their sore was still increased: the conceit of repugnancy between this which was object to his eyes, and that which faith upon promise of the law did look for, made so deep an impression and so strong, that he disputeth not the matter; but without any further inquiry or search inferreth, as we see, “The law doth fail.”

[479]
Of us who is here which cannot very soberly advise his brother? Sir, you must learn to strengthen your faith by that experience which heretofore you have had of God’s great goodness towards you: Per ea quæ agnoscas præstita, discas sperare promissa; “By those things which you have known performed, learn to hope for those things which are promised.” Do you acknowledge to have received much? Let that make you certain to receive more: Habenti dabitur; “To him that hath more shall be given.” When you doubt what you shall have, search what you have had at God’s hands. Make this reckoning, that the benefits, which he hath bestowed, are bills obligatory and sufficient sureties that he will bestow further. His present mercy is still a warrant of his future love, because, “whom he loveth, he loveth unto the end1.” Is it not thus?

Yet if we could reckon up as many evident, clear, undoubted signs of God’s reconciled love towards us as there are years, yea days, yea hours, past over our heads; all these set together have not such force to confirm our faith, as the loss, and sometimes the only fear of losing a little transitory goods, credit, honour, or favour of men, a small calamity, a matter of nothing to breed a conceit, and such a conceit as is not easily again removed, that we are clean crost out of God’s book, that he regards us not, that he looketh upon others, but passeth by us like a stranger to whom we are not known. Then we think, looking upon others, and comparing them with ourselves, Their tables are furnished day by day; earth and ashes are our bread: they sing to the lute, and they see their children dance before them; our hearts are heavy in our bodies as lead, our sighs beat as thick as a swift pulse, our tears do wash the beds wherein we lie: the sun shineth fair upon their foreheads; we are hanged up like bottles in the smoke, cast into corners like the sherds of a broken pot: tell not us of the promises of God’s favour, tell such as do reap the fruit of them; they belong not to us, they are made to others. The Lord be merciful to our weakness, but thus it is.

Well, let the frailty of our nature, the subtilty of Satan, the force of our deceivable imaginations be, as we cannot [480] deny but they are, things that threaten every moment the utter subversion of our faith; faith notwithstanding is not hazarded by these things. That which one sometimes told the senators of Rome1, Ego sic existimabam, P. C. uti patrem sæpe meum prædicantem audiveram, qui vestram amicitiam diligenter colerent, eos multum laborem suscipere, cæterum ex omnibus maxime tutos esse; “As I have often heard my father acknowledge, so I myself did ever think, that the friends and favourers of this state charged themselves with great labour, but no man’s condition so safe as theirs;” the same we may say a great deal more justly in this case: our Fathers and Prophets, our Lord and Master hath full often spoken, by long experience we have found it true; as many as have entered their names in the mystical Book of Life, Eos maximum laborem suscipere, they have taken upon them a laboursome, a toilsome, a painful profession, sed omnium maxime tutos esse, but no man’s security like to theirs. “2Simon, Simon, Satan hath desired to winnow thee as wheat;” here is our toil: “but I have prayed for thee, that thy faith fail not;” this is our safety. No man’s condition so sure as ours: the prayer of Christ is more than sufficient both to strengthen us, be we never so weak; and to overthrow all adversary power, be it never so strong and potent. His prayer must not exclude our labour. Their thoughts are vain who think that their watching can preserve the city which God himself is not willing to keep. And are not theirs as vain, who think that God will keep the city, for which they themselves are not careful to watch? The husbandman may not therefore burn his plough, nor the merchant forsake his trade, because God hath promised “I will not forsake thee.” And do the promises of God concerning our stability, think you, make it a matter indifferent for us to use or not to use the means whereby, to attend or not to attend to reading, to pray or not to pray that we “fall not into temptation?” Surely if we look to stand in the faith of the sons of God, we must hourly, continually, be providing and setting ourselves to strive. It was not the meaning of our Lord and Saviour in saying3, “Father, keep them in thy name,” that we [481] should be careless to keep ourselves. To our own safety, our own sedulity is required. And then blessed for ever and ever be that mother’s child whose faith hath made him the child of God. The earth may shake, the pillars of the world may tremble under us, the countenance of the heaven may be appalled1, the sun may lose his light, the moon her beauty, the stars their glory; but concerning the man that trusteth in God, if the fire have proclaimed itself unable as much as to singe a hair of his head, if lions, beasts ravenous by nature and keen with hunger, being set to devour, have as it were religiously adored the very flesh of the faithful man; what is there in the world that shall change his heart, overthrow his faith, alter his affection towards God, or the affection of God to him? If I be of this note, who shall make a separation between me and my God? “Shall tribulation, or anguish, or persecution, or famine, or nakedness, or peril, or sword2?” No; “I am persuaded that neither tribulation, nor anguish, nor persecution, nor famine, nor nakedness, nor peril, nor sword, nor death, nor life, nor angels, nor principalities, nor powers, nor things present, nor things to come, nor height, nor depth, nor any other creature,” shall ever prevail so far over me. “I know in whom I have believed;” I am not ignorant whose precious blood hath been shed for me; I have a Shepherd full of kindness, full of care, and full of power: unto him I commit myself; his own finger hath engraven this sentence in the tables of my heart, “Satan hath desired to winnow thee as wheat, but I have prayed that thy faith fail not.” Therefore the assurance of my hope I will labour to keep as a jewel unto the end; and by labour, through the gracious mediation of his prayer, I shall keep it.

[482]
TO THE CHRISTIAN READER1.
WHEREAS many, desirous of resolution in some points handled in this learned discourse, were earnest to have it copied out; to ease so many labours, it hath been thought most worthy and very necessary to be printed: that not only they might be satisfied, but the whole Church also hereby edified. The rather, because it will free the author from the suspicion of some errors, which he hath been thought to have favoured. Who might well have answered with Cremutius in Tacitus, “Verba mea arguuntur; adeo factorum innocens sum2.” Certainly, the event of that time, wherein he lived, shewed that to be true, which the same author spake of a worse, “Cui deerat inimicus, per amicos oppressus3;” and that there is not “minus periculum ex magna fama, quam ex mala.” But he hath so quit himself, that all may see how, as it was said of Agricola, “Simul suis virtutibus, simul vitiis aliorum, in ipsam gloriam præceps agebatur4.” Touching whom I will say no more, but that which my author said of the same man, “Integritatem, &c. in tanto viro referre, injuria virtutum fuerit.” But as of all other his writings, so of this I will add that, which Velleius spake in commendation of Piso, “Nemo fuit, qui magis quæ agenda erant curaret, sine ulla ostentatione agendi5.” So not doubting, good Christian reader, of thy assent herein, but wishing thy favourable acceptance of this work, (which will be an inducement to set forth others of his learned labours,) I take my leave; from Corpus Christi College in Oxford, the 6th of July, 1612.

Thine in Christ Jesus,
HENRY JACKSON.
[483]
A LEARNED DISCOURSE OF JUSTIFICATION, WORKS, AND HOW THE FOUNDATION OF FAITH IS OVERTHROWN1.
Habak. i. 4.
“The wicked doth compass about the righteous: therefore perverse judgment doth proceed.”

SERM. II. 1.FOR bettera manifestation of the prophet’s meaning in this place, we are, first, to consider “the wicked,” of whom he saith, that they “compass about the righteous:” secondly, “the righteous” that are compassed about by them: and thirdly, that which is inferred; “thereforeb perverse judgment proceedeth.” Touching the first, there are two kinds of wicked men, of whom in the fifth of the former to the Corinthiansc, the blessed Apostle speaketh thus2: “Do ye [484] not judge them that are within?SERM. II. 2. but God judgeth them that are without.” There are wicked, therefore, whom the Church may judge, and there are wicked whom God only judgeth; wicked within, and wicked without, the walls of the Church. If within the Church particular persons, being apparently such, cannotd otherwise be reformed, the rule of apostolicale judgment is this1, “Separate them from amongf you:” if whole assemblies, this, “Separate yourselves from amongf them: for what society hath light with darkness?” But the wicked, whom the prophet meaneth, were Babylonians, and therefore without. For which cause we have heard at large heretofore in what sort he urgeth God to judge them.

2. Now concerning the righteous, there neither is, nor everh was, any mere natural man absolutely righteous in himself: that is to say, void of all unrighteousness, of all sin. We dare not except, no not the blessed Virgin herself; of whom although we say with St. Augustine2, for the honour’si sake which we owe to our Lord and Saviour Christ, we are not willing, in this cause, to move any question ofk his mother; yet forasmuch as the schools of Rome have made it a question, we mustl answer with Eusebius Emissenus3, who speaketh of her, and to her tom this effect: “Thou didst by special prerogative nine months together entertain within the closet of thy flesh the hope of all the ends of the earth, the honour of the world, the common joy of men. He, from whom all [485] things had their beginning,SERM. II. 3. hathn had his owno beginning from thee; of thy body he took the blood which was to be shed for the life of the world; of thee he took that which even for thee he paid. ‘A peccati enim veteris nexu, per se non est immunis nec ipsa genitrix Redemptoris1:’ The mother of the Redeemer herself, otherwise than by redemption, is not loosed from the bandp of that ancient sinq2.” If Christ have paid a ransom for all, even for her, it followeth, that all without exception were captives. If one have died for all, allr were dead, dead in sins; all sinful, therefore none absolutely righteous in themselves; but we are absolutely righteous in Christ. The world then must shew a Christiant man, otherwise it is not able to shew a man that is perfectly righteous: “Christ is made unto us wisdom, justice, sanctification, and redemption3:” wisdom, because he hath revealed his Father’s will; justice, because he hath offered himselfu a sacrifice for sin; sanctification, because he hath given us ofx his Spirit; redemption, because he hath appointed a day to vindicate his children out of the bands of corruption into liberty which is glorious4. How Christ is made wisdom, and how redemption, it may be declared when occasion serveth; but how Christ is made the righteousness of men, we are now to declare.

3. There is a glorifying righteousness of men in the world to come: andy there is a justifying and a sanctifying righteousness here. The righteousness, wherewith we shall be clothed in the world to come, is both perfect and inherent. That whereby here we are justified is perfect, but not inherent. That whereby we are sanctified, inherentz, but not perfect. [486] This openeth a way to the plainz understanding of that grand question,SERM. II. 4, 5. which hangeth yet in controversy between us and the Church of Rome, about the matter of justifying righteousness.

4. First, although they imagine that the mother of our Lord and Saviour Jesus Christ were, for his honour, and by his special protection, preserved clean from all sin, yet touching the rest, they teach as we do, that all have sinneda; that infants which did neverb actually offend, have their natures defiled, destitute of justice, and averted from God. They teach as we do, that God doth justify the soul of man alone, without any other coefficient cause of justicec; that in making man righteous, none do work efficientlyd with God, but God1. They teach as we do, that unto justice no man ever attained, but by the merits of Jesus Christ. They teach as we do, that although Christ as God be the efficient, as man the meritorious cause of our justice; yet in us also there is somethinge required. God is the cause of our natural life; in him we live: but he quickeneth not the body without the soul in the body. Christ hath merited to make us just: but as a medicine which is made for health, doth not heal by being made, but by being applied; so, by the merits of Christ there can be no justification, without the application of his merits. Thus far we join hands with the Church of Rome.

The difference betwixt the Papists and us about Justification.5. Wherein then do we disagree? We disagree about the nature of the very essencef of the medicine whereby Christ cureth our disease; about the manner of applying it; about the number and the power of means, which God requireth in us for the effectual applying thereof to our soul’s comfort. [487] When they are required to shew, what the righteousness is whereby a Christian man is justified, they answer1, that it is a divine spiritual quality;SERM. II. 5. which quality received into the soul, doth first make it to be one of them who are born of God: and, secondly, endue it with power to bring forth such works, as they do that are born of him; even as the soul of man being joined untog his body, doth first make him to be inh the number of reasonable creatures, and secondly enablei him to perform the natural functions which are proper to his kind; that it maketh the soul gracious and amiablek in the sight of God, in regard whereof it is termed Grace; that it purgeth, purifieth, washeth outl, all the stains and pollutions of sinm; that by it, through the merit of Christ we are delivered as from sin, so from eternal death and condemnation, the reward of sin. This grace they will have to be applied by infusion; to the end, that as the body is warm by the heat which is in the body, so the soul might be righteous by the inherent grace: which grace they make capable of increase; as the body may be more and more warm, so the soul more and more justified2, according as grace shall be augmented; the [488] augmentation whereof is merited by good works1, as good works are made meritorious by it2. Wherefore, the first receipt of grace is in their divinityn the first justification; the increase thereof, the second justification3. As grace may be increased by the merit of good works; so it may be diminished by the demerit of sins venial4; it may be lost by mortal sin5. Inasmuch, therefore, as it is needful in the one case to repair, in the other to recover, the loss which is made; the infusion of grace hath her sundry after-meals; for whicho cause they make many ways to apply the infusion of grace. It is applied to infants6 through baptism, without either faith or works, and in them really it taketh away original sin, and the punishment due unto it; it is applied to infidels and wicked men in their firstp justification through baptism, without works7, yet not without faith; and it taketh away both sins actual and original, together with all whatsoever punishments, eternal or temporal, thereby deserved8. Unto such as have attained the first justification, that is to say, the first receipt of grace, it is applied farther by good works to the increase of former grace, which is the second justification. If they work more and more, grace doth more and more [489] increase, and they are more and more justified. To such as have diminishedq it by venial sins, it is applied by holy water, Ave Maries, crossings, papal salutations1, and such like, which serve for reparations of grace decayed. To such as have lost it through mortal sin, it is applied by the sacrament (as they term it) of penance; which sacrament hath force to confer grace anew2, yet in such sort, that being so conferred, it hath not altogether so much power3 as at the first. Forr it only cleanseth out the stain or guilt of sin committed, and changeth the punishment eternal into a temporal satisfactory punishment, here, if time do serve, if not, hereafter to be endured, except it be eithers lightened by masses, works of charity, pilgrimages, fasts, and such like; or else shortened by pardon for term, or by plenary pardon quite removed and taken away4. This is the mystery of the Man of sin. This maze the Church of Rome doth cause her followers to tread, when they ask her the way of justificationt. I cannot stand [490] now to unrip this building,SERM. II. 6. and to sift it piece by piece; only I will set a frame of apostolical erection by it in few wordsu, that itx may befall Babylon, in presencey of that which God hath builded, as itz happened unto Dagon before the ark.

6. “Doubtless,” saith the Apostle1, “I have counted all things lossa, and I dob judge them to be dung, that I may win Christ; and be foundc in him, not having mine own righteousness, but that which is through the faith of Christ, the righteousness which is of God through faith.” Whether they speak of the first or second justification, they make the essence of itd a divine quality inherent, they make it righteousness which is in us. If it be in us, then it ise ours, as our souls are ours, though we have them from God, and can hold them no longer than pleaseth him; for if he withdraw the breath of our nostrils, we fall to dust: but the righteousness wherein we must be found, if we will be justified, is not our own; therefore we cannot be justified by any inherent quality. Christ hath merited righteousness for as many as are found in him. In him God findeth us, if we be faithful; for by faith we are incorporated into himf. Then, although in ourselves we be altogether sinful and unrighteous, yet even the man which in himself is impiousg, full of iniquity, full of sin; him being found in Christ through faith, and having his sin in hatredh through repentance; him God beholdethi with a gracious eye; putteth away his sin by not imputing itk; taketh quite away the punishment due thereunto, by pardoning it; and accepteth him in Jesus Christ, as perfectly righteous, as if he had fulfilled all that isl commanded him in the law: shall I say more perfectly righteous than if himself had fulfilled the whole law? I must take heed what I say: but the Apostle saith2, “God made him which knew no sin, to be sin for usm; that we might be made the righteousness of God in him.” Such we are in the sight of God the Father, as is the very Son of God himself. Let it be counted folly, or phrensy, or [491] fury, orn whatsoever.SERM II. 7. It is our wisdom, and our comforto; we care for no knowledge in the world but this, That man hath sinned, and God hath suffered; that God hath made himself the sinp of menq, and that men are made the righteousness of God.

You see therefore that the church of Rome, in teaching justification by inherent grace, doth pervert the truth of Christ; and that by the hands of hisr Apostles we have received otherwise than she teacheth. sNow concerning the righteousness of sanctification, we deny it not to be inherent; we grant, that unlesst we work, we have it not; only we distinguish it asu a thing in nature different from vthe righteousness of justification: we are righteous the one way, by the faith of Abraham; the other way, except we do the works of Abraham, we are not righteous. Of the one, St. Paul1, “To him that worketh not, but believeth, faith is counted for righteousness.” Of the other, St. John2, “Qui facit justitiam, justus est:—He is righteous which worketh righteousness.” Of the one, St. Paul3 doth prove by Abraham’s example, that we have it of faith without works. Of the other, St. James4 by Abraham’s example, that by works we have it, and not only by faith. St. Paul doth plainly sever these two parts of Christian righteousness one from the other. For in the sixth to the Romans thus he writeth5, “Being freed from sin, and made servants tow God, yex have your fruit in holiness, and the end everlasting life.” “Ye are made free from sin, and made servants unto God;” this is the righteousness of justification: “Ye have your fruit in holiness;” this is the righteousness of sanctification. By the one we are interessed in the right of inheriting; by the other we are brought to the actual possessingy of eternal bliss, and so the end of both is everlasting life.

7. The Prophet Habakkukz doth here term the Jews “righteous men,” not only because being justified by faith they were free from sin; but also for thata they had their measure of fruitb in holiness. According to whose example [492] of charitable judgment, which leaveth it to God to discern what menb are, and speaketh of them according to that which they do professc themselves to be, although they be not holyd whom men do think, but whom God doth know indeed to be such; yet let every Christian man know, that in Christian equity, he standeth bound soe to think and speak of his brethren, as of men that have af measure in the fruit of holiness, and a right unto the titles wherewith God, in token of special favour and mercy, vouchsafeth to honour his chosen servants. So we see the Apostles of our Saviour Christ do use every where the name of saints; so the prophet the name of righteous. But let us all endeavour tog be such as we desire to be termed: Reatus impii est pium nomen, saith Salvianus1; “Godly names do not justify godless men.” We are but upbraided, when we are honoured with names and titles whereunto our lives and manners are not suitable. If we have indeedh our fruit in holiness, notwithstanding we must note, that the more we abound thereini, the more need we have to crave that we may be strengthened and supported. Our very virtues may be snares unto us. The enemy that waiteth for all occasions to work our ruin, hath everj found it harder to overthrow an humble sinner, than a proud saint. There is no man’s case so dangerous as his, whom Satan hath persuaded that his own righteousness shall present him pure and blameless in the sight of God. If we could say, “we arek not guilty of any thing at all in our ownl consciences,” (we know ourselves far from this innocency; we cannot say, we know nothing by ourselves; but if we could,) should we therefore plead not guilty inm the presence of our Judge, that sees furthern into our hearts than we ourselves are able to seeo? If our hands did never offer violence to our brethren, a bloody thought doth prove us murderers before him: if we had never opened our mouthsp to utter any scandalous, offensive, or hurtful word, the cry of our secret cogitations is heard in the ears of God. If we didq not commit the evils which we do daily and hourly, either in [493] deeds, words, or thoughtsr, yet in the good things which we do, how many defects are there intermingled! God, in that which is done, respecteth speciallys the mind and intention of the doer. Cut off then all those things wherein we have regarded our own glory, those things which wet do to please men, oru to satisfy our own liking, those things which we do with any by-respectw, not sincerely, and purely for the love of God; and a small score will serve for the number of our righteous deeds. Let the holiest and best thing we do be considered. We are never better affected unto God than when we pray; yet when we pray, how are our affections many times distracted! How little reverence do we shew to the grand majesty of thatx God, unto whom we speak! How little remorse of our own miseries! How little taste of the sweet influence of his tender mercyy do we feel! Are we not as unwilling many times to begin, and as glad to make an end, as if God zin saying, “Call upon me,” had aset us a very burdensome task?

It may seem somewhat extreme, which I will speak; therefore let every manb judge of it, even as his own heart shall tell him, and no otherwise; I will but only make a demand: If God should yield to us, not as unto Abraham, if fifty, forty, thirty, twenty, yea, or if ten good persons could be found in a city, for their sakes thatc city should not be destroyed; but, if Godd should make us an offer thus large, Search all the generations of men sithence the fall of youre father Adam, find one man, that hath done anyf one action, which hath past from himg pure, without any stain or blemish at all; and for that one man’s oneh only action, neither man nor angel shall feel the torments which are prepared for both: do you think that this ransom, to deliver men and angels, would be foundi among the sons of men? The best things we doj have somewhat in them to be pardoned. How then can we do any thing meritorious, andk worthy to be rewarded? Indeed, God doth liberally promise whatsoever appertaineth to a blessed life, unto as many as sincerely keep his law, though they be not able [494] exactlyl to keep it.SERM. II. 8, 9. Wherefore, we acknowledge a dutiful necessity of doing well; but the meritorious dignity of well doingm we utterly renounce. We see how far we are from the perfect righteousness of the law; the little fruit which we have in holiness, it is, God knowethn, corrupt and unsound: we put no confidence at all in it, we challenge nothing in the world for it, we dare not call God to a reckoningo, as if we had him in our debt-books: our continual suit to him is, and must be, to bear with our infirmities, to ppardon our offences.

8. But the people of whom the Prophet speaketh, were they all, or were the most part of them, such as had care to walk uprightly? did they thirst after righteousness? did they wish, did they long with the righteous Prophet1, “O that our ways were made so direct that we might keep thy statutes?” did they lament with the righteous Apostle2, “Miserable men, the good which we wish and purpose, and strive to do, we cannot?” No; the words of other prophetsq concerning this people do shew the contrary. How grievously doth Esay mourn over them3! “Ah sinful nation, people laden with iniquity, wicked seed, corrupt children!” All which notwithstanding, so wide are the bowels of his compassion enlarged, that he denieth us not, no not when we arer ladens with iniquity, leave to commune familiarly with him, liberty to crave and entreat, that what plagues soever we have deserved, we may not be in worse case than unbelievers, that we may not be hemmed in by pagans and infidels. Jerusalem is a sinful polluted city; but Jerusalem compared with Babylon is righteous. And shall the righteous be overborne, shall they be compassed about by the wicked? But the prophet doth not only complain; Lord, how cometh it to pass that thou handlest us so hardly, overt whom thy name is called, and bearest with the heathen nations, that despise thee? no, he breaketh out through extremity of grief, and inferreth thusu violently, This proceeding is perverse; the righteous are thus handled, “therefore perverse judgment doth proceed.”

9. Which illation containeth many things, whereof it were [495] better much both for youx to hear, and me to speak,SERM. II. 9. if necessity did not draw me to another tasky. Paul and Barnabas being requested1 to preach the same things again which once they had preached, thought it their duties to satisfy the godly desires of men sincerely affected towardsz the truth. Nor may it seem burdenous to me, or for youa unprofitable, that I follow their example, the like occasion unto theirs being offered me. When we had last the Epistle of St. Paul to the Hebrews in our handsb, and of that epistle these words2, “In these last days he hath spoken unto us by his Son;” after we had thence collected the nature of the visible Church of Christ, and had defined it to be a community of men3 sanctified through the profession of that truth which God hath taught the world by his Son; and had declared, that the scope of Christian doctrine is the comfort of them whose hearts are overcharged with the burden of sin; and had proved that the doctrine professed in the church of Rome doth bereave men of comfort, both in their lives, and atc their deaths: the conclusion in the end, whereunto we camed, was this; “The church of Rome, being in faith so corrupted, as she is, and refusing to be reformed, as she doth, we are to sever ourselves from her: the example of our fathers may not retain us in communion and fellowshipe with that church, under hope that we so continuing, mightf be saved as well as they. God, I doubt not, was merciful to save thousands of them, though they lived in popish superstitions, inasmuch as they sinned ignorantly: but the truth is now laid openg before our eyes.” The former part of this last sentence, namely, these words, “I doubt not but God was merciful to save thousands of our fathers living in popish superstitions, inasmuch as they sinned ignorantly:” this sentence I beseech you to mark, and to sift it with the stricth severity of austere judgment, that if it be found as goldi, it may stand, suitablek [496] to the precious foundation whereupon it was then laid;SERM. II. 10. for I protest, that if it prove tor be hay or stubble, my own hand shall set fire to its. Two questions have risen by occasion of thet speech before alleged: the one, “Whether our fathers, infected with popish errors and superstitions, mightu be saved:” the other, “Whether their ignorance be a reasonable inducement to make us think thatx they might.” We are thereforey to examine, first, what possibility, andz then, what probability there is, that God might be merciful unto so many of our fathers.

10. So many of our fathers living in popish superstitions, yet by the mercy of God to be saved? No; this could not be: God hath spoken by his angel from heaven unto his people concerning Babylon (by Babylon we understand the church of Rome)1: “Go out of her, my people, that ye be not partakers of her sins, and that ye receive not of her plaguesa.” For answer whereunto, first, I do not take the words to be meant only of temporal plagues, of the corporal death, sorrow, famine, and fire, whereunto God in his wrath hadb condemned Babylon; and that to save his chosen people from these plagues, he saith, “Go out;” with like intent, as in the Gospel, speaking of Hierusalem’s desolations, he saith2, “Let them that are in Judea flee unto the mountains, and them which are in the midst thereof depart out;” or, as in former times unto Lot3, “Arise, take thy wife and thy daughters which are here, lest thou be destroyed in the punishment of the city:” but forasmuch as here it is said, “Go out of Babylon, that ye be not partakers of her sins, and by consequence of her plagues;” plagues eternal being due to the sins of Babylonc; no doubtd, their everlasting destruction, which are partakers herein, is either principally meant, or necessarily implied in this sentence. How then was it possible for so many of our fathers to be saved, sith they were so far from departing out of Babylon, that they took her for their mother, and in her bosom yielded up the ghost?

[497]
SERM. II. 11.11. First, the plaguese being threatened unto them that are partakers in the sins of Babylon, we can define nothing concerning our fathers out of this sentence; unless we shew what the sins of Babylon be, and whof they be thatg are such partakers inh them, that their everlasting plagues are inevitable. The sins which may be common both to them of the church of Rome, and toi others departed thence, must be severed from this question. He which saith, “Depart out of Babylon, lest you be partakers of her sins,” sheweth plainly, that he meaneth such sins, as except we separate ourselves, we have no power in the world to avoidj; such impieties, as by lawk they have established, and whereunto all that are amongl them, either do indeed assent, or else are by powerable means forced in show and inm appearance to subject themselves. As for example, in the church of Rome, it is maintained, that the same1 credit and reverence whichn we give to the Scriptures of God, ought also to be given to unwritten verities; that the pope is supreme head ministerial2 over the universal Church militant; that the bread in the Eucharist is transubstantiated3 into Christ; that it is to be adored4, and to be [498] offered up unto God as a sacrifice propitiatory1 for quick and dead;SERM. II. 12. that images are to be worshipped, saints to be called upon as intercessors2, and such like. Now, because some heresies do concern things only believed, as transubstantiating ofn sacramental elements in the Eucharist; some concern things which are practised alsoo and put in ure, as adorationp of the elements transubstantiated: we must note that erroneously the practice of that is sometime received, whereof the doctrine whichq teacheth it is not heretically maintained. They are all partakers in the maintenance of heresies, who by word or deed allow them, knowing them, although not knowing them to be heresies; as also they, and that most dangerously of all others, who knowing heresy to be heresy, do notwithstanding, in worldly respects, make semblance of allowing that, which in heart and inr judgment they condemn: but heresy is heretically maintained, by such as obstinately hold it after wholesome admonition. Of the last sort, as alsos of the next before, I make no doubt, but that their condemnation, without actualt repentance, is inevitable. Lest any man therefore should think, that in speaking of our fathers, I speaku indifferently of them all; let my words, I beseech you, be well notedx, “I doubt not but God was merciful to save thousands of our fathers;” which thing I will now by God’s assistance set more plainly before your eyes.

12. Many are partakers of the error, which are not ofy the heresy of the church of Rome. The people following the conduct [499] of their guides, and observing as they did,SERM. II. 13. exactly that which was prescribed themy, thought they did God good service, when indeed they did dishonour him. This was their error: but the heresiesz of the Church of Rome, their dogmatical positions opposite unto Christian truth, what one man amongst ten thousand did ever understand? Of them, which understand Roman heresies, and allow them, all are not alike partakers in the action of allowing. Some allow them as the first founders and establishers of thema; which crime toucheth none but their Popes and Councils: the people are clear and freeb from this. Of them which maintain popish heresyc not as authors, but receivers of it from others, all maintain it not as Masters. In this are not the people partakers neither, but only their Predicants and theird Schoolmen. Of them which have been partakers in thee sin of teaching popish heresy, there is also a difference; for they have not all been teachers of all popish heresies. “Put a difference,” saith St. Jude1; “have compassion upon some.” Shall we lapf up all in one condition? shall we cast them all headlong, shall we plunge them all ing that infernal and ever-flamingh lake? them that have been partakers ini the errorj of Babylon, together with them withink the heresy? them which have been the authors of heresy, with them that by terror and violence have been forced to receive it? them which have taught it, with theml whose simplicity hath by sleights and conveyances of false teachers been seduced to believe it? them which have been partakers in one, with them whichm have been partakers in many? them which in many, with them which in all?

13. Notwithstanding I grant, that although the condemnation of onen be more tolerable than of anothero; yet from the man that laboureth at the plough, to him that sitteth in the Vatican; to all partakers in the sins of Babylon, our fathersp, though they did but erroneously practise that which their guides did heretically teachq; to all without exception, plagues worldlyr were due. The pit is ordinarily the end, as well of [500] the guided as the guides in blindness.SERM. II. 14. But woe worth the hour wherein we were born, except we might persuadet ourselves better things; things that accompany men’su salvation, even where we know that worse and such as accompany condemnation are due. Then must we shew some way how possibly they might escape. What way is there for sinners tox escape the judgment of God, but only by appealing to the seat of his saving mercy? Which mercy we do not with Origen extend to devils and damned spirits. God hath mercy upon thousands, but there be thousands also which he hardenedy. Christ hath therefore set the bounds, he hath fixed the limits of his saving mercy, within the compass of these twoz terms. In the thirda of St. John’s Gospel, mercy is restrained to believers1: “God sent not his Sonb to condemn the world, but that the world through him might be saved.” “2He that believeth shall not be condemned: he that believeth not, is condemned already, because he believed not in the Son of God.” In the second of the Revelation, mercy is restrained to the penitent. For of Jezebel and her sectaries thus he speaketh3: “I gave her space to repent, and she repented not. Behold, I will cast her into a bed, and them that commit fornication with her, into a great affliction, except they repent them of their works; and I will kill her children with death.” Our hope therefore of the fathers is vain, if they were altogether faithless and impenitentc.

14. They bed not all faithless that are eithere weak in assenting to the truth, or stiff in maintaining things any wayf opposite to the truth of Christian doctrine. But as many as hold the foundation which is precious, though they hold it but weakly, and as it were byg a slender thread, although they frame many base and unsuitable things upon it, things that cannot abideh the trial of the fire; yet shall they pass the fieryi trial and be saved, which indeed have builded themselves upon the rock, which is the foundation of the Church. [501] If then our fathers did not hold the foundation of faith,SERM. II. 15, 16. there is no doubt but they were faithless. If many of them held it, then is there hereink no impediment, but that1 many of them might be saved. Then let us see what the foundation of faith is, and whether we may think that thousands of our fathers livingm in popish superstitions, did notwithstanding hold the foundation.

15. If the foundation of faith do import the general ground whereupon we rest when we do believe, the writings of the Evangelists and the Apostles are the foundation of the Christian faith: “Credimus quia legimus,” saith St. Jerome1. O that the church of Rome did as2 soundly interpret thosen fundamental writings whereupon we build our faith, as she doth willingly hold and embrace them!

16. But if the name Foundationo do note the principal thing which is believed, then is that the foundation of our faith which St. Paul hath unto Timothy: “God manifested in the flesh, justified in the Spirit,” &c.3: that of Nathaniel, “Thou art the Son of the living God: thou art the king of Israel4:” that of the inhabitants of Samaria, “This is Christ the Saviour of the world:” he that directly denieth this, doth utterly razep the very foundation of our faith. I have proved heretofore, that although the church of Rome hath played the harlot worse than ever did Israel, yet are they not, as now the synagogue of the Jews, which plainly deniethq Christ Jesus, quite and clean excluded from the new covenant. But as Samaria compared with Hierusalem is termed Aholath, a church or tabernacle of her own; contrariwise, Jerusalem Aholibath, the resting place of the Lord: so, whatsoever we term the church of Rome, when we compare her tor reformed churches, still we put a difference, as then between Babylon and Samaria, as now between Rome and [502] heathenishs assemblies.SERM. II. 17. Which opinion I must and will recall; I must grant, and will, that the church of Rome, together with all her children, is clean excluded; there is no difference in the world between our fathers and Saracens, Turks, ort Painims, if they did directly deny Christ crucified for the salvation of the world.

17. But how many millions of them areu known so to have ended their mortalx lives, that the drawing of their breath hath ceased with the uttering of this faith, “Christ my Saviour, my Redeemer Jesus!” And shall we say that such did not hold the foundation of Christian faithy?

Answer is made, that this they might unfeignedly confess, and yet be far enough from salvation. For behold, saith the Apostle, “I, Paul, say unto you, that if yez be circumcised, Christ shall profit you nothing1.” Christ, in the work of man’s salvation, is alone: the Galatians were cast away by joining circumcision and othera rites of the law with Christ: the church of Rome doth teach her children to join other things likewise with him; therefore their faith, their belief, doth not profit them any thing at all.

It is true, theyb do indeed join other things with Christ; but how? Not in the work of redemption itself, which they grant that Christ alone hath performed sufficiently for the salvation of the whole world; but in the application of this inestimable treasure, that it may be effectual to their salvation: how demurely soever they confess that they seek remission of sins no otherwise than by the blood of Christ, using humbly the means appointed by him to apply the benefit of his holy blood; they teach, indeed, so many things pernicious toc Christian faith, in setting down the means whereof they speak, that the very foundation of faith which they hold, is thereby plainly overthrown2, and the force of the blood of Jesus [503] Christ extinguished.SERM. II. 18. We may therefore dispute with them, press them, urgec them even with as dangerous sequels as the Apostle doth the Galatians. But I demand, if some of those Galatians, heartily embracing the Gospel of Christ, sincere and sound in faith, this onlyd error excepted, had ended their lives before they were ever taught how perilous an opinion they held; shall we think that the damage of this error did so overweigh the benefit of their faith, that the mercy of God, his mercye, might not save them? I grant they overthrew the very foundation of faith by consequent: doth not that so likewise which the1 Lutheran churches do at this day so stiffly and so fiercelyf maintain? For mineg own part, I dare not hereuponh deny the possibility of their salvation, which have been the chiefest instruments of ours, albeit they carried to their grave a persuasion so greatly repugnant to the truth. Forasmuch therefore, as it may be said of the church of Rome, she hath yet “a little strength2,” she doth not directly deny the foundation of Christianity: I may, I trust without offence, persuade myself, that thousands of our fathers in former times, living and dying within her walls, have found mercy at the hands of God.

18. What although they repented not of their errors? God forbid that I should open my mouth to gainsay that which Christ himself hath spoken: “Except ye repent, ye shall all perish3.” And if they did not repent, they perished. But withal note, that we have the benefit of a double repentance: [504] the least sin which we commit in deed, word, or thoughti, is death, without repentance.SERM. II. 10. Yet how many things do escape us in every of these, which we do not know, how many, which we do not observe to be sins! and without the knowledge, without the observation of sin, there is no actual repentance. It cannot then be chosen, but that for as many as hold the foundation, and have all known sin and errorj in hatred, the blessing of repentance for unknown sins and errorsk is obtained at the hands of God, through the gracious mediation of Christ Jesus, for such suitors as cry with the prophet David, “Purge me, O Lord, from my secret sinsl.”

19. But we wash a wall of loam; we labour in vain; all this is nothing; it doth not prove, it cannot justify, that which we go about to maintain. Infidels and heathen men are not so godless, but that they may, no doubt, cry God mercy, and desire in general to have their sins forgiven them. To such as deny the foundation of faith, there can be no salvation, according to the ordinary course which God doth use in saving men, without a particular repentance of that error. The Galatians, thinking that except1 they were circumcised, they could not be saved, overthrew the foundations of faith directly: therefore if any of them did die so persuaded, whether before or after they werem told of their errorn, their caseo is dreadful; there is no way with them but one, death and condemnation. For the Apostle speaketh nothing of men departed, but saith generally of all, “If ye be circumcised, Christ shall profit you nothing. Ye are abolished from Christ, whosoever are justified by the law; ye are fallen from grace2.” Of them in the church of Rome the reason is the same. For whom Antichrist hath seduced, concerning them did not St. Paul speak long before, “That becausep they received not the love of the truthq, that they might be saved; therefore God would send them strong delusions to believe lies, that all they might be damned which believed not the truth, but had pleasure in unrighteousness3?” And St. John, “All that dwell upon the earth shall worship him, [505] whose names are not written in the Book of Life1?”SERM. II. 20. Indeed many of themr in former times, as their books and writings do yet shew, held the foundation, to wit, salvation by Christ alone, and therefore might be saved. Fors God hath always had a Church amongst them, which firmly kept his saving truth. As for such as hold with the church of Rome, that we cannot be saved by Christ alone without works; they do not only by a circle of consequence, but directly, deny the foundation of faith2; they hold it not, no not so much as by a slendert thread.

20. This, to my remembrance, being all that hath been as yetu opposed with any countenance or shew of reason, I hope, if this be answered, the cause in question is at an end. Concerning general repentance, therefore: what? a murderer, a blasphemer, an unclean person, a Turk, a Jew, any sinner to escape the wrath of God by a generalx “God forgive me?” Truly, it never came within my heart, that a general repentance doth serve for all sins or for all sinnersy: it serveth only for the common oversightsz of our sinful life, and for faultsa which either we do not mark, or do not know that they are faults. Our fathers were actually penitent for sins, wherein they knew they displeased God: or else they comeb not within the compass of my first speech. Again, that otherwise they could not be saved, than holding the foundation of Christian faith, we have not only affirmed, but proved. Why is it not then confessed, that thousands of our fathers, although they livedc in popish superstitions, might yet, by the mercy of God, be saved? First, if they had directly denied the very foundationd of Christianity, without repenting them particularly [506] of that sin, he which saith, there could be no salvation for them, according to the ordinary course which God doth use in saving men, granteth plainly, or at the leastwisee closely insinuateth, that an extraordinary privilege of mercy might deliver their souls from hell; which is more than I required. Secondly, if the foundation be denied, it is denied by forcef of some heresy which the church of Rome maintaineth. But how many were there amongst our fathers, who being seduced by the common error of that church, never knew the meaning of her heresies! So that ifg all popish heretics did perish, thousands of them which lived in popish superstitions might be saved. Thirdly, seeing all that held popish heresies did not hold all the heresies of the pope: why might not thousands which were infected with other leaven, live andh die unsoured byi this, and so be saved? Fourthly, if they all had heldk this heresy, many there were that held it no doubt onlyl in a general form of words, which a favourable interpreterm might expound in a sense differing far enough from the poisoned conceit of heresy. As for example; did they hold that we cannot be saved by Christ without worksn1? We ourselves do, I think, all say as much, with this construction, salvation being taken as in that sentence, “Corde creditur ad justitiam, ore fit confessio ad salutem;” except infants, and men cut off upon the point of their conversion, of the rest none shall see God, but such as seek peace and holiness, though not as a cause of their salvation, yet as a way througho which they must walk thatp will be saved. Did they hold, that without works we are not justified? Take justification so thatq it may also imply sanctification, and St. James doth say as much. rFor except there be an ambiguity in somes term, St. Paul and St. James do contradict each othert; which cannot be. Now, there is no ambiguity in the name either of [507] faith or of works, bothx being meant by them both in one and the same sense.SERM. II. 21. Finding therefore that justification is spoken of by St. Paul without implying sanctification, when he proveth that a man is justified by faith without works; finding likewise that justification doth sometimes imply sanctification also with it; I suppose nothing more soundy, than so to interpret St. James asz speaking not in that sense, but in this.

21. aWe have already shewed, that there areb two kinds of Christian righteousness: the one without us, which we have by imputation; the other in us, which consisteth of faith, hope, charityc, and other Christian virtues; and St. James doth prove that Abraham had not only the one, because the thing hed believed was imputed unto him for righteousness; but also the other, because he offered up his son. God giveth us both the one justice and the other: the one by accepting us for righteous in Christ; the other by working Christian righteousness in us. The proper and most immediate efficient cause in us of this latter, is, the spirit of adoption whiche we have received into our hearts. That whereof it consisteth, whereof it is really and formally made, are those infused virtues proper and particular unto saints; which the Spirit, in thatf very moment when first it is given of God, bringeth with it: the effects thereofg are such actions as the Apostle doth call the fruits, the worksh, the operations of the Spirit; the difference of whichi operations from the root whereof they spring, maketh it needful to put two kinds likewise of sanctifying righteousness, Habitual and Actual. Habitual, that holiness, wherewith our souls are inwardly endued, the same instant when first we begin to be the temples of the Holy Ghost; Actual, that holiness which afterward beautifieth all the parts and actions of our life, the holiness for which Enoch, Job, Zachary, Elizabeth, and other saints, are in Scripturesk so highly commended. If here it be demanded, which of these we do first receive; I answer, that the Spirit, the virtues of the Spirit, the habitual justice, [508] which is ingrafted, the external justice of Christ Jesusl which is imputed, these we receive all at one and the same time; whensoever we have any of these, we have all; they go together. Yet sith no man is justified except he believe, and no man believeth except he havem faith, and no man hath faith, unlessn he haveo received the Spirit of Adoptionp, forasmuch as theseq do necessarily infer justification, butr justification doth of necessity presuppose them; we must needs hold that imputed righteousness, in dignity being the chiefest, is notwithstanding in order the lasts of all these, but actual righteousness, which is the righteousness of good works, succeedeth all, followeth after all, both in order and int time. Which thingu being attentively marked, sheweth plainly how the faith of true believers cannot be divorced from hope and love; how faith is a part of sanctification, and yet unto justification necessary; how faith is perfected by good works, and yet no works of ours good without faithx: finally, how our fathers might hold, Wey are justified by faith alone, and yet hold truly that without goodz works we are not justified. Did they think that men do merit rewards in heaven by the works they perform on earth? The ancient Fathersa use meriting for obtaining, and in that sense they of Wittenberg have in their Confession: “We teach that good works commanded of God are necessarily to be done, and thatb by the free kindness of God they merit their certain rewards1.” Othersc therefore, speaking as our fathers did, and we taking their speech in a sound meaning, as we may take our fathers’, and oughtd, forasmuch as their meaning is doubtful, and charity doth always interpret doubtful things favourably; what should induce us to think that [509] rather the damage of the worsee construction did light upon them all, than that the blessing of the better was granted unto thousands?SERM. II. 22.

Fifthly, if in the worst construction that canf be made, they had generally all embraced it living, might not many of them dying utterly renounce it? Howsoever men, when they sit at ease, do vainly tickle their owng hearts with the wanton conceit of I know not what proportionable correspondence between their merits and their rewards, which, in the trance of their high speculations, they dream that God hath measured, weighed, and laid up, as it were, in bundles for them; notwithstanding we see by daily experience, in a number even of them, that when the hour of death approacheth, when they secretly hear themselves summoned forthwith to appear, and stand at the bar of thath Judge, whose brightness causeth the eyes of angelsi themselves to dazzle, all thosek idle imaginations do then begin to hide their faces; to name merits then, isl to lay their souls upon the rack, the memory of their own deeds is loathsome unto them, they forsake all things wherein they have put any trust andm confidence; no staff to lean upon, no ease, no rest, no comfort then, but only in Christ Jesusn.

22. Wherefore if this proposition were true, “To hold in such wise, as the church of Rome doth, that we cannot be saved by Christ alone without works, is directly to deny the foundation of faith;” I say, that if this proposition were true, nevertheless so many ways I have shewed, whereby we may hope that thousands of our fathers livingo in popish superstitionsp might be saved1. But whatq if it be not [510] true?SERM. II. 23. What if neither that of the Galatians concerning circumcision, nor this of the church of Rome aboutr works, be any direct denial of the foundation, as it is affirmed that both are? I need not wade so far as to discuss this controversy, the matter which first was brought into question being so cleareds, as I hope it is. Howbeit, because I desire that the truth even in thist also mayu receive light, I will do minex endeavour to set down somewhat more plainly: first, the foundation of faith, what it is: secondly, what it is directly to deny the foundation: thirdly, whether they whom God hath chosen to be heirs of life, may fall so far as directly to deny it: fourthly, whether the Galatians did so by admitting the error about circumcision and the law: last of all, whether the church of Rome, for this one opinion of works, may be thought to do the like, and thereupon to be no more a Christian church, than are the assemblies of Turks ory Jews.

What the foundation of faith is.23. This word foundation being figuratively used, hath always reference to somewhat which resembleth a material [511] building, as both the doctrine of the Christianityz [of Christianity] and the community of Christians do. By the Masters of civil policy nothing is so much inculcated, as that commonwealths are founded upon laws; for that a multitude cannot be compacted into one body otherwise than by a common acceptationa of laws, whereby they are to be kept in order1. The ground of all civil laws is this; “No man ought to be hurt or injured by another:” take away this persuasion, and youb take away all lawsc; take away laws, and what shall become of commonwealths? So it is in our spiritual Christian community: I do not nowd mean that body mystical2 whereof Christ is the onlye head, that building undiscernible by mortal eyes, wherein Christ is the chief corner-stone: but I speak of the visible church; the foundation whereof is the doctrine3 off the Prophets and Apostles profest. The mark whereunto their doctrine tendeth, is pointed at in thoseg words of Peter unto Christ, “Thou hast the words of eternal life:” in those ofh Paul to Timothy, “The holy Scriptures are able to make thee wise unto salvation.” It is the demand of nature itselfi, “What shall we do to have eternal life4?” The desire of immortality and ofk the knowledge of that whereby it may be attainedl, is so natural unto all men, that even they whichm are not persuaded that they shall, don notwithstanding wish that they might, know a way how to see no end of life. And because natural means are not able stillo to resist the force of death, there is no people in the earth so savage, which hath not devised some supernatural help or other, to fly unto for aid and succour in extremities, against the enemies of their livesp. A longing therefore to be saved, without understanding the true way how, hath been the cause of all the superstitions in the world. O that the miserable state of others, which wander in darkness, and wot not whither they go, [512] could give us understanding hearts, worthily to esteem the riches of the merciesq of God towards us, before whose eyes the doors of the kingdom of heaven are set wide open! Should we notr offer violence unto it? It offereth violence to us, and we gather strength to withstand it.

But I am besides my purpose when I fall to bewail the cold affection which we bear towards that whereby we should be saved; my purpose being only to set down what the ground of salvation is. The doctrine of the Gospel proposeth salvation as the end: and doth it not teach the way of attaining thereunto? Yess, the damosel possestt with a spirit of divination spake the truth: “These men are the servants of the most high God, which shew unto us the way of salvation:” “A new and living way, which Christ hath prepared for us through the vail, that is, his flesh1;” salvation purchased by the death of Christ. By this foundation the children of God, before the time of the writtenu law, were distinguished from the sons of men; the reverend patriarchs both profestx it living, and spake expressly2 of it at the hour of their death. It comforted Job3 in the midst of grief; it was afterwards likewisey the anchor-hold of all the righteous in Israel, from the writing of the law to the time of grace. Every prophet maketh mention of it. It was soz famously spoken of, about the time, when the coming of Christ to accomplish the promises, which were made long beforea, drew near, that the sound thereof was heard even amongst the Gentiles. When he was come, as many as were his acknowledged that he was their salvation; he, that long-expected hope of Israel; he, that “seed, in whom all the nations of the worldb shouldc be blestd.” So that now his name is a namee of ruin, a name of death and condemnation, unto such as dream of a new Messias, to as many as look for salvation by any other thanf by him: “For amongst men there is given no other name under heaven whereby we must be saved4.” Thus much St. Mark doth intimate by that which [513] he puttethg in the veryh front of his book,SERM. II. 24. making his entrance with these words: “The beginning of the Gospel of Jesus Christ, the Son of God.” His doctrine he termeth the Gospel, because it teacheth salvation; the Gospel of Jesus Christi, the Son of God, because it teacheth salvation by him. This is then the foundation, whereupon the frame of the Gospel is erected; that very Jesus whom the Virgin conceived of the Holy Ghost, whom Simeon embraced in his arms1, whom Pilate condemned, whom the Jews crucified, whom the Apostles preached, he is Christ, the Lord, the only Saviour of the world: “other foundation can no man lay2.” Thus I have briefly opened that principle in Christianity, which we call the foundation of our faith. It followeth now that I declare unto you, what itj is directly to overthrow it. This will better appeark, if firstl we understand, what it is to hold the foundation of faith.

24. There are which defend, that many of the Gentiles, who never heard the name of Christ, held the foundation of Christianity: and why? they acknowledged many of them the providence of God, his infinite wisdom, strength, andm power; his goodness, and his mercy towards the children of men; that God hath judgment in store for the wicked, but for the righteous that seeksn him, rewards, &c. In this which they confessed, that lieth covered which we believe; in the rudiments of their knowledge concerning God, the foundation of our faith concerning Christ lieth secretly wrapto up, and is virtually contained: therefore they held the foundation of faith, though they never heardp it. Might we not with as good colourq of reason defend, that every ploughman hath all the sciences, wherein philosophers have excelled? For no man is ignorant of ther first principles, which do virtually contain whatsoever by natural means eithers is or can be known. Yea, might we not with as goodt reason affirm, that a man may put three mighty oaks wheresoever three acorns may be put? For virtually an acorn is an oak. To avoid such paradoxes, we teach plainly, [514] that to hold the foundation is, in express terms to acknowledge it.SERM. II. 25.

25. Now, because the foundation is an affirmative proposition, they all overthrow it, who deny it; they directly overthrow it, who deny it directly; and they overthrow it by consequent, or indirectly, which hold any one assertion whatsoever, whereupon the direct denial thereof may be necessarily concluded. What is the question between the Gentiles and us, but this, Whether salvation be by Christ? What between the Jews and us, but this, Whether by this Jesus, whom we call Christ, yea, or no? This to be the main point whereupon Christianity standeth, it is clear by that one sentence of Festus concerning Paul’s accusers: “They brought no crime of such things as I supposed, but had certain questions against him of their ownu superstition, and of one Jesus which was dead, whom Paul affirmed to be alive1.” Where we see that Jesus, dead and raised for the salvation of the world, is by Jewsw denied, despised by a Gentile, and by a Christian apostle maintained. The Fathers therefore in the primitive church when they wrote; Tertullian, the book which he callethx Apologeticus; Minutius Felix, the book which he entitlethy Octavius; Arnobius, hisz seven books against the Gentiles; Chrysostom, his orations against the Jews; Eusebius his ten books of Evangelical Demonstration: they stooda in defence of Christianityb against them, by whom the foundation thereof was directly denied. But the writings of the Fathers against Novatians, Pelagians, and other heretics of the like note, refel positions, whereby the foundation of Christian faith was overthrown by consequent only. In the former sort of writings the foundation is proved; in the latter, it is alleged as a proof, which to men that had been known directly to deny it, must needs have seemed a very beggarly kind of disputing. All infidels therefore deny the foundation of faith directly: by consequent, many a Christian man, yea whole Christian churches, havec denied it, and do deny it at this present day. Christian churches denyingd the foundation of Christianity? [515] Note directly, for then they cease to be Christian churches;SERM. II. 26. but by consequent, in respect whereof we condemn them as erroneous, although, for holding the foundation, we do and must hold them Christian.

26. We see what it is to hold the foundation; what directly, and what by consequent, to deny it. The next thing which followeth is, whether they whom God hath chosen to obtain the glory of our Lord Jesus Christ, may, beingf once effectually called, and through faith truly justifiedg, afterwards fall so far, as directly to deny the foundation which their hearts have before embraced with joy and comfort in the Holy Ghost; for such is the faith, which indeed doth justify. Devils know the same things which we believe, and the minds of the most ungodly may be fully persuaded of the truth; which knowledge in the one and persuasionh in the other, is sometimes termed faith, but equivocally, being indeed no such faith as that whereby a Christian man is justified. It is the spirit of adoption which worketh faith in us, in them not; the things which we believe, are by us apprehended, not only as true, but also as good, and that to us: as good, they are not by them apprehended; as true, they are. Whereupon followeth a thirdi difference; the Christian man the more he increaseth in faith, the more his joy and comfort aboundeth: but they, the more sure they are of the truth, the more they quake and tremble at it. This begetteth another effect, whereink the hearts of the one sort have a different disposition from the other. Non ignoro plerosque conscientia meritorum, nihil se esse post1 mortem magis optare quam credere; malunt enim exstingui penitus, quam ad supplicia repararil. I am not ignorant, saith Minutius, that there are too manym, who being conscious what they are to look for, do rather wish that they might, than think that they shall, cease to ben, when they cease to live; because they hold it better that death should consume them unto nothing, than God reviveo them untop punishment. So it is in other articles of faith, whereof [516] wicked men think, no doubt, many times they are too true: on the contrary side, to the other, there is no grief norp torment greater, than to feel their persuasion weak in things, whereof, when they are persuaded, they reap such comfort and joy of spirit: such is the faith whereby we are justified; such, I mean, in respect of the quality. For touching the principal object of faith, longer than it holdeth thatq foundation whereof we have spoken, it neither justifieth, nor is; but ceaseth to be faith when it ceaseth to believe, that Jesus Christ is the only Saviour of the world. The cause of life spiritual in us, is Christ, not carnally or corporally inhabiting, but dwelling in the soul of man, as a thing which (when the mind apprehendeth it) is said to inhabit andr possess the mind. The mind conceiveth Christ by hearing the doctrine of Christianity. As the light of nature doth cause the mind to apprehend those truths which are merely rational; so that saving truth, which is far above the reach of human reason, cannot otherwise, than by the Spirit of the Almighty, be conceived. All these are implied, wheresoever any ones of them is mentioned as the cause of spiritualt life. Wherefore when we readu, that1 “the Spirit is our life;” or2, “the Word our life;” or3, “Christ our life:” we are in every of these to understand, that our life is Christ, by the hearing of the Gospel apprehended as a Saviour, and assented unto byx the power of the Holy Ghost. The first intellectual conceit and comprehension of Christ so embraced, St. Peter calleth4 the seed whereof we be new born: our first embracing of Christ, is our first reviving5 from the state of death and condemnation. “He that hath the Son hath life,” saith St. John6, “and he that hath not the Son of God, hath not life.” If therefore he which once hath the Son, may cease to have the Son, though it be buty a moment, he ceaseth for that moment to have life. But the life of them whichz live by the Son of Goda, is everlasting, not only for that it shall be everlasting7 in the world to [517] comeb, but because1 as “Christ being raised from the dead diethc no more, death hath no more power over him;” so thed justified man, being alivee to God in Jesus Christ our Lord, doth as necessarily from that time forward always live, as Christ, by whom he hath life, liveth always2.

I might, if I had not otherwhere largely done it already, shew by sundryf manifest and clear proofs, how the motions and operations of life are sometimes so undiscernible, and secretg, that they seem stone-dead, who notwithstanding are still alive unto God in Christ.

For as long as that abideth in us, which animateth, quickeneth, and giveth life, so long we live; and we know that the cause of our lifeh abideth in us for ever. If Christ, the fountain of life, may flit and leave the habitation where once he dwelleth, what shall become of his promise, “I am with you to the world’s end?” If the seed of God, which containeth Christ, may be first conceived and then cast out; how doth St. Peter3 term it immortal? How doth St. John4 affirm it abideth? If the Spirit, which is given to cherish and preserve the seed of life, may be given and taken away, how is it the earnest5 of our inheritance until redemption; how doth it continue6 with us for ever? If therefore the man which is once just by faith, shall live by faith, and live for ever, it followeth, that he which once doth believe the foundation must needs believe the foundation for ever. If hei believe it for ever, how can he ever directly deny it7? Faith holding the direct affirmation; the direct negation, so long as faith continueth, is excluded.

kBut yel will say, “That as he which to-daym is holy, may to-morrow forsake his holiness, and become impure; as a friend may change his mind, and becomen an enemy; as hope may wither: so faith may die in the heart of man, the Spirit [518] may be quenched, Grace may be extinguished, they which believe may be quite turned away from the truth.”

Then caseo is clear, long experience hath made this manifest, it needs no proof. I grant we are apt, prone, and ready, to forsake God1; but is God as ready to forsake us? Our minds are changeable; is his so likewise? Whom God hath justified, hath not Christ assured, that it is “his Father’s will to give them a kingdom?” Which kingdomp, notwithstanding, shall notq otherwise ber given them, than “2if they continue grounded and stablished in the faith, and be not moved away from the hope of the gospel;” “3if they abide in love and holiness.” Our Saviour therefore, when he spake of the sheep effectually called, and truly gathered into his fold4, “I give unto them eternal life, and they shall never perish, neither shall any pluck them out of my hands;” in promising to save them, promiseds, no doubt, to preserve them in that without the which there can be no salvation, as also from that whereby salvation ist irremediablyu lost. Every error in things appertaining tov God is repugnant unto faith; every fearful cogitation, unto hope; unto love, every straggling inordinate desire; unto holiness, every blemish wherebyx either the inward thoughts of our minds, or the outward actions of our lives, are stained. But heresy, such as that of Ebion, Cerinthus, and others, against whom the Apostles were forced to bend themselves, both by word and also by writing; that [519] repining discouragement of heart which tempteth God, whereof we have Israel in the desert for a pattern; coldness, such as that in the angely of Ephesus; foul sins, known to be expressly against the first or second table of the law, such as Noah, Manasses, David, Salomon, and Peter, committed: these are each in their kind so opposite to the former virtues, that they leave no place for salvation without an actual repentance. But infidelity, extreme despair, hatred of God and all godlinessz, obduration in sin, cannot stand where there is the leasta spark of faith, hope, love, orb sanctity; even as cold in the lowest degree cannot be, where heat in the firstc degree is found.

Whereupon I conclude, that although in the first kind, no man liveth thatd sinneth not; and in the second, as perfect as any do live, may sin: yet sith the man which is born of God hath a promise, that in him “the seed of God shall abide1;” which seed is a sure preservative against the sins ofe the third suit; greater and clearer assurance we cannot have of any thing, than of this, that from such sins God shall preserve the righteous, as the apple of his eye, for ever. Directly to deny the foundation of faith, is plain infidelity; where faith is entered, there infidelity is for ever excluded: therefore by him which hath once sincerely believed in Christ, the foundation of Christian faith can never be directly denied. Did not Peter, did not Marcellinus2, did not many others, both directly deny [520] Christ after theyf had believed, and again believe after they had denied? No doubt, as they mayg confess in wordh, whose condemnation nevertheless isi their not believing (for example we have Judas); so likewise, they may believe in heart, whose condemnation, without repentance, is their not confessing. Although therefore Peter and the rest, for whose faith Christ hadj prayed that it might not fail, did not by denial sin the sin of infidelity, which is an inward abnegation of Christ (fork if they had done this, their faith had clearly failed): yet, because they sinned notoriously and grievously, committing that which they knew to be sol expressly forbidden by the law, which saith, “Thou shalt worship the Lord thy God, and him only shalt thou serve:” necessary it was, that he which purposed to save their souls, should, as he did, touch their hearts with true unfeigned repentance, that his mercy might restore them again to life, whom sin had made the children of death and condemnation. Touching this point therefore, I hope I may safely set itm down, that if the justified err, as he may, and never come to understand his error, God doth save him through general repentance: but if he fall into heresy, he calleth himn either ato one time or other by actual repentance; but from infidelity, which is an inward direct denial of the foundation, preservethp him by special providence for ever. Whereby we may easily know what to think of those Galatians, whose hearts were so possest with loveq of the truth, that, if it had been possible, they would have plucked out their veryr eyes, to bestow upon their teachers. It is true, that they were afterwardss greatly1 changed, both [521] in persuasion and affection; so that the Galatians, when St. Paul wrote unto them, were not now the Galatians which they had been in former timess, for that through error they wandered, although they were his sheep. I do not deny, but It should deny, that they were his sheep, if I should grant, that through error they perished. It was a perilous opinion which they held, in themu whichv held it only as an error, because it overthroweth the foundation by consequent. But in them which obstinately maintainedw it, I cannot think it lessx than a damnable heresy.

We must therefore put a difference between them which err of ignorance, retaining nevertheless a mind desirous to be instructed in they truth, and them which, after the truth is laid open, persist in stubbornz defence of their blindness. Heretical defenders, froward and stiffnecked teachers of circumcision, the blessed Apostle calletha dogs: silly men, that were seduced to think they taughtb the truth, he pitieth, he taketh up in his arms, he lovingly embraceth, he kisseth, and with more than fatherly tenderness doth so temper, qualify, andc correct the speech he useth towards them, that a man cannot easily discern, whether did most abound, the love which he bare to their godly affection, or the grief which the danger of their opinion bred himd. Their opinion was dangerous; was not theirs soe likewise who thought thatf the kingdom of Christ should be earthly? was not theirs which thought thatg the gospel should be preachedh only to the Jews? What more opposite to prophetical doctrine, concerning the coming of Christ, than the one? concerning the catholic Church, than the other? Yet they which had these fancies, even when they had them, were not the worst men in the world. The heresy of freewill was a millstone about the Pelagians’ necki; shall we therefore give sentence of death inevitablek against all those Fathers in the Greek church, which being mispersuaded, died in the error of freewill1?

[522]
Of thosel Galatians, therefore, which first were justified1, and then deceived, as I can see no cause, why as many as died before admonition might not by mercy be savedm, 2even in error; so I make no doubt, but as many as lived till they were admonished, found the mercy of God effectual in converting them from their error3, lest any one that is Christ’s should perish. Of this, asn I take it, there is no controversy: only against the salvation of them which died, though before admonition, yet in error, it is objected, that their opinion was a very plain direct denial of the foundation. If Paul and Barnabas had been so persuaded, they would haply have used theiro terms otherwise, speaking of the masters themselves, who did first set that error abroach, “certain of the sect of the Pharisees which believed4.” What difference was there between these Phariseesp and other, from whom by a special description they are distinguished, but this? Theyq which came to Antioch, teaching the necessity of circumcision, were Christians; the other, enemies of Christianity. Why then should these be termed so distinctly believers, if they did directly deny the foundation of our belief; besides which, there was noner other thing, that made the rest to be unbelieverss? We need go no farther than St. Paul’s very reasoning against them, for proof of this matter5, “Seeing yet know God, or rather are known of God, how turn you again unto impotent rudiments? 6The law engendereth servants, her children are in bondage: they which are begottenu by the gospel, are free. 7Brethren, we are not children of the servant, but of the free woman, and will ye yet be under the law?” That they thought it unto salvation necessary, for the Church of Christ to observe days, and [523] months, and times, and years, to keep the ceremonies and the sacraments of the law, this was their error1. Yet he which condemneth their error, confesseth notwithstandingw, that they knew God2, and were known of him; he taketh not the honour from them to be termed sons begotten ofx the immortal seed of the gospel. Let the heaviest words whichy he useth be weighed; consider the drift of those dreadful conclusions3: “If yez be circumcised, Christ shall profit you nothing: as many as are justified by the law, yea are fallen from grace.” It had been to no purpose in the world so to urge them, had not the Apostle been persuaded, that at the hearing of such sequels, “No benefit by Christ,” “a defection from grace,” their hearts would tremble and quake within them: and why? because theyb knew, that in Christ, in gracec, their salvation lay, which is a plaind direct acknowledgment of the foundation.

Lest I should herein seem to hold that which no one godly and learnede hath done, let these words be considered, which import as much as I affirm4. “Surely those brethren which, in St. Paul’s time, thought that God did lay a necessity upon them to make choice of days and meats, spake as they believed, and could not but in words condemn that liberty, which they supposed to be brought in against the authority of divine Scripture. Otherwise it had been needless for St. Paul to admonish them, not to condemn such as eat, without scrupulosity, whatsoever was set before them. This error, if you weigh what it is of itself, did at once overthrow all Scriptures, whereby we are taught salvation by faith in Christ, all that ever the prophets did foretell, all that ever the Apostles did preach of Christ; it drew with it the denial of Christ utterly: insomuch thatf St. Paul complaineth, that his labour was lost upon the Galatiansg, unto whom this [524] error was obtruded; affirming that Christ, if so be they were circumcised, should not profit them any thing at all. Yet so far was St. Paul from striking their names out of Christ’s book, that he commanded others to entertain them, to accept them with singular humanity, to use them like brethren; he knew man’s imbecillity, he had a feeling of our blindness which are mortal men, how great it is, and being sure that they are the sons of God, whosoever be endued with his fear, would not have them counted enemies of that whereunto they could not as yet frame themselves to be friends, but did even ofh a very religious affection to the truth, unwittinglyi reject and resistk the truth. They acknowledged1 Christ to be their only and theirl perfect Saviour, but saw not how repugnant their believing them necessity of Mosaical ceremonies was to their faith in Jesus Christ.”

Hereunton replyo is made, that if they had not directly denied the foundation, they might have been saved; but saved they could not be; therefore their opinion was, not only by consequent, but directly, a denial of the foundation. When the question was about the possibility of their salvation, their denying of the foundation was brought for proofp that they could not be saved: now that the question is about their denialq, the impossibility of their salvation is alleged to prove they denied the foundation. Is there nothing which excludeth men from salvation, but only the foundation of faith denied? I should have thought, that besider this, many other things are death, except they be actually repented of: as indeed this opinion of theirs was deaths, unto as many as, being given to understandt that to cleave thereunto was to fall from Christ, did notwithstanding cleave unto it. But of this enough. Wherefore I come to the last question, “Whether theu doctrine of the Church of Rome, concerning the necessity of works unto salvation, be a direct denial of the foundation ofx our faith?”

[525]
SERM. II. 27.27. I seek not to obtrude unto you any private opinionsy of mine own. The best learned1 in our profession are of this judgment, that all the heresies andz corruptions of the Church of Rome do not prove her to deny the foundation directly; if they did, they should provea her simply to be no Christian church. “But I suppose,” saith one2, “that in the papacy some church remaineth, a church crazed, or, if you will, broken quite in pieces, forlorn, misshapen, yet some church:” his reason is this, “Antichrist must sit in the temple of God.” Lest any man should think such sentences as thisb to be true only in regard of them whom that church is supposed to have kept by the special providence of God, as it were, in the secret corners of his bosom, free from infection, and as sound in the faith, as we trust, by his mercy, we ourselves are; I permit it to yourc wise considerations, whether it be notd more likely, that as frensy, though itself take away the use of reason, doth notwithstanding prove them reasonable creatures which have it, because none can be frantic but they; so Antichristianity being the bane and plain overthrow of Christianity, may nevertheless argue the church whereine Antichrist sitteth to be Christian3. Neither have I everf hitherto heard or read any one word alleged of force to warrant, that God doth otherwise than so as hath been in the two next questions before declaredg, bind himself to keep his elect from worshipping the Beast, and from receiving his mark in their foreheads; but he hath preserved, and will preserve, them from receiving any deadly wound at the hands of the Man of sin, whose deceit hath prevailed over none unto death, but only suchh as never loved the truth, such [526] as took pleasurei in unrighteousness: they in all ages, whose hearts have delighted in the principal truth, and whose souls have thirsted after righteousness, if they received the mark of error, the mercy of God, even erring, and dangerously erring, might save them; if they received the mark of heresy, the same mercy did, I doubt not, convert them1. How far Romish heresies may prevail over God’s elect, how many God hath kept from falling into them, how many have been converted from them, is not the question now in hand: for if heaven had not received any one of that coat for these thousand years, it may still be true2, that the doctrine which atk this day they do profess, doth not directly deny the foundation, and so prove them simply to be no Christian church. One I have alleged, whose words, in my ears, sound that way; shall I add another, whose speech is plainerl? “I deny her not the name of a church,” saith another3, “no more than to a man the name of a man, as long as he liveth, what sickness soever he hath.” His reason is this: “Salvation in Jesus Christ, which is the mark that joineth the Head with the body, Jesus Christ with Hism Church, itn is so cut off by man’so merits, by the merits of saints, by the pope’s pardons, and such other wickedness, that the life of the Church holdeth by a very littlep thread,” yet still the life of the Church holdeth. A third hath these words4: “I acknowledge the church of [527] Rome, even at this present day, for a church of Christ, such a church as Israel underq Jeroboam, yet a church.”SERM. II. 28. His reason is this: “Every man seeth, except he willingly hoodwink himself, that as always, so now, the church of Rome holdeth firmly and steadfastly the doctrine of truth concerning God, and the Person of our Lord Jesusr Christ; and baptizeth in the name of the Father, the Son, and the Holy Ghost; confesseth and avoucheth Christ fors the only1 Redeemer of the world, and the Judge that shall sit upon quick and dead, receiving true believers into endless joy, faithless and godless men being cast with Satan and his angels into flames unquenchable.”

28. I may, and will, rein the question shorter than they do. Let the Pope take down his top, and captivate no more men’s souls by his papal jurisdiction; let him no longer count himself Lord Paramount over the princes of the eartht, no longer useu kings as his tenantsv paravaile2; let his stately senate submit their necks to the yoke of Christ, and cease to dye their garments, like Edom, in blood; let them, from the highest to the lowest, hate and forsake their idolatry, abjure all their errors and heresies, wherewith they have any way perverted the truth; let them strip their church, till they leave no polluted rag, but only this one about her; “By Christ alone, without works3, [528] we cannot be saved:”SERM. II. 29. it is enough for me, if I shew, that the holding of this one thing doth not prove the foundation of faith directly denied in the Church of Rome.

29. Works are an addition to the foundationx: be it so, what then? the foundation is not subverted by every kind of addition. Simply to add unto those fundamental words, is not to mingle wine with puddley, heaven with earth, things polluted with the sanctified blood of Christ: of which crime indict them, whichz attribute those operations in whole or in part to any creature, which in the work of our salvation are whollya peculiar unto Christ: and, if I open my mouth to speak in their defence, if I hold my peace, and plead not against them as long as breath is inb my body, let me bec guilty of all the dishonour that ever hath been done to the Son of God. But the more dreadful a thing it is to deny salvation by Christ alone, the more slow and fearful I am, except it be too too manifestd to lay a thing so grievous unto any man’s charge. Let us beware, lest if we make too many ways of denying Christ, we scarce leave any way for ourselves truly and soundly to confess him. Salvation only by Christ is the true foundation whereupon indeed Christianity standeth. But what if I say, yee cannot be saved only by Christ, without this addition, Christ believed in heart, confessed with mouth, obeyed in life and conversation? Because I add, do I therefore deny that which directly I didf affirm? There may be an additament of explication, which overthroweth not, but proveth and concludeth the proposition whereunto it is annexed. He thatg saith, Peter was a chief Apostle, doth prove that Peter was an Apostle: he which saith1, Our salvation is of the Lord, through sanctification of the Spirit, and faith of the truth, proveth that our salvation is of the Lord. But if that which is added, be such a privation as taketh away the very essence of that whereunto it is adjoinedh, then by sequeli it overthroweth. He which saith, Judas is a dead man, though in word he grantk Judas to be a man, yet in effect he proveth him by that very speech no man, because death depriveth him of his beingl. In like sort, he that should say, Our election [529] is of grace for our works’ sake,SERM. II. 30. should grant in sound of words, but indeed by consequent deny, that our election is of grace; for the grace which electeth us is no grace1, if it elect us for our works’ sake.

30. Now whereas the church of Rome addeth works, we must note farther, that the adding worksm2 is not like the adding of circumcision unto Christ. Christ came not to abrogate and to take awayn good works3: he did, to change circumcision; for we see that in place thereof he hath substituted holy baptism. To say, ye cannot be saved by Christ except ye be circumcised, is to add a thing excluded, a thing not only not necessary to be kept, but necessary not to be kept4 by them that will be saved. On the other side, to say, ye cannot be saved by Christ without works5, is to add things not only not excluded, but commanded, as being in theiro place and in their kind necessary, and therefore subordinated unto Christ, evenp by Christ himself, by whom the web of salvation is spun6: “Except your righteousness exceed the righteousness of the scribes and Pharisees, yeq shall not enter into the kingdom of heaven7.” They were rigorous exacters of things not utterly to be neglected and left undone8, washings and tithingsr, &c. As they were in these thingss, so must we be in judgment and the love of God. Christ, in works ceremonial, giveth more liberty, in moral much less9, than they did. Works of righteousness therefore are not so [530] repugnantly1 added in the one proposition; as in the other circumcision is.SERM. II. 31.

31. But we say, our salvation is by Christ alone; therefore howsoever, or whatsoever, we add unto Christ in the matter of salvation, we overthrow Christ. Our case were very hard, if this argument, so universally meant as it is proposed, were sound and good. We ourselves do not teach Christ alone, excluding our own faith2, unto justification; Christ alone, excluding our own works, unto sanctification; Christ alone, excluding the one or the other asx unnecessary unto salvation. It is a childish cavil wherewith in the matter of justification our adversaries do so greatly please themselves, exclaiming, that we tread all Christian virtues under oury feet, and require nothing in Christians but faith; because we teach that faith alone justifieth: whereas we by this speechz never meant to exclude either hope anda charity from being always joined as inseparable mates with faith in the man that is justified; or works from being added as necessary duties, required at the hands of every justified man: but to shew that faith is the only hand which putteth on Christ unto justification; and Christ the only garment, which being so put on, covereth the shame of our defiled natures, hideth the imperfections of our works, preserveth us blameless in the sight of God, before whom otherwise the veryb weakness of our faith were cause sufficient to make us culpable, yea, to shut us outc from the kingdom of heaven, where nothing that is not absolute can enter. That our dealing with them be not as childish as theirs with us; when we hear of salvation by Christ alone, considering that (“alone” isd an) exclusive particle, we are to note what it doth exclude, and where. If I say, “Such a judge only ought to determine such a causee,” all things incident to the determination thereof, besides the person of the judge, as laws, depositions, evidences, &c. are not hereby excluded; persons are, yet notf from witnessing herein, or assisting, but only from determining and giving sentence. How then is our salvation wrought by Christ alone? is it our [531] meaningg, that nothing is requisite to man’s salvation, but Christ to save, and he to be saved quietly without any more to doh? No, we acknowledge no such foundationi.SERM. II. 32. As we have received, so we teach that besides the bare and naked work1, wherein Christ, without any other associate, finished all the parts of our redemption, and purchased salvation himself alone; for conveyance of this eminent blessing unto us, many things are requiredj, as, to be known and chosen of God before the foundations of the world; in the world to be called, justified, sanctified: after we have left the world, to be received intok glory; Christ in every of these hath somewhat which he worketh alone. Through him, according to the eternal purpose of God before the foundation of the world2, born, crucified, buried, raised, &c., we were in a gracious acceptationl known unto God long before we were seen of men: God knew us, loved us, was kind towards usm in Christ Jesusn, in him we were elected to be heirs of life. Thus far God through Christ hath wrought in such sort alone, that ourselves are mere patients, working no more than dead and senseless matter, wood, or stone, or iron, doth in the artificer’s hando, no more than the clay, when the potter appointeth it to be framed for an honourable use; nay, not so much. For the matter whereupon the craftsman worketh he chooseth, being moved byp the fitness which is in it to serve his turn; in us no such thing. Touching the rest, thatq which is laid for the foundation of our faith, importethr farther, that by him we bes called, that we have redemption, remission of sins through his blood, health by his stripes; justice by him; that he doth sanctify his Church, and make it glorious to himself; that entrance into joy shall be given us by him; yea, all things by him alone. Howbeit, not so by him alone, as if in us, to our vocation, the hearing of the gospel; to our justification, faith; to our sanctification, the fruits of the spirit; to our entrance into rest, perseverance in hope, in faith, in holiness, were not necessary.

32. Then what is the fault of the church of Rome? Not that she requireth works at their hands that will be saved: [532] but that she attributeth unto works a power of satisfying God for sin; andt a virtue to merit both grace here, and in heaven glory. That this overthroweth the foundation of faith, I grant willingly; that it is a direct1 denial thereof, I utterly deny. What it is to hold, and what directly2 to deny, the foundation of faith, I have already opened. Apply it particularly to this cause, and there needs no more ado. The thing which is handled, if the form under which it is handled be added thereunto, it sheweth the foundation of any doctrine whatsoever. Christ is the matter whereof the doctrine of the gospel treateth; and it treateth of Christ as of a Saviour. Salvation therefore by Christ is the foundation of Christianity: as for works, they are a thing subordinate, no otherwise necessaryu than because our sanctification cannot3 be accomplished without them. The doctrine concerning them is a thing builded upon the foundation; therefore the doctrine which addeth unto them powerw of satisfying, or of meriting, addeth unto a thing subordinated, builded upon the foundation, not tox4 the very foundation itself; yet is the foundation consequently by this additiony overthrown, forasmuch as out of this addition it may negatively bez concluded, He which maketh any work good and acceptable in the sight of God, to proceed from the natural freedom of our will; he which giveth unto any good worka of ours the force of satisfying the wrath of God for sin, the power of meriting either earthly or heavenly rewards; he which holdeth works going before our vocation, in congruity to merit our vocation; works following our first, to merit our second justification, and by condignity our last reward in the kingdom of heaven, pulleth up the doctrine of faith by the roots; for out of every of these the plain direct denial thereof may be necessarily concluded. Norb this only, but what other heresy is there whichc doth not raze the very foundation of faith by consequent? Howbeit, we make a difference [533] of heresies; accounting them in the next degree to infidelity, whichd directly deny any one thing to be, which is expressly acknowledged in the articles of our belief; for out of any one article so denied, the denial of the very foundation itself is straightwaye inferred1. As for example; if a man should say, “There is no catholic Church,” it followeth immediately hereuponf, that this Jesus whom we call the Saviour, is not the Saviour of the world; because all the prophets bearg witness, that the true Messias should “shew light unto the Gentiles2;” that is to say, gather such a Church as is catholic, not restrained any longer unto one circumcised nation. In ah second rank we place them, out of whose positions the denial of any ofi the foresaid articles may be with like facility concluded; such arej they which have denied, either the divinity of Christ, with Hebion, or with Marcion, his humanity; an example whereof may be that of Cassianus defending the incarnation of the Son of God against Nestorius bishop of Antioch3, whichk held, that the Virgin, when she brought forth Christ, did not bring forth the Son of God, but a sole and a mere man. Out of which heresy the denial of the articles of Christianl faith he deduceth thus4: “If thou dost deny our Lord Jesus Christ to be [534] Godm, in denying the Son, thou canst not choose but deny the Father; for, according to the voice of the Father himself, ‘He that hath not the Son, hath not the Father.’ Wherefore denying him thatn is begotten, thou deniest him which doth beget. Again,o denying the Son of God to have been born in the flesh, how canst thou believe him to have suffered? believing not his passion, what remaineth, but that thou deny his resurrection? For we believe him not raised, except we first believe him dead: neither can the reason of his rising from the dead stand, without the faith of his death going before. The denial of his death and passion inferreth the denial of his rising from the depthp. Where upon it followeth, that thou also deny his ascension into heaven: the Apostle affirmingq, ‘That he which ascended, did first descend.’ So that, as much as lieth in thee, our Lord Jesus Christ hath neither risen from the depthr, nor is ascended into heaven, nor sitteth at the right hand of God the Father, neither shall he come at the day of final account, which is looked for, nor shall judge the quick and dead. And darest thou yet set foot in the Church? Canst thou think thyself a bishop, when thou hast denied all those things whereby thou didsts obtain a bishoply calling?” Nestorius confessed all the articles of the creed, but his opinion did imply the denial of every part of his confession. Heresies there are of a thirdt sort, such as the church of Rome maintaineth, which beingu removed by a greater distance from the foundation, although indeed they overthrow it; yet because of that weakness, which the philosopher1 noteth in men’s capacities when he saith, that the common sort cannot see [535] things which follow in reason,SERM. II. 33. when they follow, as it were, afar off by many deductions; therefore the repugnancy betweenx such heresy and the foundation is not so quickly nory so easily found, but that an heretic of this, sooner than of the former kind, may directly grant, and consequently nevertheless deny, the foundation of faith.

33. If reason be suspected, trial will shew that the church of Rome doth noz otherwise, by teaching the doctrine she doth teach concerning worksa. Offer them the very fundamental words, and what oneb man is there that will refuse to subscribe unto them? Can they directly grant, and deny directlyc one and the very selfsame thing? Our own proceedings in disputing against their works satisfactory and meritorious do shew, not only that they hold, but that we acknowledge them to hold, the foundation, notwithstanding their opinion. For are not these our arguments against them? “Christ alone hath satisfied and appeased his Father’s wrath: Christ hath merited salvation alone.” We should do fondly to use such disputes, neither could we think to prevail by them, if that whereupon we ground, were a thing which we know they do not holdd, which we are assured they will not grant. Their very answers to all such reasons, as are in this controversy brought against them, will not permit us to doubt whether they hold the foundation or no. Can any man, whiche hath read their books concerning this matter, be ignorant how they draw all their answers unto these heads? “That the remission of all our sins, the pardon of all whatsoever punishments thereby deserved, the rewards which God hath laid up in heaven, are by the blood of our Lord Jesus Christ purchased, and obtained sufficiently for all men: but for no man effectually for his benefit in particular, except the blood of Christ be applied particularly unto him by such means as God hath appointed itf to work by: That those means of themselves being but dead things, only the blood of Christ is that which putteth life, force, and efficacy in them to work, and to be available, each in his kind, to our salvation: Finally, that grace being purchased for us by the blood of Christ, and freely [536] without any merit or desert at the first bestowed upon us, the good things which we do, after grace received, areg thereby made satisfactory and meritorious.” Some of their sentences to this effect I must allege for mine own warrant. If we desire to hear foreign judgments, we find in one this confession: “He that could reckon how many the virtues and merits of our Saviour Jesus Christ haveh been, might likewise understand how many the benefits have been that are comei unto us by him, forasmuchk as men are made partakers of them all by the meanl of his passion: by him is given unto us remission of our sins, grace, glory, liberty, praise, [peace,] salvation, redemption, justification1, justice, sanctificationm, sacraments, merits, doctrinen, and all other things which we [he] had, and were behovefulo for our salvation2.” In another we have these oppositions and answers made unto them: “All grace is given by Christ Jesus. True; but not except Christ Jesus be applied. He is the propitiation for our sinsp; by his stripes we are healed; he hath offered up himselfq for us: all this [us all: this?r] is true, but apply it. We put all satisfaction in the blood of Jesus Christ; but we hold, that the means whichs Christ hath appointed for us in this case to apply it, are our penal works3.” Our countrymen in [537] Rhemes make the like answer1, that they seek salvation no other way than by the blood of Christ; and that humbly they do use prayers, fastingt, alms, faith, charity, sacrifice, sacraments, priests, only as the means appointed by Christ, to apply the benefit of his holy blood unto them: touching our good works, that in their own natures they are not meritorious, nor answerable unto the joys of heaven; it cometh by the grace of Christ, and not of the work itself, that we have by well-doing a right to heaven, and deserve it worthily. If any man think that I seek to varnish their opinions, to set the better foot of a lame cause foremost; let him know, that sinceu I began throughly to understand their meaning, I have found their halting in this doctrinex greater than perhaps it seemeth to them which know not the deepness of Satan, as the blessed Divine speaketh2. For, although this be proof sufficient, that they do not directly deny the foundation of faith; yet, if there were no other leaven in the whole lump of their doctrine but this, this were sufficient to prove, that their doctrine is not agreeable withy the foundation of Christian faith. The Pelagians, being over-great friends unto nature, made themselves enemies unto grace, for all their confessing, that men have their souls, and all the faculties thereof, their wills and the abilityz of their wills, from God. And is not the church of Rome still an adversary unto Christ’s merits, because of her acknowledging, that we have received the power of meriting by the blood of Christ? Sir Thomas More setteth down the odds between us and the church of Rome in the matter of works thus: “Like as we grant them, [538] that no good work of man is rewardable in heaven of hisb own nature, but through the mere goodness of God, that listc to set so high a price upon so poor a thing; and that this price God setteth through Christ’s passion, and for that also that they bed his own works with us; (for good works to God-ward worketh no man, without God work in him:) and as we grant them also, that no man may be proud of his works, for his owne imperfectf working; and for that in all that man may do, he can do no goodg, but is a servant unprofitable, and doth but his bare duty: as we, I say, grant unto them these things, so this one thing or twain do they grant us again, that men are bound to work good works, if they have time and power; and that whoso worketh in true faith most, shall be most rewarded: but then set they thereto, that all his rewards shall be given him for his faith alone, and nothing for his works at all, because his faith is the thing, they say, that forceth him to work well1.” I see by this of sir Thomas More, how easy it is for men of greath capacity and judgmenti to mistake things written or spoken, as well on one side as on anotherk. Their doctrine, as he thought, maketh the worksl of man rewardable in the world to come through the merem goodness of God, whom it pleaseth to set so high a price upon so poor a thing; and ours, that a man doth receive that eternal and high reward, not for his works, but for his faith’s sake, by which he worketh: whereas in truth our doctrine is no other than that whichn we have learned at the feet of Christ; namely, that God doth justify the believing man, yet not for the worthiness of his belief, but for his worthinesso which is believed; God rewardeth abundantly every one which worketh, yet not for any meritorious dignity which is, or can be, in the work, but through his mere mercy, by whose commandment he worketh. Contrariwise, their doctrine is2, that as pure water of itself hath no savour, but if [539] it pass through a sweet pipe, it taketh a pleasant smell of the pipe through which it passeth;SERM. II. 34. so, although before grace received, our works do neither satisfy nor merit; yet after, they do both the one and the other. Every virtuous action hath then power in such sortp to satisfy; that if we ourselves commit no mortal sin, no heinous crime, whereupon to spend this treasure of satisfaction in our own behalf, it turneth to the benefit of other men’s release, on whom it shall please the steward of the house of God to bestow it; so that we may satisfy for ourselves and others, but merit only for ourselves. In meriting, our actions do work with two hands: with theq one, they get their morning stipend, the increase of grace; with the other, their evening hire, the everlasting crown of glory. Indeed they teach, that our good works do not these things as they come from us, but as they come from grace in us; which grace in us is another thing in their divinity, than is the mere goodness of God’s mercy towardr us in Christ Jesus.

34. If it were not a strong deluding spirit which hath possession of their hearts; were it possible but that they should see how plainly they do herein gainsay the very grounds of apostolic faith? Is this that salvation by grace, whereof so plentiful mention is made in the sacredt Scriptures of God? was this their meaning, which first taught the world to look for salvation only by Christ? By grace, the Apostle saith, and by grace in such sort as a gift; a thing that cometh not of ourselves, not of our works, lest any man should boast [540] and say, “I have wrought out mine own salvation1.”SERM. II. 35. By grace they confess; but by grace in suchu sort, that as many as wear the diadem of bliss, they wear nothing but what they have won. The Apostle, as if he had foreseen how the church of Rome would abuse the world in time by ambiguous terms, to declare in what sense the name of grace must be taken, when we make it the cause of our salvation, saith, “He saved us according to his mercy;” which mercy, although it exclude not the washing of our new birth, the renewing of our hearts by the Holy Ghost, the means, the virtues, the duties, which God requireth atw their hands which shall be saved; yet it is so repugnant unto merits, that to say, we are saved for the worthiness of any thing which is ours, is to deny we are saved by Grace. Grace bestoweth freely; and therefore justly requireth the glory of that which is bestowed. We deny the grace of our Lord Jesus Christ; we imbasex, disannul, annihilatey the benefit of his bitter passion, if we rest in thosez proud imaginations, that life everlastinga is deservedly ours, that we merit it, and that we are worthy of it.

35. Howbeit, considering how many virtuous and just men, how many saints, how many martyrs, how many of the ancient Fathers of the church, have had their sundry perilous opinions; and among sundry of theirb opinions this, that they hoped to make God some part of amends for their sins, by the voluntary punishmentsc which they laid upon themselves; because by a consequent it may follow hereupon, that they were injurious untod Christ, shall we therefore make such deadly epitaphs, and set them upon their graves, “They denied the foundation of faith directly, they are damned, there is no salvation for them?” St. Augustine hath saide of himself, Errare possum, hæreticus esse nolo2. And, except we put a difference between them that err, and them [541] that obstinately persist in error, how is it possible that ever any man should hope to be saved? Surely, in this case, I have no respect of any person alive or dead. Give me a man, of what estate or condition soever, yea, a cardinal or a pope, whom atf the extreme point of his life affliction hath made to know himself; whose heart God hath touched with true sorrow for all his sins, and filled with love toward the Gospel of Christ; whose eyes are opened to see the truth, and his mouth to renounce all heresy and error any wayg opposite thereunto, this one opinion of merits excepted; whichh he thinketh God will require at his hands, and because he wanteth, therefore, trembleth, and is discouraged; it may be I am forgetful, ori unskilful, not furnished with things new and old, as a wise andk learned scribe should be, nor able to allege that, whereunto, if it were alleged, he doth bear a mind most willing to yield, and so to be recalled, as well from this, as from other errors: and shall I think, because of this only error, that such a man toucheth not so much as the hem of Christ’s garment? If he do, wherefore should not I have hope, that virtue may proceed from Christ to save him? Because his error doth by consequent overthrow his faith, shall I therefore cast him off, as one which hath utterly cast off Christ? one whichl holdeth not so much as by a slender thread? No; I will not be afraid to say unto a cardinal or to a popem in this plight, Be of good comfort, we have to do with a merciful God, ready to make the best of that littlen which we hold well, and not with a captious sophister, which gathereth the worst out of every thing wherein we err. Is there any reason that I should be suspected, or you offended, for this speech? Let all affection be laid aside; let the matter be indifferentlyo consideredp. Is it a dangerous thing to imagine, that such men may find mercy? The hour may come, when we shall think it a blessed thing to hear, that if our sins were asq the sins of the poper and cardinals, the bowels of the mercy of God are larger. I do not propose unto you a pope with the neck of an emperor under his foots; a cardinal riding his horse to the bridle in the blood of saints; but a pope or a cardinal sorrowful, [542] penitent, disrobed, striptt, not only of usurped power, but also delivered and recalled from error and uAntichrist, converted and lying prostrate at the feetw of Christ; and shall I think that Christ willx spurn at him? shally I cross and gainsay the merciful promises of God, generally made unto penitent sinners, by opposing the name of a pope orz cardinal? What difference is there in the world between a pope and a cardinal, and John a Stylea1, in this case? If we think it impossible for them, after they be once come within thatb rank, to be afterwards touched with any such remorse, let that be granted. The Apostle saith, “If I, or an angel from heaven, preach unto you,” &c. Let it be as likely, that St. Paul or an angel from heaven shouldc preach heresy, as that a pope or ad cardinal should be brought so far forth to acknowledge the truth; yet if a pope or cardinal should, what find we in their persons why they might not be saved? It is not theire persons, you will say, but the error wherein I suppose them to die, which excludeth them from hopef of mercy; the opinion of merits doth take away all possibility of salvation from them. What, althoughg they hold it only as an error? although they hold the truth soundlyh and sincerely in all other parts of Christian faith? although they have in some measure all the virtues and graces of the Spirit, all other tokens of God’s elect children in them? although they be far from having any proud presumptuous opinion, that they shall be saved fori the worthiness of their deeds? although the only thing which troubleth and molesteth them be but a little too much dejection, somewhat too great a fear, rising from an erroneous conceit that God will require a worthiness in them, which they are grieved to find wanting in themselves? although they be not obstinate in this persuasion? although they be willing, and would be glad to forsake it, if any one reason were brought sufficient to disprove it? although the only let, why they do not forsake it ere they die, be the ignorance of the meank wherebyl it might be disproved? although the cause why the ignorance [543] in this point is not removed, be the want of knowledge in such as should be able, and are not, to remove it?SERM. II. 36. Let me die, if ever it be proved, that simply an error doth exclude a pope or a cardinal, in such a case, utterly from hope of life. Surely, I must confess unto you, if it be an error to thinkm, that God may be merciful to save men even when they err1, my greatest comfort is my error; were it not for the love I bear unto this error, I would neithern wish to speak nor to live.

36. Wherefore to resume that mother-sentence, whereof I little thought that so much trouble would have grown, “I doubt not but God was merciful to save thousands of our fathers living in popish superstitions, inasmuch as they sinned ignorantly:” alas! what bloody matter is there contained in this sentence, that it should be an occasion of so many hard censures? Did I say, “That thousands of our fathers might be saved?” I have shewed which way it cannot be denied. Did I say, “I doubt ito not but they were saved?” I see no impiety in this persuasion, though I had no reason in the worldp for it. Did I say, “Their ignorance doth make me hope they did find mercy, and so were saved?” What doth hinderq salvation but sin? Sins are not equal; and ignorance, though it dor not make sins to be no sin, yet seeing it did make their sin the less, why should it not make our hope concerning their life the greater? We pity the most, and It doubt not but God hath most compassion over them that sin for want of understanding. As much is confessed by sundry others, almost in the selfsame words which I have used. It is but only my illu hap, that the same sentences which favourx verity in other men’s books, should seem to bolster heresy when they are once by me recited. If I be deceived in this point, not they, but the blessed Apostle hath deceived me2. What I said of others, the same he saithy of himself, “I obtainedz mercy3, for I [544] did it ignorantly.”SERM. II. 37, 38. Construe his words, and yea cannot misconstrue mine. I speak no otherwise, I meant no otherwiseb.

37. Thus have I brought the question concerning our fathers at the length unto an end. Of whose estate, upon so fit an occasion as was offered me, handling the weighty causes of separation between the church of Rome and us, and the weak motives which commonly arec brought to retain men in that society; amongst which motives the exampled of our fathers deceased is one; although I saw it convenient to utter that sentence which I did, to the end that all men might thereby understand, how untruly we are said to condemn as many as have been before us otherwise persuaded than we ourselves are: yet more than that one sentence I did not think it expedient to utter, judging it a great deal meeter for us to have regard to our own estate, than to sift over curiously what is become of other men; and fearing, lest that such questions as thise, if voluntarily they should be too far waded in, might seem worthy of that rebuke which our Saviour thought needful in a case not unlike, “What is this unto thee1?” When asf I was forced, much besides mineg expectation, to render a reason of my speech, I could not but yield at the call of others, toh proceed asi duty bound me, for the fuller satisfaction of men’s mindsk. Wherein I have walked, as with reverence, so with fear: with reverence, in regard of our fathers, which lived in former times; not without fear, considering them that are alive.

38. I am not ignorant how ready men are to feed and soothe up themselves in evil. Shall I (will the man say, that loveth thel present world more than he loveth Christ), shall I incur the high displeasure of the mightiest upon earth? shall I hazard my goods, endanger my estatem, put my life inn jeopardy, rather than yield to that which so many of my fathers haveo embraced, and yet found favour in the sight of God? “Curse Meroz,” saith the Lord, “curse her inhabitants, [545] because they helpp not the Lord, they helpq him not against the mighty1.”SERM. II. 38. If I should not only notr help the Lord against the mighty, but help to strengthen them that are mighty against the Lord; worthily might I fall under the burden of that curse, worthy I were to bear my own judgment. But if the doctrine which I teach be a flower gathered in the garden of the Lord, a part of the saving truth of the Gospel, from whence notwithstanding poisoneds creatures do suck venom; I can but wish it were otherwise, and content myself with the lot that hath befallen me, the rather, because it hath not befallen me alone. St. Paul did preacht a truth, and a comfortable truth, when he taught, that the greater our misery is in respect of our iniquities, the readier is the mercy of ouru God for our release, if we seek unto him; the more we have sinned, the more praise, and gloryw, and honour unto him that pardoneth our sin. But mark what lewd collections were made hereupon by some2: “Why then am I condemned for a xsinner?” And, saith the Apostle, “as we are blamed, and as some affirm that we say, ‘Why do we not evil that good may come of it?’ ” He was accused to teach that which ill-disposed men did gather by his teaching, though it were clean not only besidey, but against his meaning. The Apostle addeth, “Their condemnation which thus do is just.” I am not hasty to apply sentences of condemnation: I wish from my heart their conversion, whosoever are thus perversely affected. For I must needs say, their case is fearful, their estate dangerous, which harden themselves, presuming onz the mercy of God towards others. It is true, that God is merciful, but let us beware of presumptuous sins. God delivered Jonah from the bottom of the sea; will you therefore cast yourselves headlong from the tops of rocks, and say in your hearts, God shall deliver us? He pitieth the blind that would gladly see; but will Goda pity him that may see, and hardeneth himself in blindness? No; Christ hath spoken too much unto you, for youb to claim the privilege of your fathers.

[546]
SERM. II. 39.39. As for us that have handled this cause concerning the condition of our fathers, whether it be this thing or any other which we bring unto you, the counsel is good which the Wise Man giveth1, “Stand thou fast in thy sure understanding, in the way and knowledge of the Lord, and have but one manner of word, and follow the word of peace and righteousness.” As a loose tooth is a greatc grief unto him that eateth, so doth a wavering and unstable word, in speech that tendeth to instruction, offend. “Shall a wise man speak words of the wind2,” saith Eliphaz; light, unconstant, unstable words? Surely the wisest may speak words of the wind: such is the untoward constitution of our nature, that we neither dod so perfectly understand the way and knowledge of the Lord, nor so steadfastly embrace it, when it is understood; nor so graciously utter it, when it is embraced; nor so peaceably maintain it, when it is uttered; but that the best of us are overtaken sometimes through blindness, sometimes through hastiness, sometimes through impatience, sometime through other passions of the mind, whereunto (God doth know) we are too subject. We must therefore be contented both to pardon others, and to crave that others maye pardon us for such things. Let no man, whichf speaketh as a man, think himself (whilestg he liveth) always freed from scapes and oversights in his speech. The things themselves which I have spoken unto you I hopeh are sound, howsoever they have seemed otherwise unto some; at whose hands ifi I have, in that respect, received injury, I willingly forget it; although, in truthj, considering the benefit which I have reaped by this necessary searchk of truth, I rather incline unto that of the Apostle3, “They have not injured me at all.” I have cause to wish, and I do wishl, them as many blessings in the kingdom of heaven, as they have forced me to utter words and syllables in this cause; wherein I could not be more sparing inm speech than I have been. “It becometh no man,” saith St. Jerome4, “to be patient [547] in the crime of heresy.”SERM. II. 40. Patient, as I take it, we should be always, though the crime of heresy were intended; but silent in a thing of so great consequence, I could not, beloved, I durst not be; especially the love, which I bear to the truth inn Christ Jesus, being hereby somewhat called in question. Whereof I beseech them, in the meekness of Christ, that have been the first original cause, to consider that a watchman may cry “An enemy!” when indeed a friend cometh. In which caseo, as I deemp such a watchman more worthy to be loved for his care, than misliked for his error; so I have judged it my own part in this caseq, as much as in me lieth, to take away all suspicion of any unfriendly intent or meaning against the truth, from which, God doth know, my heart is free.

40. Now to you, beloved, which have heard these things, I will use no other words of admonition, than those which are offered me by St. James1, “My brethren, have not the faith of our glorious Lord Jesus Christr, in respect of persons.” Ye are not now to learn, that as of itself it is not hurtful, so neither should it be to any mans scandalous and offensive, in doubtful cases, to hear the differentt judgmentu of men. Be it that Cephas hath one interpretation, and Apollos hath another; that Paul is of this mind, andx Barnabas of that; if this offend you, the fault is yours. Carry peaceable minds, and yey may have comfort by this variety.

Now the God of peace give you peaceable minds, and turn it to your everlasting comfort.

[548]
A SUPPLICATION MADE TO THE COUNCIL
by MASTER WALTER TRAVERS1.
Right Honourable,
TRAVERS’ SUPPLICATION.THE manifold benefits which all the subjects within this dominion do at this present, and have many years2 enjoyed, under her Majesty’s most happy and prosperous reign, by [549] your godly wisdom and careful watching over this estate night and day, I truly and unfeignedly acknowledge, from the bottom of my heart, ought worthily to bind us all to pray continually to Almighty God for the continuance and increase of the life and good estate of your honours, and to be ready, with all good duties, to satisfy and serve the same to our power. Besides public benefits common unto all, I must needs, and do willingly, confess myself to stand bound by most special obligation, to serve and honour you more than anya other, for the honourable favour it hath pleased you to vouchsafe both oftentimes heretofore, and also now of late3, in a matter more dear unto me than any earthly commodity, that is, the upholding and furthering of my service in the ministering of the gospel of Jesus Christ. For which cause, as I have been always careful so to carry myself as I might by no means give occasion to be thought unworthy of so great a benefit, so do I still, next unto her majesty’s gracious countenance, hold nothing more dear and precious unto me, than that I may always remain in your honours’ favour, which hath oftentimes been helpful and comfortable unto me in my ministry, and to all such as reaped any fruit of my simple and faithful labour. In which dutiful regard I humbly beseech your honours to vouchsafe to do me this grace, to conceive nothing of me [550] otherwise than according to the duty wherein I ought to live, by any information against me, before your honours have heard my answer, and been thoroughly informed of the matter. Which, although it be a thing that your wisdoms, not in favour, but in justice, yield to all men, yet the state of the calling unto the ministry, whereunto it hath pleased God of his goodness to call me, though unworthiest of all, is so subject to misinformation, as, except we may find this favour with your honours, we cannot look for any other, but that our unindifferent parties may easily procure us to be hardly esteemed of; and that we shall be made like the poor fisher-boats in the sea, which every swelling wave and billow raketh and runneth over. Wherein my estate is yet harder than any others of my rank and calling, who are indeed to fight against flesh and blood in what part soever of the Lord’s host and field they shall stand marshalled to serve, yet many of them deal with it naked, and unfurnished of weapons: but my service was in a place where I was to encounter with it well appointed and armed with skill and with authority: whereof as I have always thus deserved, and therefore have been careful by all good means to entertain still your honours’ favourable respect of me, so have I special cause at this present, wherein misinformation to the lord archbishop of Canterbury, and other of the High Commission, hath been able so far to prevail against me, that by their letter they have inhibited me to preach, or execute any act of ministry in the Temple or elsewhere1, having never once called me before them, to understand by mine answer the truth of such things as had been informed against me. We have a story in our books, wherein the Pharisees proceeding against our Saviour Christ without having heard him is reproved by “an honourable counsellor2,” as the Evangelist doth term him, saying, “Doth our law judge a man before it hear him, and know what he hath done3?” Which [551] I do not mention, to the end that by an indirect and covert speech I might so compare those who have, without ever hearing me, pronounced a heavy sentence against me; for notwithstanding such proceedings, I purpose by God’s grace to carry myself towards them in all seeming duty agreeable to their places: much less do I presume to liken my cause to our Saviour Christ’s, who hold it my chiefest honour and happiness to serve him, though it be but among the hinds and hired servants that serve him in the basest corners of his house. But my purpose in mentioning it is, to shew, by the judgment of a prince and great man in Israel, that such proceeding standeth not with the law of God, and in a princely pattern to shew it to be a noble part of an honourable counsellor, not to allow of indirect dealings, but to loveb and affect such a course in justice as is agreeable to the law of God. We have also a plain rule in the word of God, not to proceed any otherwise against any elder of the Church; much less against one that laboureth in the word and in teaching. Which rule is delivered with this most earnest charge and obtestation, “I beseech and charge thee in the sight of God, and the Lord Jesus Christ, and the elect angels, that thou keep thesec rulesd without preferring one before another, doing nothing of partiality, or inclining to either part1;” which apostolical and most earnest charge, I refere to your honours’ wisdom how it hath been regarded in so heavy a judgment against me, without ever hearing my cause; and whether, as having God before their eyes, and the Lord Jesus, by whom all former judgments shall be tried again; and, as in the presence of the elect angels, witnesses and observers of the regiment of the Church, they have proceeded thus to such a sentence. They allege indeed two reasons in their letters, whereupon they restrain my ministry; which, if they were as strong against me as they are supposed, yet I refer to your honours’ wisdoms, whether the quality of such an offence as they charge me with, which is in effect but an indiscretion, deserve so grievous a punishment both to the Church and me, in taking away my ministry, and that poor little commodity which it yieldeth [552] for the necessary maintenance of my life; if so unequal a balancing of faults and punishments should have place in the commonwealth, surely we should shortly have no actions upon the case, nor of trespass, but all should be pleas of the crown, nor any man amerced, or fined, but for every light offence put to his ransom. I have credibly heard, that some of the ministry1 have been convicted off grievous transgressions of the laws of God and men, being of no ability to do other service in the Church than to read; yet hath it been thought charitable, and standing with Christian moderation and temperancy, not to deprive such of ministry and beneficeg, but to inflict some more tolerable punishment. Which I write not because such, as I think, were to be favoured, but to shew how unlike their dealing is with me, being through the goodness of God not to be touched with any such blame; and one who according to the measure of the gift of God have laboured now some years painfully, in regard of the weak estate of my body, in preaching the gospel, and as I hope not altogether unprofitably in respect of the Church. But I beseech your honours to give me leave briefly to declare the particular reasons of their letterh, and what answer I have to make unto it.

The first is, that, as they say, “I am not lawfully called to the function of the ministry, nor allowed to preach, according to the laws of the Church of England.”

For answer to this, I had need to divide the points. And first to make answer to the former; wherein leaving to shew what by the holy Scriptures is required in a lawful calling, and that all that is to be found in mine, that I be not too long for your other weighty affairs, I rest in this answeri.

My calling to the ministry was such as in the calling of any thereunto is appointed to be used by the orders agreed upon in the national synods of the Low Countries2, for the direction and guidance of their churches; which orders are the same with those whereby the French and Scottish churches [553] are governed; whereof I have shewed such sufficient testimonial to my lord the Archbishop of Canterbury, as is requisite in such a matter: whereby it must needs fall out, if any man be lawfully called to the ministry in those churches, then is my calling, being the same with theirs, also lawful. But I suppose, notwithstanding they use this general speech, they mean only, my calling is not sufficient to deal in the ministry within this land, because I was not made minister according to that order, which in this cause is ordained by our laws. Whereunto I beseech your honours to consider throughly of mine answer, because exception now again is taken to my ministry, whereas, having been heretofore called in question for it1, I so answered the matter, as I continued inj my ministry, and, for any thing I discerned, looked to hear that no more objected unto me. The communion of saints (which every Christian man professeth to believe) is such as, that the acts which are done in any true church of Christ’s according to his word, are held as lawful being done in one church, as in another. Which, as it holdeth in other acts of ministry, as baptism, marriage, and such like, so doth it in the calling to the ministry; by reason whereof, all churches do acknowledge and receive him for a minister of the word, who hath been lawfully called thereunto in any church of the same profession. A Doctor created in any university in Christendom, is acknowledged sufficiently qualified to teach in any country. The church of Rome itself, and the canon law holdeth it, that being ordered in Spain, they may execute that that belongeth to their order, in Italy, or in any other place. And the churches of the Gospel never made any question of it: which if they shall now begin to make doubt of, and deny such to be lawfully called to the ministry, as are called by another order than our own; then may it well be looked for, that other churches will do the like: and if a minister called in the Low Countries be not lawfully called in England, then may they say to our preachers which are there, that being made by another order than [554] theirs, they cannot suffer them to execute any act of ministry amongst them; which in the end must needs breed a schism, and dangerous division in the churches. Further, I have heard of those that are learned in the laws of this land, that by express statute to that purpose, anno 13 of her majesty’s reign1, upon subscription to the articles agreed upon, anno 1562, that they who pretend to have been ordered by another order than that which is now established, are of like capacity to enjoy any place of ministry within the land, as they which have been ordered according to that which is now by law in thisk established. Which comprehending manifestly all, even such as were made priests according to the order of the Church of Rome, it must needs be, that the law of a Christian land, professing the Gospel, should be as favourable for a minister of the word, as for a popish priest; which also was so found in Mr. Whittingham’s case2, who, notwithstanding such replies against him3, enjoyed still the benefit he had [555] by his ministry, and might have done until this day, if God had spared him life so long; which if it be understood so, and practised in others, why should the change of the person alter the right which the law giveth to all other?

The place of ministry whereunto I was called was not presentative: and if it had been so, surely they would never have presented any man whom they never knew; and the order of this church is agreeable herein to the Word of God, and the ancient and best canons, that no man should be made a minister sine titulo: therefore having none, I could not by the orders of this church have entered into the ministry, before I had a chargel to tend upon. When I was at Antwerp, and to take a place of ministry among the people of that nation, I see no cause why I should have returned again over the seas for orders here; nor how I could have done it, without disallowing the orders of the churches provided in the country where I was to live. Whereby I hope it appeareth, that my calling to the ministry is lawful, and maketh me, by our law, of capacity to enjoy any benefit or commodity, that any other, by reason of his ministry, may enjoy. But my case is yet more easy, who reaped no benefit of my ministry by law, receiving only a benevolence and voluntary contribution; and the ministry I dealt with being preaching only, which every deacon here may do being licensed, and certain that are neither1 ministers nor deacons. Thus I answer the former of these two points, whereof, if there be yet any doubt, I humbly desire, for a final end [556] thereof, that some competent judges in law may determine of itm; whereunto I refer and submit myself with all reverence and duty.

The second is, “That I preached without license.” Whereunto this is my answer: I have not presumed, upon the calling I had to the ministry abroad, to preach or deal with any part of the ministry within this church, without the consent and allowance of such as were to allow me unto it. My allowance was from the bishop of London, testified by his two several letters to the Inner Temple, who, without such testimony, would by no means rest satisfied in it: which letters being by me produced, I refer it to your honours’ wisdom, whether I have taken upon me to preach, without being allowed (as they charge) according to the orders of the realm. Thus having answered the second point also, I have done with the objection, “Of dealing without calling or license.”

The other reason they allege is, concerning a late action, wherein I had to deal with Mr. Hooker, Master of the Temple. In the handling of which cause, they charge me with an indiscretion, and want of duty, “in that I inveighed,” as they say, “against certain points of doctrine taught by him, as erroneous, not conferring with him, nor complaining of it to them.” My answer hereunto standeth, in declaring to your honours the whole course and carriage of that cause, and the degrees of proceeding in it, which I will do as briefly as I can, and according to the truth, God be my witness, as near as my best memory, and notes of remembrance, may serve me thereunto. After that I have taken away that which seemed to have moved them to think me not charitably minded to Mr. Hooker; which is, because he was brought into Mr. Alvey’s place, wherein this church desired that I might have succeeded: which place, if I would have made suit to have obtained, or if I had ambitiously affected and sought, I would not have refused to have satisfied, by subscription, such as the matter then seemed to depend upon: whereas contrariwise, notwithstanding I would not hinder the church to do that they thought to be the most for their edification and comfort, yet did I, neither by speech [557] nor letter, make suit to any for the obtaining of it, following herein that resolution, which I judge to be most agreeable to the word and will of God; that is, that labouring and suing for places and charges in the church is not lawful. Further, whereasn, at the suit of the church, some of your honours entertained the cause, and brought it to a near issue, that there seemed nothing to remain, but the commendation of my lord the archbishop of Canterbury, when as he could not be satisfied, but by my subscribing to his late articles1; and that my answer (agreeing to subscribe according to any law, and to the statute provided in that case, but praying to be respited for subscribing to any other, which I could not in conscience do, either for the Temple (which otherwise he said he would not commend me to), nor for any other place in the Church) did so little please my lord archbishop, as he resolved that otherwise I should not be commended to it: I had utterly here no cause of offence against Mr. Hooker, whom I did in no sort esteem to have prevented or undermined me, but that God disposed of me as it pleased him, by such means and occasions as I have declared.

Moreover, as I haveo taken no cause of offence at Mr. Hooker for being preferred, so there were many witnesses, that I was glad that the place was given him, hoping to live in all godly peace and comfort with him, both for acquaintance and good-will which hath been between us, and for some kindp of affinity in the marriage of his nearest kindred and mine2. Since his coming, I have so carefully endeavoured to entertain all good correspondence and agreement with him, as I think he himself will bear me witness of many earnest disputations and conferences with him about the matter; the rather, because that, contrary to my expectation, he inclined from the beginning but smally thereunto, but joined rather with such as had always opposed themselves to any good order in this churchq, and made themselves to be [558] thoughtr indisposed to thiss present state and proceedings. For, both knowing that God’s commandment charged me with such duty, and discerning how much our peace might further the good service of God and his Church, and the mutual comfort of us botht, I had resolved constantly to seek for peace; and though it should fly from me (as I saw it did by means of some, who little desired to see the good of our church), yet according to the rule of God’s word, to follow after it. Which being so (as hereof I take God to witness, who searcheth the heart and reins, and who by his Son will judge the world, both quick and dead), I hope no charitable judgment can suppose me to have stood evilaffected towards him for his place, or desirous to fall into any controversy with him.

Which my resolution I sou pursued, that, whereas I discovered sundry unsound matters in his doctrine (as many of his sermons tasted of some sour leaven or other), yet thus I carried myself towards him. Matters of smaller weight, and so covertly deliveredx, that no great offence to the Church was to be feared in them, I wholly passed by, as one that discerned nothing of them, or had been unfurnished of replies; othersy of great moment, and so openly delivered, as there was just cause of fear lest the truth and Church of God should be prejudiced and perilled by it, and such as the conscience of my duty and calling would not suffer me altogether to pass over, this was my course; to deliver, when I should have just cause by my text, the truth of such doctrine as he had otherwise taught, in general speeches, without touch of his person in any sort, and further at convenient opportunity to confer with him on such points.

According to which determination, whereas he had taught certain things concerning predestination otherwise than the Word of God doth, as it is understood by all churches professing the gospel, and not unlike that wherewith Corranus1 [559] sometime troubled thisz church, I both delivered the truth of such points in a general doctrine, without any touch of him in particular, and conferred with him also privately upon such articles. In which conference, I remember, when I urged the consent of all churches and good writers against him that I knew; and desired, if it were otherwise, to understanda what authors he had seen ofb such doctrine: he answered me, that his best author was his own reason; which I wished him to take heed of, as a matter standing morec with Christian modesty and wisdom in a doctrine not received by the Church, not to trust to his own judgment so far as to publish it before he had conferred with others of his profession labouring by daily prayer and study to know the will of God, as he did, to see how they understood such doctrine. Notwithstanding, he, with wavering, repliedd, that he would some other time deal more largely in the matter. I wished him, and prayed him not so to do, for the peace of the Church, which, by such means, might be hazarded; seeing he could not but think, that men, who make any conscience of their ministry, will judge it a necessary duty in them to teach the truth, and to convince the contrary.

Another time, upon like occasion of this doctrine of his, “That the assurance of that we believe by the word, is not so [560] certain, as of that we perceive by sense1;” I both taught the doctrine otherwise, namely, the assurance of faith to be greater, which assurede both of things above, and contrary to all sense and human understanding, and dealt with him also privately upon that point: according to which course of late, when as he had taught, “That the church of Rome is a true Church of Christ, and a sanctified Church by profession of that truth, which God hath revealed unto us by his Son, though not a pure and perfect Church;” and further, “That he doubted not, but that thousands of the Fathers, which lived and died in the superstitions of that church, were saved, because of their ignorance, which excusedf them;” misalleging to that end a text of Scripture to prove it2: the matter being of set purpose openly and at large handled by him, and of that moment, that might prejudice the faith of Christ, encourage the ill-affected to continue still in their damnable ways, and others weak in faith to suffer themselves easily to be seduced to the destruction of their souls; I thought it my most bounden duty to God and to his Church, whilst I might have opportunity to speak with him, to teach the truth in a general speech in such points of doctrine.

At which time I taught, “That such as die, or have died at any time in the church of Rome, holding in their ignorance that faith which is taught in it, and namely, justification in part by works, could not be said by the Scriptures to be saved.” In which matter, foreseeing that if I waded not warily in it, I should be in danger to be reported (as hath fallen out since notwithstanding) to condemn all the fathers, I said directly and plainly to all men’s understanding, “That it was not indeed to be doubted, but many of the fathers were saved; but the means,” said I, “was not their ignorance, which excuseth no man with God, but their knowledge and faith of the truth, which, it appeareth, God vouchsafed them, by many notable monuments and records extant of it in all ages.” Which being the last point in all my sermon, rising so naturally from the text I then propoundedg, as would have occasioned me to have delivered such matter, notwithstanding [561] the former doctrine had been sound; and being dealt in by a general speech, without touch of his particular; I looked not that a matter of controversy would have been made of it, no more than had been of my like dealing in former time. But, far otherwise than I looked for, Mr. Hooker, shewing no grief orh offence taken at my speech all the week long, the next Sabbath, leaving to proceed upon his ordinary text, professed to preach again that he had done the day before, for some question that his doctrine was drawn into, which he desired might be examined with all severity.

So proceeding, he bestowed his whole time, in that discourse, confirmingi his former doctrine, and answering the places of Scripture which I had alleged1 to prove that a man dying in the church of Romek is not to be judged by the Scriptures to be saved. In which long speech, and utterly impertinent to his text, under colour of answering for himself, he impugned directly and openly to all men’s understanding, the true doctrine which I had delivered; and, addingl to his former points some other like (as willingly one error followeth another), that is, “That the Galathians joining, with faith in Christ, circumcision, as necessary to salvation, mightm be saved; and that they of the church of Rome may be saved by such a faith of Christ as they had, with a general repentance of all their errors, notwithstanding their opinion of justification in part by their works and merits:” I was necessarily, though not willingly, drawn to say something to the points he objected against sound doctrine; which I did in a short speech in the end of my sermon, with protestation of so doing not of any sinister affection to any man, but to bear witness to the truth according to my calling; and wished, if the matter should needs further be dealt in, some other more convenient way might be taken for it. Wherein, I hope, my dealing was manifest to the consciences of all indifferent hearers of me that day, to have been according to peace, and without any uncharitableness, being duly considered.

For that I conferred notn with him the first day, I have shewed that the cause requiring of me the duty at the least not to be altogether silent in it, being a matter of such consequence, [562] the time also being short wherein I was to preach after him, the hope of the fruit of our communication being small upon experience of former conferences, and my expectation being that the Church should be no further troubled with it, upon the motion I made of taking some other course of dealing; I suppose my deferring to speak with him till some fit opportunity, cannot in charity be judged uncharitable.

The second day, his unlooked-for opposition with the former reasons, made it to be a matter that required of necessity some public answer; which being so temperate as I have shewed, if notwithstanding it be censured as uncharitable, and punished so grievously as it is, what should have been my punishment, if (without all such cautions and respects as qualified my speech) I had before all, and in the understanding of all, so reproved him offending openly, that others might have feared to do the like? which yet, if I had done, might have been warranted by the rule and charge of the1 Apostleo, “Them that offend openly, rebuke openly, that the rest may also fear;” and by his example, who, when Peter in this very case which is now between us, had, not in preaching, but in a matter of conversation, not “gone with a right foot, as was fit for the truth of the Gospel2,” conferred not privately with him, but, as his own rule required, reproved him openly before all, that others might hear, and fear, and not dare to do the like. All which reasons together weighed, I hope, will shew the manner of my dealing to have been charitable, and warrantable in every sort.

The next Sabbath day after this, Mr. Hooker kept the way he had entered into before, and bestowed his whole hour3 and more only upon the questions he had moved and maintained; wherein he so set forth the agreement of the church of Rome with us, and their disagreement from us, as if we had consented in the greatest and weightiest points, and differed only in certain smaller matters: which agreement noted [563] by him in two chief points, is not such as he would have made men believe. The one, in that he said, “They acknowledge all men sinners, even the blessed Virgin, though some of them freed her from sin;” for the council of Trent holdeth1, that she was free from sin. Another, in that he said, “They teach Christ’s righteousness to be the only meritorious cause of taking away sin, and differ from us only in the applying it:” for Thomas Aquinas their chief schoolman2, and archbishop Catherinus3, teach, “That Christ took away only original sin, and that the rest are to be taken away by ourselves;” yea, the council of Trent teacheth, “That righteousness whereby we are righteous in God’s sight, is an inherent righteousness;” which must needs be of our own works, and cannot be understood of the righteousness inherent only in Christ’s person, and accounted unto us. Moreover he taught the same time, “That neither the Galathians, nor the church of Rome, did directly overthrow the foundation of [564] justification by Christ alone, but only by consequent, and therefore might well be saved; or else neither the churches of the Lutherans, nor any which hold any manner of error could be saved; because,” saith he, “every error by consequent overthroweth the foundation.” In which discourses, and such like, he bestowed his whole time and more; which, if he had affected either the truth of God, or the peace of the Church, he would truly not have done.

Whose example could not draw me to leave the Scripture I took in hand, but standing about an hour to deliver the doctrine of it, in the end, upon just occasion of the text, leaving sundry other his unsound speeches, and keeping me still to the principal, I confirmed the believing the doctrine of justification by Christ only, to be necessary to the justification of all that should be saved, and that the church of Rome directly denieth, that a man is saved by Christ, or by faith alone, without the works of the law. Which my answer, as it was most necessary for the service of God and the Church, so was it without any immodest or reproachful speech pto Mr. Hooker: whose unsound and wilful dealings in a cause of so great importance to the faith of Christ, and salvation of the Church, notwithstanding I knew well what speech it deserved, and what some zealous earnest man of the spirit of John and James1, surnamed Boanerges, Sons of Thunder, would have said in such case; yet I chose rather to content myself in exhorting him to revisit his doctrine, as Nathan2 the prophet did the device, which, without consulting with God, he had of himself given to David, concerning the building of the temple: and, with Peter the Apostle3, to endure to be withstood in such a case, not unlike unto this. This in effect was that which passed between us concerning this matter, and the invectives I made against him, wherewith I am charged. Which rehearsal, I hope, may clear me (with all that shall indifferently consider it) of the blames laid upon me for want of duty to Mr. Hooker in not conferring with him, whereof I have spoken sufficiently already; and to the High Commission, in not revealing the matter to them, which yet now I am further to answer. My answer is, that I [565] protest, no contempt nor wilful neglect of any lawful authority stayed me from complaining unto them, but these reasons following:

First, I was in some hope, that Mr. Hooker, notwithstanding he had been over-carried, with a show of charity, to prejudice the truth, yet when it should be sufficiently proved, would have acknowledged it, or at the least induced with peace, that it might be offered withoutq any offence to him, to such as would receive it; either of which would have taken away any cause of just complaint. When neither of these fell out according to my expectation and desire, but that he replied to the truth, and objected against it, I thought he might have some doubts and scruples in himself; which yet, if they were cleared, he would either embrace soundr doctrine, or at least suffer it to have its course: which hope of him I nourished so long, as the matter was not bitterly and immodestly handled between us.

Another reason was the cause itself, which, according to the parable of the tares, (which are said to be sown among the wheat,) sprung up first in his grass: therefore, as the servants in that place are not said to have come to complain to the Lord, till the tares came to shew their fruits in their kind; so I, thinking it yet but a time of discovering ofs what it was, desired not their sickle to cut it down.

For further answer, it is to be considered, that the conscience of my duty to God, and to his Church, did bind me at the first, to deliver sound doctrine in such points as had been otherwise uttered in that place, where I had now some years taught the truth; otherwise the rebuke of the Prophet1 had fallen upon me, for not going up to the breach, and standing in it, and the peril of answering fort the blood of the city, in whose watch-tower I sate; if it had been surprised by my default. Moreover, my public protestation, in being willingu, that if any were not yet satisfied, some other more convenient way might be taken for it. And, lastly, that I had resolved (which I uttered before to some, dealing with me about the matter) to have protested the next sabbath day, that I would [566] no more answer in that place any objections to the doctrine taught by any means, but some other way satisfy such as should require it.

These, I trust, may make it appear, that I failed not in duty to authority, notwithstanding I did not complain, nor give over so soon dealing in the case. If I did, how is he clear, which can allege none of all these for himself? who leaving the expounding of the Scriptures, and his ordinary calling, voluntarily discoursed upon school points and questions, neither of edification nor of truth? Who after all this, as promising to himself, and to untruth, a victory by my silence, added yet in the next sabbath day, to the maintenance of his former opinions, these which follow:

“That no additament taketh away the foundation, except it be a privative; of which sort neither the works added to Christ by the church of Rome, nor circumcision by the Galathians, were; as one denieth him not to be a man, that saith, he is a righteous man, but he that saith he is a dead man:” whereby it might seem, that a man might, without hurt, add works to Christ, and pray also that God and St. Peter would save them.

“That the Galathians’ case is harder than the case of the church of Rome, because the Galathians joined circumcision with Christ, which God had forbidden and abolished; but that which the church of Rome joined with Christ, were good works, which God had commanded.” Wherein he committed a double fault: one, in expounding all the questions of the Galathians, and consequently of the Romans, and other Epistles, of circumcision only, and the ceremonies of the law (as they do, who answer for the church of Rome in their writings), contrary to the clear meaning of the Apostle, as may appear by many strong and sufficient reasons; the other, in that he said, “The addition of the church of Rome was of works commanded of God.” Whereas the least part of the works whereby they looked to merit, was of such works; and most were worksx of supererogation, andy works which God never commanded, but was highly displeased with, as of masses, pilgrimages, pardons, pains of purgatory, and such like. Further, “That no one sequel urged by the Apostle [567] against the Galathians for joining circumcision with Christ, but might be as well enforced against the Lutherans; that is, that for their ubiquity it may be as well said to them, If ye hold the body of Christ to be in all places, you are fallen from grace, you are under the curse of the law, saying, ‘Cursed be he that fulfilleth not all things written in this Book,’ ” with such like. He added yet further, “That to a bishop of the church of Rome, to a cardinal, yea, to the pope himself, acknowledging Christ to be the Saviour of the world, denying other errors, and being discomforted for want of works whereby he might be justified, he would not doubt, but use this speech; Thou holdest the foundation of Christian faith, though it be but by a slender thread; thou holdest Christ, though but by the hem of his garment; why shouldest thou not hope that virtue may pass from Christ to save thee? That which thou holdest of justification by thy works, overthroweth indeed by consequent the foundation of Christian faith; but be of good cheer, thou hast not to do with a captious sophister, but with a merciful God, who will justify thee for that thou holdest, and not take the advantage of doubtful construction to condemn thee. And if this (said he) be an error, I hold it willingly; for it is the greatest comfort I have in the world, without which I would not wish either to speak or live.” Thus far, being not to be answered in it any more, he was bold to proceed, the absurdity of which speech I need not to stand upon. I think the like to this, and other such in this sermon, and the rest of this matter, hath not been heard in public places within this land since Queen Mary’s days. What consequence this doctrine may be of, if he be not by authority ordered to revoke it, I beseech your honours, as the truth of God and his gospel is dear and precious unto you, according to your godly wisdom to consider.

I have been bold to offer to your honours a long and tedious discourse of these matters; but speech being like to tapestry, which, if it be folded up, sheweth but part of that which is wrought, and being unlapt and laid open, sheweth plainly to the eye all the work that is in it; I thought it necessary to unfold this tapestry, and to hang up the whole chamber of it in your most honourable senate, that so you [568] may the more easily discern of all the pieces, and the sundry works and matters contained in it. Wherein my hope is, your honours may see I have not deserved so great a punishment as is laid upon the Church for my sake, and also upon myself, in taking from me the exercise of my ministry. Which punishment, how heavy it may seem to the Church, or fall out indeed to be, I refer it to them to judge, and spare to write what I fear, but to myself it is exceeding grievous, for that it taketh from me the exercise of my calling. Which I do not say is dear unto me, as the means of that little benefit whereby I live (although this be a lawful consideration, and to be regarded of me in due place, and of the authority under whose protection I most willingly live, even by God’s commandment both unto them and unto me); but which ought to be more precious unto me than my life, for the love which I should bear to the glory and honour of Almighty God, and to the edification and salvation of his Church, for that my life cannot any other way be of like service to God, nor of such use and profit to men by any means. For which cause, as I discern how dear myz ministry ought to be unto me, so it is my hearty desire, and most humble request unto God, to your honours, and to all the authority I live under, to whom any dealing herein belongeth, that I may spend my life (according to his example1, who in a word of like sound, buta of fuller sense, comparing by it the bestowing of his life to the offering poured out) upon the sacrifice of the faith of God’s people, and especially of this church, whereupon I have already poured out a great part thereof in the same calling, from which I stand now restrained. And if your honours shall find it so, that I have not deserved so great a punishment, but rather performed the duty which a good and faithful servant ought, in such case, to do to his Lord and the people he putteth him in trust withal carefully to keep; I am a most humble suitor by these presents to your honours, that, by your godly wisdom, some good course may be taken for the restoring of me to my ministry and place again. Which so great a favour, shall bind me yet in a greater obligation of duty (which is already so great, as it seemed nothing could [569] be added unto it to make it greater) to honour God daily for the continuance and increase of your good estate, and to be ready, with all the poor means God hath given me, to do your honours that faithful service I may possibly perform. But if, notwithstanding my cause be never so good, your honours can by no means pacify such as are offended, nor restore me again, then am I to rest in the good pleasure of God, and to commend to your honours’ protection, under her Majesty’s, my private life, while it shall be led in duty; and the Church to him, who hath redeemed to himself a people with his precious blood, and is making ready to come to judge both the quick and dead, to give to every one according as he hath done in this life, be it good or evil; to the wicked and unbelievers, justice unto death; but to the faithful, and such as love his truth, mercy and grace to life everlasting.

Your Honours’ most bounden, and 
Most humble Supplicantb,
WALTER TRAVERS, 
Minister of the Gospelc.
[570]
MR. HOOKER’S ANSWER TO THE SUPPLICATION THAT MR. TRAVERS MADE TO THE COUNCIL.
To my Lord of Canterbury his Grace1.
ANSWER to TRAVERS. 1.MY duty in most humble wise remembered, may it please your Grace to understand, that whereas there hath been a late controversy raised in the Temple, and pursued by Mr. Travers, upon conceit taken at some words by me uttered with a most simple and harmless meaning; in the heat of which pursuit, after three public invectives2, silence being enjoined him by authority, he hath hereupon for defence of his proceedings, both presented the right honourable Lords and othera of her Majesty’s privy council with a writing, and also caused or suffered the same to be copied out and spread through the hands of so many, that well nigh all sorts of men have it nowb in their bosoms3; the matters wherewith I am therein charged being of such quality as they are, and myself being better known to your Grace than to any other of their Honours besides, I have chosen to offer to your Grace’s hands a plain declaration of my innocency, in all those things wherewith I am so hardly and heavilyc charged, lest if I still [571] remain silent, that which I do for quietness’ sake, be taken as an argument that I lack what to speak truly and justly in mine own defence.ANSWER to TRAVERS. 2, 3, 4.

2. First, because Mr. Travers thinketh it expedientd to breed an opinion in men’s minds, that the root of all inconvenient events which are now sprung out, is the surly and unpeaceable disposition of the man with whom he hath to do; therefore the first in the rank of accusations laid against me, is my inconformity, which have so little inclined to so many and so earnest exhortations and conferences, as myself, he saith, can witness to have been spent upon me, for my better fashioning unto good correspondence and agreement.

3. Indeed when at the first, by means of special well-willers, without any suit of mine, as they very well know, (although I do not think it had been a mortal sin, in a reasonable sort to have shewed a moderate desire that way1,) yet when by their endeavour without instigation of mine, some reverend and honourable, favourably affecting me, had procured her Majesty’s grant of the place; at the very point of my entering thereinto, the evening before I was first to preach, he came, and two other gentlemen joined with him in the charge of thise church, (for so he gave me to understand,) though not in the same kind of charge with himf: the effect of his conference then was, that he thought it his duty to advise me not to enter with a strong hand, but to change my purpose of preaching there the next day, and to stay till he had given notice of me to the congregation, that so their allowance might seal my calling. The effect of mineg answer was, that as in place where such order is, I would not break it; so here where it never was, I might not of mine own head take upon me to begin it: but liking very well the motion, for the opinion which I had of his good meaning who made it, requested him not to mislike my answer, though it were not correspondent to his mind.

4. When this had so displeased some, that whatsoever was afterwards done or spoken by me, it offended their taste, angry informations were daily sent out, intelligence given far and [572] wide, what a dangerous enemy was crept in;ANSWER to TRAVERS. 5. the worst that jealousy could imagine was spoken and written to so many, that at the length some knowing me well, and perceiving how injurious the reports were, which grew daily more and more unto my discredit, wrought means to bring Mr. Travers and me to a second conference. Wherein when a common friend unto us both had quietly requested him to utter those things wherewith he found himself any way aggrievedh, he first renewed the memory of my entering into this charge by virtue only of a human creature (for so the want of that formality1 of popular allowance was then censured); and unto this was annexed a catalogue, partly of causeless surmises, as that I had conspired against him, and that I sought superiority over him; and partly of faults, which to note, I should have thought it a greater offence than to commit, if I did account them faults, and had heard them so curiously observed in any other than myself, they are such silly things; as praying in the entrance of my sermons only, and not in the end2, naming bishops in my prayer, kneeling when I pray, and kneeling when I receive the Communion, with such like, which I would be as loth to recite, as I was sorry to hear them objected, if the rehearsal thereof were not by him thus wrested from me. These are the conferences wherewith I have been wooed to entertain peace and good agreement.

5. As for the vehement exhortations he speaketh of, I would gladly know some reason wherefore he thought them needful to be used. Was there any thing found in my speeches or dealings, whichi gave them occasion, who are studious of peace, to think that I disposedk myself to some unquiet kind of proceedings? Surely the special providence of God I do now see it was, that the first words I spake in this place should make the first thing whereof I am accused to appear not only untrue, but improbable, to as many as then heard mel with [573] indifferent ears, and do I doubt not in their consciences clear me of this suspicion. Howbeit, I grant this were nothing, if it might be shewed, that my deeds following were not suitable to my words. If I had spoken of peace at the first, and afterwards sought to molest and grieve him, by crossing him in his function, by storming if my pleasure were not asked and my will obeyed in the least occurrencesm, by carping needlessly sometimes at the manner of his teaching, sometimes at this, sometimesn at that point of his doctrine; I might then with some likelihood have been blamed, as one disdaining a peaceable hand when it hath been offered. But if I be able (as I am) to prove that myself have now a full year together borne the continuance of such dealings, not only without any manner of resistance, but also without any such complaint as might let or hinder him in his course; I see no cause in the world, why of this I should be accused, unless it be, lest I should accuse, which I meant not. If therefore I have given him occasion to use conferenceso and exhortations unto peace, if when they were bestowed upon me I have despised them, it will not be hard to shew some one word or deed wherewith I have gone about to work disturbance: one is not much, I require but one. Only I require if any thing be shewed, it may be proved, and not objected only, as this is, “That I have joined with such as have always opposed to any good order in this church, and made themselves to be thought indisposed to the present estate and proceedings.” The words have reference, as it seemeth, unto some such things, as being attempted before my coming to the Temple, went not so effectually perhaps forward as he which devised them would have wished. An order, as I learn, there was tendered, that communicants should neither kneel, as in the most places of the realm1; nor sit, as in this place the custom is; but walk to the one side of the table, and there standing till they had received, pass afterward away round about by the other2. Which being on a [574] sudden begun to be practised in the church, some sat wondering what it should mean, othersp deliberating what to do: till such time as at length by name one of them being openly calledq thereunto, requested that they might do as they had been accustomed; which was granted, and as Mr. Travers had ministered his way to the rest, so a curate was sent to minister to them after their way. Which unprosperous beginning of a thing (saving only for the inconvenience of needless alterations, otherwise harmless) did so disgrace that order in their conceit who had to allow or disallow it, that it took no place. For neither they could ever induce themselves to think it good, and it so much offended Mr. Travers, who supposed it to be the best, that he since that time, although contented himself to receive it as they do at the hands of others, yet hath not thought it meet theyr should ever receive it out of his, which would not admit that order of receiving it, and therefore in my time hath been always present not to minister but only to be ministered unto.

6. Another order there was likewise devised, an orders of much more weight and importance. This soil, in respect of certain immunities and other specialties belonging unto it, seemed likely to bear that which in other places of the realm of England doth not take. For which cause request was made to some of her majesty’s privy council, that whereas it is provided by a statute there should be collectors and sidemen in churches, which thing, or somewhat correspondent unto it, this place did greatly want, it would please their Honours to motion such a matter to the Ancients of the Temple. And, according to their honourable manner of helping forward all motionst so grounded, they wrote their letters, as I am informed, to that effect. Whereupon, although these Houses never had use of such collectors and sidemen as are appointed in other places, yet they both erected a box to receive men’s devotion for the poor, appointing the treasurer of both Houses to take care for bestowing it where need is; and grantedu further, that if any could be intreated (as in the end some were) to undertake the labour of observing men’s slackness in divine duties, they should be allowed, their complaints heard [575] at all times, and the faults they complained of, if Mr. Travers’x private admonition did not serve, then by some other means redressed, but according to the old received orders of both Housesy.ANSWER to TRAVERS. 6. Whereby the substance of their Honours’ letters wasz indeed fully satisfied. Yet because Mr. Travers intended not this, but as it seemetha, another thing; therefore notwithstanding the orders which have been taken, and for any thing I know, do stand still in as much force in this church now as at any time heretofore, he complaineth much thatb the good orders which he doth mean have been withstood. Now it were hard, if as many as any wherec oppose unto these and the like orders, in his persuasion good, do thereby make themselves to be thought dislikers of the present state and proceedings. If they whom he aimeth at have any otherwise made themselves to be thought such, it is likely he doth know wherein, and will I hope disclose to whom itd appertaineth, both the persons whom he thinketh and the causes why he thinketh them so ill-affected. But whatsoever the men be, do their faults make me faulty? They do, if I join myself with them. I beseech him therefore to declare wherein I have joined with them. Other joining than this with any man here, I cannot imagine: it may be I have talked, or walked, or eaten, or interchangeably used the duties of common humanity, with some such as he is hardly persuaded of. For I know no law of God or man, by force whereof they should be as heathens and publicans unto me, that are not gracious in the eyes of another man, perhaps without cause, or if with cause, yet such cause as he is privy unto, and not I. Could he or any reasonable man think it a charitable course in me, to observe them that shew by external courtesies a favourable inclination towards him, and if I spy out any one amongst them of whom I think not well, hereupon to draw such an accusation as this against him, and to offer it where he hath given up his against me? which notwithstanding I will acknowledge to be just and reasonable, if he or any man living shall shew, that I use as much as the bare familiar company but of one, who by word or deed hath ever given me cause to suspect or conjecture him such as here they are termed, with whom complaint is made that I join myself. [576] This being spoken therefore and written without all possibility of proof, doth not Mr. Travers give me over-great cause to stand in some fear lest he make too little conscience how he useth his tongue or pen?ANSWER to TRAVERS. 7, 8. These things are not laid against me for nothing; they are to some purpose if they take place. For in a mind persuaded that I am as he deciphereth me, one which refuse to be at peace with such as embrace the truth, and side myself with men sinisterly affected thereunto, any thing that shall be spoken concerning the unsoundness of my doctrine cannot choose but be favourably entertained. This presupposed, it will have likelihood enough which afterwards followeth, that “many of my sermons have tasted of some sour leaven or other,” that in them he hath “discovered sundry unsound matters.” A thing greatlye to be lamented, that such a place as this, which might have been so well provided for, hath fallen into the hands of one no better instructed in the truth. But what if in the end it be found that he judgeth my words, as they do colours, which look upon them with green spectacles, and think that which they see is green, when indeed that is green whereby they see.

7. Touching the first point of hisf discovery, which is about the matter of predestination, to set down that I spake, (for I have it written,) to declare and confirm the several branches thereof, would be tedious now in this writing, where I have so many things to touch that I can but touch them only. Neither is it herein so needful for me to justify my speech, when the very place and presence where I spake, doth itself speak sufficiently for my clearing. This matter was not broached in a blind alley, or uttered where none was to hear it, that had skill with authority to control, or covertly insinuated by some gliding sentence.

8. That which I taught was at Paul’s Cross; it was not huddled in amongst other matters, in such sort that it could pass without noting; it was opened, it was proved, it was some reasonable time stood upon. I see not which way my Lord of London1, who was present and heard it, can excuse so great [577] a fault, as patiently, without rebuke or controlment afterwards, to hear any man there teach otherwise than “the word of God doth,” not as it is understood by the private interpretation of some one or two men, or by a special construction received in some few books, but as it is understood “by all theg churches professing the gospel;”ANSWER to TRAVERS 9, 10. by them all, and therefore even by our own also amongst others. A man that did mean to prove that he speaketh, would surely take the measure of his words shorter.

9. The next thing discovered, is an opinion about the assurance of men’s persuasionh in matters of faith. I have taught, he saith, “That the assurance of things which we believe by the word, is not so certain as of that we perceive by sense.” And is it as certain? Yea, I taught, as he himself I trust willi not deny, that the things which God doth promise in his word are surer unto us than any thing we touch, handle, or see; but are we so sure and certain of them? if we be, why doth God so often prove his promises unto us, as he doth, by arguments taken from our sensible experiencek? We must be surer of the proof than of the thing proved, otherwise it is no proof. How is it, that if ten men do all look upon the moon, every one of them knoweth it as certainly to be the moon as another; but many believing one and the same promises, all have not one and the same fulness of persuasion? How falleth it out, that men being assured of any thing by sense, can be no surer of it than they are; whereas the strongest in faith that liveth upon the earth, hath always need to labour, and strive, and pray, that his assurance concerning heavenly and spiritual things may grow, increase, and be augmented?

10. The sermon wherein I have spoken somewhat largely of this point, was, long before this late controversy rose between him and me, upon request of some of my friends seen and read by many, and amongst many, some who arel thought able to discern; and I never heard that any one of them hitherto hath condemned it as containing unsound matter. My case were very hard, if as oft as any thing I speak displeasethm [578] one man’s tasten my doctrine upon his only word should be taken for sour leaven.ANSWER to TRAVERS. 11, 12.

11. The rest of this discovery is all about the matter now in question, wherein he hath two faults predominant, which would tire out any that should answer unto every point severally: unapt speaking of school-controversies; and of my words sometimes so untoward a reciting, that he which should promise to draw a man’s countenance, and did indeed express the parts, at leastwise the most of them, truly, but perversely place them, could not represent a more offensive visage, than unto me mine own speech seemeth in some places, as he hath ordered it. For answer whereunto, that writing is sufficient, wherein I have set down both my words and meaning in such sort, that where this accusation doth deprave the one, and either misinterpret, or without just cause mislike the other, it will appear so plainly, that I may spare very well to take upon me a new and a needless labour here.

12. Only at one thing which is there to be found, because Mr. Travers doth here seem to take such a special advantage, as if the matter were unanswerable, he constraineth me either to detect his oversight, or to confess mine own in it. In settingo the question between the church of Rome and us about grace and justification, lest I should give them anp occasion to say, as commonly they do, that when we cannot refute their opinions, we propose to ourselves such instead of theirs, as we can refute; I took it for the best and most perspicuous way of teaching, to declare first, how far we do agree, and then to shew our disagreement; not generally (as Mr. Travers his words1 would carry it, for the easier fastening [579] ofq that upon me, wherewith, saving only by him, I was never in my life touched);ANSWER to TRAVERS. 13. but about the matter of justification onlyr; for farther I had no cause to meddle at that time. What was then mine offence in this case? I did, as he saith, so set it out as if we had consented in the greatest and weightiest points, and differed only in smaller matters. It will not be found, when it cometh to the balance, a light difference when we disagree, as I did acknowledge that we do, about the very essence of the medicine, whereby Christ cureth our disease. Did I go about to make a shew of agreement in the weightiest points, and was I so fond as not to conceal our disagreement about this? I do wish that some indifferency were used by them that have taken the weighing of my words.

13. Yea, but our agreement is not such in two of the chiefest points, as I would have men believe it is: and what are they? The one is, I said, “They acknowledge all men sinners, even the Blessed Virgin, though some of them free her from sin.” Put the case I had affirmed, that only some of them free her from sin, and had delivered it as the most current opinion amongst them, that she was conceived in sin: doth not Bonaventure1 say plainly, “omnes fere,” in a manner all men do hold this? doth he not bring many reasons wherefore all men should hold it? were their voices [580] since that time ever counted, and their number found smaller which hold it, than theirs that hold the contrary? Let the question then be, whether I might say, the most of them “acknowledges all men sinners, even the Blessed Virgin herself.” To shew that their general received opinion is the contrary, the Tridentine council is alleged, peradventure not altogether so considerately. For if that council have by resolute determination freed her, if it hold, as Mr. Travers saith it doth, that she was free from sin, then must the church of Rome needs condemn them that hold the contrary. For what that council holdeth, the same they all do and must hold. But in the church of Rome, who knoweth not, that it is a thing indifferent to think and defend the one or the other? So that thist argument, the council of Trent holdeth the Virgin free from sin, ergo, it is plain that none of them may, and therefore untrue that most of them do, acknowledge her a sinner, were forcibleu to overthrow my supposed assertion, if it were true that the council did hold this. But to the end it may clearly appear, how it neither holdeth this nor the contrary, I will open what manyx do conceive of the canon that concerneth this matter. The fathers of Trent perceived, that if they should define of this matter, it would be dangerous howsoever it were determined. If they freedy her from originalz sin, the reasons against them are unanswerable, which Bonaventure and others do allege, but especially Thomas1, whose line as much as may be they follow. Again if they did resolve the other way, they should control themselves in another thing, which in no case might be altered. For they profess to keep no day holy in the honour of an unholy thing; and the Virgin’s conception they honour with a feast2, which they could not abrogate without [581] cancelling a constitution of Xystus Quartus. And that which is worse, the world might perhaps hereupon suspect, that if the church of Rome did amiss before in this, it is not impossible for her to fail in other things. In the end, they did wisely cuta out their canon by a middle thread, establishing the feast of the Virgin’s conception, and leaving the other question doubtful as they found it; giving only a caveat, that no man should take the decree which pronouncethb all mankind originally sinful, for a definitive sentence concerning the Blessed Virgin. This in my sight is plain by their own words, “Declarat hæc ipsa sancta Synodus1,” &c. Wherefore our countrymen at Rhemes, mentioning this point, are marvellous wary, how they speak; they touch it as though it were a hot coal2: “Many godly devout men judge that our blessed lady was neither born nor conceived in sin.” It is not their wont to speak so nicely of things definitively set down in that council.

cIn like sort we find that the rest which have since thed time of the Tridentine synod written of original sin, are in this point for the most part either silent or very sparing in theire speech; and when they speak, either doubtful what [582] to think, or whatsoever they think themselves, fearful to set down any certain determination. If I be thought to take the canon of thatf council otherwise than they themselves do, let him expound it whose sentence was neither last asked nor his pen least occupied in setting it down; I mean Andradiusg, whom Gregory the Thirteenth hath allowed plainly to confess1, that it is a matter which neither express evidence of Scriptureh, nor the tradition of the Fathers, nor the sentence of the Church hath determined; that they are too surly and self-willed, which, defending either opinion, are displeased with them by whom the other is maintained; finally that the Fathers of Trent have not set down any certainty about this question, but left it doubtful and indifferent.

Now whereas my words, which I had set down in writing before I uttered them, were indeed these, “Although they imagine that the Mother of our Lord Jesus Christ were for his honour and by his special protection preserved clean from all sin, yet concerning the rest they teach as we do, that all have sinned:” against my words they might with more pretence take exception, because so many of them think she had sin, which exception notwithstanding, the proposition being indefinite and the matter contingent, they cannot take, because they grant that many whom they count grave and devout amongst them think that she was clear from all [583] sin.ANSWER to TRAVERS. 14. But whether Mr. Travers did note my words himself, or take them upon the credit of some other man’si noting, the tables were faulty wherein it was noted, “All men sinners, even the Blessed Virgin;” when my speech was rather, All men except the Blessed Virgin.”

To leave this; another fault he findeth, that I said, “They teach Christ’s righteousness to be the only meritorious cause of taking away sin, and differ from us only in the applying of it.” I did say and do, “They teach as we dok, that although Christ be the only meritorious cause of our justice, yet as a medicine, which is made for health, doth not heal by being made, but by being applied; so, by the merits of Christ, there can be no life norl justification, without the application of his merits: but about the manner of applying Christ, about the number and power of means whereby he is applied, we dissent from them.” This of our dissenting from them is acknowledged.

14. Our agreement in the former is denied to be such as I pretend. Let their own words therefore and mine concerning them be compared. Doth not Andradius plainly confess1; “Our sins dothm shut, and only the merits of Christ open the entering inton blessedness?” And Soto2, “It is put for a ground, that all, since the fall of Adam, obtain salvation only by the Passion of Christ: howbeit as no cause can be effectual without applying, so neither can any man be saved, to whom the suffering of Christ is not applied.” In a word, who not? when the council of Trent3 reckoning [584] up the causes of our first justification, doth name no end but God’s glory and our felicity; no efficient but his mercy; no instrumental but baptism; no meritorious but Christ; whom to have merited the taking away of no sin but original is not their opinion: which himself will find, when he hath well examined his witnesses, Catharinus1 and Thomas. Their Jesuits are marvellous angry with the men out of whose gleanings Mr. Travers seemeth to have taken this; they openly disclaim it, they say plainly, “Of all the catholics there is no one that did ever so teach,” they make solemn protestation, “We believe and profess that Christ upon the cross hath altogether satisfied for all sins, as well original as actual2.” Indeed they teach, that the merit of Christ doth not take away actual sin in such sort as it doth original; wherein if their doctrine had been understood, I for my speech had never been accused. As for the council of Trent concerning inherent righteousness, what doth it here? No [585] man doubteth but they make another formal cause of justification than we do.ANSWER to TRAVERS. 15, 16. In respect whereof, I have shewed already that we disagree about the very essence of that which cureth our spiritual disease. Most true it is which the grand philosopher hath, “Every man judgeth well of that whicho he knoweth1;” and therefore, till we know the things throughly whereof we judge, it is a point of judgment to stay our judgment.

15. Thus much labour being spent in discovering the unsoundness of my doctrine, some pains he taketh further to open faults in the manner of my teaching, as that “I bestowed my whole hour and more, my time and more than my time, in discourses utterly impertinent to my text.” Which if I had done, it might have past without complaining of to the privy-council.

16. But I did worse, as he saith; “I left the expounding of the Scriptures, and my ordinary calling, and discoursed upon school-points and questions, neither of edification, nor of truth.” I read no lecture in the law or in physic. And except the bounds of ordinary calling may be drawn like a purse, how are they so much wider unto him than to me, that he within the limits of his ordinary calling should reprove that in me which he understood not, and I labouring that both he and others might understand, could not do this without forsaking my calling? The matter whereof I spake was such, as being at the first by me but lightly touched, he had in that place openly contradictedp, and solemnly taken upon him to disprove. If therefore it were a school-question, and unfit to be discoursed ofq there, that which was in me but a proposition only at the first, wherefore made he a problem of it? Why took he first upon him to maintain the negative of that which I had affirmatively spoken, only to shew mine own opinion, little thinking that ever it would have mader a question? Of what nature soever the question were, I could do no less than there explain myself to them, unto whom I was accused of unsound doctrine; wherein if to shew what [586] had been through ambiguity mistaken in my words, or misapplied by him in this cause against me, I used the distinctions and helps of schools, I trust that herein I have committed no unlawful thing. These school-implements are acknowledged1 by grave and wise men not unprofitable to have been invented. The most approved for learning and judgment do use them without blame; the use of them hath been well liked in some that have taught even in this very place before me; the quality of my hearers is such, that I could not but think them of capacity very sufficient for the most part to conceive harders than I used any; the cause I had in hand did in my judgment necessarily require them which were then used; when my words spoken generally without distinctions had been perverted, what other way was there for me, but by distinctions to lay them open in their right meaning, that it might appear to all men whether they were consonant to truth or no? And although Mr. Travers be so inured with the city, that he thinketh it unmeet to use any speech which savoureth of the school, yet his opinion is no canon. Though unto him, his mind being troubled, my speech did seem like fetters and manacles, yet there might be some more calmly affected which thought otherwise; his private judgment will hardly warrant his bold words, that the things which I spaket “were neither of edification nor truth.” They might edify some other, for any thing he knoweth, and be true for any thing he proveth to the contrary. For it is no proof to cry, “Absurdities, the like whereunto have not been heard in public places within this land since Queen Mary’s days.” If this cameu in earnest from him, I am sorry to see him so much offended without cause; more sorry, that his fit should be so extreme, to make him speak he knoweth not what. That I neither “affected the truth of God, nor the peace of the Church,” mihi pro minimo est. It doth not much move me when Mr. Travers doth say that, which I trust a greater than Mr. Travers will gainsay.

[587]
ANSWER to TRAVERS. 17.17. Now let all this which hitherto he hath said be granted him, let it be as he would have it, let my doctrine and manner of teaching be as much disallowed by all men’s judgmentsx as by his, what is all this to his purpose? He himselfy allegeth this to be the cause why he bringeth it in; the High Commissioners “charge him with an indiscretion and want of duty in that he inveighed against certain points of doctrine taught by me as erroneous, not conferring first with me, nor complaining of it to them.” Which faults, a sea of such matter as he hath hitherto waded in will never be able to scour from him. For the avoiding of schism and disturbance in the Church, which must needs grow if all men might think what they list and speak openly what they think; therefore by a decree1 agreed upon by the Bishops and confirmed by her Majesty’s authority2, it was ordered that erroneous doctrine, if it were taught publickly, should not be [588] publickly refuted;ANSWER to TRAVERS. 18. but that notice thereof should be given unto such as are by her Highness appointed to hear and to determine such causes. For breach of which order, when he is charged with lack of duty, all the faults that can be heaped upon me will make but a weak defence for him: as surely his defence is not much stronger, when he allegeth for himself, that “he was in some hope his speech in provingz the truth, and clearing those scruples which I had in myself, might cause me either to embrace sound doctrine, or suffer it to be embraced of others, which if I did he should not need to complain;” that “it was meet he should first discovera what I had sown, and make it manifest to be tares, and then desire their scythe to cut it down;” that conscience did bind him to do otherwise than the foresaid order requireth;” that “he was unwillingb to deal in that public manner, and wished a more convenient way were taken for it;” that “he had resolved to have protested the next sabbath-day, that he would some other way satisfy such as should require it, and not deal more in that place.” Be it imagined, (let me not be taken as if I did compare the offenders, when I do not, but their answers only,) be it imagined that a libeller did make this apology for himself; “I am not ignorant that if I have just matter against any man the law is open, there are judges to hear it, and courts where it ought to be complained of; I have taken another course against such or such a man, yet without breach of duty, forasmuch as I am able to yield a reason of my doing; I conceivedc some hope that a little discredit amongst men would make him ashamed of himself, and that his shame wouldd work his amendment; which if it did, other accusation there should not need:” could his answer be thought sufficient, could it in the judgment of discreet men free him from all blame? No more can the hope which Mr. Travers conceivede to reclaim me by public speech, justify his faultf against the established order of the church.

18. His thinking it meet “he should first openly discover to the people the tares that had been sown amongst them, and then require the hand of authority to mow them down,” [589] doth only make it a question whether his opinion that this was meet, may be a privilege or protection against that lawful constitution which had before determined of it as of a thing unmeet.ANSWER to TRAVERS. 19, 20. Which question I leave for them to discuss whom it most concerneth. If the order be such that it cannot be kept without hazarding a thing so precious as a good conscience, the peril whereof could be no greater to him than it needs must be to all others whom it toucheth in like causesg; when this is evident, it will be a most effectual motive not only for England, but also for other reformed churches, even Geneva itself, (for they have the like,) to change or take that away which cannot but with great inconvenience be observed. In the meanwhile, the breach of it may in such consideration be pardoned, (which truly I wish, howsoever it be), yet hardly defended as long as it standeth in force uncancelled.

19. Now whereas he confesseth another way had “been more convenient,” and that he found in himself secreth unwillingness to do that which he did, doth he not plainly say in effect that the light of his own understanding proved the way he took perverse and crooked; reason was so plain and pregnant against it, that his mind was alienated, his will averted to another course? yet somewhat there was which so far overruled, that it must needs be done even against the very stream: what doth thisi bewray? Finally, his purposed protestation, whereby he meant openly to make it known, that he did not allow this kind of proceeding, and therefore would satisfy men otherwise, “and deal no more in this place,” sheweth his good mind in this, that he meant to stay himself from further offending; but it servethk not his turn. He is blamed because the thing he had done was amiss, and his answer is, That which I would have done afterward had been well, if so be I had done it.

20. But as in this he standeth persuaded that he hath done nothing besides duty, so he taketh it hardly that the High Commissioners should charge him with indiscretion. Whereof1 as if he could so wash his hands, he maketh a long and a large declaration concerning the carriage of himself; how he waded in matters “of smaller weight,” and how in things of [590] greater “moment;”ANSWER to TRAVERS. 20. how warily he dealt; how “naturally he took hism things rising from the text;” how closely he kept himself “to the Scripture he took in hand;” how much pains he “took to confirm the necessity of believing justification by Christ only,” and to shew how “the church of Rome denieth that a man is saved by faith alone without works of the law;” what “the Sons of Thunder would have done” if they had been in his case; that his “answer was very temperate, without immodest or reproachful speech;” that when he might “before all have reproved me,” he did not, “but contented himself with exhorting me” before all “to follow Nathan’s example and revisit my doctrine;” when he might have followed St. Paul’s example in “reproving” Peter, he did not, but exhorted me with Peter to “endure to be withstood.” This testimony of his discreet carrying himself in the handling of his matter, being more agreeably framed and given him by another than byn himself, might make somewhat for the praise of his person; but for defence of his action unto them by whom he is thought undiscreet for not conferring privately before he spake, will it serve to answer that when he spake he did it considerately? He perceiveth it will not, and therefore addeth reasons such as they are. As namely how he purposed at the first to take another course, and that was this, “publicly to deliver the truth of such doctrine as I had otherwise taught, and at convenient opportunity to confer with me upon such points.” Is this the rule of Christ, If thy brother offend openly in his speecho, control it first with contrary speech openly, and confer with him afterwards upon it, when convenient opportunity serveth? Is there any law of God or of man whereuponp to ground such a resolution, any Church extant in the world where teachers are allowed thus to do or to be done unto? He cannot but see how weak an allegation it is, when he bringeth in his following this course, first in one matter and so afterwards in another, to approve himself now following it again. For if the very purpose of doingq a thing so uncharitabler be a fault, the deed is a greater fault; and doth the doing of it twice make it the third time fit and allowable [591] to be done? The weight of the cause, which is his third defence, relieveth him as little.ANSWER to TRAVERS. 21. The weightier it was the more it required conference, advices, and consultation, the more it stood him upon to take goodt heed that nothing were rashly done or spoken in it. But he meaneth “weighty” in regard of the wonderful danger, except he had presently withstood me, without expecting a time of conference. “This cause being of such moment that might prejudice the faith of Christ, encourage the ill-affected to continue still in their damnable ways, and other weak in faith to suffer themselves to be seduced to the destruction of their souls, he thought it his bounden duty to speak before he talked with me.” A man that should read this and not know what I had spoken might imagine that I had at the least denied the divinity of Christ. But they which were present at my speech, and can testify that nothing passed my lips more than is contained in their writings, whom for soundness of doctrine, learning, and judgment, Mr. Travers himself doth, I dare say, not only allow, but honour; they which heard and do know, that the doctrine here signified in so fearful manner, the doctrine that was so dangerous to the faith of Christ, that was so likely to “encourage ill-affected men to continue still in damnable ways,” that gave so great cause to tremble for fear of the present “destruction of souls,” was only this; “I doubt not but God was merciful to save thousands of our fathers living heretofore in popish superstitions, inasmuch as they sinned ignorantly;” and this spoken in a sermon, the greatest part whereof was against popery; they will hardly be able to discern how Christianity should herewith be so grievously shaken.

21. Whereby his fourth excuse is also taken from him. For what doth it boot him to say, “The time was short wherein he was to preach after me,” when his preaching of this matter perhaps ought, surely might have been either very well omitted, or at the least more conveniently for a while deferred, even by their judgments that cast the most favourable aspect towards these his hasty proceedings. The poison which men had taken at my hands was not so quick and strong in operation as in eight days to make them past cure; by eight days’ delay there was no likelihood that the force and [592] power of his speech could die;ANSWER to TRAVERS. 22. longer meditation might bring better and stronger proofs to mind than extemporal dexterity could furnish him with; and who doth know whether time, the only mother of sound judgment and discreet dealing, might have given that action of his some better ripeness, which by so great festination hath, as a thing born out of time, brought small joy unto him that begat it? Doth he think it had not been better that neither my speech had seemed in his eyes as an arrow sticking in a thigh of flesh1, nor his own as a child whereof he must needs be delivered by an hour? His last way of disburdening himself is, by casting his load upon my back, as if I had brought him by former conferences out of hope that any fruit would ever come of conferring with me. Loth I am to rip up those conferences, whereof he maketh but a slippery and loose relation. In one of them the question between us was, whether the persuasion of faith concerning remission of sins, eternal life, and whatsoever God doth promise unto man, be as free from doubting as the persuasion which we have by sense concerning things tasted, felt, and seen. For the negative I mentioned their example, whose faith in Scripture is most commended, and the experience, which all faithful men have continually had of themselves. For proof of the affirmative which he held I desiringu to havex some reason, heardy nothing but “all good writers” oftentimes inculcated. At the length, upon request to see some one of them, Peter Martyr’s Common Places were brought, where the leaves were turned down at a place sounding to this effect, “That the Gospel doth make truez Christians more virtuous than moral philosophy dida make heathens2:” which came not near the question by many miles.

22. In the other conference he questioned about the matter of reprobation, misliking first that I had termed God a permissive, and no positive cause of the evil, which the schoolmen do call malum culpæ; secondly that to their objection who say, “If I be elected, do what I will, I shall be saved,” I had answered, that the will of God in this thing is not absolute but conditional, to save his elect believing, fearing, and obediently [593] serving him;ANSWER to TRAVERS. 23. thirdly that to stop the mouths of such as grudge and repine against God for rejecting castaways, I had taught that they are not rejected no not in the purpose and counsel of God, without a foreseen worthiness of rejection going though not in time yet in order before. For if God’s electing do in order (as needs it must) presuppose the foresight of their being that are elected, though they be elected before they be; nor only the positive foresight of their being, but also the permissive of their being miserable, because election is through mercy, and mercy doth alwaysb presuppose misery: it followeth, that the very chosen of God acknowledge to the praise of the riches of his exceeding free compassion, that when he in his secret determination setc it down, “Those shall live and not die,” they lay as ugly spectacles before him, as lepers covered with dung and mire, as ulcers putrefied in their fathers’ loins, miserable, worthy to be had in detestation; and shall any forsaken creature be able to say unto God, Thou didst plunge me into the depthd and assign me unto endless torments only to satisfy thine own will, finding nothing in me for which I could seem in thy sight so well worthy to feel everlasting flames?

23. When I saw thate Mr. Travers carped at these things, only because they lay not open, I promised at some convenient time to make them clear as light both to him and tof all others1. Which if they that reprove me will not grant me leave to do, they must think that they are for some cause or other more desirous to have me reputed an unsound man, than willing that my sincere meaning should appear and be approved. When I was farther asked what my grounds were, I answered that St. Paul’s words concerning this cause were my grounds. His next demand, what authorg I did follow in expounding St. Paul and gathering the doctrine out of his words, against the judgment, he saith, “of all churches and all good writers.” I was well assured that to control this overreaching speech, the sentences which I might have cited out of Church Confessions, together with the best learned monuments of former times, and not the meanest of our own, [594] were mo in number than perhaps he would willingly have heard of;ANSWER to TRAVERS. 24. but what had this booted me? For although he himself in generality do much use those formal speeches, “all churches,” and “all good writers:” yet as he holdeth it in the pulpit lawful to say in general, the Painims think this, or the Heathenh that, but utterly unlawful to cite any sentencei of theirs that say it; so he gave me at that time great cause to think, that my particular alleging of other men’s words to shew their agreement with mine, would as much have displeased his mind, as the thing itself for which theyk had been alleged. For he knoweth how often he hath in public place bitten me for this, although I did never in any sermon use many of the sentences of other writers, and do makel most without any; having always thought it meetest neither to affect nor to contemn the use of them.

24. He is not ignorant, that in the very entrance to the talk which we had privately at that time, to prove it unlawful altogether in preaching, either for confirmation, declaration, or otherwise, to cite any thing but mere canonical scripture, he brought in, “The Scripture is given by inspiration, and is profitable to teach, tom improve,” &c. urging much the vigour of these two clauses, “the man of God,” and “every good work.” If therefore the work were good which he required at my hands, if privately to shew why I thought the doctrine I had delivered to be according to St. Paul’s meaning were a good work, can they which take the place before alleged for a law condemning every man of God who in doing the work of preaching any way useth human authority, like it in me, if in the work of strengthening that which I had preached, I should bring forth the testimonies and the sayings of mortal men? I alleged therefore that which might under no pretence in the world be disallowed, namely reasonn; not meaning thereby mineo own reason as now it is reported, but true, sound, divine reason; reason whereby those conclusions might be out of St. Paul demonstrated, and not probably discoursed of only; reason proper to that science whereby the things of God are known; theological reason, which out ofp principles in Scripture that are plain, soundly deduceth [595] more doubtful inferences, in such sort that being heard they neither canq be denied, nor any thing repugnant unto them received,ANSWER to TRAVERS. 25. but whatsoever was before otherwise by miscollecting gathered out of darkerr places, is thereby forced to yield itself, and the true consonant meaning of sentences not understood is brought to light. This is the reason which I intended. If it were possible for me to escape the ferula in any thing I do or speak, I had undoubtedly escaped it in this. In this I did that which by some is enjoined as the only allowable, but granted by all as the most sure and safe way whereby to resolve things doubted of, in matters appertaining to faith and Christian religion. So that Mr. Travers had here small cause given him to be weary of conferring, unless it were in other respects than that poor one which is here pretended, that is to say, the little hope he had of doing me any good by conference.

25. Yet behold his first reason of not complaining to the High Commission is, that sith I offended only through an overcharitable inclination, he conceived good hope, when I should see the truth cleared and some scruples which were in my mind removed by his diligence, I would yield. But what experience soever he had of former conferences, how small soever his hope was that fruit would come of it if he should have conferred, will any man judge this a cause sufficient why to open his mouth in public without any one word privately spoken? He might have considered that men do sometimes reap where they sow but with small hope; he might have considered that although unto me (whereof he was not certain neither) but if to me his labour should be as water spilt or poured into a torn dish, yet to him it could not be fruitless to do that which order in Christian churches, that which charity among Christian men, that which at any man’s handss even common humanity itself, at his many other things besides did require. What fruit could there come of hist open contradicting in so great haste with so small advice, but such as must needs be unpleasant and mingled with much acerbity? Surely he which will take upon him to defend that in this there was no oversight, must beware lest by such defences he leave an opinion dwelling in the minds of men that he is more [596] stiff to maintain what he hath done, than careful to do nothing but that which may justly be maintained.ANSWER to TRAVERS. 26.

26. Thus have I, as near as I could, seriously answered things of weight: with smaller I have dealt as I thought their quality did require. I take no joy in striving, I have not been nuzzledu or trained up in it. I would to Christ they which have at this present enforced me hereunto, had so ruled their hands in any reasonable time, that I might never have been constrained to strike sox much as in mine own defence. Wherefore to prosecute this long and tedious contention no further, shall I wish that your Grace and their Honours (unto whose intelligence the dutiful regard which I have of their judgments maketh me desirous that as accusations have been brought against me, so thisy my answer thereunto may likewise come) did both with the one and the other, as Constantine with thez books containing querulous matter1. Whether this be convenient to be wished or no, I cannot tell. But sith there can come nothing of contention but the mutual waste of the parties contending, till a common enemy dance in the ashes of them both, I do wish heartily that the grave advice which Constantine gave for reuniting of his clergy, so many times upon so small occasions in so lamentable sort divided, or rather the strict commandment of Christ unto his that they should not be divided at all, may at lengtha if it be his blessed will, prevail so far at the least in this corner of the Christian world, to the burying and quite forgetting of strife, together with the causes which have either bred it or brought it up; that things of small moment never disjoin them, whom one God, one Lord, one Faith, one Spirit, one Baptism, bands of greatb force, have linked; that a respective eye towards things wherewith we should not be disquieted make us not, as through infirmity the very patriarchs themselves sometimes were, full gorged, unable to speak peaceably to their own brother; finally that no strife may ever be heard of again but this, who shall hate strife most, who shall pursue peace and unity with swiftest paces.

[597]
A LEARNED SERMON OF THE NATURE OF PRIDE1.
Habak.a ii. 4.
His mind swelleth, and is not right in him: but the just by his faith shall live.

SERM. III.THE nature of man, being much more delighted to be led than drawn, doth many times stubbornly resist authority, when to persuasion it easily yieldeth. Whereupon the wisest law-makers have endeavoured always, that those laws might seem most reasonable, which they would have most inviolably kept. A law simply commanding or forbidding, is but dead in comparison of that which expresseth the reason wherefore it doth the one or the other. And, surely, even in the laws of God, although that he hath given commandment be in itself a reason sufficient to exact all obedience at the hands of men, yet a forcible inducement it is to obey with greater alacrity and cheerfulness of mind, when we see plainly that nothing is imposed more than we must needs yield unto, except we will be unreasonable. In a word, whatsoever we be taught, be it precept for direction of our manners, or article for instruction of our faith, or document any way for information of our minds, it then taketh root and abideth, when we conceive not only what God doth speak, but why. Neither is it a small thing which we derogate, as well from the honour of his truth, as from the comfort, joy, and delight which we ourselves should take by it, when we loosely slide over his speech as though it were, as our own is commonly, vulgar and trivialb. Whereas he uttereth nothing but it hath, besides the substance of doctrine delivered, a depth of wisdom in the very choice and frame of words to deliver it in. The reason whereof being [598] not perceived, but by greater intention of brain than our nice minds for the most part can well away with, fain we would bring the world, if we might, to think it but a needless curiosity to rip up any thing further than extemporal readiness of wit doth serve to reach unto. Which course if here we did list to follow, we might tell you, that in the first branch of this sentence God doth condemn the Babylonian’s pride; and in the second, teach what happiness ofc state shall grow to the righteous by the constancy of their faith, notwithstanding the troubles which now they suffer; and, after certain notes of wholesome instruction hereupon collected, pass over without detaining your minds in any further removed speculation. But, as I take it, there is a difference between the talk that beseemeth nursesd amongst children, and that which men of capacity and judgment do or should receive instruction by.

The mind of the Prophet being erected with that which hath been hitherto spoken, receiveth here for full satisfaction a short abridgment of that which is afterwards more particularly unfolded. Wherefore, as the question before disputed of doth concern two sorts of men, the wicked flourishing as the bay, and the righteous like the withered grass, the one full of pride, the other cast down with utter discouragement; so the answer which God doth make for resolution of doubts hereupon arisen, hath reference unto both sorts, and this present sentence, containing a brief abstract thereof, comprehendeth summarily as well the fearful estate of iniquity over-exalted, as the hope laid up for righteousness opprest. In the former branch of which sentence, let us first examine what this rectitude or straightness importeth, which God denieth to be in the mind of the Babylonian. All things which God did create, he made them at the first true, good, and right: true, in respect of correspondence unto that pattern of their being, which was eternally drawn in the counsel of God’s foreknowledge; good, in regard of the use and benefit which each thing yieldeth unto other; right, by an apt conformity of all parts with that end which is outwardly proposed for each thing to tend unto. Other things have ends proposed, but have not the faculty to know, judge, and esteem of them; and therefore as they tend thereunto unwittingly, so likewise in the means whereby they [599] acquire their appointed ends, they are by necessity so held that they cannot divert from them. The ende why the heavens do move, the heavens themselves know not, and their motions they cannot but continue. Only men in all their actions know what it is which they seek for, neither are they by any such necessity tied naturally unto any certain determinate mean to obtain their end by, but that they may, if they will, forsake it. And therefore, in the whole world, no creature but only man, which hath the last end of his actions proposed as a recompense and reward, whereunto his mind directly bending itself, is termed right or straight, otherwise perverse.

To make this somewhat more plain, we must note, that as they, which travel from city to city, inquire ever for the straightest way, because the straightest is that which soonest bringeth them unto their journey’s end; so we, “having here,” as the Apostle speaketh1, “no abiding city,” but being always in travel towards that place of joy, immortality, and rest, cannot but in every of our deeds, words, and thoughts, think that to be best, which with most expedition leadeth us thereunto, and is for that very cause termed right. That sovereign good, which is the eternal fruition of all good, being our last and chiefest felicity, there is no desperate despiser of God and godliness living which doth not wish for. The difference between right and crooked minds, is in the means which the one or the other do eschew or follow. Certain it is, that all particular things which are naturally desired in the world, as food, raiment, honour, wealth, pleasure, knowledge, they are subordinated in such wise unto that future good which we look for in the world to come, that even in them there lieth a direct way tending unto this. Otherwise we must think, that God, making promises of good things in this life, did seek to pervert men and to lead them from their right minds. Where is then the obliquity of the mind of man? His mind is perverse, kamf2, and crooked, not when it bendeth itself unto any of these things, but when it bendeth so, that it swerveth either to the right hand or tog the left, by excess or defect, from that exact rule whereby human actions [600] are measured. The rule to measure and judge them by, is the law of God. For this cause, the Prophet doth make so often and so earnest suit, “O direct me in the way of thy commandments”: as long as I have respect to thy statutes, I am sure not to tread amiss. Under the name of the Law, we must comprehend not only that which God hath written in tables and leaves, but that which nature hathh engraven in the hearts of men. Else how shouldi those heathenk, which never had books but heaven and earth to look upon, be convicted of perverseness? “But the Gentiles, which had not the law in books, had,” saith the Apostle1, “the effect of the law written in their hearts.”

Then seeing that the heart of man is not right exactly, unless it be found in all parts such, that God examining and calling it unto account with all severity of rigour, be not able once to charge it with declining or swerving aside (which absolute perfection when did God ever find in the sons of mere mortal men?) doth it not follow, that all flesh must of necessity fall down and confess, We are not dust and ashes, but worse; our minds from the highest to the lowest are not right; if not right, then undoubtedly not capable of that blessedness which we naturally seek, but subject unto that which we most abhor, anguish, tribulation, death, woe, endless misery. For whatsoever misseth the way of life, the issue thereof cannot be but perdition. By which reason, all being wrapped up in sin, and made thereby the children of death, the minds of all men being plainly convicted not to be right; shall we think that God hath endued them with so many excellencies, moel not only than any, but than all the creatures in the world besides, to leave them inm such estate, that they had been happier if they had never been? Here cometh necessarily in a new way unto salvation, so that they which were in the other perverse, may in this be found straight and righteous. That the way of nature, this the way of grace. The end of that way, salvation merited, presupposing the righteousness of men’s works; their righteousness, a natural abilityn to do them; that abilityn, the goodness of God which created them in such perfection. But [601] the end of this way, salvation bestowed upon men as a gift, presupposing, not their righteousness, but the forgiveness of their unrighteousness, justification; their justification, not their natural abilityo to do good, but their hearty sorrow for notp doing, and unfeigned belief in Him, for whose sake not doers are accepted, which is their vocation; their vocation, the election of God, taking them out from the number of lost children; their election, a mediator in whom to be elect; this mediation, inexplicable mercy; his mercy, their misery, for whom he vouchsafed to make himself a mediator. The want of exact distinguishing between these two ways, and observing what they have common, what peculiar, hath been the cause of the greatest part of that confusion whereof Christianity at this day laboureth. The lack of diligence in searching, laying down, and inuring men’s minds with those hidden grounds of reason, whereupon the least particulars in each of these are most firmly and strongly builded, is the only reason of all those scruples and uncertainties, wherewith we are in such sort entangled, that a number despair of ever discerning what is right or wrong in any thing. But we will let this matter rest, whereinto we stepped to search out a way, how some minds may be and are right truly even in the sight of God, though they be simply in themselves not right.

Howbeit, there is not only this difference between the just and impious, that the mind of the one is right in the sight of God, because his obliquity is notq imputed; the other perverse, because his sin is unrepented of: but even as lines that are drawn with a trembling hand, but yet to the point which they should, are thoughr ragged and uneven, nevertheless direct in comparison of them which run clean another way; so there is no incongruity in terming them right-minded men, whom though God may charge with many things amiss, yet they are not as those dismals and uglyt monsters, in whom, because there is nothing but wilful opposition of mind against God, a more than tolerable deformity is noted in them, by saying, that their minds are not right. The angel of the church of Thyatira, unto whom the Son of God sendeth this greeting, “I know thy works, and thy love, and service, [602] and faith; notwithstanding, I have a few things against thee1,” was not as he unto whom St. Peter, “Thou hast no fellowship in this business; for thy heart is not right in the sight of God2.” So that whereas the orderly disposition of the mind of man should be this; perturbations and sensual appetites all kept in awe by a moderate and sober will; will in all things framed by reason; reason directed by the law of God and nature; this Babylonian had his mind, as it were, turned upside down. In him unreasonable cecity and blindness trampled all laws, both of God and nature, under feet; wilfulness tyrannized over reason, and brutish sensuality over will: an evident token that his outrage would work his overthrow, and procure his speedy ruin. The mother whereof was that which the Prophet in these words signifieth, “His mind doth swell.”

Immoderate swelling, a token of very imminentu breach, and of inevitable destruction: pride, a vice which cleaveth so fast unto the hearts of men, that if we were to strip ourselves of all faults one by one, we should undoubtedly find it the very last and hardest to put off. But I am not here to touch thatx secret itching humour of vanity, wherewith men are generally touched. It was a thing more than meanly inordinate, wherewith the Babylonian did swell. Which that we may both the better conceive, and the more easily reap profit by,y the nature of this vice, which setteth the whole world out of course, and hath put so many even of the wisest besides themselves, is first of all to be inquired into: secondly, the dangers to be discovered which it draweth inevitablyz after it, being not cured: and, last of all, the waya to cure it.

Whether we look upon the gifts of nature or of grace, or whatsoever is in the world admired as a part of man’s excellency, adorning his body, beautifying his mind, or externally any way commending him in the account and opinion of men, there is in every kind somewhat possible which no man hath, and somewhat had which fewb can attain unto. By occasion whereof there groweth disparagement necessarily; and by occasion of disparagement, pride through men’s ignorance. First, therefore, although men be not proud of any [603] thing which is not at the least in opinion good; yet every good thing they are not proud of, but only of that which neither is common unto many, and being desired of all causeth them which have it to be honoured above the rest. Now there is no man so void of brain, as to suppose that pride consisteth in the bare possession of such things; for then to have virtue were a vice, and they should be the happiest men who are wretchedestc, because they have least of that which they would have. And though in speech we do intimate a kind of vanity to be in them of whom we say, “They are wise men and they know it;” yet this doth not prove, that every wise man is proud which doth not think himself to be blockish. What we may have, and know that we have it without offence, do we then make offensive when we take joy and delight in having it? What difference between men enriched with all abundance of earthly andd heavenly blessings, and idols gorgeously attired, but this, “The one takee pleasure in that which they have, the other none?” If we may be possessed with beauty, strength, riches, power, knowledge, if we may be privy what we are every way, if glad and joyful for our own welfare, and in all this remain unblameable; nevertheless, some there are, who, granting thus much, doubt whether it may stand with humility, to acceptf those testimonies of praise and commendation, those titles, rooms, and other honours, which the world yieldeth, as acknowledgments of some men’s excellencyg above others. For, inasmuch as Christ hath said unto those that are his, “The kings of the Gentiles reign over them, and they that bear rule over them, are called gracious lords; be yeh not so1;” the Anabaptist hereupon urgeth equality among Christians, as if all exercise of authority were nothing else but heathenish pride. Our Lord and Saviour had no such meaning. But his disciples feeding themselves with a vain imagination for the time, that the Messias of the world should in Jerusalem erect his throne, and exercise dominion with great pomp and outward stateliness, advanced in honour and terrene power above all the princes of the earth, began to think how with their Lord’s condition their own would also [604] rise; that having left and forsaken all to follow him, their place about him should not be mean; and because they were many, it troubled them much, which of them should be the greatest man. When suit was made for two by name, that of them “one might sit at his right hand, and the other at his left1,” the rest began to stomach, each taking it grievously that any should have what all did affecti: their Lord and Master, to correct this humour, turneth aside their cogitations from these vain and fanciful conceitsk, giving them plainly to understand, that they did but deceive themselves; his coming was not to purchase an earthly, but to bestow an heavenly kingdom, wherein they, if any, shall be greatest, whom unfeigned humility maketh in this world lowest, and least amongst others: “Ye are they which have continued with me in my temptations, therefore I leave unto you a kingdom, as my Father hath appointed me, that ye may eat and drink at my table in my kingdom, and sit on seats, and judge the twelve tribes of Israel2.” But my kingdom nol such kingdom as ye dream of: and therefore these hungry ambitious contentions seemlierm in heathens than in you. Wherefore from Christ’s intent and purpose nothing further removed than dislike of distinction in titles and callingn, annexed for order’s sake unto authority, whether it be ecclesiastical or civil. And when we have examined throughly what the nature of this vice is, no man knowing it can be so simple, as not to see an ugliero shape thereof apparent many times in rejecting honours offered, than in the very exacting of them at the hands of men. For, as Judas his care for the poor was mere covetousness; and that frank-hearted wastefulness spoken of in the gospel, thrift; so there is no doubt but that going in rags may be pride, and thrones be challenged with unfeigned humility.

We must go further, therefore, and enter somewhat deeper, before we can come to the closet wherein this poison lieth. There is in the heart of every proud man, first, an error of understanding, a vain opinion whereby he thinketh his own excellency, and by reason thereof his worthiness of estimation, regard, and honour, to be greater than in truth it is. This [605] maketh him in all his affections accordingly to raise up himself; and by his inward affections his outward acts are fashioned. Which if you list to have exemplifiedp, you may, either by calling to mind things spoken of them whom God himself hath in Scripture especiallyq noted with this fault; or by presenting to your secret cogitations that which you daily behold in the odious lives and manners of high-minded men. It were too long to gather together so plentiful an harvest of examples in this kind as the sacred Scripture affordeth. That which we drink in at our ears doth not so piercinglyr enter, as that which the mind doth conceive by sight. Is there any thing written concerning the Assyrian monarch in the tenth of Esay, of his swelling mind, his haughty looks, his great and presumptuous vaunts; “By the power of mine own hand I have done all things, and by mine own wisdom I have subdued the world1;” any thing concerning the dames of Sion, in the third of the prophet Esay, of their stretched-out necks, their immodest eyes, their pageant-like, stately and pompous gait; any thing concerning the practices of Cores, Dathan, and Abiron, of their impatience to live in subjection, their mutinous repiningt at lawful authority, their grudging against their superiors, ecclesiastical and civil; any thing concerning pride in any sort or sect, which the present face of the world doth not, as au glass, represent to the view of all men’s beholding? So that if books, both profane and holy, were all lost, as long as the manners of men retain the estatex they are in; for him which observeth, how after thaty men have once conceived an over-weening of themselves, it maketh them in all their affections to swell; how deadly their hatred, how heavy their displeasure, how unappeasable their indignation and wrath is above other men’s, in what manner they compose themselves to be as Heteroclites, without the compass of all such rules as thez common sort are measured by; how the oaths which religious hearts do tremble at, they affect as principal graces of speech; what felicity they take to see the enormity of their crimes above the reach of laws and [606] punishments; how much it delighteth them when they are able to appala with the cloudiness of their look; how far they exceed the terms wherewith man’s nature should be limited; how high they bear their heads over others; how they browbeat all men which do not receive their sentences as oracles, with marvellous applause and approbation; how they look upon no man but with an indirect countenance, nor hear any thing, saving their own praisesb with patience, nor speak without scornfulness and disdain; how they use their servants as if they were beasts, their inferiors as servants, their equals as inferiors, and as for superiors, acknowledge none; how they admire themselves as venerable, puissant, wise, circumspect, provident, every way great, taking all men besides themselves for ciphers, poor inglorious silly creatures, needless burthens of the earth, off-scourings, nothing: in a word, for him which marketh how irregular and exorbitant they are in all things, it can be no hard thing hereby to gather, that pride is nothing but an inordinate elation of the mind, proceeding from a false conceit of men’s excellency in things honoured, which accordingly frameth also their deeds and behaviour, unless there be cunning to conceal it. For a foul scar may be covered with a fair cloth. And as proud as Lucifer may be in outward appearance lowly.

No man expecteth grapes of thistles; nor from a thing of so bad a nature can other than suitable fruits be looked for. What harm soever in private families there groweth by disobedience of children, stubbornness of servants, untractableness in them, who, although they otherwise may rule, yet should in consideration of the imparity of their sex be also subject; whatsoever, by strifec amongst men combined in the fellowship of greater societies, by tyranny of potentates, ambition of nobles, rebellion of subjects in civil states; by heresies, schisms, divisions in the Church; naming pride, we name the mother which brought them forth, and the only nurse that feedeth them. Give me the hearts of all men humbled; and what is there that can overthrow or disturb the peace of the world? wherein many things are caused of much evil; but pride of all.

To declaim of the swarms of evils issuing out of pride, is an easy labour. I rather wish that I could exactly prescribe [607] and persuade effectually the remedies, whereby a sore so grievous might be cured thee means how the pride of swelling minds might be taken down. Whereunto so much we have already gained, that the evidence of the cause which breedeth it, pointeth directly unto the likeliest and fittest helpf to take it away. Diseases that come of fulness, emptiness must remove. Pride is not cured but by abating the error which causeth the mind to swell. Then seeing that they swell by misconceit of their own excellency: for this cause, all which tendethg to the beating down of their pride, whether it be advertisement from men, or from God himself chastisement, it then maketh them cease to be proud, when it causeth them to see their error in overseeing the thing they were proud of. At this mark Job, in his apology unto his eloquent friends, aimeth. For perceiving how much they delighted to hear themselves talk, as if they had given their poor afflicted familiar a schooling of marvellous deep and rare instruction, as if they had taught him more than all the world besides could acquaint him with; his answer was to this effect: Ye swell as though ye had conceived some great matter; but as for that which ye are delivered of, who knoweth it not? Is any man ignorant of these things? At the same mark the blessed apostle driveth1: “Ye abound in all things, ye are rich, ye reign, and would to Christ we did reign with you:” but boast not: for what have ye, or are ye of yourselves? To this mark all those humble confessions are referred, which have been always frequent in the mouths of saints, truly wading in the trial of themselves; as that of the prophet2: “We are nothing but soreness, and festered corruption;” our very light is darkness, and our righteousness itself unrighteousnessh: that of Gregory, “Let no man ever put confidence in his own deserts; sordet in conspectu Judicis, quod fulget in conspectu operantis3: in the sight of thati dreadful Judge, it is noisome, which in the doer’s judgment maketh a beautiful show:” that of [608] Anselm, “I adore thee, I bless thee, Lord God of heaven and Redeemer of the world, with all the power, abilityk, and strength of my heart and soul, for thy goodness so unmeasurably extended; not in regard of my merits, whereunto only torments were due, but of thy mere unprocured benignity.” If these Fathers should be raised again from the dust, and have the books laid open before them, wherein such sentences are found as this: “Works, no other than the value, desert, price, and worth of the joys of the kingdom of heaven; heaven, in relation to our works, as the very stipend, which the hired labourer covenanteth to have of him whose workl he doth, a thing equally and justly answering unto the time and weight of his travails, rather than a voluntarym or bountiful gift1”—if, I say, those reverend fore-rehearsed Fathers, whose books are so full of sentences witnessing their Christian humility, should be raised from the dead, and behold with their eyes such things written; would they not plainly pronounce of the authors of such writ, that they were fuller of Lucifer than of Christ, that they were proud-hearted men, and carried more swelling minds than sincerely and feelingly known Christianity can tolerate?

But as unruly children, with whom wholesome admonition prevaileth little, are notwithstanding brought to fear that ever [609] after which they have once well smarted for; so the mind which falleth not with instruction, yet under the rod of divine chastisement ceaseth to swell. If, therefore, the prophet David, instructed by good experience, have acknowledged, Lord I was even at the point of clean forgetting myself, and ofn1 straying from my right mind, but thy rod hath been my reformer; it hath been good for me, even as much as my soul is worth, that I have been with sorrow troubled: if the blessed Apostle did need the corrosive of sharp and bitter strokes, lest his heart should swell with too great abundance of heavenly revelations2: surely, upon us whatsoever God in this world doth or shall inflict, it cannot seem more than our pride doth exact, not only by way of revenge, but of remedy. So hard it is to cure a sore of such quality as pride is, inasmuch as that which rooteth out other vices, causeth this; and (which is even above all conceit) if we were clean from all spot and blemish both of other faults and of pride, the fall of angels doth make it almost a question, whether we might not need a preservative still, lest we should haplyo wax proud, that we are not proud. What is virtue but a medicine, and vice but a wound? Yet we have so often deeply wounded ourselves with medicines, that God hath been fain to make wounds medicinablep; to cure by vice where virtue hath stricken; to suffer the just man to fall, that, being raised, he may be taught what power it was which upheld him standing. I am not afraid to affirm it boldly, with St. Augustine3, that men puffed up through a proud opinion of their own sanctity and holiness, receive a benefit at the hands of God, and are assisted with his grace, when with his grace they are not assisted, but permitted, and that grievously, to transgress; whereby, as they were in over-great liking of themselves supplanted, so the dislike of that which did supplant them may establish them afterwards the surer. Ask the very soul of Peter, and it shall undoubtedly make you itself this answer: My eager protestations, made in the glory [610] of my ghostly strength, I am ashamed of; but those crystalq tears, wherewith my sin and weakness was bewailed, have procured my endless joy; my strength hath been my ruin, and my fall my stay1.

Now what we did at the first observe, the same we must here repeat unto you. As that complaint, which heretofore the prophet Abakuk hath made unto God in the person of the afflicted people of God, had two principal respects; the one to the flourishing estate of impious and cruel persecutors, the other to the woful and hard condition of saints persecuted by their cruelty; so this short abridgment of answer thereunto made hath likewise a double relation. It threateneth the one sort that their swelling pride doth prognosticate their speedy ruin: the other, which counted themselves the children of death, it reviveth, and with the hope of life laid up in store for them, it causeth their bruised hearts to rejoice. So that, whereas before, they mourned in the presence of God, and made their moan, saying2, “For thy sake we are continually slain, and are counted as sheep for the slaughter; why sleepest thou, O Lord? wake, and be not far off for ever: wherefore hidest thou thy face, wherefore dost thou forget our misery and affliction? our souls are beaten down to the dust, they cleave even to the very ground. O Lord, rise up for our succour, and redeem us for thy mercy’s sake:” all these their tears are here wiped away, and such abundance of grace consolatory ministred unto them, that they may now put off sackcloth, and anoint their heads with oil, change their doleful tunes into songs of cheerful melody, shake off that overr-depressing heaviness, and resume their wonted joys; forestalling as it were, and preoccupating that of the blessed Apostle, “Like dead men, yet behold alive3.” “For the just by his faith shall live.” For explication whereof the words themselves do offer occasion to speak, first of the promise of life; secondly, of their quality to whom life is promised; and in the last place, of that dependency whereby the life of the just is here said to hang on their faith.

[611]
In nature those things are properly said to live which do move, having in them that which giveth them their motion; as plainly appeareth to be seen in all those creatures which are commonly termed living: for they move as long as they are said to live. Neither are they moved by any external impulsive force, but a certain divine vigour, which nature hath imbreathed them with, moveth them. Touching men, of all creatures living the chiefest and most eminent, they have their natural life which the soul in the body causeth; and correspondent thereunto some amongst them a life ghostly, wrought by a force much diviner inhabiting the soul. Wherein we are to consider, first the fountain, the cause original and beginning, whereof spiritual life proceedeth: then, in what manner we do here live the life of God: and thirdly, how this life shall in the world to come be perfected.

“I have set before you,” saith Moses, “life and death. Choose life therefore, that both thou and thy seed may live by loving the Lord thy God, by obeying his voice, and by cleaving unto him, for he is thy life and the length of thy days1.” Again, “the children of men,” saith the Prophet, “they shall repose themselves under the shadow of thy wings: they shall be satisfied with the fatness of thy house, and thou shalt give them drink of the river of thy pleasures; for with thee is the well of life2.”

Now “as the Father hath life in himself, so to the Son he hath given to have life in himself also3.” Not so in himself, but that others are, by his quickening force and virtue made alive. For which cause Peter, in the third of the Apostles’ Acts, termeth him “the Lord of life.” He is the life of the world; partly, because for the world he hath suffered death, to procure it eternal life: and partly, for that the world, being really quickened by him, liveth that life which his death hath purchased. The soul which quickeneth the body is in the body. And it must be in the soul, which the soul of man liveth by. Except therefore Christ be truly in you, through him ye cannot be made alive. Hereunto all those sentences apostolic and evangelical have relation. That in the eighth to the Romans, “If Christ be in you, then is the body dead unto sin, but the spirit life for righteousness’ [612] sake.” That in the thirteenth of the second to them of Corinth, “Know ye not how Jesus Christ is in you, except ye be castaways?” That in the second to the Galatians, “Christ Jesus liveth in me.” That in the third to the Ephesians, “For this cause bow I my knees to the Father of our Lord Jesus Christ, that he may grant you according to the riches of his glory to be strengthened in the inner man, that Christ may dwell in your hearts.” That in St. John, “He that is in you is greater than he that is in the world.”

Somewhat strange it seemeth, that a thing in Scripture so often inculcated should be so hardly understood. Granted it is and agreed upon, that he which hath not the Son of God in him hath not life. But how to construe this, we are to seek: some thinking it to be a point inexplicable, a mystery which all must hold, but none is able to open or understand. Others considering, that forasmuch as the end of all speech is to impart unto others the mind of him that speaketh, the words which God so often uttereth concerning this point must needs be frivolous and vain, if to conceive the meaning of them were a thing impossible, have therefore expounded our conjunction with Christ to be a mutual participation whereby each is blended with other, his flesh and blood with ours, and ours in like sort with his, even as really materially and naturally as wax melted and blended with wax into one lump; no other difference but that this mixture may be sensibly perceived, the other not. Which gross conceit doth fight openly against reason. For are not we and Christ personally distinguished? Are we not locally divided and severed each from other? “My little children,” saith the Apostle1, “of whom I travail in birth again until Christ be formed in you.” Did the blessed Apostle mean materially and really to create Christ in them, flesh and blood, soul and body? No: Christ is in us, saith Gregory Nazianzene, not κατὰ τὸ ϕαινόμενον but κατὰ τὸ νοούμενον: not according to that natural substance which visibly was seen on earth: but according to that intellectual comprehension which the mind is capable of. So that the difference between Christ on earth and Christ in us is no less than between a ship on the sea and in the mind of him that builded it: the one a sensible thing, the other a [613] mere shape of a thing sensible. That whereby the Apostle therefore did form Christ, was the Gospel. So that Christ was formed when Christianity was comprehended. As things which we know and delight in are said to dwell in our minds and possess our hearts; so Christ knowing his sheep and being known of them, loving and being loved, is not without cause said to be in them, and they in him. And for as much as we are not on our parts hereof by our own inclination capable, God hath given unto his that Spirit which, teaching their hearts to acknowledge and tongues to confess Christ the Son of the living God, is for this cause also said to quicken. Concerning the fountain of life therefore, this may suffice.

Touching the manner of life spiritual, here begun: Of them that walk in the blind vanity of their own minds, that have their cogitations darkened through ignorance, that have hardened their hearts, that are conscienceless, that have resigned themselves over unto wantonness, that are greedily set upon all uncleanness and sin; of such it is plainly determined, they be dead. Strangers they are from the life of God. Which life is nothing else but a spiritual and divine kind of being, which men by regeneration attain unto, Christ and his spirit dwelling in them, and as the soul of their souls moving them unto such both inward and outward actions as in the sight of God are acceptable. As they that live naturally have their natural nourishment, wherewith they are sustained; so he to whom the spirit of Christ giveth life, hath whereon he also delighteth to feed. He hungereth after righteousness: it is meat and drink unto him to be exercised in doing good: “the hart is not after the rivers of water so thirsty as my soul,” saith the Prophet, “is thirsty after thee, O God.” They that live the life of God, what they delight to taste, let it by those words spoken unto Christ in the Song of Salomon be conjectured, “Honey and milk are under thy tongue;” what to smell, by those, “My beloved is as a bundle of myrrh, as a cluster of camphor:” what to hear, by those, “O let me hear thy voice, thy voice is delectable:” what to see, by those, “Shew me thy countenance, thy sight is comely.” And as the sense, so the motion, of him that liveth the life of God hath a peculiar kind of excellency. His hands are not stretched out towards his [614] enemies, except it be to give them alms: his feet are slow, save only when he travelleth for the benefit of his brethren. When he is railed upon by the wicked, his voice is not otherwise heard than the voice of Stephen, “Lord, lay not this thing to their charge.” Though we could triple the years of Methusalem or live as long as the moon doth endure; our natural life without this what were it? This altereth and changeth our corrupt nature: by this we are continually stirred up unto good things: by this we are brought to loathe and abhor the gross defilements of the wicked world; constantly and patiently to suffer whatsoever doth befall us, though as sheep we be led by flocks unto the slaughter: this dispelleth the clouds of darkness, easeth the heart of grief, abateth hatred, composeth strife, appeaseth anger, ordereth our affections, ruleth our thoughts, guideth our lives and conversations. Whence is it that we find in Abel such innocency, in Enoch such piety, in Noah such equity, in Abraham such faith, in Isaac such simplicity, such longanimity in Jacob, such chastity in Joseph, such meekness and tenderness of heart in Moses, in Samuel such devotion, in Daniel such humility, in Elias such authority, in Elizeus such zeal, such courage in Prophets, in Apostles such love, such patience in martyrs, such integrity in all true saints? did they not all live the life of God?

Which life, here begun, (to come to the last point,) shall be in the world to come finished. Whereof we have heretofore spoken largely. And when we have spoken all we can speak, all which we can speak is but this; he which hath it hath more than speech can possibly express, and as much as his heart can wish: he doth abound and hath enough. For the words of the promise of life, in the tenth of John, are these; “I came that my sheep might have life, and might abound.” Seeing therefore we are taught that life is the lot of our inheritance, and that when we have it we have enough, wherefore struggle we so much for other things which we may very well want and yet abound? When we leave the world, this hope leaves not us: it doth not forsake us, no not in the grave. Sundry are the casualties of this present world, the trials many and fearful which we are subject unto. But in the midst of all, this must be the chiefest anchor unto our souls, “The just shall live.” Wherefore this God setteth before [615] the eyes of his poor afflicted people, as having in it force sufficient to countervail whatsoever misery they either did or might sustain. Those dreadful names of troubles, wars, invasions, the very mention whereof doth so much terrify; weigh them with hearts resolved in this, that “the just shall live,” and what are they but panical terrors? If they promise great things, which are not of power and habilitie to perform the least thing promised, what wise man amongst you is there whom such presumptuous promises do not make rather to laugh than to hope? Yet behold at the threatenings of men we tremble, though we know that their rage is limited, that they cannot do what they list, that the hairs of our heads are numbered, that of so many there falleth not one to the ground without the privity and will of our heavenly Father. How often hath God turned those very purposes, counsels, and enterprises, wherewith the death of his saints hath been sought, both to the safety of their lives, and increase also of their honours! Was it not thus in Joseph, in Moses, in David, in Daniel? If cruelty, oppression, and tyranny do so far forth prevail, that they have their desires and prosper in that which they take in hand: the utmost of that evil which they can do is but that very good which the blessed Apostle doth wish, “Cupio dissolvi.” Thrice happy therefore are those men, whom, whatsoever misery befalleth in this present world, it findeth them settled in a sure expectation of that which here God promised the just, felicity and life in the world to come. Whereof God the Father make you partakers through the merits of his only-begotten Son our blessed Saviour, unto whom, with the Holy Ghost, three persons, one eternal and everliving God, be honour, glory, and praise for ever.

II.
There never was that man so carelessly affected towards the safety of his own soul, but knowing what salvation and life doth mean, though his own ways were the very paths of endless destruction, yet his secret natural desire must needs be, not to perish but to live. “What man is he,” saith the prophet David1, “which desireth, or rather what man is there which doth not desire life, and delight in days wherein [616] he may see everlasting good? Let that man keep his tongue from harm, his lips from guile: let him shun evil, embrace good, pursue peace and follow after it. For the eyes of the Lord [are] upon the righteous, and his ears unto their cry. Their cry he heareth, and delivereth them from all their troubles: near he is unto them that are contrite in heart: men afflicted in spirit he will save: the troubles of the righteous [are] great, but he delivereth out of all: their very bones so charily kept that not as much as one of them broken: such as hate them malice shall slay: the Lord redeemeth the souls of his servants, and none that trust in him shall perish.” What the prophet David largely unfoldeth, the same we have here by way of abridgment comprehended in small room. So that hearing how the just shall live, you hear no less in weight, though in sound much less be spoken. For whatsoever the watchful eye of God, whatsoever his attentive ear; whatsoever deliverance out of trouble; whatsoever in trouble nearness of ghostly assistance; whatsoever salvation, custody, redemption, safe preservation of their souls and bodies and very bones from perishing, doth import: the promise of life includeth all. And those sundry rehearsed specialties, harmlessness and sincerity in speech, averseness from evil, inclination unto good things, pursuit of peace, continuance in prayer, contrition of heart, humility of spirit, integrity, obedience, trust and affiance in God; what import they more than this one only name of justice doth insinuate? which name expresseth fully their quality unto whom God doth promise life.

Slightly to touch a thing so needful most exactly to be known, were towards justice itself to be unjust. Wherefore I cannot let slip so fit an occasion to wade herein somewhat further than perhaps were expedient, unless both the weightiness and the hardness of the matter itself did urgently press thereunto. Justice, that which flourishing upholdeth, and not prevailing disturbeth, shaketh, threateneth with utter desolation and ruin the whole world: justice, that whereby the poor have their succour, the rich their ease, the potent their honour, the living their peace, the souls of the righteous departed their endless rest and quietness: justice, that which God and angels and men are principally exalted by: justice, [617] the chiefest matter contended for at this day in the Christian world: in a word, justice, that whereon not only all our present happiness, but in the kingdom of God our future joy dependeth. So that, whether we be in love with the one or with the other, with things present or things to come, with earth or with heaven; in that which is so greatly available to both, none can but wish to be instructed. Wherein the first thing to be inquired of is, the nature of justice in general: the second, that justice which is in God: the last, that whereby we ourselves being just are in expectancy of life here promised in this sentence of the prophet, “By faith the just shall live.”

God hath created nothing simply for itself: but each thing in all things, and of every thing each part in other hath such interest, that in the whole world nothing is found whereunto any thing created can say, “I need thee not.” The prophet Osee, to express this, maketh by a singular grace of speech the people of Israel suitors unto corn and wine and oil, as men are unto men which have power to do them good; corn and wine and oil supplicants unto the earth; the earth to the heavens; the heavens to God. “In that day, saith the Lord, I will hear the heavens, and the heavens shall hear the earth, and the earth shall hear the corn and wine and oil, and the corn and wine and oil shall hear Israel.” They are said to hear that which we ask; and we to ask the thing which we want, and wish to have. So hath that supreme commander disposed it, that each creature should have some peculiar task and charge, reaching furder than only unto its own preservation. What good the sun doth, by heat and light; the moon and stars, by their secret influence; the air, and wind, and water, by every their several qualities: what commodity the earth, receiving their services, yieldeth again unto her inhabitants: how beneficial by nature the operations of all things are; how far the use and profit of them is extended; somewhat the greatness of the works of God, but much more our own inadvertency and carelessness, doth disable us to conceive. Only this, because we see, we cannot be ignorant of, that whatsoever doth in dignity and preeminence of nature most excel, by it other things receive most benefit and commodity. Which should be a motive unto the children [618] of men to delight by so much more in imparting that good which they may, by how much their natural excellency hath made them more to abound with habilitie and store of such good as may be imparted. Those good things therefore which be communicable; those which they that have do know they have them, and do likewise know that they may be derived unto others; those which may be wanting in one, and yet not without possibility to be had from some other; such are matter for exercise of justice.

And such things are of two kinds; good and desirable either simply unto him which receiveth them, as counsel in perplexity, succour in our need, comfort when we are in sorrow and grief; or, though not desired where they are bestowed, yet good in respect of a further end: so punishments, trembled at by such as suffer them, yet in public nothing more needful.

Now forasmuch as God hath so furnished the world, that there is no good thing needful but the same is also possible to be had; justice is the virtue whereby that good which wanteth in ourselves we receive inoffensively at the hands of others. I say, inoffensively: for we must note, that although the want of any be a token of some defect in that mutual assistance which should be; yet howsoever to have such want supplied were far from equity and justice. If it be so, then must we find out some rule which determineth what every one’s due is, from whom, and how, it must be had.

For this cause justice is defined, a virtue whereby we have our own in such sort as law prescribeth1. So that neither God, nor angels, nor men, could in any sense be termed just, were it not for that which is due from one to another in regard of some received law between them: some law either natural and immutable, or else subject unto change, otherwise called positive law. The difference between which two undiscerned hath not a little obscured justice. It is no small perplexity which this one thing hath bred in the minds of many, who, beholding the laws which God himself hath given, abrogated and disannulled by human authority, imagine that justice is hereby conculcated; that men take upon them to be wiser [619] than God himself; that unto their devices his ordinances are constrained to give place: which popular discourses, when they are polished with such art and cunning as some men’s wits are well acquainted with, it is no hard matter with such tunes to enchant most religiously affected souls. The root of which error is a misconceit that all laws are positive which men establish, and all laws which God delivereth, immutable. No it is not the author which maketh, but the matter whereon they are made, that causeth laws to be thus distinguished. Those Roman laws1, “Hominem indemnatum ne occidito,” “Patronus2 si clienti fraudem fecerit, sacer esto,” were laws unchangeable, though by men established. All those Jewish ordinances for civil punishment of malefactors, “the prophet that enticeth unto idolatry shall be slain3,” a false witness shall suffer the same hurt which his testimony might have brought upon another, life for life, eye for eye, tooth for tooth; all canons apostolical touching the form of church government, though received from God himself, yet positive laws and therefore alterable. Herein therefore they differ: a positive law is that which bindeth them that receive it in such things as might before have been either done or not done without offence, but not after, during the time it standeth in force. Such were those church constitutions concerning strangled and blood. But there is no person whom, nor time wherein, a law natural doth not bind. If God had never spoken word unto men concerning the duty which children owe unto their parents, yet from the firstborn of Adam unto the last of us, “Honour thy father and thy mother,” could not but have tied all. For this cause, to dispense with the one can never possibly be justice; nor other than injustice sometimes not to dispense with the other. These things therefore justice evermore doth imply; first, some good thing which is from one person due to another; secondly, a law [620] either natural or positive which maketh it due; thirdly, in him from whom it is due a right and constant will of doing it as law prescribeth.

The several kinds of justice, distributive, commutative, and corrective, I mean not presently to dwell upon. Only before we come to speak of the justice of God, this one thing generally I note concerning justice amongst men. Almost the only complaint in all men’s mouths, and that not without great cause, is, “There is no justice.” The cure of which evil, because all men do even give over in utter despair that ever any remedy can be devised to help a sore so far gone: seeing there is no hope that men will cease to offer, it remaineth that we study with patience how to suffer wrongs and injuries being offered.

And although the fault of injustice be too general, yet whom particularly we do charge with so heavy a crime, it standeth us upon to be wary and circumspect, lest our reproving do make us reprovable. What more injurious than undeservedly to accuse of injury? It cannot be denied but that cause on all sides hath been and is daily given, for each to blame other in this respect. Howbeit, patience, quietness, contentment, wise and considerate meditation, might surely cut off much from those scandalous accusations which are so often and so grievously, without regard what beseemeth either place or person, poured out in the ears of men. Wherein perhaps our kindled affection were better slaked with sober advice, than overmuch liberty taken to feed our displeased minds. No man thinketh the injuries light which himself receiveth. But first, when we seem to receive injury, how do we know that injury is done us? Whereby discern we that we have not the thing which is due? Doth not every man measure his due for the most part by his own desire? When we have not what we would, we think we should have that which we have not, and that therefore we are wronged. Might not Daniel be thus condemned for being unjust to the Babylonian: the Jews towards the Persian: our Lord and Saviour Christ himself towards the high priest Annas, before whom he stood in judgment? No man can be a competent judge of his own right. Wherefore upon our own only bare conceit to say of any man, we find him unjust, must needs be [621] rashness: which being abated, many accusations of injustice would be answered before they be made. Again; be it that we claim nothing as to ourselves or to others due more than by law we seem to have warrant for, and that in the judgment of mo than one besides ourselves. Do we think it so easy for men to define what law doth warrant?

One example I will propose unto you instead of many, to the end it may appear that there are now and then great likelihoods inducing to think that in equity warrantable which in the end proveth otherwise. A law there was sometime amongst the Grecians, that whosoever did kill a tyrant, should appoint his own reward, and demanding receive it at the hands of the chief magistrate. Another law, that a tyrant, being slain, his five nearest in blood should also be put to death. Alexander Phereus exercising tyranny was by his own wife treacherously murdered1. In lieu of this act she requireth the life of a son both hers and his, which son the same law commandeth to be executed because of his father’s tyranny, and not executed by reason of his mother’s request. The question is, whether the grant or denial of her demand, being such, were justice. On the one side, sith all commonweals do stand no less by performance of promised rewards than by taking appointed revenge, let their hope, who in such cases hazard themselves, be once defrauded, and who will undertake so dangerous attempts? Again, if in this case law have provided that none might revenge the death of tyrants by appointing so many of their nearest to die, how much more likely that such a benefit should make the son to his country ever afterwards dutiful, than his father’s deserved punishment kindle in him a desire of revenge? Besides that punishments are, if any thing, to be abridged, rewards always to be received with largest extent, what if the son had done this which the mother did, should his act by law rewardable be punished because of his near conjunction in blood? And that the father’s offence should more disadvantage the son than his mother’s deserts profit him, it seemeth hard. A bridle undoubtedly it would be to stay men from affecting tyranny for ever, if they might see that enmity with them [622] could not in any case go unrewarded. On the contrary side there is either greater or no less appearance of justice. For first, when two laws do by an unexpected casualty each control other, so that both cannot possibly be kept; what remaineth, but to keep that which cannot but with most public harm be broken? which in this case seemeth not greatly hard to discern; the one being needful unto the common safety of all, the other one body’s only benefit. Secondly, fathers being often much more careful of their children than of themselves, more afraid of the overthrow of their progeny than of their own estate and condition, they could not but be the bolder to tyrannize, if they did hope that their offspring any way might wind itself out of the evil which law inflicteth. Thirdly, were it not a thing intolerable, that so monstrous an act, as a woman to murder her husband unto whom she is so nearly linked, should not only not receive punishment, but receive what reward soever she will herself? Finally, the law bidding first generally any thing that should be demanded in way of reward to be granted, and afterwards commanding the death of the five next in blood, doth by this specialty abridge as it seemeth the former generality, and grant anything, but so that this thing be not demanded. Otherwise, what letteth but that license to exercise tyranny might be required as a reward for taking tyrants out of the way? Not therefore simply what men will ask, but what they ask with reason and without contradiction to law, that only by law doth seem granted.

This may suffice to shew how hard it is oftentimes even for the wisest and skilfullest, to see what is justice and what not. So that not only to ourselves but to others we may seem to take injury when we do not. Howbeit, even when we have not the thing which in truth and in right we should have, it may be notwithstanding that they who do us hurt, do us not that injury for which we may blame them as unjust. There is no injustice, but where wrong is wilfully offered. Is it not a rule of equity and justice, “Nullum crimen patitur is qui non prohibet quod prohibere non potest?” “we are towards them unjust, whose injustice we make complaint of for not doing that which to do they want not will but habilitie.” And when we do not receive [623] as we should at the hands of men, it may be so much even against their wills whom in such cases we think most hardly of, that their infelicity is rather to be sorrowed for, than their iniquity is to be accused.

But let it be, that men of very set purpose and malice bend themselves against us; in this case to abate the keen edge of our indignation at wrong which we suffer, it were not nothing if we did consider the wrong which we do. God we are not able to answer one of a thousand; and of a thousand if but one be unanswered us by men, we are unable to bear it.

To conclude: though we had ourselves never injured God or man, the patience and meekness of Christ in putting up injuries were worthy our imitation. His meekness were sufficient to meeken us, were the wrongs which be offered us never so grievous and unsufferable. If therefore men will not be persuaded not to do, let these persuasions induce us to take wrong with all patience, and to show ourselves just men in bearing the cross which men’s injustice doth lay upon us. Which wisdom God the Father for his Son’s sake grant; unto whom with the Holy Ghost, three Persons, one eternal and everliving God, be honour, glory, and praise, for ever.

III.
As we have spoken of the nature of justice in general, so now we must speak of the justice of God. Wherein lest any man should imagine that we term God just, not because in himself he is so, but because the liking which we have of, and love which we bear unto, ourselves, maketh us to think God such as we ourselves are; it shall not be unexpedient, first, to prove unto you that in God there is this divine virtue called Justice: secondly, to show in what sort God doth exercise that virtue in the regiment of his creatures: thirdly, what injury we do to God for want of right understanding how he doth justice unto us: last of all, what honour unto him, and us what benefit, the true knowledge of his justice addeth.

I should have a large and scopious field to walk in, if I did here endeavour with exactness either to collect so many [624] reasons as might forcibly demonstrate, or to reckon up the numbers of particularities effectual to make plain and evident, that in the thirty-third of Exodus which God himself doth insinuate, terming himself “all good.” For that mystical suit of his servant Moses, “I beseech thee, show me thy glory,” thus he answereth; “I will make all goodness to go before thee.” As therefore there can be no particular warmth which universal heat containeth not, so the infinite being of God comprehending all goodness, if justice be any part thereof, God necessarily is just. Secondly, who doth not yield unto justice more than the meanest place of reckoning and account amongst good things? Put therefore the case, that angels and men were just, God not: should they not in this part of goodness excel God, and so be better than He to whom the title, as of “greatest,” so of “best,” is confessed due? Besides, God himself being the supreme cause which giveth being unto all things that are, and every effect so resembling the cause whereof it cometh, that such as the one is the other cannot choose but be also; it followeth that either men are not made righteous by him, or if they be, then surely God himself is much more that which he maketh us; just, if a [He] be the author, fountain, and cause of our justice. Finally, seeing that we cannot conceive God without correspondence between him and creatures receiving from him whatsoever they have or are, either we must think that God cannot choose but impart good things, and then what creature would give him thanks, ever invocate, adore, and worship him? or if he distribute his graces advisedly, knowing upon whom what and wherefore he doth bestow, this being the proper function of justice, doth it not follow that God is just?

Only this doubt there is. We have already declared justice to be that virtue whereby we yield and receive good things in such sort as law prescribeth. Now God hath no superior; there is not that can lay commandment upon him; he is not subject; he standeth not bound to any higher authority and power. How then should there be any justice in his doing that which no superior’s authority or law can bind him to do? To this we could make no answer at all, if we did hold as they do who peremptorily avouch that there is no manner [625] why to be rendered of any thing which God doth, but only this, It was his absolute will to do it. True it is that thus the prophet speaketh in the Psalm1, “Our God is in heaven; and whatsoever he will, he doth.” Thus our Saviour in the Gospel2, “I give thee thanks, O Father, Lord of heaven and earth, because thou hast hid these things from the wise and men of understanding, and hast opened them unto babes. Even so, O Father, because such was thy good pleasure.” Thus the blessed Apostle often3, “God predestinateth, calleth, saveth, worketh all things, according unto the purpose of his own will.” But what infer we hereupon? That there is no other cause in any of all the works of God to be either searched or rendered but this? If so, then it seemeth that when the people do ask this question, in the fifth of Jeremy’s prophecy, “Wherefore hath the Lord our God done these things?” God should rather have closed up their mouths with sharp reproof for making any such demand, than have commanded the prophet to content and satisfy their minds by yielding a reason of his actions: Thou shalt answer them, “like as ye have forsaken me, and served strange gods in your land, so shall ye also serve strangers in a land that is not yours.” Again, let those very alleged sentences be seen into; and by sifting them it will soon appear that they rather exclude the rendering of some one cause which we are specially to beware of than import an impossibility of any reason at all to be rendered of the works of God. Our nature is prone unto haughty conceits: and when we see those blessings abundantly poured upon us, which God hath withheld from sundry others, we easily imagine that what we have more we are more worthy of than others are. To take down this proud opinion, it is so often inculcated, that whatsoever we have, the reason wherefore we have it is not our dignity, but his mercy; not the worthiness of our merit, but the goodness of his will. Yea, even in that very place where the blessed Apostle setteth down our predestination and adoption thorow Christ to have been according unto the pleasure of God’s only will, doth not himself yield a cause of this will in God, by [626] immediately adding, “unto the praise of the glory of his grace1?”

Then seeing God doth work nothing but for some end, which end is the cause of that he doth, what letteth to conclude that God doth all things even in such sort as law prescribeth? Is not the end of his actions as a law? Doth it not strictly require them to be such as always they are, so that if they were otherwise they could not be apt, correspondent, suitable unto their set and appointed end? There is no impediment therefore but that we may set it down, God is truly and properly just.

Touching the next point, how God doth exercise justice in the world, justice exhibiteth all good which congruity and right would have imparted unto equals, inferiors, or betters. Superiority and equality being excluded from all things as they are in relation unto God, at his hands we are to expect only that which justice yieldeth unto inferiors. In which consideration he taketh upon him the person of a Judge, a Lord, a Father. “He shall judge nations,” saith the prophet in the seventh Psalm. But because those future comminations seem to imply some truce and respect for the time, the wicked man through freedom from present sense of evil emboldeneth himself, taketh heart and courage, hates to be reformed, casteth the words of God behind him, runneth on his race with lost companions, for this refraineth not a whit the more, avoideth no one deed, keepeth not in any one word or syllable which his heart delighteth to utter, for fear of this; “God will judge the world,” is little cared for, though Christ our Saviour and his Apostles divinely inspired describe it in never so fearful manner. For which cause the prophet in the same Psalm addeth, that God not only shall judge nations, but is the judger of the just and of despisers of God daily. So that what criminals openly convicted are to look for from such a judge as respecteth no man’s person, standeth in awe of no man’s countenance, hateth sin extremely, knoweth every action and circumstance of action that sinners do, be it never so closely conveyed; what criminals convicted are to look for from such a judge, thereon let impenitent malefactors make their certain reckoning: for as verily as [627] God is just, his justice will show itself upon them soon or sine1, in the greatness of that judgment, which if they feel before they fear, woe worth them. God their judge, but your Lord. Wherefore, if unfeignedly ye do your endeavour to serve and please him, ye have your presidents to claim the benefit by, of protection, care, maintenance, and whatsoever good thing righteous dominion doth answer dutiful service withal. The Church, in the thirty-third of Esay, concludeth hereupon boldly and plainly, “He is our king, therefore he will save us.” Is it not much that free leave is given you to plead your causes as Ezechias pleadeth his2, “Lord, remember now how I have walked before thee in truth with a perfect heart; and have done that which is good in thy sight?” As David his3, “Preserve my soul, O Lord, thou art of great kindness unto all that serve thee: save me, for I am thy servant: O Lord, enter not into judgment with thy servant: judgment for thine enemies and them that hate thee, I am the son of thine handmaid, thy servant; O bruise not my bones, suffer not my soul to descend into hell.” Or, if the name of a Lord do not seem sufficiently gracious, unto whom God hath already imparted a spirit that giveth them cheerful courage boldly to call upon him as children upon their father, let them enlarge their hearts, and what good thing can they invent which his fatherly indulgence doth not abundantly warrant them to expect? If they thirst after consolation; behold to them it is said4, “As one whom his mother comforteth, so will I comfort you.” If they wish endless continuance of hearty affection; to them5, “I have loved you with everlasting love:” if a prosperous and flourishing estate; of them6, “I will be unto them as the dew, they shall grow as the lily, and fasten their roots like the trees of Lebanon; their branches shall spread, and their beauty like the olive-tree; they shall revive as the corn, and flourish like the pleasant vine.” It is not with God as it is with men, whose titles [628] show rather what they should be than what they are. God will not be termed that which he is not. His name doth show his nature. Were not his affection most fatherly, the appellation of a Father would offend him. Fathers lay up treasure for their children: and shall not your heavenly Father provide sufficient for you? O minds void of faith, full of distrustfulness! Fathers spend out the day in travail, and continue the night in pensiveness, ever studying how to better their children’s estate: and have the sons of God a father careless whether they think [sink] or swim? “The eye of the Lord is over the righteous.” “If a mother forget her child, (O love inexplicable!) art thou my son? of thee I will never be unmindful.” Fathers, if they be provoked unto anger, conceive not unappeasable wrath: do not the tears of their children confessing faults and craving pardon wring out oftentimes tears from their eyes? And, that which should cause even hearts of stone and iron to relent, we do not find God in Scripture so often rejoicing over the righteous, as shedding forth tears of kindness in the bosom of sinners penitent. Thus God is righteous; and his righteousness thus he showeth.

It followeth in the next place, concerning this matter of divine justice, that we consider how, for want of right understanding the reason how God doth justice unto us, injury is done unto him many ways. For by this it cometh to pass, that some beholding the present not only impunity but prosperity of sin in the world, repine at it as at a thing repugnant unto divine justice. Some, noting a difference between men departing this mortality immediately after great and grievous sin repented of, and others always leading an honest, holy, virtuous and upright life, upon conceit of inconformity with justice in God, if both ending their lives should enter forthwith and immediately into bliss, have imposed upon the souls of faithful men a kind of after-punishments satisfactory. Some, considering how God as a just and righteous judge shall hereafter reward their works, glory in them, as if, unless in themselves they were worthy of reward, they could not in justice be rewarded. These err by thinking that to be against God’s justice which is not: on the contrary side, others by thinking that not to be against [629] it which is. These not weighing how opposite it is to the justice of God either actually to condemn, or in purpose to determine condemnation, without a cause thereof presupposed in the party so condemned, have by misconstruction of some Scripture sentences with no small hazard, as well of God’s honour as men’s comfort, over-easily been led to define that so many were fore-appointed unto endless torments, only for that the will of God was to have them endlessly tormented.

IV.
What injury men do to God for want of right understanding in what sort and manner he doth administer equity and justice unto them, in no way plainlier appeareth, than first by those repining accusations wherewith the hard and heavy casualties of the righteous, contrariwise the impunity and prosperity of godless persons hath been from time to time complained of. With such kind of pleas books both profane and sacred are fraught. The motives especially inducing their minds to deem an incongruity herein, and to the justice of God a kind of repugnancy, are these. First, to that justice which we call distributive, and define to be a virtue yielding unto each person that which is due according to the difference of their quality; unto this virtue nothing more opposite than the parity of their condition in the quality of whose persons there is inequality. For which cause from God Abraham putteth off that unevenness, which blendeth these two, and maketh the one’s estate such as the other’s should be1. “Far be it from thee to slay the righteous with the wicked: that as the wicked are so the righteous should be also, far be it from thee.” If then it be a thing most unequal and unconsonant unto justice, that they which excel in virtue should not be exalted in all parts of happiness above them that are of contrary note: if it do argue an uneven hand, to bestow upon the one sort as upon the other; what may be thought, when they, whose virtues all men do admire, are in respect of the hard condition of their lives for outward things not only as the worst, which notwithstanding were greatly to be complained of, but in so far more miserable and wretched case, that these living in all abundance of whatsoever their [630] hearts can wish; they, if they perish not, as oftentimes they do, at their enemies’ will and pleasure, are found not seldom in such sort to live that their deadliest adversaries could hardly wish them greater woe than to continue as they are; doth it not stand even with reason to conclude, surely this is not that which equity and justice requireth?

Wherein, secondly, the judgment of the world doth universally so agree, that imprisonments, banishments, restraint of liberty, deprivation of honour, diminution of goods, loss of limme or life, any thing penal and unpleasant to be suffered, is by authority no where laid upon other than dangerous and pernicious malefactors. So that when contrariwise the supreme guide and governor of heaven and earth taketh a clean other course of regiment, impoverishing, depressing, and by all means keeping down the good and virtuous, but crowning the heads of malignants with honour, and heaping terrene felicity upon them, this can hardly seem just or according to righteousness. It is not therefore without cause, nor of nothing, that those so usual oppositions have in this case and question risen, some concluding if God indeed did with justice order the course of human affairs, it should be bonis benè, malis malè; well with the good, with the bad still otherwise: others crying out, Posse contra innocentiam quæ sceleratus quisque conceperit; impiety to prevail against innocency, even as far as it listeth, God himself looking on, who can but wonder and be amazed?

The state of good and bad thus continuing, what construction shall we make of God’s own promises unto the one sort, and to the other of his so heavily pronounced sentences, which he uttereth as it were emptying upon them vessels full of wrath and execration? To the one, “If thou wilt walk in my ways, and keep mine ordinances and commandments, I will lengthen and prolong thy days1:” to the other, “Thou, O God, shalt bring them down, thou shalt humble them unto the pit of corruption: bloody and deceitful men shall not live out half the time which they might by nature2.” To the one, not only long life promised, but with life prosperity and peace: to the other, not only unseasonable death, but before death woe and all kinds of misery threatened. To the one, [631] “What man is he that feareth the Lord? his soul shall dwell at ease, and his seed shall inherit the land1.” “The earth shall yield him increase of fruit; it shall be fat for his sake as oil; his cattle shall feed in large pastures2.” To the other, “Cursed shalt thou be in field, town, and city; in person, in goods, in children: the Lord shall send upon thee trouble and shame: in all that ever thou settest thy hand to, thou shalt never but suffer wrong and violence: the strangeness of those calamities which thine eyes shall behold shall take even wit and sense from thee; because thou wilt not serve the Lord thy God with a cheerful and true heart, that so thou mightest be in all things happy. Hunger and thirst, and nakedness, and want of all things necessary shall be thy undividable companions; misery shall hunt and pursue thee for ever: no peace, no prosperity for the wicked3.” These being the words of God’s own mouth, how are they performed when the righteous are hourly led as sheep to the slaughter, their goods taken from them by extortion, their persons subject unto violence, nothing about them but that which they cannot look or think upon without tears: impious despisers of God in the meanwhile rejoicing pleasantly upon their beds, living long, waxing old, increasing in honour, authority, and wealth, their houses peaceable without fear, the rod of God not upon them nor near them. Can these things cleave together, God true in his word, and we such in our estate?

This we might happily either answer with more ease, or with better contentment endure, if to the harm that such interchangeable mixture of states in the world breedeth any countervailable good did grow. But there doth not, for ought that any man living can see. The damages, losses, and inconveniences which this confusion draweth after it, they are apparent. For as the benefit but even of one man’s virtue, taking root, continuing and flourishing in the world, is invaluable not only in respect of the courage which thereby all others well inclined do take, exulting in the conscience of their own most holy resolutions to serve the Lord, when they are therein confirmed by visible assurance, that with as many as fear him from their hearts it shall undoubtedly go well; [632] but furder also in regard of the singular delight which itself doth take in being most largely beneficial, and in watching for occasions to do good, whereby it cometh to pass that the hearts of all men bless them as common fathers, and wish them, if it were possible, the very possession of heaven on earth: so on the other side, there can be no greater plague than improbity, if it come once to have any long continuance in the world, and be furnished with habilitie to annoy; because it doth not only hereby take occasion to scorn the better endeavours of more virtuously disposed minds, thinking with itself what profit have they by serving the Almighty; but maketh it even a recreation and a kind of sporting exercise, to try what wit can do in devising, and force in executing, vile, barbarous, and cruel acts, such as future ages may most wonder at and the present most rue. Sith therefore nothing doth more agree with the nature of God than to better the state of all things, what more effectual way to fill the mouths of his saints with hymns of everlasting thankfulness, to augment their joy, to illustrate his glory, to put his foes for ever to silence, and to manifest unto all generations the care which he hath of righteousness, than by making always an apparent separation between men in state according to their good or evil quality?

These are the principal inducements whereby men, as long as they do not conceive the course of divine proceedings in justice, imagine all to be out of square, because the righteous are afflicted when the contrary sort doth prosper. First, it seemeth against the rule of distributive justice, that men’s condition should not be suitable unto the quality of their persons. Secondly, the general opinion and judgment of all men disliketh to have it otherwise. Thirdly, God himself often and openly hath protested that so it should be. Finally, if it be not so, the inconveniences thereupon growing unto the world are more than mean, the virtuous not encouraged as they might be, but put out of heart, infinite good undone whereby thousands would reap benefit, impiety corroborated and made bold, no less unto God’s own dishonour than unto men’s discomfort.

It cannot be thought a labour needless that we do our endeavour to free this cause from all scruple, and to make it so [633] expedite as may suffice for our reasonable satisfaction; the minds of so many being entangled with such perplexities when they enter into these alleged considerations, through an opinion of discoherence thereby conceived between the justice of God and the state of men in this world. First therefore, touching the rule of distributive justice, which requireth that whose quality is best, their condition be not like and much less inferior unto theirs which are worst qualified, how understand we this rule of justice? Doth it require that the righteous have every desirable thing, the unrighteous nothing which is naturally good permitted them? Then that which never as yet any man was so senseless as to imagine notwithstanding must needs be; to wit, that if only the just be not beautiful, if they only be not strong, if any be healthful besides them, if they alone do not see the fruit of their bodies increased unto the third and fourth generation, God doth deal unjustly with them. How unjustly therefore with Christ, our blessed Saviour, and his only begotten Son, who, being so much more righteous than angels, saw creatures far beneath men in dignity, in some parts of outward felicity so far above him, that birds having nests, and foxes holes to hide themselves, the Son of God and man had scarce where to lay his head! Know we not that God is by nature good and gracious unto all the works of his hands? Wicked men, although they be their own workmanship as they are wicked, yet as they are men being his handywork, are not we rather injurious unto them than God to us, if so be we envy them all participation even in those things which they are capable of as men? For the favours which God extendeth towards just men, not as they are men but as they are just; such favours are so peculiarly theirs, that they neither are nor can be imparted to any other. Judge thereby therefore their estate, and is it not clear as the light, that the foresaid rule of justice is no way violated? Judge according unto this, and most evident it is that God doth not deal with the righteous as with the wicked, but always better. What should I mention him that preferred imprisonment with Cato before some other’s imperial sublimity1? It had been more than childishness in Moses to choose a fellowship in the bitter [634] afflictions of the people of God1, refusing the offered pleasures of sin, if the just man’s estate, be it whatsoever, were not by infinite degrees happier than the wicked’s in their chiefest ruff. He that sitteth at this day in Rome, kings of nations falling down before him, is his glittering estate so glorious in the eye of any good and spiritually wise man’s judgment, doth his tripled diadem adorn him as those honourable robes and garments dyed in the blood of martyrdom did beautify his first most reverend predecessors, disgraced, discountenanced, banished, murdered, rent asunder, devoured by wild beasts, put to most sharp and cruel deaths, exercised with all extremity of torture, for the name of Christ? There was not the meanest of them that would have changed his comforts in the midst of greatest woe, with all the joys and honours worldly which the flourishing rank of their successors hath acquired.

When we think otherwise, the reason of our misconceit herein is, that because all suffering is grievous, even as the contrary pleasant and acceptable unto the flesh; by occasion of this common accident, the just and unjust suffering materially the same kind of grief, by hunger, pestilence, sword, or the like, imagine that they suffer simply the same: whereas in truth their sufferings formally, and even essentially, are different. The end of God is never the same in both, howsoever upon both he seemeth to lay the same burthens. But being both in the same furnace, the one are as stubble, the other as gold: being stricken with the same rod, the one receive the torment of a judge, the other the chastisement of a father: though both seem equally forsaken, they are never equally forsaken; but the one by dereliction of probation only, the other by dereliction of reprobation. The righteous therefore may have their phancies; they may, being carried away with grief or distempered with passionate affections, conceive worse of their own estate than reason giveth: but surely there never was yet that hour, wherein, if mortal eyes could discern the things that belong unto solid happiness, the hearts of the most unhappy would not wish, as Balaam’s did, “O that we were as the just and righteous!” So that the rule of distributive justice is not violated. As for the judgment of all the world, supposing yes, what should we weigh [635] it, when we have the judgment of him who created the world, to the contrary?

Howbeit, we err, if we take the casual and unadvised sentences of men, uttering rashly that which indignation hath put in their mouths and not sound reason established their minds in, for the judgment of the whole world: whereof the wisest and skilfullest part is so far from judging God when his saints are most roughly dealt with, to give them the portion of malefactors, that they plainly and peremptorily avouch the evils which they suffer to be rather seals assuring them of everlasting bliss, than tokens arguing unto others, that God doth put no difference between them and the children of malediction.

In the words of our Saviour there is no enigmatical obscurity. “When men revile you, sclaunder you, hate you, when they cast you out of their synagogues, when they speak and practise all manner of evil against you, say not in your hearts, this lot should have fallen upon the wicked that know not God. Such sufferings do not argue your infelicity, for when ye suffer these things ye are happy, yea because you suffer them happy are you. Men shall woonder that serving a God so able to protect you, ye should be enfeebled and die daily: but ignorant they are how it cometh by the mighty hand of God to pass, that there is even in imbecility strength, and gain in the very loss of your lives.” Nor doth any thing done or suffered in this present world prejudice a whit the grand authority, or impair the sacred credit either of the promises of God containing the good things of this life which are proposed to them that serve him, or of the contrary threatenings denounced against the children of rebellion and disobedience. That which befalleth us maketh no way vain and frustrate what God speaketh. But that which is spoken and meant conditionally must be conditionally understood. The life of the just shall be long and fortunate; they shall see many and happy days; their prosperity is a sequel of their piety; but with exception, unless it be far better for them to be otherwise. That this may be far better for them, there needeth no other proof, than the very acknowledgment of men touching the fruit of their own afflictions. Minds which prosperity would make wanton, experience [636] of hard events do keep in subjection and awe. Affliction is the mother of hearty devotion. “When God humbled their hearts with heaviness,” saith the prophet, speaking of Israel, “then they cried unto the Lord.” When they loathed and abhorred their food, then they poured out their very souls in supplication unto God. Affliction is both a medicine if we sin, and a preservative that we sin not. Again, if sentence of death and temporal calamity be given against such as hate to be reformed, the certain performance thereof we must count upon; but with this caution, so far as may stand with that woonted patience which God useth ordinarily towards sinners, and so far as it may be without let and hinderance unto any greater intended good than can grow by their speedier revenge. In which considerations, if God do suffer with unweariable toleration vessels concinnate unto death, shall this, than which nothing doth more show his mercy and love towards men, by men be alleged to implead his righteousness?

“But good whereunto this tendeth, we say we discern none, sundry inconveniences being apparent.” Truth, they say, is the daughter of time1: and in time who doubteth but God may discover that, which, because we presently see not, must we needs therefore presently deny? Into the heart of Joseph, at what time his brethren made gain of his person by merchandise; into the heart of Daniel, at the hour wherein he left his native soil; hardly could it have sunk2 what good so unpleasant accidents in the end would grow unto. “The end of all things,” saith the Apostle, “is at hand.” And if till then it should lie buried in the bosom of God alone, unto what good these things in outward appearance so confused for the time may tend; yet we to be less advised than that heathen Platonic, uninstructed in the mysteries of our faith? “In that I understand concerning the works of God,” saith Plotin, “therein will I praise him; and admire him even in those things which I know no reason of3.” Do not we ourselves [637] many times that whereof our servants do see no cause? neither dare they therefore argue and dispute against our actions, because our intentions are hidden from them. As for the wicked that hereby take occasion to harden themselves, it is to their own greater woe in the end. The time is not gained; divine revenge shall come upon them so much the heavier, by how much the slower. If the virtuous do fail in courage, it is through error and misconceit. “There was a time,” saith the prophet David, “when beholding fools in prosperity, I fretted at it in my heart, saying, ‘Lo, these are wicked, yet prosper they alway, and increase in riches: surely in vain have I cleansed my heart; that I have washed my hands in innocency, to what purpose is it?’ Such was my ignorance, such my folly1.”

V.
Another sort of men, injurious unto the God of heaven for want of understanding how towards them God is righteous, are they who abridge his mercy towards sinners penitent, tormenting their minds with a fearful expectation of future anguish, tribulation, and woe; as if, how merciful soever God be in remitting, pardoning, forgiving all their transgressions, nevertheless so unappeasable is the rigour and dirity of his corrective justice, that till transgressors have endured, either in this world or another, vexation proportionable unto the pleasure which they have taken in doing evil, there is no possible rest for their souls. Upon which opinion because much dependeth, I will first endeavour to lay before you, how the favourers and defenders thereof do ground it upon a supposed exigence in the justice of God; and secondly, make manifest unto you how weakly and ungroundedly they have erected it: how the nature of divine justice doth not only not require it, but is by it plainly oppugned, denied utterly, and overthrown.

Their grounds, unto such as cast but a slight view over them, may seem to be strong and forcible, they are with such art and cunning laid. The parts of their doctrine concerning the point which now we treat of, are by their greatest masters [638] thus cemented and set together. First, most true it is, they say, and of all Christian comfort the very root, that the death of our Lord and Saviour hath duly and sufficiently paid for the sins of all the world, by that abundant price of redemption upon the cross. Which solemn entrance being such as cannot but have the full and ready approbation of all men Christian without any pause or furder deliberation gladly yielded, they smoothly proceed, adding hereunto that which cannot reasonably neither be denied; to wit, that no man was ever partaker of this benefit but in the knot and unity of his body mystical, which is the Church: that to them the streams of the holy blood of Christ and beams of his grace are in sundry manners conveyed: that upon all men, at their first incorporation into the household of the faithful, the merits of the death of Christ are so largely carried down for the remission of their sins, that were their lives before never so loaden with the most enormous offences that in this misery man may commit, yet they are not only pardoned of the same, but also perfectly acquitted for ever of all pain and punishment, which his offences by any means committed might deserve: that if men received into the favour of God and fellowship of his Church do, by sin committed after baptism, again pollute the temple of God, their estate is not such as Novatus would have it, irrecoverable, but even they may also be repaired through repentance; God most largely and mercifully promising unto his children which have erred and gone astray, if they return, if they be penitent, full remission of all their sins.

Whom we have found in so many things and so weighty true of their word, we do not easily suspect of deceit. Wherefore, as having now full possession of their hearers’ minds, they slip into that, which, being in truth utterly repugnant unto the verdicts hitherto given, they notwithstanding adjoin as consonant and agreeable thereunto. Sin, they say, committed draweth after it a double evil: First, it polluteth1, defileth, staineth the purity and dignity of our nature: secondly, it maketh the soul that sinneth obnoxious unto punishment deserved by sin. Now God remitteth indeed [639] the manifold sins of his children upon their hearty repentance, yea acquitteth them from that great pain, death and endless condemnation, which their iniquities justly deserved: howbeit doth not always, together with the remission of deadly sins and eternal punishment, exempt offenders received to his grace from all correction due1 for sin. That justice exacteth punishment for offending, even after their offences be forgiven them, there is, as it seemeth, proof sufficient mo ways than one. For first, have not just and holy men in this respect taken most sharp revenge upon themselves? Hath not the Church, for the satisfying of God’s most heavy indignation, from the very first spring of Christian religion, perpetually enjoined transgressors certain penal works of correction, either before, as the old usage was, or after the release of their offences, which now of late for grave causes hath been more used? When men do neither chastise themselves, nor are by the Church’s rod chastised, so inevitable2 is the punishment of sin, that it is a kind of constraint unto God himself to punish, yea to punish them whose sin he hath pardoned and received them into favour. Was it not thus in our first progenitors, whose grievous transgression though pardoned, yet both they did and we do smart for? For this cause the blessed Apostle plainly to them of Corinth3, “See ye not how many there are amongst you weak and feeble, how many fallen asleep:” some stricken with sickness, some with death? This we might help, if we were not careless. If we did judge ourselves, we should not be judged of God: now we are, that with the [640] world we might not perish. It cannot therefore be doubted of, but there is pain due for sin after sin be remitted. And if any debt or recompense remain to be discharged by the offender after reconcilement, it must needs rise by proportion, weight, continuance, number, and quantity of the faults committed before. Which debt we cannot say all men do fully discharge in this world. How many thousands do live at ease, secure, and altogether careless thereof? How many, by reason of their late conversion, taken out of the world before they can fully discharge this debt? So that if there were not in the next life pains satisfactory for them to endure, the case of grievous sinners till the very hower of death were much better than of small offenders converted long before: a thing not seemly to God’s justice. Unless perhaps we think that God shall be forced of necessity to remit his debt, for lack of means to punish it in another world. The punishments, which God hath reserved for his children after this life, are of two kinds1: the one, want of perfect felicity and bliss; the other, sense of fearful and grievous torments. In the former of these two Adam and all the fathers2 before Christ, till Christ’s coming, were for so many worlds together detained, to satisfy for the punishment due to the sins the guilt whereof was in this life forgiven them. Nor did only the holy patriarchs feel in this respect the lack of the abundant fruition of the majesty of God, but all the souls of the just, excepting some, who by peculiar prerogative have already received their bodies, being now in rest and unspeakable felicity, do nevertheless for sin want the increase of joy and bliss, that by receipt of their bodies lying as yet in the dust, they are hereafter undoubtedly sure of. This they term pœnam damni3. The other punishment, which hath in it [641] not only loss of joy but also sense of grief, vexation, and woe, is that whereunto they give the name of purgatory pains, in nothing different1 from those very infernal torments which the souls of castaways, together with damned spirits, do endure, saving only in this, there is an appointed term to the one, to the other none; but for the time they last, they are equal. Nor may we therefore think ourselves quite and clean discharged of all such punishment, though we do never so carefully beware of heinous offences. For the common infirmities2 and daily trespasses which defile the works of the virtuous, as immoderate laughter, excessive jesting, smaller exceedings in meats, drinks, attire, and the like, distractions of mind, wandering cogitations in holy exercise; these, though easily pardonable and venial oversights, yet deserving temporal pain, the same unforgiven here must have of necessity afterward the punishment which justice requireth. This taught in Scripture, this determined in councils general, this believed by the ancient fathers, this by the very heathens acknowledged. The doctrine which maketh either denial or doubt of this, giveth license unto evil livers, and is the very mother of presumption.

The whole sum of all this we may reduce unto these two grounds. First, the justice of God requireth, that after unto the penitent sin is forgiven, a temporal satisfactory punishment be notwithstanding for sin inflicted by God or man. Secondly, the same doth also require, that such punishment [642] being not inflicted in this world, it be in the world to come endured; that so to the justice of God full and perfect satisfaction may be made. For each of these, we have with sincerity and care touched the very principal flower of that which the wisest and learnedest on that part have hitherto alleged as proofs to stand upon. So that if this be answered unto the full contentment of reasonable men, I hope we shall not be thought unreasonable for withholding our assent from that which they urge upon the world with greater eagerness than weight of speech1.

* * * * * * *

[643]
A REMEDY AGAINST SORROW AND FEAR: DELIVERED IN A FUNERAL SERMON.
John xiv. 27.
Let not your hearts be troubled, nor fear.

SERM. IV.THE holy Apostles having gathered themselves together by the special appointment of Christ, and being in expectation to receive from him such instructions as they had been accustomed with, were told that which they least looked for, namely, that the time of his departure out of the world was now come. Whereupon they fell into consideration, first, of the manifold benefits which his absence should bereave them of; and secondly, of the sundry evils which themselves should be subject unto, being once bereaved of so gracious a Master and Patron. The one consideration overwhelmed their souls with heaviness, the other with fear. Their Lord and Saviour, whose words had cast down their hearts, raiseth them presently again with chosen sentences of sweet encouragement. “My dear, it is for your own sakes that I leave the world. I know the affections of your hearts are tender, but if your love were directed with that advised and staid judgment which should be in you, my speech of leaving the world, and going unto my Father, would not a little augment your joy. Desolate and comfortless I will not leave you; in spirit I am with you to the world’s end: whether I be present or absent, nothing shall ever take you out of these hands; my going is to take possession of that, in your names, which is not only for me but also for you prepared; where I am, you shall be. In the mean while, ‘My peace I give; not as the world giveth, give I unto you: let not your hearts be [644] troubled, nor fear.’ ” The former part of which sentence having otherwhere already been spoken of, this unacceptable occasion to open the latter part thereof here I did not look for. But so God disposeth the ways of men. Him I heartily beseech, that the thing which he hath thus ordered by his providence, may through his gracious goodness turn unto your comfort.

Our nature coveteth preservation from things hurtful. Hurtful things being present do breed heaviness, being future do cause fear. Our Saviour to abate the one speaketh thus unto his disciples, “Let not your hearts be troubled;” and to moderate the other, addeth, “Fear not.” Grief and heaviness in the presence of sensible evils cannot but trouble the minds of men. It may therefore seem that Christ required a thing impossible. Be not troubled. Why, how could they choose? But we must note, this being natural and therefore simply not reprovable, is in us good or bad according to the causes for which we are grieved, or the measure of our grief. It is not my meaning to speak so largely of this affection, as to go over all particulars whereby men do one way or other offend in it; but to teach [touch?] it so far only as it may cause the very Apostles’ equals to swerve. Our grief and heaviness therefore is reprovable sometime in respect of the cause from whence, sometime in regard of the measure whereunto it groweth.

When Christ the life of the world was led unto cruel death, there followed a number of people and women, which women bewailed much his heavy case. It was natural compassion which caused them, where they saw undeserved miseries, there to pour forth unrestrained tears. Nor was this reproved. But in such readiness to lament where they less needed, their blindness in not discerning that for which they ought much rather to have mourned, this our Saviour a little toucheth, putting them in mind that the tears which were wasted for him might better have been spent upon themselves; “1Daughters of Jerusalem, weep not for me, weep for yourselves and for your children.” It is not, as the Stoics have imagined, a thing unseemly for a wise man to be touched with grief of [645] mind; but to be sorrowful when we least should, and where we should lament there to laugh, this argueth our small wisdom. Again, when the Prophet David confesseth thus of himself, “1I grieved to see the great prosperity of godless men, how they flourish and go untouched;” himself hereby openeth both our common and his peculiar imperfection, whom this cause should not have made so pensive. To grieve at this is to grieve where we should not, because this grief doth rise from error. We err when we grieve at wicked men’s impunity and prosperity, because their estate being rightly discerned they neither prosper nor go unpunished. It may seem a paradox, it is a truth, that no wicked man’s estate is prosperous, fortunate, or happy. For what though they bless themselves and think their happiness great? Have not frantic persons many times a great opinion of their own wisdom? It may be that such as they think themselves, others also do account them. But what others? Surely such as themselves are. Truth and reason discerneth far otherwise of them. Unto whom the Jews wish all prosperity, unto them the phrase of their speech is to wish peace. Seeing then the name of peace containeth in it all parts of true happiness, when the Prophet saith plainly2, that the wicked have no peace, how can we think them to have any part of other than vainly imagined felicity? What wise man did ever account fools happy? If wicked men were wise they would cease to be wicked. Their iniquity therefore proving their folly, how can we stand in doubt of their misery? They abound in those things which all men desire. A poor happiness to have good things in possession. “3A man to whom God hath given riches and treasures and honour, so that he wanteth nothing for his soul of all that it desireth, but yet God giveth him not the power to eat thereof;” such a felicity Salomon esteemeth but as a vanity, a thing of nothing. If such things add nothing to men’s happiness where they are not used, surely wicked men that use them ill, the more they have, the more wretched. Of their prosperity therefore we see what we are to think. Touching their impunity, the same is likewise but supposed. They are oftener plagued than we are aware of. The pangs they feel are not always [646] written in their foreheads. Though wickedness be sugar in their mouths, and wantonness as oil to make them look with cheerful countenance; nevertheless if their hearts were disclosed, perhaps their glittering estate would not greatly be envied. The voices that have broken out from some of them, “O that God had given me a heart senseless, like the flint in the rocks of stone,” which as it can taste no pleasure so it feeleth no woe; these and the like speeches are surely tokens of the curse which Zophar in the Book of Job poureth upon the head of the impious man, “1He shall suck the gall of asps, and the viper’s tongue shall slay him.” If this seem light because it is secret, shall we think they go unpunished because no apparent plague is presently seen upon them? The judgments of God do not always follow crimes as thunder doth lightning, but sometimes the space of many ages coming between. When the sun hath shined fair the space of six days upon their tabernacle, we know not what clouds the seventh may bring. And when their punishment doth come, let them make their account in the greatness of their sufferings to pay the interest of that respect which hath been given them. Or if they chance to escape clearly in this world, which they seldom do; in the day when the heavens shall shrivel as a scroll and the mountains move as frighted men out of their places, what cave shall receive them? what mountain or rock shall they get by entreaty to fall upon them? what covert to hide them from that wrath, which they shall be neither able to abide nor to2 avoid? No man’s misery therefore being greater than theirs whose impiety is most fortunate; much more cause there is for them to bewail their own infelicity, than for others to be troubled with their prosperous and happy estate, as if the hand of the Almighty did not or would not touch them. For these causes and the like unto these therefore be not troubled.

Now though the cause of our heaviness be just, yet may not our affections herein be yielded unto with too much indulgency and favour. The grief of compassion whereby we are touched with the feeling of other men’s woes is of [647] all other least dangerous. Yet this is a let unto sundry duties; by this we are [apt?] to spare sometimes where we ought to strike. The grief which our own sufferings do bring, what temptations have not risen from it? What great advantage Satan hath taken even by the godly grief of hearty contrition for sins committed against God, the near approaching of so many afflicted souls, whom the conscience of sin hath brought unto the very brink of extreme despair, doth but too abundantly shew. These things wheresoever they fall cannot but trouble and molest the mind. Whether we be therefore moved vainly with that which seemeth hurtful and is not; or have just cause of grief, being pressed indeed with those things which are grievous, our Saviour’s lesson is, touching the one, Be not troubled, nor over-troubled for the other. For, though to have no feeling of that which merely concerneth us were stupidity, nevertheless, seeing that as the Author of our salvation was himself consecrated by affliction, so the way which we are to follow him by is not strewed with rushes1, but set with thorns, be it never so hard to learn, we must learn to suffer with patience even that which seemeth almost impossible to be suffered; that in the hour when God shall call us unto our trial, and turn this honey of peace and pleasure wherewith we swell into that gall and bitterness which flesh doth shrink to taste of, nothing may cause us in the troubles of our souls to storm and grudge and repine at God, but every heart be enabled with divinely inspired courage to inculcate unto itself, Be not troubled; and in those last and greatest conflicts to remember it, that nothing may be so sharp and bitter to be suffered, but that still we ourselves may give ourselves this encouragement, Even learn also patience, O my soul.

Naming patience I name that virtue which only hath power to stay our souls from being over-excessively troubled: a virtue, wherein if ever any, surely that soul had good experience, which extremity of pains having chased out of the tabernacle of this flesh, angels, I nothing doubt, have carried into [648] the bosom of her father Abraham. The death of the saints of God is precious in his sight. And shall it seem unto us superfluous at such times as these are to hear in what manner they have ended their lives? The Lord himself hath not disdained so exactly to register in the book of life after what sort his servants have closed up their days on earth, that he descendeth even to their very meanest actions, what meat they have longed for in their sickness, what they have spoken unto their children, kinsfolk, and friends, where they have willed their dead carcasses to be laid, how they have framed their wills and testaments, yea the very turning of their faces to this side or that, the setting of their eyes, the degrees whereby their natural heat hath departed from them, their cries, their groans, their pantings, breathings, and last gaspings, he hath most solemnly commended unto the memory of all generations. The care of the living both to live and to die well must needs be somewhat increased, when they know that their departure shall not be folded up in silence, but the ears of many be made acquainted with it. Again when they hear how mercifully God hath dealt with others in the hour of their last need, besides the praise which they give to God, and the joy which they have or should have by reason of their fellowship and communion of saints, is not their hope also much confirmed against the day of their own dissolution? Finally, the sound of these things doth not so pass the ears of them that are most loose and dissolute of life, but it causeth them sometime or other to wish in their hearts, “1Oh that we might die the death of the righteous, and that our end might be like his!” Howbeit because to spend herein many words would be to strike even as many wounds into their minds whom I rather wish to comfort: therefore concerning this virtuous gentlewoman only this little I speak, and that of knowledge, “She lived a dove, and died a lamb.” And if amongst so many virtues, hearty devotion towards God, towards poverty tender compassion, motherly affection towards servants, towards friends even serviceable kindness, mild behaviour and harmless meaning towards all; if, where so many virtues were eminent, any be worthy of special mention, I wish her dearest friends of that sex to be her nearest [649] followers in two things: Silence, saving only where duty did exact speech; and Patience even then when extremity of pains did enforce grief. “1Blessed are they which die in the Lord.” And concerning the dead which are blessed let not the hearts of any living be overcharged, with grief over-troubled.

Touching the latter affection of fear which respecteth evils to come, as the other which we have spoken of doth present evils; first in the nature thereof it is plain that we are not of every future evil afraid. Perceive we not how they whose tenderness shrinketh at the least rase of a needle’s point, do kiss the sword that pierceth their souls quite through? If every evil did cause fear, sin, because it is sin, would be feared; whereas properly sin is not feared as sin, but only as having some kind of harm annexed. To teach men to avoid sin, it had been sufficient for the Apostle to say, “Fly it2.” But to make them afraid of committing sin, because the naming of sin sufficed not, therefore he addeth further, that it is as a “serpent which stingeth the soul.” Again, be it that some nocive or hurtful thing be towards us, must fear of necessity follow hereupon? Not except that hurtful things do threaten us either with destruction or vexation, and that such as we have neither a conceit of ability to resist, nor of utter impossibility to avoid. That which we know ourselves able to withstand we fear not; and3 that which we know we are unable to defer or diminish, or any way avoid, we cease to fear, we give ourselves over to bear and sustain it. The evil therefore which is feared must be in our persuasion unable to be resisted when it cometh, yet not utterly impossible for a time in whole or in part to be shunned. Neither do we much fear such evils, except they be imminent and near at hand; nor if they be near, except we have an opinion that they be so. When we have once conceived an opinion or apprehended an imagination of such evils prest and ready to invade us; because they are hurtful unto our nature, we feel in ourselves a kind of abhorring; because they are, [650] though1 near yet not present, our nature seeketh forthwith how to shift and provide for itself; because they are evils which cannot be resisted, therefore she doth not provide to withstand but to shun and avoid. Hence it is that in extreme fear the mother of life contracting herself, avoiding as much as may be the reach of evil, and drawing the heat together with the spirits of the body to her, leaveth the outward parts cold, pale, weak, feeble, unapt to perform the functions of life; as we see in the fear of Belthasar king of Babel2. By this it appeareth that fear is nothing else but a perturbation of the mind through an opinion of some imminent evil threatening the destruction or great annoyance of our nature, which to shun it doth contract and deject itself.

Now because not in this place only but otherwhere often we hear it repeated, “Fear not,” it is by some made a long question, Whether a man may fear destruction or vexation without sinning? First, the reproof wherewith Christ checketh his disciples more than once, “O men of little faith, wherefore are ye afraid?” Secondly, the punishment threatened in the 21. of Revelations3, to wit, the lake, and fire, and brimstone, not only to murderers, unclean persons, sorcerers, idolaters, liars, but also to the fearful and faint-hearted: this seemeth to argue that fearfulness cannot but be sin. On the contrary side we see that he which never felt motion unto sin had of this affection more than a slight feeling. How clear is the evidence of the Spirit that “4in the days of his flesh he offered up prayers and supplications with strong cries and tears unto him that was able to save him from death, and was also heard in that which he feared!” Whereupon it followeth that fear in itself is a thing not sinful. For is not fear a thing natural and for men’s preservation necessary, implanted in us by the provident and most gracious Giver of all good things, to the end that we might not run headlong upon those mischiefs wherewith we are not able to encounter, but use the remedy of shunning those evils which we have not ability to withstand? Let that people therefore which receive a benefit by the length of their [651] prince’s days, that father or mother that rejoiceth to see the offspring of their flesh grow like green and pleasant plants, let those children that would have their parents, those men that would gladly have their friends and brethren’s days prolonged on earth, (as there is no natural-hearted man but gladly would,) let them bless the Father of lights, as in other things, so even in this, that he hath given man a fearful heart, and settled naturally that affection in him which is a preservation against so many ways of death. Fear then in itself being mere nature cannot in itself be sin, which sin is not nature, but thereof an accessary deprivation.

But in the matter of fear we may sin, and do, two ways. If any man’s danger be great, theirs greatest that have put the fear of danger farthest from them. Is there any estate more fearful than that Babylonian1 strumpet’s, that sitteth upon the tops of the2 seven hills glorying and vaunting, “3I am a queen?” &c. How much better and happier they whose estate hath been always as his who speaketh after this sort of himself, “Lord, from my youth have I borne thy yoke4!” They which sit at continual ease, and are settled in the lees of their security, look upon them, view their countenance, their speech, their gesture, their deeds: “Put them in fear, O God,” saith the Prophet, “that so they may know themselves to be but men5,” worms of the earth, dust and ashes, frail, corruptible, feeble things. To shake off security therefore, and to breed fear in the hearts of mortal men, so many admonitions are used concerning the power of evils which beset them, so many threatenings of calamities, so many descriptions of things threatened, and those so lively, to the end they may leave behind them a deep impression of such as have6 force to keep the heart continually waking. All which do shew, that we are to stand in fear of nothing more than the extremity of not fearing.

When fear hath delivered us from that pit wherein they are sunk that have put far from them the evil day, that have made a league with death and have said, “Tush, we shall feel [652] no harm;” it standeth us upon to take heed it cast us not into that wherein souls destitute of all hope are plunged. For our direction, to avoid as much as may be both extremities, that we may know as a ship-master by his card, how far we are wide, either on the one side or on the other, we must note that in a Christian man there is, first, Nature; secondly, Corruption, perverting Nature; thirdly, Grace correcting, and amending Corruption. In fear all these have their several operations. Nature teacheth simply, to wish preservation and avoidance of things dreadful; for which cause our Saviour himself prayeth, and that often, “1Father, if it be possible.” In which cases corrupt nature’s suggestions are, for the safety of temporal life not to stick at things excluding from eternal; wherein how far even the best may be led, the chiefest Apostle’s frailty teacheth. Were it not therefore for such cogitations as on the contrary side grace and faith ministereth, such as that of Job, “2Though God kill me;” that of Paul3, “Scio cui credidi, I know him on whom I do rely;” small evils would soon be able to overwhelm even the best of us. “A wise man,” saith Salomon4, “doth see a plague coming, and hideth himself.” It is nature which teacheth a wise man in fear to hide himself, but grace and faith doth teach him where. Fools care not where they hide their heads. But where shall a wise man hide himself when he feareth a plague coming? Where should the frighted child hide his head, but in the bosom of his loving father? Where a Christian, but under the shadow of the wings of Christ his Saviour? “Come, my people,” saith God in the Prophet5, “enter into thy chamber, hide thyself,” &c. But because we are in danger like chased birds, like doves that seek and cannot see the resting holes that are right before them, therefore our Saviour giveth his disciples these encouragements beforehand, that fear might never so amaze them, but that always they might remember, that whatsoever evils at any time did beset them, to him they should still repair, for comfort, counsel, and succour. For their assurance whereof his “peace he gave them, his peace he left unto them, not such peace as the [653] world offereth,” by whom his name is never so much pretended as when deepest treachery is meant; but “peace which passeth all understanding,” peace that bringeth with it all happiness, peace that continueth for ever and ever with them that have it.

This peace God the Father grant, for his Son’s sake; unto whom, with the Holy Ghost, three Persons, one eternal and everliving God, be all honour, glory, and praise, now and for ever. Amen.

[654]
DEDICATION PREFIXED TO THE FIRST EDITION OF TWO SERMONS ON PART OF ST. JUDE.
To the Worshipful M. George Summaster, Principal of Broad-Gates Hall, in Oxford, Henry Jackson wisheth all happiness.
Sir,
Jackson’s Dedication.YOUR kind acceptance of a former testification of that respect I owe you, hath made me venture to shew the world these godly sermons under your name. In which, as every point is worth observation, so some especially are to be noted. The first, that as the spirit of prophecy is from God himself, who doth inwardly heat and enlighten the hearts and minds of his holy penmen, (which if some would diligently consider, they would not puzzle themselves with the contentions of Scot and Thomas, Whether God only, or his ministering spirits, do infuse into men’s minds prophetical revelations “per species intelligibiles,”) so God framed their words also. Whence the holy father St. Augustine religiously observeth1, “That all those which understand the sacred writers, will also perceive that they ought not to use other words than they did, in expressing those heavenly mysteries which their hearts ‘conceived,’ as the blessed Virgin did our Saviour, ‘by the Holy Ghost.’ ” The greater is Castellio’sa offence,2 who hath laboured to teach the Prophets to speak otherwise than they have already. Much like to that impious king of Spain, Alphonsus the Tenthb, who found fault with God’s works3, “Si,” inquit, “creationi affuissem, mundum melius [655] ordinassem;” If he had been with God at the creation of the world, the world had gone better than now it doth. As this man found fault with God’s works, so did the other with God’s words; but, because “we have a most sure word of the Prophets1,” to which we must “take heed,” I will let his words pass with the wind, having elsewhere2 spoken to you more largely of his errors, whom, notwithstanding, for his other excellent parts, I much respect.

You shall moreover from hence understand, how Christianity consists not in formal and seeming “purity,” (under which who knows not notorious villainy to mask?) but in the heart-root. Whence the author truly teacheth, that mockers, which use religion as a cloak, to put off and on as the weather serveth, are worse than pagans and infidels. Where I cannot omit to shew how justly this kind of men hath been reproved by that renowned martyr of Jesus Christ, Bishop Latimer; both because it will be apposite to this purpose, and also free that Christian worthy from the slanderous reproaches of him3, who was, if ever any, a “mocker” of God, religion, and all good men. But first I must desire you, and in you all readers, not to think lightly of that excellent man, for usingc this and the like witty similitudes in his sermons. For whosoever will call to mind with what riff-raff God’s people were fed in those days, when their priests, “4whose lips should have preserved knowledge,” preached nothing else but dreams and false miracles of counterfeit saints, enrolled in that sottish Legend5, coined and amplified by a drowsy head between sleeping and waking: he that will consider this, and also how the [656] people were delighted with such toys, (God sending them strong delusions that they should believe lies,) and how hard it would have been for any man wholly, and upon the sudden, to draw their minds to another bent, will easily perceive, both how necessary it was to use symbolical discourse, and how wisely and moderately it was applied by that religious father, to the end he might lead their understanding so far, till it were so convinced, informed, and settled, that it might forget the means and way by which it was led, and think only of that it had acquired. For in all such mystical speeches, who knows not that the end for which they are used is only to be thought upon?

This then being first considered, let us hear the story, as it is related by Masterd Fox1: “Masterd Latimer,” saith he, “in his sermon [sermons], gave the people certain cards out of the fifth, sixth, and seventh chapters of Saint Matthew. For the chief triumph in the cards he limited the heart, as the principal thing that they should serve God withal, whereby he quite overthrew all hypocritical and external ceremonies, not tending to the necessary furtherance of God’s holy word and sacraments.” By this “he exhorted all men to serve the Lord with inward heart and true affection, and not with outward ceremonies; adding moreover to the praise of that triumph, that though it were never so small, yet it would take up the best coat-card beside in the bunch, yea, though it were the king of clubs, &c., meaning thereby, how the Lord would be worshipped and served in simplicity of the heart, and verity, wherein consisteth true Christian religion,” &c. Thus Masterd Fox.

By which it appears, that the holy man’s intention was to lift up the people’s hearts to God, and not that he made “a sermon of playing at cards, and taught them how to play at triumph, and played” (himself) “at cards in the pulpit,” as that base companion 2Parsons reports the matter in his wonted scurrilous vein of railing, whence he calleth it3 a Christmas sermon. Now he that will think ill of such allusions, may out of the abundance of his folly jest at 4Demosthenes for his story of the sheep, wolves, and dogs; and at Menenius, for his fiction of the belly5. But, hinc illæ lacrymæ, the good bishop meant that the Romish religion came not from the [657] heart, but consisted in outward ceremonies: which sorely grieved Parsons, who never had the least warmth or spark of honesty. Whether Bishop Latimer compared the bishops to the knaves of clubs, as the fellow interprets him, I know not: I am sure Parsons, of all others, deserved those colours; and so I leave him.

We see then, what inward purity is required of all Christians, which if they have, then in prayer, and all other Christian duties, they shall lift up pure hands, as the Apostle1 speaks, not as 2Baronius would have it, “washed from sins with holy water;” but pure, that is, holy, free from the pollution of sin, as the Greek word ὁσίους doth signify.

You may see also here refuted those calumnies of the papists, that we abandon all religious rites and godly duties; as also the confirmation of our doctrine touching certainty of faith, (and so of salvation,) which is so strongly denied by some of that faction, that they have told the world, “3St. Paul himself was uncertain of his own salvation.” What then shall we say, but pronounce a woe to the most strict observers of St. Francis’ rule and his canonical discipline, (though they make him even4 equal with Christ,) and the most meritorious monk that ever was registered in their

[658]
calendar of saints? But we for our comfort are otherwise taught out of the Holy Scripture, and therefore exhorted to build ourselves in our most holy faith, that so, “when our earthly house of this tabernacle shall be destroyed, we may have a building given of God, a house not made with hands, but eternal in the heavens1.”

This is that, which is most piously and feelingly taught in these few leaves, so that you shall read nothing here, but what, I persuade myself, you have long practised in the constant course of your life. It remaineth only, that you accept of these labours tendered to you by him, who wisheth you the long joys of this world, and the eternal of that which is to come.

Oxon. from Corp. Christi college, this 13 of January, 1613.

[659]
TWO SERMONS UPON PART OF ST. JUDE’S EPISTLE1.
SERMON I.
Epist. Jude, vers. 17-21.
But ye, beloved, remember the words which were spoken before of the Apostles of our Lord Jesus Christ:

How that they told you, that there should be mockers in the last time, which should walk after their own ungodly lusts.

These are makers of Sects, fleshly, having not the Spirit.

But ye, beloved, edify yourselves in your most holy faith, praying in the Holy Ghost.

And keep yourselves in the love of God, looking for the mercy of our Lord Jesus Christ, unto eternal life.

SERM. V. 1.THE occasion2 whereupon, together with the end wherefore, this Epistle was written, is opened in the front and entry of the same. There were then, as there are now, many evil and wickedly disposed persons, not of the mystical body, yet within the visible bounds of the Church, “men which were of old ordained to condemnation, ungodly men, which turned the grace of our God into wantonness, and denied the Lord Jesus.” For this cause the Spirit of the Lord is in the hand of “Jude the servant of Jesus and brother of James,” to exhort them that are called, and sanctified of God the Father, that they would earnestly “contend to maintain the faith, which was once delivered unto the saints.” Which faith because we cannot maintain, except we know perfectly, first, against whom; secondly, in what sort it must be maintained: therefore in the former three verses of that parcel of Scripture [660] which I have read, the enemiesa of the cross of Christ are plainly described;SERM. V. 2. and in the later two, they that love the Lord Jesus have a sweet lesson given them how to strengthen and stablish themselves in the faith. Let us first therefore examine the description of these reprobates concerning faith; and afterwards come to the words of the exhortation, wherein Christians are taught how to rest their hearts on God’s eternal and everlasting truth. The description of these godless persons is twofold, general and special. The general doth point them out, and shew what manner of men they should be. The particular pointeth at them, and saith plainly, these are they. In the general description we have to consider of these things; First, when they were described; “They were told of before:” Secondly, the men by whom they were described; “They were spoken of by the Apostles of our Lord Jesus Christ:” Thirdly, the days when they should be manifested unto the world; they told you they “should be in the last time:” Fourthly, their disposition and whole demeanour; Mockers and walkers after their own ungodly lusts.”

2. In the third to the Philippians1, the Apostle describeth certain; “They are men,” saith he, “of whom I have told you often, and now with tears I tell you of them, their god is their belly, their glorying and rejoicing is in their own shame, they mind earthly things.” These were enemiesa of the cross of Christ, enemiesa whom he saw, and his eyes gushed out with tears to behold them. But we are taught in this place how the Apostles spake also of enemiesa, whom as yet they had not seen, described a family of men as yet unheard of, a generation reserved for the end of the world, and for the last time; they had not only declared what they heard and saw in the days wherein they lived, but they have prophesied also of men in time to come. And “you do well,” saith St. Peter2, “in that ye take heed to the words of prophecy, so that ye first know this, that no prophecy in the Scripture cometh of any man’s own resolution.” No prophecy in Scripture cometh of any man’s own resolution. For all prophecy, which is in Scripture, came by the secret inspiration of God. But there are prophecies which are no Scripture; [661] yea, there are prophecies against the Scripture:SERM. V. 3, 4. my brethren, beware of such prophecies, and take heed you heed them not. Remember the things that were spoken of before; but spoken of before by the Apostles of our Lord and Saviour Jesus Christ. Take heed to prophecies, but to prophecies which are in Scripture; for both the manner and the matter of those prophecies do shew plainly that they are of God.

Of the spirit of prophecy received from God himself.3. Touching the manner, how men by the spirit of prophecy in holy Scripture have spoken and written of things to come, we must understand, that as the knowledge of that they spake, so likewise the utterance of that they knew, came not by these usual and ordinary means, whereby we are brought to understand the mysteries of our salvation, and are wont to instruct others in the same. For whatsoever we know, we have it by the hands and ministry of men, which lead us along like children, from a letter to a syllable, from a syllable to a word, from a word to a line, from a line to a sentence, from a sentence to a side, and so turn over. But God himself was their instructor, he himself taught them, partly by dreams and visions in the night, partly by revelations in the day, taking them aside from amongst their brethren, and talking with them as a man would talk with his neighbour in the way. Thus they became acquainted even with the secret and hidden counsels of God. They saw things which themselves were not able to utter, they beheld that whereat men and angels are astonished. They understood in the beginning, what should come to pass in the last days.

Of the Prophets’ manner of speech.4. God, which lightened thus the eyes of their understanding, giving them knowledge by unusual and extraordinary means, did also miraculously himself frame and fashion their words and writings; insomuch that a greater difference there seemeth not to be between the manner of their knowledge, than there is between the manner of their speech and ours. When we have conceived a thing in our hearts, and throughly understand it, as we think within ourselves, ere we can utter it in such sort that our brethren may receive instruction or comfort at our mouths, how great, how long, how earnest meditation are we forced to use! And after much travail and much pains, when we open our lips to speak of the wonderful works of God, our tongues do falter within our mouths, yea [662] many times we disgrace the dreadful mysteries of our faith, and grieve the spirit of our hearers by words unsavoury, and unseemly speeches:SERM. V. 4. “1Shall a wise man fill his belly with the eastern wind?” saith Eliphaz; “shall a wise man dispute with words not comely? or with talk that is not profitable?” Yet behold, even they that are wisest amongst us living, compared with the prophets, seem no otherwise to talk of God, than as if the children which are carried in arms should speak of the greatest matters of state. They whose words do most shew forth their wise understanding, and whose lips do utter the purest knowledge, so long as they understand and speak as men, are they not fain sundry ways to excuse themselves? Sometimes acknowledging with the wise man2, “Hardly can we discern the things that are on earth, and with great labour find we out the things that are before us; who can then seek out the things that are in heaven?” Sometimes confessing with Job the righteous, “intreating of things too wonderful for us, we have spoken we wist not what3.” Sometimes ending their talk, as doth the history of the Maccabees4: “If we have done well, and as the cause required, it is that we desire; if we have spoken slenderly and barely, we have done what we could.” But “God hath made my mouth like a sword,” saith Esay5. And “we have received,” saith the Apostle6, “not the spirit of the world, but the spirit which is of God, that we might know the things which are given to us of God; which things also we speak, not in words which man’s wisdom teacheth, but which the Holy Ghost doth teach.” This is that which the prophets mean by those books written full within and without; which books were so often delivered them to eat, not because God fed them with ink and paper, but to teach us, that so oft as he employed them in this heavenly work, they neither spake nor wrote any word of their own, but uttered syllable by syllable as the Spirit put it into their mouths, no otherwise than the harp or the lute doth give a sound according to the discretion of his hands that holdeth and striketh it with skill. The difference is only this: an instrument, [663] whether it be a pipe or harp, maketh a distinction in the times and sounds, which distinction is well perceived of the hearer, the instrument itself understanding not what is piped or harped.SERM. V. 5. The prophets and holy men of God not so. “I opened my mouth,” saith Ezekiel1, “and God reached me a scroll, saying, Son of man, cause thy belly to eat, and fill thy bowels with this I give thee. I ateb it, and it was sweet in my mouth as honey,” saith the prophet. Yea, sweeter, I am persuaded, than either honey or the honeycomb. For herein they were not like harps or lutes, but they felt, they felt the power and strength of their own words. When they spake of our peace, every corner of their hearts was filled with joy. When they prophesied of mournings, lamentations, and woes, to fall upon us, they wept in the bitterness and indignation of spirit2, the arm of the Lord being mighty and strong upon them.

5. On this manner were all the prophecies of holy Scripture. Which prophecies, although they contain nothing which is not profitable for our instruction, yet as one star differeth from another in glory, so every word of prophecy hath a treasure of matter in it, but all matters are not of like importance, as all treasures are not of equal price. The chief and principal matter of prophecy is the promise of righteousness, peace, holiness, glory, victory, immortality, unto “every soul which believeth that Jesus is Christ, of the Jew first, and of the Gentile3.” Now because the doctrine of salvation to be looked for by faith in Him, who was in outward appearance as it had been a man forsaken of God; in him who was numbered, judged, and condemned with the wicked; in him whom men did see buffeted on the face, scoffed at by soldiers, scourged by tormentors, hanged on the cross, pierced to the heart; in him whom the eyes of many witnesses did behold, when the anguish of his soul enforced him to roar as if his heart had rent in sunder4, “O my God, my God, why hast thou forsaken me?” I say, because the doctrine of salvation by him is a thing improbable to a natural man, that whether we preach it to the Gentile, or to the Jew, the one condemneth our faith as madness, the other as blasphemy; therefore, to establish and confirm the certainty of this saving truth in the [664] hearts of men,SERM. V. 6. the Lord, together with their preachings whom he sent immediately from himself to reveal these things unto the world, mingled prophecies of things both civil and ecclesiastical, which were to come in every age from time to time, till the very last of the latter days, that by those things, wherein we see daily their words fulfilled and done, we might have strong consolation in the hope of things which are not seen, because they have revealed as well the one as the other. For when many things are spoken of before in Scripture, whereof we see first one thing accomplished, and then another, and so a third, perceive we not plainly, that God doeth nothing else but lead us along by the hand, till he have settled us upon the rock of an assured hope, that no one jot or tittle of his word shall pass till all be fulfilled? It is not therefore said in vain, that these godless wicked ones “were spoken of before.”

6. But by whom? By them whose words if men or angels from heaven gainsay, they are accursed; by them whom whosoever despiseth, “despiseth not them but me1,” saith Christ. If any man therefore doth love the Lord Jesus, (and woe worth him that loveth not the Lord Jesus!) hereby we may know that he loveth him indeed, if he despise not the things that are spoken of by his Apostles, whom many have despised even for the baseness and simpleness of their persons.A natural man perceiveth not heavenly things. For it is the property of fleshly and carnal men to honour and dishonour, credit and discredit the words and deeds of every man, according to that he wanteth or hath without. “2If a man with gorgeous apparel come amongst us,” although he be a thief or a murderer, (for there are thieves and murderers in gorgeous apparel,) be his heart whatsoever, if his coat be of purple, or velvet, or tissue, every one riseth up, and all the reverend solemnities we can use are too little. But the man that serveth God is contemned and despised amongst us for his poverty. Herod speaketh in judgment, and the people cry out, “3The voice of God, and not of man.” Paul preacheth Christ, they term him a trifler. “4Hearken, beloved, hath not God chosen the poor of this world, that they should be rich in faith?” Hath he not chosen the refuse of the world to be heirs of his kingdom, which he hath promised to them that love him? [665] Hath he not chosen the offscourings of men to be the lights of the world, and the Apostles of Jesus Christ?SERM. V. 7. Men unlearned, yet how fully replenished with understanding? few in number, yet how great in power? contemptible in shew, yet in spirit how strong? how wonderful? “I would fain learn the mystery of the eternal generation of the Son of God,” saith Hilary1. “Whom shall I seek? Shall I get me to the schools of the Grecians? Why? I have read, Ubi sapiens? ubi scriba? ubi conquisitor hujus sæculi? These wise men in the world must needs be dumb in this, because they have rejected the wisdom of God. Shall I beseech the scribes and interpreters of the law to become my teachers? How can they know this, sith they are offended at the cross of Christ? It is death for me to be ignorant of the unsearchable mystery of the Son of God: of which mystery, notwithstanding I should have been ignorant, but that a poor fisherman, unknown, unlearned, new come from his boat with his clothes wringing wet, hath opened his mouth and taught me, ‘In the beginning was the Word, and the Word was with God, and the Word was God.’ ” These poor silly creatures have made us rich in the knowledge of the mysteries of Christ.

7. Remember therefore that which is spoken of by the Apostles. Whose words if the children of this world do not regard, is it any marvel? They are the Apostles of our Lord Jesus; not of their Lord, but of our. It is true which one [666] hath said in a certain place, Apostolicam fidem sæculi homo non capit. “A man sworn to the world is not capable of that faith which the Apostles do teach.” What mean the children of this world then to tread in the courts of our God?We must not halt between two opinions. What should your bodies do at Bethel, whose hearts are at Bethaven? The god of this world, whom ye serve, hath provided Apostles and teachers for you, Chaldeans, wizards, soothsayers, astrologers, and such like: hear them. Tell not us that ye will sacrifice to the Lord our God, if we will sacrifice to Ashtaroth or Melcom; that ye will read our Scriptures, if we will listen to your traditions; that if ye may have a Mass by permission, we shall have a Communion with good leave and liking; that ye will admit the things that are spoken of by the Apostles of our Lord Jesus, if your Lord and Master may have his ordinances observed, and his statutes kept. Salomonc took it (as well he might) for an evident proof, that she did not bear a motherly affection to her child, which yielded to have it cut in divers parts. He cannot love the Lord Jesus with his heart, which lendeth one ear to his Apostles, and anotherd to false apostles; which can brook to see a mingle-mangle1 of religion and superstition, Ministers and Massing-priests, light and darkness, truth and error, traditions and scriptures. No, we have no Lord but Jesus; no doctrine but the gospel; no teachers but his Apostles. Were it reason to require at the hands of an English subject, obedience to the laws and edicts of the Spaniard? I do marvel, that any man bearing the name of a servant of the servants of Jesus Christ, will go about to draw us from our allegiance. We are his sworn subjects; it is not lawful for us to hear the things that are not told us by his Apostles. They have told us, that in “the last days there shall be mockers,” therefore we believe it; Credimus quia legimus2, We are so persuaded, because we read it must be so. If we did not read it, we would not teach it: Nam quæ libro legis non continentur, ea [667] nec nosse debemus, saith Hilary1;SERM. V. 8, 9. “Those things that are not written in the book of the law, we ought not so much as to be acquainted with them.” “Remember the words which were spoken of before of the Apostles of our Lord Jesus Christ.”

Mockers in the last time.8. The third thing to be considered in the description of these men of whom we speak, is the time wherein they should be manifested to the world. They told you there should be mockers “in the last time.” Noah at the commandment of God built an ark, and there were in it beasts of all sorts, clean and unclean. A husbandman planteth a vineyard, and looketh for grapes, but when they come to the gathering, behold, together with grapes there are found also wild grapes. A rich man prepareth a great supper, and biddeth many; but when he sitteth him down, he findeth amongst his friends here and there a man whom he knoweth not. This hath been the state of the Church sithence the beginning. God always hath mingled his saints with faithless and godless persons; as it were the clean with the unclean, grapes with sour grapes, his friends and children with aliens and strangers. Marvel not then, if in the last days also ye see the men, with whom you live and walk arm in arm, laugh at your religion, and blaspheme that glorious name whereof you are called. Thus it was in the days of the patriarchs and prophets, and are we better than our fathers? Albeit we suppose that the blessed Apostles, in foreshewing what manner of men were set out for the last days, meant to note a calamity special and peculiar to the ages and generations which were to come. As if he should have said, as God hath appointed a time of seed for the sower, and a time of harvest for him that reapeth; as he hath given unto every herb and every tree his own fruit and his own season, not the season nor the fruit of another (for no man looketh to gather figs in the winter, because the summer is the season for them; nor grapes of thistles, because grapes are the fruit of the vine): so the same God hath appointed sundry for every generation of men, other men for other times, and for the last times the worst men, as may appear by their properties; which is the fourth point to be considered of in this description.

Mockers.9. “They told you that there should be mockers.” He [668] meaneth men that shall use religion as a cloak, to put off and on, as the weather serveth1;SERM. V. 9. such as shall with Herod hear the preaching of John Baptist to-day, and to-morrow condescend to have him beheaded; or with the other Herod say they will worship Christ, when they purpose a massacre in their hearts; kiss Christ with Judas, and betray Christ with Judas. These are mockers. For as Ishmael the son of Hagar laughed at Isaac, which was heir of the promise; so shall these men laugh at you as the maddest people under the sun, if ye be like Moses, “choosing rather to suffer affliction with the people of God, than to enjoy the pleasures of sin for a season.” And why? God hath not given them eyes to see, nor hearts to conceive that exceeding recompense of your reward. The promises of salvation made to you are matters wherein they can take no pleasure, even as Ishmael took no pleasure in that promise wherein God had said unto Abraham2, “In Isaac shall thy seed be called,” because the promise concerned not him, but Isaac. They are termed for their impiety towards God, “mockers;” and for the impurity of their life and conversation, “walkers after their own ungodly lusts.” St. Peter in his Second Epistle and third chapter soundeth the very depth of their impiety; shewing first, how they shall not shame at the length to profess themselves profane and irreligious, by flat denying the Gospel of Jesus Christ, and deriding the sweet and comfortable promises of his appearing: secondly, that they shall not be only deriders of all religion, but also disputers against God, using truth to subvert the truth; yea Scriptures themselves to disprove Scriptures. Being in this sort “mockers,” they must needs be also “followers of their own ungodly lusts.” Being atheists in persuasion, can they choose but be beasts in conversation? For why remove they quite from them the fear of God? Why take they such pains to abandon and put out from their hearts all sense, all taste, all feeling of religion? but only to this end and purpose, that they may without inward remorse and grudging of conscience give over themselves to all uncleanness.Mockers worse than Pagans and infidels. Surely the state of these men is more lamentable than is the condition of Pagans and Turks. For [669] at the bare beholding of heaven and earth the infidel’s heart by and by doth give him, that there is an eternal, infinite, immortal, and ever-living God, whose hands have fashioned and framed the world; he knoweth that every house is builded of some man, though he see not the man which built the house, and he considereth that it must be God which hath built and created all things; although because the number of his days be few, he could not see when God disposed his works of old, when he caused the light of his clouds first to shine, when he laid the corner stone of the earth, and swaddled it with bands of water and darkness; when he caused the morning star to know his place, and made bars and doors to shut up the sea within his house, saying1, “Hitherto shalt thou come, but no farther;” he hath no eyewitness of these things. Yet the light of natural reason hath put this wisdom in his reins, and hath given his heart thus much understanding. Bring a Pagan to the schools of the Prophets of God; prophesy to an infidel, rebuke him, lay the judgments of God before him, make the secret sins of his heart manifest, and he shall fall down and worship God. They that crucified the Lord of glory were not so far past recovery, but that the preaching of the Apostles was able to move their hearts and to bring them to this, “Men and brethren, what shall we do2?” Agrippa, that sat in judgment against Paul for preaching, yielded notwithstanding thus far unto him, “Almost thou persuadest me to become a Christian3.” Although the Jews for want of knowledge have not submitted themselves to the righteousness of God; yet “I bear them record,” saith the Apostle4, “that they have a zeal.” The Athenians, a people having neither zeal nor knowledge, yet of them also the same Apostle5 beareth witness, “Ye men of Athens, I perceive ye are δεισιδαιμονέστεστεροι, some way religious.” But mockers, walking after their own ungodly lusts, they have smothered every spark of that heavenly light, they have stifled even their very natural understanding. O Lord, thy mercy is over all thy works, thou savest man and beast! yet a happy case it had been for these men if they had never been born; and so I leave them.

[670]
SERM. V. 10, 11.10. St. Jude having his mind exercised in the doctrine of the Apostles of Jesus Christ, concerning things to come in the last time, became a man of a wise and staiede judgment. Grieved he was to see the departure of many, and their falling away from the faith which before they did profess;Judas vir sapiens et certi judicii. grieved, but not dismayed. With the simpler and weaker sort it was otherwise: their countenance began by and by to change, they were half in doubt they had deceived themselves in giving credit to the Gospel of Jesus Christ. St. Jude, to comfort and refresh these silly lambs, taketh them up in his arms, and sheweth them the men at whom they were offended. Look upon them that forsake this blessed profession wherein you stand: they are now before your eyes; view them, mark them, are they not carnal? are they not like to noisome carrion cast out upon the earth? is there that Spirit in them which crieth, “Abba, Father,” in your bosoms? Why should any man be discomforted? Have you not heard that there should be “mockers in the last time?” These verily are they that now do separate themselves.

11. For your better understanding what this severing and separating of themselves doth mean, we must know that the multitude of them which truly believe (howsoever they be dispersed far and wide each from other) is all one body, whereof the Head is Christ; one building, whereof he is the corner-stone, in whom they as the members of the body being knit, and as the stones of the building being coupled, grow up to a man of perfect stature, and rise to an holy temple in the Lord. That which linketh Christ to us, is his mere mercy and love towards us. That which tieth us to him, is our faith in the promised salvation revealed in the word of truth. That which uniteth and joineth us amongst ourselves, in such sort that we are now as if we had but one heart and one soul, is our love. Who be inwardly in heart the lively members of this body, and the polished stones of this building, coupled and joined to Christ, as flesh of his flesh, and bones of his bones, by the mutual bond of his unspeakable love towards them, and their unfeigned faith in him, thus linked and fastened each to other by a spiritual, sincere, and hearty affection of love, without any manner of simulation; [671] who be Jews within, and what their names be;SERM. V. 12. none can tell, save he whose eyes do behold the secret disposition of all men’s hearts. We, whose eyes are too dim to behold the inward man, must leave the secret judgment of every servant to his own Lord, accounting and using all men as brethren both near and dear unto us, supposing Christ to love them tenderly, so as they keep the profession of the Gospel, and join in the outward communion of saints. Whereof the one doth warrantize unto us their faith, the other their love, till they fall away, and forsake either the one, or the other, or both; and then it is no injury to term them as they are. When they separate themselves, they are αὐτοκατάκριτοι, not judged by us, but by their own doings. Men do separate themselves either by heresy, schism, or apostasy.Threefold separation. If they loose the bond of faith, which then they are justly supposed to do, when they frowardly oppugn any principal point of Christian doctrine, this is to separate themselves by heresy.1. Heresy. If they break the bond of unity, whereby the body of the Church is coupled and knit in one, as they do which wilfully forsake all external communion with saints in holy exercises purely and orderly established in the Church, this is to separate themselves by schism.2. Schism. If they willingly cast off and utterly forsake both profession of Christ and communion with Christians, taking their leave of all religion, this is to separate themselves by plain apostasy.3. Apostasy. And St. Jude, to express the manner of their departure which by apostasy fell away from the faith of Christ, saith, “They separated themselves;” noting thereby, that it was not constraint of others which forced them to depart, it was not infirmity and weakness in themselves, it was not fear of persecution to come upon them, whereat their hearts did fail; it was not grief of torments, whereof they had tasted, and were not able any longer to endure them. No, they voluntarily did separate themselves with a fully settled and altogether determined purpose never to name the Lord Jesus any more, nor to have any fellowship with his saints, but to bend all their counsel and all their strength to raze out their memorial from amongst men.

12. Now because that by such examples, not only the hearts of infidels were hardened against the truth, but the minds of weak brethren also much troubled, the Holy Ghost [672] hath given sentence of these backsliders, that they were carnal men, and had not the Spirit of Christ Jesus, lest any man having an overweening of their persons should be overmuch amazed and offended at their fall.SERM. V. 13. For simple men not able to discern their spirits, were brought by their apostasy thus to reason with themselves: If Christ be the Son of the living God, if he have the words of eternal life, if he be able to bring salvation to all men that come unto him, what meaneth this apostasy and unconstrained departure? Why do his servants so willingly forsake him? Babes, be not deceived, his servants forsake him not. They that separate themselves were amongst his servants, but if they had been of his servants, they had not separated themselves. “1They were amongst us, not of us,” saith St. John; and St. Jude proveth it, because they were carnal, and had not the Spirit. Will you judge of wheat by chaff which the wind hath scattered from amongst it? Have the children no bread because the dogs have not tasted it? Are Christians deceived of that salvation they looked for, because they denied the joys of the life to come which were no Christians? What if they seemed to be pillars and principal upholders of our faith? What is that to us, which know that Angels have2 fallen from heaven? Although if these men had been of us indeed (O the blessedness of a Christian man’s estate!), they had stood surer than the angels, they had never departed from their place. Whereas now we marvel not at their departure at all, neither are we prejudiced by their falling away; because they were not of us, sith they are fleshly, and have not the Spirit. Children abide in the house for ever; they are bondmen and bondwomen which are cast out.

13. It behoveth you therefore greatly every man to examine his own estate, and try whether you be bond or free, children or no children. I have told you already, that we must beware we presume not to sit as gods in judgment upon others, and rashly, as our conceit and fancy doth lead us, so to determine of this man, he is sincere, or of that man, he is an hypocrite; except by their falling away they make it manifest and known what they are. For who art thou that takest upon thee to judge another before the time? Judge thyself. God hath [673] left us infallible evidence, whereby we may at any time give true and righteous sentence upon ourselves.SERM. V. 14. We cannot examine the hearts of other men, we may our own.Infallible evidence in the faithful, that they are God’s children. “That we have passed from death to life, we know it,” saith St. John, “because we love our brethren1:” and, “Know ye not your own selves, how that Jesus Christ is in you, except ye be reprobates2?” I trust, beloved, we know that we are not reprobates, because our spirit doth bear us record, that the faith of our Lord Jesus Christ is in us.

14. It is as easy a matter for the spirit within you to tell whose ye are, as for the eyes of your body to judge where you sit, or in what place you stand. For what saith the Scripture? “Ye which were in times past strangers and enemiesf, because your minds were set on evil works, Christ hath now reconciled in the body of his flesh through death, to make you holy and unblamable and without fault in his sight; if you continue grounded and established in the faith, and be not moved away from the hope of the Gospel3.” And in the third to the Colossians, “Ye know, that of the Lord ye shall receive the reward of that inheritance; for ye serve the Lord Christ4.” If we can make this account with ourselves: I was in times past dead in trespasses and sins, I walked after the prince that ruleth in the air, and after the spirit that worketh in the children of disobedience; but God, who is rich in mercy, through his great love, wherewith he loved me, even when I was dead, hath quickened me in Christ. I was fierce, heady, proud, high-minded; but God hath made me like the child that is newly weanedg. I loved pleasures more than God; I followed greedily the joys of this present world; I esteemed him that erected a stage or theatre, more than Salomonh which built a temple to the Lord; the harp, viol, timbrel, and pipe, men-singers and women-singers, were at my feasts; it was my felicity to see my children dance before me5; I said of every kind of vanity, O how sweet art thou unto my soul! All which things now are crucified to me, and I to them: now I hate the pride of life, and pomp of this world: now “I take as great delight in the way of thy testimonies, [674] O Lord, as in all riches1;”SERM. V. 15. now I find more joy of heart in my Lord and Saviour, than the worldly-minded man, when “his wheat and oil do much abound;” now I taste nothing sweet but the “bread that came down from heaven, to give life unto the world2;” now mine eyes see nothing but Jesus rising from the dead; now my ear refuseth all kind of melody to hear the song of them that have gotten victory of the beast, and of his image, and of his mark, and of the number of his name, that stand on the sea of glass, “having the harps of God, and singing the song of Moses the servant of God, and the song of the Lamb, saying, Great and marvellous are thy works, Lord God Almighty, just and true are thy ways, O King of Saints3.” Surely, if the Spirit have been thus effectual in the secret work of our regeneration unto newness of life; if we endeavour thus to frame ourselves anew: then we may say boldly with the blessed Apostle in the tenth to the Hebrews, “We are not of them which withdraw ourselves to perdition, but which follow faith to the conservation of the soul4.” For they that fall away from the grace of God, and separate themselves unto perdition, they are fleshly and carnal, they have not God’s holy Spirit. But unto you, “because ye are sons, God hath sent forth the Spirit of his Son into your hearts5,” to the end ye might know that Christ hath built you upon a rock unmovable; that he hath registered your names in the Book of Life; that he hath bound himself in a sure and everlasting covenant to be your God, and the God of your children after you; that he hath suffered as much, groaned as oft, prayed as heartily for you, as for Peter, “O Father, keep them in thy name; O righteous Father, the world hath not known thee, but I have known thee, and these have known that thou hast sent me. I have declared thy name unto them, and will declare it, that the love wherewith thou hast loved me may be in them, and I in them6.” The Lord of his infinite mercy give us hearts plentifully fraught with the treasure of this blessed assurance of faith unto the end!

The papists falsely accuse us of heresy and apostasy.15. Here I must advertise all men, that have the testimony of God’s holy fear within their breasts, to consider how unkindly [675] and injuriously our own countrymen and brethren have dealt with us by the space of four and twenty years1, from time to time, as if we were the men of whom St. Jude here speaketh; never ceasing to charge us, some with schism, some with heresy, some with plain and manifest apostasy, as if we had clean separated ourselves from Christ, utterly forsaken God, quite abjured heaven, and trampled all truth and all religion under our feet. Against this third sort, God himself shall plead our cause in that day, when they shall answer us for these words, not we them. To others, by whom we are accused for schism and heresy, we have often made our reasonable, and in the sight of God, I trust, allowable answers. “For in the way which they call heresy, we worship the God of our fathers, believing all things which are written in the Law and the Prophets2.” That which they call schism, we know to be our reasonable service unto God, and obedience to his voice, which crieth shrill in our ears, “Go out of Babylon, my people, that you be not partakers of her sins, and that ye receive not of her plagues3.” And therefore when they rise up against us, having no quarrel but this, we need not seek any farther for our apology, than the words of Abiah to Jeroboam and his army: “O Jeroboam and Israel, hear you me: ought you not to know, that the Lord God of Israel hath given the kingdom over Israel to David for ever, even to him and to his sons, by a covenant of salt4?” that is to say, an everlasting covenant. Jesuits and papists, hear ye me: ought you not to know that the Father hath given all power unto the Son, and hath made him the only head over his Church, wherein he dwelleth as an husbandman in the midst of his vineyard, manuring it with the sweat of his own brows, not letting it forth to others? For, as it is in the Canticle, “Salomoni had a vineyard in Baalhamon, he gave the vineyard unto keepers, every one bringing for the fruit thereof a thousand pieces of silver5;” but my vineyard, which is mine, is before me, [676] saith Christ. It is true, this is meant of the mystical head set over the body, which is not seen. But as he hath reserved the mystical administration of the Church invisible unto himself, so he hath committed the mystical government of congregations visible, to the sons of David, by the same covenant; whose sons they are in the governing of the flock of Christ, whomsoever the Holy Ghost hath set over them, to go before them, and to lead them in their several pastures, one in this congregation, another in that; as it is written, “Take heed unto yourselves, and to all the flock whereof the Holy Ghost hath made you overseers, to feed the Church of God, which he hath purchased with his own blood1.”The pope’s usurped supremacy. Neither will ever any pope or papist under the cope of heaven be able to prove the Romish bishop’s usurped supremacy over all churches by any one word of the covenant of salt, which is the Scripture. For the children in our streets do now laugh them to scorn, when they force, “Thou art Peter,” to this purpose. The pope hath no more reason to draw the charter of his universal authority from hence, than the brethren had to gather by the words of Christ in the last of St. John, that the disciple whom Jesus loved should never die. “If I will that he tarry till I come, what is that to thee2?” saith Christ. Straightways a report was raised amongst the brethren, that this disciple should not die. Yet Jesus said not to him, he shall not die; but “if I will that he tarry till I come, what is that to thee?” Christ hath said in the sixteenth of St. Matthew’s Gospel to Simon the son of Jonas, “I say to thee, Thou art Peter3.” Hence an opinion is held in the world, that the pope is universal head of all churches. Yet Jesus said not, The pope is universal head of all churches; but, Tu es Petrus, “Thou art Peter.” Howbeit, as Jeroboam, the son of Nebat, the servant of Salomonk, rose up and rebelled against his Lord, and there were gathered unto him vain men and wicked, which made themselves strong against Roboaml, the son of Salomonk, because Roboam was but a child, and tenderhearted, and could not resist them; so the son of perdition and Man of Sin, (being not able to brook the words of our Lord and Saviour Jesus Christ, which forbade his disciples to [677] be like princes of nations, “They bear rule, and are called gracious, it shall not be so with you1,”) hath risen up and rebelled against his Lord; and, to strengthen his arm, he hath crept into the houses almost of all the noblest families round about him, and taken their children from the cradle to be his cardinals2; he hath fawned upon the kings and princes of the earth, and by spiritual cozenagem hath made them sell their lawful authority and jurisdiction for titles of Catholicus, Christianissimus, Defensor Fidei, and such like; he hath proclaimed sale of pardons, to inveigle the ignorant; built seminaries3, to allure young men desirous of learning; erected stews4, to gather the dissolute unto him. This is the rock whereupon his church is built. Hereby the Man is grown huge and strong, like the cedars which are not shaken with [678] the wind, because princes have been as children, over tenderhearted, and could not resist.

Hereby it is come to pass, as you see this day, that the Man of Sin doth war against us, not by men of a language which we cannot understand, but he cometh as Jeroboam against Judah, and bringeth the fruit of our own bodies to eat us up, that the bowels of the child may be made the mother’s grave, thatn hath caused no small number of our brethren to forsake their native country, and with all disloyalty to cast off the yoke of their allegiance to our dread Sovereign, whom God in mercy hath set over them; for whose safeguard, if they carried not the hearts of tigers in the bosoms of men, they would think the dearest blood in their bodies well spent. But now, saith Abiah to Jeroboam, “Ye think ye be able to resist the kingdom of the Lord, which is in the hands of the sons of David. Ye be a great multitude, the golden calves are with you, which Jeroboam made you for gods: have ye not driven away the priests of the Lord, the sons of Aaron, and the Levites, and have made you priests like the people of nations? whosoever cometh with a young bullock and seven rams, the same may be a priest of them that are no gods1.” If I should follow the comparison, and here uncover the cup of those deadly and uglyo abominations, wherewith this Jeroboam, of whom we speak, hath made the earth so drunk that it hath reeled under us, I know your godly hearts would loath to see them. For my own part, I delight not to rake in such filth, I had rather take a garment upon my shoulders, and go with my face from them to cover them. The Lord open their eyes, and cause them, if it be possible, at the length to see how they are wretched, and miserable, and poor, and blind, and naked. Put it, O Lord, in their hearts to seek white raiment, and to cover themselves, that their filthy nakedness may no longer appear. For, beloved in Christ, we bow our knees, and lift up our hands to heaven in our chambers secretly, and openly in our churches we pray heartily and hourly, even for them also: though the pope hathp given out as a judge, in a solemn declaratory sentence of excommunication against this land, that our gracious [679] Lady hath quite abolished prayers within her realm1; and his scholars, whom he hath taken from the midst of us, have in their published writings charged us not only not to have any holy assemblies unto the Lord for prayer, but to “hold a common school of sin and flattery; to hold sacrilege to be God’s service; unfaithfulness, and breach of promise to God, to give it to a strumpet, to be a virtue; to abandon fasting; to abhor confession; to mislike with penance; to like well of usury; to charge none with restitution; to find no good before God in single life, nor in no well-working;” . . . “that all men, as they fall to us, are much worsed, and more than afore corrupted.” I do not add one word or syllable unto that which Master Bristow2, a man both born and sworn amongst us, hath taught his hand to deliver3 to the view of all. I appeal to the conscience of every soul, that hath been truly converted by us, Whether his heart were never raised up to God by our preaching; whether the words of our exhortation never wrung any tear of a penitent heart from his eyes; whether his soul never reaped any joy, any [680] comfort, any consolation in Christ Jesus, by our sacraments, and prayers, and psalms, and thanksgiving; whether he were never bettered, but always worsed by us.

O merciful God! If heaven and earth in this case do not witness with us, and against them, let us be razed out from the land of the living! Let the earth on which we stand swallow us quick, as it hath done Corah, Dathan, and Abiram! But1 if we belong unto the Lord our God, and have not forsaken him; if our priests, the sons of Aaron, minister unto the Lord, and the Levites in their office; if we offer unto the Lord every morning and every evening the burnt-offerings and sweet incense of prayers and thanksgivings; if the bread be set in order upon the pure table, and the candlestick of gold, with the lamps thereof, to burn every morning; that is to say, if amongst us God’s blessed sacraments be duly administered, his holy word sincerely and daily preached; if we keep the watch of the Lord our God, and if ye have forsaken him: then doubt ye not, this God is with us as a captain, his priests with sounding trumpets must cry alarm against you; “O ye children of Israel, fight not against the Lord God of your fathers, for ye shall not prosper2.”

[681]
THE SECOND SERMON.
Epist. Jude, vers. 17-21.
But ye, beloved, remember the words which were spoken before of the apostles of our Lord Jesus Christ:

How that they told you, that there should be mockers in the last time, which should walk after their own ungodly lusts.

These are makers of Sects, fleshly, having not the Spirit.

But ye, beloved, edify yourselves in your most holy faith, praying in the Holy Ghost.

And keep yourselves in the love of God, looking for the mercy of our Lord Jesus Christ, unto eternal life.

SERM. VI. 1, 2.1. HAVING otherwhere spoken of the words of St. Jude, going next before, concerning Mockers, which should come in the last time, and backsliders, which even then fell away from the faith of our Lord and Saviour Jesus Christ; I am now, by the aid of Almighty God, and through the assistance of his good Spirit, to lay before you the words of exhortation which I have read.

2. Wherein first of all, whosoever hath an eye to see, let him open it, and he shall well perceive how careful the Lord is for his children, how desirous to see them profit and grow up to a manly stature in Christ, how lotha to have them any way misled, either by examples of the wicked, or by enticements of the world, and by provocation of the flesh, or by any other means forcible to deceive them, and likely to estrange their hearts from God. For God is not at that point with us, that he careth not whether we sink or swim. No, he hath written our names in the palm of his hand, in the signet upon his finger are we graven, in sentences not only of mercy, but of judgment also, we are remembered. He never denounceth judgments against the wicked, but he maketh some Proviso for his children, as it were for some certain privileged persons; [682] “1Touch not mine anointed, do my prophets no harm:SERM. VI. 3. Hurt not the earth, nor the sea, nor the trees, till we have sealed the servants of God in their foreheads.” He never speaketh of godless men, but he adjoineth words of comfort, or admonition, or exhortation, whereby we are moved to rest and settle our hearts on him. In the Second to Timothy, the third chapter2, “Evil men,” saith the Apostle, “and deceivers shall wax worse and worse, deceiving and being deceived. But continue thou in the things which thou hast learned.” And in the First to Timothy, the sixth chapter3, “Some men lusting after money, have erred from the faith, and pierced themselves through with many sorrows. But thou, O man of God, fly these things, and follow after righteousness, godliness, faith, love, patience, meekness.” In the Second to the Thessalonians, the second chapter4, “They that have not received the love of the truth, that they might be saved, God shall send them strong delusions, that they may believe lies. But we ought to give thanks alway to God for you, brethren, beloved of the Lord, because God hath from the beginning chosen you to salvation, through sanctification of the Spirit, and faith in the truth.” And in this Epistle of St. Jude, “There shall come mockers in the last time, walking after their own ungodly lusts. But, beloved, edify ye yourselves in your most holy faith.”

3. These sweet exhortations, which God putteth every where in the mouths of the prophets and apostles of Jesus Christ, are evident tokens, that God sitteth not in heaven careless and unmindful of our estate. Can a mother forget her child? Surely a mother will hardly forget her child. But if a mother be haply found unnatural, and do forget the fruit of her own womb; yet God’s judgments shew plainly, that he cannot forget the man whose heart he hath framed and fashioned anew in simplicity and truth to serve and fear him. For when the wickedness of man was so great, and the earth so filled with cruelty, that it could not stand with the righteousness of God any longer to forbear, wrathful sentences brake out from him, like wine from a vessel that hath no vent: “My Spirit,” saith he, “can struggle and strive no [683] longer; an end of all flesh is come before me.” Yet then did Noah find grace in the eyes of the Lord:SERM. VI. 4, 5. “1I will establish my covenant with thee,” saith God; “thou shalt go into the ark, thou, and thy sons, and thy wife, and thy sons’ wives with thee.”

4. Do we not see what shift God doth make for Lot and for his family, in the nineteenth of Genesis, lest the fiery destruction of the wicked should overtake him? Overnight the angels make inquiry, what sons and daughters, or sons-in-law, what wealth and substance he had. They charge him to carry out all, “2Whatsoever thou hast in the city, bring it out.” God seemeth to stand in a kind of fear, lest something or other would be left behind. And his will was, that nothing of that which he had, not a hoof of any beast, not a thread of any garment, should be singed with that fire. In the morning the angels fail not to call him up, and to hasten him forward; “3Arise, take thy wife and thy daughters which are here, that they be not destroyed in the punishment of the city.” The angels having spoken again and again, Lot for all this lingereth out the time still, till at the length they were forced to take “4both him, his wife, and his daughters, by the arms (the Lord being merciful unto him), and to carry them forth, and set them without the city.”

5. Was there ever any father thus careful to save his child from the flame? A man would think, that now being spoken unto to escape for his life, and not to look behind him, nor to tarry in the plain, but to hasten to the mountain, and there to save himself, he should do it gladly. Yet behold, now he is so far off from a cheerful and willing heart to do whatsoever is commanded him for his own weal, that he beginneth to reason the matter, as if God had mistaken one place for another, sending him to the hill, when salvation was in the city. “5Not so, my Lord, I beseeeh thee; behold, thy servant hath found grace in thy sight, and thou hast magnified thy mercy, which thou hast shewed unto me in saving my life. I cannot escape in the mountain, lest some evil take me and I die. Here is a city hard by, a small thing; O, let me escape thither, (is it not a small [684] thing?) and my soul shall live.” Well, God is contented to yield to any conditions.SERM. VI. 6, 7, 8. “1Behold, I have received thy request concerning this thing also, I will spare this city for which thou hast spoken; haste thee, save thee there. For I can do nothing till thou come thither.”

6. He could do nothing! Not because of the weakness of his strength (for who is like unto the Lord in power?) but because of the greatness of his mercy, which would not suffer him to lift up his arm against that city, nor to pour out his wrath upon that place, where his righteous servant had a fancy to remain, and a desire to dwell. O the depth of the riches of the mercy and love ofb God! God is afraid to offend us which are not afraid to displease him; God can do nothing till he have saved us, which can find in our hearts rather to do any thing than to serve him. It contenteth him not to exempt us when the pit is digged for the wicked; to comfort us at every mention which is made of reprobates and godless men; to save us as the apple of his own eye when fire cometh down from heaven to consume the inhabitants of the earth; except every prophet, and every Apostle, and every servant whom he sendeth forth, do come loaden with these and the like exhortations, “O beloved, edify yourselves in your most holy faith. Give yourselves to prayer in the Spirit, keep yourselves in the love of God. Look for the mercy of our Lord Jesus Christ unto eternal life.”

7. “Edify yourselves.” The speech is borrowed from material builders, and must be spiritually understood. It appeareth in the sixth of St. John’s Gospel by the Jews, that their mouths did water too much for bodily food: “2Our fathers,” say they, “did eat manna in the desert, as it is written, He gave them bread from heaven to eat; Lord, evermore give us of this bread.” Our Saviour, to turn their appetite another way, maketh them this answer: “3I am the Bread of Life; he that cometh to me shall not hunger; and he that believeth in me shall never thirst.”

8. An usual practice it is of Satan, to cast heaps of worldly baggage in our way, that whilst we desire to heap up gold as dust, we may be brought at the length to esteem vilely that [685] spiritual bliss.SERM. VI. 9. Christ, in the sixth of Matthew1, to correct this evil affection, putteth us in mind to lay up treasure for ourselves in heaven. The Apostle (1 Tim., third chapter), misliking the vanity of those women, which attired themselves more costly than beseemed the heavenly calling of such as professed the fear of God, willeth them to clothe themselves with 2shamefastness and modesty, and to put on the apparel of good works. Taliter pigmentatæ, Deum habebitis amatorem, 3saith Tertullian. Put on righteousness as a garment; instead of civetc, have faith, which may cause a savour of life to issue from you, and God shall be enamoured, he shall be ravished with your beauty. These are the ornaments, and bracelets, and jewels, which inflame the love of Christ, and set his heart on fire upon his spouse. We see how he breaketh out in the Canticles at the beholding of this attire: “4How fair art thou, and how pleasant art thou, O my love, in these pleasures!”

9. And perhaps St. Jude exhorteth us here not to build our houses, but ourselves, foreseeing by the Spirit of the Almighty which was with him, that there should be men in the last days like to those in the first, which should encourage and stir up each other to make brick, and to burn it in the fire, to build houses huge as cities, and towers as high as heaven, thereby to get them a name upon earth; men that should turn out the poor, and the fatherless, and the widow, to build places of rest for dogs and swine in their rooms; men that should lay houses of prayer even with thed ground, and make them stables where God’s people have worshipped before the Lord. Surely this is a vanity of all vanities, and it is much amongst men; a special sickness of this age. What it should mean I know not, except God have set them on work to provide fuel against that day, when the Lord Jesus shall shew himself from heaven with his mighty angels in flaming fire. What good cometh unto the owners of these things, saith Salomone, but only the beholding thereof with their eyes? “5Martha, Martha, thou busiest thyself about many things; one thing is necessary.” Ye are too busy, my brethren, with timber and brick; they [686] have chosen the better part, they have taken a better course, that build themselves.SERM. VI. 10. “1Ye are the temples of the living God, as God hath said, I will dwell in them, and will walk in them; and they shall be my people, and I will be their God.”

10. Which of you will gladly remain or abide in a mishapen, a ruinous, or a broken house? And shall we suffer sin and vanity to drop in at our eyes, and at our ears, at every corner of our bodies, and of our souls, knowing that we are the temples of the Holy Ghost? Which of you receiveth a guest whom he honoureth, or whom he loveth, and doth not sweep his chamber against his coming? And shall we suffer the chamber of our hearts and consciences to lie full of vomiting, full of filth, full of garbage, knowing that Christ hath said, “2I and my Father will come and dwell with you?” Is it meet for your oxen to lief in parlours, and yourselves to lodge in cribs? Or is it seemly for yourselves to dwell in your ceiled3 houses, and the house of the Almighty to lie waste, whose house ye are yourselves? Do not our eyes behold, how God every day overtaketh the wicked in their journeys, how suddenly they pop down into the pit? how God’s judgments for their crimes4 come so swiftly upon them, that they have not the leisure to cry, alas? how their life is cut off like a thread in a moment? how they pass like a shadow? how they open their mouths to speak, and God taketh them even in the midst of a vain or an idle word? and dare we for all this lief down, take our rest, eat our meat securely and carelessly in the midst of so great and so many ruins? Blessed and praised for ever and ever be his name, who perceiving of how senseless and heavy metal we are made, hath instituted in his Church a spiritual supper, and an holy communion to be celebrated often, that we might thereby be occasioned often to examine these buildings of ours, in what case they stand.The Sacrament of the Lord’s Supper. For sith God doth not dwell in temples which are unclean, sith a shrine cannot be a sanctuary unto him; and this supper is received as a seal unto us, that we are his house and his sanctuary; that his Christ is as truly united to me, and I to him, as my arm is united and knit unto my shoulder; that he dwelleth in me as [687] verily as the elements of bread and wine abide within me;SERM. VI. 11, 12. which persuasion, by receiving these dreadful mysteries, we 1profess ourselves to have, a due comfort, if truly; and if in hypocrisy, then woe worth us:—therefore ere we put forth our hands to take this blessed sacrament, we are charged to examine and to try our hearts whether God be in us of a truth or no: and if by faith and love unfeigned we be found the temples of the Holy Ghost, then to judge whether we have had such regard every one to our building, that the Spirit which dwelleth in us hath no way been vexed, molested, and grieved: or if it have, as no doubt sometimes it hath by incredulity, sometimes by breach of charity, sometimes by want of zeal, sometimes by spots of life, even in the best and most perfect amongst us: (for who can say, his heart is clean?) O then, to fly unto God by unfeigned repentance, to fall down before him in the humility of our souls, begging of him whatsoever is needful to repair our decays, before we fall into that desolation whereof the Prophet speaketh2, saying, “Thy breach is great like the sea, who can heal thee?”

11. Receiving the Sacrament of the Supper of the Lord after this sort (you that are spiritual, judge what I speak) is not all other wine like the water of Marah, being compared to the cup which we bless? Is not manna like to gall, and our bread like to manna? Is there not a taste, a taste of Christ Jesus, in the heart of him that eateth? Doth not he which drinketh behold plainly in this cup, that his soul is bathed in the blood of the Lamb? O beloved in our Lord and Saviour Jesus Christ, if ye will taste how sweet the Lord is, if ye will receive the King of Glory, “build yourselves.”

12. Young men, I speak this to you, for ye are his house, because by faith ye are conquerors over Satan, and have overcome that evil. Fathers, I speak it also to you; ye are his house, because ye have known him, which is from the beginning. Sweet babes, I speak it even to you also; ye are his house, because your sins are forgiven you for his name’s sakeg. Matrons and sisters, I may not hold it from you; ye are also [688] the Lord’s building, and, as St. Peter speaketh1, “heirs of the grace of life,” as well as we.SERM. VI. 13, 14, 15. Though it be forbidden you to open your mouths in public assemblies, yet ye must be inquisitive in things concerning this building which is of God, with your husbands and friends at home; not as Dalila with Samsonh, but as Sara with Abraham; whose daughters ye are, whilst ye do well, and build yourselves.

13. Having spoken thus far of the exhortation, as whereby we are called upon to edify and build ourselves; it remaineth now, that we consider the thing prescribed, namely, wherein we must be built. This prescription standeth also upon two points, the thing prescribed, and the adjuncts of the thing. And that is, our most pure and holy faith.

14. The thing prescribed is faith. For as in a chain, which is made of many links, if you pull the first, you draw the rest; and as in a ladder of many staves, if you take away the lowest, all hope of ascending to the highest will be removed: so, because all the precepts and promises in the law and in the Gospel do hang upon this, Believe; and because the last of the graces of God doth so follow the first, that he glorifieth none, but whom he hath justified, nor justifieth any, but whom he hath called to a true, effectual, and lively faith in Christ Jesus; therefore St. Jude exhorting us to build ourselves, mentioneth here expressly only faith, as the thing wherein we must be edified; for that faith is the ground and the glory of all the welfare of this building.

15. “Ye are not strangers and foreigners, but citizens with the saints, and of the household of God,” saith the Apostle2, and are built upon the foundation of the Prophets and Apostles, Jesus Christ himself being the chief cornerstone, in whom all the building being coupled together, groweth unto an holy temple in the Lord, in whom ye also are built together to be the habitation of God by the Spirit.” And we are the habitation of God by the Spirit, if we believe. For it is written3, “Whosoever confesseth that Jesus is the Son of God, in him God dwelleth, and he in God.” The strength of this habitation is great, it prevaileth against Satan, it conquereth sin, it hath death in derision; neither principalities [689] nor powers can throw it down;SERM. VI. 16. it leadeth the world captive, and bringeth every enemyi that riseth up against it to confusion and shame, and all by faith; for “this is the victory that overcometh the world, even our faith. Who is it that overcometh the world, but he which believeth that Jesus is the Son of God1?”

16. The strength of every building, which is of God, standeth not in any man’s arms or legs; it is only in our faith, as the valour of Samson lay only in his hair. This is the reason, why we are so earnestly called upon to edify ourselves in faith. Not as if this bare action of our minds, whereby we believe the Gospel of Christ, were able in itself, as of itself, to make us unconquerable, and invincible, like stones, which abide in the building for ever, and fall not out. No, it is not the worthiness of our believing, it is the virtue of him in whom we believe, by which we stand sure, as houses that are builded upon a rock. He is a wise man which hath builded his house upon a rock; for he hath chosen a good foundation, and no doubt his house will stand. But how shall it stand? Verily, by the strength of the rock which beareth it, and by nothing else2. Our fathers, whom God delivered out of the land of Egypt, were a people that had no peers amongst the nations of the earth, because they were built by faith upon the rock, which rock is Christ. “And the rock,” saith the Apostle in the First to the Corinthians, the tenth chapter3, “did follow themk.” Whereby we learn not only this, that being built by faith on Christ as on a rock, and grafted into him as into an olive, we receive all our strength and fatness from him; but also, that this strength and fatness of ours ought to be no cause why we should be highminded, and not work out our salvation with a reverentl trembling, and holy fear. For if thou boastest thyself of thy faith, know this, that Christ chose his Apostles, his Apostles chose not him; that Israel followed not the rock, but the rock followed Israel; and that thou bearest not the root, but the root thee4. So that every heart must this think, and every tongue must thus speak, “Not unto us, O Lord, not unto us,” nor unto any thing which is within us, but unto thy name only, only to thy name belongeth all the [690] praise of all the treasures and riches of every temple which is of God.SERM. VI. 17, 18, 19. This excludeth all boasting and vaunting of our faith.

17. But this must not make us careless to edify ourselves in faith. It is the Lord that delivereth men’s souls from death, but not except they put their trust in his mercy. It is God that hath given us eternal life, but no otherwise than thus, If we believe in the name of the Son of God; for he that hath not the Son of God, hath not life1. It was the Spirit of the Lord which came upon Samson, and made him strong to tear a lion, as a man would rent a kind; but his strength forsook him, and he became like other men when the razor had touched his head. It is the power of God whereby the faithful “have subdued kingdoms, wrought righteousness, obtained the promises, stopped the mouths of lions, quenched the violence of fire, escaped the edge of the sword2:” but take away their faith, and doth not their strength forsake them? are they not like unto other men?

18. If ye desire yet farther to know how necessary and needful it is that we edify and build up ourselves in faith, mark the words of the blessed Apostles3: “Without faith it is impossible to please God.” If I offer unto God all the sheep and oxen that are in the world; if all the temples that were builded since the days of Adam till this hour, were of my foundation; if I break my very heart with calling upon God, and wear out my tongue with preaching; if I sacrifice my body and my soul unto him, “and have no faith,” all this availeth nothing. “Without faith it is impossible to please God.”No pleasing of God without faith. Our Lord and Saviour therefore being asked in the sixth of St. John’s Gospel, “What shall we do that we might work the works of God?” maketh answer, “This is the work of God, that ye believe in him whom he hath sent4.”

19. That no work of ours, no building of ourselves in any thing can be available or profitable unto us, except we be edified and built in faith, what need we to seek about for long proof? Look upon Israel, once the very chosen and peculiar of God, to whom the adoption of the faithful, and the glory of Cherubins, and the covenants of mercy, and the law of Moses, and the service of God, and the promises of Christ [691] were made impropriate, who not only were the offspring of Abraham, father unto all them which do believe, but Christ their offspring, which is God to be blessed for evermore.SERM. VI. 20.

20. Consider this people, and learn what it is to build yourselves in faith. They were the Lord’s vine: “1He brought it out of Egypt, he threw out the heathen from their places, that it might be planted; he made room for it, and caused it to take root, till it had filled the earth; the mountains were covered with the shadow of it, and the boughs thereof were as the goodly cedars. She stretched out her branches unto the sea, and her boughs unto the river.” But, when God having sent both his servants and his Son to visit this vine, they neither spared the one, nor received the other, but stoned the prophets, and crucified the Lord of glory which came unto them; then began the curse of God to come upon them, even the curse whereof the prophet David hath spoken2, saying, “Let their table be made a snare, and a net, and a stumblingblock, even for a recompense unto them, let their eyes be darkened, that they do not see, bow down their backs for ever,” keep them down. And sithencem the hour that the measure of their infidelity was first made up, they have been spoiled with wars, eaten up with plagues, spent with hunger and famine; they wander from place to place, and are become the most base and contemptible people that are under the sun. Ephraim, which before was a terror unto nations, and they trembled at his voice, is now by infidelity so vile, that he seemeth as a thing cast out, to be trampled under men’s feet. In the midst of these desolations they cry, “3Return, we beseech thee, O God of hosts, look down from heaven, behold and visit this vine:” but their very prayers are turned into sin, and their cries are no better than the lowing of beasts before him. “Well,” saith the Apostle4, “by their unbelief they are broken off, and thou dost stand by thy faith. Behold therefore the bountifulness and severity of God; towards them severity, because they have fallen, bountifulness towards thee, if thou continue in his bountifulness, or else thou shalt be cut off.” If they [692] forsake their unbelief and be grafted in again, and we at any time for the hardness of our hearts be broken off, it will be such a judgment as will amaze all the powers and principalities which are above.SERM. VI. 21, 22. Who hath searched the counsel of God concerning this secret? and who doth not see, that Infidelity doth threaten Lo-ammi1 unto the Gentiles, as it hath brought Lo-ruchama2 upon the Jews? It may be that these words seem dark unto you. But the words of the Apostle, in the eleventh to the Romans, are plain enough; “3If God have not spared the natural branches, take heed, take heed, lest he spare not thee.” Build thyself in faith. Thus much of the thing which is prescribed, and wherein we are exhorted to edify ourselves. Now consider the conditions and properties which are in this place annexed unto faith. The former of them (for there are but two) is this, Edify yourselves in your faith.

21. A strange and a strong delusion it is wherewith the Man of Sin hath bewitched the world; a forcible spirit of error it must needs be, which hath brought men to such a senseless and unreasonable persuasion as this is, not only that men clothed with mortality and sin, as we ourselves are, can do God so much service, as shall be able to make a full and perfect satisfaction before the tribunal seat of God for their own sins, yea a great deal more than is sufficient for themselves; but also that a man at the hands of a bishop or a pope, for such or such a price, may buy the overplus of other men’s merits, purchase the fruits of other men’s labours, and build his soul by another man’s faith. Is not this man drowned in the gall of bitterness? Is his heart right in the sight of God? Can he have any part or fellowship with Peter, and with the successors of Peter, who thinketh so vilely of building the precious temples of the Holy Ghost? Let his money perish with him, and he with it, because he judgeth that the gift of God may be sold for money.

22. But, beloved in the Lord, deceive not yourselves, neither suffer ye yourselves to be deceived: ye can receive no more ease nor comfort for your souls by another man’s faith, than warmth for your bodies by another man’s clothes, [693] or sustenance by the bread which another doth eat.SERM VI. 23, 24, 25. The just shall live by his own faith. “Let a saint, yea a martyr content himself, that he hath cleansed himself of his own sins1,” saith Tertullian. No saint or martyr can cleanse himself of his own sins. But if so be a saint or a martyr can cleanse himself of his own sins, it is sufficient that he can do it for himself. Did ever any man by his death deliver another man from death, except only the Son of God? He indeed was able to safe-conduct2 a thief from the cross to paradise: for to this end he came, that being himself pure from sin, he might obey for sinners. Thou which thinkest to do the like, and supposest that thou canst justify another by thy righteousness, if thou be without sin, then lay down thy life for thy brother; die for me. But if thou be a sinner, even as I am a sinner, how can the oil of thy lamp be sufficient both for thee and for me? Virgins that are wise, get ye oil, while ye have day, into your own lamps. For out of all peradventure, others, though they would, can neither give nor sell. Edify yourselves in your own most holy faith. And let this be observed for the first property of that wherein we ought to edify ourselves.

23. Our faith being such, is that indeed which St. Jude doth here term faith: namely, a thing most holy. The reason is this; we are justified by faith: for Abraham believed, and this was imputed unto him for righteousness. Being justified, all our iniquities are covered; God beholdeth us in the righteousness which is imputed, and not in the sins which we have committed.

24. It is true we are full of sin, both original and actual; whosoever denieth it is a double sinner, for he is both a sinner and a liar. To deny sin, is most plainly and clearly to prove it; because he that saith he hath no sin, lieth, and by lying proveth that he hath sin.

25. But imputation of righteousness hath covered the sins of every soul which believeth; God by pardoning our sin hath taken it away: so that now, although our transgressions be multiplied above the hairs of our head, yet being justified, [694] we are as free and as clear as if there were no one spot or stain of any uncleanness in us.SERM. VI. 26, 27, 28. For it is God that justifieth; “and who shall lay any thing to the charge of God’s chosen?” saith the Apostle in the eighth chapter to the Romans.

26. Now sin being taken away, we are made the righteousness of God in Christ. For David speaking of this righteousness, saith1, “Blessed is the man whose iniquities are forgiven.” No man is blessed, but in the righteousness of God: every man whose sin is taken away is blessed; therefore every man whose sin is covered, is made the righteousness of God in Christ. This righteousness doth make us to appear most holy, most pure, most unblamable before him.

27. This then is the sum of that which I say: faith doth justify; justification washeth away sin; sin removed, we are clothed with the righteousness which is of God; the righteousness of God maketh us most holy. Every of these I have proved by the testimony of God’s own mouth. Therefore I conclude, that faith is that which maketh us most holy; in consideration whereof, it is called in this place, “Our most holy faith.”

28. To make a wicked and a sinful man most holy through his believing, is more than to create a world of nothing. Our faith most holy! Surely, Salomonn could not shew the queen of Saba so much treasure in all his kingdom, as is lapt up in these words. O that our hearts were stretched out like tents, and that the eyes of our understanding were as bright as the sun, that we might throughly know the riches of the glorious inheritance of saints, and what is the exceeding greatness of his power towards us, whom he accepteth for pure, and most holy, through our believing! O that the Spirit of the Lord would give this doctrine entrance into the stony and brazen heart of the Jew2, which followeth the law of righteousness, but cannot attain unto the righteousness of the law! Wherefore? saith the Apostle. They seek righteousness, and not by faith. Wherefore they stumble at Christ, they are bruised, shivered to pieces as a ship that hath run herself upon a rock. O that God would cast down the eyes of the proud, and [695] humble the souls of the high-minded,SERM. VI. 29. that they might at the length abhor the garments of their own flesh, which cannot hide their nakedness, and put on the faith of Christ Jesus, as he did put it on, which hath said, “1Doubtless I think all things but loss, for the excellent knowledge sake of Christ Jesus my Lord, for whom I have counted all things loss, and do judge them to be dung, that I might win Christ, and might be found in him, not having mine own righteousness, which is of the law, but that which is through the faith of Christ, even the righteousness which is of God through faith.” O that God would open the ark of mercy, wherein this doctrine lieth, and set it wide before the eyes of poor afflicted consciences, which fly up and down upon the water of their afflictions, and can see nothing but only the gulf and deluge of their sins, wherein there is no place for them to rest their feet. The God of pity and compassion give you all strength and courage, every day, and every hour, and every moment, to build and edify yourselves in this most pure and holy faith. And thus much both of the thing prescribed in this exhortation, and also of the properties of the thing, “Build yourselves in your most holy faith.” I would come to the next branch, which is of prayer; but I cannot lay this matter out of my hands, till I have added somewhat for the applying of it both to others and to ourselves.

29. For your better understanding of matters contained in this exhortation, “Build yourselves,” you must note, that every church and congregation doth consist of a multitude of believers, as every house is built of many stones. And although the nature of the mystical body of the church be such, that it suffereth no distinction in the invisible members, but whether it be Paul or Apollos, prince or prophet, he that is taught, or he that teacheth, all are equally Christ’s, and Christ is equally theirs: yet in the external administration of the church of God, because God is not the author of confusion, but of peace, it is necessary that in every congregation there be a distinction, if not of inward dignity, yet of outward degree; so that all are saints, or seem to be saints, and should be as they seem. But are all Apostles? If the [696] whole body were an eye, where were then the hearing?SERM. VI. 30, 31. God therefore hath given some to be Apostles, and some to be Pastors, &c. for the edification of the body of Christ. In which work we are God’s labourers, saith the Apostle, and ye are God’s husbandry, and God’s building.

30. The Church, respected with reference unto administration ecclesiastical, doth generally consist but of two sorts of men, the labourers and the building; they which are ministered unto, and they to whom the work of the ministry is committed; pastors, and the flock over whom the Holy Ghost hath made them overscers. If the guide of a congregation, be his name or his degree whatsoever, be diligent in his vocation, feeding the flock of God which dependeth upon him, caring for it, “1not by constraint, but willingly; not for filthy lucre, but of a ready mind;” not as though he would tyrannize over God’s heritage, but as a pattern unto the flock, wisely guiding them: if the people in their degree do yield themselves framable to the truth, not like rough stone or flint, refusing to be smoothed and squared for the building: if the magistrate do carefully and diligently survey the whole order of the work, providing by statutes and laws, and bodily punishments, if need require, that all things may be done according to the rule which cannot deceive, even as Moses provided that all things might be done according to the pattern which he saw in the Mount; there the words of this exhortation are truly and effectually heard. Of such a congregation every man will say, “Behold a people that are wise, a people that walk in the statutes and ordinances of their God, a people full of knowledge and understanding, a people that have skill in building themselves.” Where it is otherwise, there, “as by slothfulness the roof doth decay;” and as by “idleness of hands the house droppeth thorougho,” as it is in the tenth of Ecclesiastes, verse 18, so first one piece, and then another of their building shall fall away, till there be not a stone left upon a stone.

31. We see how fruitless this exhortation hath been to such as bend all their travail only to build and manage a Papacy upon earth, without any care in the world of building themselves [697] in their most holy faith.SERM. VI. 32, 33. God’s people have inquired at their mouths, “What shall we do to have eternallyp life?” Wherein shall we build and edify ourselves? And they have departed home from their prophets, and from their priests, laden with doctrines which are precepts of men; they have been taught to tire out themselves with bodily exercise: those things are enjoined them, which God did never require at their hands, and the things he doth require are kept from them; their eyes are fed with pictures, and their ears filled with melody, but their souls do wither, and starve, and pine away: they cry for bread, and behold stones are offered them; they ask for fish, and see they have scorpions in their hands. Thou seest, O Lord, that they build themselves, but not in faith; they feed their children, but not with food: their rulers say with shame, Bring, and not build. But God is righteous; their drunkenness stinketh, their abominations are known, their madness is manifest, the wind hath bound them up in her wings, and they shall be ashamed of their doings. “1Ephraim,” saith the Prophet, “is joined to idols, let him alone.” I will turn me, therefore, from the priests, which do minister unto idols, and apply this exhortation to them whom God hath appointed to feed his chosen in Israel.

32. If there be any feeling of Christ, and drop of heavenly dew, or any spark of God’s good Spirit within you, stir it up, be careful to build and edify, first yourselves, and then your flocks, in this most holy faith.

33. I say, first yourselves; for, he which will set the hearts of other men on fire with the love of Christ, must himself burn with love. It is want of faith in ourselves, my brethren, which maketh us retchless2 in building others. We forsake the Lord’s inheritance, and feed it not. What is the reason of this? Our own desires are settled where they should not be. We ourselves are like those women which have a longing to eat coals, and lime, and filth; we are fed, some with honour, some with ease, some with wealth; the gospel waxeth lothsomeq and unpleasant in our taste; how should we then have a care to feed others with that which we cannot fancy [698] ourselves?SERM. V. 34. If faith wax cold and slender in the heart of the prophet, it will soon perish from the ears of the people. The Prophet Amos speaketh of a famine, saying, “1I will send a famine in the land, not a famine of bread, nor a thirst of water, but of hearing the word of the Lord. Men shall wander from sea to sea, and from the north unto the east shall they run to and fro, to seek the word of the Lord, and shall not find it.” “2Judgment must begin at the house of God,” saith Peter. Yea, I say, at the sanctuary of God this judgment must begin. This famine must begin at the heart of the prophet. He must have darkness for a vision, he must stumble at noon-day3, as at the twilight, and then truth shall fall in the midst4 of the streets; then shall the people wander from sea to sea, and from the north unto the east shall they run to and fro, to seek the word of the Lord.

34. In the second of Haggai, “5Speak now,” saith God to his prophet, “speak now to Zerubbabel, the son of Shealtiel, prince of Judah, and to Jehoshua, the son of Jehozadak the high priest, and to the residue of the people, saying, Who is left among you that saw this house in her first glory, and how do you see it now? Is not this house in your eyes, in comparison of it, as nothing?” The prophet would have all men’s eyes turned to the view of themselves, every sort brought to the consideration of their present state. This is no place to shew what duty Zerubbabel or Jehoshua doth owe unto God in this respect. They have, I doubt not, such as put them hereof in remembrance. I ask of you, which are a part of the residue of God’s elect and chosen people, Who is there amongst you that hath taken a survey of the house of God, as it was in the days of the blessed Apostles of Jesus Christ? Who is there amongst you that hath seen and considered this holy temple in her first glory? And how do you see it now? Is it not in comparison of the other almost as nothing6? When ye look upon them that have undertaken the charge of your souls, and know how far these are for the most part grown out of kind, how few there be that tread the steps of their ancient predecessors, ye are [699] easily filled with indignation, easily drawn unto these complaints, wherein the difference of present from former times is bewailed, easily persuaded to think of them that lived to enjoy the days which now are gone1, “Surely they were happy in comparison of us that have succeeded them:SERM. VI. 34. were not their bishops men unreprovable, wise, righteous, holy, temperate, well reported of, even of those which were without? Were not their pastors, guides, and teachers, able and willing to exhort with wholesome doctrine, and to improve2 which gainsaid the truth? had they priests made of the refuse of the people? were men, like to the children which were in Ninevehr, unable to discern between the right hand and the left, presented to the charge of their congregation? did their teachers leave their flocks, over which the Holy Ghost had made them overseers? did their prophets enter upon holy things as spoils, without a reverend calling? were their leaders so unkindly affected towards them, that they could find in their hearts to sell them as sheep or oxen, not caring how they made them away?” But, beloved, deceive not yourselves. Do the faults of your guides and pastors offend you? It is your fault if they be thus faulty. Nullus, qui malum rectorem patitur, eum accuset; quia sui fuit meriti perversi pastoris subjacere ditioni, saith St. Gregory3; “Whosoever thou art whom the inconvenience of an evil governor doth press, accuse thyself, and not him: his being such is thy deserving.” “4O ye disobedient children, turn again,” saith the Lord, “and then will I give you pastors according to mine own heart, which shall feed you with knowledge and understanding.” So that the only way to repair all ruins, breaches, and offensive decays, in others, is to begin reformation at yourselves. Which that we may all sincerely, seriously, and speedily do, God the Father grant for his Son our Saviour Jesus’ sake, unto whom, with the Holy Ghost, three Persons, one eternal and everlasting God, be honour, and glory, and praise, for ever. Amen.

[700]
A SERMON, FOUND AMONG THE PAPERS OF BISHOP ANDREWS.
Matth. vii. 7, 8.
Ask, and it shall be given you; seek, and you shall find; knock, and it shall be opened unto you. For whosoever asketh, &c.

SERM. VII.AS all the creatures of God, which attain their highest perfection by process of time, are in their first beginning raw; so man, in the end of his race the perfectest, is at his entrance thereunto the weakest, and thereby longer enforced to continue a subject for other men’s compassions to work upon voluntarily, without any other persuader, besides their own secret inclination, moving them to repay to the common stock of humanity such help, as they know that themselves before must needs have borrowed; the state and condition of all flesh being herein alike. It cometh hereby to pass, that although there be in us, when we enter into this present world, no conceit or apprehension of our own misery, and for a long time after no ability, as much as to crave help or succour at other men’s hands; yet through his most good and gracious providence, which feedeth the young, even of feathered fowls and ravens, (whose natural significations of their necessities are therefore termed in Scripture “prayers and invocations1” which God doth hear), we amongst them, whom he values at a far higher rate than millions of brute creatures, do find by perpetual experience daily occasions given unto every of us, religiously to acknowledge with the [701] Prophet David1, “Thou, O Lord, from our birth hast been merciful unto us,” we have tasted thy goodness, hanging even at our mothers’ breasts.SERM. VII. 1. That God, which during infancy preserveth us without our knowledge, teacheth us at years of discretion how to use our own abilities for procurement of our own good.

“Ask, and it shall be given you; seek, and you shall find; knock, and it shall be opened unto you.” For whosoever doth ask, shall receive; whosoever doth seek, shall find; the door unto every one which knocks shall be opened.

In which words we are first commanded to “ask,” “seek,” and “knock:” secondly, promised grace answerable unto every of these endeavours; asking, we shall have; seeking, we shall find; knocking, it shall be opened unto us: thirdly, this grace is particularly warranted, because it is generally here averred, that no man asking, seeking, and knocking, shall fail of that whereunto his serious desire tendeth.

1. Of asking or praying I shall not need to tell you, either at whose hands we must seek our aid, or to put you in mind that our hearts are those golden censers from which the fume of this sacred incense must ascend. For concerning the one, you know who it is which hath said, “Call upon me2;” and of the other, we may very well think, that if any where, surely first and most of all in our prayers, God doth make his continual claim, Fili, da mihi cor tuum3, Son let me never fail in this duty to have thy heart.

Against invocation of any other than God alone, if all arguments else should fail, the number whereof is both great and forcible, yet this very bar and single challenge might suffice; that whereas God hath in Scripture delivered us so many patterns for imitation when we pray, yea, framed ready to our hands in a manner all, for suits and supplications, which our condition of life on earth may at any time need, there is not one, no not one to be found, directed unto angels, saints, or any, saving God alone. So that, if in such cases as this we hold it safest to be led by the best examples that have gone before, when we see what Noah, what Abraham, what Moses, what David, what Daniel, and the rest did; what form of prayer Christ himself likewise taught his Church, [702] and what his blessed Apostles did practise; who can doubt but the way for us to pray so as we may undoubtedly be accepted, is by conforming our prayers to theirs, whose supplications we know were acceptable?

Whoso cometh unto God with a gift, must bring with him a cheerful heart, because he loveth hilarem datorem1, a liberal and frank affection in giving. Devotion and fervency addeth unto prayers the same that alacrity doth unto gifts; it putteth vigour and life in them. Prayer proceedeth from want, which being seriously laid to heart, maketh suppliants always importunate; which importunity our Saviour Christ did not only tolerate in the woman of Canaan (Matth. xv.), but also invite and exhort thereunto, as the parable of the wicked judge sheweth (Luke xiii).

Our fervency sheweth us sincerely affected towards that we crave: but that which must make us capable thereof, is an humble spirit; for God doth load with his grace the lowly, when the proud he sendeth empty away: and therefore to the end that all generations of the world might know how much it standeth them upon to beware of all lofty and vain conceits when we offer up our supplications before him, he hath in the Gospel both delivered this caveat, and left it by a special chosen parable exemplified. 2The Pharisee and publican having presented themselves in one and the same place, the temple of God, for performance of one and the same duty, the duty of prayer, did notwithstanding, in that respect only, so far differ the one from the other, that our Lord’s own verdict of them remaineth as (you know) on record, “They departed home,” the sinful publican, through humility of prayer, just; the just Pharisee, through pride, sinful. So much better doth he accept of a contrite peccavi, than of an arrogant Deo gratias.

Asking is very easy, if that were all God did require: but because there were means which his providence hath appointed for our attainment unto that which we have from him, and those means now and then intricated, such as require deliberation, study, and intention of wit; therefore he which emboldeneth to ask, doth after invocation exact inquisition; a work of difficulty. The baits of sin every where open, ready [703] always to offer themselves; whereas that which is precious, being hid, is not had but by being sought. Præmia non ad magna pervenitur nisi per magnos labores, Bernard: straitness and roughness are qualities incident unto every good and perfect way. What booteth it to others that we wish them well, and do nothing for them? As little ourselves it must needs avail, if we pray and seek not. To trust to labour without prayer, it argueth impiety and profaneness; it maketh light of the providence of God: and although it be not the intent of a religious mind, yet it is the fault of those men whose religion wanteth light of mature judgment to direct it, when we join with our prayer slothfulness and neglect of convenient labour. He which hath said, “If any man lack wisdom, let him ask”—hath in like sort commanded also to seek wisdom, to search for understanding as for treasure. To them which did only crave a seat in the kingdom of Christ, his answer, as you know, in the Gospel, was this1; To sit at my right hand and left hand in the seat of glory is not a matter of common gratuity, but of Divine assignment from God. He liked better of him which inquired, “2Lord, what shall I do that I may be saved?” and therefore him he directeth the right and ready way, “Keep the commandments.”

I noted before unto you certain special qualities belonging unto you that ask: in them that seek there are the like: [in] which we may observe it is with many as with them of whom the Apostle speaketh3, they “are alway learning, and never able to come to the knowledge of the truth.” Ex amore non quærunt, saith Bernard; they seek because they are curious to know, and not as men desirous to obey. It was distress and perplexity of mind which made them inquisitive, of whom St. Luke in the Acts4 reporteth, that sought counsel and advice with urgent solicitation; Men and brethren, sith God hath blessed you with the spirit of understanding above others, hide not from miserable persons that which may do them good; give your counsel to them that need and crave it at your hands, unless we be utterly forlorn; shew us, teach us, what we may do and live. That which our Saviour doth say of prayer in the open streets, of causing trumpets to be [704] blown before us when we give our alms, and of making our service of God a means to purchase the praise of men, must here be applied to you, who never seek what they ought, but only when they may be sure to have store of lookers on. “On my bed,” saith the Canticles1, “there did I seek whom my soul doth love.” When therefore thou resolvest thyself to seek, go not out of thy chamber into the streets, but shun that frequency which distracteth; single thyself from thyself, if such sequestration may be attained. When thou seekest, let the love of obedience, the sense and feeling of thy necessity, the eye of singleness and sincere meaning guide thy footsteps, and thou canst not slide.

You see what it is to ask and seek; the next is “knock.” There is always in every good thing which we ask, and which we seek, some main wall, some barred gate, some strong impediment or other objecting itself in the way between us and home; for removal whereof, the help of stronger hands than our own is necessary. As therefore asking hath relation to the want of good things desired, and seeking to the natural ordinary means of attainment thereunto; so knocking is required in regard of hindrances, lets, or impediments, which are doors shut up against us, till such time as it please the goodness of Almighty God to set them open: in the mean while our duty here required is to knock. Many are well contented to ask, and not unwilling to undertake some pains in seeking; but when once they see impediments which flesh and blood doth judge invincible, their hearts are broken. Israel in Egypt, subject to miseries of intolerable servitude, craved with sighs and tears deliverance from that estate, which then they were fully persuaded they could not possibly change, but it must needs be for the better. Being set at liberty, to seek the land which God had promised unto their fathers did not seem tedious or irksome unto them: this labour and travel they undertook with great alacrity, never troubled with any doubt, nor dismayed with any fear, till at the length they came to knock at those brazen gates, the bars whereof, as they have no means, so they had no hopes, to break asunder. Mountains on this hand, and the roaring sea before their faces; [705] then all the forces that Egypt could make, coming with as much rage and fury as could possess the heart of a proud, potent, and cruel tyrant: in these straits, at this instant, Oh, that we had been so happy as to die where before we lived a life, though toilsome, yet free from such extremities as now we are fallen into! Is this the milk and honey that hath been so spoken of? Is this the paradise in description whereof so much glosing and deceiving eloquence hath been spent? Have we after four hundred and thirty years left Egypt to come to this? While they are in the midst of their mutinous cogitations, Moses with all instancy beateth, and God with the hand of his omnipotency casteth open the gates before them, maugre even their own infidelity and despair. It was not strange then; nor that they afterward stood in like repining terms: for till they came to the very brink of the river Jordan, the least cross accident, which lay at any time in their way, was evermore unto them a cause of present recidivation and relapse. They having the land in their possession, being seated in the heart thereof, and all their hardest encounters past, Joshua and the better sort of their governors, who saw the wonders which God had wrought for the good of that people, had no sooner ended their days, but first one tribe, then another, in the end all, delighted in ease; fearful to hazard themselves in following the conduct of God, weary of passing so many strait and narrow gates, [they] condescended to ignominious conditions of peace, joined hands with infidels, forsook Him which had been always the Rock of their salvation, and so had none to open unto them, although their occasions of knocking were great afterward, moe and greater than before. Concerning Issachar, the words of Jacob, the father of all the patriarchs, were these; “Issachar, though bonny and strong enough unto any labour, doth couch notwithstanding as an ass under all burdens; he shall think with himself that rest is good, and the land pleasant; he shall in these considerations rather endure the burden and yoke of tribute, than cast himself into hazard of war1.” We are for the most part all of Issachar’s disposition, we account ease cheap, howsoever we buy it. And although we can happily [i.e. haply] frame ourselves [706] sometimes to ask, or endure for a while to seek;SERM. VII. 2. yet loth we are to follow a course of life, which shall too often hem us about with those perplexities, the dangers whereof are manifestly great.

But of the duties here prescribed of asking, seeking, knocking, thus much may suffice. The promises follow which God hath made.

2. “Ask and receive, seek and find, knock and it shall be opened unto you.” Promises are made of good things to come; and such, while they are in expectation, have a kind of painfulness with them; but when the time of performance and of present fruition cometh, it bringeth joy.

Abraham did somewhat rejoice in that which he saw would come, although knowing that many ages and generations must first pass: their exultation far greater, who beheld with their eyes, and embraced in their arms, Him which had been before the hope of the whole world. We have found that Messias; have seen the salvation: “Behold here the Lamb of God, which taketh away the sins of the world1.” These are speeches of men not comforted with the hope of that they desire, but rapt with admiration at the view of enjoyed bliss.

As oft therefore as our case is the same with the prophet David’s; or that experience of God’s abundant mercy towards us doth wrest from our mouths the same acknowledgments which it did from his, “I called on the name of the Lord, and he hath rescued his servant: I was in misery, and he saved me: Thou, Lord, hast delivered my soul from death, mine eyes from tears, and my feet from falling2:” I have asked and received, sought and found, knocked and it hath been opened unto me: can there less be expected at our hands, than to take the Cup of Salvation, and bless, magnify, and extol the mercies heaped upon the heads of the sons of men? Are we in the case of them, who as yet do only ask and have not received? It is but attendance a small time, we shall rejoice then; but how? we shall find, but where? it shall be opened, but with what hand? To all which demands I must answer.

[707]
Use the words of our Saviour Christ;SERM. VII. 3. Quid hoc ad te1? what are these things unto us? Is it for us to be made acquainted with the way he hath to bring his counsel and purposes about? God will not have great things brought to pass, either altogether without means, or by those means altogether which are to our seeming probable and likely. Not without means, lest under colour of repose in God we should nourish at any time in ourselves idleness: not by the mere ability of means gathered together through our own providence, lest prevailing by helps which the common course of nature yieldeth, we should offer the sacrifice of thanksgiving for whatsoever prey we take to the nets which our fingers did weave2; than which there cannot be to Him more intolerable injury offered. Vere et absque dubio, saith St. Bernard, hoc quisque est pessimus, quo optimus, si hoc ipsum quo est optimus adscribat sibi3; the more blest, the more curst, if we make his graces our own glory, without imputation of all to him; whatsoever we have we steal, and the multiplication of God’s favours doth but aggravate the crime of our sacrilege. He, knowing how prone we are to unthankfulness in this kind, tempereth accordingly the means, whereby it is his pleasure to do us good. This is the reason why God would neither have Gideon to conquer without an army, nor yet to be furnished with too great an host. This is the cause why, as none of the promises of God do fail, so the most are in such sort brought to pass, that, if we after consider the circuit, wherein the steps of his providence have gone, the due consideration thereof cannot choose but draw from us the selfsame words of astonishment, which the blessed Apostle hath: “O the depth of the riches of the wisdom of God! how unsearchable are his counsels, and his ways past finding out4!” Let it therefore content us always to have his word for an absolute warrant; we shall receive and find in the end; it shall at length be opened unto you: however, or by what means, leave it to God.

3. Now our Lord groundeth every man’s particular assurance touching this point upon the general rule and axiom of [708] his providence, which hath ordained these effects to flow and issue out of these causes; gifts of suits, finding out of seeking, help out of knocking: a principle so generally true, that on his part it never faileth.

For why? it is the glory of God to give; his very nature delighteth in it; his mercies in the current, through which they would pass, may be dried up, but at the head they never fail. Men are soon weary both of granting and of hearing suits, because our own insufficiency maketh us still afraid, lest by benefiting of others we impoverish ourselves. We read of large and great proffers, which princes in their fond and vain-glorious moods have poured forth: as that of Herod; and the like of Ahasuerus in the Book of Hester. “Ask what thou wilt, though it reach to the half of my kingdom, I will give it thee1:” which very words of profusion do argue, that the ocean of no estate in this world doth so flow, but it may be emptied. He that promiseth half of his kingdom, foreseeth how that being gone, the remainder is but a moiety of that which was. What we give we leave; but what God bestoweth benefiteth us, and from him it taketh nothing: wherefore in his propositions there are no such fearful restraints; his terms are general in regard of making, “Whatsoever ye ask the Father in my name2;” and general also in respect of persons, “whosoever asketh, whosoever seeketh.” It is true, St. James saith3, “Ye ask, and yet ye receive not, because you ask amiss;” ye crave to the end ye might have to spend upon your lusts. The rich man sought heaven, but it was then, when he felt hell. The virgins knocked in vain, because they overslipped their opportunity; and when the time was to knock, they slept: but Quærite Dominum dum inveniri potest4, perform these duties in their due time and due sort. Let there, on our part, be no stop, and the bounty of God we know is such, that he granteth over and above our desires. Saul sought an ass, and found a kingdom. Solomon named wisdom, and God gave Solomon wealth also, by way of surpassing. “Thou hast prevented thy servant with blessings5,” saith the prophet David. [709] “He asked life, and thou gavest him long life, even for ever and ever.” God a giver; “He giveth liberally, and upbraideth none in any wise1:” and therefore he better knoweth than we the best times, and the best means, and the best things, wherein the good of our souls consisteth.

the end.
[710][711]
I.

INDEX OF TEXTS.

II.

INDEX OF AUTHORITIES QUOTED.

III.

INDEX OF PRINCIPAL MATTERS.

IV.

GLOSSARY OF WORDS.

GENERAL INDEX OF TEXTS.
asterisks Texts omitted in the following Index are those in the Preface, which are cited by Calvin as mere adaptations. Pref. ii. 3, 6. Three following Rom. xvi. 16 in Pref. iv. 4, which are only verbal parallels to that. Three in Pref. viii. 1, which are adaptations by a Barrowist. Three in Pref. viii. 6, 7, 8, cited by Anabaptists, and two in ib. 12, cited by the same. The Texts marked with an asterisk are either allusions or adaptations or mere illustrations, or else quoted by opponents, the real meaning of which is given elsewhere. The large numerals refer to the book, the smaller to the chapters, and the Arabic figures to the section, of the Ecclesiastical Polity. Other abbreviations used are the following: Ded., Hooker’s Dedication to Whitgift; Suppl., Travers’ Supplication to the Council; Answ., Hooker’s Answer to Travers; Pref., Hooker’s Preface to the Eccl. Polity; Ed. Pref., Editor’s Preface; Jack. Ded., Jackson’s Dedication to “Sermons on S. Jude.”

GENESIS.
i. 29.	I. x. 2.
ii. 17.	I. x. 2.
ii. 18.	I. ii. 3.
ii. 19.*	VII. ii. 2.
ii. 20.	I. x. 12.
iii. 8.	V. xi. 1.
iii. 15.	I. iv. 3.
iv. 2. 26.	I. x. 2.
iv. 3.	V. xi. 1.
iv. 4.	V. lxxix. 2.
iv. 8.	I. x. 3.
iv. 20-22.	I. x. 2.
v.*	I. x. 3.
vi. 3, 13. 8, 18.	Serm. VI. 3.
vi. 5.	I. x. 3.
vi. 5.	V. App. N°. 1. 1.
xiii. 4.	V. xi. 1.
xiv. 20.	V. lxxix. 7.
xvii. 14.	V. lxiv. 4.
xvii. 17.	Serm. I.
xviii. 12.	Serm. I.
xviii. 18.*	III. xi. 9.
xviii. 19.	I. x. 2.
xviii. 25.	Serm. III.
xix. 12, 15, 16.	Serm. VI. 4.
xix. 15.	Serm. ii. 10.
xix. 22.*	VI. vi. 2.
xix. 18-22.	Serm. VI. 5.
xxi. 12.	Serm. V. 9.
xxi. 33.	V. xi. 1.
xxii. 1.	V. xi. 1.
xxiv. 2.	IV. i. 3.
xxviii. 20.	V. lxxix. 7.
xxxvii. 7.*	VIII. ix. 3.
xxxix. 9.	I. viii. 8.
xlvii. 22.	VII. xxiii. 4.
xlvii. 22.	VII. xxiv. 17.
xlviii. 14.	V. lxvi. 1.
xlix.	Serm. II. 23.
xlix. 10.	III. xi. 9.
xlix. 14, 15.	Serm. VII. 1.
EXODUS.
i. 12.	V. App. N°. 1. 42.
iii. 2.	V. lvii. 3.
iii. 5.	V. lxix. 3.
iv. 21.	V. App. N°. 1. 42.
iv. 24.	V. lx. 7.
iv. 24, 25.	V. lxii. 21.
ix. 34.	V. App. N°. 1. 42.
x. 1.	V. App. N°. 1. 42.
x. 24.*	V. xix. 3.
xiii. 3.	V. lxxi. 5.
xv. 1, 21.	V. xxxix. 3.
xv. 20.*	V. xli. 1.
xvii. 12.*	IV. xiv. 7.
xviii. 19.*	VII. v. 2.
xviii. 25, 26.	VI. App.
xix. 6.	VIII. iii. 6.
xix. 8.*	V. xxxix. 1.
xxi. 6.	IV. i. 3.
xxii. 1.*	III. x. 3.
xxii. 29, 30.	V. lxxix. 13.
xxiii. 18.	V. xlii. 7.
xxiv. 3.*	V. xxxix. 1.
xxiv. 4.	I. xiii. 1.
xxiv. 7.	V. xxii. 6.
xxiv. 12.	I. xiii. 1.
xxv. 28 (39?)	V. lxxix. 5.
xxvi.	V. xi. 1.
xxvii. 3.	V. xx. 2.
xxviii. 2.*	V. xxix. 5.
xxviii. 4, 43.	II. iv. 4.
xxx. 25, 32.	V. xx. 2.
xxx. 26-28.	V. xx. 2.
xxxii.*	III. i. 8.
xxxii.	V. lxxii. 5.
xxxii. 4.*	V. lxv. 16.
xxxii. 32.	VI. App.
xxxvi. 5-7.*	VII. xxiv. 25.
xxxvii. 24.	V. lxxix. 5.
xxxix.*	II. iv. 4.
xxxix. 27.*	V. xxix. 5.
xl. 9.	V. xii. 3.
xl. 15.	V. xx. 2.
xl. 34.	V. xii. 3.
LEVITICUS.
v. 5.	VI. iv. 4.
vi. 2.	VI. v. 7.
x. 1.*	V. lxii. 13.
xi.	II. iv. 4.
xi.*	IV. vi. 2.
xi.	IV. vi. 3.
xvi.	V. lxxii. 5.
xvi. 2.	V. xii. 4.
xvi. 21.	VI. iv. 4.
xviii.	IV. xi. 7.
xviii. 3.*	IV. vi. 2.
xviii. 3.	IV. xi. 3.
xviii. 21.*	II. vi. 2.
xix. 19.*	IV. vi. 2.
xix. 19.	IV. vi. 3.
xix. 27.*	IV. vi. 2.
xix. 27, 28.	IV. vi. 3.
xix. 32.	VII. xvii. 2.
xx. 3.*	II. vi. 2.
xxi. 1.	V. lxxv. 4.
xxi. 5.	IV. vi. 3.
xxiii.	V. lxxii. 5.
xxiv. 12.	III. xi. 8.
xxv. 33, 34.	VII. xxiii. 4.
xxv. 34.	V. lxxix. 6.
xxv. 34.	VII. xxiv. 20.
xxvi. 2.	V. xii. 5.
xxvii. 11, 14.	VII. xxiii. 4.
xxvii. 25.	V. lxxix. 10.
xxvii. 28.	V. lxxix. 6.
NUMBERS.
iii. 32.	VII. vi. 6.
iv. 27.	VII. vi. 6.
v. 6.	VI. iv. 4.
v. 8.	VI. v. 7.
vi. 23.	V. xxv. 3.
vi. 23.	V. xxvi. 2.
vii. 85, 86.	V. lxxix. 5.
ix.	III. xi. 8.
ix. 13.*	V. lxviii. 1.
x. 2.	V. xx. 2.
xi. 17.	V. lxxvii. 8.
xi. 25.	VI. App.
xi. 33.	V. xlviii. 3.
xii. 14.	VI. v. 4.
xiv.	V. lxxii. 5.
xiv. 22.	VI. v. 4.
xv. 32.	V. lxxi. 8.
xv. 33-35.	III. xi. 8.
xvi. 3.	VI. i. 4.
xvi. 3.*	VII. xvii. 1.
xvi. 10.*	V. lxii. 13.*
xvii. 8.*	V. iii. 4.
xviii. 8-28.	VII. xxiii. 4.
xviii. 19, 24.	VII. xxiii. 5.
xviii. 24-28.	VII. xxiii. 1.
xviii. 32.	VII. xxiv. 20.
xx. 12.	VI. v. 4.
xxi. 8.	V. lx. 4.
xxii. 28.*	V. ix. 1.
xxiii. 10.	V. xlvi. 2.
xxiii. 10.	Serm. IV.
xxiii. 19.*	II. vi. 1.
xxvii.	III. xi. 8.
xxvii. 18.	V. lxvi. 1.
xxvii. 21.*	II. vi. 3.
xxxi. 48-54.	VII. xxiii. 1.
xxxv.	V. lxxix. 6.
xxxv. 7.	VII. xxiii. 4.
DEUTERONOMY.
iv. 2.	III. v. 1.
iv. 2.	VIII. vi. 5.
iv. 5.	III. xi. 6.
iv. 6.*	VII. xviii. 7.
iv. 12, 14.	III. xi. 6.
v. 22.	III. xi. 6.
v. 27.	III. xi. 6.
v. 27.*	V. xxxix. 1.
v. 28-31.	III. xi. 6.
vi. 5.	I. viii. 7.
vii. 15.	V. xlviii. 6.
x. 9.	VII. xxiii. 5.
xii. 2.*	V. xvii. 1.
xii. 2.	V. xvii. 5.
xii. 4, 5.	V. xvii. 5.
xii. 5-7.	V. xi. 1.
xii. 32.	III. v. 1.
xii. 32.	VIII. vi. 5.
xiii.	Serm. III.
xiv. 1.	IV. vi. 3.
xiv. 7.*	IV. vi. 2.
xiv. 7.	IV. vi. 3.
xiv. 22.	V. lxxix. 7.
xvi. 14.	V. lxx. 3.
xvi. 18.	VI. App.
xvii. 2.	VIII. ix. 3.
xvii. 5.	VIII. ix. 3.
xvii. 8.	Pref. vi. 2.
xvii. 8-13.	VI. App.
xvii. 12.	Pref. vi. 4.
xviii. 8.	VII. xxiii. 4.
xviii. 10.*	II. vi. 2.
xix. 15.	II. vii. 2.
xxii. 10, 11.*	III. x. 1.
xxii. 11.*	IV. vi. 2.
xxii. 11.	IV. vi. 3.
xxvi. 17.*	V. xxxix. 1.
xxviii.	Serm. III.
xxviii. 20.	V. xvii. 2.
xxviii. 21, 22, 27.*	V. lxxvi. 4.
xxx. 9.	V. xlviii. 6.
xxx. 15.	V. App. N°. 1. 5.
xxx. 19.	I. vii. 2.
xxx. 19.	Serm. III.
xxxi. 11-13.	V. xxii. 4.
xxxi. 11-13.	V. xxii. 13.
xxxii. 7.	V. vii. 1.
xxxiii.	V. xlvi. 1.
xxxiii. 11.*	VII. xxiv. 26.
JOSHUA.
i. 5.	Serm. I.
i. 18.	VIII. viii. 3.
vii. 19.	V. xlii. 7.
vii. 19.	VI. iv. 4.
ix. 14.	II. vi. 3.
xiii. 14.	VII. xxiii. 5.
xiv. 4.	VII. xxiii. 4.
xxii. 10.	III. xi. 15.
xxii. 10.*	V. lxv. 16.
xxiv.	V. xlvi. 1.
xxiv. 15.*	III. i. 10.
xxiv. 16.*	V. xxxix. 1.
JUDGES.
v. 23.*	Serm. II. 38.
vi. 13.	V. xvii. 2.
xi. 40.	III. xi. 15.
xx. 26.	V. lxxii. 5.
RUTH.
iv. 7.	IV. i. 3.
1 SAMUEL.
i. 19 (9-18).	V. lxxxi. 2.
vii.*	V. lxxii. 5.
viii. 7.	V. xlviii. 3.
xii. 23.*	V. xxiii.
xiii. 11.*	V. lxii. 13.
xiii. 14.	V. i. 4.
xv. 15.*	V. xvii. 1.
xvii. 39.*	III. viii. 4.
xxiii. 11, 12.	V. App. N°. 1.
xxxi. 13.	V. lxxii. 5.
2 SAMUEL.
i. 19.	V. lxxv. 3.
vi. 2.*	V. xli. 1.
vi. 6.*	V. lxii. 13.
vii. 2.	IV. ii. 4.
vii. 2-5.*	Trav. Sup.
xii. 6.*	III. x. 3.
xii. 13.	VI. vi. 8.
xii. 14.	IV. xii. 2.
xii. 14.	VI. v. 4.
xv. 30.*	V. lxxv. 2.
xvi. 12.*	VII. xxiv. 16.
1 KINGS.
ii.	V. xlvi. 1.
iii. 11.*	V. xxxv. 2.
iii. 14.	Serm. III.
iv. 29, 30.*	III. viii. 9.
viii.	V. xii. 4.
viii. 1.	VIII. v. 1.
viii. 11.	V. xii. 3.
x. 1.*	I. x. 12.
xi. 7.	V. lxv. 17.
xix. 18.*	III. i. 8.
2 KINGS.
v. 11.	V. lxvi. 1.
v. 13.	V. lx. 4.
vii. 2.	Serm. I.
xviii. 3, 4.	V. lxv. 12.
xviii. 3, 6.	V. lxv. 17.
xviii. 4.*	III. i. 8.
xxii. 2.	V. lxv. 17.
xxii. 17.*	III. i. 8.
xxiii. 7.*	V. lxv. 19.
xxiii. 13.	V. lxv. 17.
1 CHRONICLES.
x. 12.	V. lxxii. 5.
xiii. 5.*	V. xli. 1.
xv. 3, 4.	VIII. v. 1.
xvii. 2.	II. vi. 3.
xvii. 6.	II. vi. 3.
xix. 2.	V. lxxv. 3.
xxi. 1.	I. iv. 3.
xxii. 14.	V. xi. 1.
xxiii. 3.	VII. xxiii. 4.
xxiii. 31.	V. lxx. 5.
xxviii. 14.	V. xv. 4.
xxix. 2, 3, 6, 9, 14.	V. xv. 4.
xxix. 3.	VII. xxii. 4.
xxix. 3.	VIII. i. 5.
xxix. 3, 4.	V. xi. 1.
xxix. 17.	V. vi. 1.
xxix. 17, 18.	V. xi. 2.
xxix. 18.	V. i. 4.
xxxix. 2-7.	V. lxxix. 5.
2 CHRONICLES.
ii. 5.	IV. ii. 4.
ii. 5.	V. vi. 2.
ii. 5.	V. xv. 4.
iii. 1.	V. xi. 1.
iv. 3.*	V. lxv. 16.
vi. 7.	V. xi. 1.
vi. 20.*	V. xxv. 5.
ix. 1.*	I. x. 12.
xiii. 4, 9, 10, 11.*	III. i. 10.
xiii. 5.	Serm. V. 15.
xiii. 8-12.	Serm. V. 15.
xiv. 3.*	V. xvii. 1.
xv. 9.	VIII. v. 1.
xvii. 6.*	V. xvii. 1.
xix.	VII. xxiv. 21.
xix. 5.	VIII. viii. 6.
xix. 5-11.	VI. App.
xix. 6.*	V. i. 2.
xix. 6.	VIII. App. N°. 1.
xix. 8, 11.*	VIII. i. 4.
xix. 11.	VII. vi. 6.
xx.*	V. lxxii. 5.
xx. 3.*	V. xli. 1.
xx. 7.*	V. iii. 1.
xxiv. 4-9.	VIII. viii. 3.
xxiv. 5.	VIII. v. 1.
xxvi. 16.*	V. lxii. 13.
xxix. 30.*	V. xxv. 5.
xxix. 30.	V. xl. 3.
xxx. 1.	VIII. v. 1.
xxx. 6.	VIII. viii. 3.
xxx. 13.*	V. lxviii. 10.
xxx. 27.	V. xxv. 3.
xxix. (xxxi?)*	V. xvii. 1.
xxxi. 10.	VII. xxii. 7.
iii. (xxiv?)*	V. xvii. 1.
xxxiv. 7.*	V. lxv. 16.
xxxiv. 18, 19, 21.	V. xxii. 4.
xxxiv. 29.	VIII. v. 1.
xxxv. 6.*	V. lxviii. 1.
xxxv. 10. 14.*	V. lxviii. 4.
EZRA.
ii. 68, 69.	V. lxxix. 5.
iii. 12.	V. xi. 1.
vi. 6.	V. xii. 4.
viii.*	V. lxxii. 5.
viii. 26.	V. lxxix. 5.
NEHEMIAH.
ii. 17.	VIII. i. 5.
vii. 70.	V. lxxix. 5.
viii. 3.	V. xxxii. 4.
viii. 3, 12.	V. lxxii. 7.
viii. 9. (10?)	V. lxx. 3.
x. 32.	V. lxxix. 5.
xiii. 15.*	V. lxxi. 8.
ESTHER.
vii. 2.*	Serm. VII. 3.
ix.	V. lxxi. 5.
ix. 27.	V. lxx. 6.
JOB.
i. 5.	V. lxxii. 16.
i. 6.	I. iv. 1.
i. 7.	I. iv. 3.
i. 12.	V. xlviii. 3.
i. 21.	VII. xxii. 1.
ii. 2.	I. iv. 3.
ii. 6.	V. xlviii. 3.
ii. 11.	V. lxxv. 3.
iv. 12.	II. iv. 6.
xii. 12.	V. vii. 1.
xiii. 7.	VIII. App. N°. 11.
xiii. 15.	Serm. I.
xiii. 15.	Serm. IV.
xv. 2.*	Serm. II. 39.
xv. 2, 3.	Serm. V. 4.
xix.	Serm. II. 23.
xx. 16.	Serm. IV.
xxi. 11.*	Serm. V. 14.
xxviii. 26.	I. iii. 2.
xxix. 8.*	V. xlvii. 3.
xxix. 21, 22, 25.*	VII. i. 3.
xxx. 1-9.*	VII. i. 3.
xxxi. 33.*	V. xxxvi. 2.
xxxi. 33.	VI. iv. 4.
xxxii. 6.*	V. xlvii. 3.
xxxiii. 2.	V. xlviii. 9.
xxxiv. 3.	I. xvi. 7.
xxxiv. 20.	V. xlvi. 1.
xxxviii. 7.	I. iv. 1.
xxxviii. 11.*	Serm. V. 9.
xl. 4, 5.	Pref. IX. 2.
xl. 4, 5.*	VIII. App. N°. 11.
xlii. 3.	Serm. V. 4.
PSALMS.
i. 2.	V. xxii. 13.
i. 3.	V. i. 2.
i. 4.	V. xlviii. 6.
ii. 8.	V. xlviii. 5.
ii. 8.	VIII. iv. 5.
v. 5.	V. App. N°. 1. 29.
v. 8.*	VII. xxiv. 15.
vi. 6.*	VI. vi. 17.
vii. 6.*	VII. xxiv. 2.
viii. 4.	I. xi. 3.
viii. 6.	V. lv. 8.
ix. 20.	Serm. IV.
xix. 5.*	I. iii. 2.
xix. 8.	II. iv. 1.
xix. 12.*	Serm. II. 18.
xxi. 3, 4.	Serm. VII. 3.
xxii. 1.*	Serm. I.
xxii. 9.	Serm. VII. Intr.
xxii. 23.	V. xlii. 7.
xxv. 13.	Serm. III.
xxvi. 12.*	V. xxiv. 2.
xxvii. 4.*	V. xxiv. 2.
xxx. 4.*	V. xxiv. 2.
xxxii. 1.	Serm. VI. 26.
xxxii. 2.	VI. v. 4.
xxxii. 5.	VI. iv. 16.
xxxii. 11. 7.	V. xlviii. 6.
xxxiv.	Serm. III.
xxxiv. 1.*	V. xxiv. 2.
xxxiv. 13.	V. App. N°. 1. 13.
xxxv. 13.*	V. lxxii. 17.
xxxvi. 7.	Serm. III.
xxxvii. 25.*	V. vii. 1.
xxxvii. 25.*	V. lxxvi. 5.
xxxix. 5.	V. xlvii. 2.
xl. 8.	V. xlviii. 9.
xlii. 4.*	V. xxiv. 2.
xliv. 23-27.	Serm. III.
l. 10.	VII. xxii. 1.
l. 13, 14.	VII. xxii. 5.
l. 15.	Serm. VII. 1.
l. 18.*	VII. xxi. 2.
li. 1.*	VI. vi. 2.
li. 4.	VIII. App. N°. 1.
li. 10-12.*	V. lxvi. 6.
lv. 14.	V. xxxix. 1.
lv. 17.*	V. xxiv. 1.
lv. 23.*	V. lxxvi. 4.
lv. 23.	Serm. III.
lxviii. 18.	V. lxxviii. 9.
lxix. 10.*	V. lxxii. 18.
lxix. 22, 23.	Serm. VI. 20.
lxxii. 3, 6.*	VIII. App. N°. 11.
lxxii. 10, 11.	VII. xxii. 5.
lxxii. 15.	VII. xvii. 3.
lxxiii.	Serm. III.
lxxiii. 3.	Serm. IV.
lxxiii. 3-9.	V. App. N°. 1. 42.
lxxiii. 5.*	V. lxxii. 18.
lxxiii. 28.	Serm. I.
lxxvii. 20.	VII. xviii. 3. 11.
lxxviii. 19.	Serm. I.
lxxix. 3.	V. lxxv. 3.
lxxix. 9.*	Serm. I.
lxxx. 8, 9.*	VII. viii. 7.
lxxx. 8-11.	Serm. VI. 20.
lxxx. 14.	Serm. VI. 20.
lxxxi. 13.	V. xvii. 2.
lxxxiv. 1.*	V. xxiv. 2.
lxxxvi.	Serm. III.
lxxxviii. 15.	Serm. IV.
lxxxix. 28, 32.	Serm. I.
xc. 10.	V. lxxix. 17.
xci. 11, 12.	I. iv. 1.
xcv. 7.	V. App. N°. 1. 42.
xcvi. 6.	V. xxv. 2.
xcvi. 9.	V. xvi. 2.
xcvi. 9.*	V. xxiv. 2.
cii. 13, 14.*	VII. vii. 2.
civ. 4.	I. iv. 1.
cv. 15.	Serm. VI. 2.
cv. 24, 25.*	VII. xxiv. 22.
cvi. 19, 20.*	III. i. 8.
cvi. 30.*	V. lxii. 21.
cxv. 3.	V. App. N°. 1. 26.
cxv. 3.	Serm. III.
cxv. 8.	V. xvii. 2.
cxvi. 4-8.	Serm. VII. 2.
cxviii. 24.	V. lxix. 3.
cxviii. 24.	V. lxx. 2.
cxix. 5.*	Serm. II. 8.
cxix. 14.	Serm. V. 14.
cxix. 16.	V. xxii. 13.
cxix. 22.*	V. xlviii. 13.
cxix. 33. 35.*	V. xxii. 13.
cxix. 71.*	V. xlviii. 13.
cxix. 71.	Serm. III.
cxix. 98.	I. xv. 4.
cxix. 99.	II. i. 4.
cxxii. 1.*	V. xxiv. 2.
cxxii. 1.*	V. lxviii. 10.
cxxii. 6.	VIII. App. N°. 11.
cxxxii. 3-5.	V. xi. 1.
cxxxii. 9.	V. xxv. 3.
cxxxv. 18.	I. viii. 11.
cxxxvi. 10, 15, 18.	V. App. N°. 1. 44.
cxxxvii. 7.*	V. xvii. 1.
cxxxix. 2.	V. App. N°. 1. 23.
cxxxix. 7, 8.	V. lv. 3.
cxli.	V. App. N°. 1. 13.
cxliv. 2.*	V. i. 2.
cxlv. 15, 16.*	I. xvi. 7.
cxlvii. 9.	Serm. VII. Intr.
cxlvii. 20.	V. App. N°. 1. 40.
cxlviii. 2.	I. iv. 1.
cxlviii. 2.	I. iv. 2.
cxlviii. 7-9.	I. xvi. 5.
cxlix. 2.*	V. xxix. 5.
PROVERBS.
i. 2-4.	V. xxii. 6.
i. 32.*	V. xlviii. 13.
ii. 4.	I. vii. 7.
ii. 9.	II. i. 3, 4.
iii. 9.	V. lxxix. 2.
iii. 9.	VII. xxii. 2.
iii. 9.	VII. xxiv. 21.
iii. 10.	VII. xxxii. 7.
iv.	V. App. N°. 1. 13.
v. 22.	VI. vi. 8.
vi. 20.	III. ix. 3.
vi. 20.*	VIII. vi. 5.
viii. 15.	I. xvi. 2.
viii. 16.	VIII. iv. 6.
viii. 22.	I. ii. 5.
x. 7.*	V. lxxvi. 4.
xi. 29.*	VII. xviii. 12.
xvi. 4.	I. ii. 4.
xxii. 3.	Serm. IV.
xxii. 15.	VIII. App. N°. 11.
xxiii. 22.	VII. xvii. 2.
xxiii. 26.	Serm. VII. 1.
xxviii. 13.	VI. iv. 4.
xxix. 18.	V. xxii. 11.
xxx. 8.*	V. lxxvi. 5.
xxxi. 6.	V. lxxv. 3.
ECCLESIASTES.
iii. 1.	V. lxxiii. 4.
iii. 9, 10.	I. iii. 2.
iv. 9.*	V. viii. 3.
iv. 12.	VI. App.
v. 2.	I. ii. 2.
vi. 2.	Serm. IV.
vii. 2-4.	V. lxxii. 16.
ix. 7.*	V. lxxii. 17.
ix. 8.	V. lxxv. 2.
x. 1.*	Pref. iv. 8.
x. 18.	Serm. VI. 30.
xii. 9, 10.*	V. i. 2.
xii. 13.*	V. lxxii. 15.
CANTICLES.
iii. 1.	Serm. VII. 1.
vii. 6.	Serm. VI. 8.
viii. 11.	Serm. V. 15.
ISAIAH.
i. 4.*	III. i. 8.
i. 4.*	Serm. II. 8.
i. 6.	Serm. III.
i. 13.	V. lxx. 2.
i. 13.	V. lxxii. 15.
i. 18.	VI. v. 4.
i. 21, 23.	VIII. i. 5.
ii. 2.	V. xvii. 2.
iii. 5.*	VII. xviii. 12.
vi. 1-3.	V. xxxix. 3.
vi. 3.	I. iv. 1.
vi. 8.	V. lxxvii. 11.
vi. 10, 11.	V. App. N°. 1. 42.
vii. 9.	VIII. iv. 4.
vii. 16.	I. vi. 1.
viii. 20.	V. xxii. 2.
viii. 21.	V. xvii. 2.
ix. 6.	V. li. 1.
x. 1.	I. x. 10.
x. 13.	Serm. III.
xi. 2.	V. liv. 7.
xxvi. 20.	Serm. IV.
xxix. 13.	I. xv. 2.
xxix. 14.*	III. xi. 15.
xxx. 1, 2.	II. vi. 3.
xxx. 2.*	V. xxix. 5.
xxx. 23.	Serm. III.
xxxviii.	Serm. III.
xl. 26.	V. lvi. 5.
xli. 22, 23.	V. App. N°. 1. 23.
xli. 24.	V. xvii. 2.
xliii. 25.	VI. vi. 4.
xliv. 18, 19.	I. viii. 11.
xliv. 20.*	III. viii. 8.
xlv.	VIII. ii. 5.
xlv. 20.	V. xvii. 2.
xlvi. 10.	V. App. N°. 1. 26.
xlviii. 22.	Serm. IV.
xlix. 3.	Serm. V. 4.
xlix. 15.	I. x. 2.
xlix. 23.	VII. xv. 5.
liii.	V. App. N°. 1. 33.
liii. 1.*	V. xxii. 4.
liii. 5.	V. lvi. 11.
liii. 10.	V. xlviii. 10.
liv. 3.	V. l. 1.
lv. 6.	Serm. VII. 3.
lvii. 3.*	III. i. 8.
lviii. 3.	V. lxxii. 15.
lviii. 6, 7.*	V. lxxii. 15.
lx. 15.*	III. i. 8.
lxi. 1.	V. liv. 7.
lxv. 12.	V. App. N°. 1. 29.
lxvi.	Serm. III.
lxvi. 21.	V. lxxviii. 3.
JEREMIAH.
ii. 17.*	V. xvii. 2.
iii. 14, 15.	Serm. VI. 34.
v. 3.	V. App. N°. 1. 42.
v. 22.	I. iii. 2.
vi. 26.	VI. vi. 17.
xi. 11.*	V. xxv. 3.
xi. 13.*	III. i. 8.
xiii. 11.	III. i. 8.
xv. 1.	V. xlix. 3.
xvi. 7.	V. lxxv. 3.
xvii. 24.	V. xii. 5.
xxiii. 1-4.	V. lxxxi. 2.
xxiii. 6.	V. li. 1.
xxiii. 24.	V. lv. 3.
xxix. 13.	VI. vi. 18.
xxix. 26.	VII. xv. 2.
xxxi.	Serm. III.
xxxvi.*	V. lxxii. 5.
li. 9.*	IV. x. 3.
LAMENTATIONS.
ii. 13.*	VII. i. 1.
ii. 13.	Serm. VI. 10.
ii. 18.	VI. vi. 17.
EZEKIEL.
iii. 2, 3.	Serm. V. 4.
iii. 14.	Serm. V. 4.
viii. 18.*	V. xxv. 3.
ix. 4.	V. lxv. 7.
xviii. 23-32.	V. App. N°. 1. 32.
xxii. 30.*	Trav. Sup.
xxiv. 1, 2.	V. lxxii. 5.
xxxiii. 6.*	Trav. Sup.
xxxiii. 14.	VI. v. 4.
xxxiv. 2, 8, 10.	V. lxxxi. 2.
xxxvi. 20.	IV. xii. 2.
xlv. 1-4.	V. lxxix. 13.
xlviii. 14.	VII. xxiv. 20.
DANIEL.
i. 17.*	III. viii. 9.
ii. 21.	VIII. ii. 5.
iii. 29.*	V. ii. 2.
iv.	VIII. ii. 5.
iv. 13.	I. iv. 1.
v. 6.	Serm. IV.
vii. 10.	I. iv. 1.
ix. 3.*	V. xxiv. 1.
ix. 20.*	V. xxiii.
ix. 23.	I. iv. 1.
x. 2, 3.	V. lxxii. 6.
xii. 3.*	VIII. App. N°. 11.
HOSEA.
i. 6, 9.	Serm. VI. 20.
ii. 8.	VII. xxii. 1.
iv. 6.	V. lxxxi. 2.
iv. 6.	V. App. N°. 1. 39.
iv. 17, 15.*	III. i. 10.
iv. 17.	Serm. VI. 31.
viii. 8.	V. App. N°. 1. 39.
viii. 12.	I. xiii. 1.
ix. 15.	V. App. N°. 1. 39.
xiii. 9.	V. App. N°. 1. 39.
xiv.	Serm. III.
xiv. 2.	V. xxiii.
xiv. 4. (3?)	V. xvii. 2.
JOEL.
ii. 12.	VI. vi. 18.
ii. 15.*	V. xli. 1.
ii. 15.	V. lxxi. 5.
ii. 16.	V. lxxiii. 4.
ii. 17.*	V. xxv. 5.
AMOS.
iii. 6.	V. App. N°. 1. 32.
viii. 11, 12.	Serm. VI. 33.
OBADIAH.
5.	VII. xxiv. 25.
JONAH.
iii. 5.	VI. iii. 2.
iii. 7.	V. lxxii. 6.
iii. 9.	VI. iii. 4.
iv. 11.*	V. xxiv. 1.
MICAH.
i. 8, 9.	VI. vi. 17.
v. 2.	V. xix. 3.
vii. 19.	VI. vi. 8.
HABAKKUK.
i. 4.	Serm. I.
i. 4.	Serm. II.
i. 16.	Serm. VII. 2.
ii. 4.	Serm. III.
ii. 17.	VII. xxiv. 20.
HAGGAI.
ii. 2.	V. xi. 1.
ii. 2, 3.	Serm. VI. 34.
ii. 3.	V. lxxix. 5.
ii. 5, 9.	V. xv. 3.
ZECHARIAH.
iv. 6.*	IV. xiv. 7.
iv. 7.*	V. lxxi. 2.
iv. 10.*	V. lxxi. 2.
viii. 17.	V. App. N°. 1. 29.
viii. 19.	V. lxxii. 5.
MALACHI.
i. 7, 8.	VIII. i. 5.
i. 8.	VII. xxii. 3.
i. 8.	V. xv. 4.
i. 8, 14.	V. xxxiv. 3.
ii. 7.	Pref. iii. 2.
ii. 7.	Jack. Ded.
ii. 9.*	Pref. iii. 3.
iii. 8.	V. lxxix. 13.
iii. 8.	VII. xxiv. 17.
iii. 9.	VII. xxiv. 20.
iii. 10.	V. lxxix. 8.
iii. 10.	VII. xxii. 1.
iii. 10.	VII. xxii. 7.
iii. 15.	VI. vi. 8.
APOCRYPHA.
JUDITH.
viii. 6.	V. lxxii. 6.
ESTHER.
xiii. 9.	V. App. N°. 1. 26.
WISDOM.
ii. 21.	V. ii. 1.
iv. 9.*	V. vii. 1.
iv. 11.*	V. lxxvi. 5.
iv. 11.	V. App. N°. 1. 24.
iv. 13.*	IV. xiv. 7.
v. 4.*	Pref. iii. 14.
vi. 10.	V. vi. 1.
vi. 24.*	Pref. viii. 3.
vii. 23.	V. lvi. 5.
vii. 27.	I. v. 3.
viii. 1.	I. ii. 3.
viii. 1.	V. App. N°. 1. 2.
viii. 12.	V. App. N°. 1. 27.
ix. 6.	Serm. V. 4.
ix. 15, 16.	I. vii. 7.
x. 20.	V. xli. 1.
xi. 20.	I. ii. 3.
xiii. 17.	I. viii. 11.
xiv. 15, 16.	I. viii. 11.
xiv. 31.	V. i. 3.
xvi. 7.	V. lvii. 4.
xvii. 11.*	V. i. 2.
xvii. 12.	V. iii. 1.
ECCLESIASTICUS.
v. 10.*	Serm. II. 39.
vii. 6.*	V. lxxvii. 14.
xv. 11.	V. App. N°. 1. 29.
xv. 14.	V. App. N°. 1. 28.
xxiii. 19, 20.	V. App. N°. 1. 23.
xxv. 6.	VII. xvii. 2.
xxvi. 28.*	Pref. viii. 3.
xxxiii. 7-12.	V. lxix. 3.
xxxviii. 1.	VII. xvii. 2.
xxxix. 4.	V. xv. 3.
xxxix. 18.	V. App. N°. 1. 46.
xxxix. 19, 20.	V. App. N°. 1. 23.
xlv. 7.	V. xxix. 4.
xlv. 7.	VII. xx. 3.
xlvii. 8, 9.	V. xxxviii. 2.
li. 26, 27.*	V. xxii. 17.
HISTORY OF SUSANNA.
9.*	V. ii. 1.
1 MACCABEES.
ii. 40.*	V. lxxi. 8.
iv. 54.	V. lxx. 6.
iv. 55, 59.	V. lxxi. 7.
xiv. 42.	VIII. i. 1.
xiv. 44.	VIII. v. 1.
2 MACCABEES.
vi. 24.*	V. xlii. 3.
xiii. 12.	V. lxxii. 5.
xv. 36.	V. lxxi. 6.
xv. 38.	Serm. V. 4.
ST. MATTHEW.
i. 25.	V. xlv. 2.
ii. 6.	V. xix. 3.
ii. 11.	VII. xxii. 5.
iii. 6.	VI. iv. 5.
iii. 9.*	V. xxii. 17.
v. 4.	V. lxxii. 16.
v. 6.*	V. xxii. 17.
v. 12.	I. xi. 5.
v. 14.*	III. xi. 20.
v. 14.*	VII. xviii. 7.
v. 20, 21.	Serm. II. 30.
v. 46.	II. viii. 2.
v. 48.	I. v. 3.
vi. 2.	I. vii. 1.
vi. 5, 6.*	V. xxv. 1.
vi. 7.	V. xxxii. 1.
vi. 10.	I. iv. 1.
vi. 13.	VIII. iv. 6.
vi. 16.	V. lxxii. 4.
vi. 20.	Serm. VI. 8.
vi. 24.	V. lxxxi. 2.
vi. 29.	V. xv. 4.
vi. 33.	I. x. 2.
vi. 33.*	VIII. i. 4.
vii. 24.*	III. i. 9.
vii. 25.	Serm. VI. 16.
ix. 2.	VI. vi. 1.
ix. 2.	VI. vi. 4.
ix. 13.	V. lxi. 4.
ix. 14.	V. lxxii. 4.
ix. 14.*	V. lxxii. 15.
ix. 18.	V. lxvi. 1.
ix. 21.*	V. lxvii. 12.
x. 6.	V. App. N°. 1. 41.
x. 11, 12.	V. xlix. 3.
x. 28.	VI. vi. 8.
x. 42.	II. viii. 4.
xi. 20.	VI. iii. 2.
xi. 21.	V. App. N°. 1. 41.
xi. 25.	Serm. III.
xii. 20.*	V. xxii. 17.
xii. 30.*	III. i. 9.
xii. 31.	VI. vi. 15.
xii. 42.*	I. x. 12.
xiii. 24.	III. i. 8.
xiii. 24, 27.	V. lxviii. 6.
xiii. 47.	III. i. 8.
xiii. 52.	III. viii. 9.
xv. 13.	Pref. viii. 6.
xv. 14.*	Pref. iii. 3.
xv. 14.	V. lxxxi. 2.
xv. 19.	VI. vi. 8.
xvi. 16.	V. li. 2.
xvi. 17.	V. xxi. 5.
xvi. 17.*	V. lxiii. 1.
xvi. 18.*	III. i. 9.
xvi. 18.	Serm. V. 15.
xvi. 19.	V. xxii. 12.
xvi. 19.	VI. iv. 1.
xvi. 26.	I. viii. 5.
xvii. 5.*	II. vi. 4.
xvii. 10.	II. vii. 7.
xviii. 6.	IV. xii. 2.
xviii. 10.	I. iv. 1.
xviii. 10.	V. xlii. 7.
xviii. 15.	VIII. ix. 3.
xviii. 16.	II. vii. 2.
xviii. 17, 18.	VI. iv. 1.
xviii. 18.	VI. vi. 2.
xviii. 20.	V. xxiv. 1.
xix. 4-9.*	V. lxxvii. 3.
xix. 13.	V. lxvi. 1.
xix. 14.	V. lxiv. 3.
xix. 16, 17.	Serm. VII. 1.
xix. 17.	V. App. N°. 1. 29.
xix. 26.*	Serm. I.
xix. 28.	V. lxxviii. 3.
xix. 28.	VII. vi. 5.
xx. 21.	Serm. III.
xx. 23.	Serm. VII. 1.
xx. 25.	VII. xvi. 9.
xxi. 13.	V. xii. 5.
xxi. 13.	V. xxv. 2.
xxi. 13.	V. lxxix. 13.
xxi. 23, 25, 26.	VII. xi. 9.
xxii. 1, 2.*	V. xix. 4.
xxii. 20.*	V. lxv. 16.
xxii. 30.	I. xi. 3.
xxii. 38.	I. viii. 7.
xxii. 40.	I. viii. 7.
xxii. 43.	III. viii. 17.
xxiii. 3.	VIII. App. N°. 1.
xxiii. 6, 7.	VII. xx. 2.
xxiii. 8, 10.*	II. vi. 4.
xxiii. 23.*	Pref. vi. 5.
xxiii. 23.*	III. iii. 4.
xxiii. 23.*	IV. i. 1.
xxiii. 27.	V. lxxv. 3.
xxiii. 33.	V. lxi. 5.
xxiii. 37.	I. vii. 7.
xxiv. 16.	Serm. ii. 10.
xxv. 34.	V. App. N°. 1. 46.
xxv. 46.	I. xi. 3.
xxvi. 13.	VII. xxii. 5.
xxvi. 26-28.	V. lxvii. 6.
xxvi. 30.	V. xxvi. 2.
xxvi. 39.	V. xlviii. 5.
xxvi. 39.	Serm. IV.
xxvi. 40.	II. viii. 1.
xxvi. 53.	I. iv. 1.
xxvi. 53.	I. iv. 2.
xxvii. 46.	V. xlviii. 9.
xxvii. 46.	V. liv. 6.
xxvii. 46.*	Serm. V. 5.
xxviii. 1.	V. lxxi. 11.
xxviii. 18.	V. lv. 8.
xxviii. 18.	V. lxxvii. 6.
xxviii. 19.*	III. i. 9.
xxviii. 19.	VI. ii. 1.
xxviii. 19.	VII. iv. 4.
xxviii. 20.	VIII. vi. 7.
ST. MARK.
ii. 27.	V. lxxi. 8.
ii. 7.	VI. vi. 1.
ii. 7.	VI. vi. 4.
iii. 17.*	Trav. Sup.
iii. 30.	VI. vi. 15.
v. 21.	VI. vi. 1.
v. 23.	V. lxvi. 1.
vi. 23.*	Serm. VII. 3.
vii. 7, 8.	Serm. VII.
vii. 9.*	V. iii. 4.
viii. 22.	V. lxvi. 1.
ix. 24.	Serm. I.
x. 4.	I. viii. 8.
x. 13.	V. lxvi. 1.
xi. 16.	V. xii. 5.
xii. 6.*	V. xxx. 3.
xii. 42.*	VI. vi. 17.
xiv. 22.	V. lxvii. 6.
xiv. 36.	V. xlviii. 5.
xv. 43.*	Trav. Sup.
xvi. 1.	V. lxxi. 11.
xvi. 5.*	V. xxix. 5.
xvi. 15.	VI. ii. 1.
xvi. 16.	V. lx. 4.
xvi. 17.	V. lxvi. 2.
xvi. 18.	VI. iv. 5.
ST. LUKE.
i. 6.	VII. xi. 10.
i. 23.*	V. iv. 3.
i. 26.	V. lxx. 8.
ii. 8.	V. lxxxi. 2.
ii. 13.	I. iv. 1.
ii. 13.	I. iv. 2.
ii. 14.	V. lxxi. 11.
ii. 21.	V. lxx. 8.
ii. 28.*	Serm. II. 23.
ii. 47.	V. liv. 6.
iii. 22.	V. lvi. 4.
iv. 18.	V. liv. 7.
v. 6, 7.	V. xix. 3.
v. 21.	VI. vi. 1.
v. 21.	VI. vi. 4.
v. 35.	V. lxxii. 8.
vi. 4.*	V. ix. 1.
vi. 39.*	III. xi. 20.
vii. 5.*	V. lxxix. 4.
vii. 8.*	II. viii. 6.
vii. 12.	V. lxxv. 3.
vii. 27. (29?)	VI. vi. 8.
vii. 44-46.*	V. lxv. 5.
vii. 47.	VI. iv. 4.
viii. 32.	V. xlviii. 3.
viii. 39.	V. xviii. 1.
ix. 54.	I. vii. 7.
x. 16.	Serm. V. 6.
x. 41, 42.	Serm. VI. 9.
xi. 1.	V. xxxv. 3.
xi. 17.	VIII. ii. 2.
xi. 31.*	I. x. 12.
xi. 31.*	V. xxii. 17.
xi. 39.*	Serm. II. 30.
xii. 3.	V. xviii. 1.
xii. 5.	VI. vi. 8.
xii. 14.	VII. xv. 11.
xii. 42.	V. lxxvi. 10.
xii. 56, 57.	Pref. iii. 1.
xiii. 3.*	Serm. II. 18.
xiii. 24.	I. vii. 7.
xiv. 23.*	V. lxviii. 10.
xv. 7.	I. iv. 1.
xv. 18.	VI. iii. 4.
xvi. 31.	V. xxii. 4.
xviii. (viii).	VI. vi. 11.
xviii. 1.*	V. xxiii.
xviii. 10-14.	Serm. VII. 1.
xviii. 11.*	Serm. I.
xviii. 12.	V. lxxii. 4.
xviii. 13.*	VI. vi. 2.
xviii. 15.	V. lxvi. 1.
xxi. 21.	Serm. II. 10.
xxii. 25, 26.	Serm. III.
xxii. 25, 26.	Serm. V. 15.
xxi. 27.	V. lv. 8.
xxii. 28, 30.	Serm. III.
xxii. 31, 32.	Serm. I.
xxii. 33.*	Serm. I.
xxii. 42.	V. xlviii. 5.
xxii. 43.	V. xlviii. 11.
xxiii. 28.	Serm. IV.
xxiv. 1.	V. lxxi. 11.
xxiv. 49.	V. lxxvii. 7.
ST. JOHN.
i. 4-9.	V. lvi. 7.
i. 4, 10.	V. lvi. 5.
i. 9.	III. ix. 3.
i. 12, 13.*	V. lxiii. 1.
i. 14.	V. li. 2.
i. 14.	V. lii. 3.
i. 16.	V. lvi. 10.
i. 18.	V. lvi. 2.
i. 25.	VII. xi. 8.
i. 29.	Serm. VII. 2.
i. 46.	Serm. I.
i. 47.*	III. i. 2.
i. 49.	Serm. II. 16.
iii. 2.	VII. xiv. 11.
iii. 3, 5.	V. lx. 1.
iii. 5.	V. lix. 1.
iii. 5.	VI. vi. 10.
iii. 13.	V. liii. 4.
iii. 17, 18.	Serm. II. 13.
iii. 18.	V. lx. 5.
iii. 34, 35.	V. lvi. 4.
iii. 35.	V. liv. 3.
iv. 10.	V. liv. 3.
iv. 24.	V. vi. 1.
iv. 24.*	VII. xxiii. 2.
iv. 34.	V. lxxii. 2.
iv. 39.*	II. vii. 2.
iv. 42.	Serm. II. 16.
v. 4.	V. lvii. 3.
v. 17.	I. ii. 3.
v. 20.	V. lvi. 4.
v. 21.	V. li. 1.
v. 26.	V. liv. 3.
v. 39.	V. xxii. 2 (3).
v. 39.	V. xxii. 10.
vi.	V. App. N°. 1. 33.
vi. 25.*	V. lxvii. 3.
vi. 26.	Serm. III.
vi. 28, 29.	Serm. VI. 18.
vi. 29.	I. xi. 6.
vi. 31-35.	Serm. VI. 7.
vi. 33.	Serm. V. 14.
vi. 39.	V. App. N°. 1. 45.
vi. 46. (45?)	V. xxi. 5.
vi. 53.	V. lxvii. 1.
vi. 57.	V. lvi. 7.
vi. 63.	V. lxvii. 9.
vi. 68.*	Serm. II. 23.
vii. 51.*	Trav. Sup.
viii. 11.	VII. xv. 11.
viii. 44.	I. iv. 3.
viii. 44.	V. App. N°. 1. 29.
ix.	VI. vi. 11.
ix. 31.*	V. xxv. 3.
x. 4.	V. lxxxi. 2.
x. 11.*	V. xix. 4.
x. 15.	V. xlviii. 10.
x. 17.	V. lvi. 4.
x. 22.	III. xi. 15.
x. 22.	V. lxx. 6.
x. 28.	III. i. 2.
x. 28.	Serm. II. 26.
x. 38.	II. iv. 1.
xi. 35, 36.*	V. lxxv. 2.
xii. 27.	V. xlviii. 9.
xiii.*	III. vi.
xiii. 1.	Serm. I.
xiii. 13.	III. i. 4.
xiii. 27.	I. iv. 3.
xiv. 2.	V. xlv. 1.
xiv. 6.	I. xi. 6.
xiv. 9.	V. lvi. 7.
xiv. 15.	I. ii. 2.
xiv. 17, 19.	Serm. II. 26.
xiv. 20.	V. lvi. 7.
xiv. 23.	V. lvi. 5.
xiv. 23.	Serm. VI. 10.
xiv. 27.	I. x. 14.
xiv. 27.	Serm. IV.
xiv. 31.	V. lvi. 4.
xv. 4.	V. lvi. 7.
xv. 5, 6.	V. lvi. 7.
xv. 9.	V. lvi. 7.
xv. 10.	V. lvi. 4.
xv. 16.	V. lxxvii. 13.
xv. 16.*	VII. xxii. 6.
xv. 21.*	III. i. 4.
xvi. 2, 4.*	III. i. 4.
xvi. 13-15.	I. ii. 2.
xvi. 15.	V. li. 1.
xvi. 23.	Serm. VII. 3.
xvii. 1, 2.	V. xlviii. 5.
xvii. 5.	VIII. iv. 6.
xvii. 8.	III. xi. 3.
xvii. 9, 20.	V. App. N°. 1. 46.
xvii. 11.	Serm. I.
xvii. 24.	V. xlv. 1.
xvii. 25, 26.	Serm. V. 14.
xviii. 4.	V. xlviii. 8.
xviii. 36, 37.*	III. xi. 11.
xix. 40.	III. xi. 15.
xix. 40.	V. lxxv. 3.
xx. 1.	V. lxxi. 11.
xx. 22.*	V. lxvi. 9.
xx. 22.	V. lxxvii. 6.
xx. 23.	V. lxxvii. 7.
xx. 23.	VI. iv. 1.
xx. 23.	VI. iv. 14.
xx. 23.	VI. vi. 2.
xx. 25.	II. iv. 1.
xx. 27.	V. liv. 8.
xx. 29.	II. iv. 2.
xx. 31.	I. xiv. 4.
xx. 31.	V. xxii. 6.
xxi. 11.	V. xix. 3.
xxi. 15.*	III. i. 2.
xxi. 15.	III. xi. 11.
xxi. 15, 16.	VII. iv. 2.
xxi. 22.*	Serm. II. 37.
xxi. 22.	Serm. VII. 2.
xxi. 22, 23.	Serm. V. 15.
ACTS.
i. 5.	V. lix. 5.
i. 8.*	V. lxvi. 9.
i. 13.	V. xi. 2.
i. 15.*	V. xxx. 4.
i. 20.	VII. xi. 3.
i. 21, 22.	VII. iv. 4.
ii. 1, 46.	V. xi. 2.
ii. 3.	V. lvii. 3.
ii. 3.	V. lix. 5.
ii. 15.*	III. vi.
ii. 15.	V. lxxii. 7.
ii. 34.	III. viii. 16.
ii. 36.	III. i. 4.
ii. 37.	VI. iii. 2.
ii. 37.	Serm. V. 9.
ii. 37.	Serm. VII. 1.
ii. 38.	V. lx. 1.
ii. 38.	VI. vi. 15.
ii. 41.	III. i. 6.
ii. 41, 47.*	V. lxxviii. 4.
ii. 42.*	III. i. 14.
ii. 44.	VII. xxiii. 8.
iii. 21.	V. lv. 8.
iv. 12.	Serm. II. 23.
iv. 27.	V. liv. 7.
iv. 34.	VII. xxiii. 1.
iv. 34, 35.	II. viii. 4.
iv. 35.	VII. xxiii. 8.
iv. 37.	I. viii. 8.
v. 2.	VII. xxiv. 17.
v. 3.	I. iv. 3.
v. 4.	I. viii. 8.
v. 4.*	II. iv. 4.
v. 4.	V. lxxix. 12.
v. 38, 39.*	II. i. 1.
vi. 4.	VII. xv. 12.
vi. 13, 14.	IV. xi. 4.
vii. 22.*	III. viii. 9.
vii. 51.	V. App. N°. 1. 42.
vii. 60.	VI. vi. 8.
viii. 12-17.	V. lxvi. 5.
viii. 15, 17.	V. lxvi. 9.
viii. 17, 18.	V. lxvi. 2.
viii. 18.	VI. vi. 11.
viii. 21.	Serm. III.
viii. 22, 23.	VI. vi. 8.
viii. 26.	VII. v. 10.
viii. 31.	V. xxii. 14.
viii. 38.	III. i. 6.
ix. 17.	V. lxxviii. 7.
x. 3.	I. iv. 1.
x. 4.	V. xxiii.
x. 9.*	V. xxiv. 1.
x. 30.*	V. xxiii.
x. 31.*	VI. vi. 17.
xi. 27.	V. lxxviii. 6.
xi. 30.	VII. xxiii. 1.
xii. 2.	VII. iv. 2.
xii. 17.	VII. xxiii. 1.
xii. 22.	Serm. V. 6.
xiii. 2.	VII. iv. 2.
xiii. 2.	VII. v. 10.
xiii. 15.	V. xx. 1.
xiii. 15.	V. xx. 3.
xiii. 36.	III. viii. 16.
xiii. 41-44.*	Serm. II. 9.
xiii. 46.	V. App. N°. 1. 42.
xiv. 4.	V. xvii. 2.
xiv. 15.	III. viii. 17.
xiv. 17.	V. App. N°. 1. 40.
xiv. 22.*	V. xlviii. 13.
xiv. 23.	V. lxxx. 2.
xiv. 23.	VII. vi. 3.
xiv. 23.	VII. xiv. 6.
xv.	Pref. vi. 2.
xv.	III. x. 7.
xv.	IV. xi. 4.
xv. 5.	Serm. II. 26.
xv. 7, 13-23.	VIII. vii. 1.
xv. 8.	III. viii. 17.
xv. 20.	I. xvi. 7.
xv. 21.	V. xix. 1.
xv. 21.	V. xx. 1.
xv. 21.	V. xx. 3.
xv. 24.	IV. xi. 4.
xv. 28.	I. x. 14.
xv. 28.	III. x. 2.
xv. 28.	VIII. vi. 7.
xv. 28, 29.	IV. xi. 5.
xv. 36.	V. lxxx. 2.
xvi. 4.	IV. xi. 5.
xvi. 4.	VIII. vi. 7.
xvi. 6.	V. App. N°. 1. 41.
xvi. 6, 7.	VII. v. 10.
xvi. 14.	V. xxi. 5.
xvi. 17.	Serm. II. 23.
xvii. 11.	Pref. iii. 1.
xvii. 18.	Serm. V. 6.
xvii. 22.	Serm. V. 9.
xvii. 26.	V. App. N°. 1. 41.
xvii. 28.	I. iii. 4.
xvii. 28, 29.	V. lvi. 5.
xvii. 31.	V. App. N°. 1. 26.
xviii. 4, 11.	III. viii. 10.
xviii. 14, 15.	VIII. i. 5.
xviii. 24.	V. lxxviii. 7.
xix. 6.	V. lxvi. 2.
xix. 17, 19.	VI. iii. 2.
xix. 18.	VI. iv. 5.
xix. 19.*	Pref. viii. 4.
xx. 2.	V. lxxxi. 2.
xx. 9.	V. xxxii. 4.
xx. 28.	III. xi. 11.
xx. 28.	V. lxxxi. 2.
xx. 34.	V. lxxxi. 8.
xx. 28.	VI. ii. 1.
xx. 28.	VII. ii. 2.
xx. 28.	VII. v. 1.
xx. 28.	Serm. V. 15.
xx. 30.	VII. v. 2.
xx. 36, 37.	VII. v. 1.
xxi. 10.	V. lxxviii. 6.
xxi. 18.	VII. xxiii. 1.
xxi. 20.	IV. xi. 4.
xxi. 25.	IV. xi. 4.
xxii. 3.*	III. viii. 9.
xxii. 16.	III. i. 6.
xxiv. 14.	Serm. V. 15.
xxv. 11.	VII. xiii. 5.
xxv. 19.	III. viii. 6.
xxv. 19.	Serm. II. 25.
xxvi. 23.	Serm. II. 32.
xxvi. 24.*	Pref. iii. 14.
xxvi. 24.	III. viii. 6.
xxvi. 27.	III. viii. 12.
xxvi. 28.	Serm. V. 9.
xxvii. 38.*	V. ix. 1.
xxviii. 8.	VI. iv. 5.
xxviii. 11.	V. xiii. 4.
ROMANS.
i. 8.*	III. i. 10.
i. 9.*	V. xxiii.
i. 14, 21, 24.	V. App. N°. 1. 42.
i. 16.	V. xxii. 6.
i. 16.	Serm. V. 5.
i. 19.	III. ix. 3.
i. 21.	I. xvi. 5.
i. 21, 32.	III. viii. 6.
i. 24.	V. xvii. 2.
i. 24.	V. lxxiii. 7.
i. 32.	VII. xi. 10.
ii. 5.	V. App. N°. 1. 42.
ii. 5.	VI. v. 4.
ii. 9.	I. ix. 1.
ii. 14.	I. viii. 3.
ii. 14, 15. (?)	VII. iv. 2.
ii. 14, 15.	Serm. III.
ii. 15.	I. xvi. 5.
ii. 15.*	III. ii. 1.
ii. 15.	III. ix. 3.
ii. 23.	II. ii. 3.
ii. 24.	II. ii. 3.
ii. 24.	IV. xii. 2.
iii. 7, 8.	Serm. II. 38.
iii. 17.	Pref. vi. 1.
iii. 17.*	VIII. App. N°. 2.
iv.	Serm. II. 6.
iv. 5.	Serm. II. 6.
iv. 20.	Serm. I.
v. 2.	V. xlvii. 4.
vi. 10.	Serm. II. 26.
vi. 21.*	V. lxv. 6
vi. 22.	Serm. II. 6.
vii. 19, 24.*	Serm. II. 8.
viii. 7.	V. App. N°. 1. 8.
viii. 9.	V. lvi. 11.
viii. 10.	V. lvi. 7.
viii. 10.	Serm. II. 26.
viii. 10.	Serm. III.
viii. 14.*	III. ix. 3.
viii. 14.	VIII. App. N°. 1.
viii. 15.	V. xlvii. 4.
viii. 21.	Serm. II. 2.
viii. 23.	V. lvi. 11.
viii. 26, 27.	Serm. I.
viii. 28.	V. App. N°. 1. 46.
viii. 30.	V. lx. 3.
viii. 30.	V. App. N°. 1. 46.
viii. 33.	V. App. N°. 1. 46.
viii. 35, 38, 39.	Serm. I.
ix. 3.	VI. App.
ix. 3. 8.	V. xlix. 3.
ix. 5.	V. li. 1.
ix. 11-24.	V. App. N°. 1. 43, 44.
ix. 20.	V. App. N°. 1. 25.
ix. 22.	V. App. N°. 1. 31.
ix. 31-33.	Serm. VI. 28.
x. 1.	V. xlix. 3.
x. 2.	Serm. V. 9.
x. 10.	VIII. vi. 5.
x. 14, 15.	V. xxii. 9.
xi. 1, 7.	V. App. N°. 1. 45.
xi. 6.	Serm. II. 29.
xi. 8.	V. App. N°. 1. 26.
xi. 9.	V. App. N°. 1. 42.
xi. 9, 10.	Serm. VI. 20.
xi. 17.	III. xi. 4.
xi. 18.	Serm. VI. 16.
xi. 20, 22.	Serm. VI. 20.
xi. 21.	Serm. VI. 20.
xi. 28.*	III. i. 10.
xi. 33.	I. ii. 5.
xi. 33.	V. App. N°. 1. 41.
xi. 33.	Serm. VII. 2.
xi. 33, 34.*	III. xi. 21.
xi. 36.	I. ii. 6.
xii. 1.*	V. iv. 3.
xii. 5.	V. lvi. 11.
xii. 8.*	V. lxxviii. 5.
xiii.	VIII. ii. 5.
xiii. 1.	I. xvi. 5.
xiii. 1.	VIII. iv. 6.
xiii. 1.	VIII. App. N°. 1.
xiii. 7.*	VII. xviii. 7.
xiv. 1.*	V. xxii. 17.
xiv. 2.	V. lxxii. 6.
xiv. 5.	Pref. iii. 1.
xiv. 6, 7.	III. vii. 1.
xiv. 9.	V. lv. 8.
xiv. 10.	IV. xi. 5.
xiv. 17.	V. lxxii. 3.
xiv. 17.*	V. lxxii. 15.
xiv. 20, 15-20.	IV. xii. 6.
xiv. 23.	II. iv. 1.
xv. 1.	IV. xii. 6.
xv. 5.*	V. lxviii. 6.
xv. 6.*	III. i. 10.
xv. 13.	Serm. I.
xvi. 5, 7.*	IV. xiii. 1.
xvi. 5, 7.*	IV. xiii. 9.
xvi. 16, &c.	Pref. iv. 4.
1 CORINTHIANS.
i. iii-vi.	III. i. 10.
i. 10.*	V. lxviii. 6.
i. 11.*	II. vii. 2.
i. 19.*	III. viii. 4.
i. 21.	V. xxii. 9.
i. 27.*	Pref. iii. 14.
i. 27, 28.*	Serm. I.
i. 30.	Serm. II. 2.
ii. 4.*	III. viii. 4.
ii. 4, 5.	III. viii. 10.
ii. 8.	V. liii. 4.
ii. 12, 13.	Serm. V. 4.
ii. 14.*	III. viii. 4.
ii. 14.	III. viii. 6.
iii. 6.	V. xxii. 12.
iii. 11.	Serm. II. 23.
iii. 19.*	III. viii. 8.
iii. 22, 23.	VIII. iv. 6.
iv. 1.	V. lxxvi. 10.
iv. 8.	Serm. III.
iv. 12.	V. lxxxi. 8.
v. 1.	IV. xi. 7.
v. 3.	VI. iv. 1.
v. 11.	IV. xi. 7.
v. 11.*	V. lxviii. 1.
v. 11.*	V. lxviii. 5.
v. 12, 13.	VIII. ix. 3.
v. 12, 13.	Serm. II. 1.
vi. 1-7.	VII. xv. 3.
vi. 11.	VI. vi. 8.
vi. 12.	II. iv. 4.
vi. 12.	IV. xii. 6.
vii. 4.	V. lxxiii. 7.
vii. 5.	V. lxxii. 8.
vii. 5.	V. lxxiii. 4.
vii. 8, 26.*	III. viii. 2.
vii. 14.	V. lx. 6.
vii. 24.	V. lxxxi. 2.
vii. 25.	VII. vi. 2.
vii. 39.*	II. v. 6.
viii. 6.	I. iii. 5.
viii. 6.	V. xlii. 9.
viii. 8.	V. lxxii. 3.
ix. 13.	VII. xxiii. 6.
ix. 16.	VII. iv. 2.
x. 4.	Serm. VI. 16.
x. 15.	Pref. iii. 1.
x. 15.	III. viii. 11.
x. 31.	II. ii. 1.
x. 31.	III. vii. 1.
x. 32.	II. ii. 3.
x. 32.	III. vii. 1.
x. 33.	II. ii. 2.
xi.	Serm. III.
xi. 1.*	VII. xxiii. 11.
xi. 10.	I. xvi. 4.
xi. 10.	V. xxv. 2.
xi. 13.	Pref. iii. 1.
xi. 22.	III. ix. 3.
xi. 22.	V. xii. 5.
xi. 24.	VI. ii. 1.
xi. 27.	VI. v. 8.
xii. 3.	V. xxi. 5.
xii. 3, 4, 13.	V. xlii. 9.
xii. 12.	V. lvi. 7.
xii. 13.	III. i. 3.
xii. 27.	V. lvi. 11.
xii. 28.	V. lxxviii. 8.
xii. 28.	VI. App.
xiii. 7.*	V. xlix. 2.
xiv. 15.	V. xxvi. 3.
xiv. 16.	V. xxxvi. 3.
xiv. 16, 23, 24.	V. lxxvii. 2.
xiv. 26.	III. vii. 1.
xiv. 34.	V. lxii. 2.
xiv. 36.*	IV. xiii. 1.
xiv. 36.*	IV. xiii. 9.
xiv. 40.	III. vii. 1.
xiv. 40.	VIII. ii. 2.
xv. 20.	V. xlii. 7.
xv. 21.*	V. lxviii. 12.
xv. 22, 45.	V. lvi. 8.
xv. 24.	V. lv. 8.
xv. 24.	VIII. iv. 6.
xv. 39.*	V. lxxviii. 2.
xv. 47.	V. lvi. 6.
xv. 48.	V. lvi. 7.
xvi. 1.	IV. xiii. 1.
xvi. 1.	IV. xiii. 5.
xvi. 1.	VII. viii. 3.
xvi. 2.	V. lxxi. 11.
xvi. 13.	I. vii. 7.
2 CORINTHIANS.
i. 11.	V. xxiv. 1.
i. 18.	II. iv. 1.
i. 21.	V. liv. 7.
ii. 6.	VI. iv. 1.
ii. 14-16.	V. xxii. 12.
iii. 3, 6.	VIII. vi. 7.
iii. 7, 8.	VII. xxiii. 6.
iii. 7, 8.*	VIII. iii. 2.
iv. 6.	V. xxi. 5.
iv. 17.	I. viii. 5.
v.	V. App. N°. 1. 33.
v. 1.	Jack. Ded.
v. 19.	V. li. 3.
v. 21.	Serm. II. 6.
vi. 5.	V. lxxii. 8.
vi. 9.	Serm. III.
vi. 14-17.	Serm. II. 1.
vi. 16.	Serm. VI. 9.
vii. 11.	VI. v. 6.
viii. 5.	VII. xxiii. 1.
ix. 7.	Serm. VII. 1.
x. 10.	III. viii. 10.
xi. 2, 3.	Serm. I.
xi. 3.	I. vii. 7.
xi. 27.	V. lxxii. 8.
xi. 28.	VII. viii. 3.
xii. 7.	Serm. III.
xii. 7-9.	V. xlviii. 3.
xii. 9.	V. App. N°. 1. 4.
xiii. 5.	Serm. III.
xiii. 5.	Serm. V. 13.
xiii. 7.	V. xlviii. 12.
xiii. 13.	V. lvi. 7.
GALATIANS.
i. 1.	VII. iv. 4.
i. 6.*	III. i. 10.
i. 8.	Pref. vi. 3.
i. 8, 9.	V. xxii. 10.
ii. 8.	VII. iv. 2.
ii. 11, 14.*	Trav. Sup.
ii. 20.	V. lvi. 10.
ii. 20.	Serm. III.
iii. 16.*	II. vi. 4.
iv. 5.*	Serm. I.
iv. 6.	V. lvi. 11.
iv. 6.	Serm. V. 14.
iv. 8.*	V. lxiii. 1.
iv. 9, 10, 24, 31.	Serm. II. 26.
iv. 10.	V. lxx. 7.
iv. 12.*	Serm. II. 39.
iv. 19.	Serm. III.
iv. 26.	V. l. 1.
v. 2.	Serm. II. 17.
v. 2, 4.	Serm. II. 19.
v. 2, 4.	Serm. II. 26.
v. 3.	V. lxiv. 4.
v. 19.	IV. xi. 7.
vi. 8.	I. xi. 1.
EPHESIANS.
i. 1.	V. lx. 3.
i. 3, 4.	V. lvi. 6.
i. 5.	V. liv. 3.
i. 7.	I. ii. 4.
i. 7.	V. lvi. 11.
i. 11.	I. ii. 5.
i. 11.	Serm. II. 31.
i. 11, 6.	Serm. III.
i. 14.	V. lvi. 11.
i. 14.	Serm. II. 26.
i. 20-23.	V. lv. 8.
i. 20-23.	VIII. iv. 3.
i. 21.	VIII. iv. 2.
i. 21, 22.	VIII. iv. 5.
i. 23.	Serm. II. 23.
i. 23.	V. lvi. 10.
ii. 3, 12.	V. lx. 3.
ii. 5.	Serm. II. 26.
ii. 8.	VI. vi. 10.
ii. 10.*	V. lxxii. 15.
ii. 12-16.	III. xi. 4.
ii. 14.*	IV. vi. 2.
ii. 16.	III. i. 3.
ii. 19-22.	Serm. VI. 15.
ii. 20.	Serm. II. 23.
iii. 2.	V. lxxvi. 10.
iii. 5.	V. App. N°. 1. 40.
iii. 6.	III. i. 3.
iii. 10.	I. iv. 1.
iii. 10.	I. xvi. 4.
iii. 10.	V. App. N°. 1. 7.
iii. 14-17.	Serm. III.
iii. 14.*	V. lxvi. 9.
iii. 15.	V. liv. 2.
iv. 5.	III. i. 3.
iv. 5.	I. x. 14.
iv. 5.	V. lxii. 4.
iv. 7, 8, 11, 12.	V. lxxviii. 9.
iv. 9.	V. lv. 8.
iv. 15.	V. lvi. 11.
iv. 15.	Serm. II. 23.
iv. 17, 18.	I. viii. 11.
iv. 23.	I. vii. 1.
iv. 23.	I. viii. 6.
iv. 25.	V. lvi. 11.
v. 8.	V. lx. 3.
v. 12.*	V. lxv. 6.
v. 14.	I. vii. 7.
v. 19.	V. xxvi. 3.
v. 19.	V. xxxix. 4.
v. 19.	V. xliii. 3.
v. 23.	V. lvi. 7.
v. 26.	V. lx. 1.
v. 29.	I. xiv. 3.
v. 29.	I. xvi. 3.
v. 29.	II. viii. 2.
v. 30.	V. lvi. 7.
PHILIPPIANS.
i. 1.	VII. ii. 2.
i. 1.	VII. ix. 3.
i. 1.	VII. xi. 1.
i. 6.*	V. xxii. 17.
ii. 8, 9.	V. lv. 8.
ii. 9.	V. liv. 3.
ii. 16.	Serm. II. 26.
ii. 17.*	Trav. Sup.
iii. 8, 9.	Serm. II. 6.
iii. 8, 9.	Serm. VI. 28.
iii. 11.*	V. lxviii. 12.
iii. 16.*	VII. xxiii. 11.
iii. 18, 19.	Serm. V. 2.
iii. 19.	I. xi. 4.
iv.	V. App. N°. 1.
iv. 12.*	Pref. iv. 3.
iv. 18.	VII. xxii. 5.
iv. 19.	I. ii. 4.
COLOSSIANS.
i. 15-18.	V. li. 3.
i. 16.	VIII. iv. 3.
i. 18.	VIII. iv. 2.
i. 18.	VIII. iv. 3.
i. 19.	V. liv. 3.
i. 21-23.	Serm. V. 14.
i. 23.	Serm. II. 26.
ii. 3.	I. ii. 4.
ii. 3.	V. liv. 7.
ii. 8.*	III. viii. 4.
ii. 8.	III. viii. 7.
ii. 9.	V. li. 1.
ii. 10.	V. lvi. 7.
ii. 16.*	V. lxxii. 15.
ii. 22.*	III. xi. 15.
iii. 4.	Serm. II. 26.
iii. 5.	V. lxxix. 1.
iii. 16.	V. xliii. 3.
iii. 24.	III. i. 4.
iii. 24.	Serm. V. 14.
iv. 1.	III. i. 4.
iv. 3.	V. lxxii. 8.
iv. 16.	V. xxii. 2.
1 THESSALONIANS.
ii. 7, 9.	II. viii. 4.
ii. 9.	V. lxxxi. 8.
ii. 17.	V. lxxxi. 2.
iii. 10.*	V. xxii. 17.
iv. 13.	IV. vi. 3.
iv. 17.*	V. lxviii. 12.
iv. 18.*	V. xxii. 17.
v. 12.*	VIII. i. 4.
v. 17.*	V. xxiii.
v. 21.*	Pref. i. 2.
v. 27.	V. xxii. 2.
2 THESSALONIANS.
ii. 10, 11.	V. App. N°. 1. 42.
ii. 10-12.	Serm. II. 19.
ii. 10, 11, 13.	Serm. VI. 2.
ii. 11.*	Pref. iii. 10.
ii. 13.	Serm. II. 29.
iii. 8.	I. viii. 8.
iii. 8.	V. lxxxi. 8.
1 TIMOTHY.
i. 5.*	V. xxii. 17.
i. 5.*	III. i. 2.
i. 17.	VIII. iv. 6.
i. 18.	VII. v. 10.
i. 20.	VI. iv. 1.
ii. 1.*	V. xxxii. 2.
ii. 3.	V. xlix. 1.
ii. 8.*	V. xxv. 3.
ii. 8.	V. lxxviii. 7.
ii. 8.	Jack. Ded.
ii. 9, 10.	Serm. VI. 8.
ii. 12.	V. lxii. 2.
ii. 15.	Serm. II. 26.
iii. 1.	V. lxxvii. 10.
iii. 1.	V. lxxvii. 13.
iii. 2.	V. lxxxi. 2.
iii. 2.	VI. App.
iii. 5.	VII. xi. 1.
iii. 10.	VII. xiv. 6.
iii. 15.	V. lxxviii. 7.
iii. 16.	V. lxviii. 6.
iii. 16.	Serm. II. 16.
iv. 3, 4, 5.	II. iii. 1.
iv. 8.	V. lxxii. 15.
iv. 10.	V. App. N°. 1. 33.
iv. 10.	VIII. iv. 6.
iv. 12.	V. lxxxi. 2.
iv. 14.	III. xi. 11.
iv. 14.	VII. ix. 3.
iv. 16.	VII. xxiv. 15.
v. 8.	I. x. 2.
v. 8.	II. viii. 2.
v. 9.*	III. xi. 11.
v. 9.	V. lxxviii. 11.
v. 9.	VII. vi. 2.
v. 14.	V. lxxviii. 7.
v. 17.	V. lxxxi. 16.
v. 17.	VI. App.
v. 17.*	VII. xviii. 6.
v. 17.	VII. xxiii. 6.
v. 18.	VII. xxiii. 7.
v. 19.	VI. ii. 1.
v. 19.	VII. vi. 8.
v. 19.	VII. xi. 6.
v. 20, 21.*	Trav. Sup.
v. 21.	I. iv. 3.
v. 21.	I. xvi. 4.
v. 22.	VII. vi. 3.
v. 22.	VII. xiv. 6.
vi. 8.	I. x. 2.
vi. 10, 11.	Serm. VI. 2.
vi. 13, 14.	III. xi. 11.
vi. 20.	III. xi. 11.
2 TIMOTHY.
i. 6. note	VII. ix. 3.
i. 12.	Serm. IV.
ii. 4.	VII. xv. 12.
ii. 13.	I. ii. 6.
ii. 15.	V. xxii. 12.
ii. 15.	V. lxxxi. 2.
ii. 15.	V. lxxxi. 11.
iii. 4.	V. lxiii. 2.
iii. 6.	Pref. iii. 13.
iii. 7.*	Pref. viii. 7.
iii. 7.	Serm. VII. 1.
iii. 8.*	I. xiv. 3.
iii. 12.*	V. xlviii. 13.
iii. 13, 14.	Serm. VI. 2.
iii. 14.	I. xiv. 4.
iii. 15.	I. xiv. 4.
iii. 15.	V. xxi. 3.
iii. 15.	V. xxii. 6.
iii. 15.*	Serm. II. 23.
iii. 16.	V. xxii. 10.
iii. 16.	II. i. 4.
iv. 1.	V. lxxxi. 2.
iv. 1, 2.	III. xi. 11.
iv. 5, 9.	V. lxxviii. 7.
iv. 7, 8.	III. xi. 11.
iv. 8.	I. xi. 3.
TITUS.
i. 5.	V. lxxx. 2.
i. 5.	VII. iv. 2.
i. 5.	VII. vi. 3.
i. 5.	VII. xi. 1.
i. 5.	VII. xi. 6.
i. 7.	V. lxxvi. 10.
i. 12.*	I. xiv. 3.
i. 9.	V. lxxxi. 2.
i. 9.	V. lxxxi. 5.
i. 9, 11.	III. viii. 8.
iii. 5.	V. lx. 1.
iii. 5.	VI. vi. 8.
iii. 11.	III. viii. 8.
PHILEMON.
19.	V. lxxvi. 10.
HEBREWS.
i. 2.	Serm. II. 9.
i. 3.	V. xlii. 7.
i. 3.	V. lvi. 5.
i. 5-13.*	II. vi. 1.
i. 6.	I. iv. 1.
i. 7.	I. iv. 1.
i. 9.	V. liv. 7.
i. 14.	I. iv. 1.
ii. 5-8.*	II. vi. 1.
ii. 8.	V. lv. 8.
ii. 9.	V. lv. 8.
ii. 9.	V. App. N°. 1. 33.
ii. 10.	V. li. 3.
ii. 16.	V. lii. 3.
ii. 17.	V. lxxvii. 2.
iii. 6.	III. xi. 2.
iv. 12.	I. xii. 2.
iv. 12.*	III. viii. 4.
iv. 12.	III. viii. 10.
iv. 12.	V. xxii. 10.
iv. 13.	V. App. N°. 1. 23.
iv. 15.	V. li. 3.
v. 1.*	VIII. i. 4.
v. 1.	VIII. viii. 6.
v. 4.*	V. lxii. 13.
v. 6.	V. lxxvii. 12.
v. 7.	Serm. IV.
v. 9.	VIII. iv. 6.
v. 9.	V. lvi. 7.
vi. 2.	V. lxvi. 4.
vi. 2.	V. lxvi. 9.
vi. 6.	V. App. N°. 1. 45.
vi. 6.	VI. vi. 15.
vi. 16.	Pref. vi. 1.
vi. 17.	I. ii. 6.
vi. 18.*	II. vi. 1.
vii. 3.	VII. xxiii. 1.
vii. 7.	V. lxvi. 6.
ix. 14.	V. lvi. 8.
ix. 25.	VIII. iv. 6.
x. 9.	V. lxxvii. 12.
x. 19.	V. xlvii. 4.
x. 20.	Serm. II. 23.
x. 24.*	V. xxii. 17.
x. 26.	VI. vi. 15.
x. 39.	Serm. V. 14.
xi. 6.	III. viii. 11.
xi. 6.	Serm. VI. 18.
xi. 21.	V. xlvi. 1.
xi. 25.	Serm. III.
xi. 33, 34.	Serm. VI. 17.
xi. 38.*	VII. xxiii. 3.
xii. 1, 12.	I. vii. 7.
xii. 11.*	V. xlviii. 13.
xii. 14.*	V. lxxii. 15.
xii. 22.	I. iv. 1.
xii. 22.	I. iv. 2.
xii. 22, 24.	VIII. iv. 6.
xiii. 4.	IV. xi. 7.
xiii. 5.	Serm. I.
xiii. 14.*	Serm. III.
xiii. 17.	VIII. App. N°. 1.
ST. JAMES.
i. 2, 3.	V. xlviii. 13.
i. 5.	Serm. VII. 3.
i. 13.	V. App. N°. 1. 29.
i. 14 (13?)	V. App. N°. 1. 29.
i. 17.	I. xvi. 1.
i. 17.	V. liv. 2.
i. 27.*	V. lxxii. 15.
ii.	Serm. II. 6.
ii. 1.*	Pref. i. 3.
ii. 1.*	Serm. II. 40.
ii. 2, 5.	Serm. V. 6.
iv. 3.	Serm. VII. 3.
v. 14, 16.	VI. iv. 5.
v. 16.	VI. iv. 7.
1 ST. PETER.
i. 12.	I. xvi. 4.
i. 12.	V. App. N°. 7.
i. 23.	Serm. II. 26.
ii. 8.	IV. xii. 2.
ii. 9.	VIII. iii. 6.
ii. 12.	II. ii. 3.
ii. 13.	VIII. App. N°. 1.
ii. 14.	VII. xvii. 2.
ii. 17.	VII. xvii. 2.
iii. 7.	Serm. VI. 12.
iii. 15.	III. viii. 16.
iii. 21.	V. lxiii. 3.
iv. 10.*	V. xxii. 17.
iv. 10.	V. lxxvi. 10.
iv. 17.	Serm. VI. 33.
v. 1.	V. lxxviii. 3.
v. 1, 2.	VII. xi. 1.
v. 2.	V. lxxxi. 2.
v. 2.	Serm. VI. 30.
v. 4.	I. xi. 3.
v. 8.	I. iv. 3.
v. 10.*	V. xxii. 7.
2 ST. PETER.
i. 4.	V. lvi. 7.
i. 19.	Jack. Ded.
i. 19, 20.	Serm. V. 2.
ii. 1.	V. App. N°. 1. 33.
ii. 4.	I. iv. 3.
ii. 4.*	I. xiv. 3.
ii. 5.*	I. x. 3.
ii. 12.*	Pref. iii. 3.
iii. 3.	V. ii. 2.
1 ST. JOHN.
i. 3.	VII. iv. 4.
i. 3.	VIII. iv. 6.
i. 5.	II. vi. 1.
i. 5.	V. App. N°. 1. 29.
i. 9.	VI. iv. 5.
ii.	V. App. N°. 1. 33.
ii. 9.	Serm. I.
ii. 16.	V. App. N°. 1. 29.
ii. 19.	Serm. V. 12.
ii. 19.	V. lxviii. 6.
ii. 20, 27.	V. liv. 7.
iii. 1.	V. lvi. 6.
iii. 4.	VI. vi. 8.
iii. 7.	Serm. II. 6.
iii. 9.	V. lvi. 11.
iii. 9.	Serm. II. 26.
iii. 14.	Serm. V. 13.
iv. 1.	Pref. iii. 10.
iv. 4.	Serm. III.
iv. 6.*	Pref. iii. 14.
iv. 15.	Serm. VI. 15.
v. 4, 5.	Serm. VI. 15.
v. 11.	V. lvi. 7.
v. 12.	V. lvi. 7.
v. 12.	Serm. VI. 17.
v. 12, 13.	Serm. II. 26.
v. 20.	V. li. 1.
v. 20.	V. liv. 3.
2 ST. JOHN.
i.*	V. lxiv. 3.
ST. JUDE.
6.	I. iv. 3.
6.	V. App. N°. 1. 29.
10.*	Pref. iii. 3.
10.*	I. x. 10.
12.	Pref. iv. 4.
17-21.	Serm. V. VI.
18.	V. ii. 2.
20, 21.*	V. xxii. 17.
22.*	Serm. II. 12.
REVELATION.
i. 3.	V. xxii. 15.
i. 5.	VIII. iv. 6.
i. 6.	VI. v. 3.
i. 6.	VIII. iii. 6.
i. 8.	VIII. iv. 6.
i. 10.	V. lxxi. 11.
i. 11.	I. xiii. 1.
i. 20.	V. lxxx. 2.
ii.*	III. i. 10.
ii.	VII. v. 2.
ii. 1.	VII. xi. 3.
ii. 2, 4.*	Serm. I.
ii. 4.	VI. iii. 3.
ii. 13.	III. i. 5.
ii. 19, 20.	Serm. III.
ii. 21-23.	Serm. II. 13.
ii. 24.*	Serm. II. 33.
iii. 8.*	Serm. II. 17.
iii. 14.	II. iv. 1.
iii. 20.	VI. iii. 2.
iv. 4.	V. lxxviii. 3.
v. 8.	V. xxiii.
v. 12.	V. lv. 8.
vi. 9.*	V. xxiii.
vii. 3.	V. lxv. 7.
vii. 3.	Serm. VI. 2.
viii. 10.*	III. viii. 4.
ix. 4.	V. lxv. 7.
ix. 11.	I. iv. 3.
xiii. 8.	Serm. II. 19.
xiv. 6.	I. xv. 3.
xiv. 13.	I. xiii. 1.
xiv. 13.	Serm. IV.
xv. 2, 3.	Serm. V. 14.
xv. 4.*	Pref. iv. 8.
xv. 6.*	V. xxix. 5.
xviii. 4.	Serm. II. 10.
xviii. 4.	Serm. V. 15.
xviii. 7.	Serm. IV.
xix. 10.	I. xvi. 4.
xx. 8.	I. iv. 3.
xxi. 8.	V. xvii. 2.
xxi. 14.	V. lxxviii. 3.
xxi. 14.	VII. iv. 4.
xxii. 9.	I. iv. 2.
xxii. 18.*	III. v.[730][737][792][793]
GLOSSARY. 
Incorporating Mr. Furnivall’s Glossary.
A.
Abate (an opinion), II. vi. 3.

abidden, V. lxxxi. 6.

ableness, V. App. 1. 1.

abroach, to set, V. ii. 2: Serm. II. 26, p. 522.

absolute (= perfect), II. vi. 1: V. lxxvi. 9: Serm. II. 31.

accidents, I. viii. 5; VIII. i. 5.

addle speech, III. viii. 10.

adjoin, I. x. 15.

adunited, VIII. i. 6.

affect, I. v. 2; x. 4.

affiance, V. lxv. 20.

afford, V. lxxv. 3.

after-meal, Serm. II. p. 488.

after-wit, VI. vi. 11.

agnize, V. lxxi. 11.

agree unto, VIII. iv. 6.

ancient, the, V. xxxvii. 2; xxxix. 5; xli. 2; xlii. 13; xlv. 1, etc.

ancient (plural), V. lxi. 1.

ancients = elders, IV. xiii. 9.

antichristianity, IV. iii. 2: Serm. II. 27.

“anvil, strike on this,” V. lxi. 3; lxv. 7.

apostata, III. i. 12 (1st ed.): VI. vi. 15.

appale (? = appal), Serm. I. p. 481: Serm. III. p. 606.

apparance, V. xii. 1; lx. 6 (1st ed.): Serm. II. 11.

apparency, V. lx. 6.

apparent (= manifest), VII. v. 4.

apparently (= manifestly), I. xiv. 1: IV. i. 1: V. xxii. 20.

appendent, III. iii. 4.

appliable, Pref. viii. 7: V. lv. 8.

art, word of, V. xviii. 1.

ascertain, I. xvi. 5: VI. vi. 17.

assay (noun), V. lxxi. 2.

assecure, V. lxii. 19.

assecured, V. lxii. 19: VI. vi. 1.

assoil, VI. vi. 5, 12.

attendance (= waiting), Serm. VII. p. 706.

attendancy, VII. xx. 4.

available, I. x. 8: II. vii. 8.

awe (= protection), VII. xxiv. 22.

B.
Bag and baggage, II. v. 7.

baggage, Serm. VI. p. 684.

bane (verb), V. xv. 2.

battle (adj.), V. iii. 4, vid. note 2.

bear in hand, VI. vi. 13.

beat on, II. iv. 3: V. ii. 1.

befool, V. lxxvii. 5.

behoveful, I. viii. 9; x. 4: III. x. 8.

being, III. xi. 20: V. ii. 1; lxviii. 6.

bent (strain), V. xxxii. 4.

better foot of a lame cause, Serm. II. p. 537.

bishoply, VIII. vii. 1: Serm. II. 32.

blockish, Serm. III. p. 603.

bloom (verb active), V. iii. 4.

bolster, III. viii. 8.

bolt out, V. lxv. 15.

bonny, Serm. VII. p. 705.

bow of two strings, V. lxxx. 9.

brayed, V. xxii. 12.

briars, out of the, IV. iv. 1: VI. vi. 13: VII. viii. 5; xiv. 2.

burdenous, Serm. II. 9.

C.
Cam (or kam), Serm. III. p. 599.

captivate, Serm. II. 28.

card, I. ii. 5: Serm. IV. p. 652.

casteth him therewith in the teeth, IV. ix. 1.

casualties, Pref. vii. 11.

cecity, Serm. III. p. 602.

chair, “imagination, the peculiar chair of memory,” V. lxv. 7.

chariness, Serm. I. p. 473.

check, IV. xi. 3.

[794]
chiefty, VII. ii. 3; vi. 6: VIII. ii. 11; vi. 12.

circuitions, V. ix. 2.

circulatory, V. liii. 4.

circumstance of the place (= context), II. v. 3; = qualification, V. xlv. 2.

civet (sp. civit), Serm. VI. p. 685.

civil, I. xv. 4.

clay colour, IV. xiii. 6.

coaction, V. lxviii. 10.

coagmentation, VIII. ii. 2.

coat, “any of that,” Serm. II. 27, p. 526.

co-efficient (noun subst.), V. App. 1, p. 554; (adj.), Serm. II. 4.

collection, V. lviii. 4.

commander, V. lxxi. 4; lxxix. 14: VIII. ii. 1; vi. 13.

commandress, V. viii. 1.

commissionary, VIII. viii. 3.

common sense, I. vi. 5; viii. 4.

commonwealth’s-men (= citizens), VII. xxiv. 20.

complement, V. lviii. 4; lxiv. 4; lxv. 5: VIII. iv. 3.

conceit = opinion (passim), spelled “conceipt,” v. V. xlvii. 2.

conceited (“strongly”), V. i. 3.

concinnate, Serm. III. 4.

condescend (= agree), V. lxxix. 9: Serm. V. p. 668: VII. p. 705: VIII. ii. 7, 11.

conditioned, VII. xviii. 10.

conjuring, V. lxxxi. 2.

conjuring exhortations, V. lxxxi. 2.

conscience = consciousness, II. vii. 2: VI. iv. 6.

consort, V. v. 1.

conster, III. v. 1 (1st ed.): IV. xi. 7; xii. 3.

contentation, I. xi. 4.

continent, V. lxxix. 7.

continent, “of all she possesseth,” V. lxxix. 7.

control, II. vii. 10: V. lxi. 3; lxv. 15; lxxvii. 5: VII. xi. 11.

conveniency, II. iv. 5.

convented, VII. xxiii. 4.

conveyance, III. v. 1; vii. 5.

convocate, VII. viii. 12.

cope, Serm. V. 15.

copesmates, VI. v. 9.

corps, V. lxxx. 11.

corrosive, IV. x. 1.

corse, V. lxxv. 4.

countenance, III. xi. 18.

countervailed, V. ix. 1.

course, words of, III. xi. 7.

crazedness, Pref. iii. 8.

crime = charge, Serm. II. 39.

curry favour, IV. vii. 4.

D.
Dally, VII. xv. 10: VIII. ix. 5.

damnify, V. lxxxi. 16.

decease, III. x. 2.

defeat, I. iii. 2: III. i. 12: V. xxii. 13; lxii. 13.

delicates, IV. vii. 2.

demi-premisses, V. lxxxi. 4.

demurely, Serm. II. 17.

deodate, VII. xxii. 4.

derive, V. lxxvii. 8: VIII. vi. 11.

detecteth, VI. iv. 9.

device, V. Ded. 2.

devolution, VIII. vi. 14.

dint, Pref. iii. 3.

dirity, Serm. III. 5.

disauthorize, vol. iii. p. 467.

dischurch (an “unusual word”), G. Cranmer, vol. iii. p. 111.

dischurched, VI. App. p. 111.

discoherence, vol. iii. p. 633.

discommend, V. xlvi. 1: VII. xv. 6.

discommodious, V. lxxi. 8.

discommoned, VIII. i. 6.

discourse, I. vi. 4; vii. 7; xiv. 1.

discover, I. i. 2; iii. 4.

disgorge, V. lxiv. 6.

disgrace, Pref. iii. 3: I. vii. 7; xvi. 2: II. i. 4: III. viii. 4: V. xxxiv. 3; xxxviii. 3; lxii. 14.

dislike, II. vii. 2.

disparagement, Serm. III. p. 603.

displeasant, Pref. vii. 3.

distract, VIII. i. 4, 6.

disusage, IV. xiv. 3: VII. xxiv. 11.

ditty, V. xxxviii. 1.

dive (= dip), IV. xii. 3.

divinely, I. viii. 1.

divisibly, VIII. iv. 6.

E.
Economy, V. liv. 6.

elevate (= disparage), II. vii. 8: V. lx. 3.

elide, IV. iv. 1.

embase (“imbase,” Serm. II. 34), VII. xi. 9.

emprese, Pref. iv. 3.

enable, III. viii. 10: V. xxvii. 3.

endammage, V. xlii. 12.

ensignes, V. lxiv. 6.

ensorceled (reading of 1st ed.), Serm. I. p. 477.

ensue (transitive), V. lxv. 18.

enthronize, VI. vi. 13: VIII. ii. 13.

epicure, VIII. ii. 15.

every of these, I. xvi. 5: III. xi. 13.

evict (= prove), VIII. ix. 5.

evitable, I. viii. 8.

exagitate, III. xi. 16.

excellency, IV. ix. 3.

exequies, V. lxxv. 4.

[795]
exhibit, V. lxiv. 5; lxvii. 6.

exigent, Pref. iv. 7.

exorbitant, III. xi. 8; Serm. III. 1.

exquisite, I. v. 2: Serm. I. p. 475.

extemporality, vol. iii. p. 464.

extraordinancy, VII. xv. 8.

extreme, V. ix. 1.

F.
Fact, III. xi. 15.

fall into (= be incident to), I. xi. 3: III. x. 3.

famously (= notoriously), VI. iv. 10, 13: VIII. ii. 11; ix. 5.

feeling, I. xii. 2: V. xxxix. 1; li. 3 (sense): VII. xxiv. 15.

festination, Answ. to Trav. 21.

flannel, “for gold hath flannel,” V. lxxix. 16.

flit, Serm. II. 26, p. 517.

float, Pref. ix. 4: V. lxxi. 7.

foot, “set the better foot of a lame cause foremost,” Serm. II. 33.

foreceable, V. App. 1. 33.

forcible (= efficacious), V. xxii. 6, cp. V. lxvii. 1; VI. iv. 13: VII. xv. 14.

foreprized, V. lxxi. 4.

foreslow, VI. iii. 4; iv. 6.

forlorn, V. lxv. 17; lxvi. 9.

formal, V. lxiv. 4.

formalize, V. lvi. 11.

formally, Serm. II. 21.

forsaken, III. i. 8: V. xlii. 13: VI. iii. 5: VII. xvi. 9: Answ. to Trav. 22.

forth (“have their forth?”), V. lxii. 8.

fortuned, VII. v. 5.

framable, vol. iii. p. 696.

frankness, V. lxxii. 11.

frequent = crowded, V. lxxx. 7.

frier’s-gray, IV. xiii. 6; opposed to “clay-colour.”

from (= away from), VIII. iv. 7.

fumbling shifts, V. lxii. 14.

fumingly, V. xxii. 7; lxii. 21.

furious (= mad), I. ix. 1: V. lxiv. 4: VI. v. 8.

G.
General (of the whole kind), II. viii. 1: IV. vi. 3: V. lv. 1.

genitive (1) no mark of inflexion, “work sake,” Pref. i. 1; “distinction sake,” V. iv. 3 (in old edd.); (2) “his,” “Novatianus his conceit,” V. lxii. 5; “Dionysius his navigation,” lxxix. 15; “Glaucus his charge,” lxxix. 16.

gentility (heathendom), V. ii. 4.

gestured, V. xxvii. 1.

girdler, VII. viii. 11.

glass, Pref. vii. 1.

“Glaucus his change,” V. lxxix. 16.

glorious, V. lxxi. 7.

gloses, V. xxii. 10; lxii. 14.

glosing, V. iv. 2.

gravelled, VI. iv. 12.

grisly, VI. vi. 15.

guard (ward), V. lxiv. 6.

H.
Habilitie, Serm. III. 2 (so passim in original edd.)

handfast, Serm. I. vol. iii. p. 476.

hands, “work with two hands,” Serm. II. 33.

handsel, V. lvi. 11.

happily, for haply, Pref. ii. 3, and passim in 1st edd.

haps, VI. iv. 6.

hardlier, V. lxxxi. 6.

“have their forth” (?), V. lxiii. 1.

heaved at, VII. xxiv. 2.

her (instead of its), “the appetite,” I. v. 3; “discipline,” Pref. viii. 3: V. lxiii. 1.

heteroclites, vol. iii. p. 605.

his (instead of its), “that which is of God—his kind,” I. xiv. 5; “operation,” I. xvi. 5; “creature,” V. lv. 2; “body,” lviii. 4.

hold out with, III. xi. 19.

hungry, V. xxii. 19.

I.
Idea, I. iv. 1.

idol (so, 1597-1616, instead of “idle”), V. xi. 3, vid. note 1.

ifs or ands, Pref. ii. 6.

illation, Serm. II. 9.

imbecility, V. xxii. 17; xxv. 1.

imbreathed with, vol. iii. p. 611.

impaled, VIII. i. 4.

impardonable, V. lxv. 1.

implead (raise a plea against), VI. iv. 10.

implement, I. x. 2.

import (“their souls”), VI. iv. 15.

impotent, Pref. ii. 4: IV. ix. 1.

impreparation, V. ii. 2.

impression, I. iii. 3; xvi. 3.

improve (= improbare), V. xxii. 10; vol. iii. p. 699.

imps (and limbs of Satan), III. i. 7.

incantations, IV. iv. 1.

incident into, I. iii. 3: II. iii. 1; vii. 5: V. lxii. 14 (passim) (unto only in the later xviith century edd.)

inclinable, VII. xiv. 2.

incommodious, IV. vi. 2.

inconformitie, IV. xi. 4.

[796]
incredibility, V. App. 1. 36.

indifferent, II. i. 3.

infested, III. xi. 9.

inflammations, V. xxxiv. 1.

infringe, V. lxxxi. 3.

ingenuity, V. xx. 20.

injuried, I. i. 2, 4: V. xvi. 1.

injury (to), III. viii. 9: VI. iii. 3; v. 2.

inn, V. lxvii. 10.

innocent (n.), I. vi. 3: III. viii. 11: V. lx. 7; lxiv. 2, 3, 5.

inrailed, IV. xiii. 7.

instinct, VII. v. 7.

intentive, I. xi. 4.

intercourse (= alternation, “day and night”), Serm. I. vol. iii. p. 474.

interessed, V. Ded.; V. xl. 3; lxiv. 5; lxxx. 9.

interlace, V. xxvi. 2; lxii. 14.

inure with, I. i. 2; vi. 3: III. xi. 3: V. xl. 3: Answ. to Trav. 16.

inure unto, V. xlii. 11.

invective (adj.) V. lxxii. 12.

irefully, Pref. ii. 2.

its (I. iii. 5: V. xxix. 6 in Keble’s text; but in early edd. the).

J.
John a Style, Serm. II. 35.

judicials, I. xv. 1: III. x. 4.

jump, I. viii. 8.

K.
Kind, “grown out of,” vol. iii. p. 698.

known of his faults, VI. iv. 5.

L.
Lapt up, Serm. VI. 28.

leave or liking, Pref. viii. 13.

leisurable -y, V. xlvi. 1, 2.

“let or hindered,” I. ii. 6: II. ii. 3.

lets, I. i. 1.

lifted at, VII. xxiv. 26.

like of (to), I. iv. 3: VIII. vi. 14.

limbs, III. i. 7.

list (= border), V. xx. 10.

list (v.), V. xxii. 9; lxx. 1; lxxi. 4; lxxvii. 3.

listed, VII. viii. 4.

litigious (of things, “feast of Easter”), IV. xi. 12.

livery, V. lxxi. 7.

loam, “wash a wall of,” Serm. II. 19.

loden, Pref. iii. 13.

long of them, V. i. 1.

look (interj.), I. viii. 10: VII. vi. 9: VIII. vi. 6.

loover, Pref. iv. 4.

M.
Maims, IV. xii. 6: V. lxv. 7; lxx. 4.

malapert, VII. xv. 15.

malignants, III. vii. 10: V. ii. 4.

manner (“all, no, manner”), Pref. viii. 6: I. iv. 1; viii. 10: II. vii. 4; v. 2: V. vi. 1; liv. 7: VIII. ii. 13.

manuary, V. lxxxi. 8.

manumised, V. lxxxi. 15.

markable, VIII. vi. 14.

mast, VIII. iii. 2.

medled (= mixed), IV. viii. 1.

meeken, to, vol. iii. p. 623.

mel-pell, VIII. ix. 5.

merry, “more merry than wise,” V. lxxiv. 3.

meslin, IV. vi. 3.

mettle, “softer,” V. lxv. 6 (cf. lxv. 15, 1st ed.; lxxix. 5).

mincingly, I. xi. 6.

minerals, I. iv. 3.

mingle-mangle, Serm. V. 7.

miscollecting, vol. iii. p. 595.

misconceit, Pref. i. 2: III. i. 10: VIII. i. 4.

miscreants, III. i. 8: V. lxiii. 1: VI. v. 8.

misdesert, V. lxxvii. 3.

misdistinguish, III. ii. 2; iii. 1.

miserable (= miser), V. lx. 20.

misinfer, V. lii. 4.

misordered, VI. v. 9.

moe, Pref. ix. 4: II. v. 5: III. v. 1; xi. 21: IV. ii. 2; xiii. i. 9: V. Ded. p. 3; xxii. 8; xxxv. 2; lvi. 10; lxxviii. 12; lxxx. 4, 11.

momentany, I. viii. 5.

montanize, IV. vii. 4.

mother-cause, v. I. iii. 2; viii. 6: VIII. ii. 12.

mother of life, vol. iii. p. 650.

mother-sentence, Serm. II. 36.

motioner, VIII. viii. 4.

N.
Namely, VI. iv. 4.

natural, “a mere n. man,” III. viii. 6.

nemo scit, a, V. lxxix. 5.

nephew (= grandson), V. xx. 11: VI. App. p. 133.

nice, made it not, VI. iv. 2.

nocive, Serm. IV. p. 649.

noon’s meal, V. lxxii. 6.

not no, I. xii. 2; III. xi. 9.

no, not, V. xxii. 14; lxxi. 8.

no, no, VI. iv. 14.

note, (of this n.), vol. iii. p. 481.

notional, V. lxxxi. 5.

nuzzled, Answ. to Trav. 26.

O.
Object (adj.), Serm. I. vol. iii. p. 478: E. P. III. i. 2: IV. i. 3.

observants, I. iv. 1.

[797]
occasioned to, Pref. ii. 1.

occurrents, V. Ded. i. 3.

odds (sing.), what odds there is, I. viii. 2; VIII. iv. 5.

opinative, V. lx. 5.

opposite (subs.), I. xvi. 5: III. xi. 9: IV. vii. 6: V. vii. 3.

oratorial, III. viii. 9.

organize (“soul of the body”), V. lviii. 1.

orient, VIII. ii. 8.

over-carried, vol. iii. p. 565.

overcast, to (v.), V. xxxii. 3.

over-having (“an over-having disposition”), VII. xxiii. 5.

over-seeing (cf. oversight), vol. iii. p. 607.

overseen, I. viii. 3.

oversight, III. i. 9, 12.

overskip, Pref. iii. 2.

overslip, V. lxxii. 14.

oversway, IV. xiii. 9.

P.
Pageants, V. lxxvii. 14.

pale (“the common pale”), IV. xiii. 7.

panical (terrors), vol. iii. p. 615.

paramount, paravaile, Serm. II. 28, p. 527.

participate (transitive), V. lxv. 20; lxxi. 4.

party, IV. i. 4: V. i. 3; xliii. 3; lxxx. 5, 12: VI. iv. 7; v. 8.

peers (“two cases be peers”), V. lxii. 13.

pensive, V. lxxii. 1: VI. iii. 3; iv. 6; v. 4.

perceivance, Serm. I. vol. iii. p. 477.

permit to, Serm. II. 27, p. 525.

person (= mask?), V. ii. 3.

petit, V. lxxiv. 4.

petitionary, V. xlviii. 2.

pew-fellows, VI. iv. 10.

phrenetical, V. App. 1. 38.

pin, III. iv. 1.

pinch, IV. xiii. 1, 9.

pitch, a field, V. xxxi. 1.

platform, III. vii. 4.

politician, VII. xxiv. 3.

politicly, VII. xxiv. 22.

politics, V. App. 2. 5: VII. xxiv. 22.

powerable, VII. xviii. 9: Serm. II. 11.

preach, a (n.), V. xxviii. 3.

precincts (= limits), V. lxxx. 2 (v. VII. ii. 1; viii. 3).

preconceit, Pref. iii. 9.

predicants, Serm. II. 12.

prejudice, Pref. ii. 8: I. vii. 6; x. 13.

prest, Serm. IV. p. 649.

pretence, V. lxii. 8.

pretend (= put forward), II. v. 6; (= claim) VI. v. 1.

prettily, V. xxii. 7.

prime, II. iv. 6: V. lxv. 2: VII. xxii. 6.

proctor, IV. ix. 3: V. Ded. 7.

propense, Pref. iii. 13.

puddle, Serm. II. 28, p. 528.

puissance, V. xxxviii. 1.

punned (= bruised), V. xxii. 12 (T. C.)

purchase (= obtain), Pref. ii. 8: I. vii. 1: V. xxiii.

purity, I. iii. 4.

put up injuries, VII. xv. 3.

Q.
Querulous (= quarrelsome), III. xi. 10.

quite and clean, I. xii. 3: III. i. 8, 13: IV. xii. 7: V. ii. 1; xx. 2: VIII. i. 6.

quite, to, I. xi. 5.

R.
Race, “the race of Christ,” IV. v.; V. lvi. 11.

rake (“billows raking a boat”), Travers, vol. iii. p. 550.

ransom, put to his (opposed to “amerced” or fined), III. p. 552.

readunited, VIII. i. 6.

reasoned, be, VII. xv. 14.

rebukeable, IV. vii. 5.

recharge, III. xi. 13.

recidivation, Serm. VII. 1.

reckless (= rechless, wretchless), V. lxxi. 9.

recognizance, V. xlii. 10.

recomforted, V. lxxv. 3.

redeem, IV. xiv. 3.

refel, VI. vi. 6.

regalities, VIII. vii. 7.

rein, “the question shorter,” Serm. II. 28, p. 527. Cp. V. xliii. 4.

religion, “they of the religion in France,” IV. viii. 4.

rely myself on, V. lxvii. 12.

remonstrances, V. lxxvi. 6.

remorse, V. lx. 6.

rent (verb), VI. iii. 5.

resemble unto us, I. iv. 1: V. vi. 2; xxxviii. 1: III. iii. 4 (?).

respected (regarded), III. xi. 20: V. lxvi. 4.

respective, V. i. 1; xxix. 8.

respectively (with respect of person), Pref. ii. 4.

revisit (= review), vol. iii. p. 564.

rewardable, I. ix. 1.

riotous, I. xiii. 3.

rise, III. viii. 10: V. iii. 4.

ruff (“in their chiefest r.”) Serm. III. 4.

[798]
S.
Sabboth, for Sabbath, III. viii. 10, vid. note: IV. xiii. 1, passim.

safeconduct, to, vol. iii. p. 693.

sallet (1st and 2nd edd. = salad K.), V. lxxvi. 8.

salt apology, VI. vi. 6.

sanctimony, VI. v. 6.

says (= essays), V. lxx. 4 (cf. lxxi. 2).

scapes and oversights, Serm. II. 39.

scholies -y, III. viii. 2, 16; xxxi. 3; lxxxviii. 2.

scholy, to (v.), V. xxii. 7: VI. iv. 10.

scopious (?), Serm. III. p. 623.

sea (“a sea of such matter”), vol. iii. p. 587: E. P. Pref. i. 1; viii. 11: I. xi. 3: V. lxxi. 7: VI. iii. 3.

sear, VI. iv. 6.

sedulity, V. iii. 1.

seedsmen (= sowers), VIII. ii. 8.

several, I. x. 13; xiv. 3: IV. xiii. 1: V. xiv. 1.

severed from, IV. xiii. 1.

shadowish, VIII. iii. 1.

shame (= to be ashamed), VII. xxiv. 22.

“shorten the reins of their censure,” V. xliii. 4.

“shorter commons,” V. lxxviii. 5.

side respect, I. x. 7.

side (= page), vol. iii. p. 661.

silly, III. viii. 10; xi. 8.

sith, IV. xii. 5 and passim.

skilleth, III. vii. 3.

sleight, Pref. viii. 10.

slight, Pref. iii. 16.

slips (branches), II. i. 1: V. lxxviii. 5: VI. xv. 13.

slought, V. App. 1. 5.

smally, III. xi. 5.

soder, V. xxix. 7.

soon or sine, Serm. III. p. 627.

soonest, with the, VII. xiii. 2.

sophisticate, V. lxxvii. 14.

sound that way, Pref. iii. 9: V. ii. 1; xx. 12; xxxviii. 3.

sound to, Pref. iii. 9: V. xxxviii. 3.

sound towards, Pref. iii. 9: V. xx. 12; xlii. 10: VIII. ii. 15.

spit-venom, V. ii. 2.

sponged out, V. xix. 2; lxvi. 9.

square (out of), III. i. 10: V. lxv. 11; lxxxi. 1.

stand to, Pref. viii. 2: I. viii. 7.

stand with, III. ix. 3; xi. 18: V. lxii. 22.

stand upon, II. iv. 1: III. iii. 4; v. 1: V. lxv. 20: VII. xxiv. 13: Answ. to Trav. 8.

stand in stead, III. ix. 1: V. lxxii. 2.

“stews of idols,” V. lxii. 17.

stint (= limit), IV. xiv. 3: VII. vxxiii. 10, 11.

stomach, Pref. ii. 6: II. v. 7: V. xlii. 2.

stormingly, V. App. 1. 44.

stroke, VIII. vi. 13.

stupidity (ἀναισθησία), VI. vi. 6.

sugared, vol. iii. p. 474.

suit (of one), Pref. viii. 7: III. iii. 2.

sup up words, V. lxii. 14.

suppage, V. lxxii. 6.

suppled, V. lxviii. 11.

suppositum, vol. ii. p. 230, note 1.

suspense (adj.), Pref. ii. 2.

T.
Teeth, “from the t. outwards,” V. lxviii. 6.

teeth, “with t. and all,” VIII. vi. 2.

tempts (= attempts), V. lxxvi. 7.

tender (v.), IV. xi. 5.

tenor, IV. ii. 2.

tenure, I. iii. 2: III. i. 12.

thought, “half a thought the better,” IV. xiii. 10.

too too (manifest), Serm. II. 29, p. 528; vid. T. C. quoted, vol. ii. p. 94.

touch, hold the, VI. iv. 11.

touch of mercy, V. li. 3.

touch of his person, vol. iii. p. 559.

toy, IV. i. 3: V. lix. 3.

toyish, V. lxiv. 1, 4.

tract of time, IV. xiv. 1: V. xxix. 7; lxxviii. 5.

treatable, V. xlvi. 1: -y, V. lxxix. 16.

trencher-mates, V. ii. 2.

U.
Unbeseeming, I. viii. 9.

unbuilded (conclusions), II. vii. 5.

uncapable, I. iii. 3.

uncomfortable, I. iv. 1.

uncommanded, VIII. ii. 1.

unconscionable -ly (= against conscience), VII. xxiv. 25.

unconsonant, V. li. 3.

uncredible, vol. iii. p. 476.

unculpable, III. vii. 2.

underlie, VIII. i. 2.

underset, V. xv. 5.

undispensable, VII. xiv. 4.

undistinctly, V. lxviii. 9.

undividable, VII. xxiv. 20: Serm. III. 4.

unemptiable, II. i. 4.

unfallible, VI. vi. 5.

unforcible, V. lxv. 9.

unframable, I. xvi. 6.

ungroundedly, vol. iii. p. 627.

unindifferent, IV. vii. 4: vol. iii. p. 550.

unlapt, vol. iii. p. 567.

[799]
unpartial, VI. vi. 5.

unperfect, V. ix. 3.

unrequisite, III. xi. 16.

unresistable, Pref. ii. 3: V. i. 3; lxi. 4.

unsensible, VII. xiv. 2.

unseparable, VII. xxii. 5.

unstrengthened, V. viii. 4.

unsubject, VIII. ii. 13; viii. 1.

unweariable, Pref. iii. 12.

upshot, V. lxv. 12: VI. v. 5.

ure, “put in,” Pref. ii. 2: II. vii. 9: V. lxxiii. 8: Serm. II. 11.

— out of, VII. xiv. 2.

V.
Vice-agent, V. xli. 1.

voyage (French sense), V. lxxix. 7.

vulgar, VI. vi. 8.

W.
Wade, I. ii. 2; iii. 2: V. lxv. 13; lxvii. 4; lxxxi. 4: Serm. II. 37.

warfaring, VIII. iv. 6.

warrantize, Serm. V. 11.

wash a wall of loam, Serm. II. 19.

wave in and out, V. xliii. 5.

weeds, IV. xiii. 6: V. xxix. 3; lxx. 4.

well-willers, V. lxxii. 14; lxxvi. 2: VIII. iv. 10.

willinger (more willing), V. i. 2.

withal = with, III. xi. 10: IV. vii. 2.

woe worth, Serm. II. 13: Serm. V. 6: VI. 10.

worsed, vol. iii. p. 679.

wreath (= strand of a cable), VII. xviii. 10.

wretchless, Serm. VI. 33.

wringeth, V. App. ii. 4.

writhed, IV. xiii. 5.

the end.
[800]
a
The words which is, are inserted from the Dublin MS. (which will be designated in these notes by the letter E.)

1
[Although the present editor is convinced, for the reasons assigned in the preface, that the sixth book completed by Hooker is now almost or altogether lost, still he has judged it best on consideration to leave the following pages in their usual place: first, because the early part of them does appear to have formed part of some rough draft of the book on lay elders; secondly, because it seemed safer to await the judgment of literary men in general, before expunging so large a portion of the treatise: thirdly, because he believes the whole to be Hooker’s, though wrongly inserted into his great work.]

b
conflicts Ed. 1651.

c
striving E.

1
[After 1593, in which year were published the first portion of Hooker’s work, and the two treatises of Bancroft, there was a pause for a while in the Puritan controversy.]

2
[See Pref. iv. 5: and note 17 (?87, which is note 1, p. 160, in vol. i. of this Ed.).]

3
Lib. vi.

4
Lib. vii.

5
Lib. viii.

d
not om. E.

e
only om. E.

6
[It may seem that there is some omission here: for the following sentence implies that a summary had been given of the Puritan “plot set down for the office of the ministry,” as being the end, for which the objections about ceremonies were a pretence, and the agitation for lay elders a mean.]

f
suppose E.

g
favour it the more. Fulman in the margin of a copy of the first edition in C. C. C. library.

7
[See Bancroft’s Dang. Pos. b. iv. c. 12.]

h
so om. E.

i
bear E.

1
[Eccl. Disc. fol. 120-125.]

2
Numb. xvi. 3.

k
doth D.

l
sufficient om. E.

1
Acts xx. 28.

2
1 Tim. v. 19.

3
Mark xvi. 15; Matt. xxviii. 19; 1 Cor. xi. 24.

4
Τίμα μὲν τὸν Θεὸν, ὡς αἴτιον τω̑ν ὅλων καὶ κύριον· Ἐπίσκοπον δὲ, ὡς ἀρχιερέα, Θεου̑ εἰκόνα ϕορου̑ντα· κατὰ μὲν τὸ ἄρχειν, Θεου̑, κατὰ δὲ τὸ ἱερατεύειν, Χριστου̑. Epist. [interpol.] ad Smyrn. [c. 9.] [This note in E forms part of the text.]

m
iii. D.

n
limation made limitation by Abp. Ussher in D.

o
ampliation D.

p
inconstant E.

q
bringeth E. which spoils the sentence. Fulman conjectured offered; though instead of offered through.

r
predecessor E.

s
doctrine E.

t
concerns E.

u
step towards om. E.

x Penitency E.
y
iv. D.

z
then om. E.

a
toward E.

1
[This clause, “in matters of ecclesiastical cognizance,” is no doubt inserted with especial purpose of qualifying the general expression before, of “reforming all injuries, &c.:” and so avoiding the claim of extreme prerogative, which the Puritans urged in order to draw all causes into their spiritual courts. See Pref. c. vii. 4. In the statement supposed to be the Lord Keeper Puckering’s, Stryp. An. iv. 201, among other opinions held by the Puritans against the state and policy of the realm, is set down, “That all matters arising in their several limits, (though they be mere civil and temporal,) if there may happen to be breach of charity, or wrong offered by one unto another, may and ought to be composed by the eldership.”]

b
sin deprives E.

c
farther E.

1
“Pœnitentiæ secundæ, et unius, quanto in actu [arcto] negotium est, tanto operosior [“potior” E.] probatio est, ut non sola conscientia proferatur, [“præferatur,” D.] sed aliquo etiam actu administretur.” “Second penitency, following that before baptism, and being not more than once admitted in one man, requireth by so much the greater labour to make it manifest, for that it is not a work which can come again in trial, but must be therefore with some open solemnity executed, and not left to be discharged with the privity of conscience alone.” Tertull. de Pœnit. [c. 9.]

d
the E.

1
[Judging by internal evidence, (which is almost all we have,) it may perhaps appear that at this point, if not before, the collections of Hooker for the 6th book cease, and that what remains is taken indeed from papers of his, but wrongly assigned to a treatise on lay-elders.]

e
v. D.

2
[Rev. iii. 20.]

3
[Comp. Fragment of an Answer to a Christian Letter, above, t. ii. p. 540.]

4
[See E. P. V. lvi. 12.]

5
Acts ii. 37. [om. D.]

1
Jonah iii. 5. [om. D.]

f
f. wrought. Fulm.

2
S. Matth. xi. 21. [om. D.]

3
[Acts xix. 17, 19.]

g
any E.

h
endeavours E.

4
[Rev. ii. 4.]

1
now om. E.

1
Cassian. Col. 20. c. 4. [Bibl. Patr. Colon. t. v. p. 2. 206. “Ita ut Deo, præteritis facinoribus offenso, jamque justissimam pœnam pro tantis criminibus inferenti, (si dici fas est) quodammodo obsistat, et quasi inviti (ut ita dixerim) dextram suspendat ultoris.”]

2
Basil. Episc. Seleuc. (circ. 451,) p. 106, [ed. Commelin. 1596.] Φιλάνθρωπον βλέμμα προσιου̑σαν αἰδει̑ται μετάνοιαν. Chry. in 1 Cor. Hom. 8. [§ 4. t. x. p. 71 C. ed. Bened.] Οὐ τὸ τρωθη̑ναι οὕτω δεινὸν, ὡς τὸ τρωθέντα μἠ βούλεσθαι θεραπεύεσθαι. Marc. Erem. († circ. 410) [de Pœnit. ap. Biblioth. Patr. Par. 1624, t. i. p. 915 D.] Οὐδεὶς κατεκρίθη, εἰ μὴ μετανοίας κατεϕρόνησε, καὶ οὐδεὶς ἐδικαιώθη, εἰ μὴ ταύτης ἐπεμελήσατο.

3
Fulg. (Bp. of Ruspe, 467-533,) de Remis. Peccat. lib. ii. cap. 15. [“Ecce Saul dixit, Peccavi; David quoque dixit, Peccavi. Cum ergo in confessione peccati utriusque una vox fuerit, cur non una est utriusque concessa remissio? nisi quia in similitudine confessionis videbat Deus dissimilitudinem voluntatis.” in Bibl. Patr. Colon. vi. 119.]

k
God’s D.

1
Jon. c. iii. 9.

2
[Luke xv. 18.]

l
vi. D.

3
Senten. lib. 4. d. 14.

m
vii. D.

n
nor slight E.

o
viii. D.

p
A space of half a page is left here in D.

q
i. D.

1
Matt. xvi. 19.

r
therein E.

1
Matt. xviii. 17.

2
Matt. xviii. 18; John xx. 23; 1 Cor. v. 3; 2 Cor. ii. 6.

3
1 Tim. i. 20.

s
ii. D.

t
their.

u
other E.

x
iii. D.

1
[Concil. Later. iv. ad 1215, under Innocent III. can. 21. “Omnis utriusque sexus fidelis, postquam ad annos discretionis pervenerit, omnia sua solus peccata confiteatur fideliter, saltem semel in anno, proprio sacerdoti et injunctam sibi pœnitentiam studeat pro viribus adimplere, suscipiens reverenter ad minus in Pascha Eucharistiæ sacramentum.” t. xi. p. 172, 3.]

y
a thing thus made E.

z
Penitence E.

2
Soto [Spanish Dominican, 1494-1560,] in iv. Sent. d. 14. q. 1. art. 1. [“Est enim pœnitentia, sacramentum remissionis peccatorum quæ post baptismum committuntur.” p. 344. ed. Douay, 1613.]

3
Idem, ead. dist. q. 2. art. i. [p. 359. “Est detestatio, et odium, et abominatio commissi peccati, cum firmo proposito emendandi vitam, spe veniæ divinitus obtinendæ. Hæc namque pro pœnitentiæ virtutis definitione habenda est.”]

1
Scot. [John Duns, Franciscan Schoolman, † 1308,] Sent. iv. d. 14. q. 4. [“Pono hanc rationem nominis: pœnitentia est absolutio hominis pœnitentis, facta certis verbis, cum debita intentione prolatis a sacerdote, jurisdictionem habente ex institutione divina, efficaciter significantibus absolutionem animæ a peccato.” t. ix. p. 81. ed. Wading. This reference is by Ussher.]

2
[Tho. Aquin. Summ. Theol. p. iii. de Sacram. q. xxv. al. 84. art. iii. “Oportet quod ea quæ sunt ex parte pœnitentis, sive sint verba sive facta, sint quædam materia hujus sacramenti, ea vero, quæ sunt ex parte sacerdotis, se habeant per modum formæ.” p. 291. Venet. 1596.]

3
Sess. xiv. c. 3. “Docet sancta Synodus sacramenti pœnitentiæ formam, in qua præcipue ipsius vis sita est, in illis ministri verbis positam esse, Ego te absolvo. Sunt autem quasi materia hujus sacramenti ipsius pœnitentis actus, nempe contritio, confessio, et satisfactio.”

a
parts E.

4
[In iv. sent. d. xvi. q. 1. § 4. “Præter materiam et formam in sacramentis non est dare alias partes proprie dictas; sed contritio et satisfactio non sunt materia neque forma sacramenti pœnitentiæ: forma enim consistit in verbis absolutionis; materia vero siqua sit in verbis confessionis, quibus pœnitens suam conscientiam aperit sacerdoti: ergo contritio et satisfactio non sunt partes sacramenti pœnitentiæ, proprie loquendo.”]

b
inpenitency D.

c
iv. D.

d
iniquity E.

1
Luc. vii. 47.

2
Job xxxi. 33.

3
“Tantum relevat confessio delictorum, quantum dissimulatio exaggerat. Confessio autem [enim] satisfactionis consilium est, dissimulatio contumaciæ.” Tertull. de Pœnit. [c. 8. fin.]

1
Chry. hom. 30. in Epist. ad Hebr. [Opp. tom. iv. 589. 20. ed. Savil. ἁμαρτία γὰρ ὁμολογουμένη ἐλάττων γίνεται· μὴ ὁμολογουμένη δὲ, χείρων· ἂν γὰρ προσλάβῃ τὴν ἀναισχυντίαν καὶ τὴν ἀγνωμοσύνην, οὐδέποτε στήσεται· πω̑ς δαὶ ὅλως ὁ τοιου̑τος δυνήσεται ϕυλάξασθαι πάλιν μὴ τοι̑ς αὐτοι̑ς περιπεσει̑ν, ὁ τὸ πρότερον οὐκ εἰδὼς ὅτι ἥμαρτε; . . . μὴ ἁμαρτωλοὺς καλω̑μεν ἑαυτοὺς μόνον, ἀλλὰ καὶ τὰ ἁμαρτήματα ἀναλογιζώμεθα, κατ’ εἰ̑δος ἕκαστον ἀναλέγοντες. οὐ λέγω σοι, ἐκπόμπευσον σαὐτὸν, οὐδὲ παρὰ τοι̑ς ἄλλοις κατηγόρησον· . . . ἐπὶ του̑ Θεου̑ ταυ̑τα ὁμολόγησον, ἐπὶ του̑ δικαστου̑ ὁμολόγει τὰ ἁμαρτήματα, εὐχόμενος, εἰ καὶ μὴ τῃ̑ γλώττῃ ἀλλὰ τῃ̑ μνήμῃ, καὶ οὕτως ἀξίου ἐλεηθη̑ναι. . . . . οὐ του̑το δὲ λέγω, ἐὰν ᾐ̑ς πεπεισμένος σαὐτὸν ἁμαρτωλὸν εἰ̑ναι· οὐχ οὕτω του̑το δύναται ταπεινω̑σαι ψυχὴν, ὡς αὐτὰ ἐϕ’ ἑαυτω̑ν τὰ ἁμαρτήματα, καὶ κατ’ εἰ̑δος ἐξεταζόμενα· . . . . οὐ ϕρονήσεις μέγα, οὐ περιπεσῃ̑ πάλιν τοι̑ς αὐτοι̑ς, σϕοδρότερος ἐσῃ̑ πρὸς τὰ ἀγαθά. . . . . Οἰ̑δα ὅτι οὐκ ἀνέχεται ἡ ψυχὴ τη̑ς μνήμης τη̑ς οὕτω πικρα̑ς. ἀλλὰ ἀναγκάζωμεν αὐτὴν, καὶ βιαζώμεθα· βέλτιον δάκνεσθαι αὐτὴν τῃ̑ μνήμῃ νυ̑ν, ἢ κατ’ ἐκει̑νον τὸν καιρὸν τῃ̑ τιμωρίᾳ.]

e
doth but grow E.

f
often om. E.

g
thy E.

h
or D.

i
held om. E.

2
Levit. xvi. 21.

1
“All Israel is bound on the day of expiation to repent and confess.” R. Mos. in lib. Mitsuoth haggadol. par. 2. præ. 16. [Comp. Tract. Teshuboth, c. ii. § 9. p. 52. ed. Clavering.]

2
“On the day of expiation the high-priest maketh three express confessions.” Idem, eodem loco. [E. gives this note as part of the text. See Clavering’s notes, p. 137. and Talmud, Cod. Joma, as cited by him.]

k
at the times E.

l
book call E.

3
Num. v. 6.

4
Lev. v. 5.

5
Misne Tora, Tractatu Teshuba, cap. 1. [t. i. fol. 7. Venet. 1550.] et R. M. in lib. Misnoth, par. 2. cap. 16.

6
Mos. in Misnoth. par. 2. præ. 16. [This note in E. is part of the text. Comp. Tract. Teshuboth, c. i. § 3.] “None of them, whom either the house of judgment hath condemned to die, or of them which are to be punished with stripes, can be clear by being executed or scourged, till they repent and confess their faults.” [Ibid.] “To him which is sick and draweth towards death, they say, Confess.”

1
Jos. vii. 19.

2
[Maimonid. in Tract. Teshuboth, c. ii. § 6.]

3
[Prov. xxviii. 13.]

m
v. D.

1
Matt. iii. 6.

2
Acts xix. 18. [Alleged by Bellarmine, de Pœnit. iii. c. 4.]

3
James v. 14, 16.

4
Mark xvi. 18.

5
Acts xxviii. 8.

1
Ambros. de Pœnitentia, lib. i. cap. 8. [“Cur ergo manus imponitis, et benedictionis opus creditis, si quis forte revaluerit ægrotus? Cur præsumitis aliquos a colluvione Diaboli per vos mundari posse? Cur baptizatis, si per hominem peccata dimitti non licet.”]

2
[In loc. “Nec hic est sermo de confessione sacramentali: (ut patet ex eo quod dicit, ‘confitemini invicem.’ Sacramentalis enim confessio non fit invicem, sed sacerdotibus tantum;) sed de confessione, qua mutuo fatemur nos peccatores, ut oretur pro nobis; et de confessione hinc et inde erratorum, pro mutua placatione et reconciliatione.” fol. 419. Ludg. 1556.]

3
Annot. Rhem. in Jac. 5. [“It is not certain that he speaketh here of sacramental confession, yet the circumstance of the letter well beareth it, and very probable it is that he meaneth of it.” p. 653. ed. 1582.]

4
[De Pœnit. lib. iii. c. 4.]

5
1 John i. 9.

6
[Bellarm. ubi sup. “Verba illa, ‘Fidelis est et justus,’ referuntur ad promissionem divinam: ideo enim Deus fidelis et justus dicitur, dum peccata confitentibus remittit, quia state promissis suis, nec fidem fallit. At promissio de remittendis peccatis iis qui confitentur Deo peccata sua, non videtur ulla exstare in divinis literis: exstat autem promissio apertissima iis qui ad illos accedunt, quibus dictum est Joannis XXmo, ‘Quorum remiseritis peccata, remittuntur eis.’ ”]

n
his E.

o
vi. D.

1
[De Pœnit. c. ix. “Exomologesis prosternendi et humilificandi hominis disciplina est, conversationem injungens misericordiæ illicem.”]

2
“Plerosque hoc opus ut publicationem sui aut suffugere, aut de die in diem differre præsumo pudoris magis memores quam salutis; velut illi qui, in partibus verecundioribus corporis contracta vexatione, conscientiam medentium vitant, et ita cum erubescentia sua pereunt.” Tertull. de Pœnit. [c. 10.]

1
[Idem ibid. “Inter fratres atque conservos, ubi communis spes, metus, gaudium, dolor, passio, (quia communis spiritus de communi Domino et Patre) quid tu hos aliud quam te opinaris? Quid consortes casuum tuorum ut plausores fugis? Non potest corpus de unius membri vexatione lætum agere: condoleat universum, et ad remedium conlaboret necesse est.”]

2
[De Laps. c. 14. “Quanto et fide majores et timore meliores sunt, qui quamvis nullo sacrificii aut libelli facinore constricti, quoniam tamen de hoc vel cogitaverunt, hoc ipsum apud sacerdotes Dei dolenter, et simpliciter confitentes, exomologesin conscientiæ faciunt, animi sui pondus exponunt, salutarem medelam parvis licet et modicis vulneribus exquirunt.”]

3
[Qui necessitatem sacrificandi pecunia apud magistratum redimebant, accepta securitatis syngrapha Libellatici dicebantur.]

1
[For an account of the literary history of these Homilies, and of the various opinions which have been entertained regarding their origin, see Oudin. Comment. de Scriptor. Eccles. i. 390-426. He does not mention Salvian [† c.485.] as one of the supposed authors, but after deciding against the claims of Eucherius [of Lyons, † c. 449.] and Hilary of Arles [† c. 449.], acquiesces in that of Faustus Regiensis [of Riez, † 493].]

2
Hom. i. de initio Quadragesimæ, [tom. v. par. 1. p. 552. Biblioth. Patr. Col. Agripp. 1618. “Quod autem, charissimi, videmus aliquoties etiam illas animas pœnitentiam petere, quæ ab ineunte adolescentia consecrata pretiosum Deo thesaurum devoverunt, inspirare hoc Deum pro Ecclesiæ nostræ profectibus noverimus: ac medicinam quam invadunt sani discant quærere vulnerati: ut bonis etiam parva deflentibus, ingentia ipsi mala lugere consuescant: ac si quando jam illa persona quæ forte minus indiget pœnitentia aliquid fide dignum atque compunctum sub oculis Ecclesiæ gerit, fructum suum etiam de aliena ædificatione multiplicat, et meritum suum de lucro proficientis accumulat; ut dum perfectione illius emendatur alterius vita, spiritali fœnore ad ipsum boni operis recurrat usura.”]

1
Hom. 10, ad Monachos, [p. 586, 7. “Si levia sunt fortasse delicta; verbi gratia, si homo vel in sermone, vel in aliqua reprehensibili voluntate, si oculo peccavit, aut corde; verborum et cogitationum maculæ quotidiana oratione curandæ, et privata compunctione tergendæ sunt. Si vero quisque conscientiam suam intus interrogans, facinus aliquod capitale commisit, aut si fidem suam falso testimonio expugnavit ac prodidit, ac sacrum veritatis nomen perjurii temeritate violavit; si velum baptismi vel tunicam et speciosam virginitatis holosericam cœno commaculati pudoris infecit; si in semet ipso novum hominem nece hominis occidit; si per augures et divinos atque incantatores captivum se Diabolo tradidit: hæc atque hujusmodi commissa expiari penitus communi et mediocri vel secreta satisfactione non possunt, sed graves causæ] graviores et acriores et publicas curas requirunt.”

2
Hom. 8. ad Monach. [p. 585. “Respondeat mihi illa anima, quæ peccatum suum confusione mortifera in conspectu fratrum sic agnoscere erubuit, quomodo vitare debuisset; quid faciet, cum ante tribunal divinum, cum ante cælestis militiæ fuerit præsentata consessum?”]

1
Lib. ii. de Pœnitentia, c. 9. [“Plerique futuri supplicii metu, peccatorum suorum conscii, pœnitentiam petunt; et cum acceperint, publicæ supplicationis revocantur pudore.” t. ii. p. 434 e.]

2
[De Ecclesiasticis dogmatibus, in Appendix to S. Augustine’s works, ascribed doubtfully to Gennadius, † 493.] Cap. 53. [“Quamvis quis peccato mordeatur, peccandi non habeat cætero voluntatem, et communicaturus satisfaciat lacrymis et orationibus, et confidens de Domini miseratione, qui peccata piæ confessioni donare consuevit, accedat ad Eucharistiam intrepidus et securus. Sed hoc de illo dico quem capitalia et mortalia peccata non gravant: nam quem mortalia crimina post Baptismum commissa premunt, hortor prius publica pœnitentia satisfacere, et ita sacerdotis judicio reconciliatum communioni sociari, si vult non ad judicium et ad condemnationem sui Eucharistiam percipere.”]

p
with conscience E.

q
now held E.

1
Cypr. Epist. 12. [al. 17. c. 1. ap. Bellarmin. de Pœnit. lib. iii. c. 7. “Cum in minoribus delictis, quæ non in Dominum committuntur, pœnitentia agatur justo tempore, et exomologesis fiat, inspecta vita ejus, qui agit pœnitentiam, nec ad communicationem venire quis possit, nisi prius illi ab Episcopo et Clero manus fuerit imposita, quanto magis in his gravissimis et extremis delictis caute omnia et moderate secundum disciplinam Domini observari oportet!” t. ii. 39. ed Fell.]

2
“Inspecta vita ejus qui agit pœnitentiam.”

3
Conc. Nic. par. 2. c. 12. “Pro fide et conversatione pœnitentium.” [ἐϕ’ ἅπασι δὲ τούτοις προσήκει ἐξετάζειν τὴν προαίρεσιν καὶ τὸ εἰ̑δος τη̑ς μετανοίας. ὅσοι μὲν γὰρ καὶ ϕόβῳ καὶ δάκρυσι καὶ ὑπομονῃ̑ καὶ ἀγαθοεργίαις τὴν ἐπιστροϕὴν ἔργῳ καὶ οὐ σχήματι ἐπιδείκνυνται, οὑ̑τοι πληρώσαντες τὸν χρόνον τὸν ὡρισμένον τη̑ς ἀκροάσεως εἰκότως τω̑ν εὐχω̑ν κοινωνήσουσι· μετὰ του̑ ἐξει̑ναι τῳ̑ ἐπισκόπῳ καὶ ϕιλανθρωπότερόν τι περὶ αὐτω̑ν βουλεύσασθαι. t. ii. 36.]

1
*De Pœnitent. dist. i. cap. Mensuram. [in Corp. Jur. Can. p. 368. “Mensuram temporis in agenda pœnitentia idcirco non satis aperte præfigunt canones pro uno quoque crimine, ut de singulis dicant qualiter unumquodque emendandum sit, sed magis in arbitrio sacerdotis intelligentis relinquendum statuunt, quia apud Deum non tam valet mensura temporis quam doloris.”]

*
The heading of the last note inserted here. D.

2
Ambros. de Pœnitentia, lib. ii. cap. 10. [“Facilius inveni qui innocentiam servaverint quam qui congrue egerint pœnitentiam.” t. ii. 436.]

r
Nice E.

3
Greg. Nyss. Orat. in eos qui alios acerbe judicant. [tom. ii. p. 136. ed. Par. 1638. “Eadem in vultu hilaritas, idem in corporis cultu victuque splendor. Somno ad satietatem usque indulgemus, negotiis et occupationibus animo sedulitatis oblivionem injicimus, pœnitentiæ nomen inane duntaxat, et nullis expressum factis retinemus.”]

s
vii. D.

1
Origen. in Psal. xxxvii. [Hom. ii. § 6. “Circumspice diligentius cui debeas confiteri peccatum tuum. Proba prius medicum, cui debeas causam languoris exponere, . . . ut ita demum si quid ille dixerit, qui se prius et eruditum medicum ostenderit et misericordem, si quid consilii dederit, facias, et sequaris, si intellexerit, et præviderit talem esse languorem tuum qui in conventu totius Ecclesiæ exponi debeat et curari, ex quo fortassis et cæteri ædificari poterunt, et tu ipse facile sanari, multa hoc deliberatione, et satis periti medici illius consilio procurandum est.” t. ii. 688.]

2
Ambros. de Pœnitentia, lib. ii. cap. 9. “Hi non tam se solvere cupiunt quam. Sacerdotem ligare.”

3
Aug. in Hom. de Pœnit. [Serm. 351. c. 4. “Ab ipsa mente talis sententia proferatur, ut se indignum homo judicet participatione corporis et sanguinis Domini: ut qui separari a regno cælorum timet per ultimam sententiam summi Judicis, per ecclesiasticam disciplinam a sacramento cælestis panis interim separetur. . . . Cum ipse in se protulerit severissimæ medicinæ, sed tamen medicinæ sententiam, veniat ad antistites, per quos illi in Ecclesia claves ministrantur; et tanquam bonus jam incipiens esse filius, maternorum membrorum ordine custodito, a præpositis sacramentorum accipiat satisfactionis suæ modum.” tom. v. 1356, 1359. Hooker quotes from the Decret. Gratian. de Pœnit. dist. i. c. “in actione.” “Cum tanta est plaga peccati, atque impetus morbi, ut medicamenta corporis et sanguinis Domini differenda sint, auctoritate antistitis debet se quisque ab altari removere ad agendam pœnitentiam, et eadem auctoritate reconciliari.” col. 1673. ed. Lugd. 1572.]

1
Hom. de Pœnit. Ninivit. [Bibl. Patr. Col. t. v. par. i. p. 569. “Dicit Novatianus, ‘Pœnitentiam agere debeo, non accipere; necessaria mihi non est vel admonitio vel intercessio sacerdotis.’ Non ita est. Nam Deus qui erudiendis peccatoribus per prophetam adjutoria procurat, neminem sibi per se sufficere posse confirmat. Errant itaque qui inter dantem et accipientem velut corporale intervenire arbitrantur officium. Quid est enim dare, nisi remedia demonstrare peccatis? Quid est accipere, nisi obedire præceptis, lacrymis et jejuniis interpellare miserationis auditum?”]

2
Aug. Hom. de Pœnit. [i. Serm. 351, c. 4. § 9.] citatur a Grat. [de Pœnit.] dist. 1. c. judicet.

1
[“Judicet ergo seipsum homo in istis voluntate, dum potest, et mores convertat in melius : . . . et tanquam bonus incipiens esse . . . ] a præpositis sacramentorum accipiat satisfactionis suæ modum.”

2
James v. 16.

3
Cassian. coll. 20. c. 8. [7. Bibl. Patr. Col. t. v. pars ii. 207 E. “Si te fragilem fecerit quælibet mentis ignavia, oratione saltem atque intercessione sanctorum remedia vulneribus tuis humilitatis affectu submissus implora.”]

t
Nisse E.

4
Greg. Nyss. Orat. in eos qui alios acerbe judicant, [t. ii. p. 137. “Afflige te, fratresque benevolos atque unanimes adhibe, qui simul doleant, adjumentoque sint, ut libereris. Ostende mihi amaras atque uberes lacrymas tuas, ut meas ego quoque commisceam.” ed. Paris. 1638. This homily has not been published in Greek.]

5
[Ibid.]

u
thine affection D.

1
Leo i. Ep. 7, 8. [al. 136, t. i. 718, ed. Quesnel.] ad Episc. Campan. citat. a Grat. de Pœn. d. 1. c. sufficit. [“Sufficit illa confessio, quæ primum Deo offertur, tum etiam sacerdoti, qui pro delictis pœnitentium precator accedit.”]

2
Ambros. lib. ii. de Pœnit. c. 10. [“Fleat pro te Mater Ecclesia, et culpam tuam lacrymis lavet; videat te Christus mœrentem, ut dicat, Beati tristes, quia gaudebitis. Amat ut pro uno multi rogent.” t. ii. p. 436.]

x
suppliant E.

3
Tertull. de Pœnit. [c. 10. “In uno et altero Ecclesia est, Ecclesia vero Christus. Ergo cum te ad fratrum genua protendis, Christum contractas, Christum exoras. Æque illi cum super te lacrymas agunt, Christus patitur, Christus Patrem deprecatur. Facile impetratur semper, quod Filius postulat.”]

y
supplicant E.

z
their om. E.

a
viii. D.

1
Leo i. Ep. 7, 8. [“Quamvis plenitudo fidei videatur esse laudabilis, quæ propter Dei timorem apud homines erubescere non veretur: tamen quia non omnium hujusmodi sunt peccata ut ea quæ pœnitentiam poscunt non timeant publicare, removeatur tam improbabilis consuetudo, ne multi a pœnitentiæ remediis arceantur, dum aut erubescunt aut metuunt inimicis suis sua facta reserari, quibus possint legum constitutione percelli . . . . Tunc enim demum plures ad pœnitentiam poterunt provocari, si populi auribus non publicetur conscientia confitentis.” Ep. 136. t. i. 719.]

2
[E. H. vii. 16. ἐν τῳ̑ παραιτει̑σθαι συνομολογει̑ν τὴν ἁμαρτίαν χρεὼν, ϕορτικὸν, ὡς εἰκὸς, ἐξ ἀρχη̑ς τοι̑ς ἱερευ̑σιν ἔδοξεν, ὡς ἐν θεάτρῳ ὑπὸ μάρτυρι τῳ̑ πλήθει τη̑ς ἐκκλησίας τὰς ἁμαρτίας ἐξαγγέλλειν· πρεσβύτερον δὲ τω̑ν ἄριστα πολιτευομένων ἐχέμυθόν τε καὶ ἔμϕρονα, ἐπὶ του̑το τετάχασιν· ᾠ̑ δὴ προσίοντες οἱ ἡμαρτηκότες, τὰ βεβιωμένα ὡμολόγουν. ὁ δὲ, πρὸς τὴν ἑκάστου ἁμαρτίαν, ὅ, τι χρὴ ποιη̑σαι ἢ ἐκτίσαι ἐπιτίμιον θεὶς ἀπέλυε, παρὰ σϕω̑ν αὐτω̑ν τὴν δίκην εἰσπραξομένους.]

3
Rather Nicephorus, referring apparently to Socrates. His words are, (lib. xii. c. 28.) Ναυατιανοι̑ς οὐδεμία περὶ τούτου ἔστι σπουδή. λόγος γε μὴν ἔχει καὶ δι’ αὐτοὺς μα̑λλον του̑τ’ ἐπινοηθη̑ναι τὸ ἔργον, μὴ θελήσαντας κοινωνη̑σαι τοι̑ς ἐπὶ τῳ̑ διωγμῳ̑ Δεκίου ἀρνησαμένοις, ἔπειτα μεταμεληθει̑σιν· οἱ γὰρ τηνικάδε ἐπίσκοποι τῳ̑ Ναυάτου ἀντιϕερόμενοι δόγματι τὸν ἐπὶ τω̑ν μετανοούντων πρεσβύτερον ἐϕ’ ἑκάστῃ ἐκκλησίᾳ κατέστησαν, τῳ̑ ἐκκλησιαστικῳ̑ κανόνι ἑπόμενοι.]

b
more E.

c
specially E.

4
Facinoris viam monstrat innoxiis, qui nocentibus post scelera blanditur. [from D.]

d
ix. D.

1
[From the schism of Novatian, circ. ad 253, to the episcopate of Nectarius, circ. 391.]

2
[Soc. v. 19. Γυνή τις τω̑ν εὐγενω̑ν προση̑λθεν τῳ̑ ἐπὶ τη̑ς μετανοίας πρεσβυτέρῳ· καὶ κατὰ μέρος ἐξομολογει̑ται τὰς ἁμαρτίας, ἃς ἐπεπράχει μετὰ τὸ βάπτισμα. Ὁ δὲ πρεσβύτερος παρήγγειλε τῃ̑ γυναικὶ νηστεύειν καὶ συνεχω̑ς εὔχεσθαι, ἵνα σὺν τῃ̑ όμολογίᾳ καὶ ἔργον τι δεικνύειν ἔχῃ τη̑ς μετανοίας ἄξιον. Ἡ δὲ γυνὴ προβαίνουσα καὶ ἄλλο πται̑σμα έαυτη̑ς κατηγόρει· ἔλεγε γὰρ, ὡς εἴῃ συγκαθευδήσας αὐτῃ̑ τη̑ς ἐκκλησίας διάκονος. Soz. vii. 16. Προσταχθει̑σα παρὰ τούτου του̑ πρεσβυτέρου νηστεύειν καὶ τὸν Θεὸν ἱκετεύειν, τούτου χάριν ἐν τῃ̑ ἐκκλησίᾳ διατρίβουσα, ἐκπεπορνευ̑σθαι παρ’ ἀνδρὸς διακόνου κατεμήνυσεν. By this latter account it appears not only that the exposure gave offence, but also that the method of penance prescribed in the case led to a new crime. Such is the construction put on the words of Sozomen by Nicephorus, E. H. xii. 28, and in Hist. Tripart. ix. 35, as also by Valesius in his note on the place of Socrates.]

3
E. H. v. 19. ἠγανάκτουν γὰρ οὐ μόνον ἐπὶ τῳ̑ γενομένῳ, ἀλλ’ ὅτι καὶ τῃ̑ ἐκκλησίᾳ βλασϕημίαν ἡ πρα̑ξις καὶ ὕβριν προὐξένησεν. Διασυρομένων δὲ ἐκ τούτου τω̑ν ἱερωμένων ἀνδρω̑ν, Εὐδαίμων τις τη̑ς ἐκκλησίας πρεσβύτερος, Ἀλεξανδρεὺς τὸ γένος, γνώμην τῳ̑ ἐπισκόπῳ δίδωσι Νεκταρίῳ περιελει̑ν μὲν τὸν ἐπὶ τη̑ς μετανοίας πρεσβύτερον· συγχωρη̑σαι δὲ ἕκαστον, τῳ̑ ἰδίῳ συνειδότι τω̑ν μυστηρίων μετέχειν· οὕτω γὰρ μόνως ἔχειν τὴν ἐκκλησίαν τὸ ἀβλασϕήμητον. This statement, made by Socrates of the cause of the abolition of the office of penitentiary in the time of Nectarius, Hooker seems to have referred to its establishment in the third century.]

1
[Ubi supr. Ἤδη τη̑ς ἀρχαιότητος, οἰ̑μαι, καὶ τη̑ς κατ’ αὐτὴν σεμνότητος καὶ ἀκριβείας, εἰς ἀδιάϕορον καὶ ἡμελημένον ἠ̑θος κατὰ μικρὸν διολισθαίνειν ἀρξαμένης· ἐπεὶ πρότερον, ὡς ἡγου̑μαι, μείω τὰ ἁμαρτήματα ἠ̑ν, ὑπό τε αἰδου̑ς τω̑ν ἐξαγγελλόντων τὰς σϕω̑ν αὐτω̑ν πλημμελείας, καὶ ὑπὸ ἀκριβείας τω̑ν ἐπὶ του̑το τεταγμένων κριτω̑ν.]

e
open E.

f
thought E.

2
[Socr. Hist. Eccles. lib. v. cap. 19. fin. Ταυ̑τα παρὰ του̑ Εὐδαίμονος ἀκούσας ἐγὼ τῃ̑ γραϕῃ̑ τῃ̑δε παραδου̑ναι ἐθάρρησα· . . Ἐγὼ δὲ πρὸς τὸν Εὐδαίμονα πρότερον ἔϕην· ἡ συμβουλή σου, ὡ̑ πρεσβύτερε, εἰ συνήνεγκεν τῃ̑ ἐκκλησίᾳ ἢ εἰ μὴ, Θεὸς ἂν εἰδείη. Ὁρω̑ δὲ ὅτι πρόϕασιν παρέσχε του̑ μὴ ἐλέγχειν ἀλλήλων τὰ ἁμαρτήματα, μηδὲ ϕυλάττειν τὸ του̑ Ἀποστόλου παράγγελμα τὸ λέγον, Μηδὲ συγκοινωνει̑τε τοι̑ς ἔργοις τοι̑ς ἀκάρποις του̑ σκότους, μα̑λλον δὲ καὶ ἐλέγχετε.]

g
mine E.

1
[Sozom. Hist. Eccles. l. vii. c. 16. ἐπηκολούθησαν δὲ σχεδὸν οἱ πανταχου̑ ἐπίσκοποι . . . καὶ ἐξ ἐκείνου του̑το κρατου̑ν διέμεινεν.]

h
x. D.

2
“Tanta hæc Socrati testanti præstanda est fides, quanta cæteris hæreticis de suis dogmatibus tractantibus; quippe Novatianus secta cum fuerit, quam vere ac sincere hæc scripserit adversus pœnitentiam in Ecclesia administrari solitam, quemlibet puto posse facile judicare.” Baron. tom. i. ann. Chr. 56. [c. 26.]

“Sozomenum eandem prorsus causam fovisse certum est.” Ibid.

“Nec Eudæmonem illum alium quam Novatianæ sectæ hominem fuisse credendum est.” Ibid. [c. 27.]

“Sacerdos ille merito a Nectario est gradu amotus officioque depositus, quo facto Novatiani (ut mos est hæreticorum) quamcunque licet levem, ut sinceris dogmatibus detrahant, accipere ausi occasionem, non tantum Presbyterum pœnitentiarium in ordinem redactum, sed et pœnitentiam ipsam una cum eo fuisse proscriptam, calumniose admodum conclamarunt, cum tamen illa potius theatralis fieri interdum solita confessio peccatorum fuerit abrogata.” Ibid. [c. 34.]

i
so forward E.

k
both om. E.

l
corresponding E.; correspondency E. 1648.

1
[Ubi supr. Ναυατιανοι̑ς μὲν, οἱ̑ς οὐ λόγος μετανοίας, οὐδὲν τούτου ἐδέησεν.]

2
[Lib. vii. cap. 16. Τὸ μὴ παντελω̑ς ἁμαρτει̑ν θειοτέρας ἢ κατὰ ἄνθρωπον ἐδει̑το ϕύσεως· μεταμελουμένοις δὲ καὶ πολλάκις ἁμαρτάνουσι συγγνώμην νέμειν ὁ θεὸς παρεκελεύσατο.]

3
[Socr. v. 10. ὁ βασιλεὺς (Theodosius) θαυμάσας αὐτω̑ν τὴν περὶ τοὺς οἰκείους κατὰ τὴν πίστιν ὁμόνοιαν, νόμῳ ἐκέλευε τω̑ν μὲν οἰκείων κρατει̑ν ἀδεω̑ς εὐκτηρίων τόπων, ἔχειν δὲ καὶ προνόμια τὰς ἐκκλησίας αὐτω̑ν, ἅπερ καὶ οἱ τη̑ς αὐτου̑ πίστεως ἔχουσιν.]

1
Socrat. Hist. Eccles. lib. v. c. 19*. [μόνοι οἱ του̑ Ὁμοουσίου ϕρονήματος, καὶ οἱ τούτοις κατὰ τὴν πίστιν ὁμόϕρονες Ναυατιανοὶ, τὸν ἐπὶ τη̑ς μετανοίας πρεσβύτερον παρῃτήσαντο.]

*
This reference not in D.

m
xi. D.

2
Bellarm. de Pœnit. lib. iii. c. 14. [p. 1399, 1400.] “Apud veteres nomine pœnitentium, soli publici pœnitentes, intelligi solebant.”

“Nullo modo fieri potuit, ut unus presbyter satisfaceret tantæ multitudini, quantam Constantinopoli, vel in aliis civitatibus, pœnitentiæ remedio indigebat: non igitur omnes eum Presbyterum adire cogebantur, sed ii solum, qui pœnitentiam publicam suscipiebant.”

“Sozomenus, ubi disertis verbis affirmasset, constitutionem de Presbytero pœnitentiali, quam prisci Episcopi invexerant, et Nectarius postea Constantinopoli abrogaverat, Romæ potissimum accurate servari; continuo explicare cœpit ritum pœnitentiæ publicæ, quæ Romæ suo tempore servabatur: igitur constitutio illa ad solos pœnitentes publicos pertinebat.”

“Colligimus, constitutionem Episcoporum, de qua historici loquuntur, id solum complexam, ut qui publice lapsi essent post Baptismum, ii ad sacram Eucharistiam non accederent, nisi Presbytero pœnitentiario privatim omnia peccata sua confessi essent, et deinde ad ejus arbitrium publice coram cœtu Ecclesiæ peccata publica detexissent, et pœnitentiam publicam egissent . . . Ante exortam hæresin Novati, nemo cogebatur certum Presbyterum adire, neque peccata ulla publice confiteri . . . Cæterum post Novati hæresin excitatam, placuit Episcopis aliquid addere, ne Novatiani Catholicos reprehendere possent quod nimis facile lapsos ad communionem admitterent.”]

n
of om. D.

1
Τοὺς ἐπὶ τη̑ς μετανοίας περιελει̑ν πρεσβυτέρους. [Hist. Eccles. lib. v. c. 19.]

o
convict E.

p
But E.

q
whensoever E.

1
Sozom. Hist. Eccles. lib. vii. c. 16. [vid. supr. p. 33, note 2.]

r
Prelate E.

s
The following clause to the repetition of the word burthensome is omitted in E.

t
confessions E.

u
in the first E.

2
Fab. Decret. Ep. 2. tom. i. Conc. p. 358. [“Illi qui illa perpetrant, de quibus Apostolus ait ‘Quoniam qui talia agunt regnum Dei non consequentur,’ valde cavendi sunt, et ad emendationem, si voluntarie noluerint, compellendi; quia infamiæ maculis sunt aspersi, et in barathrum delabuntur, nisi eis sacerdotali auctoritate subventum fuerit.” Conc. ed. Labb. et Cossart. i. 643. The epistle is believed to be spurious.]

x
xii. D.

y
Hessels E.

1
Theological professor at Louvain: present at the Council of Trent, where he died 1551. (Fleury, Hist. Eccl. l. 147. c. 104.) Not to be confounded with J. Hessels. v. Biog. Univ. Fleury, l. 170. c. 13.

1
[De Pœnit. iii. 14. p. 1399.]

2
“Non [nec E.] est quod sibi blandiantur illi de facto Nectarii, cum id potius secretorum peccatorum confessionem comprobet, et non aliud quam Presbyterum pœnitentialem illo officio suo moverit; uti amplissime deducit D. Johannes Hasselius” [so E.; v. note y above]. Pamel. in Cypr. lib. [de Lapsis, p. 251.] annot. 98. et in lib. Tertull. de Pœnit. annot. 1. [p. 200. Paris. 1598.]

z
xiii. D.

a
fourthly ins. E.

b
fifthly ins. E.

c
sixthly ins. E.

d
profession D.

1
“Sacerdos imponit manum subjecto, reditum Spiritus Sancti invocat, atque ita eum qui traditus fuerat Satanæ in interitum carnis, ut spiritus salvus fieret, indicta in populum oratione altari reconciliat.” Hieron. advers. Lucif. [§ 5. t. ii. p. 175. a. ed. Vallarsii.]

2
Ambros. de Pœnit. lib. ii. cap. 10. [“An quisquam ferat ut erubescas Deum rogare, qui non erubescis rogare hominem? et pudeat te Deo supplicare, quem non lates, cum te non pudeat peccata tua homini, quem lateas, confiteri? An testes precationis et conscios refugis, cum si homini satisfaciendum sit, multos necesse est ambias obsecres, ut dignentur intervenire; ad genua te ipse prosternas, osculeris vestigia, filios offeras culpæ adhuc ignaros, paternæ etiam veniæ precatores? Hoc ergo in ecclesia facere fastidis, ut Deo supplices, ut patrocinium tibi ad obsecrandum sanctæ plebis requiras: ubi nihil est quod pudori esse debeat, nisi non fateri, cum omnes simus peccatores; ubi ille laudabilior, qui humilior, ille justior, qui sibi abjectior.” t. iii. 435.]

1
Chrys. Hom. Περὶ μετανοίας καὶ ἐξομολογήσεως. Παρὰ τοι̑ς λογισμοι̑ς γενέσθω τω̑ν πεπλημμελημένων ἡ ἐξέτασις· ἀμάρτυρον ἔστω τὸ δικαστήριον· ὁ Θεὸς ὁράτω μόνος ἐξομολογούμενον. [See hereafter on § 16.]

e
punishment E.

2
Cassian. Collat. 20. c. 8. [7. Bibl. Pat. Colon. t. v. p. ii. 207. “Quod si verecundia retrahente revelare coram hominibus erubescis, illi quem latere non possunt confiteri ea jugi supplicatione non desinas, . . . qui et absque illius verecundiæ publicatione curare, et sine improperio peccata donare consuevit.”]

3
Prosper de Vita Contempl. lib. ii. c. 7. [“Deum sibi facilius placabunt illi, qui non humano convicti judicio sed ultro crimen cognoscunt: qui aut propriis illud confessionibus produnt, aut nescientibus aliis quales occulti sunt, ipsi in se voluntariæ excommunicationis sententiam ferunt; et ab altari cui ministrabant non animo sed officio separati vitam tanquam mortuam plangunt, certi quod reconciliato sibi efficacis pœnitentiæ fructibus Deo non solum amissa recipiant, sed etiam cives supernæ civitatis effecti ad gaudia sempiterna perveniant.” Bibl. Patr. Colon. t. v. pars iii. p. 63.]

f
First ins. E.

g
Secondly E; and the mistake is continued throughout this enumeration.

h
sin E.

i
offenders E.

k
xiiii. D.

1
Calv. Inst. lib. iii. cap. 4. § 7. [“Miror autem qua fronte ausint contendere confessionem de qua loquuntur juris esse divini; cujus equidem vetustissimum esse usum fatemur, sed quem facile evincere possumus olim fuisse liberum.”]

l
to man om. E.

1
“Sed tantum ut Ecclesiæ sit aliqua ratione satisfactum, et omnes unius pœnitentia confirmentur, qui fuerant unius peccatis et scandalis vulnerati.” Sadeel. [i. e. Antoine la Roche de Chandieu, a leading French Protestant teacher, first at Paris, then at Geneva, 1534-1591. He hebraized his name—Sadeel, ‘Chant de Dieu,’ Zamariel, ‘Champ de Dieu.’ Biog. Univ.] in Psal. xxxii. ver. 5. [Op. p. 906. ed. 1599.]

2
Harm. Confess. sect. viii. ex 5. cap. Confess. Bohem. [p. 143. Genev. 1581.] “Docetur et hoc apud eos; quorum peccatum est publicum, atque ideo scandalum publicum, quando Deus iis largitur pœnitentiæ spiritum, externam pœnitentiæ testificationem non debere abesse; et hac quidem de causa, ut sit argumentum et testimonium, quo probetur seu planum fiat lapsos peccatores qui pœnitentiam agunt vere se convertere. Etiam ut sit nota reconciliationis cum Ecclesia et proximo; atque exemplo aliis, quod reformident et vereantur.”]

m
example E.

3
[Ibid. “Ita instituuntur pœnitentes, ut curatores animarum suarum accedant, et coram ipsis confiteantur Deo, peccata sua . . . . ut hoc modo dolorem suum, quo afficiuntur, et quam sibi propter peccata displiceant, indicare, et consilium et doctrinam quomodo deinceps ea effugiant, et institutionem atque consolationem impeditis conscientiis suis, itemque absolutionem ex potestate clavium, et remissionem peccatorum per ministerium evangelii a Christo institutum peculiariter singuli expetere possint, et a Deo suo consequi se sciant, et quando hæc a ministris eis præstantur, accipere ab eis, tanquam rem a Deo ad commodandum ipsis et salubriter inserviendum institutam, cum fiducia debent, et remissione peccatorum sine dubitatione frui, secundum verbum Domini, ‘Cui peccata remiseritis, remittuntur eis.’ Atque hac fide indubitata nitentes, certi et animo confirmato esse debent per ministerium harum clavium, de potestate Christi et verbo ipsius omnia ipsis peccata remitti.” The Saxon confession runs thus: “De confessione privata facienda pastoribus adfirmamus ritum privatæ absolutionis in Ecclesia retinendum esse; et constanter retinemus, propter multas graves causas.” Ap. Syntagm. Confess. pars ii. p. 77. Genev. 1654.]

n
all om. E.

1
[John xx. 23. ap.] cap. 5. Confess. Bohem.

o
they om. D.

p
xv. D.

q
guilty om. E.

1
As for private confession, abuses and errors set apart, we condemn it not, but leave it at liberty. Jewel, Defens. p. 156. [158. ed. 1611. “Abuses and errors removed, and especially the priest being learned, as we have said before, we mislike no manner confession, whether it be private or public.”]

r
save E.

s
themselves fearful E.

t
the om. D.

1
“Nos a communione quenquam prohibere non possumus, quamvis hæc prohibitio nondum sit mortalis, sed medicinalis, nisi aut sponte confessum, aut aliquo sive seculari sive ecclesiastico judicio accusatum atque convictum. Quis enim sibi utrumque audet assumere, ut cuiquam ipse sit et accusator et judex?” [Rhenan. Admon. de Dogm. Tertull. [Basel. 1521. Paris 1566.] inter Opp. Tertull. p. 903, ed. Par. 1635.]

u
it is not in us E.

x
or om. E.

1
“Non enim temere et quodammodo libet, [quomodolibet?] sed per judicium, ab Ecclesiæ communione separandi sunt mali, ut si per judicium auferri non possint, tolerentur potius, velut paleæ cum tritico.” [et paullo supra.] “Multi corriguntur, ut Petrus; multi tolerantur, ut Judas; multi nesciuntur, donec veniat Dominus, et illuminabit abscondita tenebrarum.” Rhenan. [Beatus Rhenanus 1485-1547] admonit. de dogmat. Tertull. [Ibid.]

y
fors. elude. Fulm.

1
Lib. iii. de Pœnit. [called in the old editions of St. Ambrose, “Exhortatio ad Pœnitentiam:” but omitted by the Benedictine editors on the ground of its being found word for word in St. Augustin’s Works, t. v. 1506-8, Hom. cccxciii: ascribed by some to Cæsarius of Arles. “Qui positus in ultima necessitate ægritudinis suæ acceperit pœnitentiam, et mox reconciliatus fuerit, et vadit, i.e. exit de corpore; fateor vobis, non illinegamus quod petit, sed non præsumo dicere quia bene hinc exit. Non præsumo, non polliceor, non dico, non vos fallo, non vos decipio, non vobis promitto . . . Nunquid dico, damnabitur? non dico: sed nec liberabitur dico. . . . Prorsus nescio de Dei voluntate. Vis te frater a dubio liberari? vis quod incertum est evadere? Age pœnitentiam dum sanus es. . . Si autem tunc agere vis ipsam pœnitentiam quando peccare jam non potes, peccata te dimiserunt, non tu illa.”]

z
clearly om. E.

a
xvi. D.

1
“Non dico tibi, ut te prodas in publicum, neque ut te apud alios accuses, sed obedire to volo Prophetæ dicenti, (Ps. xxxii. 5.) ‘revela Domino viam tuam.’ Ante Deum confitere peccata tua.” Chrysost. Hom. 31. ad Hebr. [t. iv. p. 198. ed. Froben. Basil] “Peccata tua dicito, ut ea deleat; si confunderis alicui dicere quæ peccasti, dicito ea quotidie in anima. Non dico ut confitearis conservo qui exprobret; Deo dicito qui ea curat.” [Idem in Ps. l. t. i. p. 708, 10. ed. Savile.] “Non necesse est præsentibus testibus confiteri; solus te Deus confitentem videat.” Id. Hom. de Pœnit. et Confess. [t. v. 512.] “Rogo et oro ut crebrius Deo immortali confiteamini, et enumeratis vestris delictis veniam petatis. Non te in theatrum conservorum duco, non hominibus peccata tua conor detegere. [detegere cogo.] Repete coram Deo conscientiam tuam, te explica, ostende medico præstantissimo vulnera tua, et pete ab eo medicamentum.” Hom. 5. de incompreh. Dei Natura, itemque Homil. de Lazaro. [t. ii. 402; i. 77.]

b
reckoning up D.

2
Psalm xxxii. 5.

c
then om. E. inserted in D. by Archbishop Ussher.

d
i. D. with a considerable blank after the last line.

1
Tertull. de Pœnit. [cap. 5. “Qui per delictorum pœnitentiam instituerat Domino satisfacere, diabolo per aliam pœnitentiæ pœnitentiam satisfaciet: eritque tanto magis perosus Deo, quanto æmulo ejus acceptus.”]

e
fors. from Fulm.

2
Chrysost. in 1 Cor. Hom. 8 Τὸν Θεὸν ἐξιλεώσασθαι. [πω̑ς οὐ̑ν δυνήσῃ τὸν Θεὸν ἐξιλεώσασθαι, ὅταν μηδὲ ὅτι ἥμαρτες εἰδῃ̑ς; t. x. 71 E. ed. Bened.]

f
St. om. E.

3
Cypr. Ep. 8. [al. 11. c. 2. “Virgas et flagella sentimus, qui Deo nec bonis factis placemus, nec pro peccatis satisfacimus.” ii. 23, 24. ed. Fell.]

g
this E.

1
Cyp. Ep. 26. [al. 31. c. 5. “Illi ipsi oculi, qui male simulacra conspexerunt, quæ illicita commiserant, satisfacientibus Deo fletibus deleant.” p. 64.]

2
Sent. lib. iv. dis. 16. [cap. 1. “In perfectione pœnitentiæ tria observanda sunt; s. compunctio cordis, confessio oris, satisfactio operis, . . . ut sicut tribus modis Deum offendimus, sc. corde, ore, et opere, ita tribus modis satisfaciamus. Sunt enim ‘tres peccati differentiæ’ (ut ait Augustinus:” [De Serm. Dom. in Monte, [Editor: gap in text] 12. t. iii. pars ii. p. 180.]) “in corde, et in facto, et in consuetudine vel verbo; tanquam tres mortes. Una quasi in domo, cum in corde consentitur libidini: altera quasi prolata jam extra potam, cum in factum procedit libidini assensio; tertia cum malæ consuetudinis tanquam mole premitur animus, vel noxiæ defensionis clypeo armatur, ‘quasi in sepulchro jam fœtens. Hæc sunt tria genera mortuorum, quæ Deus legitur suscitasse.’ Huic ergo triplici morti triplici remedio occurritur.” fol. 174. ed. Colon. 1513.]

h
works om. E.

i
ii. D.

1
Bonavent. [S. Bonaventura, great Franciscan Schoolman, 1221-1274.] in Sent. lib. iv. dist. xv. q. 9. [q. i. t. iii. pars ii. p. 199. “Etsi divina misericordia relinquat offensam homini dando gratiam, non tamen ita omnino relinquit, quin etiam exigat de offensa satisfactionem per justitiam Et quia homo non potuit pro tanta offensa satisfacere, ideo Deus dedit ei mediatorem qui satisfaceret pro offensa. Unde in sola fide passionis Christi remittitur omnis culpa, et sine fide ejus nullus justificatur. Et secundum hoc dicunt, quod omnis satisfactio nostra virtutem habet a satisfactione Christi.” Ed. Rom. 1596.]

k
iii. D.

l
thereby made E.

m
fruits E.

2
[De Pœnit. c. 9.]

n
iv. D.

1
Apoc. i. 6.

2
Cassian. col. 20. c. 8. [Bibl. Patr. Colon. t. v. p. ii. 207. “Etiamsi hæc omnia fecerimus, non erunt idonea ad expiationem scelerum nostrorum, nisi ea bonitas Domini clementiaque deleverit. Qui cum religiosi conatus obsequia supplici mente a nobis oblata perspexerit, exiguos parvosque conatus immensa libertate prosequitur, dicens, Ego sum, ego sum, qui deleo iniquitates tuas propter me, et peccatorum tuorum jam non recordabor.”]

o
v. D.

1
Basil. Hom. in Psalm. xxxvii. Παντὸς γὰρ πάθους ἀλλότριον τὸ Θει̑ον. [Πολλάκις δὲ εἴρηται, ὡς ὀργὴ καὶ θυμὸς του̑ Θεου̑ λεγόμενα ἐν ται̑ς θεοπνεύστοις γραϕαι̑ς οὐ πάθη σημαίνει (παντὸς γὰρ πάθους ἀλλότριον τὸ Θει̑ον·) κατὰ μεταϕορὰν δὲ τὰ τοιαυ̑τα εἴωθεν ὀνομάζειν ὁ τη̑ς γραϕη̑ς λόγος, ὡς καὶ ὀϕθαλμοὺς Θεου̑, κ.τ.λ. . . . οὕτως οὐ̑ν καὶ τὰς ἐπαγομένας πονηρίας τοι̑ς ἁμαρτάνουσι κατὰ Θεου̑ κρίσιν, σκυθρωπὰς οὔσας καὶ ἀλγεινὰς τοι̑ς πάσχουσιν, ὡσανεὶ ἐξ ὀργη̑ς καὶ θυμου̑ ἐπαγομένας ὑποτυπου̑ται. App. ad t. i. p. 363. bc The Homily is considered spurious.]

2
“Cum Deus irasci dicitur, [irascitur E.] non ejus significatur perturbatio qualis est in animo irascentis hominis, sed ex humanis motibus translato vocabulo, vindicta ejus, quæ non nisi justa est, iræ nomen accepit.” Aug. t. 3. Ench. c. 33. [t. vi. 209.]

3
“Pœnitentiæ compensatione redimendam proponit impunitatem Deus.” Tertull. de Pœniten. [c. 6.]

p
takes E.

q
which E.

4
Numb. xiv. [22]; xx. 12; xii. 14; 2 Sam. xii. 14.

5
“Cui Deus vere propitius est, non solum condonat [donat] peccata ne noceant ad futurum seculum, sed etiam castigat, ne semper peccare delectet.” Aug. in Psal. xcviii. [§ 11. iv. 1067.]

1
“Plectuntur quidam, quo cæteri corrigantur; exempla sunt omnium, tormenta paucorum.” Cypr. de Lapsis. [c. 13.]

r
punishment E.

s
now here D. E.

2
Ezech. xxxiii. 14.

t
turneth E.

3
Rom. ii. 5.

4
Isai. i. 18. [Abp. Ussher in E.]

u
as E.

5
“Si texit Deus peccata, noluit advertere; si noluit advertere, noluit animadvertere.” [in Psal. xxxi. (Heb. xxxii.) 1. t. iv. 176.]

6
[Psal. xxxii. 2.]

x
not E.

y
God E.

1
“Mirandum non est, et mortem corporis non fuisse eventuram homini, nisi præcessisset peccatum, cujus etiam talis pœna consequeretur, et post remissionem peccatorum eam fidelibus evenire, ut ejus timorem vincendo exerceretur fortitudo justitiæ . . . . Sic et mortem corporis propter hoc peccatum Deus homini inflixit, et post peccatorum remissionem propter exercendam justitiam non ademit.” Aug. de Pecc. Mer. et Rem. lib. ii. c. 34. [t. x. 69.]

2
“Ante remissionem esse illa supplicia peccatorum, post remissionem autem certamina exercitationesque justorum.” [August. ibid. p. 68.]

z
This word is erased by Abp. Ussher in D, and seemeth written in its place.

1
Cypr. Epist. 53 [52 ed. Pamel. 55 ed. Fell. p. 111. “Unus ille et verus Pater. . . . lætatur in pœnitentia filiorum suorum; nec iram pœnitentibus, aut plangentibus et lamentantibus pœnam comminatur, sed veniam magis et indulgentiam pollicetur.”]

a
himself E. (?)

b
meditation E.

c
fruit E.

d
vi. D.

e
Prayers E.

f
have E.

g
those E.

1
2 Cor. vii. 11.

2
Ἡμω̑ν γὰρ αὐτω̑ν δίκην λάβωμεν, ἡμω̑ν αὐτω̑ν κατηγορήσωμεν· οὕτως ἐξιλεωσόμεθα τὸν κριτήν. Chrys. Hom. 30. [31.] in Ep. ad Heb. [t. xii. 289. a.]

h
are E.

3
Cypr. de Lapsis. [c. ult. “Orare oportet impensius et rogare, . . in cilicio et sordibus volutari; post indumentum Christi perditum, nullum hic jam velle vestitum; post diaboli cibum malle jejunium; justis operibus incumbere, quibus peccata purgantur; eleemosynis frequenter insistere, quibus a morte animæ liberentur.” t. i. p. 137, 138.]

i
some E.

1
Salv. ad Eccl. Cathol. lib. i. [p. 367. tom. v. par. iii. Biblioth. Patr. Colon. “Nec offerat cum redemptionis fiducia, sed cum supplicationis officio: . . . Non pretio, sed affectu placent.”]

k
our E. [not 1648.]

l
vii. D.

m
sometime E.

n
dealing E.

o
habiletee D.

2
Levit. vi. 2.

p
neighbours E.

q
taken D.

r
swear E.

s
sometime E.

t
have E.

u
any om. E.

1
Num. v. 8.

2
[Maimon. tract. Teshuboth. § ii. in fine.]

x
that E.

y
guides E.

z
oppressions D.

3
“Quamdiu enim res, propter quam peccatum est, non redditur, si reddi potest; non agitur pœnitentia sed fingitur.” Sent. iv. d. 15. [c. 5. fol. 173. from S. Aug. Ep. ad Maced. 153. c. 6. t. ii. 532.]

a
viii. D.

b
suffices.

c
for E.

1
Cyp. Ep. lii. [al. 55, c. 10. “Si nos aliquis pœnitentiæ simulatione deluserit; Deus qui non deridetur, et qui cor hominis intuetur, de his quæ nos minus perspeximus judicet, et servorum sententiam Dominus emendet.” t. ii. p. 108.]

2
Basil. Ep. ad Amphil. c. 76. [77. ep. 217. t. iii. 329. Ὁ καταλιμπάνων τὴν νομίμως αὐτῳ̑ συναϕθει̑σαν γυναι̑κα, καὶ ἑτέραν συναγόμενος, κατὰ τὴν του̑ Κυρίου ἀπόϕασιν, τῳ̑ τη̑ς μοιχείας ὑποκει̑ται κρίματι· κεκανονίσται δὲ παρὰ τω̑ν πατέρων ἡμω̑ν, τοὺς τοιούτους ἐνιαυτὸν προσκλαίειν, διετίαν ἐπακροα̑σθαι, τριετίαν ὑποπίπτειν· τῳ̑ δὲ ἑβδόμῳ συνίστασθαι τοι̑ς πιστοι̑ς· καὶ οὕτω τη̑ς προσϕορα̑ς καταξιου̑σθαι, ἐὰν μετὰ δακρύων μετανοήσωσι· ὁ δὲ αὐτὸς τύπος κρατείτω καὶ ἐπὶ τω̑ν τὰς δύο ἀδελϕὰς λαμβανόντων εἰς συνοικέσιον, εἰ καὶ κατὰ διαϕόρους χρόνους.]

1
Concil. Nicen. can. 11. [περὶ τω̑ν παραβάντων χωρὶς ἀνάγκης, ἢ χωρὶς ἀϕαιρέσεως ὑπαρχόντων, ἢ χωρὶς κινδύνου, ἤ τινος τοιούτου, ὅ γέγονεν ἐπὶ τη̑ς τυραννίδος Λικινίου· ἔδοξε τῃ̑ συνόδῳ, κἂν ἀνάξιοι ἠ̑σαν ϕιλανθρωπίας, ὅμως χρηστεύσασθαι εἰς αὐτούς· ὅσοι οὐ̑ν γνησίως μεταμέλονται, τρία ἔτη ἐν ἀκροωμένοις ποιήσουσιν οἱ πιστοὶ, καὶ ἕπτα ἔτη ὑποπεσου̑νται· δύο δὲ ἔτη χωρὶς προσϕορα̑ς κοινωνήσουσι τῳ̑ λαῳ̑ τω̑ν προσευχω̑ν. Conc. t. i. 327. ed. Harduin.]

2
Καθόλου καὶ περὶ παντὸς οὑτινοσου̑ν ἐξοδεύοντος, αἰτου̑ντος [δὲ] μετασχει̑ν Εὐχαριστίας, ὁ ἐπίσκοπος μετὰ δοκιμασίας μεταδιδότω τη̑ς προσϕορα̑ς. can. 13. μετὰ δοκιμασίας, id est, manifestis indiciis deprehensa peccatoris seria conversione ad Deum. [ib. 329.]

3
Canon 12. [μετὰ του̑ ἐξει̑ναι τῳ̑ ἐπισκόπῳ καὶ ϕιλανθρωπότερόν τι περὶ αὐτω̑ν βουλεύσασθαι· ὅσοι δὲ ἀδιαϕόρως ἤνεγκαν, καὶ τὸ σχη̑μα του̑ μὴ εἰσιέναι εἰς τὴν ἐκκλησίαν ἀρκει̑ν αὐτοι̑ς ἡγήσαντο πρὸς τὴν ἐπιστροϕὴν, ἐξ ἅπαντος πληρούτωσαν τὸν χρόνον.]

d
Thirdly, being E.

e
ill E.

f
was E.

g
St. om. E.

1
[De Laps. c. 12.] “Jacens stantibus, et integris vulneratus, minatur, [et quod non statim Domini corpus inquinatis manibus accipiat, aut ore polluto Domini sanguinem bibat, sacerdotibus sacrilegus irascitur. Atque O tuam nimiam, furiose, dementiam. Irasceris ei qui abs te averter e Dei iram nititur, ei minaris qui pro te Domini misericordiam deprecatur.”]

1
[“Mandant aliquid martyres fieri, sed si justa, si licita, si non contra ipsum Dominum a Dei sacerdote facienda, si obtemperantis facilis et prona consensio, si petentis fuerit religiosa moderatio . . . Quid vero justius Noe? . . . Quid gloriosius Daniele? . . . Quid Job in operibus promptius? . . . Nec his tamen, si rogarent, concessurum se Deus dixit . . . Adeo non omne quod petitur in præjudicio petentis sed in dantis arbitrio est.” p. 187. ed. Baluz.] Exod. xii. [xxxii?] 31; Jerem. vii. 15. [16.]; Ezek. xiv. 14.

h
did himself E.

2
[Ibid. p. 186. “Emersit, fratres dilectissimi, novum genus cladis; et quasi parum persecutionis procella sævierit, accessit ad cumulum sub misericordiæ titulo malum fallens et blanda pernicies. Contra evangelii vigorem, contra Domini ac Dei legem temeritate quorundam laxatur incautis communicatio, irrita et falsa pax, periculosa dantibus, et nihil accipientibus profutura. Non quærunt sanitatis patientiam, nec veram de satisfactione medicinam. Pœnitentia de pectoribus excussa est, gravissimi extremique delicti memoria sublata est. Operiuntur morientium vulnera, et plaga lethalis altis et profundis visceribus infixa dissimulato dolore contegitur. A diaboli aris revertentes ad sanctum Domini sordidis et infectis nidore manibus accedunt. Mortiferos idolorum cibos adhuc pœne ructantes, exhalantibus etiam nunc scelus suum faucibus, et contagia funesta redolentibus, Domini corpus invadunt, quando occurrat scriptura divina et clamat et dicat, . . .‘Quicunque ederit carnem aut biberit calicem Domini indigne, reus erit corporis et sanguinis Domini.’ Spretis his omnibus atque contemptis, ante expiata delicta, ante exomologesin factam criminis, ante purgatam conscientiam sacrificio et manu sacerdotis, ante offensam placatam indignantis Domini ac minantis, vis infertur corpori ejus et sanguini . . . Pacem putant esse, quam quidam verbis fallacibus venditant. . . Quid injuriam beneficium vocant? Quid impietatem vocabulo pietatis appellant? . . . Non concedit pacem facilitas ista, sed tollit: . . . Persecutio est hæc alia et alia tentatio, per quam subtilis inimicus impugnandis adhuc lapsis occulta populatione grassatur: ut lamentatio conquiescat, ut dolor sileat, ut delicti memoria evanescat, ut comprimatur pectorum gemitus, statuatur fletus oculorum, nec Dominum graviter offensum longa ac plena pœnitentia deprecetur.”]

i
rigour E.

k
altar E.

1
1 Cor. xi. 27.

l
rise E.

m
ix. D.

n
[Here the Dublin MS. goes on. “For against ye guiltines of sinne, and ye danger of everlasting condemnation thereby incurred, confession and absolution succeeding ye same, are, (as they take it,) a remedie sufficient, and therefore, what their pœnitentiaries doe thinck good to impose further, whether it be, a matter of Ave Maries dayly to be scored up, a journey of pilgrimage to bee undertaken, some few dishes of ordinarie dyet to be exchanged, offrings to be made att ye shrines of Saints, or a little to be scraped of from men’s superfluitie, for releife of poore people, as in liew or exchange wth God, whose Justice oweth us still (they say) notwithstanding our pardon, some temporall punishment, to be susteyned in ye life to come, except wee quitt ourselves here with workes of ye former kind, continued till ye ballance of God’s most strict severity, shall finde ye paynes wee have taken, æquivallent, with ye faults for which wee satisfye.” All this passage Abp. Ussher has drawn lines through and across, with his pen; and noted in the margin, “(This followeth afterward, more properly, in the viith section of the next head, touching absolution.)” (See below, pp. 83, 84.)]

o
and children E.

p This marginal note om. E.
q satisfaction E. (?)
r indulgence E. (?)
s
unpossible E.

1
See below, p. 84.

t
unestimable E.

u
those E.

x
true om. E.

y
i. D.

1
Matt. ix. 2.

2
Marc. v. 21. [ii. 7]; Luc. v.21.

z
much as om. E.

a
ii. D.

b
due E.

1
“Ipsius (pœnitentis scil.) actio non est pars sacramenti, nisi quatenus potestati sacerdotali subjicitur, et a sacerdote dirigitur vel jubetur.” Bellarmin. de Pœnit. lib. i. c. 16. [t. iii. 942.]

2
“Christus instituit sacerdotes judices super terram cum ea potestate, ut, sine ipsorum sententia, nemo post baptismum lapsus reconciliari possit.” Bellarmin. de Pœnit. lib. iii. c. 2. [t. iii. 1028.]

3
[Matt. xviii. 18; John xx. 23.]

4
“Quod si possent rei [ei E.] sine sacerdotum sententia absolvi, non [enim E.] esset vera Christi promissio, Quæcunque,” &c. Bellarm. ibid. [p. 1031.]

c
hampred himself so E.

5
[Luke xviii. 13.]

1
[Psalm li. 1.]

d
compassion E.

2
[Gen. xix. 22.]

e
iii. D.

f
is E.

g
granteth E.

3
“Christus ordinariam suam potestatem in apostolos transtulit; extraordinariam sibi reservavit. Ordinaria enim remedia in Ecclesia ad remittenda peccata sunt ab eo instituta, sacramenta; sine quibus peccata remittere Christus potest, sed extraordinarie et multo rarius hoc facit, quam per sacramenta. Noluit igitur homines [eos E.] extraordinariis remediis remissionis peccatorum confidere, quæ et rara sunt et incerta: sed ordinaria, et ut ita dicam, visibilia sacramentorum quærere remedia.” Maldonat. in Matt. xvi. 19. [p. 343.]

h
iv. D.

1
[The insertion of this paragraph here is probably a mistake; the whole of it except the quotation from St. Clement being found in other parts of this book.]

2
Matt. ix. 2; Mark ii. 7; Luke v. 21; Cypr. de Laps. c. 11*.

*
14. D.

3
Πάντα ὀνίνησιν ὁ Κύριος καὶ πάντα ὠϕελει̑, καὶ ὡς ἄνθρωπος, καὶ ὡς Θεός. Τὰ μὲν ἁμαρτήματα ὡς Θεὸς ἀϕιεὶς, εἰς δὲ τὸ μὴ ἐξαμαρτάνειν παιδαγωγω̑ν ὡς ἄνθρωπος. Clem. Alexandr. Pædag. lib. i. cap. 3.

i
a man D.

4
Esai. xliii. 25.

5
“Veniam peccatis, quæ in ipsum commissa sunt, solus potest ille largiri, qui peccata nostra portavit, qui pro nobis doluit, quem Deus tradidit pro peccatis nostris.” [de Laps. c. 11.]

k
v. D.

l
unfallable E.

1
Victor. de Persecut. Vandal. [lib. ii. ap. Bibl. Patr. Colon. t. v. pars iii. p. 655-6. Hunneric, king of the Arian Vandals in Africa, had by one edict driven into exile bishops, priests, deacons and other members of the church catholic to the number of 4961. “Quantæ tunc multitudines de diversis regionibus et civitatibus ad visendos Dei martyres occurrerent populorum testantur viæ vel semitæ; quæ cum agmina venientium nequaquam caperent, per vertices montium et vallium occurrens turba fidelium inæstimabilis descendebat, cereos manibus gestantes, suosque infantulos vestigiis martyrum projicientes, ista voce clamabant: ‘Quibus nos miseros relinquitis, dum pergitis ad coronas? qui hos baptizaturi sunt parvulos fontibus aquæ perennis? qui nobis pœnitentiæ munus collaturi sunt, et reconciliationis indulgentia obstrictos peccatorum vinculis soluturi? quia vobis dictum est, ‘Quæcunque solveritis super terram erunt soluta et in cœlis.’ Qui nos solennibus orationibus sepulturi sunt morientes? a quibus divini sacrificiritus exhibendus est consuetus? vobiscum et nos licebat pergere, si liceret ut tali modo filios a patribus nulla necessitas separaret.”]

m
universally E.

n
freed E.

o
favours E.

p
loosened D.

q
censures E.

r
offenders D.

s
habilitie D.

t
then more E.

u
vi. D.

1
[De Pudicit. c. ii. “Causas pœnitentiæ delicta condicimus. Hæc dividimus in duos exitus. Alia erunt remissibilia, alia irremissibilia. . . . Secundumquod nemini dubium est alia castigationem mereri alia damnationem. Omne delictum aut venia expungit aut pœna: venia ex castigatione; pœna ex damnatione. . . . Secundum hanc differentiam delictorum, pœnitentiæ quoque conditio discriminatur. Alia erit, quæ veniam consequi possit, in delicto scilicet remissibili. Alia, quæ consequi nullo modo possit, in delicto scilicet irremissibili.” And, c. xviii. “Pœnitentia ad se clementiam invitat, salva illa pœnitentiæ specie post fidem, quæ aut levioribus delictis veniam ab episcopo consequi poterit, aut majoribus et irremissibilibus a Deo solo.”]

2
[Ibid. c. iii. “Ad Dominum remissa [Pœnitentia] et illi exinde prostrata, hoc ipso magis operabitur veniam, quod eam a solo Deo exorat, quod delicto suo humanam pacem sufficere nec credit, quod Ecclesiæ mavult erubescere quam communicare.”]

1
[Ib. c. v. “Est et mali dignitas, quod in summo aut in medio pessimorum collocatur. Pompam quandam atque suggestum adspicio mœchiæ, hinc ducatum idololatriæ antecedentis, hinc comitatum homicidii insequentis.” Tertullian’s copies, as many of the LXX do now, apparently transposed the sixth and seventh commandments. Comp. Rom. xiii. 9.]

2
[Cap. i. “Durissime nos infamantes Paracletum disciplinæ enormitati, digamos foris sistimus: eundem limitem liminis mœchis quoque et fornicatoribus figimus; jejunas pacis lacrymas profusuris, nec amplius ab Ecclesia quam publicationem dedecoris relaturis.”]

3
[Ibid. cap. 9. “Quis enim timebit prodigere quod habebit postea recuperare? Quis curabit perpetuo conservare quod non perpetuo poterit amittere.] Securitas delicti, etiam libido est ejus.”

4
[Ibid. c. xxi. “Si et ipsos beatos Apostolos tale aliquid indulsisse constaret, cujus venia a Deo, non ab homine, competeret, non ex disciplina, sed ex potestate fecissent. Nam et mortuos suscitaverunt, quod Deus solus; et debiles redintegraverunt, quod nemo nisi Christus: immo et plagas inflixerunt, quod noluit Christus. Non enim decebat eum sævire, qui pati venerat. Percussus est Ananias et Elymas; Ananias morte, Elymas cæcitate. . . . Exhibe igitur et nunc mihi, Apostolice, prophetica exempla, et agnoscam divinitatem; et vindica tibi delictorum ejuscemodi remittendorum potestatem. Quod si disciplinæ solius officia sortitus es, nec imperio præsidere sed ministerio, quis aut quantus es indulgere? qui neque Prophetam nec Apostolum exhibens, cares ea virtute cujus est indulgere.”]

x
the impotent E.

1
Concil. Neocæsar. c. 12. [t. i. 1484. Ἐὰν νοσω̑ν τις ϕωτισθῃ̑, εἰς πρεσβύτερον ἄγεσθαι οὐ δύναται· οὐκ ἐκ προαιρέσεως γὰρ ἡ πίστις αὐτου̑, ἀλλ’ ἐξ ἀνάγκης· εἰ μὴ τάχα διὰ τὴν μετὰ ταυ̑τα αὐτου̑ σπουδὴν καὶ πίστιν, καὶ διὰ σπάνιν ἀνθρώπων. ad 314.]

2
Sozom. [Socrat. D.E.] lib. iv. cap. 23. Concil. Nicen. c. 30. [In this reference the editor apprehends some mistake.] Socrat. lib. i. cap. 7. [Quoted from the Bibl. Patr. In Valesius’ edition it is Socr. i. 10. Sozom. i. 22.]

1
[Ὡς ἄρα οὐ χρὴ τοὺς μετὰ τὸ βάπτισμα ἡμαρτηκότας ἁμαρτίαν, ἣν πρὸς θάνατον καλου̑σιν αἱ θει̑αι γραϕαὶ, τη̑ς κοινωνίας τω̑ν θείων μυστηρίων ἀξιου̑σθαι· ἀλλ’ ἐπὶ μετάνοιαν μὲν αὐτοὺς προτρέπειν· ἐλπίδα δὲ τη̑ς ἀϕέσεως μὴ παρὰ τω̑ν ἱερέων ἀλλὰ παρὰ του̑ Θεου̑ ἐνδέχεσθαι, του̑ δυναμένου καὶ ἐξουσίαν ἔχοντος συγχωρει̑ν ἁμαρτήματα. Socr. i. 10.]

2
[De Pudic. c. 1. fin.: vid. supr. note 2, p. 80.]

3
Can. viii. [Περὶ τω̑ν ὀνομαζόντων μὲν ἑαυτοὺς καθαρούς ποτε, προσερχομένων δὲ τῃ̑ καθολικῃ̑ ἐκκλησίᾳ, ἔδοξε τῃ̑ ἁγίᾳ καὶ μεγάλῃ συνόδῳ, ὥστε χειροθετουμένους αὐτοὺς, μένειν οὕτως ἐν τῳ̑ κλήρῳ· πρὸ πάντων δὲ του̑το ὁμολογη̑σαι αὐτοὺς ἐγγράϕως προσήκει, ὅτι συνθήσονται καὶ ἀκολουθήσουσι τοι̑ς τη̑ς καθολικη̑ς καὶ ἀποστολικη̑ς ἐκκλησίας δόγμασι· του̑τ’ ἔστι, καὶ διγάμοις κοινωνει̑ν, καὶ τοι̑ς ἐν τῳ̑ διωγμῳ̑ παραπεπτωκόσιν. Conc. ii. 32.]

y
vii. D.

z
absolution E.

a
not E.

b
practices E.

c
also in them E.

d
al. impose. Archbishop Ussher in MS. D, over the word inioyne. His authority for the correction was probably the use of the word in the same MS. where this passage had before occurred: see above, c. v. § 9. p. 71.

e
quite E.

f
which we E.

g
that om. E.

1
[See above, pp. 71, 72.]

h
viii. D.

2
“In peccato tria sunt; actio mala, interior macula, et sequela.” Bonav. Sent. lib. iv. d. 17. [Q. i. pars i. art. i.] q. 3. [p. 240 e. Rom. 1596.]

3
1 John iii. 4.

4
Matt. xv. 19.

1
Acts viii. [22,] 23.

i
that om. E.

2
Prov. v. 22.

k
hand E.

3
“Sacerdotes opus justitiæ exercent in peccatores cum eos justa pœna ligant; opus misericordiæ cum [dum D. E. not 1648] de ea aliquid relaxant, vel Sacramentorum communioni conciliant; alia opera in peccatores exercere nequeunt.”

Sent. lib. iv. dis. 18. [c. 5. fol. 178. Basil. 1513.]

l
bonds E.

4
Acts vii. 60; Mic. vii. 19.

5
1 Cor. vi. 11; Tit. iii. 5.

6
Luke xii. 5; Matt. x. 28.

7
2 Sam. xii. 13.

8
Luke vii. 27. [29?]

9
Mal. iii. 15.

1
Sent. [Peter Lombard, † 1164.] lib. iv. dis. 18. [c. iii, iv. fol. 178. “Hoc sane dicere ac sentire possumus, quod solus Deus dimittit peccata et retinet; et tamen Ecclesiæ contulit potestatem ligandi et solvendi: verum aliter ipse solvit vel ligat, aliter Ecclesia. Ipse enim per se tantum dimittit peccata: quoniam et animas mundat ab interiori macula, et a debito æternæ mortis solvit. Non autem hoc sacerdotibus concessit, quibus tamen tribuit potestatem solvendi et ligandi, i. e. ostendendi homines ligatos vel solutos. . . . Quoniam etsi aliquis apud Deum sit solutus, non tamen in facie Ecclesiæ solutus habetur nisi per judicium sacerdotis. . . . Ligant quoque sacerdotes dum satisfactionem pœnitentiæ confitentibus imponunt: solvunt cum de ea aliquid dimittunt, vel per eam purgatos ad sacrorum communionem admittunt.”]

m
power to the Church E.

n
to om. E.

o
in the Church of God so E.

p
satisfaction E.

2
[Ibid.]

3
Hier. t. vi. Comment. in 16. Matt. [“Legimus in Levitico de Leprosis, ubi jubentur ut ostendant se sacerdotibus, et si lepram habuerint, tunc a sacerdote immundi fiant: non quo sacerdotes leprosos faciant et immundos, sed quo habeant notitiam leprosi et non leprosi, et possint discernere qui mundus quive immundus sit. Quomodo ergo ibi leprosum sacerdos mundum vel immundum facit, sic et hic alligat vel solvit Episcopus et Presbyter, non eos qui insontes sunt vel noxii; sed pro officio suo, cum peccatorum audierit varietates, scit qui ligandus sit, quive solvendus.” t. vii. p. 125. ed. Vallarsii.]

q
ix. D.

1
[I. e. in the degree of Pope Eugenius [iv. 1431-47] addressed to the Armenians [in 1439], t. xiii. 534. “Sacramenta antiquæ legis non causabant gratiam, sed eam solum per passionem Christi dandam esse figurabant: hæc vero nostra et continent gratiam et ipsam digne suscipientibus conferunt.”]

r
ecclesiastical E.

s
doth seem E.

2
[In 4 Sent. dist. 1. q. i. art. 4. “Principale agens respectu justificationis Deus est, nec indiget ad hoc aliquibus instrumentis ex parte sua, sed propter contrarietatem ex parte hominis justificandi . . . utitur sacris quasi quibusdam instrumentis justificationis. Hujusmodi autem materialibus instrumentis competit aliqua actio ex natura propria, sicut aquæ abluere, et oleo facere nitidum corpus: sed ulterius, in quantum sunt instrumenta divinæ misericordiæ justificantis, pertingunt instrumentaliter ad aliquem effectum in ipsa anima, qui primo correspondet sacramentis, sicut est character, vel aliquid hujusmodi. Ad ultimum autem effectum, qui est gratia, non pertingunt etiam instrumentaliter, nisi dispositive, in quantum hoc, ad quod instrumentaliter effective pertingunt, est dispositio, quæ est necessitas, quantum in se est, ad gratiæ susceptionem. Et quia omne instrumentum agendo actionem naturalem, quæ competit sibi in quantum est res quædam, pertingit ad effectum, qui competit sibi in quantum est instrumentum, sicut dolabrum dividendo suo acumine pertingit instrumentaliter ad formam scamni; ideo etiam materiale elementum exercendo actionem naturalem, secundum quam est signum interioris effectus, pertingit ad interiorem effectum instrumentaliter. Et hoc est quod Augustinus dicit, quod aqua baptismi corpus tangit, et cor abluit; et ideo dicitur quod sacramenta efficiunt quod figurant.” Op. t. vii. ed. Venet. 1593.]

t
concept E.

1
Scot. Sent. lib. iv. Solut. ad 4. Quæst. et 5. [t. viii. 89, &c. ed. Wading. “Susceptio sacramenti est dispositio necessitans ad effectum signatum per sacramentum, non quidem per aliquam formam intrinsecam, per quam necessario causaret terminum vel aliquam dispositionem præviam, sed tantum per assistentiam Dei causantis illum effectum, non necessario absolute, sed necessitate respiciente potentiam ordinatam: disposuit enim universaliter, et de hoc Ecclesiam certificavit, quod suscipienti tale sacramentum ipse conferret effectum signatum.” p. 95.] Occam. in i. qu. quart. [quanti D. In iv. Sent. qu. 1. Lyons, 1495.] Alliac. Quæst. 1. in 4. Sent. [fol. 224-6. ed. Paris.]

u
them? Fulm.

x
x. D.

y
book E.

2
“Lutherani in [de E.] hac re interdum ita scribunt ut videantur a catholicis non dissentire; interdum autem apertissime scribunt contraria: at semper in eadem sententia manent, Sacramenta non habere immediate ullam efficientiam respectu gratiæ, sed esse nuda signa, tamen mediate aliquid efficere quatenus excitant et alunt fidem. . . quod ipsum non faciunt nisi repræsentando, ut Sacramenta per visum excitent fidem, quemadmodum prædicatio Verbi per auditum.” Bellarm. de Sacram. in genere, lib. ii. c. 2. [t. iii. 112.]

“Quædam signa sunt theorica, non ad alium finem instituta, quam ad significandum; alia ad significandum et efficiendum, quæ ob id practica dici possunt. . . Controversia est inter nos et Hæreticos, quod illi faciunt Sacramenta signa prioris generis. Quare si ostendere poterimus esse signa posterioris generis, obtinuimus causam.” Ib. c. viii. [p. 126. These two quotations are somewhat abridged.]

z
hath E.

1
John iii. 5. [om. E.]

2
“Semper memoria repetendum est Sacramenta nihil aliud quam instrumentales esse conferendæ nobis gratiæ causas.” Calv. in Ant. con. Trid. sess. 7. c. 5. [p. 344. ed. Gen. 1597.] “Si qui sint qui negent Sacramentis contineri gratiam quam figurant, illos improbamus.” Ibid. c. 6.

a
else is E.

3
“Iste modus non transcendit rationem signi, cum Sacramenta novæ Legis non solum significent sed causent gratiam.” [Summ. Theol.] pars iii. q. 62. art. 1. [xii. 192.]

1
Alexand. pars iv. 1. 8. memb. 3. art. v. sec. 1, et 2. [p. 94. ed. Col. Agrip. 1622.] Th. de Verit. q. 27. art. iii. [4? “Si sic se habeant sacramenta novæ legis ad gratiam, sequitur quod sint solum signa gratiæ, et ita nihil habebunt præ sacramentis veteris legis.” t. viii. 474.] Alliac. in iv. Sent. qu. 1. Capreolus in 4. d. 1. q. 1. [“In Sacramentis novæ legis est aliqua virtus gratiæ causativa.” p. 2. Venet. 1588.] Palud. [i. e. Petrus de Palude,] ibidem. Ferrar. [Ferrarius in Tho. Aquin.?] lib. iv. cont. Gent. c. 57. [Op. Aquin. t. ix. 493.]*

*
[Alexander of Hales, † 1245. Peter d’Ailly, 1354-1425. John Capreolus, Dominican of Toulouse, † 1444. Peter de la Palu, Dominican of Paris, † 1342.]

2
Eph. ii. [8.]

3
[Tho. Aquin. de Verit. 27. art. 3.]

4
“Necesse est ponere aliquam virtutem supernaturalem in Sacramentis.” [Aquin. in] Sent. iv. d. 1. q. 1. art. iv. [fol. 4. g.] “Sacramentum consequitur spiritualem virtutem cum benedictione Christi, et applicatione ministri ad usum Sacramenti.” [Id. Summ. Th.] pars iii. q. 62. art. iv. Concil. [t. xii. 193 G.] “Virtus sacramentalis habet esse transiens ex uno in aliud et incompletum.” Ibidem. “Ex Sacramentis duo consequuntur in anima, unum est character, sive aliquis ornatus; aliud est gratia. Respectu primi, Sacramenta sunt causæ aliquo modo efficientes; respectu secundi, sunt disponentes. Sacramenta causant dispositionem ad formam ultimam, sed ultimam perfectionem non inducunt.” [Idem in] Sent. iv. d. 1. q. 1. art. iv. [p. 4 A.] “Solus Deus efficit gratiam, adeo quod nec angelis, qui sunt nobiliores sensibilibus creaturis, hoc communicetur.” Ibid. [pag. 3. i.]

b
habilitie D; hability E.

1
Ad Donat. c. 3.* [“Postquam undæ genitalis auxilio superioris ævi labe detersa, in expiatum pectus serenum ac purum desuper se lumen infudit; postquam cœlitus Spiritu hausto in novum me hominem nativitas secunda reparavit; mirum in modum protinus confirmare se dubia, patere clausa, lucere tenebrosa, facultatem dare quod prius difficile videbatur, geri posse quod impossibile putabatur, ut esset agnoscere terrenum fuisse quod prius carnaliter natum delictis obnoxium viveret, Dei esse cœpisse quod jam Spiritus Sanctus animaret.” p. 2. ed. Baluzii.]

*
Epl. 2. D.

c
me om. E.

d
xi. D.

2
[In 4 Sent. d. 1. pars 1. art. i. qu. 4. p. 12.] “Cavendum est ne dum nimis damus corporalibus signis ad laudem, subtrahamus honorem causæ curanti et animæ suscipienti.”

1
Luke xviii. [viii.]; John ix.

e
virtue om. E.

2
Bellarm. de Sacr. in genere, lib. ii. c. 1. [Having quoted Luther for the words, “Omnes in hoc concedunt, sacramenta esse efficacia signa gratiæ,” he goes on, “Hoc sufficit ad fidem, et ad legitimum usum Sacramentorum; quomodo in miraculis Christi non requirebatur, ut homines qui curandi essent scirent in quo genere causæ fimbria Christi sanaret; . . . neque opus erat ut ipsi Apostoli, qui manus imponendo curabant, scirent quomodo id fieret: ita quoque non est necesse ut vel ministri vel qui suscipiunt sacramenta sciant quomodo Sacramenta sint causæ justificationis.”]

3
“Dicimus gratiam non creari a Deo, . . . sed produci . . . ex aptitudine et potentia naturali animæ, sicut cætera omnia quæ producuntur in subjectis talibus, quæ sunt apta nata ad suscipiendum accidentia.” Allen. [of Oriel College, Oxford, 1532-1594. Cardinal, 1587. Archbp. Mechlin, 1589.] de Sacr. in Gen. c. 37. [p. 132. Antwerp. 1576.]

f
thereby om. E.

1
[Decr. Eugen. ap. Concil. t. xiii. p. 534. “Hæc nostra [sacramenta] et continent gratiam, et eam digne suscipientibus conferunt.”]

2
[Sess. vii. de Sacram. can. 6. “Si quis dixerit, sacramenta novæ legis non continere gratiam quam significant, aut gratiam ipsam non ponentibus obicem non conferre; quasi signa tantum externa sint acceptæ per fidem gratiæ vel justitiæ, et notæ quædam Christianæ professionis quibus apud homines discernuntur fideles ab infidelibus, anathema esto.” xiv. 777.]

3
[The obvious corruption of the text here may perhaps be rightly removed by leaving out the word “which.”]

4
Tho. de Verit. q. 27. art. iii. resp. ad 16. [“Manus impositio non causat Sp. Sancti adventum; sed simul cum manus impositione Sp. Sanctus advenit. Unde non dicitur in textu quod Apostoli imponentes manus darent Sp. Sanctum, sed quod imponebant manus, et illi accipiebant Sp. Sanctum. Si tamen impositio manuum dicatur aliquo modo causa acceptionis Sp. Sancti per modum quo Sacramenta sunt causa gratiæ . . . . hoc non habebit manus impositio in quantum est ab homine, sed ex institutione divina.” t. viii. 472. i.] Acts viii. 18.

1
[t. xiii. 534. “Illa non causabant gratiam, sed eam solum per passionem Christi dandam esse figurabant.”]

2
“Quod ad circumcisionem sequebatur remissio, fiebat, [per accidens ratione signi,] ratione rei adjunctæ et ratione pacti divini, eodem plane modo quo non solum Hæretici, sed etiam aliquot vetustiores Scholastici voluerunt nova Sacramenta conferre gratiam.” Allen. de Sacr. in Gen. c. 39.

3
“Bonaventura, Scotus, Durandus, Richardus, Occamus, Marsilius, Gabriel, — volunt solum Deum producere gratiam ad præsentiam Sacramentorum.” Bellarm. de Sacr. in Gen. lib. ii. c. 11.

4
“Puto longe probabiliorem et tutiorem sententiam quæ dat sacramentis veram efficientiam. Primo quia Patres passim docent, sacramenta non agere nisi prius a Deo virtutem seu benedictionem seu sanctificationem accipiant, et referunt effectum sacramenti ad omnipotentiam Dei, et conferunt cum versi causis efficientibus. Secundo, quia non esset differentia inter modum agendi sacramentorum, et signorum magicorum. Tertio, quia tunc non esset homo Dei minister in ipsa actione sacramentali, sed homo præberet signum actione sua, et Deus alia actione viso eo signo infunderet gratiam, ut cum unus ostendit syngrapham mercatori, et ille dat pecunias. At Scripturæ docent, quod Deus baptizat per hominem.” Bellarm. lib. ii. cap. 11.

g
own om. E.

h
copartner E.

i
xii. D.

k
my E.

l
Rabbies E.

1
Conc. Trid. Sess. xiv. c. 4. “[Docet . . . etsi contritionem hanc aliquando caritate perfectam esse contingat, hominemque Deo reconciliare, priusquam hoc sacramentum actu suscipiatur; ipsam nihilominus reconciliationem ipsi contritioni sine sacramenti voto, quod in illa includitur, non esse ascribendam.” t. xiv. 817.]

2
Bellarm. de Pœnit. lib. ii. c. 13. [“Scriptura passim docet, eos qui toto corde ad Dominum convertuntur sine mora veniam peccatorum accipere.” “Veram conversionem nullas pati veniæ moras, proinde continuo remitti contritis peccata, etiam antequam absolutio sacerdotalis accedat.”]

m
if man D.

n
thought E.

1
“Hæc expositio, Ego te absolvo, id est, Absolutum ostendo, partim quidem vera est, non tamen perfecta. Sacramenta quippe novæ legis non solum significant, sed efficiunt quod significant.” Soto, Sent. lib. iv. dist. 14. q. 1. art. iii. [p. 350. Douay, 1613. from Aquin. 3 Summ. qu. 84. art. 3. resp. ad 5.]

o
I D.

p
xiii. D.

2
“Attritio solum dicit dolorem propter pœnas inferni; . . . dum quis accedit attritus, per gratiam Sacramentalem fit contritus.” Soto, Sent. iv. dist. 14. q. 1. art. i. [p. 347.]

1
“Dum accedit vere contritus propter Deum, illa etiam contritio non est contritio, nisi quatenus prius natura informetur gratia per Sacramentum in voto.” Soto, Sent. iv. dist. 14. q. 1. art. i.

2
“Legitima contritio votum Sacramenti pro suo tempore debet inducere, atque adeo in virtute futuri Sacramenti peccata remittit.” Idem, art. iii. [p. 350.]

q
due E.

3
“Tunc sententia sacerdotis judicio Dei et totius cœlestis curiæ approbatur, et confirmatur, cum ita ex discretione procedit, ut reorum merita non contradicant.” Sent. l. iv. d. 18. [“Quoscunque ergo solvunt vel ligant adhibentes clavem discretionis reorum meritis, solvunt vel ligant in cœlis: i.e. apud Deum.” c. 4. fol. 178.]

4
“Non est periculosum sacerdoti dicere, Ego te absolvo, illis in quibus signa contritionis videt, quæ sunt dolor, de præteritis, et propositum de cætero non peccandi; alias absolvere non debet.” Tho. Opusc. 22. [c. 3. t. xvii. p. 195.]

r
St. Cyprian’s E.

5
Cypr. de Lapsis. [See above, p. 69, note 2.]

s
sacraments D.

t
xiv. D.

1
[Sess. xiv. c. 4: vid. supr. p. 96, note 1.]

u
virtue E.

x
afterwards D.

y
should seem a thousand times to absolve E.

z
Ancients E.

1
“A reatu mortis æternæ absolvitur homo a Deo per contritionem; . . . manet autem reatus ad quandam pœnam temporalem, et minister ecclesiæ quicunque virtute clavium tollit reatum cujusdam partis pœnæ illius.” Abulens. [Tostatus, Bp. of Avila, 1400-1455] in Defensor. p. i. c. 7. [Opusc. ad calc. Comment. t. xii. p. 9. Venet. 1596.]

2
“Signum hujus Sacramenti est causa effectiva gratiæ sive remissionis peccatorum; non simpliciter, sicut ipsa prima pœnitentia, sed secundum quid; quia est causa efficaciæ gratiæ qua fit remissio peccati, quantum ad aliquem effectum in pœnitente, ad minus quantum ad remissionem sequelæ ipsius peccati, scilicet pœnæ.” Alex. p. iv. q. 14. memb. 2. [art. i. § 2. p. 467.]

3
“Potestas clavium proprie loquendo non se extendit supra culpam . . . . Ad illud quod objicitur, Joan. 20; ‘Quorum remiseritis peccata;’ dicendum, quod vel illud de remissione dicitur quantum ad ostensionem vel solum quantum ad pœnam.” Bon. Sent. lib. iv. d. 18. [pars i. art. 2.] q. i. [p. 273.]

4
“Ab æterna pœna nullo modo solvit sacerdos, sed a purgatoria*; neque hoc per se, sed per accidens, quod cum in pœnitente virtute clavium minuitur debitum pœnæ temporalis, non ita acriter punietur in purgatorio sicut si non esset absolutus.” [Bonav. in] Sent. lib. iv. d. 18. q. 3. [p. 274.]

*
purgatorio E.

a
xv. D.

b
enforce him E.

1
[Tacit. Annal. lib. vi. c. 6. “Quid scribam vobis, patres conscripti, aut quomodo scribam, aut quid omnino non scribam hoc tempore, Dii me Deæque pejus perdant, quam perire me quotidie sentio, si scio.”]

c
suspicion E.

d
vertuous E.

e
inordinate E.

f
xvi. D.

g
that D.

h
a om. E.

i
was E.

1
Matt. 21. [xii.] 31; Mark iii. 30.

k
very om. E.

l
apostates E.

2
Acts ii. 38.

m
the imposition E.

1
Heb. vi. 6.

2
Heb. x. 26.

n
xvii. D.

1
Jer. vi. 26; Micah i. 8, 9; Lam. ii. 18. “Quam magna deliquimus, tam granditer defleamus. Alto vulneri diligens et longa medicina non desit; pœnitentia crimine minor non sit.” Cypr. de Laps. [p. 192. ed. Baluz.] “Non levi agendum est contritione, ut debita illa redimantur, quibus mors æterna debetur; nec transitoria opus est satisfactione pro malis illis propter quæ paratus est ignis æternus.” Euseb. Emissenus, vel potius Salvian. f. 106. [Ad Monach. Hom. V. in Bibl. Patr. Colon. t. v. pars i. 582. g.]

o
command E.

p
for a heart E.

q
out om. D, E.

r
object om. D.

2
Psal. vi. 6.

3
Mark xii. 42.

s
near to the mark E.

4
Acts x. 31.

t
his D.

u
xviii. D.

v
secondly D.

w
they doe, which E.

x
that om. E.

y
devotions D.

1
Jer. xxix. 13; Joel ii. 12.

2
Chrys. de repar. Laps. lib. ad Theodor. [ap. Grat. Decr.] de Pœnit. dist. 3. c. Talis. [“Talis, mihi crede, talis est erga homines pietas Dei: nunquam spernit pœnitentiam si ei sincere et simpliciter offeratur. Etiamsi ad summum quis perveniat malorum, et inde tamen velit reverti ad virtutis viam, suscipit et libenter amplectitur, facit omnia quatenus ad priorem revocet statum: et quod est adhuc præstantius et eminentius, etiam si non potuerit quis explere in præsenti satisfaciendi ordinem, quantulamcunque tamen et quamlibet brevi tempore gestam non respuit pœnitentiam. Suscipit etiam ipsam, nec patitur quamvis exiguæ conversionis perdere mercedem.” See the original (from which this is much altered) in the Benedictine Chrysostom, t. i. p. 5.]

z
Saint om. E, not 1648.

1
Aug. in Psal. cxxxviii. [§ 21. “In libro tuo omnes scribentur: non solum perfecti sed etiam imperfecti. Non timeant imperfecti, tantum proficiant.” t. iv. p. 1546.]

1
[This paper is preserved in the library of C.C.C. Oxford; “N°. 297. W. C. 2. 11.” It is indorsed as follows: “Mr. S. and Mr. Cr. Notes upon the 6 and 7 bookes.” Then in Fulman’s hand, “Written with their own hands and given me by my friend M. Isaac Walton 1673. W. F.”]

2
[In Fulman’s hand. The rest of the paper in Cranmer’s hand.]

3
[See St. Jerome, Ep. ad Nepotian. § 5. t. i. p. 256. ed. Vallars.]

4
[This is one of the very few instances in which it is possible that the copy before Cranmer may have agreed with the present (so called) sixth book: in the early part of which a citation from St. Ignatius occurs. See before, p. 4.]

1
[See it quoted E. P. V. xx. 3. note 6; lxxiii. 6. note 4.]

2
[c. i. 7, 13.]

3
[The word comes in the margin of the MS. which is defaced here.]

4
[Evidently in Rom. ix. 3.]

5
[See Exod. xxxii. 32.]

1
[This may be noted as a second instance in which the note might possibly refer to the sixth book as it stands: p. 29.]

2
[c. xii. 2.]

1
[In some private letter: for of Some’s published tracts the only one which from its date could have referred to Hooker is “Questions wherein is handled that Christ died for the elect alone,” &c. Camb. 1596: in which Hooker is not mentioned.]

2
[Lucr. iii. 14, &c.]

3
[In the margin, opposite the words, “is to be sayd in defence thereof,” is the following note, as it seems in Cranmer’s hand.

“Excommunication with us you knowe is exercised by a lay commissary, although for fashion sake a minister be called in to reade the sentence. But in their discipline suppose the lay elders be of mynd to excommunicate any man, the pastor, not; Shall the pastor have a negative voice, or shall excommunication be exercised by the laymen only?”]

4
[See 2d Admonition, p. 6, 7, ed. 1617; Milton, of Reform. in Engl. Prose Works, vol. i. p. 27. ed. 1738.]

1
[Eccles. iv. 12. Comp. E. P. b. vii. c. 18. § 10.]

1
[Exod. xviii. 25, 26.]

2
[Num. xi. 25.]

1
[The MS. adds “Samuel,” with a pen drawn across it.]

2
[2 Chron. xix. 5, 6, 7.]

3
Deut. xvi. [ver. 18.]

4
[2 Chron. xix. 8-11.]

5
Deut. xvii. [ver. 8-13.]

6
[Ver. 10.]

7
[xvii. 8.]

8
[Ver. 11.]

1
[See Sutcliffe, “de Presbyterio,” p. 20, 29; and “False Semblant,” &c p. 80.]

1
[This sentence has a pen drawn across it in the MS.]

1
Deut. xvii.

1
[This seems to be a reference to Heb. xii. 2.]

2
A Fruitfull Sermon upon Rom. xii. 3-8. Lond. Waldegrave, 1584. Especially p. 34, 35. “If you ask me, how many members there be in the body, what they be, and how they be named and called, and what be their duties and callings: the Apostle himself will answer plainly in the next verses, and perfectly and fully determine all these quotations, saying, These members are either doctors to teach, pastors to exhort, elders to rule, deacons to distribute, attenders upon the poor strangers and the sick; or else the people and saints, which are taught, exhorted, ruled, and receive alms and relief. These are all, no mo, no fewer. So the necessity of relation plainly proveth; and these are such as are able to execute andperform any duty belonging to the perfect building of and adorning of the mystical body of Christ, as shall hereafter appear more at large in the particular handling of every several office.”

And p. 54. “He reduceth all the ordinary functions which were ordained in the Church, and which are perpetually to be retained for the happy success and preservation thereof, unto two general heads, that is to say, unto prophets and officers, dividing either of them into their several branches.”]

1
[In 1 Cor. xii. 28; to which this note clearly refers. See b. v. c. 78. § 8.]

2
[Viz. 1 Tim. v. 17, comp. T. C. i. 173; Def. 626; T. C. iii. 32.]

1
1 Tim. iii. 2. “For at that tyme I thinke the word presbyter and episcopus were used promiscuously. And in episcopo it was required to be able to teach.”

2
“In this point I perceave that I have mistaken Mr. Car. meaning. See what I have written in the end of all.”

3
[T. C. iii. 35; comp. Def. 628.]

4
[Pseud. Ambros. in 1 Tim. v. 1. “Apud omnes utique gentes honorabilis est senectus: unde et synagoga, et postea Ecclesia seniores habuit, quorum sine consilio nihil agebatur in Ecclesia. Quod qua negligentia obsoleverit, nescio; nisi forte doctorum desidia, aut magis superbia dum soli volunt aliquid videri.” ed. Bened. t. ii. App. 298. vid. Whitgift’s Answ. 132; T. C. i. 182; Def. 651; T. C. iii. 44.]

1
[Tert. Ap. 39; T. C. iii. 41.]

2
[S. Cyp. Ep. 1. p. 2; T. C. iii. 42. Cf. V. lxxx. 11. note 1; VII. xxiii. 9.]

3
[T. C. iii. 43. quoting Socr. H. E. v. 22. “At Alexandria, after Arius was convicted of heresy, it was decreed that the elders should no more teach: by which decree they did as it were covertly confess that they had received the reward of breaking the order of God, in permitting that the elder should teach in the Church.”]

4
[Hier. adv. Lucif. 9; T. C. iii. 43.]

1
[T. C. iii. 45, quoting S. Ign. ad Trall. Ep. interp. c. 3. ὁ ἐπίσκοπος, του̑ Πατρὸς τω̑ν ὅλων τύπος ὑπάρχει· οἱ δὲ πρεσβύτεροι, ὡς συνεδρίον Θεου̑, καὶ σύνδεσμος ἀποστόλων Χριστου̑· χωρὶς τούτων ἐκκλησία ἐκλεκτὴ οὐκ ἔστιν. ap. Coteler. ii. 61.]

2
[T. C. iii. 42. “Valerius . . . did contrary to the custom of theAfrican church, in that he committed the office of teaching unto Augustin which was an elder.” He refers to Posidonius, or Possidius, Vit. Aug. c. 5. “Valerius Augustino presbytero potestatem dedit coram se in ecclesia evangelium prædicandi contra usum. . . Africanarum ecclesiarum.”]

1
[E. H. V. 22.+]

1
[Edwin Sandys in Fulman’s hand; the rest in Sandys’ own hand.]

2
Vid. in p. 16. [referring to Hooker’s MS.]

1
[“Non intendimus judicare de feudo” is Innocent the Third’s disavowal of temporal jurisdiction in the dispute between king John and Philip Augustus, ad 1204. See Decretal. ix. i. 13. col. 489. ed. Lugd. 1572.]

1
[Probably Mr. Serjeant Yelverton, who was chosen Oct. 27, 1597, and continued in office till the 9th Feb. following, when the parliament was dissolved. Cobbett’s Parliamentary Hist. i. 895, 905.]

1
[Qu. Hanoth? vid. Lightf. t. i. p. 1062, and Buxtorf. voc. הָנוה.]

2
[i.e. Walter Travers.]

1
The harvest greate, the labourers few: i. e. Preachers.

1
[“Si quæris quare in ecclesia baptizatus nisi per manus Episcopi non accipiat Sp. Sanctum,” &c. Adv. Lucif. § 9. t. ii. 182. ed. Vallarsii.]

2
[Ubi supra; (speaking of baptism;) “Frequenter, (si tamen necessitas cogit,) scimus etiam licere laicis.” t. ii. 139. ed. Frob. Basil.]

1
[On 1 Tim. v. 1.]

1
[This anecdote must relate either to France or Scotland: the editor has not succeeded in tracing it in either history. From the tone in which Hooker describes the state of the country, it would seem that France was meant; comp. b. iii. c. xi. § 14. The anecdote might be one of the many reports of what had passed in the conference at Poissy, 1561.]

1
[Lam. ii. 13.]

1
[Job xxix. 21, 22, 25.]

2
[Job xxx. 1-9.]

3
Cyp. lib. i. Ep. 3. [al. Ep. 59. c. 3. “Exaltatio, et inflatio, et arrogans ac superba jactatio non de Christi magisterio, qui humilitatem docet, sed Antichristi spiritu nascitur.” ii. 127. ed. Fell.]

4
[Bed. Hist. Eccl. i. 4.]

1
Sulpit. Sever. lib. ii. [c. 55. “Missis per Illyricum, Italiam, Africam, Hispanias, Galliasque magistris officialibus, acciti ac in unum coacti quadringenti et aliquanto amplius Occidentales Episcopi, Ariminum convenere; quibus omnibus annonas et cellaria dare imperator præceperat; sed id nostris, i. e. Aquitanis, Gallis ac Britannis indecens visum, repudiatis fiscalibus, propriis sumptibus vivere maluerunt. Tres tantum ex Britannia, inopia proprii, publico usi sunt.” More than three it seems were present from Britain, but three only received the public allowance.]

2
Beda Eccl. Hist. lib. ii. c. 2. [“Augustinus . . . . convocavit ad suum colloquium episcopos sive doctores proximæ Britonum provinciæ . . . Convenerunt, ut perhibent, septem Britonum episcopi et plures viri doctissimi.”]

3
An. 1066.

4
“Alfredus Eboracensis Archiepiscopus Gulielmum cognomento Nothum spirantem adhuc minarum et cædis in populum mitem reddit: et religiosis pro conservanda repub. tuendaque ecclesiastica disciplina sacramentis adstrinxit.” Neubrig. l. i. c. 1. [William of Newbridge, 1135-1208.] [ap. Rer. Britannic. Script. Heidelberg, 1587. p. 357.]

5
[Rather Albinus the successor of Festus; of whom Josephus writes, B. J. ii. 14. ed. Huds. Οὐκ ἔστι δὲ ἥν τινα κακουργίας ἰδέαν παρέλιπεν· . . . τοιου̑τον δὲ ὄντα τὸν Ἀλβι̑νον ἀπέδειξεν ὁ μετὰ του̑τον ἔλθων Γέσσιος Φλω̑ρος ἀγαθώτατον κατὰ σύγκρισιν.]

1
Οἱ παρ’ Ἀθηναίων εἰς τὰς ὑπηκόους πόλεις ἐπισκέψασθαι τὰ παρ’ ἑκάστοις πεμπόμενοι, Ἐπίσκοποι καὶ ϕύλακες ἐκαλου̑ντο· οὓς οἱ Λάκωνες ἁρμοστὰς ἔλεγον. Suid. [voc. ἐπίσκοπος.] Κατέστησεν ἐϕ’ ἑκάστοις [ἑκάστου] τω̑ν πάγων ἄρχοντα ἐπίσκοπόν τε καὶ περίπολον τη̑ς ἰδίας μοίρας. Dionys. Halicar. de Numa Pompilio, Antiq. lib. ii. [c. 76.] “Vult me Pompeius esse quem tota hæc Campania et maritima ora habeat Ἐπίσκοπον, ad quem delectus et negotii summa referatur.” Cic. ad Attic. lib. vii. Epist. 11.

1
Acts xx. 28; Phil. i. 1.

2
“And God brought them unto Adam, that Adam might see or consider what name it was meet he should give unto them.” Gen. ii. 19.

3
So also the name deacon, a minister appropriated to a certain order of ministers.

4
The name likewise of a minister was common to divers degrees, which now is peculiarly among ourselves given only to pastors, and not, as anciently, to deacons also.

1
[Othello, Act ii. sc. 1. “O most lame and impotent conclusion!” The date of this play is 1611. The phrase is a translation of “manca et debilis,” which had somehow become proverbial. Cf. in Facciolati, Plaut. Mercator. 3. 4. 45, “Cæcus, mutus, mancus, debilis;” Liv. 7, 13, “mancorum et debilium ducem;” Cic. pro Mil. 9, “mancam ac debilem Præturam.”]

1
“Meminisse diaconi debent, quoniam apostolos, id est, episcopos et præpositos, Dominus elegit.” Cypr. l. iii. ep. 9. [al. ep. 65. p. 113, ed. Baluz.]

2
Rom. ii. 14, 15; 1 Cor. ix. 16; John xxi. 15, 16.

1
Gal. ii. 8.

2
Him Eusebius doth name the governor of the churches in Asia, lib. iii. Hist. Eccles. c. 16. [i. Θω̑μας μὲν, ὡς ἡ παράδοσις περιέχει, τὴν Παρθίαν εἴληχεν, Ἀνδρέας δὲ τὴν Σκυθἰαν, Ἰωαννὴς τὴν Ἀσίαι· πρὸς οὕς καὶ διατρίψας, ἐν Ἐϕέσῳ τελευτᾳ̑.] Tertullian calleth the same churches St. John’s foster-daughters, advers. Marcion. [lib. iv. c. 5. “Si constat, id verius quod prius, id prius quod et ab initio, id ab initio quod ab apostolis; pariter utique constabit, id esse ab apostolis traditum quod apud ecclesias apostolorum fuerit sacrosanctum. Videamus quod lac a Paulo Corinthii hauserint; ad quam regulam Galatæ sint recorrecti; quid legant Philippenses, Thessalonicenses, Ephesii; quid etiam Romani de proximo sonent, quibus evangelium et Petrus et Paulus sanguine quoque suo signatum reliquerunt. Habemus et Joannis alumnas ecclesias. Nam etsi Apocalypsin ejus Marcion respuit, ordo tamen episcoporum ad originem recensus in Joannem stabit auctorem. Sic et cæterarum generositas recognoscitur.”]

3
“Jacobus, qui appellatur frater Domini, cognomento Justus, post passionem Domini statim abapostolis Hierosolymorum episcopus ordinatus est.” Hieron. Scrip. Eccles. Catal. ii. [al. De Viris Illustr. c. 2. t. ii. 815, ed. Vallars.] “Eodem tempore Jacobum primum sedem episcopalem Ecclesiæ, quæ est Hierosolymis, obtinuisse memoriæ traditur.” [τότε δη̑τα καὶ Ἰάκωβον, τὸν του̑ Κυρίου λεγόμενον ἀδελϕὸν. . . πρω̑τον ἱστορου̑σι τη̑ς ἐν Ἱεροσολύμοις ἐκκλησίας τὸν τη̑ς ἐπισκοπη̑ς ἐγχειρισθη̑ναι θρόνον.] Euseb. Hist. Ecclesiast. lib. ii. cap. 1. The same seemeth to be intimated, Acts xv. 13; xxi. 18.

1
Acts xii. 2; xiii. 2.

2
Titus i. 5.

3
This appeareth by those subscriptions which are set after the epistle to Titus, and the second to Timothy, and by Euseb. Eccles. Hist. lib. iii. cap. 4. [§ 2. Τιμόθεός γε μὴν τη̑ς ἐν Ἐϕέσῳ παροικίας ἱστορει̑ται πρω̑τος τὴν ἐπισκοπὴν εἰληχέναι· ὡς καὶ Τίτος τω̑ν ἐπὶ Κρήτης ἐκκλησιω̑ν.]

4
Iren. lib. iii. cap. 3. [“Habemus annumerare eos qui ab Apostolis instituti sunt Episcopi in Ecclesiis.”]

5
[Ibid. § 3. οἰκοδομήσαντες οἱ μακάριοι ἀπόστολοι τὴν ἐκκλησίαν, Λίνῳ τη̑ς ἐπισκοπη̑ς λειτουργίαν ἐνεχείρισαν.]

6
[Ibid. § 4. Πολύκαρπος . . . οὐ μόνον ὑπὸ ἀποστόλων μαθητευθεὶς, καὶ συναναστραϕεὶς πολλοι̑ς τοι̑ς τὸν Χριστὸν ἑωρακόσιν, ἀλλὰ καὶ ὑπὸ ἀποστόλων κατασταθεὶς εἰς τὴν Ἀσίαν ἐν τῃ̑ ἐν Σμύρνῃ ἐκκλησία ἐπίσκοπος.]

7
In Ep. [adscript.] ad Antioch. [c. 7. μνημονεύσατε Εὐοδίου του̑ ἀξιομακαρίστου ποιμένος ὑμω̑ν, ὁς πσω̑τος ἐνεχειρίσθη παρὰ τω̑ν ἀποστόλων τὴν ὑμετέραν προστασίαν· μὴ καταισχύνωμεν τὸν πατέρα· γενώμεθα γνήσιο παι̑δες, ἀλλὰ μὴ νόθοι.]

1
Hieron. ep. 85. [al. 101, § 1. “Omnes Apostolorum successores sunt.”]

2
Cypr. Ep. ad Flor. [ep. 66. c. 3. ed. Fell.]

3
Theod. in 1 Tim. iii. [1. τοὺς αὐτοὺς ἐκάλουν τοτὲ πρεσβυτέρους καὶ ἐπισκόπους· τοὺς δὲ νυ̑ν καλουμένους ἐπισκόπους, ἀποστόλους ὠνόμαζον.]

4
“Ipsius apostolatus nulla successio. Finitur enim legatio cum legato, nec ad successores ipsius transit.” Stapl. [Thomas Stapleton, 1535-1598, a famous Roman Catholic controversialist.] Doct. Prin. lib. vi. cap. 7. [Opp. i. 213.]

5
Acts i. 21, 22; 1 John i. 3; Gal. i. 1; Apoc. xxi. 14: Matt. xxviii. 19.

6
[“Omnia Dei dona quæ fuerunt in Apostolis et Evangelistis propius erunt inspicienda, ut sciamus, quid Apostolis eorumque temporibus fuerit peculiare, quid commune futurum sit cæteris omnibus Ecclesiæ ministris, usque ad consummationem sæculi. Primum quod in Apostolis nobis est considerandum, est vocatio illa extraordinaria, quæ proxime a Deo est facta: deinde, legatio nullis circumscripta finibus: tertium, quod in iis omnibus quæ ad ipsorum spectabant officium, infallibilem habuerunt directorem, Sp. Sanctum, qui suggessit ipsis quæcunque prius a Domino audiverant, et omnia quæ ad hominum salutem et ecclesiæ ædificationem erant necessaria, adeo ut in ipsorum potestate non fuerit a veritate deflectere. Postremum est ipsius apostolatus potestas.

“Priora illa tria fuerunt necessaria ponendis fundamentis ecclesiarum super quæ alii superstruerent, quæ nisi certam conjunctam Sp. Sancti haberent firmitatem, labasceret quicquid ab aliis postea superstructum fuit. Edendi miracula gratiam prætereo, quod illa data sit non Apostolis tantum aliisque Ecclesiæ pastoribus, sed quibusvis ut Deo visum fuit fidelibus, ut de fide in Filium Dei certam et indubitatam fidem facerent. Ex omnibus his donis nihil successoribus communicare potuerunt præter evangelii ministerium: quod cum Apostolicæ potestati conjunctum sit, eam simul ad posteros transmiserunt: utpote quæ non tantum propagandis, verum etiam conservandis ecclesiis sit necessaria. Sine verbi Dei prædicatione, et sacramentorum usu, ac ecclesiastico regimine, nulla ecclesia recte potest subsistere. Quemadmodum prædicatio verbi Dei, baptismus, et cœna Domini non sunt data Ecclesiæ, ut tantum servirent temporibus Apostolorum, sed etiam futuris sæculis usque ad Domini adventum; sic etiam regiminis forma quæ ab ipso Domino fuit instituta, et ab Apostolis tradita, et usu patrum confirmata, permanere debet. Illa autem habuit inferiores et superiores pastores: ergo id in Ecclesia Christi retinendum est.” Saravia de Div. Ministr. Grad. c. 14. p. 33.]

1
Acts xx. 36, 37.

2
Acts xx. 28.

3
As appeareth both by his sending to call the presbyters of Ephesus before him as far as to Miletum (Acts xx. 17) which was almost fifty miles, and by his leaving Timothy in his place with his authority and instructions for ordaining of ministers there (1 Tim. v. 22); and for proportioning their maintenance (ver. 17, 18); and for judicial hearing of accusations brought against them (ver. 19) and for holding them in an uniformity of doctrine (ch. i. 3).

1
Acts xx. 30.

2
Rev. ii.

3
Cypr. iv. Epist. 9. [al. ep. 66. c. 6.]

1
Hieron. epist. ad Evag. [101. ad Evang. “Cum Apostolus perspicue doceat, eosdem esse presbyteros, quos et episcopos. . . . Quod autem postea unus electus est, qui cæteris præponeretur, in schismatis remedium factum est, ne unusquisque ad se trahens Christi ecclesiam rumperet.”]

2
Exod. xviii. 19.

3
Ep. ad Januar. [108. al. 54. c. i. t. ii. 124.]

1
Ep. ci. ad Evagr. [ad Evan. § 1. “Nam et Alexandriæ a Marco Evangelista usque ad Heraclam et Dionysium episcopos, presbyteri semper unum ex se electum, in excelsiore gradu collocatum, episcopum nominabant.”]

2
T. C. lib. i. p. 82. “It is to be observed that Jerome saith, it was so in Alexandria; signifying that in other churches it was not so.”

1
[Euseb. E. H. vii. 11.]

2
[Id. vii. 32.]

3
Socr. E. H. i. 5.

4
[Ibid. Καί ποτε παρόντων τω̑ν ὑπ’ αὐτὸν πρεσβυτέρων καὶ τω̑ν λοιπω̑ν κληρικω̑ν, ϕιλοτιμότερον περὶ τη̑ς ἁγίας Τριάδος, ἐν Τριάδι Μονάδα εἰ̑ναι ϕιλοσοϕω̑ν, ἐθεολόγει. Ἄρειος δέ τις πρεσβύτερος τω̑ν ὑπ’ αὐτὸν ταττομένων, ἀνὴρ οὐκ ἄμοιρος τη̑ς διαλεκτικη̑ς λέσχης . . . γοργω̑ς ὑπήντῃσε πρὸς τὰ παρὰ του̑ ἐπισκόπου λεχθέντα.]

5
Unto Ignatius, bishop of Antioch, Hero a deacon there was made successor. [Euseb. E. H. iv. 36. 3. Ign. ep. adscr. ad Heron. t. ii. p. 108, ed. Coteler.] Chrysostom, being a presbyter of Antioch, was chosen to succeed Nectarius in the bishopric of Constantinople. [Soc. vi. 2.]

1
[Ep. cxlvi. ad Evag. “Quid patitur mensarum ac viduarum minister, ut supra eos se tumidus efferat, ad quorum preces Christi corpus sanguisque conficitur? . . .Manifestissime comprobatur, eundem esse episcopum atque presbyterum. Quod autem postea unus electus est, qui cæteris præponeretur, in schismatis remedium factum est. . . . Nam et Alexandriæ . . . presbyteri semper unum ex se electum . . . episcopum nominabant . . . Nec altera Romanæ urbis ecclesia, altera totius orbis existimanda est. Et Galliæ, et Britanniæ, et Africa, et Persis, et Oriens, et India, et omnes barbaræ nationes unum Christum adorant, unam observant regulam veritatis. Si auctoritas quæritur, orbis major est urbe. Ubicunque fuerit episcopus, sive Romæ, sive Eugubii, sive Constantinopoli, sive Rhegii, sive Alexandriæ, sive Tanis, ejusdem meriti, ejusdem est et sacerdotii. Potentia divitiarum et paupertatis humilitas vel sublimiorem vel inferiorem episcopum non facit . . . Sed dices, quomodo Romæ ad testimonium diaconi presbyter ordinatur? Quid mihi profers unius urbis consuetudinem? Quid paucitatem, de qua ortum est supercilium, in leges Ecclesiæ vindicas? . . . Diaconos paucitas honorabiles, presbyteros turba contemptibiles facit. Cæterum etiam in ecclesia Romæ, presbyteri sedent et stant diaconi. . . . Qui provehitur, de minori ad majus provehitur. Aut igitur ex presbytero ordinetur diaconus, ut presbyter minor diacono comprobetur; aut si ex diacono ordinatur presbyter, noverit se lucris minorem, sacerdotio esse majorem. Et ut sciamus traditiones Apostolicas sumtas de veteri Testamento; quod Aaron et filii ejus atque Levitæ in templo fuerunt, hoc sibi episcopi et presbyteri et diaconi vendicent in Ecclesia.” t. i. 1074-77, ed. Vallars.]

1
V. 5. [t. vii. 694 E. “Antequam Diaboli instinctu studia in ecclesia [religione] fierent, et diceretur in populis, Ego sum Pauli, ego Apollo, ego autem Cephæ; communi presbyterorum consilio Ecclesiæ gubernabantur. Postquam vero unusquisque eos quos baptizaverat suos putabat esse, non Christi, in toto orbe decretum est, ut unus de presbyteris electus superponeretur cæteris, ad quem omnis ecclesiæ cura pertineret, et schismatum semina tollerentur.” Saravia remarks on this passage, “Quod hic dicitur communi presbyterorum consilio ecclesias in principio fuisse gubernatas, non diffiteor: sed hoc non arguit dominicæ institutionis episcopos non fuisse postea præpositos ecclesiæ, non magis quam presbyteros et diaconos non ex ordinatione divina creatos ab Apostolis, quia ecclesiæ absque presbyteris et diaconis sub apostolis regebantur, antequam crearentur diaconi et presbyteri.” c. 23. p. 51. “Inde non sequitur, ab apostolis, ubi viros idoneos Deus dederit, non fuisse præfectos singulis ecclesiis singulos episcopos supra ipsos presbyteros, qui in apostolorum locum succederent, et illa eadem præstarent, quæ ipsi præstitissent, si ubique semper præsentes ecclesiis adesse, aut semper vivere potuissent.” p. 52.]

2
[Sarav. Tract on diverse Degrees of Ministers, Eng. Transl. p. 65. Lond. 1591. “But now those factions begun under the apostles, and therefore that custom began in good time, and the Apostles themselves for the avoiding of schism altered (if not abrogated) the Lord’s institution. The which, methinks, were more than absurd to say. Our Saviour, no doubt, who is the wisdom of His Father, knew much better than the Apostles what was needful and commodious for the preventing of schism. Whom as it did not beseem to seem more wise than their master, so was it not their parts for the default of one church to alter God’s institution. Again, how knew Hierome, that before those schisms brake forth the church of Corinth had their elders, by whose council they were ruled . . . Neither do we read at any time that the elders of the church of Corinth gave the occasion of this schism, but that it was taken of the people by reason of that opinion they had of their pastors and elders . . . . They for whose sake this schism was set abroad at Corinth were not at Corinth: so that for the avoiding of this schism the elders which were to be set in some better order under one bishop were Paul himself and Apollos and Cephas,” &c. And p. 67. “The error of Hierom and Aërius grew of the . . . confused use of these titles (a Bishop and an Elder) as they were then in use. But when the same thing befalleth the title of an Apostle also, is it not strange that they should rather err in the one than the other? For whereas Barnabas, Epaphroditus, and many others are called apostles; yet no man thereby ever thought that there was no difference between them and the twelve apostles.”]

1
Ibid. v. 5. [vii. 695 E. “Sicut ergo Presbyteri sciunt se ex Ecclesiæ consuetudine ei qui sibi præpositus fuerit, esse subjectos; ita episcopi noverint se magis consuetudine, quam dispositionis Dominicæ veritate, Presbyteris esse majores, et in commune debere Ecclesiam regere.”]

2
Bishops he meaneth by restraint; for episcopal power was always in the Church instituted by Christ himself, the apostles being in government bishops at large; as no man will deny;—having received from Christ himself that episcopal authority. For which cause Cyprian hath said of them: “Meminisse diaconi debent quoniam apostolos, id est episcopos et præpositos, Dominus elegit: diaconos autem post ascensum Domini in cœlos apostoli sibi constituerunt episcopatus sui et ecclesiæ ministros.” Lib. iii. Ep. 9. [al. Ep. 3. c. 2.]

*
[A new paragraph begins here in Gauden’s ed.]

1
[It is obvious that this sentence is an insertion by mistake into the text of a note on the rough draft of the work, either by Hooker or by some friend (most probably the latter): according to the remark of Dr. Mac Crie, Life of Melville, vol. i. p. 462. The following sentences, down to “perpetual continuance thereof,” are by Gauden printed in Italics, probably because he found them underscored in Hooker’s MS. But the sense, it is apprehended, will be more exactly given by omitting the Italics, (which were probably an insertion of the critic,) and reading the whole as one paragraph with the exception of the supposed marginal note.]

1
[Saravia’s remark however is, “Privatam fuisse Hieronymi opinionem, consentaneam cum Aërio, et Dei verbo contrariam.” c. 23.]

2
Lib. ii. Hæres. 66. [c. 20.]

3
De Præscript. advers. Hæret. [c. 32. “Edant ergo origines ecclesiarum suarum, evolvant ordinem episcoporum suorum, ita per successiones ab initio decurrentem, ut primus ille episcopus aliquem ex apostolis, vel apostolicis viris, qui tamen cum apostolis perseveraverit, habuerit auctorem et antecessorem. Hoc enim modo Ecclesiæ apostolicæ census suos deferunt: sicut Smyrnæorum Ecclesia Polycarpum ab Joanne conlocatum refert.”]

*
[So printed, as a parenthesis, in Gauden’s ed.]

1
Acts xiii. 2.

2
Acts viii. 26.

3
Acts xvi. 6.

4
Ver. 7.

5
1 Tim. 1. 18.

1
[Sutcliffe de Presbyt. 119. “Ex istis hæ eliciuntur conclusiones: episcoporum supra presbyteros gradum, cum a synodis confirmetur, a Patribus tanquam divina probetur, cœperitque Apostolorum temporibus, et nunquam nisi nuper, a nuper exortis tenebrionibus condemnata fuerit, omnesque qui contra senserunt pro hæreticis habiti sint: divinam esse ejusdem originem; presbyterium vero, cum a synodis et Patribus ignoretur, figmentum esse humanum.”]

2
Aug. Ep. 19. [al. 82. c. 4. fin.] ad Hieron. [t. ii. 202. “Quanquam secundum honorum vocabula, quæ jam Ecclesiæ usus obtinuit, episcopatus presbyterio major sit, tamen in multis rebus Augustinus Hieronymo minor est.”] et de Hæres. 53. [t. viii. 18. “Aërius . . . dicebat etiam presbyterum ab episcopo nulla differentia debere discerni.”]

1
1 Cor. vii. 25; 1 Tim. v. 9.

2
Tertull. de vel. Virg. [c. 9. “Scio alicubi virginem in viduatu ab annis nondum viginti collocatam; cui si quid refrigerii debuerat episcopus, aliter utique salvo respectu disciplinæ præstare potuisset.”]

3
Epiph. lib. iii. Hær. 75. [c. 4. speaking of Aërius. Ὅτι μὲν ἀϕροσύνης ἐστὶ τὸ πα̑ν ἔμπλεων, τοι̑ς σύνεσιν κεκτημένοις, του̑το δη̑λον· τὸ λέγειν αὐτὸν ἐπίσκοπον καὶ πρεσβύτερον ἰ̑σον εἰ̑ναι· καὶ πω̑ς ἔσται του̑το δυνατόν; ἡ μὲν γάρ ἐστι πατέρων γεννητικὴ τάξις· πατέρας γὰρ γεννᾳ̑ τῃ̑ ἐκκλησίᾳ· ἡ δὲ πατέρας μὴ δυναμένη γεννᾳ̑ν, διὰ τη̑ς του̑ λουτρου̑ παλιγγενεσίας τέκνα γεννᾳ̑τῃ̑ ἐκκλησίᾳ, οὐ μὴν πατέρας, ἢ διδασκάλους.]

4
Acts xiv. 23.

5
Tit. i. 5.

6
1 Tim. v. 22.

1
“Apud Ægyptum presbyteri consignant, si præsens non sit episcopus.” Com. q. vulgo Ambros. dic. in 4. ep. ad Ephes. [§ 9. in App. 241. ed. Bened.]

2
[Concil. Carthag. iv. can. 3. t. i. 979. ed. Harduin. ad 398. “Presbyter cum ordinatur, episcopo eum benedicente, et manum super caput ejus tenente, etiam omnes presbyteri qui præsentes sunt manus suas juxta manum episcopi super caput illius teneant.”]

3
[Matt. xix. 28.]

1
Numb. iii. 32.

2
Numb. iv. 27.

3
2 Chron. xix. 11.

4
Joseph. Antiq. p. 612. [του̑τον θεραπεύουσι μὲν διὰ παντὸς οἱ ἱερει̑ς, ἡγει̑ται δὲ τούτων ὁ πρω̑τος ἀεὶ κατὰ γένος. οὑ̑τος μετὰ τω̑ν συνιερέων θύσει τῳ̑ Θεῳ̑, ϕυλάξει τοὺς νόμους, δικάσει περὶ τω̑ν ἀμϕισβητουμένων, κολάσει τοὺς ἐλεγχθέντας ἐπ’ ἀδίκῳ· ὁ δέ γε τούτῳ μὴ πειθόμενος, ὑϕέξει δίκην ὡς εἰς τὸν Θεὸν αὐτὸν ἀσεβω̑ν. Contr. Apion. II. 23.]

5
[E. g. Beza, Respons. ad Saraviam, De divers. Grad. Ministr. Evang. c. 14. § 2. in Tract. Sarav. p. 136. “Respondeo non fuisse æquale neque sacerdotum neque Levitarum inter se ministerium. Fuerunt enim aliæ et eminentiores summi sacerdotis, quam aliorum infra ipsum, partes; ut cui soli sacrarium ingredi liceret, ut Jesu Christi ecclesiæ suæ capitis unici typo.” comp. “De Triplici Sacerdotio,” p. 60.]

1
[De Brés, “Racine, Source, et Fondement des Anabaptistes.” p. 822. “Plusieurs de nos Anabaptistes pensent bien d’échapper de tant de témoignages qui sont contr’eux, disant, que tous ces témoignages sont pris du Vieil Testament, et qu’ils ne doivent avoir lieu au Nouveau, en tant que notre Seigneur requiert une perfection plus grande en l’église Chrétienne qu’il n’a pas fait au peuple Judaique.” Comp. p. 825. “Les Anabaptistes pensent bien tout renverser, quand ils nous repliquent le dire du Prophête Esaie, ii. 4; xi. 6.” &c.]

2
Cypr. l. iii. Ep. 9. [65. ed. Baluz.] ad Rogatianum. [“Tu quidem honorifice circa nos et pro solita tua humilitate fecisti, ut malles de eo nobis conqueri, cum pro episcopatus vigore et cathedræ auctoritate haberes potestatem qua posses de illo statim vindicari, . . . habens circa hujusmodi homines præcepta divina, cum Dominus Deus in Deuteronomio dicat, ‘Et homo quicunque fecerit in superbia, ut non exaudiat sacerdotem aut judicem quicunque fuerit in diebus illis,’ &c. . . . Et ut sciamus hanc Dei vocem cum vera et summa majestate ejus processisse ad honorandos ac vindicandos sacerdotes suos, cum adversus Aaron sacerdotem tres de ministris, Chore, et Dathan, et Abiron ausi sunt superbisse et cervicem suam extollere, et sacerdoti præposito se adæquare, hiatu terræ absorpti ac devorati pœnas statim sacrilegæ audaciæ persolverunt. . . . Ut probaretur sacerdotes Dei ab eo qui sacerdotes facit vindicari.”]

1
Hier. Ep. 85. [al. 146. fin. vid. supr. c. v. § 6. p. 160, note 1.]

2
Ep. ad Smyr. [c. 9. vid. supr. b. vi. c. ii. § 1. p. 4, note 4.]

3
1 Tim. v. 19. “Against a presbyter receive no accusation under two or three witnesses.”

4
Ignat. [adscr.] Epist. ad Antioch. [c. 8.]

5
Apud Cypr. Ep. ii. 7. [31. “Quanquam nobis differendæ hujus rei necessitas major incumbat, quibus post excessum nobilissimæ memoriæ viri Fabiani nondum est episcopus propter rerum et temporum difficultates constitutus, qui omnia ista moderetur, et eorum qui lapsi sunt possit cum auctoritate et consilio habere rationem.”]

1
“Episcopi universæ plebi mandare jejunia assolent.” Tertull. advers. Psychic. [c. 13.]

2
Cypr. Ep. 27. [al. 33. “Dominus noster, cujus præcepta et monita observare debemus, episcopi honorem et ecclesiæ suæ rationem disponens in evangelio loquitur et dicit Petro, ‘Ego tibi dico quia tu es Petrus,’ &c. . . . . . Inde per temporum et successionum vices episcoporum ordinatio et ecclesiæ ratio decurrit, ut ecclesia super episcopos constituatur, et omnis actus ecclesiæ per eosdem præpositos gubernetur.”]

3
Cypr. Ep. 39. [al. 5. ed. Baluz. Fretus et dilectione et religione vestra, quam satis novi, his literis et hortor et mando, ut vos, quorum minime illic invidiosa et non adeo periculosa præsentia est, vice mea fungamini circa gerenda ea quæ administratio religiosa deposcit.”]

4
Vide Ignat. ad Magnes. [c. vi. προκαθημένου του̑ ἐπισκόπου εἰς τόπον Θεου̑, καὶ τω̑ν πρεσβυτέρων εἰς τόπον συνεδρίου τω̑ν ἀποστόλων, καὶ τω̑ν διακόνων, τω̑ν ἐμοὶ γλυκυτάτων, πεπιστευμένων διακονίαν Ἰησου̑ Χριστου̑ . . c. vii. ὥσπερ οὐ̑ν ὁ Κύριος ἄνευ του̑ Πατρὸς οὐδὲν ἐποίησε, ἡνωμένος ὢν, οὔτε δι’ αὐτου̑, οὔτε διὰ τω̑ν Ἀποστόλων· οὕτως μηδὲ ὑμει̑ς ἄνευ του̑ ἐπισκόπου καὶ τω̑ν πρεσβυτέρων μηδὲν πράσσετε. . . c. xiii. ὑποτάγητε τῳ̑ ἐπισκόπῳ καὶ ἀλλήλοις, ὥσπερ Ἰησου̑ς Χριστὸς τῳ̑ Πατρὶ κατὰ σάρκα, καὶ οἱ ἀπόστολοι τῳ̑ Χριστῳ̑ καὶ τῳ̑ Πατρὶ καὶ τῳ̑ Πνεύματι, ἵνα ἕνωσις ᾐ̑ σαρκική τε καὶ πνευματική.]

1
“Quod Aaron et filios ejus, hoc episcopum et presbyteros esse noverimus.” Hier. ad Nepotianum, ep. 2. [al. 52. § 7. t. i. p. 260. ed. Vallarsii.]

2
“Ita est, ut in episcopis Dominum, in presbyteris Apostolos recognoscas.” Auctor Opusc. de septem Ordinib. Eccl. inter Opera Hieron. [t. xi. 123.]

3
Ignat. [interp.] Ep. ad Trall. [c. 7. τί γάρ ἐστιν ἐπίσκοπος, ἀλλ’ ἢ πάσης ἀρχη̑ς καὶ ἐξουσίας ἐπέκεινα πάντων κρατω̑ν, ὡς οἱ̑όν τε ἄνθρωπον κρατει̑ν, μιμητὴν γινόμενον κατὰ δύναμιν Χριστου̑ του̑ Θεου̑.]

4
Instit. lib. iv. cap. 4. § 2. [“Quibus docendi munus injunctum erat, eos omnes nominabant presbyteros. Illi ex suo numero in singulis civitatibus unum eligebant, cui specialiter dabant titulum episcopi; ne ex æqualitate, ut fieri solet, dissidia nascerentur. Neque tamen sic honore et dignitate superior erat episcopus ut dominium in collegas haberet; sed quas partes habet consul in senatu, ut referat de negotiis, sententias roget, consulendo, monendo; hortando, aliis præeat, authoritate sua totam actionem regat, et quod decretum communi consilio fuerit exsequatur; id muneris sustinebat episcopus in presbyterorum cœtu.”]

1
[T. C. i. 109. al. 83. “That he meaneth nothing less than to make any such difference between a bishop and a minister as is with us, . . . I will send you to Chrysostom upon 1 Tim. iii. where he saith, ‘The office of a bishop differeth little or nothing from an elder’s:’ and a little after, ‘That a bishop differeth nothing from an elder or minister but by the ordination only.’ ” Whitgift, Def. 387. “Chrysostom in that place maketh degrees in the ministry, and placeth the bishop in degree above the minister, which utterly overthroweth your equality.”]

2
Hieron. Ep. ad Evagr. [Evang.] 85. [al. 146. § 1. “Quid enim facit excepta ordinatione episcopus, quod presbyter non faciat?”]

3
Chrysost. Hom. x. [xi.] in 1 Tim. 3. [t. xi. p. 604. ed. Ben. Ἃ περὶ ἐπισκόπων εἰ̑πε, ταυ̑τα καὶ πρεσβυτέροις ἁρμόττει· τῃ̑ γὰρ χειροτονίᾳ μόνῃ ὑπερβεβήκασι, καὶ τούτῳ μόνον δοκου̑σι πλεονεκτει̑ν τοὺς πρεσβυτέρους.]

1
[Ep. ad Nepot. 2. al. 52. § 7.]

1
“Velut in aliqua sublimi specula constituti, vix dignantur videre mortales et alloqui conservos suos.” In 4. c. Epist. ad Gal. [v. 13. t. vii. 458.]

2
“Nemo peccantibus episcopis audet contradicere; nemo audet accusare majorem; propterea quasi sancti et beati et in præceptis Domini ambulantes augent peccata peccatis. Difficilis est accusatio in episcopum. Si enim peccaverit, non creditur; et si convictus fuerit, non punitur.” In cap. 8. Ecclesiast. v. 11. [iii. 454. The later editions of St. Jerome omit the first clause.]

3
“Pessimæ consuetudinis est, in quibusdam ecclesiis tacere presbyteros et præsentibus episcopis non loqui; quasi aut invideant aut non dignentur audire.” Ep. ad Nepotian. [52. § 7.]

4
Ep. 53. ad Ripar. [al. 109. § 2. i. 720.]

5
Hier. ad Nepot. [52. § 7. “Esto subjectus Pontifici tuo, et quasi animæ parentem suscipe. . . Illud etiam dico, quod episcopi, sacerdotes se esse noverint, non dominos; honorent clericos quasi clericos, ut et ipsis a clericis quasi episcopis honor deferatur. Scitum illud est oratoris Domitii, ‘Cur ego te, inquit, habeam ut principem, quum tu me non habeas ut senatorem?’ Quod Aaron et filios ejus, hoc esse episcopum et presbyteros noverimus. Unus Dominus, unum Templum; unum sit etiam ministerium.” i. 260.]

1
No bishop may be a lord in reference unto the presbyters which are under him, if we take that name in the worst part, as Jerome here doth. For a bishop is to rule his presbyters, not as lords do their slaves, but as fathers do their children.

2
[§ 9. “Ecclesiæ salus in summi sacerdotis dignitate pendet; cui si non exsors quædam et ab omnibus eminens detur potestas, tot in ecclesiis efficientur schismata, quot sacerdotes.” ii. 182.]

3
In Vita Chrys. per Cassiod. Sen. [in Hist. Eccles. Tripart. (a Latin compilation from Socrates, Sozomen, and Theodoret, made or translated at the instance of Cassiodorus (470-565) by his friend Epiphanius Scholasticus) l. x. c. 18.] 1886.

1
Pallad. (367-431.) in Vita Chrys. [c. 9. t. xiii. p. 29 E. ed. Bened. ἐμηνύθησαν δύο πρεσβύτεροι του̑ Ἰωάννου, . . . λέγοντες, “ἐδήλωσέν σοι ἡ σύνοδος· ‘πέρασον πρὸς ἡμα̑ς, ἀπολογησόμενος τὰ ἐγκλήματα.’ ” πρὸς ταυ̑τα ὁ Ἰωάννης ἀντεδήλωσεν δι’ ἐπισκόπων ἑτέρων· “ποίᾳ ἀκολουθίᾳ δικάζετε, οἱ μήτε τοὺς ἐχθρούς μου ἐξεώσαντες, καὶ διὰ τω̑ν ἐμω̑ν κληρικω̑ν μεταστελλόμενοι;”]

2
Ὥσπερ σύμπονοι δοθέντες τῳ̑ ἐπισκόπῳ. Zonaras, 1029-1118. in Can. Apost. [Can. 58. ap. Beveridge, Synod. i. 38. καὶ τὸ ὄνομα δὲ του̑ ἐπισκόπου εἰς νη̑ψιν αὐτὸν διεγείρει· σκοπὸς γὰρ ὠνόμασται· τὸν δὲ σκοπὸν ἐγρηγορέναι δει̑, ἀλλ’ οὐ ῥαθυμει̑ν· διὰ του̑το τοι̑ς ἐπισκόποις ἐν τῳ̑ θυσιαστηρίῳ καθέδρα ἐϕ’ ὕψους ἵδρυται, δηλου̑ντος του̑ πράγματος, οἱ̑ον εἰ̑ναι του̑τον, καὶ ὅτι δει̑ τὸν ὑπ’ αὐτὸν λαὸν ὁρᾳ̑ν ἀϕ’ ὕψους, καὶ ἐπισκοπει̑ν ἀκριβέστερον· καὶ οἱ πρεσβύτεροι συνιστάναι ἐκει̑ τῳ̑ ἐπισκόπῳ καὶ συγκαθη̑σθαι ἐτάχθησαν, ἵνα καὶ οὑ̑τοι διὰ τη̑ς ἀϕ’ ὕψους καθέδρας ἐνάγωνται εἰς τὸ ἐϕορᾳ̑ν τὸν λαὸν, καὶ καταρτίζειν αὐτὸν, ὥσπερ σύμπονοι δοθέντες τῳ̑ ἐπισκόπῳ.]

3
[Cap. 7. τί δὲ πρεσβυτερίον, ἀλλ’ ἢ σύστημα ἱερὸν, σύμβουλοι καὶ συνεδρευταὶ του̑ ἐπισκόπου; . . . ὁ τοίνυν τούτων παρακούων, ἄθεος πάμπαν εἴη ἂν, καὶ δυσσεβής, καὶ ἀθετω̑ν Χριστὸν, καὶ τὴν αὐτου̑ διάταξιν σμικρύνων. ap. Coteler. ii. 63.]

1
“Cum episcopo presbyteri sacerdotali honore conjuncti.” Ep. 28. [qu. 68? p. 118. ed. Baluzii. “Nec hoc in episcoporum tantum et sacerdotum, sed et in diaconorum ordinationibus observasse apostolos animadvertimus . . . nequis ad altaris ministerium vel ad sacerdotalem locum indignus obreperet.”] “Ego et compresbyteri nostri qui nobis adsidebant.” Ep. 27. [66. p. 114.]

2
[It should be “Cornelius unto Cyprian.”]

3
[Ep. 46. p. 60. ed. Baluz. “Posteaquam Urbanus et Sidonius confessores ad compresbyteros nostros venerunt, affirmantes Maximum confessorem et presbyterum secum pariter cupere in ecclesiam redire, . . . . ex ipsorum ore et confessione ista quæ per legationem mandaverant placuit audiri. Qui cum venissent, et a presbyteris quæ gesserant exigerentur. . . . circumventos se esse affirmaverunt, . . . . qui cum hæc et cætera eis fuissent exprobrata, ut abolerentur et de memoria tollerentur deprecati sunt. Omni igitur actu ad me perlato, placuit contrahi presbyterium. Adfuerunt etiam episcopi quinque, . . . ut firmato consilio quid circa personam eorum observari deberet consensu omnium statueretur.”]

1
Cypr. Ep. 93. [5. p. 11. “Ad id quod scripserunt mihi compresbyteri nostri Donatus et Fortunatus, Novatus, et Gordius, solus rescribere nihil potui, quando a primordio episcopatus mei statuerim nihil sine consilio vestro et sine consensu plebis mea privatim sententia gerere.”]

2
[Ibid. “Sed cum ad vos per Dei gratiam venero, tunc de iis quæ vel gesta sunt vel gerenda, sicut honor mutuus poscit, in commune tractabimus.”]

1
Cypr. Ep. [65. al. 3. c. 1. vid. supr. c. vi. § 7. p. 172, note 2.]

2
Such a one was that Peter whom Cassiodore writing the life of Chrysostom doth call the archpresbyter of the church of Alexandria under Theophilus at that time bishop. [Hist. Eccles. Tripartit. lib. x. cap. 10.]

3
Psalm cii. 13, 14.

1
L. 36. C. de Episc. et Cler. [Cod. Just. i. 3. de Episc. et Cler. 36. p. 35. ed. Gothofr. 1688. Hooker gives almost verbatim the Greek version of Photius, (†890.) Nomocanon, p. 85. ed. Paris. 1620.] Ἑκάστη πόλις ἴδιον ἐπίσκοπον ἐχέτω· καὶ κᾂν διὰ θείας ἀντιγραϕη̑ς τολμήσῃ τις ἀϕελέσθαι πόλιν του̑ ἰδίου ἐπισκόπου ἢ τη̑ς περιοικίδος αὐτη̑ς ἢ τινὸς ἄλλου δικαίου, γυμνου̑ται τω̑ν ὄντων καὶ ἀτιμου̑ται. Ἐξῄρηται δὲ ἡ Τομέων Σκυθίας πόλις. Ὁ γὰρ ἐπίσκοπος αὐτη̑ς καὶ τω̑ν λοιπω̑ν προνοει̑. Καὶ ἡ Λεοντόπολις Ἰσαυρίας ὑπὸ τὸν ἐπίσκοπόν ἐστιν Ἰσαυροπόλεως. Besides, Cypr. Ep. 52. [p. 73. ed. Baluz. al. 55. c. 14.] “Cum jampridem per omnes provincias et per urbes singulas ordinati sunt episcopi.”

2
“Ubi ecclesiastici ordinis non est consessus, et offert et tingit sacerdos qui est ibi solus.” Tertull. Exhort. ad Castit. [c. 7.]

1
Cypr. Ep. 25. [40. ed. Baluz. p. 53. “Cum semel placuerit tam nobis quam confessoribus et clericis urbicis, item universis episcopis vel in nostra provincia vel trans mare constitutis,” &c.]

2
Hieron. advers. Lucifer. [§ 9. “Non quidem abnuo hanc ecclesiarum esse consuetudinem, ut ad eos qui longe in minoribus urbibus per presbyteros et diaconos baptizati sunt, episcopus ad invocationem Sancti Spiritus manum impositurus excurrat.”]

3
Cypr. Ep. 49. [al. 52. c. 1. “Didicimus, atque docere et instruere cæteros cœpimus, Evaristum de episcopo, jam nec laicum remansisse, cathedræ et plebis extorrem, et de ecclesia Christi exsulem.” p. 63. ed. Baluz.]

4
[So Keble. Johnson quotes Whitgift and Raleigh for this form. So Gauden’s text 1676, but in ed. 1682 it is diocese.] 1886.

5
[Ἕκαστον ἐπίσκοπον ἐξουσίαν ἔχειν τη̑ς ἑαυτου̑ παροικίας, διοικει̑ν τε κατὰ τὴν ἑκάστῳ ἐπιβάλλουσαν εὐλάβειαν καὶ πρόνοιαν ποιει̑σθαι πάσης τη̑ς χω̑ρας τη̑ς ὑπὸ τὴν ἑαυτου̑ πόλιν, ὣς καὶ χειροτονει̑ν πρεσβυτέρους καὶ διακόνους, καὶ μετὰ κρίσεως ἕκαστα διαλαμβάνειν· περαιτέρω δὲ μηδὲν πράττειν ἐπιχειρει̑ν δίχα του̑ τη̑ς μητροπόλεως ἐπισκόπου, μηδὲ αὐτὸν ἄνευ τη̑ς τω̑ν λοιπω̑ν γνώμης.] Conc. Antioch. cap. 9. [ad 341. t. i. 597. ed. Harduin.] Ἀκλήτους δὲ ἐπισκόπους ὑπὲρ διοίκησιν μὴ ἐπιβαίνειν, ἐπὶ χειροτονίᾳ ἤ τισιν ἄλλαις οἰκονομίαις ἐκκλησιαστικαι̑ς. Conc. Const. can. 2. [ad 381. t. i. 809.] Του̑το γὰρ πρότερον διὰ τοὺς διωγμοὺς ἐγίνετο ἀδιαϕόρως. Socr. lib. v. cap. 8.

1
“As I have ordained in the churches of Galatia, the same do ye also.” 1 Cor. xvi. 1.

2
2 Cor. xi. 28.

3
Chrys. in i. ad Tit. [εἰ δὲ ὁ μίαν ψυχὴν σκανδαλίζων, συμϕέρει αὐτῳ̑ ἵνα μύλος ὀνικὸς κρεμασθῃ̑ εἰς τὸν τράχηλον αὐτου̑, καὶ καταποντισθῃ̑ ἐν τῳ̑ πελάγει τη̑ς θαλάσσης. ὁ τὰς τοσαύτας ψυχὰς σκανδαλίζων, πόλεις ὁλοκλήρους, καὶ δήμους, καὶ μυρίας ψυχὰς, ἄνδρας, γυναι̑κας, παι̑δας, πολίτας, γεωργοὺς, τοὺς ἐν αὐτῃ̑ τῃ̑ πόλει, τοὺς ἐν ἑτέραις ται̑ς ὑπ’ ἐκείνην τὴν πόλιν, τί ὑποστήσεται;]

4
Pallad. in Vita Chrys. [c. 7.] καὶ τὸν μὲν χειροτονει̑ ἐπίσκοπον, ἐγκατατάξας κωμυδρίῳ, πόλιν οὐκ ἔχων· ἀδεω̑ς γὰρ καὶ τὰς καινοτομίας εἰργάζετο, ἄλλον ἑαυτὸν Μωσέα ὀνομάζων. ap. Chrys. ed. Bened. t. xiii. 22 F.]

a
bishops, Keble; so Gauden 1676.

b
vicegerent, Keble; and so 1676.

1
Concil. Antioch. ad 341. can. 10. [τοὺς ἐν ται̑ς κώμαις, ἢ ται̑ς χώραις, ἢ τοὺς καλουμένους χωρεπισκόπους, εἰ καὶ χειροθεσίαν εἰ̑εν ἐπισκόπων εἰληϕότες, ἔδοξε τῃ̑ ἁγίᾳ συνόδῳ εἰδέναι τὰ ἑαυτω̑ν μέτρα, καὶ διοικει̑ν τὰς ὑποκειμένας ἑαυτοι̑ς ἐκκλησίας, καὶ τῃ̑ τούτων ἀρκει̑σθαι ϕροντίδι καὶ κηδεμονίᾳ, καθιστα̑ν δὲ ἀναγνώστας, καὶ ὑποδιακόνους, καὶ ἐϕορκιστὰς, καὶ τῃ̑ τούτων ἀρκει̑σθαι προαγωγῃ̑· μήτε πρεσβύτερον, μήτε διάκονον χειροτονει̑ν τολμᾳ̑ν, δίχα του̑ ἐν τῃ̑ πόλει ἐπισκόπου, ᾑ̑ ὑπόκεινται αὐτός τε καὶ ἡ χώρα· εἰ δὲ τολμήσειέ τις παραβη̑ναι τὰ ὁρισθέντα, καθαιρει̑σθαι αὐτὸν καὶ ἣς μετέχει τιμη̑ς· χωρεπίσκοπον δὲ γίνεσθαι ὑπὸ του̑ τη̑ς πόλεως, ᾑ̑ ὑπόκειται, ἐπισκόπου. t. i. p. 597. ed. Harduin. ad 341.]

1
[T. C. i. 85. apud Whitgift. Def. 392. al. 110. “If any man will call this a rule or presidentship, and him that executeth the office a president or moderator, or a governor, we will not strive, so that it be with these cautions, that he be not called simply governor or moderator, but governor or moderator of that action and for that time, and subject to the orders that others be, and to be censured by the company of the brethren as well as others, if he be judged any way faulty. And that after that action ended and meeting dissolved, he sit him down in his old place, and set himself in equal estate with the rest of the ministers. Thirdly, that this government or presidentship, or whatsoever like name you will give it, be not so tied unto that minister, but that at the next meeting it shall be lawful to take another if another be thought meeter.”]

c
So ed. 1676, 1682.

1
[Of Archbishops, see Admon. ap. Whitg. Def. 298; Answ. ibid. al. 95-103; T. C. i. 61. al. 82; Def. 297, &c.; T. C. ii. 453-514.]

2
“Si quid habebis cum aliquo Hellespontio controversiæ, ut in illam διοίκησιν rejicias.” Cic. Fam. Ep. 53. lib. xiii. The suit which Tully maketh was this, that the party in whose behalf he wrote to the proprætor, might have his causes put over to that court which was held in the diocess of Hellespont, where the man did abide, and not to his trouble be forced to follow them at Ephesus, which was the chiefest court in that province.

d
mo, ed. 1676; more, 1682.

1
[Notit. Imp. Orient. per Pancirollum, p. 78. ed. 1593.]

2
[According to the Notitia, p. 153, Africa had but five provinces; according to Sextus Rufus, six; ap. Gruter. Script. Hist. Rom. p. 1194.]

3
[Lib. xlv. c. 29.]

4
[Theodoret. E. H. v. 17; Cod. Theodos. xi. tit. i. 33.]

5
Cic. ad Attic. lib. v. ep. 13. Item, l. Observ. D. de Officio Proconsulis et Legati. [“Imperator noster Antoninus Augustus ad desideria Asianorum rescripsit, proconsuli necessitatem impositam per mare Asiam applicare, καὶ τω̑ν μητροπολέων Ἔϕεσον, i.e. inter matrices urbes Ephesum primam attingere.” ap. Gothofred. Corp. Jur. Civ. p. 28. ed. 1688.]

6
“Sancimus . . . ut sicut Oriens atque Illyricum, ita et Africa prætoriana maxima potestate specialiter a nostra clementia decoretur. Cujus sedem jubemus esse Carthaginem . . . et ab ea, auxiliante Deo, septem provinciæ cum suis judicibus disponantur.” Lib. i. tit. 27. l. i. sect. 1, 2. [Cod. Justinian. p. 100. ed. Gothofr. 1688.]

e
So ed. 1676, 1682.

1
Psalm lxxx. 8, 9.

2
Concil. Antiochen. can. 9. Τοὺς καθ’ ἑκάστην ἐπαρχίαν ἐπισκόπους εἰδέναι χρὴ τὸν ἐν τῃ̑ μητροπόλει προεστω̑τα ἐπίσκοπον καὶ τὴν ϕροντίδα ἀναδέχεσθαι πάσης τη̑ς ἐπαρχίας, διὰ τὸ ἐν τῃ̑ μητροπόλει πανταχόθεν συντρέχειν πάντας τοὺς τὰ πράγματα ἔχοντας, ὅθεν ἔδοξε καὶ τῃ̑ τιμῃ̑ προηγει̑σθαι αὐτόν. [t. i. 595. ed. Harduin. ad 341.]

1
Vilierius de Statu primitivæ Ecclesiæ. [“Hæc quidem Ecclesiæ Christianæ instituta adusque cccc amplius xxx annos integra atque inviolata permanserunt . . . . At paucis post annis, Constantinopolitanus Episcopus ambitione et cupiditate regnandi ac census ausus est præclaram illam Ecclesiæ descriptionem et œconomiam convellere. Cum enim imperatores sedem imperii sui, senatumque in ea civitate constituissent, ille artibus suis perfecit, ut ea . . . dignitatem quoque et potestatem aliquam præter cæteras metropoles eximiam ac perpetuam obtineret. Itaque quod Constantinopolitani primi cap. 2°. constitutum erat, ut Asiæ, Ponti, et Thraciæ metropolitæ, suæ quisque provinciæ procurationem gererent, . . . proximo universali concilio, i. e. Chalcedonensi, funditus abrogatum est, et novo more, nullo exemplo constitutum, ut harum omnium provinciarum metropolitas solus Constantinopolitanus episcopus constitueret: qua lege . . . nemo non videt . . . æquabilitatem provinciarum, quæ a majoribus conservata ac tradita fuerat, turpissime confusam ac perturbatam.” fol. 143. ad calcem Reg. Poli, Def. Eccl. Unit.; Argentorat. 1555. The tract was written in reality by François Hotman, the distinguished French protestant lawyer, and was first printed at Geneva, 1553: Hotman being then Professor of Law at Strasburg. Vid. Gesneri Biblioth. as epitomized by Simler, Zurich, 1574. p. 202; et Biogr. Univ. art. Hotman.]

2
[I. e. the council of Chalcedon, ad 451; in its 28th canon, cited below.]

1
[Can. vi. Τὰ ἀρχαι̑α ἔθη κρατείτω, τὰ ἐν Αἰγύπτῳ καὶ Λιβύῃ καὶ Πενταπόλει, ὥστε τὸν Ἀλεξανδρείας ἐπίσκοπον πάντων τούτων ἔχειν τὴν ἐξουσίαν, ἐπειδὴ καὶ τῳ̑ ἐν τῃ̑ Ῥώμῃ ἐπισκόπῳ του̑το σύνηθές ἐστιν. ὁμοίως δὲ καὶ κατὰ τὴν Ἀντιόχειαν, καὶ ἐν ται̑ς ἄλλαις ἐπαρχίαις, τὰ πρεσβει̑α σώζεσθαι ται̑ς ἐκκλησίαις. Conc. Harduin. i. 325.]

2
Socr. lib. v. c. 8.

3
[ad 381. Can. ii. Κατὰ τοὺς κανόνας, τὸν μὲν Ἀλεξανδρείας ἐπίσκοπον τὰ ἐν Αἰγύπτῳ μόνον οἰκονομει̑ν· τοὺς δὲ τη̑ς ἀνατολη̑ς ἐπισκόπους τὴν ἀνατολὴν μόνην διοικει̑ν, ϕυλαττομένων τω̑ν ἐν τοι̑ς κανόσι τοι̑ς κατὰ Νικαίαν πρεσβείων τῃ̑ Ἀντοιχέων ἐκκλησίᾳ. And Can. iii. Τὸν μέντοι Κωνσταντινουπόλεως ἐπίσκοπον ἔχειν τὰ πρεσβει̑α τη̑ς τιμη̑ς μετὰ τὸν τη̑ς Ῥώμης ἐπίσκοπον, διὰ τὸ εἰ̑ναι αὐτὴν νέαν Ῥώμην. Conc. i. 809.]

4
Can. 28. [ad 451. Τὰ αὐτὰ καὶ ἡμει̑ς ὁρίζομεν καὶ ψηϕιζόμεθα περὶ τω̑ν πρεσβείων τη̑ς ἁγιωτάτης ἐκκλησίας Κωνσταντινουπόλεως, ψέας Ῥώμης· καὶ γὰρ τῳ̑ θρόνῳ τη̑ς πρεσβυτέρας Ῥώμης, διὰ τὸ βασιλεύειν τὴν πόλιν ἐκείνην, οἱ πατέρες εἰκότως ἀποδεδώκασι τὰ πρεσβει̑α, καὶ τῳ̑ αὐτῳ̑ σκόπῳ κινουμένοι οἱ ρν́ θεοϕιλέστατοι ἐπίσκοποι, τὰ ἰ̑σα πρεσβει̑α ἀπένειμαν τῳ̑ τη̑ς νέας Ῥώμης ἁγιωτάτῳ θρόνῳ, εὐλόγως κρίναντες, τὴν βασιλείᾳ καὶ συγκλήτῳ τιμηθει̑σαν πόλιν, καὶ τω̑ν ἴσων ἀπολαύουσαν πρεσβείων τῃ̑ πρεσβυτέρᾳ βασιλίδι Ῥώμῃ, καὶ ἐν τοι̑ς ἐκκλησιαστικοι̑ς ὡς ἐκείνην μεγαλύνεσθαι πράγμασιν, δευτέραν μετ’ ἐκείνην ὑπάρχουσαν. Ibid. ii. 612.]

1
Can. 36. [There is an historical oversight here. The council meant is that called Quinisextum, or “in Trullo,” ad 706; of which the 36th canon appeals to the 630 bishops assembled at Chalcedon. Ἀνανεούμενοι τὰ παρὰ τω̑ν ρν́ ἁγίων πατἐρων τω̑ν ἐν τῃ̑ θεοϕυλάκτῳ ταύτῃ καὶ βασιλίδι πόλει συνελθόντων, καὶ τω̑ν χλ́. τω̑ν ἐν Χαλκήδονι συνελθόντων νομοθετηθέντα, ὁρίζομεν, ὥστε τὸν Κωνσταντινουπόλεως θρόνον τω̑ν ἴσων ἀπολαύειν πρεσβείων του̑ τη̑ς πρεσβυτέρας Ῥώμης θρόνου, καὶ ἐν τοι̑ς ἐκκλησιαστικοι̑ς ὡς ἐκει̑νον μεγαλύνεσθαι πράγμασι, δεύτερον μετ’ ἐκει̑νον ὑπάρχοντα· μεθ’ ὃν ὁ τη̑ς Ἀλεξανδρέων μεγαλοπόλεως ἀριθμείσθω θρόνος· εἰ̑τα ὁ τη̑ς Ἀντιοχέων· καὶ μετὰ του̑τον ὁ τη̑ς Ἱεροσολυμιτω̑ν πόλεως. Ibid. iii. 1676.]

2
[E. g. of Theodosius ii. xvi. Cod. Theodos. tit. ii. l. 45. ad 421. “Omni innovatione cessante, vetustatem et canones pristinos ecclesiasticos qui nunc usque tenuerunt, per omnes Illyrici provincias servari præcipimus: tum, si quid dubietatis emerserit, id oporteat non absque scientia viri reverendissimi sacrosanctæ legis antistitis urbis Constantinopolitanæ (quæ Romæ veteris prærogativa lætatur) conventui sacerdotali sanctoque judicio reservari.” t. vi. 89. ed. Gothofred. And of Justinian, Novell. cxxxi. c. 1, 2. ad 541. “Sancimus vicem legum obtinere sanctas ecclesiasticas regulas, quæ a sanctis quatuor conciliis expositæ sunt, aut firmatæ . . . Ideoque sancimus secundum earum definitiones sanctissimum senioris Romæ Papam primum esse omnium sacerdotum; beatissimum autem Archiepiscopum Constantinopoleos novæ Romæ secundum habere locum post sanctam apostolicam senioris Romæ sedem; aliis autem omnibus sedibus præponatur.” p. 275. ed. Gothofr. 1688.]

3
Novell. cxxiii. 22. [“Si quis vero sanctissimorum episcoporum ejusdem synodi dubitationem aliquam ad invicem habeat, sive pro ecclesiastico jure, sive pro aliis quibusdam rebus, prius metropolita eorum cum aliis de sua synodo episcopis causam examinet et judicet; et si non rata habuerit utraque pars ea quæ judicata sunt, tunc beatissimus Patriarcha diœceseos illius inter eos audiat, et illa determinet, quæ ecclesiasticis canonibus et regulis consonant, nulla parte ejus sententiæ contradicere valente. Si autem et a clero, aut alio quocunque aditio contra episcopum fiat propter quamlibet causam; apud sanctissimum ejus metropolitam secundum sacras regulas et nostras leges causa judicetur; et siquis judicatis contradixerit, ad beatissimum archiepiscopum et patriarcham diœceseos illius referatur causa, et ille secundum canones et leges huic præbeat finem. Si vero contra metropolitam talis aditio fiat ab episcopo aut clero, aut alia quacunque persona, diœceseos illius beatissimus patriarcha simili modo causam judicet.” p. 259. ed. Gothofr. 1688. ad 541.]

1
Conc. Nic. c. 6. [t. i. 325. ed. Harduin. vid. supr. § 9. p. 193, note 1.]

1
Ejusd. Conc. c. 7. [ἐπειδὴ συνήθεια κεκράτηκε καὶ παράδοσις ἀρχαία, ὥστε τὸν ἐν Αἰλίᾳ ἐπίσκοπον τιμα̑σθαι, ἐχέτω τὴν ἀκολουθίαν τη̑ς τιμη̑ς, τῃ̑ μητροπόλει σωζομένου του̑ οἰκείου ἀξιώματος. It appears that Hooker’s copy placed the comma after μητροπόλει.]

2
[Vide Sarav. de Divers. Ministr. Evang. Grad. c. 20.]

3
“What! no mention of him in Theophilus bishop of Antioch? none in Clemens Alexandrinus? none in Ignatius? none in Justin Martyr? in Irenæus, in Tertullian, in Origen, in Cyprian? in those old historiographers, out of which Eusebius gathered his story? Was it for his baseness and smallness that he could not be seen amongst the bishops, elders, and deacons, being the chief and principal of them all? Can the cedar of Lebanon be hidden amongst the box-trees?” T. C. lib. i. 92. [al. 70.]

1
T. C. lib. i. ubi supra. “A metropolitan bishop was nothing else but a bishop of that place which it pleased the emperor or magistrate to make the chief of the diocess or shire; and as for this name, it makes no more difference between a bishop and a bishop, than when I say a minister of London and a minister of Newington.”

f
more, so in edd. 1676, 1682.

2
[“Girdler, ‘a maker of girdles.’ ‘Talk with the girdler, or with the milliner.’ Beaum. and Fletcher, Honest Man’s Fortune.” Todd’s Johnson’s Dict.]

3
Conc. Nicen. c. 6. “Illud autem omnino manifestum, quod siquis absque metropolitani sententia factus sit episcopus, hunc magna synodus definivit episcopum esse non oportere.” [καθόλου δὲ πρόδηλον ἐκει̑νο, ὅτι εἴτις χωρὶς γνώμης του̑ μητροπολίτου γένοιτο ἐπίσκοπος, τὸν τοιου̑τον ἡ μεγαλὴ σύνοδος ὥρισε μὴ δει̑ν εἰ̑ναι ἐπίσκοπον.] Can. 4. [ἐπίσκοπον προσήκει μάλιστα μὲν ὑπὸ πάντων τω̑ν ἐν τῃ̑ ἐπαρχίᾳ καθίστασθαι· εἰ δὲ δυσχερὲς εἴη τὸ τοιου̑το, ἢ διὰ κατεπείγουσαν ἀνάγκην, ἢ διὰ μη̑κος ὁδου̑, ἐξ ἅπαντος τρει̑ς ἐπὶ τὸ αὐτὸ συναγομένους, συμψήϕων γινομένων καὶ τω̑ν ἀπόντων, καὶ συντιθεμένων διὰ γραμμάτων, τότε τὴν χειροτονίαν ποιει̑σθαι· τὸ δὲ κυ̑ρος τω̑ν γινομένων δίδοσθαι καθ’ ἑκάστην ἐπαρχίαν τῳ̑ μητροπολίτῃ. t. i. 324. ed. Harduin.]

1
Novell. cxxiii. can. 10. [“Ut omnis ecclesiasticus status et sacræ regulæ diligenter custodiantur; jubemus unumquemque beatum archiepiscopum et patriarcham et metropolitam sanctissimos episcopos sub se constitutos in eadem provincia semel aut secundo per singulos annos ad se convocare, et omnes causas subtiliter examinare, quas episcopi aut clerici aut monachi ad invicem habeant.” p. 255.]

2
Novell. cxxiii. cap. 9. [“Interdicimus Deo amabilibus episcopis proprias relinquere ecclesias, et ad alias regiones venire. Si vero necessitas faciendi hoc contigerit, non aliter, nisi cum literis beatissimi Patriarchæ aut Metropolitæ, aut per imperialem videlicet jussionem hoc faciant.” ibid.]

3
Novell. lxxix. cap. 2. [“Imp. Just. Aug. Mennæ Archiep. Constantinop. . . . Tua celsitudo . . . . utatur ad Deo amabiles civitatum metropolitanos (quorum ipse suscepisti ordinationem) proponens propriis literis hanc nostram sacram legem. Verum illi sub se constitutis episcopis hæc nuncient, ut ex paucis literis una continuatio legis ad omnem perveniat ditionem.” p. 165.]

4
Novell. cxxiii. cap. 22. [vid. supr. § 10. p. 194, note 3.]

5
Novell. cxxiii. cap. 23. [“Œconomos autem et xenodochos, nosocomos, ptochotrophos, et aliorum venerabilium locorum gubernatores, et alios omnes clericos jubemus pro creditis sibi gubernationibus apud proprium episcopum, cui subjacent, conveniri, et rationem suæ gubernationis facere et exigi. . . Si vero putaverint se gravari, post repetitionem metropolita causam examinet. Si vero metropolita . . . debitum exegerit, et exactus putaverit se gravatum, diœceseos illius beatissimus patriarcha causam examinet.” p. 259.]

1
Can. 9. [τοὺς καθ’ ἑκάστην ἐπαρχίαν ἐπισκόπους εἰδέναι χρὴ, τὸν ἐν τῃ̑ μητροπόλει προεστω̑τα ἐπίσκοπον καὶ τὴν ϕροντίδα ἀναδέχεσθαι πάσης τη̑ς ἐπαρχίας . . . ὅθεν ἔδοξε . . . μηδὲν πράττειν περιττὸν τοὺς λοιποὺς ἐπισκόπους ἄνευ αὐτου̑, κατὰ τὸν ἀρχαι̑ον κρατήσαντα τω̑ν πατέρων ἡμω̑ν κανόνα, ἢ ταυ̑τα μόνα, ὅσα τῃ̑ ἑκάστου ἐπιβάλλει παροικίᾳ καὶ ται̑ς ὑπ’ αὐτὴν χώραις. i. 595. ed. Hard.]

2
Can. 16. [τελείαν δὲ ἐκείνην εἰ̑ναι σύνοδον, ᾑ̑ συμπάρεστι καὶ ὁ τη̑ς μητροπόλεως. i. 599.]

3
Can. 4. τὸ κυ̑ρος τω̑ν γινομένων. [i. 324.]

4
Can. 23. [28. ad 397. 3. Concil. Carthag. “Placuit, ut Episcopi transmare non proficiscantur nisi consulto primæ sedis Episcopo, sive cujusque provinciæ primate, ut ab eo præcipue possint sumere formatam, sive commendationem.” t. i. 964.]

5
Can. 34. [33. τοὺς ἐπισκόπους ἑκάστου ἔθνους εἰδέναι χρὴ τὸν ἐν αὐτοι̑ς πρω̑τον, καὶ ἡγει̑σθαι αὐτὸν ὡς κεϕαλὴν, καὶ μηδέν τι πράττειν περιττὸν ἄνευ τη̑ς ἐκείνου γνώμης. Conc. Harduin. i. 17.]

1
Cassiod. in Vita Chrysost. [Hist. Eccles. Tripart. lib. x. c. 4. from Theodoret. H. E. v. 18.]

2
Hieron. Ep. 9. [al. lib. contr. Joan. Hierosolym. § 37. t. ii. 447. ed. Vallarsii.]

3
Aug. de Hær. ad Quodvultdeum. [t. viii. 18. Hær. 53.] “Aëriani ab Aërio quodam sunt, qui quum esset presbyter, doluisse fertur, quod episcopus non potest ordinari; [et in Arianorum hæresin lapsus, propria quoque dogmata addidisse nonnulla, dicens, offerri pro dormientibus non oportere, nec statuta solenniter celebranda esse jejunia, sed cum quisque voluerit jejunandum, ne videatur esse sub lege.] Dicebat etiam presbyterum ab episcopo nulla differentia debere discerni.”

1
[Epiphan. Hæres. 75. c. 3. ἠ̑ν δὲ αὐτου̑ ὁ λόγος μανιώδης μα̑λλον, ἤπερ καταστασέως ἀνθρωπίνης, καί, ϕησι, τί ἐστιν ἐπίσκοπος πρὸς πρεσβύτερον; οὐδὲν διαλλάττει οὑ̑τος τούτου· μία γάρ ἐστι τάξις, καὶ μία, ϕησὶ, τιμὴ, καὶ ἓν ἀξίωμα· χειροθετει̑, ϕησὶν, ἐπίσκοπος, ἀλλὰ καὶ ὁ πρεσβύτερος· λουτρὸν δίδωσιν ὁ ἐπίσκοπος, ὁμοίως καὶ ὁ πρεσβύτερος· τὴν οἰκονομίαν τη̑ς λατρείας ποιει̑ ὁ ἐπίσκοπος, καὶ ὁ πρεσβύτερος ὡσαύτως· καθέζεται ὁ ἐπίσκοπος ἐπὶ του̑ θρόνου, καθέζεται καὶ ὁ πρεσβύτερος.]

1
Ἐν τούτῳ πολλοὺς ἠπάτησε. [Hær. 75. § 3.]

2
[Hær. 75. § 5.]

3
As in that he saith, the Apostle doth name sometime presbyters and not bishops, 1 Tim. iv. 14. sometime bishops and not presbyters, Phil. i. 1. because all churches had not both, for want of able and sufficient men. In such churches therefore as had but the one, the Apostle could not mention the other. Which answer is nothing to the latter place abovementioned: for that the church of Philippi should have more bishops than one, and want a few able men to be presbyters under the regiment of one bishop, how shall we think it probable or likely?

4
1 Tim. iv. 14. “With the imposition of the presbytery’s hand.” Of which presbytery St. Paul was chief, 2 Tim. i. 6. And I think no man will deny that St. Paul had more than a simple presbyter’s authority.

1
Phil. i. 1. “To all the saints at Philippi, with the bishops and deacons.” For as yet in the church of Philippi, there was no one which had authority besides the Apostles, but their presbyters or bishops were all both in title and in power equal.

1
Titus i. 5; 1 Tim. iii. 5; Phil. i. 1; 1 Pet. v. 1, 2. [See this argument urged, T.C. i. 79. al. 103. ii. 515, &c. Comp. Calvin, Instit. iv. 3, 8.]

1
[Marsilius of Padua, [a Franciscan canonist, who defended the claims of the Emperor, Louis of Bavaria, against Pope John XXII. † 1328.] Def. Pacis, pars ii. c. xvi. (vid. infra, § 8. note.) “Ostendemus, primum, Apostolorum neminem ad alios habuisse præeminentiam in essentiali dignitate sacerdotali. . . Ex quibus etiam per necessitatem deducemus, episcoporum sibi successorum neminem singulariter auctoritatem seu potestatem aliquam. . . in reliquos sibi coepiscopos seu compresbyteros habere.” p. 241.]

2
[Can. 6, 7.]

3
[Vid. supr. c. viii. 12. p. 199, note 1.]

4
T. C. lib. i. p. 62, [al. 83. Whitgift’s Defence, 303.] “So that it appeareth that the ministry of the Gospel, and the functions thereof ought to be from heaven: from heaven, I say, and heavenly, because although it be executed by earthly men, and ministers are chosen also by men like unto themselves, yet because it is done by the word and institution of God, it may well be accounted to come from heaven and from God.”

1
Acts i. 20.

2
Rev. ii. 1.

1
1 Tim. v. 19.

2
Tit. i. 5.

1
They of Walden. Æn. Syl. [Æneas Sylvius Piccolomini, Pius II. 1458-1463] Hist. Bohem. [c. 35. (speaking of John Huss and his partisans in the university of Prague); “Proruperunt in blasphemias, et cum aliquibus ignavis fortasse ac vitiosis maledicere possent, in omnes latrare sacerdotes cœpere; et ab ecclesia Catholica recedentes, impiam Valdensium sectam atque insaniam amplexi sunt. Hujus pestiferæ et jampridem damnatæ factionis dogmata sunt, Romanum præsulem reliquis episcopis parem esse; inter sacerdotes nullum discrimen; presbyterum non dignitatem sed vitæ meritum efficere potiorem.” p. 141. ap. Dubravium, Rerum Bohem. Scriptores, Hanoviæ, 1602.] Marsilius Defens. Pac. [“Marsilii Menandrini Patavini, Defensor Pacis ad Imp. Ludovic. iv. adversus usurpatam Rom. Pontif. Jurisdict.” circ. ad 1324. [v. Fleury, H. E. l. 93. 19, 39.] Dictio seu Pars ii. c. xv. “Post Apostolorum tempora numero sacerdotum notabiliter aucto, ad scandalum et schisma evitandum elegerunt sacerdotes unum ex ipsis, qui alios dirigeret et ordinaret quantum ad ecclesiasticum officium et servitium exercendum, et oblata distribuendum, ac reliqua disponendum convenientiori modo. . . Hic ex posteriorum consuetudine retinuit sibi soli nomen episcopi. . . Jam dicta electio seu institutio per hominem sic electo nihil amplioris meriti seu sacerdotalis auctoritatis. . . . tribuit, sed solum ordinationis œconomicæ in domo Dei seu templo potestatem quandam, alios sacerdotes, diaconos, et officiales reliquos regulandi et ordinandi.” ap. Goldastum, Tract. de Monarch. S. Rom. Imp. t. ii. p. 240. Francof. 1621.] Nicl. [Wicl. ap.] Thom. Wald. [Thomas Netter, of Walden in Essex, † 1431, Provincial of the Carmelites: employed by Richard II, Henry IV, and Henry V.] c. 1. lib. ii. art. 3. c. 60. [quoting Wicl. Trialog. lib. i. c. 10. for the following sentiment; “Unum audacter assero; quod in primitiva ecclesia ut tempore Pauli suffecerunt duo ordines clericorum, scil. sacerdos atque diaconus. Secundo dico quod in tempore Apostoli idem fuit presbyter ac episcopus . . . . Tunc enim non fuit inventa distinctio papæ, et cardinalium, patriarcharum, et archiepiscoporum, episcoporum, et archidiaconorum, officialium, et decanorum, cum cæteris officiariis . . . Certum videtur quod superbia Cæsarea hos gradus et ordines adinvenit. Si enim fuissent necessarii ecclesiæ, Christus et ejus Apostoli non in expressione eorum ac detentione (sic) officii reticerent.” p. 326. Venet. 1571. In the edition of Wicliffe, 1525, the passage occurs lib. iv. c. 15. fol. 124.] Calvin. Com. in 1. ad Tit. [v. 7. “Locus hic abunde docet, nullum esse presbyteri et episcopi discrimen: quia nunc secundo nomine promiscue appellat quos prius vocavit presbyteros. . . Hinc perspicere licet, quanto plus delatum hominum placitis fuerit quam decebat, quia abrogato Sp. Sancti sermone, usus hominum arbitrio inductus prævaluit. Mihi quidem non displicet quod statim ab ecclesiæ primordiis receptum fuit ut singula episcoporum collegia unum aliquem moderatorem habeant: verum nomen officii quod Deus in commune omnibus dederat in unum solum transferri, reliquis spoliatis, et injurium est et absurdum. Deinde sic pervertere Sp. Sancti linguam, ut nobis eædem voces aliter quam voluerit significent, nimis profanæ audaciæ est.” p. 537. ed. Genev. 1600.] Bullinger, (1504-1575.) Decad. v. Serm. 3. [p. 296. Tigur. 1577. “Non ita multis post mortem Apostolorum sæculis visa est in ecclesia longe alia hierarchia quam fuerat ab initio. Quamvis principia illa videantur tolerabiliora fuisse quam sint hodie istius ordinis omnia. . . In qualibet urbe et regione præstantissimus quisque cæteris præponebatur. Ejus functio erat superintendere presbyteris et universo gregi. Non habebat . . . in collegas vel presbyteros dominium, sed sicut consul in senatu partes habet interrogandi colligendique suffragia, leges item ac jura tuendi, ac curandi ne subnascantur inter senatores factiones; ita non aliud in ecclesia episcopo officium fuit: per cætera, communia habuit cum sacerdotibus. Nisi vero longius processisset consequentibus temporibus sacerdotum audacia, et episcoporum ambitio, ne verbo quidem reclamaremus.”] Juel. Def. Apol. part. 2. c. 9. di. 1. [Harding, in the course of an argument for tradition, had remarked, that “they which denied the distinction of a bishop and a priest were condemned of heresy.” (p. 196.) Jewel replies, (p. 202,) “What meant M. Harding here to come in with the difference between priests and bishops? Thinketh he that priests and bishops hold only by tradition? Or is it so horrible an heresy as he maketh it, to say that by the Scriptures of God a bishop and a priest are all one? Or knoweth he how far, and unto whom, he reacheth the name of an heretic?” He then proceeds to quote S. Chrysostom, S. Jerome, &c. and concludes, “All these and other mo holy Fathers, together with S. Paul the apostle, for thus saying, by M. Harding’s advice, must be holden for heretics.” ed. 1609.] Fulk. Answ. to the Test. Tit. 1. 5. [The Rhemish note on this verse is, “Though priests or bishops may be nominated and elected by the princes, people, or patrons of places, . . . yet they cannot be ordered and consecrated but by a bishop who was himself rightly ordered or consecrated before, as this Titus was by St. Paul. . . . . That the ordering of priests or imposition of hands to that purpose belongeth only to bishops. . . . . is plain by the apostolic practice set down in the Scriptures, viz. in the Acts and in the Epistles to Timothy and Titus.” Fulke’s reply: “The people had their elections, moderated by the wisdom and gravity of the clergy, among whom, for order and seemly government, there was always one principal, to whom by long use of the church the name of bishop or superintendant hath been applied, which room Titus exercised in Crete, Timothy in Ephesus, and others in other places. Therefore although in the Scripture a bishop and an elder is of one authority in preaching of the word and administration of the sacraments, . . . yet in government by ancient use of speech he is only called a bishop, which is in the Scripture called προιστάμενος, προεστὼς, or ἡγούμενος, i. e. chief in government, to whom the ordination or consecration by imposition of hands was always principally committed. Not that imposition of hands belongeth only to him, for the rest of the elders that were present at ordination did lay on their hands, or else the bishop did lay on his hands in the name of the rest.” p. 718, 19. ed. 1633.]

1
John i. 25. [ap. T. C. i. 62. al. 83.]

2
Matt. xxi. 23. 25, 26.

1
Lib. i. [c. 14.]

2
Rom. i. 32.

3
Luke i. 6.

1
Confess. 169. [“Fides mea nititur cum primis et simpliciter verbo Dei, deinde nonnihil etiam communi totius veteris Catholicæ ecclesiæ consensu, si ille cum sacris literis non pugnet: credo enim quæ a piis Patribus in nomine Domini congregatis, communi omnium consensu, citra ullam sacrarum literarum contradictionem definita et recepta fuerunt, ea etiam (quanquam haud ejusdem cum sacris literis authoritatis) a Sp. Sancto esse. Hinc fit ut quæ sint ejusmodi, ea ego improbare nec velim nec audeam bona conscientia. Quid autem certius ex historiis, ex conciliis, et ex omnium Patrum scriptis, quam illos ministrorum ordines, de quibus diximus, communi totius Reip. Christianæ consensu in ecclesia constitutos receptosque fuisse? Quis autem ego sum, qui quod tota ecclesia approbavit improbem? Sed neque omnes nostri temporis docti viri improbare ausi sunt, quippe qui norunt et licuisse hæc ecclesiæ, et ex pietate atque ad optimos fines pro electorum ædificatione ea omnia fuisse perfecta et ordinata.” Quoted also by Bishop Cooper, Adm. 82, 83; by Saravia, De Divers. Min. Grad. c. 23; by Bancroft, Survey, &c. p. 108; and by Bridges, Def. of Gov. established, &c. p. 424. It was Zanchius’ deliberate opinion, in answer to an exception which Beza had taken to a clause in his (Zanchius’) draught of a Confession for theReformed Churches.]

2
Epist. [ad Reg. Polon. (Non. Decemb. 1554.) p.] 190. [“Vetus quidem ecclesia patriarchias instituit, et singulis etiam provinciis quosdam attribuit primatus, ut hoc concordiæ vinculo melius inter se devincti manerent episcopi.” ed. Gen. 1617 = p. 140. ed. Gen. 1576.]

1
Ep. 3, lib. i. [al. 59. c. 10. “Cum statutum sit ab omnibus nobis et æquum sit pariter ac justum ut uniuscujusque causa illic audiatur ubi est crimen admissum, . . . oportet . . . agere illic causam . . . ubi et accusatores habere et testes . . . possint.” p. 86. ed. Baluz.]

1
“The bishop which Cyprian speaketh of, is nothing else but such as we call pastor, or as the common name with us is, parson, and his church whereof he is bishop is neither diocess nor province, but a congregation which met together in one place, and to be taught of one man.” T. C. lib. i. p. 99, 100. [76. ap. Whitg. Def. 360.]

1
[“Judicio Dei ac plebis favore, ad officium Sacerdotii et Episcopatus gradum adhuc neophytus, et ut putabatur, novellus electus est . . . Humiliter ille secessit, antiquioribus cedens, et indignum se titulo tanti honoris existimans.” Pont. Vit. S. Cypr. § 5. p. cxxxvii. ed. Baluz.]

2
[Ibid. § 6. cxxxviii. “Viderint pietatis antistites, seu quos ad officium boni operis instruxit ipsius ordinis disciplina, seu quos sacramenti religio communis ad obsequium exhibendæ religionis arctavit. Cyprianum de suo talem accepit cathedra, non fecit.”]

1
“Etsi fratres pro dilectione sua cupidi sunt ad conveniendum et visitandum confessores bonos, quos illustravit jam gloriosis initiis divina dignatio, tamen caute hoc, et non glomeratim nec per multitudinem simul junctam, puto esse faciendum: ne ex hoc ipso invidia concitetur, et introeundi aditus denegetur, et cum insatiabiles multum [totum] volumus, totum perdamus: consulite ergo et providete ut cum temperamento hoc agi tutius possit; ita ut presbyteri quoque, qui illic apud confessores offerunt, singuli cum singulis diaconis per vices alternent, quia et mutatio personarum, et vicissitudo convenientium minuit invidiam.” Ep. 5. [4. p. 9. ed. Baluz.]

g
bitter, so edd. 1676, 1682.

1
[As T. C. does, in reply to a paper of Jewel’s, ap. Whitg. Def. 428.]

2
[About ad 1147. vid. Pet. Cluniacens. Epist. ap. Bibl. Patr. Colon. t. xii. pars 2. p. 206 H. “Templorum vel ecclesiarum fabricam fieri non debere, factas insuper subrui oportere, nec esse necessaria Christianis sacra loca ad orandum, quoniam æque in taberna et in ecclesia, in foro et in templo, ante altare vel ante stabulum invocatus Deus audit, et eos qui merentur exaudit.” Fleury, E. H. lxix. 24. t. xiv. 600.]

h
So edd. 1676, 1682.

1
Cypr. lib. i. Ep. 3. [al. 59. c. 10. vid. supr. c. xii. p. 215, note 1.]

2
Acts xxv. 11.

1
Liv. lib. i. [c. 49.]

1
[So edd. Query, reverence.] 1886.

1
[Not “ancients:” comp. b. v. lxi. 1; infra, xiv. 13. xv. 1, 12.]

2
[Ep. xxxiii. p. 46. ed. Baluz. “Presbyteris et diaconibus et universæ plebi salutem. In ordinationibus clericis, fratres clarissimi, solemus vos ante consulere, et mores ac merita singulorum communi consilio ponderare. Sed expectanda non sunt testimonia humana cum præcedunt divina suffragia. Aurelius frater noster, illustris adolescens, a Domino jam probatus et Deo carus . . . bis confessus et bis confessionis suæ victoria gloriosus, &c. . . . Merebatur talis clericæ ordinationis ulteriores gradus et incrementa majora . . . Sed interim placuit ut ab officio lectoris incipiat.”]

3
[“Plus vident oculi quam oculus” vid. Erasm. Colloq. Opp. t. i. p. 824. ed. Clerici. Lugd. Bat. 1703.]

1
Acts xiv. 23.

2
1 Tim. v. 22.

3
[1 Tim. iii. 10.]

4
Lamprid. in Alex. Sever. [p. 130. B. ed. Salmas. Paris. 1620. “Ubi aliquos voluisset vel rectores provinciis dare, vel præpositos facere, vel procuratores, i. e. rationales, ordinare, nomina eorum proponebat, hortans populum ut siquis quid haberet criminis, probaret manifestis rebus; si non probasset subiret pœnam capitis: dicebatque, grave esse, quum id Christiani et Judæi facerent in prædicandis sacerdotibus qui ordinandi sunt, non fieri in provinciarum rectoribus, quibus et fortunæ hominum committerentur et capita.”]

1
Dec. Quando Epis. sect. Igitur. [pars i. distinct. 24. p. 114. Lugd. 1572. from the council of Nantes, of uncertain date. “Episcopus quando ordinationem facere disponit, omnes qui ad sacrum ministerium accedere volunt, feria quarta ante ipsam ordinationem evocandi sunt ad civitatem, una cum archipresbyteris qui eos repræsentare debent. Et tunc episcopus e latere suo dirigere debet sacerdotes et alios prudentes viros, gnaros legis divinæ, et exercitatos in ecclesiasticis sanctionibus, qui ordinandorum vitam, genus, patriam, ætatem, institutionem, locum ubi educati sunt, si sint bene literati, si in lege Domini instructi, diligenter investigent. Ante omnia, si fidem catholicam firmiter teneant, et verbis simplicibus asserere queant. Ipsi autem, quibus hoc committitur, cavere debent, ne aut favoris gratia, aut cujuscunque muneris cupiditate allecti, a vero devient, ut indignum et minus idoneum ad sacros gradus suscipiendos episcopi manibus applicent. Quod si fecerint; et ille qui indigne accessit ab altari removebitur, et illi qui donum Sp. sancti vendere conati sunt, coram Deo jam condemnati ecclesiastica dignitate carebunt. Igitur per tres continuos dies diligenter examinentur; et sic sabbato, qui probati inventi sunt, episcopo repræsententur.” Concil. Harduin. vi. pars i. 459.]

2
Cf. V. lxxii. 8.

3
Eccl. Discipl. p. 34. [or p. 22. Cartwright’s Translation, 1617.]

1
Eccl. Discipl. fol. 41. [or p. 27 of Cartwright’s version.]

1
[Eccl. Discipl. transl. by T. C. p. 28. “I would not that the judgment of the rest of the Church should be contemned and neglected, or that the council or elders of the Church should of their own authority set one over the Church whom they list against the Church’s will, but that the elders going before the people also follow, and having heard and understood their sentence and decree, may either by some outward token or else by their silence allow it if it be to be liked of, or gainsay it if it be not just and upright.”]

2
Eccles. Discipl. p. 41. [Ibid. “And not only gainsay it, but if just cause of their disliking may be brought, make it altogether void and of none effect, until at the last a meet one may be chosen by the authority and voices of the elders, and allowed of by the consent and approbation of the rest of the Church. So that herein there is no cause to complain that by the bringing in of the rule of a few the majesty of the whole Church is diminished.”]

1
“Neque enim fas erat aut licebat, ut inferior ordinaret majorem.” Comment. q. Ambros. tribuuntur, in 1 Tim. 3 [§ 7.]

2
[ad 1561. Thuanus, lib. 28. t. ii. p. 45. Gen. 1620. “Claudius Espencæus, vir doctus et pacis ecclesiæ studiosus, a Lotaringo loqui jussus, postquam præfatus est expetivisse jam a multo tempore ut colloquendi copia fieret, et interea semper a suppliciis, quibus ob religionem miseri homines antea afficiebantur abhorruisse; demirari se subinde sæpius dixit, qua auctoritate Protestantes et a quo vocati et instituti ad ministerium essent: et cum neminem citarent, a quo manus impositionem suscepissent, quomodo legitimi pastores censeri possent: nam manifestum esse, vocatione ordinaria minime institutos; cum autem ad extraordinariam miraculis opus sit, nec ea ipsi edant, necessario sequi, nec secundum ordinem nec extra ordinem eos in domum Dei ingressos esse.]

3
[Apol. con. Arian. c. 12. sqq. t. i. 134. ed. Bened. Πόθεν οὐ̑ν πρεσβύτερος Ἰσχύρας; τίνος καταστήσαντος, ἀ̑ρα Κολλούθου; του̑το γὰρ λοιπόν· ἀλλ’ ὅτι Κόλλουθος πρεσβύτερος ὢν ἐτελεύτησε, καὶ πα̑σα χεὶρ αὐτου̑ γέγονεν ἄκυρος, καὶ πάντες οἱ παρ’ αὐτου̑ κατασταθέντες ἐν τῳ̑ σχίσματι λαικοὶ γεγόνασι, καὶ οὕτως συνάγονται, δη̑λον, καὶ οὐδενὶ καθέστηκεν ἀμϕίβολον· πω̑ς οὐ̑ν ἰδιώτης ἄνθρωπος, καὶ οἴκισκον οἰκω̑ν ἰδιωτικὸν, ποτήριον ἔχειν μυστικὸν πιστενθείη;]

1
Ἐπισκοπη̑ς χειροθεσίαν.

2
John iii. 2.

3
[Sleidan. Comment. v. p. 58. Argent. 1556. “Cum ejectus e Saxonia finibus. . . . . Muncerus oberraret, ac rumor increbuisset eum cogitare Mulhusium, Lutherus. . . datis ad senatum literis, graviter monet, ne recipiat . . . recte facturum senatum, si roget ex illo, quis docendi munus ipsi commiserit, quis evocavit? et si Deum nominet authorem, tum jubeant hanc suam vocationem aliquo evidenti signo comprobare, quod si repræsentare non possit, ut tum repudietur: hoc enim esse Deo proprium et familiare, quoties formulam consuetam et rationem ordinariam velit immutari, ut tum voluntatem suam aliquo signo declaret.” v. Fleury, l. 128. c. 45. a. 1523.]

1
[See b. V. c. lxxx. § 11.]

2
[1 Adm. p. 2. ed. 1617. “Then election was made by the elders with the common consent of the whole Church: now every one picketh out for himself some notable good benefice, he obtaineth the next advowson, by money or by favour, and so thinketh himself to be sufficiently chosen. Then, the congregation had authority to call ministers; instead thereof now they run, they ride, and by unlawful suit and buying prevent other suitors also. Then no minister placed in any congregation but by the consent of the people; now that authority is given into the hands of the bishop alone.” Whitg. Answ. 42; T. C. i. 28, al. 43, &c.; Def. 154, &c.; T. C. ii. 194, &c.]

1
[1 Adm. ap. Whitg. Def. 662. “Then it was said, ‘Tell the Church;’ now it is spoken, ‘Complain to my lord’s grace, primate and metropolitan of all England, or to his inferior my lord bishop of the diocese; if not to him, shew the chancellor or official or commissary.’ ” Answ. ibid. “In that place of Matthew, as all learned interpreters both old and new do determine, the Church signifieth such as have authority in the Church.” T. C. i. 146, al. 183, &c.; Def. 662-671; T. C. iii. 77-88.]

2
[T. C. i. 147. al. 184. “It must needs be the meaning of our Saviour Christ that the excommunication should be by many and not by one; and by the Church and not by the minister of the Church alone.” Ibid. 183. “That the charge of excommunication belongeth not unto one, to the minister, but chiefly to the eldership and pastor, it appeareth by that which the authors of the Admonition allege out of St. Matthew, xviii. 17; which place I have proved before to be necessarilyunderstanded of the elders of the Church.”]

3
Concil. Carthag. iv. c. 23. [ad 398. “Ut episcopus nullam causam audiat absque præsentia clericorum suorum; alioquin irrita erit sententia episcopi, nisi clericorum præsentia confirmetur.” i. 980. ed. Harduin.]

4
Cypr. lib. iii. Ep. 10. [5. ed. Baluz. p. 11. “Solus rescribere nihil potui, quando a primordio episcopatus mei statuerim nihil sine consilio vestro et sine consensu plebis mea privatim sententia gerere.”] 14. [11. Baluz. p. 21. “Audio quosdam de presbyteris, nec evangelii memores, nec quid ad nos martyres scripserint cogitantes, nec episcopo honorem sacerdotii sui et cathedræ reservantes, jam cum lapsis communicare cœpisse et offerre pro illis et eucharistiam dare, quando oporteat ad hoc per ordinem perveniri. Nam cum in minoribus delictis quæ non in Deum committuntur pœnitentia agatur justo tempore, et exomologesis fiat inspecta vita ejus qui agit pœnitentiam, nec ad communicationem venire quis possit nisi prius illi ab episcopo et clero manusfuerit imposita: quanto magis in his gravissimis et extremis delictis caute omnia et moderate secundum disciplinam Domini observari oportet! . . . Audiant quæso patienter consilium nostrum, expectent regressionem nostram, ut cum ad vos per Dei misericordiam venerimus, convocati coëpiscopi plures, secundum Domini disciplinam et confessorum præsentiam et vestram quoque sententiam, beatorum martyrum literas et desideria examinare possimus.”] Lib. ii. Ep. 8. [59. Baluz. p. 97. “Significasti de Victore quodam presbytero, quod ei, antequam pœnitentiam plenam egisset, et Domino Deo, in quem deliquerat, satisfecisset, Therapius collega noster immaturo tempore et præpropera festinatione pacem dederit. Quæ res nos satis movit, recessum esse a decreti nostri auctoritate, ut ante legitimum et plenum tempus satisfactionis, et sine petitu et conscientia plebis, nulla infirmitate urgente ac necessitate cogente, pax ei concederetur. Sed librato apud nos diu consilio, satis fuit objurgare Therapium collegam nostrum, quod temere hoc fecerit, et instruxisse, ne quid tale de cætero faciat. Pacem tamen quomodocunque a sacerdote Dei semel datam non putavimus auferendam.” These passages are producted by T. C. i. 149. al. 187, and maintained, iii. 87-89, in order to shew that the bishop might not absolve alone. Comp. Whitg. Def. 674.]

1
[Compare De Presbyterio et Excommunicatione: p. 112, 113. Gen. 1590. “Hoc veluti fræno, (sc. presbyterio) coercebatur tum ipsorum pastorum tum etiam populi potestas, ne illa quidem in oligarchiam, ista vero in ochlocratiam degeneraret. Itaque mihi quidem ecclesia Christiana, ut et vetus illa Israelitica, ex illo triplici statu divinissime constituta videtur: cujus caput est et monarcha longe supra omnia eminens unicus ille noster Pontifex æternus, cujus figura fuit Leviticus ille Pontifex . . . Isti vero cœtus divinissimam aristocratiam referunt. Universa denique multitudo, qua conscia, et ex cujus consensu cœtus ipsi aristocratici constituuntur, cælestis democratiæ perfectum exemplum præbet.” And Epist. xii. p. 220. ad 1567. Tract. t. iii. Gen. 1582. “Aiunt . . . excommunicationes et absolutiones in curiis quibusdam episcopalibus in Anglia fieri non ex presbyterii (quod nullum ibi sit) sententia, neque ex Dei verbo, sed ex quorundam jurisconsultorum et aliorum ejusmodi, immo etiam interdum unius cujuspiam auctoritate . . . Respondemus, nobis pæne incredibile videri, ejusmodi abusum perversissimi moris et exempli adhuc in eo regno usurpari, ubi puritas doctrinæ vigeat. Jus enim excommunicandi ante papisticam illam tyrannidem nunquam penes unum fuisse comperietur, sed penes presbyterium, et quidem non excluso penitus populo.” This is the epistle to the leading Puritans, which was so industriously circulated in England. Vid. vol. ii, p. 133, note 3.]

1
[Adm. ap. Whitg. Def. 749. “In that they have civil offices joined to the ecclesiastical, it is against the word of God. As for an archbishop to be a lord president, a lord bishop to be a county palatine, a prelate of the garter, who hath much to do at St. George’s feast when the Bible is carried before the procession in the cross’s place, a justice of peace, a justice of quorum, an high commissioner, &c. And therefore they have their prisons, as Clinks, Gatehouses, Colehouses, towers and castles; which is against all the Scriptures; Luke ix. 60, 61; xii. 14; Rom. xii. 7; 1 Tim. vi. 11; 2 Tim. ii. 3, 4;” Answ. 114, &c.; T. C. i. 206, al. 165, &c.; Def. 749, &c.; T. C. iii. 1-31; Decl. of Discipl. 39-44, ed. 1617.]

2
Jer. xxix. 26.

1
1 Cor. vi. 1-7.

1
Vide Barnab. Brisson. [Bernabé Brisson, an eminent French lawyer under Henry III, hung by the Leaguers Nov. 1591.] Antiq. Jur. lib. iv. c. 16. [“Conjunctam olim fuisse juris divini et humani scientiam. Ridiculum videtur nonnullis jurisprudentiam rerum divinarum et humanarum notitiam ab Ulpiano definiri, quod existimant rerum divinarum cognitionem nihil cum juris civilis scientia commune habere. Atqui ex veteribus memoriis certissimum est in utriusque facultatis cognitione consultos pares fuisse, tenuisseque et edocuisse eos quibus hostiis, quibus diebus, quo ritu, ad quæ templa sacra facienda essent, quæ sepulchrorum monumentorumque jura, quæ justorum funebrium solemnia essent. Quæ ad jus publicum et divinum referebantur omnia.” p. 136. Paris. 1606.]

2
Aug. de Oper. Monach. c. 29. [t. vi. 499. “Dominum Jesum, in cujus nomine securus hæc dico, testem invoco super animam meam, quoniam quantum attinet ad meum commodum, multo mallem per singulos dies, certis horis, quantum in bene moderatis monasteriis constitutum est, aliquid manibus operari, et cæteras horas habere ad legendum et orandum aut aliquid de divinis literis agendum liberas, quam tumultuosissimas perplexitates causarum alienarum pati, de negotiis sæcularibus vel judicando dirimendis, vel interveniendo præcidendis: quibus nos molestiis idem affixit Apostolus, non utique suo, sed ejus qui in eo loquebatur arbitrio; quas tamen ipsum perpessum fuisse non legimus: aliter enim se habebat Apostolatus ejus discursus . . . Sapientes ergo qui in locis consistebant, fideles et sanctos, non qui hac atque illac propter evangelium discurrebant, talium negotiorum examinatores esse voluit. Unde nunquam de illo scriptum est quod aliquando talibus vacaverit, a quibus nos excusare non possumus, etiamsi contemptibiles sumus, quia et hos collocari voluit, si sapientes defuissent, potius quam ut negotia deferrentur in forum. Quem tamen laborem non sine consolatione Domini suscipimus, pro spe vitæ æternæ, ut fructum feramus cum tolerantia.” Quoted by Bp. Jewel in Whitg. Answ. 325. See T. C. i. 171; Def. 771; T. C. iii. 26; Sarav. de Hon. Præs. c. 20.]

1
Isaiah xlix. 23.

1
Zanchius [Jerome Zanchi of Bergamo † 1590: he taught theology at Strassburg and Heidelberg 1553, 1568], p. 274. Observ. in Confess. [t. viii. 547. c. xxv. aphorism. 21. “Non diffitemur, episcopos, qui simul principes sunt, præterauctoritatem ecclesiasticam, sua etiam habere jura politica, sæcularesque potestates, quemadmodum et reliqui habent principes jus imperandi sæcularia, jus gladii, nonnullos jus eligendi confirmandique reges et imperatores, aliaque politica constituendi et administrandi, subditosque sibi populos ad obedientiam sibi præstandam cogendi. Ac proinde fatemur, politicis horum mandatis, quæ sine transgressione legis divinæ servari possunt, a subditis obtemperandum esse, non solum propter timorem sed etiam propter conscientiam.” And Append. p. 584. “Duæ longe diversæ sunt quæstiones, utrum episcopis liceat etiam esse principibus, principibusque esse episcopis, suis retentis principatibus; et, an qui episcopi jam sunt simul et principes, ii præter auctoritatem ecclesiasticam jura etiam habeant politica in cives sibi subjectos; eoque an subditi illis tanquam principibus obedire debeant necne. In meo aphorismo nihil prorsus de priori quæstione locutus sum, quia non fuit necesse, sed tantum de posteriori. Quis autem illis omnino obediendum esse, quo jure, quaque injuria principes fuerint creati, ex testimoniis a me allatis non videat aperte demonstrari? Cur enim qui subditi sunt Moguntino, Coloniensi, Trevirensi principibus Imperii simul et archiepiscopis, in rebus cum pietate Christiana minime pugnantibus non obtemperent? Seditiosorum certe fuerit non obtemperare. Quod si istis, cur non etiam Romano, iisdem in rebus et eandem ob causam, qui sub ejus vivunt imperio? Eadem enim horum omnium est ratio. De priori quæstione nihil (ut ante dixi) disserui; sed neque etiam nunc in hac mea brevi confessione disputare constitui; cum sciam, non omnium eandem esse sententiam; et in utramque partem multa dici possint.” ed. 1605.]

1
[Especially in the two embassies to Maximus, ad 383, and 387. vid. ep. xxiv. ed. Bened. t. ii. 888-891.]

1
[Hooker seems to refer to the conference at Paris, Dec. 1329, between the archbishop of Sens and Bertrand bishop of Autun as representatives of the Church, and Pierre de Cugnières as advocate for the royal and baronial authority: the particulars of which may be seen, Concil. Harduin. vii. 1544; or abstracted in the continuation of Fleury, liv. xciv. c. 2-5. Goldastus, Monarch. S. R. I. t. iii. p. 1383, having inserted the document, adds, “Sic re aliquamdiu ultro citroque agitata, cum episcopi et prælati se suo solito more reformarent, ita nempe, ut specie ac verbis injuriarum quandam alleviationem simularent, re autem ipsa ea potius augerent et aggravarent quam diminuerent; demum rex severam quandam legem fert, qua talem prælatorum audaciam et tyrannidem cohibet, seque ac suos in libertatem asserit.” But it seems clear from a papal letter to the king, quoted in Raynaud’s Continuation of Baronius, ad 1329, that this latter statement (which is similar to Hooker’s) must be erroneous. No authority for it is given. But in the proceedings of the conference complaint is made by the clergy, “quod quædam præconizationes factæ erant in præjudicium jurisdictionis ecclesiasticæ, quas supplicabant revocari. Tum dominus rex respondit ore proprio, quod non erant factæ de suo mandato, nec aliquid sciebat, nec eas ratas habebat.” Possibly the statement in the text may be traced to some of these ordinances, either spurious at first, or such as it was found convenient to disavow. Henault’s account is, “Le roi est favorable aux ecclésiastiques, mais cette querelle est le fondement de toutes les disputes qui se sont élevées depuis par rapport à l’autorité des deux puissances, et dont l’effet a été de restraindre la jurisdiction ecclésiastique dans des bornes plus étroites.” Abrégé Chronol. de l’Hist. de France, t. i. p. 52, Paris, 1768.]

1
[Eccl. Disc. fol. 57-64. “Episcopi nomen, a Græca voce ἐπισκοπει̑ν deductum, speculatorem aut vigilem significat, qui castris custodiendis, aut ad urbis vigilias ad hostium adventum denunciandum designatus est. . . Est autem episcopus, si vere illum definire volumus, minister ecclesiæ in rebus divinis, et ad Deum pertinentibus. . . Sic Timothei (quamvis evangelistæ) munus Paulus domus Dei gubernatione et administratione definivit. Et Apostolus ad Hebræos animarum procuratione τω̑ν ἡγουμένων curam complexus est. . . Videamus, recte ne eorum munus religione et cærimoniis tractandis definitum sit. Vetus enim opinio est, et ab antiquis ducta temporibus, episcopos non ita rei divinæ faciendæ terminis circumscribi, quin etiam humana tractare possint, ac simul quidem ecclesiam et rempublicam administrare. Hinc apud nos episcopi pacis et otii communis conservandi auctoritatem habent, et ejus violatores in carcerem atque vincula conjiciendi, testamentorum lites, et alias civilium controversias in suo foro audiendi, disceptandi judicandique potestatem.” &c. Decl. of Disc. 75-77, 85.]

2
[Eccl. Disc. 60, “Quum utraque potestas primo in Mose confusa esset, Deus, republ. Mosi relicta, ecclesiæ gubernationem ad Aaronem fratrem transtulit.” Decl. of Disc. 79.]

3
[T. C. iii. 7. “In saying that ‘although the godly magistrate ruleth in the Lord over us, yet that this title is given by excellency (1 Thess. v. 12.) to ecclesiastical officers,’ I do not dally; it is the distinction of the Holy Ghost himself. For albeit they that handle commonwealth matters serve the Lord, and do things tending to his glory, yet the Scripture comparing both these governments together giveth this title as a note to discern the ecclesiastical officers from the civil; as appeareth in the Chronicles, (2 Chr. xix. 11,) from whence (it is like) the Apostle took this manner of speech.”]

1
[Whitg. Answ. 217. ap. Def. 767. “What say you to Eli and Samuel? were they not both priests and judges?” T. C. i. 170, al. 211. “As for Eli and Samuel, they are extraordinary examples, which may thereby appear, for that both these offices first meeting in Melchisedech and afterward in Moses were by the commandment of God severed, when as the Lord took from Moses the priesthood, and gave it to Aaron and his successors.” Whitg. Def. 767. “It is not certain whether Moses were ever priest or no. . . Howsoever the priesthood and civil magistracy were divided in Moses and Aaron, yet met they both together again not only in Eli and Samuel, but in Esdras, Nehemias, Matthias and some other.” T. C. iii. 21. “Such were extraordinarily raised up of God, and not by any established order or election of men.”]

2
[Whitg. Def. 769. “Remember I pray you what you said before in the treatise of Seniors: you there set it down that they are ecclesiastical persons; and yet M. Beza as I have there declared saith that noblemen and princes may be of the seigniory; wherefore either may civil and ecclesiastical offices meet together in ecclesiastical persons (which you deny); or else cannot noblemen and princes be of your seigniory, as M. Beza affirmeth.”]

1
[S. John viii. 11. alleged by T. C. iii. 3.]

2
[S. Luke xii. 14. alleged by Adm. see Ans. 264, 266, al. 215; T. C. i. 165; Def. 751; T. C. iii. 2.]

3
[“He, because he came not but to be a Mediator between God and man, would not become a common divider and judge of every secular cause of title of land: . . . ‘Neither my heavenly Father sent me to that end, neither have I commission from thy brother to send thee into the moiety of the possession.’ Besides, if he had intermeddled in the matters of the commonwealth, it would have strengthened the conceit, that he sought an earthly kingdom, and to dispossess the Romans. . .Christ did not condemn the woman taken in the act of adultery: shall not therefore officers ecclesiastical condemn any such sinner? Christ refused to divide the inheritance: it was because he would not use the authority that he had as Lord of heaven and earth, when he came as a servant: not because either a Christian magistrate or minister should after his example lay aside all authority: τίς μὲ κατέστησε; implieth rather that if he had been appointed by both the parties, he might have done it; and so may any minister arbitrate and compound a controversy civil that is committed unto him.” Sutcliffe, Rem. to Dem. of Disc. 179.]

4
2 Tim. ii. 4. [quoted in Adm. See Answ. 216; T. C. i. 166; Def. 754; T. C. iii. 6.]

1
[Acts vi. 4. ap. T. C. i. 167, al. 208; Def. 758; T. C. iii. 10.]

2
[Hooker here forsakes the rendering of the Geneva Bible, which he commonly adopts, and translates the verse for himself.]

3
“Convenit hujusmodi eligi et ordinari sacerdotes, quibus nec liberi sunt nec nepotes. Etenim fieri vix potest, ut vacans hujus vitæ quotidianæ curis, quas liberi creant parentibus maxime, omne studium omnemque cogitationem circa divinam liturgiam et res ecclesiasticas consumat.” [Cod. Justin. lib. i. tit. iii.] xlii. sect. 1. de Episc. et Cler.

1
[Can. Apost. 72. Εἴπομεν, ὅτι μὴ χρὴ ἐπίσκοπον καθει̑ναι ἑαυτὸν εἰς δημοσίας διοικήσεις, ἀλλὰ προσευκαιρει̑ν ται̑ς ἐκκλησιαστικαι̑ς χρείαις· ἢ πειθέσθω οὐ̑ν του̑το μὴ ποιει̑ν, ἢ καθαιρείσθω. οὐδεὶς γὰρ δύναται δυσὶ κυρίοις δουλεύειν, κατὰ τὴν κυριακὴν παρακέλευσιν. Ed. Coteler. t. i. 452. Conc. Chalc. can. 3. ἠ̑λθεν εἰς τὴν ἁγίαν σύνοδον, δτι τω̑ν ἐν τῳ̑ κλήρῳ κατειλεγμένων τινὲς δι’ οἰκείαν αἰσχροκερδείαν ἀλλοτρίων κτημάτων γίνονται μισθωταὶ, καὶ πράγματα κοσμικὰ ἐργολαβου̑σι, τη̑ς μὲν του̑ Θεου̑ λειτουργίας καταρραθυμου̑ντες, τοὺς δὲ τω̑ν κοσμικω̑ν ὑποτρέχοντες οἴκους, καὶ οὐσιω̑ν χειρισμοὺς ἀναδεχόμενοι διὰ ϕιλαργυρίαν. ὥρισε τοίνυν ἡ ἁγία συνόδος, μηδένα του̑ λοιπου̑, μὴ ἐπίσκοπον, μὴ κληρικὸν, μὴ μονάζοντα, ἢ μισθου̑σθαι κτήματα ἢ πράγματα, ἢ ἐπεισάγειν ἑαυτὸν κοσμικαι̑ς διοικήσεσι· πλὴν εἰ μή που ἐκ νόμων καλοι̑το εἰς ἀϕηλίκων ἀπαραίτητον ἐπιτροπὴν, ἢ ὁ τη̑ς πόλεως ἐπίσκοπος ἐκκλησιαστικω̑ν ἐπιτρέψοι ϕροντίζειν πραγμάτων, ἢ ὀρϕάνων καὶ χηρω̑ν ἀπρονοήτων, καὶ τω̑ν προσώπων τω̑ν μάλιστα τη̑ς ἐκκλησιαστικη̑ς δεομένων βοηθείας, διὰ τὸν ϕόβον του̑ Κυριου̑. εἰ δέ τις παραβαίνειν τὰ εἰρημένα του̑ λοιπου̑ ἐπιχειρήσοι, ὁ τοιου̑τος ἐκκλησιαστικοι̑ς ὑποκείσθω ἐπιτιμίοις. t. ii. 601, ed. Harduin. And can. 7, τοὺς ἅπαξ ἐν κλήρῳ κατειλεγμένους, ἢ καὶ μονασάντας, ὡρίσαμεν, μήτε ἐπὶ στρατείαν, μήτε ἐπὶ ἀξίαν κοσμικὴν ἔρχεσθαι· ἢ του̑το τολμω̑ντας, καὶ μὴ μεταμελουμένους, ὥστε ἐπιτρέψαι ἐπὶ του̑το, ὃ διὰ θεὸν πρότερον εἵλοντο, ἀναθεματίζεσθαι. ibid. 603. ap. T. C. i. 168, al. 210; Def. 762; T. C. iii. 15; who refers also to 4 Conc. Carthag. can. 20. “Ut episcopus nullam rei familiaris curam ad se revocet, sed lectioni et orationi et verbi Dei prædicationi tantummodo vacet.” ibid. i. 986.]

2
[S. Cypr. 1 Ep. ed. Fell. “Graviter commoti sumus. . . cum cognovissemus quod Geminius Victor frater noster de sæculo excedens . . . Presbyterum tutorem testamento suo nominaverit: cum jam pridem in concilio episcoporum statutum sit, ne quis de clericis et Dei ministris tutorem vel curatorem testamento suo constituat, quando singuli divino sacerdotio honorati et in clerico ministerio constituti non nisi altari et sacrificiis deservire et precibus atque orationibus vacare debeant. Scriptum est enim, ‘Nemo militans Deo obligat se molestiis sæcularibus.’ . . . Quod cum de omnibus dictum sit, quanto magis clerici molestiis et laqueis sæcularibus obligari non debent! . . . Quod episcopi antecessores nostri religiose considerantes, et salubriter providentes, censuerunt ne quis frater excedens, ad tutelam vel curam clericum nominaret: ac si quis hoc fecisset, non offerretur pro eo, nec sacrificium pro dormitione ejus celebraretur.” Ap. T. C. i. 166, al. 207; Def. 754; T. C. iii. 6. He quotes also S. Ambr. de Offic. i. 38. (36.) “Non te implices negotiis sæcularibus, quoniam Deo militas. Etenim si is qui imperatori militat a susceptionibus litium, actu negotiorum forensium, venditione mercium prohibetur humanis legibus: quanto magis qui fidei exercet militiam ab usu negotiationis abstinere debet: agelluli sui contentus fructibus, si habet; si non habet, stipendiorum suorum fructu.” And S. Jer. on Zephaniah, c. 1. “Eos, qui adorant Dominum et Melchom: qui sæculo pariter et Domino putant se posse servire, et duobus Dominis satisfacere, Deo et Mammonæ; qui militantes Christo obligant se negotiis sæcularibus, et eandem imaginem offerunt Deo et Cæsari, et cum Christi sacerdotes se esse dicant, filios consecrant Melchom, i. e. regi suo.” t. vi. 680.]

1
“Cum multa divinitus, pontifices, a majoribus nostris inventa atque instituta sunt, tum nihil præclarius, quam quod vos eosdem et religionibus deorum immortalium, et summæ reipub. præesse voluerunt.” Cic. pro Domo sua ad Pontific. [c. 1.]

2
“Honor sacerdotii firmamentum potentiæ assumebatur.” Tacit. Hist. lib. v. [c. 8. fin.] He sheweth the reason wherefore their rulers were also priests. The joining of these two powers, as now, so then likewise, profitable for the public state, but in respects clean opposite and contrary. For whereas then divine things being more esteemed, were used as helps for the countenance of secular power; the case in these latter ages is turned upside down, earth hath now brought heaven under foot, and in the course of the world, hath of the two the greater credit. Priesthood was then a strengthening to kings, which now is forced to take strength and credit from far meaner degrees of civil authority. “Hic mos apud Judæos fuit, ut eosdem reges et sacerdotes haberent, quorum justitia religioni permixta incredibile quantum evaluere.” Just. Hist. lib. xxxvi. [c. 2.]

1
Cod. Justin. I. iii. de Episcopis, &c. 42. § 2. [“De his vero episcopis, qui nunc sunt, vel futuri sunt, sancimus, nullo modo habere eos facultatem testandi vel donandi vel per aliam quamcunque excogitationem alienandi quid de rebus suis, quas postquam facti fuerint episcopi possederint et acquisierint, vel ex testamentis, vel ex donationibus, vel alia quacunque causa: exceptis duntaxat his, quas ante episcopatum habuerint ex quacunque causa, vel quæ post episcopatum a parentibus et theiis, h. e. patruis vel avunculis, et a fratribus ad ipsos pervenerunt, perventuraque sunt: quæcunque enim post ordinationem ex quacunque causa extra præfatas personas ad ipsos pervenerunt, ea jubemus ad sanctissimam ecclesiam, cujus episcopatum tenuerint, pertinere.” ad 528.]

2
T. C. lib. i. p. 126. [98, ap. Whitg. Def. 452. “I have done, only this I admonish the reader, that I do not allow of all those things which I before alleged in the comparison between our archbishops and the archbishops of old time, &c. . . . Only my intent is to show that although there were corruptions, yet in respect of ours they be much more tolerable.”]

1
[Adm. ap. Whitg. Def. 57. “The lordly lords, archbishops, bishops, suffragans, deans, doctors, archdeacons, chancellors, and the rest of that proud generation, whose kingdom must down, hold they never so hard, because their tyrannous lordship cannot stand with Christ’s kingdom. And it is the special mischief of our English church, and the chief cause of backwardness, and of all breach and dissention. For they whose authority is forbidden by Christ, will have their stroke without their fellow servants. Matt. xx. 25, 26; xxiii. 8, 9; Mark x. 42, 43; Luke xxii. 15, &c.” Ans. 37-39, al. 13, &c.; T. C. i. 10, al. 22; Def. 61-75; T. C. ii. 421-436.]

2
[T. C. i. 10, al. 22. “Our Saviour Christ upon occasion of the inordinate request of the sons of Zebedee, putteth a difference between the civil and ecclesiastical function. He placeth the distinction of them in two points; whereof the one is in their office, and the other is in their names and titles. The distinction of the office he noteth in these words: ‘The kings of the gentiles, &c.’ Whereupon the argument may be thus gathered; That wherein the civil magistrate is severed from the ecclesiastical officer doth not agree to one minister over another. But the civil magistrate is severed from the ecclesiastical officer by bearing dominion; therefore bearing dominion doth not agree to one minister over another.”]

3
[De Brés, “La Racine, &c. Des Anabaptistes, &c.” p. 841.]

1
[“Horum verborum verus et simplex hic est sensus: Vestra gubernatio diversa erit ab ea quæ est regum propria. . . Si quis locus citari potest ex evangelicis scriptis ad probandam superioritatem inter evangelii ministros, hic unus est. . . ubi omnes sunt futuri pares, præcepto nihil opus quo moderatio mandatur in præcipua dignitate constituto. Sensus igitur hujus præcepti est, Quanto quis inter vos major erit tanto submissius inter suos fratres se gerat. Tametsi omnes Apostoli ejusdem ordinis et potestatis fuerint, ætatis discrimen et donorum Sp. Sancti magnum inter eos fuit.” Sar. de divers. Min. Grad. c. 15. vid. etiam de Honore Præsulibus debito, c. 2.]

2
T. C. lib. i. 100. [al. 76. ap. Whitg. Def. 361.]

1
Lib. iv. ep. 9. [ii. p. 166. ed. Fell. “Quis autem nostrum longeest ab humilitate? utrumne ego, qui quotidie fratribus servio, et venientes ad ecclesiam singulos benigne et cum voto et gaudio suscipio? an tu qui te episcopum episcopi, et judicem judicis ad tempus a Deo dati constituis?”]

2
[T. C. i. 72.] Ὥστε τὸν τη̑ς πρὼτης καθέδρας ἐπίσκοπον μὴ λέγεσθαι ἔξαρχον τω̑ν ἱερέων ἢ ἀκρὸν ἱερέα ἢ τοιουτότροπόν τί ποτε, ἀλλὰ μόνον ἐπίσκοπον τη̑ς πρώτης καθέδρας. Can. 39. [Cod. Can. Eccl. Afr. 39. ap. Harduin. Conc. i. 884. or 3 Conc. Carth. can. 26. p. 964. ad 397.]

3
[S. Cyp. Ep. 49. p. 63. ed. Baluz. “Novatus, qui apud nos primum discordiæ et schismatis incendium seminavit; qui quosdam istic ex fratribus ab episcopo segregavit; . . cum sua tempestate Romam quoque ad evertendam ecclesiam navigans similia illic et paria molitus est, a clero portionem plebis avellens . . . Damnare nunc audet sacrificantium manus, cum sit ipse nocentior.”]

1
[Ibid. Ep. 69. p. 121. “Prævaluit apud te contra divinam sententiam et contra conscientiam nostram fidei suæ viribus nixam inimicorum et malignorum commentum, quasi apud lapsos et prophanos et extra ecclesiam positos, de quorum pectoribus excesserit Sp. Sanctus, esse aliud possit nisi mens prava et fallax lingua et odia venenata et sacrilega mendacia; quibus qui credit, necesse est cum iis inveniatur cum judicii dies venerit.”]

2
[Ib. Ep. 69. p. 122. “Quæ mentis inflatio, ad cognitionem suam præpositos et sacerdotes vocare, ac nisi apud te purgati fuerimus et sententia tua absoluti, ecce jam sex annis nec fraternitas habuerit episcopum, nec plebs præpositum, nec grex pastorem, nec ecclesia gubernatorem, nec Christus antistitem, nec Deus sacerdotem.”]

1
Concil. Carthag. de Hæret. baptizandis. [p. 329. ed. Baluz. Superest ut de hac ipsa re singuli quid sentiamus proferamus, neminem judicantes, aut a jure communicationis aliquem, si diversum senserit, amoventes. Neque enim quisquam nostrum episcopum se esse episcoporum constituit, aut tyrannico terrore ad obsequendi necessitatem collegas suos adigit; quando habeat omnis episcopus pro licentia libertatis et potestatis suæ arbitrium proprium, tamque judicari ab alio non possit, quam nec ipse potest alterum judicare.”]

2
Lib. ii. Ep. i. [72. “Hæc ad conscientiam tuam, frater carissime, et pro honore communi et pro simplici dilectione pertulimus, credentes etiam tibi pro religionis tuæ et fidei veritate placere quæ et religiosa pariter et vera sunt. Cæterum scimus quosdam quod semel imbiberint nolle deponere, nec propositum suum facile mutare, sed salvo inter collegas pacis et concordiæ vinculo quædam propria quæ apud se sunt semel usurpata retinere. Qua in re nec nos vim cuiquam facimus aut legem damus, quando habeat in ecclesiæ administratione voluntatis suæ arbitrium liberum unusquisque præpositus, rationem actus sui Domino redditurus.” p. 129.]

1
Ωστε τὰ ἐν τῃ̑ Νικαέων συνόδῳ ὁρισθέντα παντὶ τρόπῳ παραϕυλαχθήσεται. [Conc. Hard. i. 868.]

2
T. C. lib. i. p. 113. [al. 87. ap. Whitg. Def. 408. Whitgift (Answ. 72.) had quoted from S. Ign. interp. ad Smyrnæos, c. 9: Τίμα τὸν θεὸν, ὡς αἰτίον τω̑ν ὅλων καὶ κύριον· ἐπίσκοπον δὲ, ὡς ἀρχιερέα, θεου̑ εἰκόνα ϕορου̑ντα: T. C. replies, “As for Ignatius’ place, it is sufficiently answered before, in that which was answered to Cyprian his place: for when he saith, ‘the bishop hath rule over all,’ he meaneth no more all in the province, than in all the world, but meaneth that flock or congregation whereof he is bishop or minister. And when he calleth him ‘prince of the priests,’ although the title be excessive and big, condemned by Cyprian and the council of Carthage, yet he meaneth no more the prince of all in the diocese as we take it, or of the province, than he meaneth the prince of all the priests in the world: but he meaneth those fellow ministers and elders that had the rule and government of that particular church and congregation whereof he was bishop.”]

1
Theod. Hist. Eccles. lib. i. cap. 7. Ἀρχιερει̑ς. [Ἀϕίκετο καὶ αὐτὸς εἰς τὴν Νικαίαν, ἰδει̑ν τὴν τω̑ν ἀρχιερέων πληθὺν ἐϕιέμενος . . . ὀκτωκαίδεκα δὲ καὶ τριακόσιοι συνη̑λθον ἀρχιερει̑ς.]

2
Hieronymus contra Luciferian. “Salutem ecclesiæ pendere,” dicit, “a summi sacerdotis dignitate,” id est, episcopi. [c. ix. t. ii. 182. ed. Vallarsii.] Idem est in Hieronymo summus sacerdos quod ἄκρος ἱερεὺς in Carthaginensi Concilio.

3
Vide C. omnes. 38 dist. [Decret. Gratian. pars i. d. 38, p. 184. Lugd. 1572. “Ex septima synodo.” . . . . “Substantia summi sacerdotii nostri sunt eloquia divinitus tradita, i. e. vera divinarum scripturarum disciplina: quemadmodum magnus perhibet Dionysius.” Comp. 2 Concil. Nicæn. ad 787. can. ii. οὐσία τη̑ς καθ’ ἡμα̑ς ἱεραρχίας ἐστὶ τὰ θεοπαράδοτα λογία, εἴτουν ἡ τω̑ν θείων γραϕω̑ν ἀληθινὴ ἐπιστήμη, κάθως ὁ μέγας ἀπεϕήνατο Διονύσιος. t. iv. 48. ed. Hard. comp. Dionys. de Eccl. Hierarch. c. v. § 7.] Item c. Pontifices, [Decr. Grat. pars ii. causa] xii. qu. 3. [p. 1001. “Pontifices quibus in summo sacerdotio constitutis ab extraneis aliquid . . . donatur . . . inter facultates ecclesiæ computabunt.” In Consilio Agathensi [can. 6. ad 506. t. ii. 998. Hard.] Item c. De his. [Decr. Grat. pars iii.] De Consecr. dist. 5. [p. 1991, from a supposed decretal epistle of Melchiades. “Utrum majus sit sacramentum manus impositionis episcoporum aut baptismi: scitote utrumque magnum esse sacramentum: et sicut unum majoribus, i. e. summis Pontificibus, est accommodatum . . . ita et majori veneratione venerandum.” Comp. Conc. i. 245.]

2
Hieronymus contra Luciferian. “Salutem ecclesiæ pendere,” dicit, “a summi sacerdotis dignitate,” id est, episcopi. [c. ix. t. ii. 182. ed. Vallarsii.] Idem est in Hieronymo summus sacerdos quod ἄκρος ἱερεὺς in Carthaginensi Concilio.

3
Vide C. omnes. 38 dist. [Decret. Gratian. pars i. d. 38, p. 184. Lugd. 1572. “Ex septima synodo.” . . . . “Substantia summi sacerdotii nostri sunt eloquia divinitus tradita, i. e. vera divinarum scripturarum disciplina: quemadmodum magnus perhibet Dionysius.” Comp. 2 Concil. Nicæn. ad 787. can. ii. οὐσία τη̑ς καθ’ ἡμα̑ς ἱεραρχίας ἐστὶ τὰ θεοπαράδοτα λογία, εἴτουν ἡ τω̑ν θείων γραϕω̑ν ἀληθινὴ ἐπιστήμη, κάθως ὁ μέγας ἀπεϕήνατο Διονύσιος. t. iv. 48. ed. Hard. comp. Dionys. de Eccl. Hierarch. c. v. § 7.] Item c. Pontifices, [Decr. Grat. pars ii. causa] xii. qu. 3. [p. 1001. “Pontifices quibus in summo sacerdotio constitutis ab extraneis aliquid . . . donatur . . . inter facultates ecclesiæ computabunt.” In Consilio Agathensi [can. 6. ad 506. t. ii. 998. Hard.] Item c. De his. [Decr. Grat. pars iii.] De Consecr. dist. 5. [p. 1991, from a supposed decretal epistle of Melchiades. “Utrum majus sit sacramentum manus impositionis episcoporum aut baptismi: scitote utrumque magnum esse sacramentum: et sicut unum majoribus, i. e. summis Pontificibus, est accommodatum . . . ita et majori veneratione venerandum.” Comp. Conc. i. 245.]

2
1 Pet. ii. 17.

3
Ecclus. xxxviii. 1.

4
Levit. xix. 32.

1
Ecclus. xxv. 6.

2
Prov. xxiii. 22.

3
1 Pet. ii. 14.

4
Psalm lxxii. 15.

1
[Compare b. v. c. lxxvi. § 2.]

2
“Quis est tam vecors, qui aut cum suspexerit in cœlum, Deos esse non sentiat, et ea, quæ tanta mente fiunt ut vix quisquam arte ulla ordinem rerum ac necessitudinem persequi possit, casu fieri putet; aut, cum Deos esse intellexerit, non intelligat eorum numine hoc tantum imperium esse natum et auctum et retentum.” Cic. Orat. de Harusp. Resp. [c. 9.]

1
Tit. 1. l. iii. C. [lib. i.] de summa Trinit.

2
L. 43 C. [Cod. lib. i. tit. iii.] de Episc. et Cler. [lex 43.]

3
L. 34 C. de Episcopali Audientia. [Ibid. i. iv. 34.]

4
Psalm lxxvii. 20.

1
“Qui sacerdotes in Veteri Testamento vocabantur, hi sunt qui tunc presbyteri appellantur: et qui tunc princeps sacerdotum, nunc episcopus vocatur.” Raban. Maur. de Instit. Cler. lib. i. cap. 6. [Opp. t. vi. 5. ed. Colon. 1526.]

1
1 Tim. v. 17.

2
Rom. xiii. 7.

1
Deut. iv. 6.

2
Matt. v. 14.

1
Petr. Blesens. Ep. 5. [t. xii. par. ii. p. 704. Biblioth. Patr. Colon. “Ego, qui conscius secretorum fui, verbum illud confidenter in communem deduco notitiam: verbum enim memorabile est. ‘Sciat,’ inquit, ‘Dominus Archiepiscopus, quod si meus filius electus, aut aliquis episcopus terræ, vel comes, vel aliqua persona illustris, suæ voluntati aut dispositioni contrarie præsumpserit, aut impedierit quo minus opus sibi commissæ legationis adimpleat, inveniet me sui contemptus persecutorem et vindicem, ac si in coronam meam proditorie commisisset.’ ” Peter of Blois was archdeacon of Bath in the reign of Henry II. and wrote this letter from court to Richard, who succeeded Thomas Becket in the primacy, and held it from 1174 to 1183.]

1
Psal. lxxvii. 20.

1
Isa. iii. 5.

2
[Comp. Proverbs xi. 29.]

1
Præf. lib. v. Silv. [i.e. Statius, Publ. Papinius, Præf. lib. v. Sylvarum.] 1886.

1
Ἀρχιερει̑ς.

2
[T. C. i. 61. ap. Def. 300. al. 81. “The title of archbishop is only proper to our Saviour Christ, therefore no man may take that unto him. That it is proper to our Saviour Christ, appeareth by that which St. Peter saith, when he calleth him ἀρχιποιμένα. . . And in the Hebrews where he is called the great shepherd of the sheep, and in the Acts and Hebrews archleader of life and salvation, which . . . are proper titles of his mediation, and therefore cannot be without bold presumption applied unto any mortal man.” Whitgift, Def. ibid. “What name is more proper unto God than is this name God? and yet is the same also attributed unto man.”]

1
[T. C. ii. 408. “The Greek word signifying prince, which name he confesseth proper unto the civil magistrate; it must follow that the name of archbishop, which is as much as prince of bishops, breaketh upon the possession of the magistrate.”]

2
Hist. Eccles. lib. v. c. 8. [rather c. 9. in which he gives the synodical epistle of the council of Constantinople to the Roman synod, beginning Κυρίοις τιμιωτάτοις.]

3
[Τὴν σὴν ἁγιωσύνην. Cod. Justin. I. 1.] de summa Trinit. l. vii. [“tuæ sublimitatis.” ibid. I. 3.] de Episc. et cler. l. xxxiii.

4
[“Sacrosanctam hujus religiosissimæ civitatis ecclesiam . . . privilegia et honores omnes super episcoporum creationibus, et jure ante alios residendi, et cætera . . . habere sancimus.” Impp. Leo et Anthemius in Cod. Just. i. 2.] de sacros. Eccles. l. xvi.

5
[Conc. Nic. Canones Arabici, can. lvii. “Locus episcopi in oratione sit in summo sacelli sive capellæ intra locum altaris, ut qui sit pastor et gubernator: post eum sit archidiaconus ad latus dextrum, ut qui sit loco ejus, et præsit omnibus quæ ad orationem et ecclesiam pertinent: chorepiscopus autem sit post archidiaconum ad alterum latus, quia ipse etiam est loco episcopi super villas, et monasteria et sacerdotes villarum,” t. i. 472. ed. Hard. iv Conc. Carth. ad 398. can. 35. “Ut episcopus in ecclesia, et in consessu presbyterorum, sublimior sedeat. Intra domum vero, collegam se presbyterorum esse agnoscat.” t. i. 981.]

1
“They love to have the chief seats in the assemblies, and to be called of men Rabbi.” Matt. xxiii. 6, 7. [quoted in Adm. ap. Def. 57; Answ. 40. al. 15; T. C. i. 12. al. 24; Def. 71, 72.]

2
Ecclus. xlv. 7.

1
[Ign. ad Philadelph. Ep. interp. c. xi; ad Smyrn. c. x; Martyr. Ign. c. 3. ἐδεξιου̑ντο τὸν ἅγιον, διὰ τω̑ν ἐπισκόπων καὶ πρεσβυτέρων καὶ διακόνων, αἱ τη̑ς Ἀσίας πόλεις καὶ ἐκκλησίαι, πάντων ἐπειγομένων πρὸς αὐτόν. ap. Coteler. t. ii. 159.]

2
[Before the middle of the third century, vid. (e. g.) S. Cypr. Ep. 55. “Cyprianus Cornelio. Legi literas tuas, quas per Saturum fratrem nostrum acolythum misisti;” et Ep. 36. “Peregrinis sumptus suggeratis . . . misi . . . . per Naricum acolythum aliam portionem;” et S. Cornelius ap. Euseb. H. E. vi. 43. who reckons up ἀκολούθους δύο καὶ τεσσαράκοντα among the officers of the Roman church. Comp. Duarenus de Sacr. Eccles. Minist. i. 14. as quoted by Bingham, b. iii. c. iii. § 3. “Extat rescriptum cujusdam Pontificis ap. Gratianum, quo præcipitur Episcopis ut comites semper aliquos (duo Presbyteros et tres Diaconos) secum ducant famæ suspicionisque vitandæ causa. Can. Jubemus; de Consecr. dist. i. (p. 1868.) Vel ideo acoluthi appellati sunt, quod funus una cum canonicorum et ascetriarum cœtu comitarentur. Nam eos id munus obire solitos fuisse satis perspicuum est ex constitutione Justiniani: Novell. ix.” (c. iv. p. 136. ed. Gothofr.)]

3
Novel. vi. [c. 2. του̑το οὐκ ἂν ἔχοι πρέποντα λογισμὸν, τὸ μετὰ πλήθους (ὅπερ ἀνάγκη τὰς τω̑ν ἐπισκόπων ἔχειν θεραπείας) περινοστει̑ν . . . ἐν ξένῃ. vid. p. 18. ed. Gothofr. 1688.]

1
T. C. l. i. p. 126. [al. 98: ap. Whitg. Def. 451. “Another reason of this pomp and stateliness of the bishops was that which almost brought in all poison and popish corruption into the church, and that is a foolish emulation of the manners and fashions of the idolatrous nations. . . . Galerius Maximinus the emperor to the end that he might promote the idolatry and superstition whereunto he was addicted, chose of the choicest magistrates to be priests, and that they might be in great estimation gave each of them a train of men to follow them: and now the Christians and Christian emperors thinking that that would promote the Christian religion that promoted superstition, . . . . . endeavoured to make their bishops encounter and match with those idolatrous priests, and cause that they should not be inferior to them in wealth and outward pomp. Eusebius, lib. viii. cap. 15.” 14. ed. Reading. p. 399. Ἱερέας τε εἰδώλων κατὰ πάντα τόπον καὶ πόλιν· καὶ ἐπὶ τούτων ἑκάστης ἐπαρχίας ἀρχιερέα τω̑ν ἐν πολιτείαις ἑνά γέ τινα, μάλιστα τὸν ἐμϕανω̑ς διὰ πάσης ἐμπρέψαντα λειτουργίας, μετὰ στρατιωτικου̑ στίϕους καὶ δορυϕορίας ἐκτάσσων. ἀναιδήν τε πα̑σι γόησιν, ὡς ἂν εὐσεβέσι καὶ θεω̑ν προσϕιλέσι, ἡγεμονίας καὶ τὰς μεγίστας προνομίας δωρούμενος. Whitgift: “There is not one word, that any Christian prince took any example of him to do the like in Christianity. It rather appeareth that Maximinus did in this point imitate the Christians, who had their metropolitans, and one chief bishop in every province long before this time.” The conduct of Julian afterwards seems to warrant this conjecture.]

1
L. 12. C. de sacros. Eccles. [This is a law of Valentinian the Third and Marcian, ad 454, confirming all former church privileges, annulling encroachments, and especially enjoining the payment of allowances.] L. 5. ibid. [A law of Honorius and Theodosius ii. ad 412. “Placet . . . . præscribere, a quibus specialiter necessitatibus singularum urbium ecclesiæ habeantur immunes. Prima quippe illius usurpationis contumelia depellenda est: ne prædia usibus cælestium secretorum dedicata, sordidorum munerum fœce vexentur.”] L. 2. C. de Episc. et Cler. [A law of Constantius (ad 357.) reenacting former immunities, and extending them to the wives and families of clergymen.] L. 10. ibid. [Arcadius and Honorius, ad 398, enjoin on provincial officers immediate regard to all cases of sacrilege, and add, “Nec expectet (provinciæ moderator), ut episcopus injuriæ propriæ ultionem deposcat, cui sanctitas ignoscendi gloriam dereliquit. Sitque cunctis laudabile, factas atroces sacerdotibus aut ministris injurias veluti crimen publicum persequi, ac de talibus reis ultionem mereri.”]

1
Psal. l. 18.

2
Hos. ii. 8.

3
Psal. l. 10.

4
Job i. 21.

1
Mal. iii. 10.

2
Prov. iii. 9.

3
Seneca. [Epist. 95. p. 604. ed. Lipsii, Antwerp 1615. “Vis Deos propitiare? bonus esto. Satis istos coluit, quisquis imitatus est.”]

1
Mal. i. 8. [Comp. b. v. c. xxxiv. § 3; b. viii. c. i. § 5.]

2
[Compare b. v. c. lxxviii.]

3
“Because,” saith David, “I have a delight in the house of my God, therefore I have given thereunto of mine own both gold and silver to adorn it with.” 1 Chron. xxix. 3.

1
Psal. l. 13, 14.

2
Phil. iv. 18.

3
Psal. lxxii. 11.

4
Ver. 10.

5
Matt. ii. 11.

6
Matt. xxvi. 13.

7
John xv. 16.

1
[See E. P. b. v. c. lxxix. 14.]

2
Aug. c. 15. de Mendac. [t. vi. 437. “Sicut illud, Nolite cogitare de crastino: et, Nolite itaque cogitare quid manducetis, et quid bibatis, et quid induamini. Cum autem videmus et ipsum Dominum habuisse loculos, quo ea quæ dabantur mittebantur, ut servari possent ad usus pro tempore necessarios; et ipsos Apostolos procurasse multa fratrum indigentiæ, non solum in crastinum, sed etiam in prolixius tempus impendentis famis, sicut in Actis Apostolorum legimus; satis elucet illa præcepta sic intelligenda, ut nihil operis nostri temporalium adipiscendorum amore vel timore egestatis tanquam ea necessitate faciamus.”]

3
C. 12. qu. 1. c. 15 et 16. [“Futuram ecclesiam Apostoli in gentibus prævidebant: idcirco prædia in Judæa minime sunt adepti, sed pretia tantummodo ad fovendos egentes.” Decr. Grat. pars ii. causa xii. qu. 1. p. 958. can. “Futuram.” This decretal, ascribed to Miltiades, or Melchiades, who was bishop of Rome from ad 311 to 320, bears evident marks of having been composed long after Christianity had been established in the empire.]

1
[Decret. Grat. pars ii. caus. xiii. qu. 2. § Siquis irascitur. “Qui unum filium habet, putet Christum alterum; si duos habet, putet Christum tertium; si decem habet, undecimum Christum faciat; et suscipio.” From S. Aug. Serm. i. de Vita Clericorum, § 4: t. v. 1382.]

2
Prov. iii. 10.

3
Mal. iii. 10.

4
2 Chron. xxxi. 10. Tho. Waldensis, tom. i. [Doctrinale Fidei] lib. iv. c. 39. [and 40; quoting inter al. Wicliffe, Trialog. iv. § 18. (of which the title is, “Sæculares propter dotationem sunt puniendi.”) “Nos autem dicimus illis, quod nedum possunt auferre temporalia ab ecclesia habitudinaliter delinquente, nec solum quod illis licet hoc facere, sed quod debent sub pœna damnationis gehennæ, cum debent de sua stultitia pœnitere, et satisfacere pro peccato quo Christi ecclesiam macularunt.” fol. 131. ed. 1525. And, § 19. “Facilitatem autem faciendi hanc eleemosynam et debitum sic potes cognoscere. Constat ex regalibus regis Angliæ, quod decedente episcopo vel abbate, aut quocunque notabiliter dotato de Anglia, temporalia sua, ad denotandum regalia regis, cadere debent in manu sua, et non procedetur ad electionem, nisi obtenta regis licentia; nec habebuntur ab electo mortificata regni dominia, nisi rege noviter approbante. Contineat ergo se rex ab innovatione derelicti maximi progenitorum suorum, et in brevitate erit totum regnum purgatum a mortificatione stolida bonorum temporalium, quæ jam sunt in manu mortua.” fol. 132. The passages which Thomas of Walden produces in c. 39, do not occur in the copy of Wicliffe here quoted: but their tenor is exactly that of his whole argument. E. g. c. 18. fol. 129. “Dic, rogo, utrum sæculares sunt propter dotationem hujusmodi increpandi?” . . . “Tene firmiter et nullatenus dubites, quin temporales domini in isto graviter peccaverunt. . . Non solum cooperati sunt ad istam dotationem, sed multipliciter consenserunt.” . . . . . fol. 130. “Cito post ascensionem ejus, infra annum cccc ejus ordinationem præcipuam in dotando ecclesiam reversarunt, et per consequens Antichristum in deturpationem sponsæ suæ multipliciter procrearunt. Unde narrant Chronica, quod in dotatione ecclesiæ vox audita est in aere angelica, tunc temporis sic dicentis, Hodie effusum est venenum in ecclesiam sanctam Dei.” Compare the following, quoted by Walden from the Speculum militantis Ecclesiæ, cap. 9. Juxta prædicta, erubescerent Antichristus et sui maculare sacerdotes Christi contra ordinationem quam ipse fecit; et domini sæculares et alii fatui qui hic adjuvant Antichristum, erubescerent de sic adjuvando sicut erubescent in finali judicio; et iste pudor erit major pro dolore hypocrisis, quia dicunt in factis suis quod faciunt ista ob honorem Christi, quia Christus male instituit, et Domini sæculares emendant eum, sicut Imperator Romanus quando fecit sacerdotes suos dominos, ipse correxit statum Apostolorum super ordinationem Christi. Sed totum hoc sapit blasphemiam.” Among the errors of Wicliffe condemned at the Council of Constance, one head is, Contra dotationem Ecclesiæ; of which the following are specimens: “Domini temporales possunt licite auferre temporalia ab ecclesia delinquente.” “Non est major hæreticus vel Antichristus, quam ille qui docet quod licitum sacerdotibus et Levitis gratiæ sit dotari in possessionibus et temporalibus.” Quantum ad chartas et concessiones sæcularium dominionum patet quod clerus erubesceret inniti tam culpabili fundamento: quia in nullo valet humana concessio, nisi præhabita licentia a domino capitali: et cum non possint docere quod domini de hoc habeant licentiam a Christo, patet quod lege tam humana quam divina, est talis donatio stulta sentienda, et ita illicita et Catholicis respuenda.” Fasciculus, &c. Gratii: ed. Browne, p. 271.]

1
Gen. xxviii. 20-22.

1
[Apol. c. 39.]

2
[Justinian. Inst. II. i. 7. “Nullius autem sunt res sacræ et religiosæ, et sanctæ: quod enim divini juris est, id nullius in bonis est.”]

3
Num. xviii. 24-28.

4
Num. xxxi. [48-54.]

5
Heb. vii. 3.

6
Acts iv. 34.

1
2 Cor. viii. 5.

2
Acts xi. 30.

3
Acts xxi. 18. xii. 17.

4
Ἐπίσκοπον ἔχειν τω̑ν τη̑ς ἐκκλησίας πραγμάτων ἐξουσίαν, ὥστε διοικει̑ν εἰς πάντας δεομένους μετὰ πάσης εὐλαβείας καὶ ϕόβου Θεου̑. Can. 40. [Προστάττομεν ἐπίσκοπον ἐξουσίαν ἔχειν τω̑ν τη̑ς ἐκκλησίας πραγμάτων· εἰ γὰρ τὰς τιμίας τω̑ν ἀνθρώπων ψυχὰς αὐτῳ̑ πιστευτέον, πολλῳ̑ ἂν μα̑λλον δέοι ἐπὶ τω̑ν χρημάτων ἐντέλλεσθαι, ὥστε κατὰ τὴν αὐτου̑ ἐξουσίαν πάντα διοικει̑σθαι, καὶ τοι̑ς δεομένοις διὰ πρεσβυτέρων· καὶ διακόνων ἐπιχορηγει̑σθαι, μετὰ ϕόβου του̑ Θεου̑ καὶ πάσης εὐλαβείας· μεταλαμβάνειν δὲ καὶ αὐτὸν τω̑ν δεόντων (εἴγε δέοιτο) εἰς τὰς ἀναγκαίας αὐτου̑ χρείας, καὶ τω̑ν ἐπιξενουμένων ἀδελϕω̑ν, ὡς κατὰ μηδένα τρόπον αὐτοὺς ὑστερει̑σθαι. t. i. 20. ed. Hard.] et Conc. Antioch. [can. 25. ibid. p. 604, 5. ad 341.]

1
John iv. 24.

2
[Vid. supr. c. xiii. § 4.]

3
Heb. xi. 38.

4
[Comp. b. v. c. xv.]

1
Num. xviii. 15.

2
Ver. 12.

3
Ver. 13.

4
Ver. 15.

5
Ver. 8; Leviticus xxvii. 11, 14; Num. xviii. 8.

6
Ver. 8.

7
Ver. 9.

8
Ver. 21.

9
Ver. 28.

10
1 Chron. xxiii. 3.

1
Gen. xlvii. 22.

2
Num. xxxv. 7.

3
Josh. xiv. 4.

i
[So corrected, ed. 1676, 1682.]

4
Deut. xviii. 8; Lev. xxv. 33, 34.

1
Deut. x. 9.

2
Josh. xiii. 14.

3
Num. xviii. 24.

4
Ver. 19.

5
1 Cor. ix. 13.

6
1 Tim. v. 17.

7
2 Cor. iii. 7, 8. Vide [Tho. Aquin.] 2m. 2. [i. e. secundæ Summ. Theol. pars ii.] qu. 77. [87.] art. 1. [“Utrum homines teneantur dare decimas ex necessitate præcepti.”] . . . Respons. ad 1. [“Determinatio decimæ partis solvendæ est authoritate ecclesiæ tempore novæ legis instituta, secundum quandam humanitatem, ut scil. non minus populus novæ legis ministris novi testamenti exhiberet, quam populus veteris testamenti exhibebat: cum tamen populus novæ legis ad majora obligetur, secundum illud Matth. v. ‘Nisi abundaverit,’ &c.: et cum ministri novi testamenti sint majoris dignitatis quam ministri veteris testamenti, ut probat Apostolus 2 ad Corinth. iii.” fol. 205, ed. Venet. 1593.]

1
1 Tim. v. 18.

1
Acts iv. 35.

2
Acts ii. 44.

3
Dispens. [“Dispensator;” so called by] Prosper: [?] Julianus Pomerius † 493.] de Vita Contempl. l. ii. c. 12. [in Bibl. Patr. Colon. t. v. part. iii. p. 64.] “Œcon.” [“Œconomus”] L. 14. C. de sacr. Eccles. [Cod. Justin. lib. i. tit. 2. lex 14.] et Novel. vii. in princip. [μήτε ἀρχιεπίσκοπον . . . μήτε οἰκόνομον, πιπράσκειν . . . ἢ ἄλλως ἐκποιει̑ν πρα̑γμα ἀκίνητον.]

4
Prosp. de Vita Contempl. l. ii. c. 16. [p. 65. “Ut uno solicitudines omnium in sua societate viventium sustinente, omnes, qui sub eo sunt, fructuosa vacatione potiantur spiritualiter et quiete . . . . Etiam in hoc Deo serviunt, quia si Dei sunt ea quæ conferuntur ecclesiæ, Dei opus agit, qui res Deo consecratas non alicujus cupiditatis, sed fidelissimæ dispensationis intentione non deserit.”]

1
Cypr. l. iv. ep. 5. [34. p. 48. Baluz.] “Presbyterii honorem designasse nos illis jam sciatis, ut et sportulis iisdem cum presbyteris honorentur, et divisiones mensurnas æquatis quantitatibus partiantur, sessuri nobiscum provectis et corroboratis annis suis.” Which words of Cyprian do shew, that every presbyter had his standing allowance out of the church-treasury; that besides the same allowance called sportula, some also had their portion in that dividend which was the remainder of every month’s expense; thirdly, that out of the presbyters under him, the bishop as then had [a] certain number of the gravest, who lived and commoned always with him.

1
Prosp. [v. note 4, p. 296.] de Vita Contempl. l. ii. c. 9. [“Expedit facultates ecclesiæ possideri, et amore perfectionis proprias contemni. Non enim propriæ sunt, sed communes ecclesiæ facultates, et ideo quisquis omnibus quæ habuit dimissis aut venditis fit rei suæ contemptor, cum præpositus fuerit sanctæ ecclesiæ, omnium quæ habuit ecclesia efficitur dispensator. Deinde Sanctus Paulinus, ut ipsi melius nostis, ingentia prædia, quæ fuerunt sua, vendita pauperibus erogavit; sed cum postea factus esset episcopus, non contempsit ecclesiæ facultates, sed fidelissime dispensavit. Quo facto satis ostendit, et propria debere propter perfectionem contemni, et sine impedimento perfectionis posse quæ sunt communia ecclesiæ possideri.”]

2
[Ibid. “Quid S. Hilarius? nonne et ipse omnia bona sua aut parentibus reliquit, aut vendita pauperibus erogavit? Is tamen cum merito perfectionis suæ fieret ecclesiæ Arelatensis episcopus, quod illa tunc habebat ecclesia non solum possedit, sed etiam acceptis fidelium numerosis hæreditatibus ampliavit. Isti ergo tam sancti tam perfecti pontifices factis evidentibus clamant, posse et debere fieri quod fecerunt. Qui utique homines tam sæcularium quam divinarum literarum sine ambiguitate doctissimi, si scirent res ecclesiæ deberi contemni, nunquam ea debuerant, qui omnia sua reliquerant, retinere. Unde datur intelligi, quod tales ac tanti viri, (qui volentes esse Christi discipuli renunciaverunt omnibus quæ habebant) non ut possessores sed ut procuratores facultates ecclesiæ possidebant.”]

3
Pont. Diacon. in vita Cypr. [“Statim rapuit quod invenit promerendo Deo profuturum. Distractis rebus suis ad indigentiam pauperum sustentandam, tota prædia pretio dispensans, duo bona simul junxit, ut et ambitionem sæculi sperneret, qua perniciosius, nihil est; et misericordiam, quam Deus etiam sacrificiis suis prætulit, quam nec ille qui legis omnia mandata servasse se dixerat, fecit, impleret.” col. cxxxvi. ed. Baluz.]

4
[Capitul. Hincmar. Archiepisc. Remens. ad 852. series ii. cap. xvi. Concil. Hard. t. v. 396. “Ut ex decimis 4 portiones fiant juxta institutionem canonicam . . . et ut de duabus portionibus, ecclesiæ et episcopi, ratio reddatur per singulos annos, quid inde profecerit in ecclesia.” Conc. Namnet. temp. incert. can. x. “Instruendi sunt presbyteri pariterque admonendi quatenus noverint decimas et oblationes, quas a fidelibus accipiunt, pauperum et hospitum et peregrinorum esse stipendia, et non quasi suis sed quasi commendatis uti . . . . Qualiter vero dispensari debeant canones sancti instituunt; sc. ut 4 partes inde fiant; una ad fabricam ecclesiæ relevandam, altera pauperibus distribuenda, tertia presbytero cum suis clericis habenda, quarta episcopo reservanda, ut quidquid exinde jusserit prudenti consilio fiat.” t. vi. pars i. 459. The “Excerptiones” of Egbert, archbishop of York, ad 747. (t. iii. 1962.), the Canons of Charlemagne, cap. vii. (t. iv. 958.) those of Ælfric, can. xxiv. (vi. pars i. 982.); and those of a bishop of Basle, (can. xv. ib. 1243.) recognise a threefold division, considering the bishop’s portion and that of his clergy as one. The latter refers to 9 Conc. Tolet. can. vi. ad 655: which speaks of the bishop’s third as a received institution. (t. iii. 974.)]

1
[So the word stands in E. (Gauden’s edd.) It should be “their,” or some equivalent word.]

2
Lact. de Vera Sap. lib. iv. c. 30. [“li, quorum fides fuit lubrica, cum Deum nosse se et colere simularent, augendis opibus et honori studentes, affectabant maximum sacerdotium; et a potioribus victi, decedere cum suffragatoribus suis maluerunt, quam eos ferre præpositos, quibus concupierant ipsi ante præponi.”]

1
[Conc. Antioch. can. 25. t. i. 604. ad 341. Ἐπίσκοπον ἔχειν τω̑ν τη̑ς ἐκκλησίας πραγμάτων ἐξουσίαν, ὥστε διοικει̑ν εἰς πάντας τοὺς δεομένους μετὰ πάσης εὐλαβείας καὶ ϕόβου Θεου̑· μεταλαμβάνειν δὲ καὶ αὐτὸν τω̑ν δεόντων, (εἴγε δεοίτο,) εἰς τὰς ἀναγκαίας αὐτου̑ χρείας, καὶ τω̑ν παρ’ αὐτῳ̑ ἐπιξενουμένων ἀδελϕω̑ν, ὡς κατὰ μηδένα τρόπον αὐτοὺς ὑστερει̑σθαι, κατὰ τὸν θει̑ον ἀπόστολον λέγοντα· ἔχοντες διατροϕὰς καὶ σκεπάσματα, τούτοις ἀρκεσθησόμεθα· εἰ δὲ μὴ τούτοις ἀρκοι̑το, μετάβαλλοι δὲ τὰ πράγματα εἰς οἰκικὰς αὐτου̑ χρείας, καὶ τοὺς πόρους τη̑ς ἐκκλησίας ἢ τοὺς τω̑ν ἀγρω̑ν καρποὺς μὴ μετὰ γνώμης τω̑ν πρεσβυτέρων ἢ τω̑ν διακόνων χειρίζοι, ἀλλ’ οἰκείοις αὐτου̑ καὶ συγγενέσιν ἢ ἀδελϕοι̑ς ἢ υἱοι̑ς παράσχοιτο τὴν ἐξουσίαν, εἰς τὸ διὰ τω̑ν τοιούτων παραλεληθότως βλάπτεσθαι τοὺς λόγους τη̑ς ἐκκλησίας· του̑τον εὐθύνας παρέχειν τῃ̑ συνόδῳ τη̑ς ἐπαρχίας. 4 Conc. Carthagin. c. 14, 15. t. i. 980. ad 398. “Ut episcopus non longe ab ecclesia hospitiolum habeat:” “Ut episcopus vilem supellectilem, et mensam ac victum pauperem habeat, et dignitatis suæ auctoritatem fide ac vitæ meritis quærat.” T. C. i. 124. al. 97. ap. Whitg. Defence, 446, quotes these two canons, as also 3 Conc. Turon. c. 5. ad 813. t. 4. 1024. “Episcopum non oportet nimium profusis incumbere conviviis.”]

1
[Wicliff. Trial. iv. 16. fol. 127. “Non volet negare, quia oportet omnes fideles sequi Christum in moribus. Patet quod in gradu suo oportet clericos specialiter in paupertate humili sequi ipsum.” Penry, Humble Motion, p. 108. (1590.) “As touching the lord bishops and great clergymen, which have so laden themselves with thick clay, that they have much ado to get up into the pulpit of God; do they not know that it is their duty, that they may please Him who hath chosen them to be soldiers, not to entangle themselves with the affairs of this life, and that they ought for the peace and wealth of the Church to follow the example of their Lord and Master, &c. . . . then is it a small matter for them to leave their thousands and be content with their hundreds.”]

2
[Wicliff. fol. 126. “Christus dicit Matth. 8. ‘Quod filius hominis non habet,’ &c. hoc est, non habet humanitus sæculariter et proprietarie dominando. Qua ergo fronte episcopi nostri cæsarii audent in dominio civili se sic extollere super Christum?”]

3
[“Disputant, aliam hujus temporis sub Christi Evangelio esse rationem, ac olim fuit sub lege Mosaica: divitias et honores Deum priscis indulsisse sacerdotibus, at Evangelii ministros pauperes et inglorios vivere oportere, ad Christi Servatoris nostri exemplum et Apostolorum.” Saravia, de Hon. Præsul. et Presbyteris debito, c. 4. Vid. Wicliff. ubi sup.]

1
[“Sarcasmus est Juliani Apostatæ et hostium Christianæ religionis, deprædatis ecclesiarum opibus, doctoribus et pastoribus ecclesiarum insultare, ac dicere ipsos esse pauperes oportere, ad Christi Servatoris et Apostolorum ipsius exemplum.” Sar. de Hon. Præs. &c. c. 5. p. 90.]

2
1 Cor. xi. 1.

3
Phil. iii. 16.

1
Psalm vii. 8.

2
Epiph. contra Hæres. lib. iii. hær. 70. c. 1. [διαϕανής τις κατὰ τὴν ἑαυτου̑ πατρίδα, διὰ τὸ ἀκραιϕνὲς του̑ βίου, καὶ κατὰ Θεὸν ζηλου̑ καὶ πίστεως· ὃς πολλάκις θεώμενος τὰ ἐν ται̑ς ἐκκλησίαις γενόμενα, εἰς πρόσωπον ἐπισκόπων τε καὶ πρεσβυτέρων ἐλεγκτικω̑ς ἀντετίθει τοι̑ς τοιούτοις λέγων· οὐ χρὴ ταυ̑τα οὕτως γενέσθαι, οὐκ ὀϕείλει ταυ̑τα οὕτως πράττεσθαι· ὡς ἀνὴρ ἀληθεύων, καὶ ὁποι̑α ϕιλει̑ ὑπὸ τω̑ν ϕιλαληθω̑ς ἐλευθεροστομούντων ἀνδρω̑ν, τω̑ν μάλιστα τὸν βίον ἀκρότατα βιούντων . . . εἴτινα γὰρ εἰ̑δε τω̑ν ϕιλοχρηματούντων του̑ κλήρου, ἢ ἐπίσκοπον, ἢ πρεσβύτερον, ἢ ἕτερόν τινα του̑ κανόνος, πάντως ἐϕθέγγετο· καὶ εἰ ἑώρα τινὰ ἐν τρυϕῃ̑ καὶ σπατάλῃ, ἤ τινα παραχαράττοντα τὰ ἐν τῳ̑ ἐκκλησιαστικῳ̑ κηρύγματι καὶ θεσμῳ̑ τη̑ς ἐκκλησίας, μἠ ϕέρων ὁ ἀνὴρ, προεβάλλετο, ὡς ἔϕην, τὸν λόγον. καὶ ἠ̑ν του̑το τοι̑ς μὴ τὸν βιὸν δεδοκιμασμένον ἔχουσιν ἐπαχθές· ὑβρἰζετο δὲ ἕνεκα τούτου, καὶ ἀντελέγετο, ἐμισει̑το, ἔϕερε κλυδωνιζόμενος τε καὶ ὠθούμενος, καὶ ἀτιμαζόμενος, ἕως χρόνου ἱκανου̑ ἐν ται̑ς ἐκκλησίαις συναγόμενος, ἕως ὅτε δεινω̑ς ἐνεγκάντες τινὲς ἐξεου̑σι τὸν ἄνδρα διὰ τὴν τοιαύτην αἰτίαν· ὁ δὲ οὐκ ἠνείχετο, ἀλλ’ ἐβιάζετο μα̑λλον ἀλήθειαν μὲν λέγειν, μὴ ἀναχωρει̑ν δὲ του̑ συνδέσμου τη̑ς μια̑ς ἑνώσεως τη̑ς ἁγίας καθολικη̑ς ἐκκλησίας· ὡς δὲ ἐτύπτετο, . . . τὰ δει̑να τε ἔπασχε, βαρυστονήσας, σύμβουλον λαμβάνει ἑαυτῳ̑ τὴν ἀνάγκην τω̑ν ὑβρέων· ἑαυτὸν γὰρ ἀϕορίζει τη̑ς ἐκκλησίας.]

1
Ammian. Marcel. lib. xxvii. [c. iii. (ad 367.) “Damasus et Ursinus [Ursicinus] supra humanum modum ad rapiendam episcopatus sedem ardentes, scissis studiis asperrime conflictabantur, adusque mortis vulnerumque discrimina adjumentis utriusque progressis . . . constatque in basilica Sicinini, ubi ritus Christiani est conventiculum, uno die cxxxvii. reperta cadavera peremptorum; efferatamque diu plebem ægre postea delenitam.” p. 480. ed. Vales.]

2
Vide in Vita Greg. Naz. [p. 22. præfix. ed. Par. 1630.]

3
“Nemo gradum sacerdotii pretii venalitate mercetur; quantum quisque mereatur, non quantum dare sufficiat, æstimetur. Profecto enim, quis locus tutus et quæ causa esse poterit excusata, si veneranda Dei templa pecuniis expugnentur? Quem murum integritatis aut vallum [fidei] providebimus, si auri sacra fames in penetralia veneranda proserpat? quid denique cautum esse poterit aut securum, si sanctitas incorrupta corrumpatur? Cesset altaribus imminere profanus ardor avaritiæ, et a sacris adytis repellatur piaculare flagitium. Itaque castus et humilis nostris temporibus eligatur episcopus, ut quocunque locorum pervenerit, omnia vitæ propriæ integritate purificet. Nec pretio sed precibus ordinetur antistes.” L. 31. C. de Episc. et Cler. [Cod. Just. lib. i. tit. 3. lex 31.]

1
[Dr. Bridges, Def. of the Government, &c. p. 488, takes notice of a similar oversight: “With dutiful submission to their authority, we wish that some even of our bishops had been so careful in this long time that they had not admitted some though prettily learned yet too headstrong and newfangled ministers, that since they have entered into the ministry, forgetting the oath of their canonical obedience to their bishops, and of their loyal obedience to their prince, have, and do make, all, or the most part, of these stirs. But their carelessness in admitting such, hath been since meetly well punished by these their disobedient and unthankful children. And some also they have admitted into this function too unlearned (we confess) and unworthy ministers, and so are not altogether clear of maintaining the continual nurseries of ignorance and ignorant pastors. Yet neither have they been maintained, but greatly rebuked, for their so careless doings, and thereupon laws and provisions have been made, and stand in force, to repress such unlearned ministers, and the makers of them.”]

2
[Gibson, Codex, 784, note. “This writ lies, when one hath an advowson, and the parson dies, and another presents a clerk, or disturbs the rightful patron to present.” He gives the form of the writ.]

3
[Bp. Cooper, Adm. p. 147. “As for the corruption in bestowing other meaner livings, the chief fault thereof is in patrons themselves. For it is the usual manner of the most part of these (I speak of too good experience) though they may have good store of able men in the Universities, yet if an ambitious or greedy minister come not unto them to sue for the benefice, if there be an unsufficient man, or a corrupt person within two shires of them, whom they think they can draw to any composition for their own benefit, they will by one means or other find him out. And if the bishop shall make courtesy to admit him, some such shift shall be found by the law, either by Quare impedit or otherwise, that whether the bishop will or no, he shall be shifted into the benefice. I know some bishops unto whom such suits against the patrons have been more chargeable in one year, than they have gained by all the benefices they have bestowed since they were bishops, or I think will do while they be bishops.”]

1
Can. Apost. 76. [ap. Beveridge, Synodicon, i. 50. Οὐ χρὴ ἐπίσκοπον τῳ̑ ἀδελϕῳ̑, ἢ τῳ̑ υἱῳ̑, ἢ ἑτέρῳ συγγενει̑ χαριζόμενον, εἰς τὸ ἀξίωμα τη̑ς ἐπισκοπη̑ς χειροτονει̑ν ὃν βούλεται· κληρονόμους γὰρ τη̑ς ἐπισκοπη̑ς ποιει̑σθαι οὐ δίκαιον, τὰ του̑ Θεου̑ χαριζόμενον πάθει ἀνθρωπίνῳ. οὐ γὰρ τὴν του̑ Θεου̑ ἐκκλησίαν ὑπὸ κληρονόμους ὀϕείλει τιθέναι· εἰ δέ τις του̑το ποιήσει, ἄκυρος μὲν ἔστω ἡ χειροτονία· αὐτὸς δὲ ἐπιτιμάσθω ἀϕορισμῳ̑.]

1
[Sueton. Nero. c. 38.]

1
Egisip.* l. ii. c. 12. “[Nactus igitur Herodes regnum, quod a Romanis pro oppugnatæ vel proditæ patriæ mercede acceperat, in locum Antigoni . . . substituit successores in sacerdotium, non Asamonæi generis, quos clarioris fuisse prosapiæ accepimus, sed ignobiles quosque quos aut libido aut casus dedisset . . . In hujusmodi ordinationibus Archelaus secutus paternæ speciem consuetudinis, angustioris animi tenuit sententiam: more quodam insito mortalibus, ut apud eos minus suspecta sit ignavia hebetiorum, quam gratia bonorum.” in Bibl. P. Colon. II. 1003 F.]

*
[I. e. Hegesippus (or Josippus), de Bello Judaico: a compilation from Josephus in Latin, and in some MSS. ascribed to S. Ambrose. It was printed several times in the sixteenth century, from 1511 to 1583, and by Weber, Marburg, 1864. Vid. Cave, Hist. Lit. i. 216; Ceillier. ii. 2. § 5; Weiss in Biog. Univ.] 1886.

1
[See Theodoret. E. H. i. 11. Φιλαπεχθήμονες ἄνδρες ἐγράψαντο τω̑ν ἐπισκόπων τινὰς, καὶ τῳ̑ βασιλει̑ τὰς ἐγγράϕους κατηγορίας ἐπέδοσαν· ὁ δὲ πρὸ τη̑ς γεγενημένης ὁμονοίας ταύτας δεξάμενος, εἰ̑τα δεσμὸν ἐπιθεὶς καὶ τῳ̑ δακτυλίῳ σημηνάμενος, ϕυλαχθη̑ναι προσέταξεν· ἔπειτα τὴν σύμβασιν ἐργασάμενος, ταύτας κομίσας παρόντων αὐτω̑ν κατέκαυσεν, ὁμωμόκως ἠ̑ μὴν μηδὲν τω̑ν ἐγγεγραμμένων ἀνεγνωκέναι· οὐ γὰρ ἔϕη χρη̑ναι τω̑ν ἱερέων τὰ πλημμελήματα δη̑λα γίνεσθαι τοι̑ς πολλοι̑ς, ἵνα μὴ σκανδάλου πρόϕασιν ἐντευ̑θεν λάβοντες, ἀδεω̑ς ἁμαρτάνωσι· ϕασὶ δὲ αὐτὸν καὶ τόδε προσθει̑ναι, ὡς εἰ αὐτοπτὴς ἐπισκόπου γάμον ἀλλότριον διορύττοντος γίνοιτο, συγκάλυψαι ἂν τῳ̑ πορϕυρίῳ τὸ παρανόμως γινόμενον, ὡς ἂν μὴ βλάψῃ τοὺς θεωμένους τω̑ν δρωμένων ἡ ὄψις.]

1
[1 Tim. iv. 16.]

2
[Psalm v. 8.]

3
[c. xliv. 4. σοϕοὶ λόγοι ἐν παιδείᾳ αὐτω̑ν.]

1
2 Sam. xvi. 12.

2
Plat. in Phæd. [Μισανθρωπία ἐνδύεται ἐκ του̑ σϕόδρα τινὶ πιστευ̑σαι ἄνευ τέχνης, καὶ ἡγήσασθαι παντάπασί γε ἀληθη̑ εἰ̑ναι καὶ ὑγιη̑ καὶ πιστὸν τὸν ἄνθρωπον, ἔπειτα ὄλιγον ὕστερον εὑρει̑ν του̑τον πόνηρόν τε καὶ ἄπιστον, καὶ αὐ̑θις ἕτερον· καὶ ὅταν του̑το πολλάκις πάθῃ τις, καὶ ὑπὸ τούτων μάλιστα οὓς ἂν ἡγήσατο οἰκειοτάτους τε καὶ ἑταιροτάτους, τελευτω̑ν δὴ θαμὰ προσκρούων, μισει̑ τε πάντας, καὶ ἡγει̑ται οὐδένος οὐδὲν ὑγιὲς εἰ̑ναι τοπαράπαν . . . . καὶ δη̑λον ὅτι ἄνευ τέχνης τη̑ς περὶ τὰ ἀνθρώπεια ὁ τοιου̑τος χρη̑σθαι ἐπιχειρει̑ τοι̑ς ἀνθρώποις· εἰ γάρ που μετὰ τέχνης ἐχρη̑το, ὥσπερ ἔχει, οὕτως ἂν ἡγήσαιτο, τοὺς μὲν χρηστοὺς καὶ πονηροὺς σϕόδρα ὀλίγους εἰ̑ναι ἑκατέρους, τοὺς δὲ μεταξὺ, πλείστους. t. i. 89. c. ed. Serran.]

1
Merc. Trism. in Pimandro, dial. vi. [§ 3. ed. Patricii, Lond. 1611, fol. 14.]

2
Mal. iii. 8.

k
So corrected ed. 1676.

3
Acts v. 2.

1
Gen. xlvii. 22.

1
Numb. xviii. 32.

2
Lev. xxv. 34.

3
Ezek. xlviii. 14.

4
Habak. ii. 17.

5
Mal. iii. 9.

6
Prov. iii. 9.

7
2 Chron. xix.

l
So all Gauden’s edd.

1
[De Præscript. Hæret. c. xxx. “Agnosco maximam virtutem eorum, qua Apostolos in perversum æmulantur.”]

1
Psal. cv. 24, 25.

2
[See in Penry’s “Humble Motion,” p. 94, &c. a detailed plan for the redistribution of church property.]

3
[Cf. Elizabeth’s motto, Semper eadem.]

m
time and; all Gauden’s edd.

4
Lib. x. Ep. 54. DDD. Valent. Theodos. et Arcad. [p. 289. Paris. 1604.]

5
Cod. Just. I. 2. de Sacros. Eccles. l. 14.

1
[So 1662: bishopricks, 1676, ’82.

1
[c. viii. 12.]

2
[Justinian. Instit. lib. i. tit. viii. § 2. “Expedit enim reip. ne sua re quis male utatur.”]

3
“Pudet dicere, sacerdotes idolorum, aurigæ, mimi et scorta hæreditates capiunt, solis clericis et monachis id lege prohibetur, et prohibetur non a persecutoribus sed principibus Christianis. Nec de lege conqueror, sed doleo quod meruerimus hanc legem.” Ad Nepot. 2. [§ 6. t. i. 258. ed. Vallars.]

1
[Exod. xxxvi. 5-7.]

2
Obad. ver. 5.

3
Flor. lib. iii. c. 13. [23.]

1
Deut. xxxiii. 11.

*
[Q.L.C.D. stand for MSS. described vol. i. p. xliv. E. for the ed. 1651; see vol. i. p. xxxiii. There was an earlier ed. 1648, here marked E′. which was followed by Gauden, 1662.] 1886.

a
E. adds containing.

b
to E C.

†
[Archdeacon Cotton has transmitted to the editor, from a MS. [D. 3. 3.] in the library of Trinity College, Dublin, the following extract, being part of a kind of analysis of the eighth book, written out by Abp. Ussher as for his own use.

“Of Kings and their Power Ecclesiastical generally.

1. “An Admonition concerning Men’s Judgments about the Question of regal Power.
2. “What their Power of Dominion is.
3. “By what 1 Right, after what 2 Sort, in what 3 Measure, with what 4 Conveniency, and according to what 5 Example, Christian Kings may have it. In a word, their manner of holding Dominion.
“Of the Kings of England particularly.

4. “Of the Title of Headship, which we give to the Kings of England in relation unto the Church.
5. “Of their Prerogative to call general Assemblies about the affairs of the Church.
6. “Of their Power in making Ecclesiastical Laws.
7. “Of their Power in making Ecclesiastical Governors, (the chief Ministers of Ecclesiastical Jurisdiction).
8. “Of their Power in Judgment Ecclesiastical.
9. “Of their Exemption from Judicial kinds of Punishment‡ by the Clergy.”]
‡
[Censures Ecclesiastical written underneath this clause.]

c
in both Q.

1
1 Maccab. xiv. 42.

d
their E.Q.L.C.

e
over their works, and country, and weapons, but also, &c. E.

f
all om. E.

2
Vid. inf. c. iii. 1.

g
haply it be E.C.L.

h
to E.

i
only om. E.

k
Ezekias om. E.C.L.

l
the om. E.

m
matters E.Q.L.C.

n
officers C.L.

o
kings E.

p
things E.C.

q
prophets E.

r
nor E.C.L.

s
at any time could E.L. any time could C.

t
with D.

u
thing E.C. change Fulm.

x
the supremacy E.C.

y
even om. E.C.

z
high om. E.C.L.Q.

a
the E.C.

b
and E.Q.L.C.

c
do imagine E.C.

1
[1 Adm. ap. Whitg. Def. 694. “To these three jointly, i. e. to the ministers, seniors, and deacons, is the whole regiment of the Church to be committed.” Answ. ibid. “Methinks I hear you whisper that the prince hath no authority in ecclesiastical matters.” T. C. i. 153. al. 192. “The prince or civil magistrate hath to see that the laws of God touching his worship and touching all matters and orders of the Church be exercised and duly observed, and to see that every ecclesiastical person do that office whereunto he is appointed, and to punish those which fail in their office accordingly. As for the making of the orders and ceremonies of the Church, they do (where there is a constituted and ordered Church) pertain unto the ministers of the Church and to the ecclesiastical governors . . . But if those to whom that doth appertain make any orders not meet, the magistrate may and ought to hinder them and drive them to better.”]

d
callings E.C.

e
the om. D.

2
See below, c. ii. 1.

f
who E.Q.C.L.

g
functions E.

h
in om. D.

i
matter E.

k
the om. E.

kk
opinions E′.

l
the om. E.C.

ll
substance, ed. 1676, ’82.

m
that E.Q.C.L. which E′.

n
so being E.C.

o
E.Q.L.

p
every one of them E.

q
sever E. [swerve Fulm.]

r
most om. E.

s
through E.C.

t
Christian om. C.L.

u
E. reads the church and commonwealth are two corporations, independently subsisting [each Fulm.] by itself. The correction is made upon the authority of all the MSS.

x
man a om. E.C.

y
triangle E.

z
kind om. E.

a
the Church E.L.

b
to E.Q.C.

c
both. Nay, it is so E. The MSS. read as above, only C. omits so.

d
also to be E.C.

e
the om. E.Q.C.L.

ee
that—do between brackets E′.

f
the bishops E.

g
with E.

h
which does not hinder, om. E. inserted from Q.D.

i
a om. E.C.L.

1
[“Etsi duo sunt gubernationis genera, alterum civitatis, alterum ecclesiæ, tamen utrumque ab eodem profectum est auctore. Quod quamvis diversa fiat ratione, et illud a Deo sit quatenus Creator et Moderator rerum humanarum, hoc quatenus Redemptor est humani generis, et unumquodque suum habeat finem; tamen quando eadem societas ecclesia est et civitas, sicut ab eodem utriusque regiminis auctoritas manat, ita ad eundem postremum finem respicit, et eodem se recolligit. Unde fit, ut multa habeant communia, quæ nunquam recte nisi communi consilio et assensu possunt perfici. Evangelii minister a Deo Servatore regiminis in ecclesiam habet auctoritatem: magistratus a Deo omnium Creatore in cives. Qui quoties simul amice conspirant, et eodem sua consilia referunt, optime cum civitate, optime agitur cum ecclesia.” Saravia de Divers. Ministr. Grad. c. xi. p. 27.]

k
errors E.Q.C.L.

l
such as are E. [such; are Fulm.]

m
in om. E.D.

n
hath E.C.

o
dependencie D.

p
government E.

1
[“Dicunt, ecclesiam et rempublicam res distinctas esse: quod nos quidem fatemur nonnunquam personis et ratione fieri, nonnunquam ratione tantum. Ubi universa civitas aut resp. fidem Christi profitetur et amplectitur, ratione tantum differunt cives regni Dei, et reip. Ubi civitas et princeps est infidelis, ibi non ratio tantum, sed personæ civium utriusque regni diversæ sunt. Quo in loco diversæ possunt esse summæ jurisdictiones. Ubi vero eadem est resp. et ecclesia, minime hoc fieri potest. Cum igitur in nostro regno iidem sint cives regni Dei, et reip. una debet esse summa potestas, nisi subditorum et præsidum alia sit ratio.” Sutcliffe, de Presbyterio, p. 42.]

2
[Apol. doctissimi Viri D. Gulielmi Alani [i. e. Cardinal Allen, v. p. 92, note 3] pro Sacerdotibus Societatis Jesu, et aliis Seminariorum Alumnis: Augustæ Trevirorum, 1583. cap. iv. p. 64, 65: “Est error manifestus, omni eruditione tam humana quam divina damnatus, affirmare primatum in causis ecclesiasticis naturalibus aut Christianis legibus in potestate et jure regis temporalis includi, aut hujusmodi dignitatem unquam in gubernatorem aliquem civilem jure collatam aut conferri posse, sive is ethnicus, sive Christianus fuerit, asserere. . . Sub Nerone, præcipui Apostoli ecclesiam Romanam gubernabant: quemadmodum modo, ubi regna ab avita fide desciverunt, ecclesia suam spiritalem necessario habeat gubernationem, quæ a regibus ethnicis, quibus tamen in rebus sæcularibus obtemperant Christiani, minime dependeat. Quapropter omnia quæ a Protestantibus ex sacris literis adferuntur, non plus principi Christiano quam ethnico quoad hanc potestatem favent.”]

q
did bare D.

r
church’s E.C.

s
live she om. E.C.L.

t
spiritual E.

u
the om. E.C.L.

1
Polit. [lib. iii. cap. 6.] p. 102. [τὸ κοινῃ̑ συμϕέρον συνάγει, καθ’ ὅσον ἐπιβάλλει μέρος ἑκαστῳ̑ του̑ ζῃ̑ν καλω̑ς· μάλιστα μὲν οὐ̑ν του̑τ’ ἔστι τέλος, καὶ κοινῃ̑ πα̑σι, καὶ χωρίς· συνέρχονται δὲ καὶ του̑ ζῃ̑ν ἕνεκεν αὐτου̑.]

x
saith E.C.L.

y
the life E.

z
which the life hath need of E. as this life hath need of C.

zz
needeth E′.

2
S. Matt. vi. [33. This reference from Q.]

a
to be sought first for E.

b
sought E.

3
Arist. Pol. p. 196. [om. E. Q. C. L.]

c
heathens E.Q.C.L.

d
affairs E.

4
Arist. Pol. lib. iii. cap. 20. [123. l. 10. et 181. l. 28. D. vi. 8. t. iii. 566. c. ed. Duval. Ἀλλὸ δ’ εἰ̑δος ἐπιμελείας ἡ πρὸς τοὺς θεούς· οἱ̑ον, ἱερει̑ς τε καὶ ἐπιμεληταὶ τω̑ν περὶ τὰ ἱερά.] Liv. lib. i. c. 20. [“(Numa) sacerdotibus creandis animum adjecit . . . Pontificem . . . legi eique sacra omnia exscripta exsignataque attribuit . . . ut esset quo consultum plebis veniret.”]

e
states amongst D.

f
considerations E.Q.

g
selfsame E.L.

h
reasons E.Q.C.L.

i
there be E.

k
already [alleged Fulm.]

l E. reads, Three kinds of their proofs are [1. Fulm.] taken from the difference of affairs and offices. L. as in the text, only reading, officers for affairs D.C. and Q. give the reading above.
kk
the holy E′.

ll
commonweal in E′. throughout this §.

m
their E.C.L.Q.

n
were E.

o
or E.

p
evermore to be E.C.Q.

q
was that E. that was C.L.Q.

r
that E.Q.C.L.

s
dependence E.C.

t
the om. E.C.L.Q.

tt
the om. E′.

u
a church E.Q.C.L.

x
rest E.

y
continued E. corr. in 1662.

z
subsistencie D.

a
are in this case therefore E.Q.C.L.

b
as it liveth under E.C.L.

c
Jesus om. E.C.L.

d
forsomuch E.C.L.

e
so om. E.

1
2 Chron. xix. 8, 11; Heb. v. 1; 1 Thess. v. 12; T. C. iii. 151.

f
from D.

g
a little om. Q.

2
[“A true, sincere, and modest Defence of English Catholics that suffer for their faith both at home and abroad; against a false, seditious and slanderous libel, entitled ‘the Execution of Justice in England.’ ” c. v. p. 98, 99. “Though the state, regiment, policy and power temporal be in itself always of distinct nature, quality, and condition from the government ecclesiastical and spiritual commonwealth called the Church or body mystical of Christ, and the magistrate spiritual and civil divers and distinct, and sometime so far that the one hath no dependance of the other, nor subalternation to the other in respect of themselves, (as it is in the churches of God residing in heathen kingdoms, and was in the Apostles’ times under the pagan emperors,) yet now when the laws of Christ are received, and the bodies politic and mystical, the Church and civil state, the magistrate ecclesiastical and temporal, concur in their kinds together, (though ever of distinct regiments, natures and ends) there is such a concurrence and subordination betwixt both, that the inferior of the two (which is the civil state) must needs (in matters pertaining any way either directly or indirectly to the honour of God and benefit of the soul,) be subject to the spiritual, and take direction from the same. The condition of these two powers (as St. Gregory Nazianzen most excellently resembleth it) is like unto the distinct state of the spirit and body or flesh in a man. . . The spirit may and must command, overrule, and chastise the body . . . So likewise, the power political,” &c.]

h
temporal princes C.

1
[“Nor yet the spiritual turned into the temporal, or subject by perverse order (as it is now in England) to the same; but the civil, which indeed is the inferior, subordinate, and in some cases subject to the ecclesiastical; though so long as the temporal state is no hinderance to eternal felicity and the glory of Christ’s kingdom, the other intermeddleth not with his actions.” Allen, ubi supra.]

i
Christian om. E.C.L.

k Proofs . . . commonwealth om. E.
2
T. C. l. iii. p. 151.

l
opposed E.

m
both Church and commonwealth E.C.L.

n
which saith om. E. that saith Q.

3
Socr. lib. 5. præfat. Sozom. lib. 3. c. 20. [These two references from D.]

o
to be adjudged E.

p
and another E.Q.C.L.

4
Euseb. de Vita Constant. lib. iii. [c. 65.]

q
of om. E.C.L.

r
Church, doth E.Q.C.L.

5
Aug. Ep. 167. [al. 89.]

rr
not E′. edd. 1662, ’76. corr. 1682.

s
may om. E.C.

t
there om. E.Q.C.L.

tt
accident E′.

u
always should E.C.L.

x
lovingly dwell E.L.D.

y
be E.C.L.

yy
sometimes E′.

z
residence E.

a
in those accidents they are to be divers E.

aa
that om. E′.

b
inabstracted, betoken E.; so in 1662-82.

c
the accidents themselves E.Q.C.L.

d
the om. E.

e
those E.C.L.

f
in one man E.Q.C.

g
in divers E. not in E′.

h
these E. such C.L.

i
as E.Q.

k
that E.C.L.

l
therefore the Church E.C.L.

m
a om. E.

n
without any affairs; besides, when, E.

o
when ... church doth flourish om. E.C.L.

p
when in both [of] them, we then say E.

1
Isai. i. 21, 23.

q
justice and judgment E.Q.L. [The Geneva Bible, D. and C. read as in the text.]

2
Mal. i. 7, 8. [cf. VII. xxii. 4.]

r
evil Gen. Bible, E.Q.C.L.

s
you D.

t
treasure E.C.

3
1 Chron. xxix. 3.

u
bestowed E.

x
did E.C.L.

y
which E.C.L.

4
Nehem. ii. 17.

yy
Nehemiah E′.

5
Acts xviii. 14.

6
Ver. 15.

z
reciteth E.C.L. [rejecteth Fulm.]

a
thereof E.C. of those matters L. of those things, Gen. Bible.

b
this difference E.

c Proofs ... Church om. E.
d and released om. E. or released Q.C.
e
or E.

f
is not therefore E.Q. is therefore C.L.

g
from om. E.C.

1
T. C. l. iii. p. 152. [151. “A man may, by excommunication, be sundred from the Church, which forthwith loseth not of necessity his burgessship or freedom in the city, or commonwealth . . . The civil magistrate may by banishment cut off a man from being a member of the commonwealth, whom the Church cannot by and by cast out by excommunication . . . When one is for his misbehaviour deprived of his privileges both in the Church and commonwealth; albeit the Church be upon his repentance bound to receive him in again as member thereof, yet the commonwealth is at her liberty whether she will restore him or no.”]

gg
admit? If it chance the same man be shut out of both, division E′. 1666.

h
both, divisions E.

hh
execution. After E′.

i
none of E. none of is gone from L. cut off is gone from C. once that way gone from Q.

k
fit E. [still Fulm.]

l
that E.

m
a om. E.D.C.L.

n
state D.

o
very om. E.C.

p
which om. E.

q
it must E.

r
utterly separate E. sever C.

s
from the other also E.

t
may E.

u
do clearly E.

x
the affairs of om. E.C.L.

y
by E.

z
which E.C.L.

a
same om. E.

b
same E.C.

c
which having E.L.Q. which having been both D.A. man which hath both been C.

d
reunited E.L. received C.

e
admitted E.C.

f
both parts E.

g
these E.C.

h
received E.L. receives C.

i
man om. E.C.L.Q.

k
or E.L. on C.Q. E′. 1666.

kk
hath E′. 1666.

l
of these things in E. of those things of C.L.

m
said D.

n
the dominions E.C.L.

o
and E.C.L.

p
subjected E. submitted C.

q
dependence from E.C.L.

r
when he is suffered to rule E.C.L. where he, &c. Q.

s
all om. D. all did L.

t
thought fit E.

u
as it is a Church om. E.C.L.Q.

x
order thereby E.C.L.Q.

y
and to dispose E.C.L.

z
so far as E.C.L.

a
government E.C.

b
over the Church om. E.C.

c
and power E.C.

d
degrees E.C.L.Q.

dd
Law E′. 1666.

e
Head and om. E.C.L.Q.

f
E.C.L. insert “secondly, thirdly, fourthly, fifthly,” to mark the respective clauses of this sentence; to which C. and L. add (as would be correct) sixthly before the word according; but E. in that place has a full stop, for which in the current text and has been substituted. Q. notes the numbers in the margin. The whole stands here as in the Dubl. MSS.

g
in what inconveniency E. in what conveniency C.

h
generals E.C.

i
secondly, the prerogative, &c. E.C.L. (and so in the following clauses of this enumeration).

k
great E.

l
law E.

m
inevitable om. E.Q.C.L.

1
Luke xi. 17.

2
1 Cor. xiv. 40.

n
except E.C.L.

o
and E.C.L.

p
Yea om. E.C.L.Q.

pp
coaugmentation E′. 1666. corr. 1676.

q
to the other E.Q.

r
policy E.Q.C.L.

s
have E.C.L.

t
matters E.

u
so...extend om. D. do om. E.

x
an authority E.

y
cases E.C.L.

z
there om. E.C.L.

a
of all Dominion? om. E. 1666.

b
Besides—dominion om. E. him om. D.

c
order E. [Fulm. under.]

cc
old axiom E′.

d
et dominium om. E. potestatem, dominium C.

1
[Bracton (circ. 1244.) de Leg. the reading in the former quotation Angl. i. 8. fol. 5. ed. 1569; where is “dominationem et potestatem.”]

e
not om. E.C.

f
state E.

g
any thing hereunto to the contrary E.C.

h
not E.C.L.Q.

i
meant to exclude E. (Fulm. inserts “but”).

k
in E.

l
consisteth E.

m
of om. E.

n
On the authority of the Dublin MS. confirmed by internal evidence, the section headed, “By what rule,” is omitted here, and inserted § 17. Of this arrangement a relic remains in E.Q. and L. viz. the marginal note, “By what rule,” inserted in that place, without any section to which it might refer. Fulm. notes in the margin there, “des.” which probably means “desunt [quædam].”

o The right which men give, God ratifies, E. In Q., on a separate paper, in another hand, (perhaps Bishop Barlow’s,) this side-notestands thus: “By what right kings hold supreme power over causes ecclesiastical in their own dominions; namely, though such as men do give, yet God doth ratify.”
p
which E.

q
against the other E.Q.C.L.

r
must E.Q.

s
God supreme E.Q.C.L.

t
band E.C.L.

u
full om. E.

1
[Comp. Allen, Apol. c. iv. p. 67. “Oportet ecclesiam . . . illam retinere et conservare gubernandi rationem, quam Christus ipse immediate instituit, quamque nec elegit nec ordinavit populi decretum et consensus qui origo omnium statuum humanorum est et formarum politiæ.” It is the principle of the Roman law: “Quod principi placuit, legis habet vigorem: utpote cum lege regia, quæ de ejus imperio lata est, populus ei et in eum omne suum imperium et potestatem conferat.” Dig. i. iv. 1.]

x
kinds D.

y
society E.C.L.

z
they E. he C.L.

a
unto E.Q.C.L.

b
to the E.C. to that Q.

c
Sometime D.

d
which E.C.

e
have it om. E.C.L. [which insert it after “God.”]

f
freely E.

g
governors E.C.L.Q.

h
estates E.C.

i
for D. [see p. 346, line 5.]

2
Dan. ii.21.iv; Is. xlv; Rom.xiii.

k
power which they have to be his E.C.L.Q.

3
“Corona est potestas delegata a Deo.” Bracton. [The editor has not been able to find these words in the Book De Legibus Angliæ, but the sentiment occurs continually. E. g. ed. 1569, fol. 1. “Rex vicarius Dei;” et fol. 5. “Quod sub lege esse debeat, cum sit Dei vicarius, evidenter apparet ad similitudinem Jesu Christi, cujus vices gerit in terris;” and fol. 55. Habet omnia jura in manu sua, quæ ad coronam et laicalem pertinent potestatem . . . ut ex jurisdictione sua, sicut Dei minister et vicarius, tribuat unicuique quod suum fuerit. . . . Est enim corona regis facere justitiam et judicium, et tenere pacem;” and fol. 107. lib. iii. cap. 9. throughout.] “ ‘Rex’ ” “(inquit Sthenidas [Ecphantus] Locrus de Regno) τὸ μὲν [σκα̑νος] τοι̑ς λοιποι̑ς ὅμοιος οἵα γεγονὼς ἐκ τα̑ς αὐτα̑ς ὕλας ὑπὸ τεχνίτα δ’ εἰργασμένος λώστω ὃς ἐτεχνίτευσεν αὐτὸν ἀρχετύπῳ χρώμενος ἑαυτῳ̑.” [Ap. Stobæum, ii. 321. ed. Gaisford*.]

*
E. and C. omit this note; L. gives the following version. “A king, in regard of the tabernacle of his body, is like to other men, as made of the same matter, but fashioned by the best workman, who artificially framed him, using himself for the pattern.” The word σκα̑νος therefore seems to have been inadvertently omitted by the copyist. It may be questioned, however, whether this version be Hooker’s. In MS. D. a space is left here.

l
even om. E.Q.C.L.

m
own om. E.C.L.

n
hath E.Q.C.L.

o
the om. E.C.

p
the thing om. E. [Fulm. f. the power.]

q
assigned and established E.

r
affairs E.

s
by E.C.

t
the om. E.

u
the om. E.

x
being now made E.Q.C.L.

y
proposed E.

z
states E. estates C.

a
laws E.

b
so by the same law they E.

c
state E.

d
hath E.Q.C.L.

e
power and authority E. f. power over C.

f
the kingly E.

g
right om. E.Q.C.L.

h
place of estate E. the place of state C. that place Q.

i
right to exact E.Q.C.L.

j
and E.Q.C.L.

k
the om. E.

l Inserted from D.Q.L.
m
behoveth further E.

n
in E.Q.C.L.

o
of E.

1
[Vindic. contr. Tyr. p. 63, 65. “Cum de universo populo loquimur, intelligimus eos qui a populo auctoritatem acceperunt, magistratus . . . intelligimus etiam comitia, quæ nil aliud sunt, quam regni cujusque epitome, ad quæ publica omnia negotia referuntur . . . Illi vero ut singuli rege inferiores sunt, ita universi superiores.”]

p
herein om. E.Q.C.L.

q
injury done unto him E.Q.C.L.

r
not om. E.C.L.

s
the same E.C.L.

t
μετὰ E. Fulm. by birth, μετὰ L. μεία E′. both Greek words om. C.

u
is capital, and E. is capable, and C.

1
Junius Brutus, Vindic. p. 83. [“Vindiciæ contra Tyrannos, sive, de Principis in Populum Populique in Principem legitima Potestate; Stephano Junio Bruto, Celta, sive, ut putatur, Theodoro Beza, auctore.” P. 112, ed. Amstelod, 1660. “Etsi, ex quo virtutem patrum imitati filii nepotesve regna sibi quasi hæreditaria fecisse videntur, in quibusdam regionibus electionis libera facultas desiisse quodammodo videatur; mansit tamen perpetuo in omnibus regnis bene constitutis ea consuetudo, ut demortuis non prius succederent liberi, quam a populo quasi de novo constituerentur; nec tanquam sui hæredes patribus agnascerentur, sed tum demum reges censerentur, cum ab iis, qui populi majestatem repræsentarent, regni investituram quasi per sceptrum et diadema accepissent.” The first edition of this work bears date 1579. It appears by the prefixed epistle to have been completed 1577: and from internal evidence to have been written soon after the coronation of the Duke of Anjou (afterwards Henry III.) as king of Poland. See p. 223, ed. 1660; and compare a dissertation by Le Clerc at the end of Bayle’s Dictionary, Eng. Transl. 1734, in which, from this and other circumstances, he seems to have established in opposition to Bayle that Du Plessis Mornay, not Hubert Languet, was the probable author of the Vindiciæ. Sutcliffe in his Answer to the Petition to the Queen, 1591, mentions it repeatedly as the work either of Beza or Hotoman: p. 75, 79, 81. Dr. Mac Crie in his life of Melville, p. 425 (Edinb. 1819), says that the Vindiciæ is properly an enlargement of Beza’s suppressed treatise of De Jure Magistratuum. This, Mr. Gibbings suggests, may be the reason why Hooker seems to have been inclined to ascribe the book to Beza: see above, Editor’s Preface, p. xxii. At one time it was ascribed to the Jesuit Saunders: see Bancroft, Survey, c. 22. It is an essay to settle four questions: 1. “An subditi teneantur aut debeant principibus obedire, si quid contra legem Dei imperent.” 2. “An liceat resistere principi, legem Dei abrogare volenti, ecclesiamve vastanti. Item quibus, quomodo et quatenus.” 3. “An et quatenus principi remp. aut opprimenti aut perdenti resistere liceat. Item, quibus id, quomodo, et quo jure permissum sit.” 4. “An jure possint aut debeant vicini principes auxilium ferre aliorum principum subditis, religionis puræ causa afflictis, aut manifesta tyrannide oppressis.”]

x
that om. E.

y
deceased om. E.C.L.

z
king’s E.

a
sceptre and a diadem E.

1
Junius Brutus, Vindic. p. 85. [116. “In summa: omnes omnino reges ab initio electi fuerunt. Et qui hodie per successionem regnum adire videntur, prius a populo constituantur necesse est. Denique etsi populus ob egregia quædam merita ex aliqua stirpe reges sibi deligere in quibusdam regionibus solet; stirpem ipsam, non surculum deligit; nec ita deligit, quin, si degeneret, aliam eligere non [?] possit. Qui vero ex ea stirpe etiam proximi sunt, non tam reges nascuntur, quam fiunt; non tam reges, quam regum candidati habentur.” p. 81. [110.] “Si stirpem spectas, hæreditarium certe fuisse; at sane si personas, omnino electivum.”]

b
that E.C.L.

c
we E.C.L.

d
themselves om. E.C.L.

e
selected E.

2
Page 78. [105, &c.]

f
and E.Q.C.L.

3
[See this subject treated of at large by Dr. Saravia, “De Imperandi Auctoritate, et Christiana Obedientia,” lib. iii. cap. 1-17; against William Reynolds, of Rheims, who had maintained the contrary doctrine on the part of a Roman Catholic clergy and people in his work, “De Reip. Christianæ Potestate super Reges,” published 1592, under the name of G. Gul. Rossæus. It appears to have been the standing doctrine of the extreme papal party in their contentions with the imperialists.]

ff
given with E′.

g
and sceptres om. E.Q.C.L.

gg
have. I say these om. E′.

h
specified E.C.L.

i
either om. E.Q.C.L.

k
And om. E.

l
doth E.Q.C.

m
all these new elections and investings E.Q.C.L.

n
in E.Q.C.L.

o
entire om. E. inserted in C. by an after hand.

p
have rule Q.C.L. E′. om. E.

q
the D.E.C.L. [Q. reads his.]

r
of his sovereign E.C.L.

s
in om. E.Q.C.L.

t
from E.C.

u
into D.

x
births E.Q.C.L.

y
incapable D.C.L.

z
is om. E. which gives the whole sentence in italics.

1
Vide Cicer. de Offic. [ii. 12.]

a
it did always flow by original E.

b
the cause of kings’ E.

c
lords E.C.

d
fall unto them by escheat E.C.

e
follow rightly E. rightly om. C.

f
and E.Q.C.L.

g
the om. E.Q.C.L.

h
may a body politic then E.C.

i
the E.C.L.

k
inconveniences do E.Q.C. conveniences do L.

l
by any just means should be able E.

m
the line of E. [underscored by Fulm.]

n
lawfully om. E.Q.C.L.

o
is to E.Q.C.L.

p
only the articles E.C.L.

q
do by little and little growE. do grow by little, &c. Q.C.L.

r
most om. E.

s
to be om. E.

1
Arist. Pol. lib. iii. cap. 1*. [cap. 10. E. cap. 16, ed. Duval, t. iii. 477. B. βασιλείας μὲν οὐ̑ν εἴδη ταυ̑τα, τέτταρα τὸν ἀριθμόν· μία μὲν, ἡ περὶ τοὺς ἡρωϊκοὺς χρόνους· αὕτη δ’ ἠ̑ν έκόντων μὲν, ἐπί τισι δ’ ὡρισμένοις· στρατηγὸς γὰρ ἠ̑ν καὶ δικαστὴς ὁ βασιλεὺς, καὶ τω̑ν πρὸς τοὺς θεοὺς κύριος.]

*
Pol. l. i. c. 10. D.

ss
sustained, and endued with . . E′. and edd.

2
Pythagoras apud Ecphant. de Regno. Ὁ κατ’ ἀρετὰν ἐξάρχων καλέεται [τε] βασιλεὺς, καὶ ἔντι, ταύταν ἔχων ϕιλίαν τε καὶ κοινωνίαν ποτὶ τὼς ὑπὸ αὐτὸν, ἅνπερ ὁ Θεὸς ἔχει ποτί τε τὸν κόσμον καὶ τὰ ἐν αὐτῳ̑. [ap. Stob. Floril. ii. 323. ed. Gaisford.] “He that ruleth according to virtue is called a king, and hath such friendship and community towards those that be under him, as God hath towards the world and those things that be in it†.”

†
This extract is wanting in E; the Greek in C; the English in D.Q.L.

t
judge E.C.L.

u
that D.

x
scape D.

y
another E.Q.C.L.

3
Polit. iii. 14. [Ἡ ἐν τῃ̑ Λακωνικῃ̑ πολιτείᾳ δοκει̑ μὲν εἰ̑ναι βασιλεία μάλιστα τω̑ν κατὰ νόμον, οὐκ ἔστι δὲ κυρία πάντων· ἀλλ’ ὅταν ἐξέλθῃ τὴν χώραν, ἡγεμὼν ἔστι τω̑ν πρὸς τὸν πόλεμον. ἔτι δὲ τὰ πρὸς τοὺς θεοὺς ἀποδέδοται τοι̑ς βασιλευ̑σιν. comp. c. 15. init.]

z
have E.Q.C.L.

a
they were most tied to law, and so [had C.L.Q.] the most restrained power E.C.L.Q.

b
very om. E.C.L.

c
the state E.C.L.

d
αὐτῳ̑ μὲν, and afterwards δὲ, om. E.C.

1
[The margin of the Queen’s Coll. MS. has here, “Ecphantus Pythagoricus.” Vid. Stob. Floril. ed. Gaisford, II. 326. The whole passage is, Ὅπερ ἔντι μὲν τῳ̑ Θεῳ̑, ἕντι καὶ τῳ̑ βασιλει̑, αὐτῳ̑ μὲν ἄρχειν (ἀϕ’ ὠ̑περ καὶ ὁ αὐτάρκης καλέεται) ἄρχεσθαι δ’ ὑπ’ οὐδενός.]

e
predominancy E.

f
always om. E.C.L.

g
E.Q.C. insert “both for them and the people,” as does L, repeating “best” before that clause.

h
The reading of C. here is, “I mean not only the law of nature and the law of God, but the national consent thereunto.” Q, as in the text, omitting “very.” L and E, “I mean not only the law of nature and of God, but the national law consonant thereunto.” The text is from D.

hh
hand ins. E′.

hhh
Happier—science ital. E′.

i
and degrees ins. E.Q.C.L.

j
every of which E. later of which Q.C.L.

k
another E.C.L.

l
δὲ E.

m
ἀπόλυπος E.

n
ἡ δὲ ὅλη E.Q. [C. omits the Greek.]

2
[Ap. Stob. Floril. II. 166.] “The king ruling by law, the magistrate following, the subject free, and the whole society happy*.”]

*
This English in text of E. om. D. in marg. Q.C.L.

o
a king E.C.

p
grows D.

q
by him, or commanded E.

r
unto E.Q.L.

1
[Καὶ τούτων παραβάσει μὲν βασιλευς, τύραννος· ὁ δὲ ἄρχων, ἀνακόλουθος· ὁ δ’ ἀρχόμενος, δου̑λος· ἁ δ’ ὅλα κοινωνία, κακοδαίμων. Id. ibid.]

s
the E.C.

t
manner of person E.Q.C.L.

u
the om. D.

x
are to take E.C.

y
crown E.

z
pointed E.C.L.

a
same E.Q.C.L.

b
power and authority E.Q.C.L.

c
and om. E.

d
military dominion E.Q.C.L.

e
and E.Q.L.

f
such om. E.C.L.

g
may lawfully E.C.

2
Stapl. de Doct. Princip. [Controv. II.] lib. v. c. 17. [“Non negatur principi, magistratui, vel communitati potestas, perversa docentes corporali pœna puniendi, legesque pro ecclesiæ pace ferendi, dogmata promulgandi, defendendi, et contra violatores vindicandi.” p. 189. Paris, 1579.]

h
punishments D.

i
from violation om. E. which inserts it after themselves.

j
the very E.

3
Choppin. [René Chopin, 1537-1606.] de Sacra Politia forensi. Par. 1577, and 1589. Præfat. [This reference is from the Dubl. MS. Hooker quotes from the dedication of the edition of 1589, addressed to cardinal Bourbon under the name of Charles X. “Regium istud est, civiliumque magistrorum munus, ecclesiæ decreta tueri, conservare, tum latis legibus omnes sacris addictos continere in officio; urgendos etiam ad canonum ecclesiasticorum veterisque cultum disciplinæ, principali non minus auctoritate quam pontificali. . . . Laudatus est enim vel ex eo Joas Hebræorum rex, quod metuens ne sacerdotes nummos interverterent, qui offerebantur a populo ad tutelam templi, eos primum in arcam clausam inferri jussisset, deinde scriba suo præsente fabris ac cæmentariis erogari. Sed longe augustius illud, Christianæque utilius reip. regem ipsius adeo religionis cultusque divini custodem se profiteri, nedum sacri ærarii: qualem se gessisse Constantinum Magnum accepimus, et Galliæ tuæ heroas præstantissimos plerosque.” The writer was a lawyer of eminence in the parliament of Paris, and a vehement partisan of the League. v. Biog. Univ.]

k
rule E.

kk
Joash E′. 1666.

l
was E.Q.

m
treasure E.

n
whole om. E.Q.C.L.

o
of om. E.Q.C.L.

p
God’s ecclesiastical E.Q.C.L.

q
fear E.Q.C.L.

r
thing E.C.

s
themselves E.C.L.

t
never hereunto E.Q.C.L.

u
discourse E.C.L.

x
to be far E.

1
T. C. lib. i. p. 192. [154.]

y
that om. E.

z
all orders E.C.

a
to be E.

b
that om. E.

2
*Fenner’s “Defence of the godly Ministers [against the slanders of D. Bridges, contained in his answer to the preface before the Discourse of Ecclesiastical Government.” 1587. Sign. E. 1.]

*
Farmer’s E.C. Fennar’s D. Fermor’s Q. Fenner’s L.

c
do om. E.

d
towards E.

3
Humble Motion, p. 63.

e
which om. E.Q.C.L.

f
and preeminence E.C.

g
insomuch E.Q.C.L.

h
no kind of motion om. E.

4
Cicero, lib. i. de Nat. Deor. [c. 44. “Posidonius disseruit in libro quinto, nullos esse deos, Epicuro videri; quæque is de diis immortalibus dixerit, invidiæ detestandæ gratia dixisse: neque enim tam desipiens fuisset, ut fingeret . . . omnino nihil curantem, nihil agentem . . . Re tollit, oratione relinquit, deos.” Lactant. Epit. 36. “Verbo reliquisti, re sustulisti.”]

k
and om. E.

l
those D.

m
afore alleged E′. grant E.C. ground Q.

n
shew E. some C.

o
opinion E.Q.C.L.

p
the establishment E.Q.C.L.

q
that such words only sound towards all kind of fulnes of power E. All the MSS. read as in the text, except that C has a kind of fulness of power.

r
above E.

s
only E.

t
thereunto E.Q.C.L.

u
kinds Q.L. In the margin of E. Kinds stand here, as if the title of a section; perhaps by the printer’s mistake, from its being inserted in his copy as a probable emendation.

x
power and order, and of spiritual jurisdiction E. the power of order and of spiritual jurisdiction C.L.

y
hath E. which has no stop at joined.

z
inseparably E.Q.C.L.

a
those E.Q.C.L.

b
proceeding E.

c
such E.C. some L.

d
wish’t D.

e
E. omits not.

f om. D.
1
[Vid. supra, § 2, 3.]

g
in om. E.Q.C.L.

h
sundry great things E.Q.C.L.

i
and by command E.

j
the law doth so much E.

k
sometimes E.

l
the king, &c. (as a quotation) E.

m
right E.C.L.

n
his lords and commons in parliament E.C.

o
of om. E.

p
either divine Q.

q
human L.

r
the kings in themselves C.

s
not any . . . both, but om. E.

t
have a privilege therein E.Q.C.L.

u
restrain E.Q.C.L.

x
laws E.Q.L.

y
not E.

z
in E.C.L.

a
supernatural E.

b
church, whether E. church; where even C.

1
Ambros. Ep. 32. d. 160*. [II. 873. N. B. The word “bonus” is not in the MSS. of St. Ambrose.]

*
This marginal reference from C.

c
proceeding E.Q.C.L.

d
spiritual om. E.Q.C.L.

e
business E.C.L.

f
liberty E.Q.C.L.

g
C. has that church.

h
never E.

i
alter it; yet E. (Fulm. yea.)

j
supreme om. E.

k
any of om. E.Q.

l
throughout all E.Q.L. throughout the whole C.

m
law E. laws Q.C.L.

n
ways E.Q.C.L.

o
so om. E.

p
so om. E.

q
determined to be E.Q.C.L. [The phrase of the statute is, “adjudged to be.”]

1
An. 1. Reg. Eliz. [1 Eliz. c. 1. § 36.]

r
that in E.Q.C.L.

s
four first E.C.

t
churches, and evermore E.

u
making E.L.

x
it D.

y
that E.

2
[“It hath been generally holden that although the high commission court was abolished by the statute of 16 Car. I. c. 11, yet these rules will be good directions to ecclesiastical courts in relation to heresy.” 1 Hawkins, 4. ap. Burn. Eccl. Law, II. 277. ed. 1788.]

z This marginal note om. E.C. [Fulm. For what inconveniency.]
zz
case E′. 1662.

a
into E.C.L.

b
inconvenience E.L.

c
hurts E.

d
themselves E.

e
the procurement E.C.

f
invested E. arrayed marg. Q.

g
prerogative honour E.

h
as part to the whole perfection E. as part of L.C. as part to Q.

i
yet om. E.C.L.

k
E inserts be here, and omits it after societies.

l
the om. E.

1
Ob utilitatem publicam Reip. per unum consuli oportere, prudentissimi jurisconsulti docuerunt*. Just. Dig. i. 2. de Orig. Juris. 2. § 11†. [quoted in substance. The words are, “Novissime, sicut ad pauciores juris constituendi via transisse ipsis rebus dictantibus videbatur, per partes evenit, ut necesse esset reip. per unum consuli: nam senatus non perinde omnes provincias probe gerere poterat. Igitur constituto Principe, datum est ei jus, ut quod constituisset, ratum esset.”]

*
docent E.C.L. Jurisconsulti . . Just. Dig. om. E′.

†
L. ii. § novissimè. ¶ de orig. Juris. D. E′. adds Civilis.

m
all in E.

n
inconveniency E.

o
have E.C.L.Q.

p
these E.Q.C.L.

q
one’s E.C.

r
then in such case thou shouldest be E.

s or pattern om. E.C.L.Q.
t
comparable E.C.

u
policy E.

x
the Israelites E.

y
chosen E.

z
himself om. E.C.L.

a
from E.C.L.

b
of E.C.L.

c
power om. E.

d
First, unless E.C.

1
Staplet[on] de Princ. Doct. p. 197. [194. Opp. t. i. Controv. 11.] lib. v. c. 22. “Primum, ut Judæorum sacerdotium imperfectius erat, quia umbraticum tantum ac melioris præfigurativum, suoque tempore in melius commutandum: sic ipsius sacerdotii regimen imperfectius fuit*, ut illud viz. etiam aliqua ex parte ad Reges pertinere non incongrue posset.”

*
This quotation om. E.

e
that the Jews [first C.] religion E.C.

f
is om. E.

2
Stapl. ibid. [“Rursum, sacerdotium vetus habuit suas leges, sacrificia, ritus, et cæremonias omnes a Moyse præscriptas atque conscriptas, quibus nefas erat vel addere vel detrahere quicquam: ut hic nulla fere alia re opus esset, quam præscriptos cultus et leges executioni mandare; in quo genere Reges concurrere commodissime possent. Nam ardua et sublimiora fidei mysteria, quæ sacerdotum judicia maxime desiderarent, nondum erant necessario ab omnibus explicite credenda, sed tantum a majoribus, a cæteris autem in fide majorum . . . At in ecclesia Christi et quam plurima accesserunt mysteria explicite credenda, etiam a minoribus et vulgo fidelium . . . et præterea cultus divini externique regiminis ratio, ritus, ac cæremoniæ, scriptæ omnino non fuerunt.”]

g
and their E.C.L. Rights E′.

h
so generally om. Q.

i
government E.C.L.

k
our E.Q.C.L.

1
Idem ibid. [“Tertio, synagogæ disciplina erat gladius, et pœnæ temporales . . . Ut totus ille status servorum erat, non filiorum; sic terrore et externis pœnis, non amore et spiritualibus pœnis ducebantur. ‘Quod enim tunc fiebat gladio, lapidationibus, aliisque corporeis censuris, illud’ (ait Augustinus*) ‘degradationibus et excommunicationibus faciendum esse significatum est hoc tempore; cum in ecclesiæ disciplina visibilis fuerit gladius cessaturus.’ Hæc ille. Hinc ergo factum est, ut propter disciplinam illam corporalem, et visibilis gladii, qui in manu regum erat, reges ipsi causis ecclesiasticis non solum pie, sed etiam necessario sese nonnihil immiscuerint. Nunc vero, cum visibilis gladius non pertineat amplius ad disciplinam ecclesiæ, ut docuit Augustinus, datur intelligi non amplius ad reges disciplinam ecclesiæ et regimen pertinere; sed ad illos tantum quorum est ligare et solvere, et cætera.”]

*
[De Fid. et Oper. c. 3.]

l
as E.

m
the E.Q.C.L.

n
and om. E.

o
now om. E.

2
Stapl. ibid. [“Quarto, cum synagoga vetus in uno populo concluderetur, et in uno loco sub illo sacrificaretur, non erat incommodum, ut uni quoque regi synagogæ cura magna ex parte committeretur. At in ecclesia multarum gentium ut idem fiat impossibile est . . . . Cum unitate religionis Christianæ bene constat multitudo regnorum.”]

p
whilst E.Q.C.L.

q
into E.C.

r
king E.C.

s
inconveniences E.

t
must E.C.L.Q.

u
of nature E.

x
the om. E.

y
The following paragraphs, to “kings and priests” in p. 367, are inserted here on the authority of the Dublin MS.; and collated with Clavi Trabales, pp. 64-71.

z
for om. Cl. Trab. D.

1
[See book v. c. 76, § 4.]

1
2 Cor. iii. 7, 8.

2 [The editor has substituted this from Cl. Trab. for “ad primum,” which stands here in the MS. by mistake.]
a
which om. Cl. Trab.

b
seem D.

1
[See in Hector Boeth. Scot. Hist. lib. xii. fol. 250. ed. Paris. 1574; circ. ad 1050, the third law of Maccabæus (or Macbeth): “Qui pontificis authoritatem annum totum execratus contempserit, neque se interim reconciliarit, hostis reip. habetor: qui vero duos annos in ea contumacia perseverarit, fortunis omnibus multator.” This may be seen in the Councils, Hard. t. vi. p. 1. pag. 974: with his other canons, the one transferring all judicature over Christians to the clergy, the other confirming their right to tithes and oblations.]

1
[Whitaker. adv. Campian, p. 201. “Pontifex Romanus ille est Nemrodes, robustus venator ecclesiæ.”]

1
[In 1 Pet. ii. 9. Comm. in Epist. omnes Canonicas, Antwerp, 1591. fol. 270. “Cum dicitur, Exod. xix, Vos eritis in regnum sacerdotale, quare dicatur hic regale sacerdotium? Resp. Ad innuendam prærogativam novi testamenti respectu veteris: in novo enim testamento sacerdotium præeminet regno; sicut spiritus præeminet corpori. Regnum enim consistit in regimine corporali, sacerdotium vero in regimine spirituali. Ideo potestas sacerdotalis ponitur in substantivo, regalis autem in adjectivo.” This work is omitted in the Roman and Venetian editions of Aquinas, and is ascribed by many critics to Thomas Anglicus; i. e. to Thomas Gualensis or Wallensis, a Dominican of Oxford, about ad 1332: whose nomen gentilitium may have been confounded with Angelicus, the well-known epithet of Aquinas. See Wharton ap. Cave, Hist. Lit. i. 728, and App. 10, 29, ed. 1668; Sixt. Senens. Biblioth. i. 482. Neap. 1742; Lorinus, in S. Jac. Præf. § 11. The same doctrine however is clearly enough taught in the treatise De Regimine Principum, Aquin. t. xvii. Opusc. xx. lib. i. c. 14. “Ab eo (Christo) regale sacerdotium derivatur . . . Quia in veteri lege promittebantur bona terrena . . . religioso populo exhibenda, ideo et in lege veteri sacerdotes regibus leguntur fuisse subjecti. Sed in nova lege est sacerdotium altius, per quod homines traducuntur ad bona cœlestia: unde in lege Christi reges debent sacerdotibus esse subjecti.” Wharton however doubts the genuineness of this treatise also.]

2
Exod. xix. [6.]

3
1 Pet. ii. [9.] Thomas in eum locum.

4
Revelat. i. 6.

c
first considered thus E. first thus considered C.L.

d
easier E.C.L.

e
government E.

f
with us om. C.

g
do understand E.

h
and E.C.

i
lawful E.C. princes lawfully Q.

k
in E.C.Q.L.

l
spiritual E. [Fulm. special] D.

m
the om. E.C.Q.

n
dominion or [of C.] supreme E.Q.C.L.

o
Again to E.

p
it will peradventure E.L.Q.

1
T. C. lib. ii. p. 411. [See also T. C. i. 35; Def. 181; and in Bristow, Motives to the Catholic Faith, fol. 157. ed. 1599, almost the same argument alleged on the part of the Church of Rome.

The following memoranda are found in the Dubl. MS. fol. 154. with a reference, in Archbishop Ussher’s handwriting, to this part of the treatise.

“The name of ‘Head of the Church of England,’ to give to the prince, they count it injurious unto Christ. See Mr. Cartw. second book, p. 411.” (Here Abp. Ussher adds a note; “vid. supr. pag. 47:” i. e. p. 47. of the MS.) “See Counterpoison, pag. 173, what authority they leave to princes.”*

“The cause of this doubt is a conceit that the Church and commonweale in respect of regiment must needs be always two distinct bodies; so that the head of the one cannot be the head of the other also. Their reason frivolous, that because Christ is properly termed the Head of the Church, therefore the Prince may not be called the Head of this Church under Christ. What the name of Headship doth import being attributed unto Christ; that his headship over all churches doth not exclude the authority of governors placed as heads over each particular church for the visible regiment thereof. That a Christian prince within his dominions hath supreme power, authority, and headship, over all governors, and that in causes of whatsoever kind, no less if they belong to the Church of Christ than if they merely concern the temporal and civil state.

“Their minds, I doubt not, are far from treason. Howbeit, in the days of Henry VIII. to have held that which now is maintained concerning the prince’s power, had then been adjudged a capital offence.

“Out of the principles which the learneder sort of them deliver the simpler* may draw, as some have done, that by just execution of law hath cost them their lives. A hard case, and to them small comfort which have taught these silly persons such doctrine as being unsaid they have notwithstanding suffered death.”

It will be perceived that most of these notes are expanded more or less entirely in the book as we now have it. Some of the topics however do not there occur. The memoranda are exactly of the same sort as those in the C. C. C. copy of the Christian Letter, inserted here and there in the notes on the five first books. This is a confirmation (if any were needed) of their genuineness.]

*
[“For his” (Cosin’s Answer to the Abstract, p. 207.) “slander that we agree with the papists ‘to give Christian princes power of fact, but not of law, and authority to promote and set forward, not to intermeddle in causes ecclesiastical;’ we esteem it no more than a foul untruth, which every man of judgment can convince. For if they have authority in our judgment by the word of God to see to their ministry, and to cause them to make such laws as they know to be agreeable to God’s word; to authorize such and disannul the contrary; cause them to make good when they would make ill; or orderly to procure such as can and will be present in the action, and give their consent if it please them (all which are given by T. C. (ii. [iii.?] 167.) and by us all unto the magistrate): then do we grant them no more than ‘power of fact?’ than ‘to promote matters?’ ”]

*
[e. g. Penry, Coppinger, Arthington.]

q
fit om. D.

r
entitled Head of the Church, which was given E.

1
Ephes. i. 21, Col. i. 18.

s
rules, dominions, titles E. rules or dominions C.

t
to civil magistrates E.

u
termed also E.

x
of all E.

y
articles E.

z
much as om. E.C.L.Q.

a
none E.

b
magistrate E.Q.C.L.

c
to E.C.

d
used and urged E.C.

1
Ephes. i. 20-23.

e
had [hath C.] set on his E.C.

f
fulness E.

2
Col. i. 18.

g
he mentioned before E.C.L. named before Q.

3
Col. i. 16.

h
By E.

i
on D.

k
dominions E.C.

l
which lifteth...... of head om. E.

m
ways E.

n
therein om. E.

o
usually E.

p
self om. E.C.

q
equalizing E. equalling [of om.] C.

1
Apol. [adv. Gent. c. 34.] “Dicam plane Imperatorem Dominum sed quando non cogor ut Dominum Dei vice dicam*.”

*
This note om. E.

r
call E.

s
may be also as well D.

t
doth E.C.L.

u
or E.

x
Christ, that the E.C.L.

y
where E.

z
called E. termed C.

a
in meaner degrees E.C. in meaner degree L.

b
the om. D.

c
there is E. is the C.L.

d
that could not well be E.

e
nor E.

f
the E.C.L.

g
which om. E.C.L.

h
must E. [might Fulm.]

i
plain E.

k
personally spoken E. unusually taken C.

1
Capita papaverum, primores civitatis. Liv. I. [54.] Roma κεϕαλὴ συμπάσης Ἰταλίας. Dionys. Halic. Antiq. lib. II. Pekah is termed the Head of Samaria, which was the seat of his throne and kingdom*. Esai. vii. 9.

*
This note, except the reference to Isaiah, om. E.

l
nothing E.

m
even om. E.

n
Christian kings E.

o
others D.

p
other, although it be E.C. although Q.L.

q
than that which E.Q.L.

2
Confess. c. 5. art. 23† [“Eorum qui publico munere funguntur in ecclesia, alii . . . partim administrant civilia negotia, partim ecclesiæ tranquillitatem in genere procurant ac tuentur, et quidem accepta in hos usus gladii potestate:”] et 32. [“Civili magistratui obnoxii sunt omnes, cujus etiam potestas est suo respectu ἀρχιτεκτονικὴ, quatenus pacem et εὐταξίαν procurare debet, præsertim in iis quæ primam tabulam respiciunt.” Tract. Theol. i. 42. 46. Gen. 1570.]

†
This reference om. E.

r
command E.

s
if even E.

t
they E.

u
the E.C.

v
it should not E.

x
could E.

y
whole om. E.

z
is properly E.Q.C.L.

b
Apostle......doth E.

c This side-note om. E.
d
First, It differeth in order, because E. in measure......in kind Q.L. first in order...secondly...thirdly C.

e
above all om. E.C.

f
τη̑ς om. E. ὑ. π. τ. ἀ. om. L. all the Greek om. C.

1
Ephes. i. 21, 22.

g
principalities......powers E.C.

h
subordinated D.

i
Secondly, again E.

2
Psal. ii. 8.

k
both in E.C.L.

l
own E.C.

m
for ever om. E.C.

n
other E.C.Q.L.

o
How their power E.C.

p
Thirdly, The last E.

q
and greatest E.C.

r
the E.C.

s
the om. E.C.

3
Θειότατον καὶ τω̑ν ἐν ἡμι̑ν πάντων δεσποτου̑ν. Plat. in Tim.*

*
This note om. E. in English C.

q
the Head of the Church E.C.L.

r
they om. E.C.

s
chiefest E.C.

t
is om. E.C.L.

u
is always knit to it E. is inseparably knit with it. L.

x
motion unto E.

y
quickeneth us E.

z
Church affairs E.C.L.

a
any possibility E.C.Q.

b
one besides E.C.

c note om. E.Q.
1
T. C. lib. ii. p. 411. [and i. 167.]

d
not E.

e
over his, and over kingdoms, E. [Fulm. “ ‘other kingdoms,’ i. e. over his own, and over other kingdoms.”]

f
of E.C.L.Q.

g
to commonwealths E.

h
there E.C.

i
of Christ om. E.C.

1
T. C. lib. ii. p. 418. [Of this and the passage last referred to, the substance is given, not the very words.]

j
him E.C.

k
that om. E.C.Q.L.

l
it om. E.

m
in word om. E.C. in words Q.

n
of Christ E.C.Q.L.

o
first om. E.C.L.Q.

2
Matt. vi. 13.

3
1 Tim. i. 17.

p
the invisible D.

q
in om. D.

r
both om. E.C.L.

s
which made E.

1
Apoc. i. 8.

s
hath E.

t
as he was man E.

u
requireth E.

2
John xvii. 5.

x
now om. E.C.L.

y
he E.

z
Further, it is not necessary E.C.L.Q.

a
properly and truly E.

b
and E.C.

c
and E.C.

d
conditions E.C.L.

e
wholly E.

f
and therefore E.C.

g
proportion E.

3
1 Tim. iv. 10.

h
succour E1. 1666, corr. 1676.

4
Heb. v. 9.

i
high and ghostly E.C.L.Q.

5
1 John i. 3.

6
Heb. xii. 22[-24.]

k
we account them E.C.L.Q.

l
and that live E.C.L.Q.

m
as over dutiful and loving subjects E.C.L.Q.

n
the E.

1
[T. C.] ii. 411. l. 14.

o
his E.

p
the E.C.

q
in governing E.Q.L.C.

r
of om. E. [not E1.]

s
kingdoms E.

2
T. C. lib. ii. p. 418. l. 10. [rather 416 . . 418.]

t
magistrates authority E.C.L.

u
Christ’s E.C.L.Q.

x
is subordinate E.

y
the preservation E.C.

z
of om. E.L.C.Q.

a
also E.

b
obstinate and rebelliousE.C.

3
T. C. ii. 417. l. 12.

c
that E. the C.L.Q.

g
there E.L. om. C.

h
superiors E.Q.C.L.

i
of the commonwealth E.

k
to E.Q. until C.

l
them om. E.

m
as E.

n
on the earth E.C.L.

o
reigneth now E.

p
any longer under him E.C.L.Q.

q
the E.

r
providence and kingdom E.

1
[T. C.] lib. ii. p. 411. lin. 16. [D.]

s
regiment E.

t
sacrifices E.

2
Heb. ix. 25.

u
sleight E.Q.C.L.

3
T. C. lib. ii. p. 415.

x
immediately E.C.

4
Rom. xiii. 1.

y
nor with any subordination to God, nor doth any thing from God, but by the hands of our Lord, &c. E.

5
Prov. viii. 16. Humble Motion, p. 63. [“Seeing her highness doth acknowledge Christ to be her head, and renounceth the pope, is it not for her safety, by her authority, to set up that which remaineth of Christ’s most holy laws, and to banish all the pope’s canons! May not her princely mind perceive it to be so, if she remember that it is said of Christ, ‘By me kings reign, and princes decree justice: by me princes rule,’ ” &c.]

z
By me ......justice om. E.

6
Rev. i. 5.

a
amongst......government om. E.C.L.

b
and have......unto him om. E.

c
as om. E.

d
of necessity E.C.

e
govern and guide E.C.

f
special E.

g
namely om. E.

1
1 Cor. iii. 22, [23.]

h
E. reads kings are Christ’s as saints, because they are of the Church, if not collectively, &c. C. reads, as saints, because they are of the Church: as kings, because they are in authority over the Church, &c. in which L. agrees. D. and Q. give it as in the text.

i
It E. [The mistake might arise from the old way of abbreviating “that.”]

k
surely E.C.L.

l
reacheth E.C.

m
may have and lawfully exercise it E.

o
where they speak E. where C.L.Q.

1
T. C. lib. ii. p. 413.

p
over E.

q
seeing E.

n This side-note om. E.Q.
r
heads E.L.

s
the E1.

t
may om. E. [Fulm. “may be”] C.

u
more E1.

v
provinces E.C.

w
to om. E.

x
the earth E.

y
is om. D.

z
the E.

a
always remaineth E.C.L.

b
indeed E.

c
a visible E.

1
T. C. lib. ii. p. 419.

d
that are of the Church make E.

2
Ut Hen. 8. 6. 9. [26 Hen. viii. cap. 1.?]

e
customs E.C.L.

f
misconceiving E.Q.C.L.

g
or E.Q.C.L.

h
from E.C.L.

i
the head E.Q.C.L.

k
therefore even E.

l
be om. E.Q.C.L.

m
of the Church om. D.

n
the Church therefore next this, is E.

o
the E.

p
a superior or head E.C.L.

q
do grant E.C.

r
chief E.

s
order E.L. the order C.

t
termed in E.Q. termed within [C. in] his own dominions C.L.

u
own om. E.Q.C.L.

v
any E.

x
a om. E1.

1
T. C. ii. 412.

y
heads E.

z
monsters together E.C.L.Q.

a
the skilful in nature’s mysteries have been used to term it, The womb, &c.

b
it is E.

c
appears D.

d
how E.C.L.

e
shall E.

f
if Christians om. E.

g
the om. E.

h
each E.

i
each om. E.

k
a head also E.

l
and om. E.

m
more E1.

n
is E.C.

o
nor yet uncomely E.Q.C.

p
perfect body E.C.

q
him that God E.C.L.

r
to itself should E. should be to itself C.

s
ought E.

t
the E.

u
have power possibly E.C.

x
we see therefore E.C.

y
as that E.C.

1
[This section stands here on the authority of the Dublin MS. But it must be apparent to every reader that it is out of its place. Probably it was a note made to be inserted, in substance, somewhere in the treatise, but the place of insertion not determined. The conclusion of the whole subject, in p. 392, seems no improper place for it. But without MS. authority it might be too great a liberty to transpose it. The Dublin MS. bears marks of unusual inattention in this part.]

z
This paragraph is inserted before “these things,” p. 368. E.Q.C.L. There it is clearly incongruous, and here the transition would be clearer without it.

a
state E.C. [style Fulm.]

b
should E.

c
any om. E.

d
great E.

1
G. Courin. in Epist. de Morte T. Mori, et Episcopi Roffensis, p. 517. [ap. “Thomæ Mori, Angliæ Ornamenti eximii Lucubrationes.” Basil. 1563.*]

*
This note, except “Roffens. Epist. p. 517.” om. E.Q.C.L. “p. 517.” om. D.

e
a secular E.

2
[“Illud dico, me septem annis intendisse animum studiumque meum in istam causam, verum hactenus in nullo doctorum ab ecclesia probatorum reperi scriptum,] quod laicus, aut, ut vocant, sæcularis, possit aut debeat esse caput status spiritualis aut ecclesiastici.”

f
even om. E.

3
Præf. Cent. 7. [t. iv. p. 11. Basil. 1567. “Non sint capita ecclesiæ, quia istis,” &c.]

4
Calvin. in Com. in Amos vii. 13. [Quoted by T. C. ii. 413. “Qui initio tantopere extulerunt Henricum regem Angliæ, certe fuerunt inconsiderati homines: dederunt illi summam rerum omnium potestatem: et hoc me semper graviter vulneravit. Erant enim blasphemi, qui vocarent eum summum caput ecclesiæ sub Christo. Hoc certe fuit nimium. Sed tamen sepultum hoc maneat, quia peccarunt inconsiderato zelo . . . Faciunt illos nimis spirituales. Et hoc vitium passim regnat in Germania. In his etiam regionibus nimium grassatur . . .Principes, et quicunque potiuntur imperio, putant se ita spirituales esse, ut nullum sit amplius ecclesiasticum regimen. Non putant se posse regnare, nisi aboleant omnem ecclesiæ auctoritatem, et sint summi judices, tam in doctrina, quam in toto spirituali regimine.” p. 282. ed. 1610.]

g
and thought . . . . we do om. E.

h
protesteth E.L.Q.

i
through D.

j
was E.C.

k
government E.

l
the authority E.C.L.Q.

m
Here the Dublin MS. proceeds as in p. 388, line 20. “Their meaning is,” to “whole or any part,” in p. 392: and then inserts what follows in this place, as far as “spiritual government,” p. 388.

n This side-note om. E.Q.C. Against the third difference L.
o
made om. E.

p
give E.Q.C.L.

q
these D.

r
the E.C.

1
[Whitg. Def. 300, 301. “Christ is the only head of the Church, if by the head you understand that which giveth the body life, sense, and motion: for Christ only by his Spirit doth give life and nutriment to his body. He only doth pour spiritual blessings into it, and doth inwardly direct and govern it. Likewise he is only the head of the whole Church, for that title cannot agree to any other. But if by the head you understand an external ruler and governor of any particular nation or church, (in which signification head is usually taken) then I do not perceive why the magistrate may not as well be called the head of the church, i.e. the chief governor of it in the external policy, as he is called the head of the people, and of the commonwealth. And as it is no absurdity to say, that the civil magistrate is head of the commonwealth, next and immediately under God, (for it is most true,) so is it none to say, that under God also he is head of the church, i. e. chief governor, as I have before said.”]

s
there om. E.C.

t
exceptions D.

u
the outward E.C.

x
of head D.

y
to him with E.C.L.Q.

1
T. C. ii. 414. [“It is first to be noted from whom this provision was brought him. For as Harding borrowed it of Pighius, so the doctor’s purveyors had it from Harding, or from both.”]

z
to om. E.Q.C.L.

a
it E.

b
already hath E.

c
sufficiently been E.C.L.Q.

d
therefore be E.Q.C.

e
convicted in some things E.Q.C.L.

f
thought E.

g
confess E.C.

2
T. C. lib. iii. p. 168.

h
in E.C.

i
of E. in E′.

k
is there E.Q.C.L.

l
mislike E.

m
therein E.

n
and E.C.

o
that om. E.Q.C.L.

p
reasons E.C.L.

q
any om. E.

r
to be D.

1
T. C. lib. ii. p. 415.

s
the om. E.Q.C.L.

t
also as it is E. as it is also C.

u
in distinguishing, they think E. as they think, in so distinguishing C.

x
doth D.E′.

y
wonted om. E.C.

z
kind of affection E.C.L.

a
be able to om. E.

b
henceforward E.C.L.Q.

c
invisible, exercised E.C.L.

d
the om. E.

e
particularly E.Q.C.L.

f
him only do we acknowledge E. him therefore only (L. only therefore) do we C.L.

g
the E.C.

h
graces E.

i
the sacraments E.C.L.Q.

k
means E.C.L.

l
those D.

m
such properly concerns E.

n
regiment D.

o
nor E.Q.C.L.

p
seeing E.

q
his E.

r
differing om. E.

1
T. C. lib. ii. p. 415.

s
in om. E.

t
societies E.Q.

u
gathered together E.L.

x
must be their head E. must needs be their head C.L.

y
not their E.C. there E′.

z
societies E.Q.C.

a
as God E.C.

b
not there E. there om. C.

c
very om. E.C.

d
hereby E.C.L.

e
as D.

f
and not the D.

g
that there E.Q.C.L.

h
that E.Q.C.L.

i
that E.

k
the om. E.L.

l
Church E.

m
the head E.Q.C.L.

1
T. C. ii. 413. [“As it hath certain ground in the Scripture that this title of Head of the Church is too high to be given unto any man, so hath it been confirmed from time to time by writers both old and new, which have had the honour of Christ in any convenient estimation . . . Cyprian saith, ‘there is but one head of the Church.’ De Simplicitate Prælatorum,” (i.e. de Unitate Ecclesiæ: “Ecclesia Domini . . . ramos suos in universam terram copia ubertatis extendit . . . Unum tamen caput est, et origo una.” p. 195. ed. Baluz.) “The bishop of Sarisbury affirmeth the same. Apol. p. 2. c. 2. div. i.” (“Christ alone is the prince of this kingdom; Christ alone is the head of this body; Christ alone is the bridegroom of this spouse.”) “Augustine proveth that the minister which baptizeth cannot be the head of him which is baptized, because Christ is the Head of the whole Church. Contr. Lit. Petil. i. [4.] 5.” (“Id enim agunt isti, ut origo, radix, et caput baptizati non nisi ille sit a quo baptizatur . . . O humana temeritas et superbia . . . . Cur non sinis ut semper sit Christus origo Christiani, in Christo radicem Christianus infigat, Christus Christiano sit caput? . . . .An vero Apostolus Paulus caput est et origo eorum quos plantaverat . . . cum dicat, nos multos unum esse corpus in Christo, ipsumque Christum caput esse universi corporis?” t. ix. 208. comp. lib. iii. c. 42. p. 322.)

n
it om. D.

o
into E.

p
besides can be so E.

q
Here the Dublin MS. goes back to p. 386. “The last difference . . . . . spiritual government.” p. 388.

r
E.C.L.Q. begin this paragraph with the word “amongst,” in lin. 14, transposing all that goes before it so as to come in after “and others.”

1
Polyb. lib. vi. de Milit. ac Domest. Rom. Discipl. [c. 12.]

s
other om. E.C.

t
there E.

2
1 Macc. xiv. 44.

u
sort E.Q.C.L.

3
1 Chr. xv. 3, 4.

4
1 Reg. viii. 1.

5
2 Chr. xv. 9; xxiv. 5; xxx. 1; xxxiv. 29.

x
Before this paragraph two insertions are made in E.Q.C.L. 1. From “The Consuls” to “Wherefore,” noticed above note r. 2. From “The clergy” to “shall not need,” as below, p. 395. The two are connected thus: “Wherefore the clergy,” &c.

6
Dig. xlvii. 22. De Collegiis Illicitis [et Corporibus.] L. i. [1. Mandatis principalibus præcipitur præsidibus provinciarum, ne patiantur esse collegia sodalitia, neve milites collegia in castris habeant . . . ne sub prætextu hujusmodi illicitum collegium coeant . . . Sed religionis causa coire non prohibentur: dum tamen per hoc non fiat contra senatus consultum.” 3. “Nisi ex senatus consulti auctoritate, vel Cæsaris, collegium, vel quodcunque tale corpus coierit: contra senatus consultum, et mandata, et constitutiones collegium celebrat.”] Cod. Just. i. 3. De Episc. et Presbyt. [et Cler. L. 15.] De Illicit. Conventiculis. [“Conventicula illicita etiam extra Ecclesiam in privatis ædibus celebrari prohibemus; proscriptionis domus periculo imminente, si dominus ejus in ea clericos nova ac tumultuosa conventicula extra ecclesiam celebrantes susceperit.” ad 404.]

y
Christians E.C.L.

z
general synod E.C.

a
meeting E.Q.C.L.

b
consisting E. consists C.

c
not accounted E.Q.C.L.

1
[Albert Pighius, of Kempen in Holland, (1490-1542.) “Aucun controvertiste n’a poussé plus loin le zèle pour les prétensions de la cour Romaine.” (Biog. Univ.) The work quoted is Hierarchiæ Ecclesiasticæ assertio, 1544, several times reprinted.] Hierarch. lib. vi. cap. 1. [“Constantini principis pius religiosusque zelus prima eorundem causa et origo extitit.”]

d
businesses E.

2
Constant. concilium a Theodosio sen. indictum: Theod. l. i. [5.] c. 9. Ephesinum 1. nutu Theodosii jun. convenit. Evagr. i. 2. [i. 3.] Sardicen. concil. a Constant. [Sardicense Constantius indicit. D.] Theod. ii. 4. Chalcedon. impetratum a Martiano. Leo, Ep. 43*.

*
These references are in part supplied by the MSS. D. and L.

e
used D.

3
Hieron. cont. Ruffinum, lib. ii. [§ 20. St. Jerome, as appears by the context, was rather disputing the existence than the authority of the alleged synod. “Responde, quæso, synodus, a qua excommunicatus est (S. Hilarius), in qua urbe fuit? Dic episcoporum vocabula; profer sententias subscriptionum . . Doce qui eo anno consules fuerint, quis imperator hanc synodum jusserit congregari: Galliæne tantum episcopi fuerint, an et Italiæ et Hispaniæ: certe quam ob causam synodus congregata sit. Nihil horum nominas.” t. ii. 513. ed. Vallars.]

f
the om. D.

f
the om. D.

g
and E.Q.C.L.

1
Sozomen. lib. vi. cap. 7. [Οἱ περὶ Ἑλλήσποντον καὶ Βιθυνίαν ἐπίσκοποι, καὶ ὅσοι ἄλλοι ὁμοούσιον τῳ̑ Πατρὶ τὸν Υἱὸν λέγειν ἠξίουν, προβάλλονται πρεσβεύειν ὑπὲρ αὐτω̑ν Ὑπατιανὸν . . . ὥστε ἐπιτραπη̑ναι συνελθει̑ν ἐπὶ διορθώσει του̑ δόγματος· προσελθόντος δὲ αὐτου̑, καὶ τὰ παρὰ τω̑ν ἐπισκόπων διδάξαντος, ὑπολαβὼν Οὐαλεντινιανός, ἐμοὶ μὲν, ἔϕη, μετὰλαου̑ τεταγμένῳ, οὐ θέμις ἐστὶ τοιαυ̑τα πολυπραγμονει̑ν. οἱ δὲ ἱερει̑ς οἱ̑ς ταυ̑τα μέλει καθ’ ἑαυτοὺς ὅπη βούλονται συνίτωσαν.] Ambros. Epist. 32. [21. t. ii. 860. Ad Valentinian. ii. “Augustæ memoriæ pater tuus non solum sermone respondit sed etiam legibus suis sanxit, in causa fidei vel ecclesiastici alicujus ordinis eum judicare debere, qui nec munere impar sit nec jure dissimilis; hæc enim verba rescripti sunt, hoc est, sacerdotes de sacerdotibus voluit judicare . . . . Pater tuus, Deo favente . . . dicebat, Non est meum judicare inter episcopos.”] Quanquam longe aliter Nicephorus, lib. vii. c. 12*. [xi. 3. where Valentinian is represented as saying, Ἐμοὶ, πράγμασιν ἐνειλημμένῳ, καὶ τὰ του̑ πλήθους ἐπιτετραμμένῳ, οὐκ εὐχερὲς τὰ τοιαυ̑τα διερευνα̑σθαι.]

*
This reference om. E.C.

h
matters E.C.

i
willed E.Q.L. called C.

k
belongeth E.Q.C.L.

l
together E.C.

m
where E.

n
together om. E.

o
the om. E.Q.C.

p
and Valentinian E.C.

q
east unto the west parts E.

r
to D.

s
there the bishops E.

t
was E.C.

u
very small E.C.L.Q.

x
grow thereby E.C.

y
means E.C.L.

z
he E.Q.C.L.

a
unto them E.Q.C.L.

b
even om. E.

c
else om. E.C.L.

d
now therefore E.Q.C.L.

e
This passage, from “The clergy” to “shall not need,” in E.C.L.Q. occurs before, viz. after “the other. Wherefore” in p. 392.

f
that afterward E. [Fulm. del.]

g
seemeth D.

h
thereof E.C.

i
breaketh E.Q.C.L.

l
will E.Q.C.L.

k This whole §, down to the words “laws thereof,” is inserted here from the Dublin MS. [It does not appear in E. 1648, 1651, or Gauden, 1662.] It might not improperly be marked as a fragment, as it evidently has not been brought into coherence with what comes before and after. It appears to be the introduction of this part of the treatise, as re-written by the author, but not yet finished off so as to smooth the transitions and avoid repetition. The marginal heading is transferred, as the subject seemed to require, from the beginning of the following section, “The case is,” &c.
m
hir D.

n
or E.Q.C.L.

1
Soto in 4 Sent. [ubi infra. “Gerson in Tract. Potest. Eccles.” (cons. iv, xi.) . . . “atque alii fautores illius opinionis, quod concilium est supra papam, arbitrati suam opinionem ex hoc fundamento pendere, aiunt, potestatem ecclesiasticam jurisdictionis in utroque foro residere in tota universitate Ecclesiæ, hoc est, in toto corpore . . . Jure enim naturæ potestas regendi rempubl. in tota ipsa est, et in nullo seorsim membro, nisi ab ipsa eligatur, ut est videre in antiquo regimine Romanorum . . . Nisi quod illæ quæ rege gubernantur ipsum elegerunt, in quem suam transtulerunt auctoritatem, quæ jure hæreditario perpetuo succederet in suam sobolem, juxta tenorem legis, Quod principi placuit. Sic ergo aiunt existere potestatem in corpore Ecclesiæ immediate.” Which opinion he proceeds to combat on the ground of the apostolical charter granted in Scripture.]

2
Potestas jurisdictionis ecclesiasticæ non residet in toto corpore immediate, sed in prælatis. Caiet. [Thomas de Vio, of Gaeta, Dominican theologian, 1469-1534.] in Opusc. de comp. Pap. et Concil. [t. i. tract. i. c. xii.] Turrecr. [John Torquemada of Valladolid, Dominican theologian, 1388-1468.] Summ. Eccl. l. 2. c. 71. [fol. 196, 197. Venet. 1561. apud] Soto in 4 Sent. Dist. 20 q. 1. art. 4.

1
[Possibly this paragraph might case it should stand as § 1 of this be meant as a transition from the chapter. It is here given as in the former chapter to this: in which Dubl. MS.]

o
This portion of the work, to “assent not asked?” p. 407, is omitted in the edition of 1651, but found in part in Clavi Trabales, p. 73-76, &c. and was inserted by Bishop Gauden in his edition of Hooker’s works, 1662. It occurs in MSS. Q.C.L. but much later, viz. where Bishop Gauden inserted it, after the words “defence of the truth therein,” at the end of c. viii. On the authority of the MS. D, confirmed by internal evidence, it is now placed here.

p
cause E.

q
not om. D. It had been “unlike,” but the “un” is erased.

r
of om. Cl. Trab.

s
they are om. E.

t
quality E.Q.C.L.

u
belongs E.

1
[Eccl. Disc. transl. by T. C. p. 4. ed. 1617; comp. T. C. i. 84. al. 63. ap. Whitg. Def. 305. “Moses that was the overseer of the work was a wise and a godly man; the artificers that wrought it, Bezaleel and Aholiab, most cunning workmen: and yet observe how the Lord leaveth nothing to their will, but telleth not only of the boards, of the curtains, of the apparel; but also of the bars, of the rings, of the strings, of the hooks, of the besoms, of the snuffers,” &c. . . . “If in the shadows, how much more in the body . . . Is it a like thing . . . that he that then remembered the pins did here forget the master builders?”]

x
as E.Q. Cl. Trab.

y
also E.Q.C.L. Cl. Trab.

z
strait E.

1
Deut. iv. 2; xii. 32; [quoted in Admonit. p. 1. ed. 1617;] Jos. i. 7.

a
to Cl. Trab. you D.

b
had E.

c
law E.

2
Thom. ii.* [2 Sum. pars i.] quæst. 108. art. 2. [p. 709. Venet. 1596.]

*
1. 2. D.

d
said Cl. Trab. D.

e
of scripture D.

f
must needs take D.

g
so great E.Q.C.L. Cl. Trab.

h
as a conclusion E.

i
as E.Q.C.

k
for manner in E.C.L. Cl. Trab. for manner of Q.

l
a duty E.

m
it om. E.C.L.

n
as om. E.

1
[Rom. x. 10.]

o
man’s laws have E.

p
their contradiction E.Q.C.L. Cl. Trab.

q
man afterwards is E.C.L. Cl. Trab.

r
man of wisdom apply those words of D.

2
Prov. vi. 20.

s
The English first E.Q.

t
makes E.Q.C.L.

u
thou not E.Q.C.L.

x
a thing even undoubtedly E.C.L.

y
ordained D.

1
Δει̑ τὸν νόμον τὰ περὶ Θεοὺς καὶ δαίμονας καὶ γονέας, καὶ ὅλως τὰ καλὰ καὶ τίμια, πρω̑τα [πρα̑τα] τίθεσθαι· δεύτερον δὲ τὰ συμϕέροντα· τὰ γὰρ μήονα τοι̑ς μείζοσιν ἀκολουθει̑ν καθήκει. [ποθάκει*.] Archyt. de Leg. et Justit. That is, “It behoveth the law first to establish or settle those things which belong to the gods, and divine powers, and to our parents, and universally those things which be virtuous and honourable; in the second place, those things that be convenient and profitable: for it is fit that matters of the less weight should come after the greater†.” [Ap. Stob. Floril. II. 169. ed. Gaisford.]

*
This word is erased D.

†
Translation om. D.

z
must needs E.Q.C.L.

a
Here the fragment in Cl. Trab. breaks off.

b
it must of necessity retain the same, being of the Christian religion E. of necessity being [of 1676] Christian Religion, Gauden, 1662.

c
that om. E.C.Q. inserted L.D.

2
Act. xv. 7. 13-23.

d
which om. E.

e
afterwards E.Q.C.L.

3
[See App. No. iv.]

f
belongeth to the prelates E. The MSS. all give it as above: except that the before bishops is omitted in D.

g
alone om. E.

h
has E.C.L. hath Gauden.

i
since D.

1
Acts xvi. 4.

k
since D.

2
Acts xv. 28.

3
Matt. xxviii. 20.

l
the E.C.

4
2 Cor. iii. 3, 6.

m
more D.

n
no om. E.

o
are, as now E.C.L.

p
laws E.Q.C.L.

1
Cap. Dilecta, de Excess. Prælator. [Decretal. Greg. v. 31, 14. c. 1642. Lugd. 1572. This is an inhibition of Pope Honorius III. to the clergy of Jouars, in the diocese of Meaux, forbidding them to make or use a common seal without the consent of the abbess of Jouars, who was “ipsorum caput et patrona.”] L. Per fundum [Tit. de servitutib.] rusticor. Præd. [Digest. lib. viii. tit. iii. l. 11. “Per fundum, qui plurium est, jus mihi esse eundi, agendi, potest separatim cedi: ergo subtili ratione non aliter meum fiet jus, quam si omnes cedant: et novissima demum cessione superiores omnes confirmabuntur.”] et § Religiosum. De rerum divis. [Inst. II. 1. § 9. “Religiosum locum unusquisque sua voluntate facit. In communem autem locum purum invito socio inferre non licet.”]

2
Gloss. [in verb. Pertinet.] Dist. 96. c. Ubinam*. [fol. xcix. Lugd. 1509. Bonifac. viii. De Regulis Juris, ad calc. lib. 6i Decretal. Lugd. 1572. Reg. xxix. col. 742.]

*
This note from D.

1
[Decr. Gratian. pars i. d. 96. col. 468, from a letter of Nicholas I. to the Greek emperor Michael III, reproving him for having been a party to the proceedings of the provincial synod which deposed Ignatius patriarch of Constantinople without any charge of heresy, and substituted Photius in his place. ad 865. Concil. Hard. v. 158 C.]

q
quæ univ. . . . communis est om. E.

r
as om. E.Q.C.L.

s
minuere E.C.L.

2
Extrav. de Judic. C. Novit. (Extra de judiciis novit, Gauden in text.) [This passage does not appear in the Extravagantes, Tit. De Judiciis, ad calc. vi. Decretal. ed. 1573. The forty-second canon of the fourth Lateran council, which was drawn up by Innocent III, ad 1215, runs thus: “Sicut volumus ut jura clericorum non usurpent laici, ita velle debemus, ne clerici jura sibi vindicent laicorum. Quo circa universis clericis interdicimus, ne quis prætextu ecclesiasticæ libertatis suam de cætero jurisdictionem extendat in præjudicium justitiæ sæcularis.” Conc. Hard. vii. 49. In the title De Judiciis, Decretal. Greg. ix. lib. ii. tit. i. cap. 13, (which begins, Novit ille qui nihil ignorat) the following passage is given of the letter from Innocent to the bishops of France; by which he interfered between king John and Philip Augustus, ad 1204; “Non putet aliquis quod jurisdictionem illustris regis Francorum perturbare aut minuere intendamus, cum ipse jurisdictionem nostram nec velit nec debeat impedire.” col. 489. Lugd. 1572.]

t
saith Pope Innocent E.Q.C.L.

u
conditions E.L.

x
right E.C.

y
in making laws E.Q.C.L.

z
whereby E.Q.C.L.

a
in om. E.C.L.

b
Tridental E.

c
laws E.

1
Boet. Epo, Heroic. Quæst. lib. i. sect. 284. [“Ecclesiasticarum sive Heroicarum Quæstionum libri sex.” No date, but some time before 1588, in which year were published three additional books, “De jure Sacro.” The author was Boetius Epo, a native of Friesland, [1529-1599] Professor of Canon Law at Douay, 1578. The editor has not obtained a sight of the work here quoted. It appears from the continuation of it, that the writer was a strenuous assertor of the pope’s plenary power: and from the preface to his “Antiquit. Ecclesiast. Syntagmata,” that he had once been a Protestant. (Moreri; Hurter, Nomenclator Liter. i. 228.)]

d
touching either E.Q.C.L.

e
right E.Q.C.L.

2
[It should seem from Strada’s account, b. iv. p. 106, 107, that no formal exception was made, but from Fra Paolo, viii. 85, that the publication took place in the king’s name and not in the pope’s; and from Brandt, (Hist. of the Reform. in the Low Countries, b. v. Eng. Transl. t. i. 153,) that the “temporal magistrates were directed to assist the prelates. . . and to be conformable to the canons of the council in every thing, save only where they might seem to derogate from his majesty’s prerogatives or from the rights of any of his vassals.” This statement is confirmed by the original documents as they stand in Le Plat, Monum. Hist. Concil. Trid. t. vii. especially the king’s final letter to the duchess of Parma, p. 91. The points specified by Hooker about patronage, &c. are specified not in the king’s letter, but in various memorials, given by Le Plat, from the councils of Namur, Brabant, &c. (p. 71, 83,) and forwarded by the duchess to Philip: which memorials occasioned the letter.]

f
again D.

g
follows E.C.L.

h
king’s E.C. Kings Gaud.

i
face E.L.

k
a corporation E.

l
Here the printed editions since Gauden, and all the MSS., insert a passage, which will be found below, as a note by way of Appendix to this book. The reasons for omitting it here will be found elsewhere. The Dublin MS. then proceeds as in § 14. “And concerning,” to “over the Church.” But as that MS. is clearly erroneous and incoherent in one part of this arrangement, the transposition has not been adopted.

m
account E.Q.C.L.

n
those E.C.L.

1
[Allen, Apol. 1583, c. iv. p. 69. “Veritas est, nec regem nec parlamentum habere potestatem legitimam præscribendi ordinem ecclesiæ vel clero in hac parte, magis quam hierarchiis angelorum in cœlo commorantium.” The points which he had just been mentioning were the royal supremacy and the validity of the protestant episcopal orders.]

1
[Ibid. p. 64. “Parlamentum autem est conventus plane civilis, in quo nec episcopi aliter quam ut regni barones jus suffragandi obtinent, nec ut barones ullam habent tractandi aut definiendi negotia, aliam quam quæ ad civilem status gubernationem spectant, potestatem: cum omnis potestas, quam vel episcopi vel alii in illo loco exercent, sit a Principe et Rep. civili derivata; ad quos nec lege divina nec naturali hujusmodi rerum definitio spectat.”]

o
of power om. E.

p
notwithstanding om. E.C.L.

q
so E.

2
[Ibid. 65. “Non ad paganos imperatores hoc spectabat, (quamvis non minus olim imperiales et regales quam nunc temporis extiterint) nec ab illis expetebatur; nam sub Nerone, præcipui Apostoli ecclesiam Romanam gubernabant.”]

3
[Ibid. 67. “Hoc itaque regimen non est jus regi terreno, principi, aut statui ulli debitum: hi enim omnes (si Christiani sunt) tenentur subesse pastoribus animarum suarum et ecclesiæ Christi.”]

r
laws here E.C.L.Q. and D. read as in the text.

s
in such sort om. E.C.

4
[Ibid. “Nec eam ecclesia concessit, nec unquam concedere potest, cum nec a natura illis, ut patet in ethnicis, competat, nec jure Christianitatis, cujus virtute omnes quotquot in universo orbe vivunt, ecclesiæ Christi obedire tenentur, non eidem imperare; nec ulla civilis resp. eam principi suo auctoritatem largiri potest quam nulla vi naturæ possidet: unde princeps cum hanc potestatem nec a populo nec a majoribus per naturalem propagationem aut alia ratione acceptam consecutus sit, eam parlamento haud communicare potest, et consequenter nullas ferre leges, nec audire nec determinare, per se vel per parlamentum aut aliud quodcunque tribunal modo jam dicto sibi subjectum, quidquam de ecclesiæ gubernatione potest.”]

t
For they . . . . unto om. E.

1
[Saravia. de Honore Præsulibus et Presbyteris debito, c. 25. “Coriarii, tinctores, textores, coctores cervisiæ, fabri, fullones, mercatores, comitia celebrant, de Republ. sententiam dicunt (quod equidem in libero populo non improbo): sed pastores ecclesiarum excludi, contra æquabile jus civium est, qui sub iisdem legibus et magistratu vivunt, et communia ferunt cum cæteris civibus onera: de quorum vita et fortunis, de iisque omnibus a quibus tum ipsorum privata salus, tum ecclesiarum publica pendet, non minus deliberatur, quam de pannis, de lana, de piscibus, de coriis cæterisque mercibus importandis aut exportandis. Num minor pastoribus ecclesiarum cura Reip. esse debet, quam Burgimagistris?”]

u
Jule D.

x
the om. E.Q.C.L.

xx
should E′.

y
of E.Q.C.L.

z
in themselves om. E.

a
foundation D.

b
into E.

c
the laws E.C.

d
grand D.

1
An. 1 et 2 Phil. et Mar. c. 8.

e
there om. D.

f
neither did they or the cardinal imagine E. or the cardinal himself, as they imagine, any thing commit Q; commit any thing C.L.

g
of om. E.Q.C.L.

h
public om. E.C.L.

i
the Christian E.

j
rites om. E.

k
the devising E.

l
This clause om. E.

l
that E.Q.C.L.

m
kings but E.

n
states of om. E.

o
the om. E.

p
no E.C.L.

q
of E.Q.C.L.

r
utterly om. E.C.L.

s
regal E.C.

t
unto their emperors E.Q.C.L.

u
means E.C.L.Q.

1
“Quod principi placuit, legis habet vigorem: cum lege Regia, quæ de ejus imperio lata est, populus ei et in eum omne imperium suum et potestatem concedat*.” Inst. [lib. i. t. 2.] de J. N. G. et C. [§ 6.]

*
“cum . . . concedat” om. E.C.L.

x
endued E.

y
thought E.

z
nor E.Q.C.L.

a
the E.

b
to om. E.C.L.

1
T. C. lib. i. p. 92. (292 D.) [al. 154. ap. Whitg. Def. 695. “As for the making of the orders and ceremonies of the Church, they do (where there is a constituted and ordered church,) pertain unto the ministers of the Church and to the ecclesiastical governors; and as they meddle not with the making of civil laws for the commonwealth, so the civil magistrate hath not to ordain ceremonies pertaining to the Church.”]

c
the om. E.Q.

d
these E.Q.C.L.

e
be lords om. D.

f
of all other is E. is most proper of all other C.

g
ecclesiastical persons E.C.

h
and E. and the C.L.

1
[De Rep. iii. ap. Lactant. vi. 8.]

i
part D.

k
his E.C.L. [Fulm. this.]

l
the E.L.

m
though E. however C.

n
of om. C.L.

o
especially is E.

p
establisheth them E.C.L.Q.

q
deliver om. E.Q.C.L.

s
is E.

t
that om. D.

u
did forbid E.

x
them E.

y
thereby E.Q.

z
laws E.C.

a
the om. E.Q.C.L.

b
deprive himself thereof E.

c
a om. E.Q.C.L.

d
own om. E.Q.C.L.

e
his E. [Fulm. this] L.

ee
head E′. Gauden, ’62, ’76, ’82.

f
on D.

g
needs om. E.Q.C.L.

h
unto E.Q.

i
of E.

k
matters E.C.L.

l
idolatrous E.Q.C.L.

m
and E.C.

n
as om. E.

o
true and Christian D.

p
The passage which follows, down to “over the Church,” p. 419, is placed by the Dublin MS. before “There are which wonder,” &c. c. vi. 9. The margin of D. has, “Power to make laws.”

q
the E.C.L.

r
for D.

1
T.C. lib. iii. p. 159 (51 E.) [T. C. i. 193. al. 155. ap. Whitg. Def. 701, says, “We say, that if there be no lawful ministry to set good orders (as in ruinous decays and overthrows of religion,) that then the prince ought to do it; and if (when there is a lawful ministry) it shall agree of any unlawful or unmeet order, that the prince ought to stay that order, and not to suffer it, but to drive them to that which is lawful and meet.” And iii. 159: (quoting Jewel and Nowell;) “ ‘Christian princes have rather to do with these matters than ignorant and wicked priests . . . In case of necessity (meaning when the ministry is wicked) the prince ought to provide for convenient remedy:’ the very selfsame thing which we maintain, in saying, when there is no lawful ministry, that then the prince ought to take order in these things.”

s
endued E.

t
remarkable E.C.L.

u
wheresoever E.Q.C.L.

x
is no E.

y
ministry? D.

z
quality? D.

a
dealing for ever with affairs E.

b
transfers E.

c
the om. E.C.Q.

1
[See at the end of Greenwood’s “Answer to G. Gifford’s pretended Defence of Read Prayers,” 1590, a circular letter from the bishop of London (Aylmer) to his clergy, with “A Brief of the Positions holden by the new sectory of Recusants:” of which the 10th is, “That if the prince, or magistrate under her, do refuse, or defer to reform, such faults as are amiss in the Church, the people may take the reforming of them into their own hands, before or without her authority.” And in a subsequent paper, Art. 6. “They affirm that the people must reform the Church and not tarry for the magistrate.” Their own reply is, “We go not about to reform your Romish bishopricks, deans, officers, advocates, courts, canons, neither your popish priests, half priests, ministers, all which come out of the bottomless pit: but we leave those merchantmen and their wares with the curse of God upon them until they repent . . . We are to obey God rather than man, and if any man be ignorant let him be ignorant still. We are not to stay from doing the Lord’s commandment upon the pleasure or offence of any.”]

d
to D.

e
and to the D.E.

f
the great, by the poor and the simple; some Kniperdoling, &c. E. Gauden. Kimperdoling E. G. Kniperdoling 1676. But all the MSS. omit by: which seems to indicate the change of punctuation here adopted.

2
[Bernard Knipperdoling, of Munster, one of the leaders of the anabaptists in the tumult of 1533, and designated by Sleidan as ‘facile primus ejus factionis.” Commentar. b. x. f. 106. ed. Argentorat. 1559. “Vaticinatur Cnipperdolingus, fore ut in summo gradu collocati deturbentur, alii autem e sordibus et infimis emergant subselliis: deinde jubet omnia templa destrui.” Ibid.]

g
this E.C.L.

h
and orders om. D.

i
very om. E.

k
so om. E.

l
qu. contentious?

m
stifling E. [trifling Fulm. Q. in marg.]

n
authorized kings E.

o
till it were well E.

p
strong E.Q.C.L.

q
whereof E. hereof C.

r
indeed lawful for kings E.C.L.

s
these E.C.L.

t
foresaid om. E.C.L.Q.

u
reason C.

1
T. C. lib. i. p. 192. [al. 153. ap. Whitg. Def. 694.]

x
vid. p. 17. [marg. D.]

2
Apol. fol. 40*. p. 2. [c. iv. p. 67. “Ad terrenam spectat potestatem, quam Deus illis largitus est, ecclesiæ leges defendere, negotiorum suscipere executionem, et punire rebelles atque transgressores.”]

*
4 D.

y
punish rebels and transgressors E.Q.C.L.

z
the laws E.Q.C.L.

a
the church D.

z
that om. E.Q.C.L.

a
doth D.

1
[Here in E.Q.C.L. ends the treatise on Legislative Supremacy, and the section “Touching the king’s supereminent authority,” &c. (c. viii.) begins. But in D. the following passage is inserted: which, occurring as it does afterwards, the first part of it almost verbatim, was probably put here as a note in the copy from which that MS. was transcribed, and got by mistake into the text. (It appears also in Cl. Trab. p. 71.) “Wherein it is, from the purpose altogether, alleged, that Constantine,” &c. (as in c. viii. § 8. to “a matter of theirs:) all which hereupon may be inferred reacheth no further than only unto the administration of church affairs, or the determination of strifes and controversies* rising about the matter† of religion: it proveth that in former ages of the world it hath been judged most convenient for church officers to have the hearing of causes merely ecclesiastical, and not the emperor himself in person to give sentence of them. No one man can be sufficient for all things. And therefore public affairs are divided, each kind in all well-ordered states allotted unto such kind of persons as reason presumeth fittest to handle them. Reason cannot presume kings ordinarily so skilful as to be personal judges meet for the common hearing and determining of church controversies; but they which are hereunto appointed, and have all their proceedings authorized by such power as may cause them to take effect. The principality of which power in making laws, whereupon all these things depend, is not by any of these allegations proved incommunicable unto kings.”]

*
controversy Cl. Tr.

†
matters Cl. Tr.

2
[In a second instance here the order of the Dublin MS. fol. 107; and of Cl. Trab. p, 72, has been departed from; the following passage to the end of this section, as they give it, is quite incoherent, followingthe extract given above, (note 1,) in this way: “The principality of which power in making laws whereupon all these things depend, is not by any of these allegations proved incommunicable unto kings, although not both in such sort,” &c. This being clearly wrong, and the passage as it stands in the text fitting in tolerably well, perhaps the insertion of it on conjecture may not seem too bold.]

3
T. C. lib. i. p. 193. [al. 154. ap. Def. 698. “By the emperor’s epistle in the first action of the council of Constantinople...it appeareth that it was the manner of the emperors to confirm the ordinances which were made by the ministers, and to see them kept.”]

b
These sentences from “although not both,” p. 418, l. 4, occur only in D. They are followed by the passage “There are which wonder,” &c. c. vi. 9. to “Christian religion,” p. 415. After which at an interval, that MS. proceeds with the words “Touching the advancement,” &c. as in the text.

c This side-note from Cl. Trab. as are all the various readings in this seventh chapter.
1
[Vid. Sarav. De Imp. Auct. et Christian. Obedient. lib. iii. c. 37. “Sacerdotii præcipua pars relicta regibus.” In the coronation of the emperors of Germany at Aix la Chapelle, after their anointing, they put on a deacon’s habit: (Goldast. Polit. Imp. p. 71, 80, 95.) “quem amictum quondam imperator Carolus Magnus gestaverat.” ibid. p. 144.]

d
the people.

e
the.

f
only om.

g
sometimes.

1
Pseud. Ambros. in 4 ad Ephes. [v. 11, 12. “Non per omnia conveniunt scripta apostoli ordinationi quæ nunc in ecclesia est: quia hæc inter ipsa primordia sunt scripta. Nam et Timotheum presbyterum a se creatum episcopum vocat; quia primi presbyteri episcopi appellabantur; ut recedente eo, sequens ei succederet . . . Sed quia cœperunt,” &c. t. ii. Ap. 241.]

h
judicio om.

i
the favour.

2
In Vit. Cypr. [§ 5.]

k
chose.

3
Nulla ratio. Dist. 63. [it should be 62. § 1. Dec. Grat. pars i. p. 311. He adds, “Nec a comprovincialibus episcopis cum metropolitani judicio consecrati.” See his Canonical Epistle to Rusticus, archbishop of Narbonne, t. i. 406, ed. Quesnel. circ. ad 450: and compare the canonical letter of Cœlestine to the bishops of Gaul, ad 428; can. v. “Nullusinvitis detur episcopus: cleri, plebis, et ordinis consensus et desiderium requiratur.” Conc. Hard. i. 1260.]

4
Ep. Honor. Imp. ad Bonif. Concil. tom i. [col. 1238. ed. Hard. “Beatitudine tua prædicante, id ad cunctorum clericorum notitiam volumus pervenire, ut si quid forte religioni tuæ (quod non optamus) humana sorte contigerit, sciant omnes ab ambitionibus esse cessandum. At si duo contra fas temeritate certantes fuerint ordinati, nullum ex his futurum penitus sacerdotem, sed illum solum in sede apostolica permansurum, quem ex numero clericorum nova ordinatione divinum judicium et universitatis consensus elegerit.” Circ. ad 419.]

l
places.

m
munificence.

n
seasonable.

o
the law.

1
25 Ed. 3. [c. 6. A Statute of Provisors, reciting the Statute of Carlisle, 25 Edw. i. c. 4. preamble: “Whereas the holy Church of England was founded in the estate of prelacy within the realm of England, by king Edward and his progenitors, and the earls, barons, and other nobles of his said realm, and their ancestors, to inform them and their people of the law of God, and to make hospitalities, alms, and other works of charity, in the places where the churches were founded, for the souls of the founders, their heirs, and all Christians; and certain possessions, as well in fees, lands, rents, as in advowsons, which do extend to a great value, were assigned by the said founders to the prelates and other people of the holy Church of the said realm, to sustain the same charge, and especially of the possessions which were assigned to archbishops, bishops, abbots, priors, religious and all other people of holy Church, by the kings of the said realm, earls, barons, and other great men of his realm; the same kings, earls, barons and other nobles, as lords and advowees, have had and ought to have the custody of such voidances, and the presentments and the collations of the benefices being of such prelacies.” &c. ad 1350.]

p
to.

1
Ibid. [§ iii. “The election was first granted by the king’s progenitors upon a certain form and condition, as to demand licence of the king to chuse, and after the election to have his royal assent, and not in other manner.” Stat. at Large, by Ruffhead and Runnington, t. i. 260, 62.]

2
25 Hen. VIII. c. 20. [§ iv. “Be it ordained and established by the authority aforesaid, that at every avoidance of every archbishoprick or bishoprick . . . the king . . . may grant to the prior and convent, or the dean and chapter of the cathedral churches or monasteries where the see . . . shall happen to be void, a licence under the great seal . . . to proceed to election . . . with a letter missive, containing the name of the person which they shall elect.” § vii. “If the prior and convent of any monastery, or dean and chapter of any cathedral church, . . . proceed not to election and signify the same according to the tenor of this act, within the space of twenty days next after such licence shall come to their hands: or else if any archbishop or bishop, . . . shall refuse, and do not confirm, invest, and consecrate, with all due circumstance . . . every such person as shall be so elected, nominate, or presented . . . . within twenty days next after the king’s letters patents . . . . shall come to their hands . . . . then every prior and particular person of his convent, and every dean and particular person of the chapter, and every archbishop and bishop, and all other persons so offending . . . . shall run in the dangers, pains, and penalties of the estatute of Provision and Præmunire:” i.e. imprisonment, outlawry, and forfeiture of lands and goods.]

3
C. Nullus, Dist. 63. [Decret. Gratian. pars i. dist. 62. § 3. “Nullus in episcopum nisi canonice electum consecret. Quod si præsumptum fuerit, et consecrans et consecratus absque recuperationis spe deponatur.” This is the tenth Canon of the first Lateran council, held under Calixtus II, ad 1123. See Concil. Hard. t. vi. pars ii. p. 1112.]

q
for Cl. Tr.

1
Tom. i. Concil. [i. 1237. ed. Hard. “Ecclesiæ meæ, cui Deus noster meum sacerdotium, vobis res humanas regentibus, deputavit, cura constringit, ne causis ejus, quamvis adhuc corporis incommoditate detinear, propter conventus, qui a sacerdotibus universis et clericis, et Christianæ plebis perturbatoribus agitantur, apud aures Christianissimi principis desim.”]

2
Onuphr. [Onuphrius Panvinius, of Verona, 1529-1568, annotated and continued the Lives of the Popes, by Platina, 1421-1481] in Pelag. II. [in his note on Platina’s life of that pope, who was next before S. Gregory the Great; and of whom Platina had remarked, that owing to the Lombards who beset the city, he was elected without the emperor’s consent; ad 577: “Nil enim tum a clero in eligendo Pontifice actum erat nisi ejus electionem Imperator approbasset.” On which Onuphrius observes, “Gotthis Italia omni per Narsem Patricium pulsis, eaque cum urbe Roma Orientalis imperii parte facta sub Justiniano Imperatore, ex auctoritate Papæ Vigilii, novus quidam in comitiis Pontificiis mos inolevit. Is fuit, ut mortuo Papa, nova quidem electio more majorum statim a clero S.P.Q.R. fieret, verum electus Romanus Pontifex non ante consecrari atque ab Episcopis ordinari posset, quam ejus electio ab Imperatore Constantinopolitano confirmata esset, ipseque literis suis patentibus licentiam electo Pontifici concederet, ut ordinari et consecrari posset.” p. 75. ed. Colon. 1626.]

3
[Benedict II. ad 684. “Ad hunc Constantinus Imperator hominis sanctitate permotus, sanctionem misit, ut deinceps quem clerus, populus, exercitusque Romanus in Pontificem delegisset, eundem statim verum Christi vicarium esse omnes crederent; nulla aut Constantinopolitani Principis aut Italiæ exarchi exspectata auctoritate, ut antea fieri consueverat.” Ibid. p. 93.]

1
[Grat. Decr. pars i. dist. 63. c. Hadrianus. (ad 774.) Carolus . . . “constituit synodum cum Hadriano papa in patriarchatu Lateranensi, in ecclesia Sancti Salvatoris: quæ synodus celebrata est a cliii episcopis religiosis et abbatibus. Hadrianus autem papa cum universa synodo tradiderunt Carolo jus et potestatem eligendi pontificem, et ordinandi apostolicam sedem . . . Insuper archiepiscopos et episcopos per singulas provincias ab eo investituram accipere definivit; ut nisi a rege laudetur et investiatur episcopus, a nemine consecretur: et quicunque contra hoc decretum ageret, anathematis vinculo eum innodavit.” col. 322. Lugd. 1572. This seems to have been altogether false, though a story current in the time of Gratian, (ad 1131,) who took it from an interpolated copy of the Chronicle of Sigebert. (ad 1101.) Vid. Pagi in Ann. Baron. iii. 341.]

2
[In council at Rome, ad 1080, in which Henry IV. was finally deposed, and Rodolph of Suabia confirmed emperor in his place. Canon i. “Sequentes statuta sanctorum patrum . . . decernimus . . . ut siquis deinceps episcopatum vel abbatiam de manu alicujus laicæ personæ susceperit, nullatenus inter episcopos vel abbates habeatur . . . Insuper etiam ei gratiam S. Petri et introitum ecclesiæ interdicimus” . . . ii. “Item, si quis imperatorum, regum, ducum, marchionum, comitum, vel quilibet sæcularium potestatum ac personarum investituram episcopatuum vel alicujus ecclesiasticæ dignitatis dare præsumpserit, ejusdem sententiæ vinculo se obstrictum esse sciat.” Conc. Hard. t. vi. pars i. col. 1587.]

r
needeth.

3
C. Reatin. Dist. 63. [Decr. Grat. pars i. d. 63. § 16. “Reatina ecclesia, quæ per tot temporum spatia pastoralibus curis destituta consistit, dignum est ut brachio amplitudinis vestræ sublevetur, ac gubernationis regimine protegatur. Unde salutationis alloquio præmisso, vestram mansuetudinem deprecamur, quatenus Colono humili diacono eandem ecclesiam adregendam concedere dignemini: ut vestra licentia accepta, ibidem eum, Deo adjuvante, consecrare, valeamus episcopum.” circ. ad 847. The Church was greatly depressed at that time, the Saracens often ravaging Italy to the very gates of Rome.]

s
of om.

t
please, &c.

u
highnesses.

1
Walthramus [Waleran, Bp. of Naumburg, 1089-1111] Naumburgensis, de Investit. Episcoporum per Imperator. facienda. [ap. Schardium, “Sylloge Historico-Politico-Ecclesiastica, de Discrimine Potestatis imperialis et ecclesiasticæ.” pp. 72-74, Argentorat. 1618, [published by Ulric Hutten, 1520.] The tract was written, ad 1109: by a German bishop, a strong partisan of the imperial side.]

2
[Plat. vit. Greg. VII. p. 165. ad 1373. “Adeptus pontificatum Gregorius, statim Henricum imperatorem admonet, ne deinceps largitione corruptus, episcopatus et beneficia alicui per simoniacam cupiditatem committat, aliter se usurum in se et delinquentes censuris ecclesiasticis.”]

x
dealing.

3
[Viz. Sylvester, Gregory I. Adrian I. Leo (III?) Leo (IV?) and Benedict (III?) Walthram, 73 A.]

y
further om.

z
other om.

a
hereunto.

4
[Ibid. “Legitur etiam de episcopis Hispaniæ, Scotiæ, Angliæ, Ungariæ, quomodo ex antiqua institutione, usque ad modernam novitatem, per reges introierint, cum pace temporalium, pure et integre.”]

b
the.

1
[Ibid. p. 72. “Qui a primo Constantino gesta et decreta revolvit, patenter inveniet, quod per reges et imperatores et devotos laicos Romana ecclesia, aliæque in orbe terrarum ecclesiæ, in fundis et mobilibus ditatæ et exaltatæ sint; sibique tutelas et defensiones contra tyrannos et raptores retinuerint, ut gladius regalis et stola Petri sibi invicem subveniant, quasi duo cherubin conversis vultibus respicientia in propitiatorium.”]

c
understood.

2
[Ibid. p. 73. “Episcopatus qui sub Romano degunt imperio, majoribus fundis et amplioribus vigent justitiis: et ideo propter majus scandalum a stola Petri disertius tractandi sunt: quia non omnes sunt Petrus, qui tenent sedem Petri.” . . “Postquam a Sylvestro per Christianos reges et imperatores dotatæ, ditatæ, et exaltatæ sunt ecclesiæ in fundis et aliis mobilibus, et jura civitatum in teloneis, monetis, villicis, &c. . . . . per reges delegata sunt episcopis; congruum fuit et consequens ut rex qui unus est in populo, et caput populi, investiat et inthronizet episcopum: et contra irruptionem hostium sciat cui civitatem suam credat, cum jus suum in domum illorum transtulerit.”]

3
[Ibid. “Longe ante decretum Adriani papæ, ejusque successorum, reges, qui erant uncti, et majores domus, investituras episcoporum fecerunt.”]

4
[Ibid. “Nihil refert, sive verbo, sive præcepto, sive baculo, sive alia re quam in manu tenuerit, investiat aut inthronizet rex et imperator episcopum, quo die consecrationis veniens, annulum et baculum ponit super altare, et in curam pastoralem singula accipit a stola et authoritate S. Petri. Sed congruum magis est per baculum, qui est duplex, i. e. temporalis et spiritualis.”]

d
save.

1
[Francisci Duareni, [1509-1559.] Biturig. “De Beneficiis et ad ea pertinentibus, libri viii.” [Paris, 1551.] ap. Tract. Illustr. Jurisc. Ven. 1584. t. xv. pars ii. The author was accounted by Thuanus one of the most distinguished of the French jurists of the sixteenth century.]

2
[Jean Papon, a lawyer in the service of Catharine de’Medici, and author of a work called Notaire, or Secrets de Notaire, in three parts, in the third of which, b. iii. p. 155, &c. is a statement and vindication of the rights of the Crown of France in the matter of presentation to benefices.]

3
[“De Sacra Politia forensi,” [Par. 1577.] 1589. Vid. supr. c. ii. § 14. note 3.]

4
[Ægidius de Columna, archbishop of Bourges, †1316, contemporary with Boniface VIII. and tutor to Philip the Fair: in his “Quæstio de Utraque Potestate,” inserted by Goldastus in Monarch. S. Rom. Imp. t. iii. 95, &c.]

5
[Ægidius Magister, “De Regaliis,” in Tract. Illustr. Jurisc. t. xiii. pars ii. p. 437, &c.]

6
[Arnulphus Ruzæus, “De Jure Regaliæ.” [Par. 1534, 1551.] Ibid. t. xii. 357, &c.]

7
[Petrus Costalius, “Adversaria ex Pandect. Justin.” lib. i. p. 49. Colon. 1560.]

8
[Philippus Probus [= Prudhomme], Bituricus, “De Jure Regaliæ,” in Tract. Illustr. Jurisc. t. xii. 389, &c. v. Biog. Univ. Supplem. art. Ruzé].]

9
Cap. general. de Elect. i. 6. [In 2 Conc. Lugd. ad 1274, can. 12, Generali constitutione sancimus, universos et singulos, qui regalia, custodiam, sive guardiam advocationis, vel defensionis titulum, in ecclesiis, monasteriis, sive quibuslibet aliis piis locis, de novo usurpare conantes, bona ecclesiarum, monasteriorum, aut locorum ipsorum vacantium occupare præsumunt, quantæcunque dignitatis honore præfulgeant, . . . eo ipso excommunicationis sententiæ subjacere. . . . Qui autem ab ipsarum ecclesiarum cæterorumque locorum fundatione, vel ex antiqua consuetudine, jura sibi hujusmodi vindicant, ab illorum abusu sic prudenter abstineant, et suos ministros in eis solicite faciant abstinere, quod ea quæ non pertinent ad fructus sive reditus provenientes vacationis tempore non usurpent; nec bona cætera, quorum se asserunt habere custodiam, dilabi permittant, sed in bono statu conservent.” Conc. Hard. vii. 711.]

n
prerogative.

o
the om.

1
Hieron. adv. Jovin. i. [19. “Nonnunquam errat plebis vulgique judicium, et in sacerdotibus comprobandis unusquisque suis moribus favet, ut non tam bonum quam sui similem quærat præpositum.”]

p
dishonest.

2
L. 7. Ep. 5. [“Ecclesia” (Bituricarum, i. e. Bourges,) “nuper summo viduata pontifice, utriusque professionis ordinibus ambiendi sacerdotii quodammodo classicum cecinit. Fremit populus per studia divisus: pauci alteros, multi sese non offerunt solum, sed inferunt. Si aliquid pro virili portione secundum Deum consules, veritatemque, omnia occurrunt levia, varia, fucata: et quid dicam? sola est illic simplex impudentia.” In Bibl. Patr. Colon. t. v. pars i. p. 1022.]

1
Theod. l. v. c. 27. Sozom. l. viii. c. 2. [ψηϕισαμένων δὲ του̑το του̑ λαου̑ καὶ του̑ κλήρου, καὶ ὁ βασιλεὺς συνῄνει. Nectarius was his predecessor, not his competitor.]

2
[Amm.] Marcell. l. xv. [p. 24. c. 3. “Nec corrigere sufficiens nec mollire, coactus magna vi secessit in suburbanum.”] Socr. lib. ii. c. 27. et iv. c. 29. [(after the election,) συμπληγάδες τω̑ν ὀχλω̑ν ἐγίνοντο· ὥστε καὶ ἐκ τη̑ς παρατριβη̑ς πολλοὺς ἀποθανει̑ν, καὶ διὰ του̑το πολλοὺς λαικούς τε καὶ κληρικοὺς ὑπὸ του̑ τότε ἐπάρχου Μαξιμίνου τιμωρηθη̑ναι. ad 366.] Sozom. lib. vi. c. 23.

3
Socr. ii. 27. [Μακεδόνιος τω̑ν ἐκκλησιω̑ν ἐγκρατὴς. . . . Χριστιανικὸν ἐκίνησε πόλεμον, οὐχ ἥττονα ἢ ὑπὸ τὸν αὐτὸν χρόνον ἐποίουν οἱ τὑραννοι. ad 356.] Soz. iv. 11. [ὡς εἰσήλαυνεν εἰς Ῥώμην ὁ βασιλεὺς, . . . πολὺς ἠ̑ν ὁ ἐνθάδε δη̑μος περὶ Λιβερίου ἐκβοω̑ν, καὶ δεόμενος αὐτὸν ἀπολαβει̑ν.] Theodor. ii. 15, 16, 17: [concerning the expulsion of Liberius bishop of Rome by the emperor Constantius, and the discontent of the people in his absence, ad 357.]

4
Pontius in Vit. Cypr. c. 5. [“Invitus dico, sed dicam necesse est. Quidam illi restiterunt, etiam ut vinceret; quibus tamen quanta lenitate, quam patienter, quam benevolenter indulsit! quam clementer ignovit, amicissimos eos postmodum, et inter necessarios computans, mirantibus multis! Cui enim posset non esse miraculo tam memoriosæ mentis oblivio?”]

q
our om.

1
C. Sacror. Can. dist. 63. [Grat. Decr. i. from Capitul. Carol. et Ludovic. l. i. “Sacrorum canonum non ignari, ut in Dei nomine sancta Ecclesia suo liberius potiretur honore, assensum ordini ecclesiastico præbuimus, ut scil. episcopi, per electionem cleri et populi, secundum statuta canonum, de propria diœcesi, remota personarum et munerum acceptione, ob vitæ meritum et sapientiæ donum eligantur, ut exemplo et verbis sibi subjectis undequaque prodesse valeant.”]

2
C. Lectis. dist. 63. [from a letter of Stephen to a count Guido, relating to the consecration of a bishop for the church of Reate. “Scientes ecclesiam Dei sine proprio pastore non debere consistere, gloriæ vestræ mandamus, quoniam aliter nos agere non debuimus, ut a vestra solertia imperiali (ut prisca consuetudo dictat) percepta licentia, et nobis, quemadmodum vos scire credimus, imperatoria directa epistola, tunc voluntati vestræ de hoc parebimus, et eundem electum, Domino adjuvante, consecrabimus.”]

3
Archbishop Ussher has corrected this to fourth.

r Their om. E.C. no marginal head Q.
s whatsoever om. E.C.
t
The Dublin MS. has an interval of seven pages between this and the preceding dissertation.

u
the judging D.

x
the sacraments E.

y
the bishops E.C.

z
the Church E. the courts D.

a
do rise E.

b
matter D.

c
account E.Q.C.L.

d
this E.Q.C.L.

e
an E.

f
this E.Q.L.

g
the order E.

h
of om. E.Q.C.L.

i
serves D.

k
Josiah E.C.

1
2 Chron. xxiv. 4-9.

l
“Go out, &c.” (not giving the quotation at length.) D.

m
all om. E.

n
“the Lord” in later editions, “God” E.C.

o
the Lord God E.C.

oo
Balaam E′.

p
the Lord E.C.

2
2 Chr. xxx. 6.

q
these E.C.L.Q.

r
sometimes E.Q.L.

s
solemn om. E.C.L.

3
Josh. i. 18.

t
“and will. . . . courage” om. D.

u
that om. E.

x
law D.

y
supreme governors E.C.

z
who E.Q.C.L.

a
any E.C.L.Q.

b
same law E.

1
Just. Instit. l. iv. tit. 1. de Offic. Judic.

c
aut om. E.Q.C.L.

d
ut Imperator Justinianus E.C.

e
own om. E.Q.

f
the D.

g
belonging to E.C.

h
any of E. any thing the practice C.L.Q.

i
there om. D.

k
imparting E.C.

l
and D.

m
that D.

n
cause E.

1
1 Eliz. cap. 1.

o
schisms, heresies E. schisms om. C.

p
and E.Q.C.L.

q
the violation E.Q.C.L.

r
laws E.C.

s
motion E.Q.C.L.

t
commission E. commissioners Q.C.L.

u
appeal: and E.Q.C.L.

x
jurisdictions E.Q.C.L.

2
[Alexander III. in the arrangement made after the murder of the Archbishop of Canterbury, ad 1172.]

y
sometimes E.Q.C.L.

z
the om. E.

a
have appeal E.

1
Machiavel. Hist. Florent. lib. i. [“Che dovesse annullare tutte le cose fatte nel suo regno in disfavore della libertà ecclesiastica; e dovesse acconsentire, che qualunque suo soggietto potesse volendo appellare a Roma: le quali cose furono tutte da Enrico accettate, e sottomessesi a quel giudicio un tanto Re, che oggi un uomo privato si vergognarebbe a sottomettersi.” p. 21. ed. Genev. 1550.]

a
causes E.

b
appeals, but appeals made E.Q.C.L.

2
25 Hen. VIII. c. 19.

c
any certain particular E.Q.C.L.

d
in E.C.L.

e
favour or preferment E.Q.C.

f
the D.

g
of E.

h
judgment in E.

o
things, persons E.Q.C.L.

oo
supreme power E.

p
jurisdictions E.Q.C.L.

q
incite E.

r
we are herein E.C.

1
T. C. l. iii. p. 154. 2 Chron. xix. 5. Heb. v. 1.

s
Apostles E. Apostle to the Hebrews Q.

t
high om. D.

u
but E.Q.C.L.

x
his E.Q.L.

y
ergo E.Q.C.L.

z
three om. E.C.

a
the Church D.

2
Heb. v. i.

b
sin E.Q.C.L.

c
alone only } D.

d
but in spiritual or church affairs, (as hath been already shewed) it was E. The whole clause om. from “Church affairs” just before C.L.

e
kings only E.

f
so E.Q.C.L.

ff
so E′. politie Gauden.

g
priest E.C.

h
they say om. E.

i
whereof E.C.L.Q.

k
to E.

l
the pastors E.

m
the E.C.L.Q.

n
nor E.

o
at war E.D. in Q.C.L.

p
notwithstanding his power must E.

q
unto, even E.C.L.

r
more either E.Q.C.L.

s
of D.

t
plainly om. E.

1
Staunf. Pleas of the Crown, l. ii. c. 3. [fol. 54. ed. 1574. “Le Roy in person ne peut estre judge ne seer in judgment in treason ou felony, eo quod il est un des parties al judgment.”]

u
there E.C.L.

x
What follows does not appear in the first edition, but was added, in 1662, by Bishop Gauden.

y
consideration D.

z
consideration D.

a
hath transcendent E.Q.C.L.

b
when E.Q.C.L.

c
enemies E.C.

d
this unresistible E.Q. an unresistable C.

e
the E.Q.C.L.

f
that E.C.L.

1
T. C. lib. iii. p. 155.

1
Euseb. de Vita Constant. lib. iv. [c. 24. Ἐν ἑστιάσει ποτὲ δεξιούμενος ἐπισκόπους, λόγον ἀϕη̑κεν, ὡς ἄρα εἴη καὶ αὐτὸς ἐπίσκοπος, ὡ̑δέ πη αὐτοι̑ς εἰπὼν ῥήμασιν ἐϕ’ ἡμετέραις ἀκοαι̑ς· ἀλλ’ ὑμει̑ς μὲν τω̑ν εἴσω τη̑ς ἐκκλησίας, ἐγὼ δὲ τω̑ν ἐκτὸς ὑπὸ Θεου̑ καθιστάμενος ἐπίσκοπος ἂν εἴην.]

2
Aug. Ep. 162. [al. 43. c. 7. t. ii. 297. “Neque enim ausus est Christianus imperator sic eorum tumultuosas et fallaces querelas suscipere, ut de judicio episcoporum qui Romæ sederant ipse judicaret; sed alios, ut dixi, episcopos dedit.”] Ep. 166. [al. 105. c. 2.] t. ii. 299, [43. 20. “Eis” (Donatistis) “ipse cessit, ut de illa causa post episcopos judicaret, a sanctis antistitibus postea veniam petiturus.” t. ii. 97.]

3
Besides these testimonies of antiquity which Mr. Cartwright bringeth forth, D. Stapleton, who likewise (Doct. Prin. l. 5. cont. 2. c. 18.) citeth them one by one to the same purpose, hath augmented the number of them by adding other of the like nature: namely, how Hosius the bishop of Corduba (apud Athan. in. Ep. ad Solit. Vit. agentes*) answered the emperor, saying, “God hath committed to thee empire; with those things that belong to the Church he hath put us in trust.” How Leontius bishop of Tripolis (Suid. in verb. Leontius†) also told the selfsame emperor as much: “I wonder how thou, which art called unto one thing, takest upon thee to deal in another. For being placed in military and politic affairs, in things that belong unto bishops alone thou wilt bear rule.”

*
[Hist. Arian. ad Monach. t. i. 371. ed. Bened. Μὴ τίθει σεαυτὸν εἰς τὰ ἐκκλησιαστικὰ, μηδὲ σὺ περὶ τούτων ἡμι̑ν παρακελεύου· ἀλλὰ μα̑λλον παρ’ ἡμω̑ν σὺ μάνθανε ταυ̑τα· σοὶ βασίλειαν ὁ Θεὸς ἐνεχείρισεν, ἡμι̑ν δὲ τὰ τη̑ς ἐκκλησίας ἐπίστευσε.]

†
[Θαυμάζω, ὅπως ἕτερα διέπειν ταχθεὶς, ἑτέροις ἑπιχειρει̑ς· στρατιωτικω̑ν μὲν καὶ πολιτικω̑ν πραγμάτων προεστηκὼς, ἐπισκόποις δὲ περὶ τω̑ν εἰς μόνους ἐπισκόπους ἡκόντων διαταττόμενος. This is conjectured to be an extract from Philostorgius.]

4
Hilar. ad Constant. lib. i. § 1. [“Provideat et decernat clementia tua, ut omnes ubique judices, quibus provinciarum administrationes creditæ sunt, ad quos sola cura et solicitudo publicorum negotiorum pertinere debet, a religiosa se observantia abstineant.” col. 1218. ed. Bened.]

r
only commonwealth matters E.

5
Ambros. lib. v. Ep. 33. [al. 20. § 16. by an error of the press in the Benedictine edition, for § 19. “Ad imperatorem palatia pertinent, ad sacerdotem ecclesiæ. Publicorum tibi mœnium jus commissum est, non sacrorum.” II. 857.]

s
the authority E.Q.C.L.

1
[“Ambrose hath a worthy saying, wherein he plainly noteth both what a Christian prince may do in these things that appertain unto the Church, and how a godly bishop should in that case behave himself. ‘When it was proposed unto me,’ saith he, ‘that I should deliver the plate or vessel of the Church, I made this answer: If there were any thing required that was my own, either land, house, gold or silver, being of my own private right, that I would willingly deliver it: but that I could not pull any thing from the Church of God. And moreover I said, that in so doing I had regard to the emperor’s safety, because it was not profitable either for me to deliver it, or for him to receive it. Let him receive the words of a free minister of God: if he will do that is for his own safety, let him forbear to do Christ injury.’ ” Bishop Cooper’s Adm. p. 212.]

2
[T. C. i. 193. al. 154. ap. Whitg. Def. 700.]

3
[See Epistle 21, throughout.]

t
consistories E.C.L.

u
D inserts here in the text, “Besides these testimonies,” &c. (as in note 3, p. 440.)

x
means E.C.L.Q.

y
custom E.Q.C.L.

z
fit and lawful C.

a
hath E.

b
ratifieth E.Q.L.

c
the bishops E.Q.C.L.

d
special E.

1
Ep. 68. [D. al. 88. § 3. t. ii. 162. C, D. Ed. Bened. Antwerp. 1700.]

e
it pleased C.L.

f
therewith om. E.Q.C.L.

g
sort modestly D.

h
is E.C.

i
odds was between E.Q. is C.

k
his E.Q.C. this L.

2
Ep. xx. § 16. (19.) “Mandatur denique, ‘Trade basilicam.’ Respondeo, ‘Nec mihi fas est tradere, nec tibi accipere, imperator, expedit. Domum privati nullo potes jure temerare, domum Dei existimas auferendam?’ Allegatur, imperatori licere omnia, ipsius esse universa. Respondeo, ‘Noli te gravare, imperator, ut putes te in ea, quæ divina sunt, imperiale aliquod jus habere. Noli te extollere, sed si vis diutius imperare, esto Deo subditus. Scriptum est, quæ Dei Deo, quæ Cæsaris Cæsari.’ ” t. ii. 857.]

l
that are E. which are Q.C.L.

m
whomsoever he pleaseth, but E.

n
habitation E.Q.C.

o
do E.

p
wicked heretics E.Q.C.L.

q
decision E.L. correction C.

r
prelates E.D.

s
or D.

t
the head E.D.L.

tt
gave E.

u
may E.Q.C.L.

x
either the nature E.Q.C.L.

y
had om. E.

1
See the statute of Edw. I. and Edw. II. [13 Edw. I. st. 4. Circumspecte agatis; 24 Edw. I. De Consultatione; 9 Edw. II. st. 1.] and Nat. Brev. touching Prohibition, [p. 30. Lond. Tottell, 1584.] See also in Bracton these sentences, lib. v. [Tract. v.] cap. 2. “Est jurisdictio quædam ordinaria, quædam delegata, quæ pertinet ad sacerdotium, et forum ecclesiasticum, sicut in causis spiritualibus et spiritualitati annexis. Est etiam alia jurisdictio ordinaria vel delegata, quæ pertinet ad coronam, et dignitatem regis, et ad regnum in causis et placitis rerum temporalium in foro seculari.” Again: “Cum diversæ sint hinc inde jurisdictiones, et diversi judices, et diversæ causæ, debet quilibet ipsorum imprimis æstimare, an sua sit jurisdictio, ne falcem videatur ponere in messem alienam.” Again: “Non pertinet ad regem injungere pœnitentias, nec ad judicem secularem; nec etiam ad eos pertinet cognoscere de iis, quæ sunt spiritualibus annexa, sicut de decimis et aliis ecclesiæ proventionibus.” Again: “Non est laicus conveniendus coram judice ecclesiastico de aliquo, quod in foro seculari terminari possit et debeat.” [fol. 400, 401. ed. 1569.]

z
the law E.C.L.

a
and in all E.Q.C.L.

b
What follows is all found in D. alone of the MSS. with an interval of a blank leaf. But § 1, 2. is printed in Clavi Trabales, p. 92-94: as far as “to any,” p. 446.

c Harding om. E. (?)
d
disorder Cl. Tr.

e
that.

f
moved.

g
immoveable.

h
it om.

i
kingdom.

1
Deut. xvii. 15. Matt. xviii. 15.

2
1 Cor. v. 12, 13.

3
Def. Apol. part 6. p. 720. [c. 12. div. 1. p. 582. ed. 1611.]

4
[Jewel, and the Counterpoison, both read priest, not high priest.]

5
Tom. ii. f. 53. [“The Reproof of M. Dorman his proof of certain Articles of Religion, &c. continued by Alexander Nowell. With a Defence of the chief Authority and Government of Christian Princes as well in causes ecclesiastical as civil within their own dominions, by M. Dorman maliciously oppugned.” Lond. 1566. f. 51. “We profess, as doth Calvin, that the prince himself ought to be obedient to the ecclesiastical minister executing these his offices according to God’s word; yea though it be against the prince himself, according as Theodosius the emperor was in this case obedient to St. Ambrose.”]

6
Euseb. l. vi. c. 14. Theod. v. c. 18.

7
Counter[poison,] page 174. [Comp. T. C. iii. 93, for the whole of this except the reference to Bp. Jewel. And Eccl. Disc. 142, 143. “Neque vero hic magistratus, etsi in reliqua ecclesia politicæ auctoritatis ratione emineant, se ab hoc parendi et ecclesiasticis magistratibus obediendi præcepto et mandato eximendos esse arbitrentur. Quum enim non minus de magistrorum quam de aliorum salute illos solicitos esse oporteat, et illius etiam animam, ut cæterorum, sua cura contineant, illis etiam non minus quam reliquis parendum est, et ecclesiasticorum magistratuum justæ auctoritati obtemperandum. Atque cum illi Jesu Christi non solum auctoritate præsint, sed ipsam quodammodo personam sustineant, quum nullo suo imperio, sed illius solo verbo et mandato omnia administrent; annon æquum est, illis vel summos magistratus et reges ipsos obtemperare? Huic enim omnes orbis principes et monarchæ fasces suos submittere et parere decet [debent]; quem Deus regni sui hæredem, et cœli atque terræ Dominum constituit.” Then he proceeds to give examples, and dwells especially upon the cases of Philip and Theodosius.]

1
Gen. xxxvii. 7.

1
[T. C. iii. 92. “Who could be ignorant that our Saviour Christ speaketh generally when he saith, ‘if thy brother,’ &c. whereby he comprehendeth all those that are members of one church and children of one heavenly Father. In which number the Scripture reckoneth the king, whilst in that he is both called a brother, and calleth his subjects brethren. Or who could be ignorant that St. Paul subjecteth all unto this order, saving those only which are strangers from the Church. So that to say that princes are not subject unto this order, is all one as if he should say that princes pertain not to the kingdom of heaven, are none of the Church, have no part with Christ, &c. Thus is both Christ robbed of his honour, which in contempt of his order (as though it were too base for princes to go under) is himself contemned; and princes defrauded of a singular aid of salvation, and way to draw them to repentance, when they, through the common corruption, fall into such diseases against which this medicine was prepared.”]

2
Deut. xvii. 2.

a
D. has a space of half a page here.

3
Ὁ βασιλεὺς νόμοις οὐχ ὑπόκειται, ἤγουν ἁμαρτήσας οὐ κολάζεται. Καὶ κατὰ βασιλέως οἱ γενικοὶ ἤγουν οἱ καθολικοὶ κρατείτωσαν νόμοι. Harmenop. [Promptuarium Juris] l. 1. c. i. § 48 et 39. [ed. Gothofred. 1587.]

1
Def. p. 6. c. 12. div. 1.

1
Eus. l. vi. c. 33. [34. Του̑τον κατέχει λόγος Χριστιανὸν ὄντα, ἐν ἡμέρᾳ τη̑ς ὑστάτης του̑ πάσχα παννυχίδος, τω̑ν ἐπὶ τη̑ς ἐκκλησίας εὐχω̑ν τῳ̑ πλήθει μετασχει̑ν ἐθελη̑σαι, οὐ πρότερον δὲ ὑπὸ του̑ τηνικάδε προεστω̑τος ἐπιτραπη̑ναι εἰσβαλει̑ν, ἢ ἐξομολογήσασθαι, καὶ τοι̑ς ἐν παραπτώμασιν ἐξεταζομένοις μετανοίας τε χώραν ἴσχουσιν ἑαυτὸν καταλέξαι· ἄλλως γὰρ μὴ ἄν ποτε πρὸς αὐτου̑, μὴ οὐχὶ του̑το ποιήσαντα, διὰ πολλὰς τω̑ν κατ’ αὐτὸν αἰτίας παραδεχθη̑ναι. Καὶ πειθαρχη̑σαί γε προθύμως λέγεται, τὸ γνήσιον καὶ εὐλαβὲς τη̑ς περὶ τὸν θει̑ον ϕόβον διαθέσεως ἔργοις ἐπιδεδειγμένον. Comp. Chron. Alex. ad 253. p. 270. ed. Du Fresne. S. Chrys. t. xi. 531 . . 45. Suid. voc. Βαβυλα̑ς. Philostorg. vii. 8. Of which conflicting accounts the first is the only one which gives any countenance to the narration of Eusebius.]

2
Sozom. [Theod.] l. v. c. 18. [Ἀϕικόμενον εἰς τὴν Μεδιόλανον τὸν βασιλέα, καὶ συνήθως εἰς τὸν θει̑ον εἰσελθει̑ν βουληθέντα νεὼν, ὑπαντήσας (Ἀμβρόσιος) ἔξω τω̑ν προθύρων, ἐπιβη̑ναι τω̑ν ἱερω̑ν προπυλαίων τοιάδε λέγων ἐκώλυσεν· “οὐκ οἰ̑σθα, ὡς ἔοικεν, ὡ̑ βασιλευ̑, τη̑ς εἰργασμένης μιαιϕονίας τὸ μέγεθος, οὐδὲ μετὰ τὴν του̑ θυμου̑ παυ̑λαν ὁ λογισμὸς ἐπέγνω τὸ τολμηθέν. οὐκ ἐᾳ̑ γὰρ ἴσως τη̑ς βασιλείας ἡ δυναστεία ἐπιγνω̑ναι τὴν ἁμαρτίαν, ἀλλ’ ἐπιπροσθει̑ ἡ ἐξουσία τῳ̑ λογισμῳ̑· χρὴ μέντοι εἰδέναι τὴν ϕύσιν, καὶ τὸ ταύτης θνητόν τε καὶ διάρρεον, καὶ τὸν πρόγονον χου̑ν ἐξ οὑ̑ γεγόναμεν, καὶ εἰς ὃν ἀπορρέομεν· καὶ μὴ τῳ̑ ἄνθει τη̑ς ἀλουργίδος ἀποβουκολούμενον, ἀγνοει̑ν του̑ καλυπτομένου σώματος τὴν ἀσθένειαν. ὁμοϕυω̑ν ἄρχεις, ὡ̑ βασιλευ̑, καὶ μὲν δὴ καὶ ὁμοδούλων. εἱ̑ς γὰρ ἁπάντων δεσπότης καὶ βασιλεὺς, ὁ τω̑ν ὅλων δημιουργός. ποίοις τοίνυν ὀϕθαλμοι̑ς ὄψει τὸν του̑ κοινου̑ δεσπότου νεών; ποίοις δὲ ποσὶ τὸ δάπεδον ἐκει̑νο πατήσεις τὸ ἅγιον; πω̑ς δὲ τὰς χει̑ρας ἐκτενει̑ς, ἀποσταζούσας ἔτι του̑ ἀδίκου ϕόνου τὸ αἱ̑μα; πω̑ς δὲ τοιαύταις ὑποδέξῃ χερσὶ του̑ δεσπότου τὸ πανάγιον σω̑μα; πω̑ςδὲ τῳ̑ στόματι προσοίσεις τὸ αἱ̑μα τὸ τίμιον, τοσου̑τον διὰ τὸν του̑ θυμου̑ λόγον ἐκχέας παρανόμως αἱ̑μα; ἄπιθι τοίνυν, καὶ μὴ πειρω̑ τοι̑ς δευτέροις τὴν προτέραν αὔξειν παρανομίαν· καὶ δέχου τὸν δεσμὸν ᾠ̑ ὁ Θεὸς ὁ τω̑ν ὅλων δεσπότης ἄνωθεν γίνεται σύμψηϕος· ἰατρικὸς δὲ οὑ̑τος, καὶ πρόξενος ὑγιείας.” Τούτοις εἴξας ὁ βασιλεὺς τοι̑ς λόγοις· (τοι̑ς γὰρ θείοις λογίοις ἐντεθραμμένος, ᾔδει σαϕω̑ς τἰνα μὲν τω̑ν ἱερέων, τίνα δὲ τω̑ν βασιλέων ἴδια·) στένων καὶ δακρύων ἐπανη̑λθεν εἰς τὰ βασίλεια· χρόνου δὲ συχνου̑ διελθόντος· ὀκτὼ γὰρ ἀναλώθησαν μη̑νες· κατέλαβεν ἡ του̑ σωτη̑ρος ἡμω̑ν γενέθλιος ἑορτή. ὁ δὲ βασιλεὺς ἐν τοι̑ς βασιλείοις ὀλοϕυρόμενος καθη̑στο, τὴν τω̑ν δακρύων ἀναλίσκων λιβάδα· του̑το θεασάμενος Ῥουϕι̑νος· μάγιστρος δὲ τηνικαυ̑τα ἠ̑ν καὶ πολλη̑ς μέτασχε παρρησίας, ἅτε δὴ συνηθέστερος ὢν, προσελθὼν ἤρετο τω̑ν δακρύων τὸ αἴτιον· ὁ δὲ πικρω̑ς ἀνοιμώξας, καὶ σϕοδρότερον προχέας τὸ δάκρυον, “σὺ μὲν,” ἔϕη, “Ῥουϕι̑νε, παίζεις, τω̑ν γὰρ ἐμω̑ν οὐκ ἐπαισθάνῃ κακω̑ν· ἐγὼ δὲ στένω καὶ ὀλοϕύρομαι τὴν ἐμαυτου̑ συμϕορὰν λογιζόμενος, ὡς τοι̑ς μὲν οἰκέταις καὶ τοι̑ς προσαίταις ἄνετος ὁ θει̑ος νεὼς, καὶ εἰσίασιν ἀδεω̑ς, καὶ τὸν οἰκει̑ον ἀντιβολου̑σι δεσπότην· ἐμοὶ δὲ καὶ οὑ̑τος ἄβατος, καὶ πρὸς τούτῳ μοι ὁ οὐρανὸς ἀποκέκλεισται· μέμνημαι γὰρ τη̑ς δεσποτικη̑ς ϕωνη̑ς ἣ διαρρήδην ϕησὶν, ὃν ἂν δήσητε ἐπὶ τη̑ς γη̑ς, ἔσται δεδεμένος ἐν τοι̑ς οὐρανοι̑ς.” Comp. S. Ambr. Ep. li. t. ii. 997 ad circ. 390. and Paulin. vit. S. Ambros. c. 2. ibid. App. col. vii.]

1
[This passage, down to the word “evangelists,” is found verbatim in E. P. III. 9. 3. For this reason, and on account of its general irrelevancy to the subject of this Book, the editor has ventured to treat it as a separate fragment, probably of a Sermon on Obedience to Governors, annexed by mistake to the eighth book in all the MSS. but not appearing in the first edition, which breaks off abruptly in c. viii. 6. at the words “give judgment.”]

2
Rom. viii. 14.

a
admit of no E.*

*
[E. here stands for Ganden’s ed. 1662, not as before for the ed. princeps, 1648, 1651.]

b
great om. E.

c
pains as law E. pain as law L. pain as the law C.

d
the om. E.

1
[Contra Faustum, lib. xxii. 27. “Peccatum est factum vel dictum vel concupitum aliquid contra æternam legem. Lex vero æterna est ratio divina vel voluntas Dei.” t. viii. 378. f.]

e
laws E.Q.C.L.

f
the spots E.Q.C.L.

g
feel E.Q.L.

2
Psalm li. 4.

h
his word E.Q.C.L.

i
in D.

k
itself E.Q.C.L.

3
1 Pet. ii. 13.

4
Rom. xiii. 1.

5
“Verum ac proprium civis a peregrino discrimen est, quod alter imperio ac potestate civili obligatur; alter jussa principis alieni respuere potest. Illum princeps ab hostium æque ac civium injuria tueri tenetur; hunc non item nisi rogatus et humanitatis officiis impulsus,” saith Bodin, de Rep. lib. i. cap. 6. non multum a fine p. 61 B. edit. Lugd. in fol. 1586.* [Bodin was a French jurist, and secretary to the duke of Alençon, brother to Henry III. His work “de Republica” had such credit as to be used for a text book in lectures at Cambridge. Biog. Univ.]

*
Note om. D.

l
kinds E.

1
Matt. xxiii. 3.

m
ye E.C.L.

n
whatsoever simply D.

o
in om. E.

p
powers D.

2
Rom. xiii. 1.

q
orders D.

r
instituting E.Q.C.L.

s
The quotations in marg. D.

1
“A sceptre-swaying king, to whom even Jupiter himself hath given power and commandment.” Hom. II. lib. a. [ver. 279.]

t
undeniably E.C.

2
2 Chron. xix. 6.

u
him, and even E.Q.C.L.

x
thereof E.C.L.

y
be E.C.L.Q.

z
jurisdictions E.

a
clearly E.

b
to govern E.Q.

c
services E.

d
rule E.

1
Heb. xiii. 17.

e
of E.L.

f
them E.

2
[Prefixed to “A Summarie view of the government both of the Old and New Testament, whereby the episcopal government of Christ’s Church is vindicated: out of the rude draughts of Lancelot Andrews, late bishop of Winchester.” Oxford, printed by Leonard Lichfield, ad 1641. This is part of a collection entitled, “Certain brief Treatises, written by diverse learned men, concerning the ancient and modern Government of the Church: wherein both the primitive institution of Episcopacy is maintained, and the lawfulness of the Ordination of the Protestant Ministers beyond the seas likewise defended.” The other fragments are, “The original of Bishops and Metropolitans, briefly laid down by Martin Bucer, John Reinolds and James archbishop of Armagh;” “A Disquisition touching Proconsular Asia and its seven Churches,” by Ussher; “A Declaration of the Patriarchal Government of the ancient Church,” by Edward Brerewood; “A brief Declaration of the several forms of Government received in the Reformed Churches beyond the seas,” by John Durel; and “The Lawfulness of the Ordination of the Ministers of those Churches, maintained against the Romanists,” by Francis Mason. If the fragment in question be Hooker’s, (a point on which the editor does not feel entitled to express any decided opinion; but is rather inclined to hold the negative,) it may have been sketched by way of hints for the conclusion of the whole work: and for that reason it is inserted here. Compare the latter part of Cranmer’s letter to Hooker, subjoined to the fifth book in this edition.

Archdeacon Cotton informs the editor, that this paper is in the library of Trinity College, Dublin, in MS. (D. 3. 3.) in the handwriting of some person unknown, “but certainly,” Mr. Gibbings adds, the same amanuensis, who copied the latter portions of the Sermon on Pride, and also the Appendix i. to B. v. together with B. vi. This may afford a reason for ascribing the Paper to Hooker.” “The marginal references to Scripture are in Ussher’s hand, as likewise several slight corrections in the text. It is highly probable that this is the very MS. from which the printed copy was taken; more especially as at p. 5. line 22, Ussher has added a side-note to the printer, ‘a larger space betwixt these:’† which has been followed: the space left there being wider than between any other two paragraphs of the tract.” Mr. Gibbings adds that the Title or Heading is Ussher’s. But it makes no mention of Hooker, standing as follows: “The Causes of the Continuance of these Contentions concerning Church-Government.”]

†
But Ussher afterwards erased the direction:—as Mr. Gibbings informs the Editor.

1
[Possevin de Rebus Muscoviticis, p. 5. ad 1581. “Concionatores non habent, sed tantum, quas diximus, vitas sanctorum, vel eorum, quos pro sanctis venerantur, atque homiliæ partem ut dixi (a D. præsertim Chrysostomo) a Poppis suis audiunt.” Herberstein, Rerum Moscovitic. Comment. p. 31. “Doctores quos sequuntur sunt Basilius magnus, Gregorius, et Joannes Chrysostomus.” Concionatoribus carent. “Satis esse putant interfuisse sacris, ac evangelii, epistolarum, aliorumque doctorum verba, quæ vernacula lingua recitat sacrificus audivisse: ad hoc, quod varias opiniones ac hæreses, quæ ex concionibus plerumque oriuntur, sese effugere credunt.” ap. Rer. Mosc. Auct. varii, Francof. 1600. It appears from King’s Greek Church, p. 433, that Iwan Basilowitz held a synod in 1542, in which possibly the law in question might be enacted. He was very jealous of the progress of Lutheranism in Livonia. See in the same collection, p. 220, Hist. Belli Livonici, per Tilm. Bredenbach, 1563.]

1
Prov. xxii. 15.

u
the easier reuniting. So in D.

2
2 Cor. viii. 18.

3
Dan. xii. 3.

1
[Socr. E. H. i. 6. πρὸς ὀργὴν ἐξάπτεται.]

2
Lib. ii. § [61.] “Is vero sine modo, et ultra quam oportuit, Instantium sociosque ejus lacessens, facem quandam nascenti incendio subdidit: ut exasperaverit malos potius quam oppresserit.”]

y
Instantius D.

z
deformities D.

3
[Bristow’s “Fifty-one Demands to be proposed by the Catholics to the Heretics.” Lond. 1592. 4to.]

1
[The same author’s “Sure ways to find out the Truth, or Motives unto the Catholic Faith.” Antwerp, 1574. 8vo.]

2
[Campian’s “Censure upon two books written in answer to Edmund Campian’s offer of Disputation.” Douay, 1581; and Defence of the same by Parsons, 1582.]

3
[Allen’s “Apology of the English Seminaries at Rome and Rheims.” Mons, 1581.]

4
Job xiii. 7.

a
ye D.

5
Rom. iii. 17.

6
Job xl. 4, 5.

7
Psalm lxxii. 3, 6.

b
for peace D.

c
there is now no way D.

1
Psalm cxxii. 6.

1
[Compare B. vii. xi. 8. p. 211.]

1
[This and the Discourse of Justification, are now placed first among Hooker’s Opuscula, as having probably been earliest written. See Travers’s Supplication to the Council, in Dobson’s Hooker, ii. p. 464 (infra, p. 559). “Upon . . . occasion of this doctrine of his, that the assurance of that we believe by the word is not so certain as of that we perceive by sense, I . . . taught the doctrine otherwise. — According to which course of late, when as he had taught, ‘that the church of Rome is a true church,’ &c.” Compare Hooker’s Answer, § 9, 10, 11. It should seem as if these two, and the Sermons on Pride, were portions of a series on the Prophecy of Habakkuk preached in the Temple Church, 1585-6; and the present arrangement sets them in the order of their texts. (It has here been compared with the first ed. 1612.)]

2
[The name is usually, but not always, in the first ed. Abacuc.]

1
[Job xiii. 15.]

2
Psalm lxxiii. 28.

1
[Rom. iv. 20. οὐ διεκρίθη τῃ̑ ἀπιστίᾳ.]

2
[2 Kings vii. 2.]

3
Gen. xvii. 17.

1
[St. Mark ix. 24.]

2
[1 John ii. 9.]

1
[Rom. xv. 15.]

1
[Psal. xxii. 1.]

2
[Luke xviii. 11.]

3
[Rom. viii. 26, 27.]

1
[Luke xxii. 33.]

2
[Written Galathians, ed. 1612, 1618.]

3
[Gal. iv. 5.]

4
[to, ed. 1612, 1618.]

5
[Apoc. ii. 2, 4.]

6
2 Cor. xi. 2, 3.

1
Jos. i. 5; Heb. xiii. 5.

2
[Sathan, throughout 1612, 1618.]

3
Psal. lxxviii. 19.

4
Gen. xviii. 12.

5
John i. 46.

6
[So ed. 1612, 1618; bewitched, Gauden, 1662, and so Keble.]

1
Matt. xix. 26.

2
1 Cor. i. 27, 28.

3
[Ps. lxxxix. 28, 32.]

4
[Ps. lxxix. 9.]

1
John xiii. 1.

1
Sallust. Jugurth. c. 14.

2
Luke xxii. 31, 32.

3
[John xvii. 11.]

1
[apaled, ed. 1612, 1618; appaled, Gauden, 1662, 1676: v. p. 606. v. Murray’s Dictionary, appal.]

2
[Rom. viii. 35, 38, 39.]

1
[Prefixed to the first publication of the Sermon on Justification, 1612, by Henry Jackson, of C. C. C., to whom Hooker’s papers had been intrusted by Dr. Spenser, to be arranged for the press. In the first collection of Hooker’s Opuscula [dated 1618], subjoined to the five books of Ecclesiastical Polity, 1617, this sermon comes after Travers’s Supplication and Hooker’s Answer: the present order has been adopted as being that in which they were written, and because the two latter suppose a knowledge of this sermon.]

2
Lib. iv. Annal. [c. 34.]

3
Lib. i. Hist. [c. 2.]

4
In Vita Agric. [c. 41.]

5
Lib. ii. [c. 98.]

1
[From a passage in Hooker’s answer to Travers’s Supplication, § 5, we know that this sermon was preached in the first year of Hooker’s mastership of the Temple. For he says, “I am able to prove that myself have now for a full yeer together borne the continuance of such dealings,” &c. And it appears from Strype’s Collections, inserted in Walton’s Life of Hooker, that the sermon was preached the 28th March, and that Travers’s notes of exception to it were “set down and shewed” March 30, 1585: but a MS. in the Harleian Collection, quoted above, vol. i. 59, gives March 1, 1585, as the date of the sermon; erroneously, since the sermon was preached on a Sunday, (see Travers, Supplication, p. 561, 562, infra,) and the 1st March did not fall on a Sunday in either of those years. The 28th did, in 1586. And this agrees with what Travers in his Supplication states, “that Hooker according to his course had of late taught that the church of Rome is a true church of Christ.” He had been made Master of the Temple March 17, 1584, 5. The sermon was collated by Archdeacon Cotton for the edition of 1836, with a MS. (A. 5, 6.) in Trin. Coll. Dublin, here designated by D.: the results of which collation, revised by Dr. Todd and Mr. Gibbings, are given in the margin below*.]

*
[E. stands for edition of 1613 (v. Pref. i. p. liii). It agrees with the edition of 1618, except in the readings here marked. The ed. 1618 is here marked F.] 1886.

a
the better E.

b
Ergo D.

c
Corinthes D.

2
1 Cor. v. 12, 13†.

†
Om. D.

d
be apparently such as cannot E.

e
of the apostolical E.

1
2 Cor. vi. 14-17*.

*
Om. D.

f
amongste D.

h
never D.

2
[De Nat. et Grat. contra Pelag. § 42. x. 144. G. “Commemorat cos, qui non modo non peccasse, verum etiam juste vixisse referantur, Abel, Enoch, Melchisedech, &c. Adjungit etiam fœminas, . . . ipsam etiam Domini ac Salvatoris nostri matrem, quam dicit sine peccato confiteri necesse esse pietati. Excepta itaque sancta virgine Maria, de qua propter honorem Domini nullam prorsus, cum de peccatis agitur, haberi volo quæstionem, (unde enim scimus, quid ei plus gratiæ collatum fuerit ad vincendum omni ex parte peccatum, quæ concipere ac parere meruit, quem constat nullum habuisse peccatum?) hac ergo virgine excepta, si omnes illos sanctos et sanctas, cum hic viverent, congregare possemus, et interrogare utrum essent sine peccato; quid fuisse responsuros putamus; utrum hoc quod iste dicit, sive quod Joannes Apostolus?”]

i
honour D.F.

k
about D.

l
may E.

3
Or whosoever it be that was the author of those Homilies that go under his name†.

†
Note om. D.

m
in E.

n
hath om. E.

o
own om. E.

1
Knowing how the schoolmen hold this question, some critical wits may perhaps half suspect that these two words, per se, are inmates. But, if the place which they have be their own, their sense can be none other than that which I have given them by a paraphrastical interpretation*.

*
Both notes om. D. the latter om. F.

p
bond E.

q
is not otherwise loosed from the bond of ancient sin, than by redemption. E.

2
Hom. 2. de Nativ. Dom.* [t. v. pars ii. p. 545. Biblioth. Patr. Colon. “Spem terrarum, decus sæculorum, commune omnium gaudium, peculiari munere novem mensibus sola possides: initiator omnium rerum abs te initiatur, et profundendum pro mundi vita sanguinem de corpore tuo accepit, ac de te sumpsit, quod etiam pro te solvat. A peccati enim,” &c.]

r
then all E.

s
were dead in sin E.

t
righteous E.

3
[1 Cor. i. 30.]

u
offered up himself E.

x
of om. E.

4
[Rom. viii. 21.]

y
as E.

z
is inherent E.

z
plain om. E.

a
that all have sinned om. E.

b
never did E.

c
This clause in marg. E. which also reads coeffective for coefficient.

d
work efficiently E.

1
*“Deus sine medio coeffectivo animam justificat.” Casal. de quadr. 1. part. Just. lib. cap. 8 [pars I. lib. i. cap. 8. p. 24. G. ed. Venet. 1599. first published 1563.] Idem, lib. 3. c. 9. [“Salvator noster est nostra justificatio, quia nos justificat effective secundum naturam divinam; estque nostra justificatio, quia nos justificat meritorie secundum naturam humanam.” p. 304. Casal was bishop of Leiria and Coimbra in Portugal, and was distinguished at the Council of Trent. † 1587. See in Fra Paolo, vi. 53, his arguments for conceding the eucharistical cup to the laity; and vii. 32, his assertion of the divine right of episcopacy.]

*
Note om. D.

e
there is also somewhat D.

f
nature and essence E.

1
*Tho. Aquin. Summ. Theol. ii. pars i. quæst. 100. “Gratia gratum faciens, id est, justificans, est in anima quiddam reale et positivum; qualitas quædam (art. ii. concl.) supernaturalis, non eadem cum virtute infusa, ut Magister, sed aliquid (art. iii.) præter virtutes infusas, fidem, spem, charitatem, [110. art. 1.] habitudo quædam (art. iii. ad 3.) quæ præsupponitur in virtutibus istis sicut earum principium et radix;” essentiam animæ tanquam subjectum occupat, non potentias, sed “ab ipsa” (art. iv. ad 1.) “effluunt virtutes in potentias animæ, per quas potentiæ moventur ad actus.” Plur. vid. quaest. 113. de Justificatione. [t. xi. 253-255; 259, &c. ed. Antwerp. 1612. Comp. Concil. Trident. Sess. vi. Decr. de Justificatione, cap. vii. “Justificationis unica formalis causa est, justitia Dei; non qua ipse justus est, sed qua nos justos facit, qua videlicet ab eo donati renovamur spiritu mentis nostræ, et non modo reputamur, sed vere justi nominamur et sumus; justitiam in nobis recipientes unusquisque suam, secundum mensuram quam Spiritus Sanctus partitur singulis prout vult, et secundum propriam cujusque dispositionem et cooperationem.” Ibid. can. xi. “Si quis dixerit, homines justificari vel sola imputatione justitiæ Christi, vel sola peccatorum remissione, exclusa gratia et caritate, quæ in cordibus eorum per Spiritum Sanctum diffundatur atque illis inhæreat, aut etiam gratiam, qua justificamur, esse tantum favorem Dei; anathema sit.”]

*
Note om. D.

g
to E.

h
of E.

i
inhable D.

k
amiable and gracious E.

l
and washeth out E.

m
sins E. not F.

2
[Concil. Trident. ubi supr. cap. 10. “Mortificando membra carnis suæ, et exhibendo ea arnia justitiæ in sanctificationem, per observationem mandatorum Dei et ecclesiæ, in ipsa justitia, per Christi gratiam accepta, cooperante fide bonis operibus, crescunt, atque magis justificantur: sicut scriptum est, ‘Qui justus est, justificetur adhuc.’ ” And can. xxiv. Si quis dixerit, justitiam acceptam non conservari, atque etiam non augeri coram Deo per bona opera; sed opera ipsa fructus solummodo et signa esse justificationis adeptæ, non autem ipsius augendæ causam; anathema sit.”]

1
[Ibid. cap. xvi. “Bene operantibus usque in finem, et in Deo sperantibus proponenda est vita æterna et tanquam gratia filiis Dei per Christum Jesum misericorditer promissa, et tanquam merces ex ipsius Dei promissione bonis ipsorum operibus et meritis fideliter reddenda.”]

2
[Ibid. “Cum ipse Christus Jesus, tanquam caput in membra, et tanquam vitis in palmites, in ipsos justificatos jugiter virtutem influat, quæ virtus bona eorum opera semper antecedit, et comitatur et subsequitur, et sine qua nullo pacto Deo grata et meritoria esse possent,” &c.]

n
in their divinity is E.

3
[Catherinus, Dialog. de Justif. ad calc. Summ. Doctr. de Prædest. Rom. 1550. p. 60.]

4
[See in Aquinas (2 Summ. pars ii. qu. xxiv. art. 10; t. xi. pars ii. p. 63 A. Antwerp, 1612.) with what qualification this must be taken.]

5
[Id. ibid. art. 11, 12.]

o
the which E.

6
[Id. 3 Summ. qu. lxix. art. 6. (t. xii. 221.)]

p
the first E.

7
[Id. ibid. qu. lxviii. art. 5. fol. 219.]

8
[Id. ibid. qu. lxix. art. 1, 2, 3. f. 220.]

q
diminish E. not F.

1
[Id. ibid. qu. lxxxvii. art. 3. fol. 292. “Triplici ratione aliqua causant remissionem venialium peccatorum. Uno modo, in quantum in eis infunditur gratia;—et hoc modo. . .per omnia sacramenta novæ legis . . . peccata venialia remittuntur. Secundo, in quantum sunt cum aliquo motu detestationis peccatorum: et hoc modo confessio generalis, tunsio pectoris, et oratio Dominica, operantur ad remissionem venialium peccatorum. . .Tertio modo, in quantum sunt cum aliquo motu reverentiæ in Deum, et ad res divinas; et hoc modo benedictio episcopalis, aspersio aquæ benedictæ, quælibet sacramentalis unctio, oratio in ecclesia dedicata, et si aliqua sunt hujusmodi, operantur ad remissionem venialium peccatorum.”]

2
[Conc. Trid. Sess. vi. Decr. de Justif. cap. xiv. “Qui ab accepta justificationis gratia per peccatum exciderunt, rursus justificari poterunt, cum excitante Deo, per pœnitentiæ sacramentum, merito Christi, amissam gratiam recuperare procuraverint.”]

3
[Ibid. “Docendum est, Christiani hominis pœnitentiam post lapsum multo aliam esse a baptismali, eaque contineri non modo cessationem a peccatis, et eorum detestationem, aut cor contritum et humiliatum; verum etiam eorundem sacramentalem confessionem, saltem in voto, et suo tempore faciendam, et sacerdotalem absolutionem; itemque satisfactionem, per jejunia, eleemosynas, orationes, et alia vitæ spiritalis exercitia, non quidem pro pœna æterna, quæ vel sacramento vel sacramenti voto una cum culpa remittitur, sed pro pœna temporali, quæ, ut sacræ literæ docent, non tota semper, ut in baptismo fit, dimittitur illis, qui gratiæ Dei, quam acceperunt, ingrati, Sp.Sanctum contristaverunt, et templum Dei violare non sunt veriti.” Comp. Sess. xiv. decr. de Pœnit. cap. 9, et can. 13.]

r
first, for D.

s
either om. E.

4
[Ibid. Sess. xxv. Decr. de Purgatorio; et Decr. de Indulgentiis. Comp. Aquin. in iv. Sent. dist. xx. qu. i. art. 3.]

t
to justification E.

u
pass it by in few words E.

x
that E.

y
in the presence E.

z
it om. E.

1
Phil. iii. 8, 9.

a
lost E.

b
I do om. E.

c
to be found E.

d
it om. F. it the essence of a F.

e
is it E.

f
Christ E.

g
is impious in himself E.

h
remitted E.

i
upholdeth F.

k
it om. F.

l
was E.

2
2 Cor. v. 21.

m
to be sin for us, who knew no sin E.

n
or om. E.

o
whatsoever; it is our comfort, and our wisdom E.

p
son E.

q
man E.

r
the Apostles E.

s
D. begins the section here.

t
without D.

u
as om. F.

v
different in nature E.

1
[Rom. iv. 5.]

2
[1 John iii. 7.]

3
[Rom. iv.]

4
[James ii.]

5
Rom. vi. 22.

w
unto D.

x
you D.

y
possession E.

z
Abakuk D. Abak. F.

a
because E.

b
fruits E.

b
we E.

c
they professt D.

d
holy men E.

e
for E.

f
a om. F.

g
endeavour to om. E.

1
[De Gubern. Dei, lib. iv. p. 341. D; in Bibl. Patr. Colon. t. v. part. iii.]

h
indeed we have E.

i
herein D.

j
ever om. E.

k
were E.

l
own om. E.

m
before E.

n
farder D.

o
ourselves can do E.

p
mouth E.

q
do F.

r
If we did [do F.] not commit the sins which daily and hourly, either in deed, word, or thoughts we do commit E.

s
specially om. E.

t
men E.

u
and E.

w
by any respect E.

x
that om. E.

y
mercies E.

z
God om. E.

a
he had set E.

b
one E.

c
this E. (?)

d
and F. if he E.

e
our E.

f
any om. E.

g
past him D.

h
one om. F.

i
could be found to be E.

j
which we do E.

k
or E.

l
exactly able E.

m
doing well E.

n
knowes D.

o
to reckoning E.

p
and E.

1
[Psalm cxix. 5.]

2
[Rom. vii. 19, 24.]

q
the other prophet E.

3
[Isa. i. 4.]

r
were E.

s
loden D.

t
of E.

u
thus om. E.

x
ye F.

y
stake D.

1
Acts xiii. 41-44.

z
to E.

a
ye E.

b
in hand E.

2
*Heb. i. 2.

*
Notes om. D.

3
*By sanctification, I mean a separation from others not professing as they do. For true holiness consisteth not in professing, but in obeying the truth of Christ.

c
in E.

d
the conclusion whereunto in thend we came of all D.

e
and fellowship om. E.

f
may E.

g
open om. E.

h
strict om. E.

i
to be gold E.

k
be suitable E.

r
prove to om. E.

s
on it E.

t
this E.

u
may E.

x
that om. E.

y
then E.

z
and om. E.

1
Apoc. xviii. 4.

a
that ye be not partaker of her plagues E.

b
in wrath hath D. hath F.

2
Matt. xxiv. 16; S. Luke xxi. 21.

3
Gen. xix. 15.

c
that ye be not....sins of Babylon om. E.

d
we doubt E.

e
for the plagues E.

f
what E.

g
which E.

h
of E.

i
to om. D.

j
no stop after avoid in F.

k
by their law E.

l
amongste D.

m
in om. E. (?) apparance F.

1
[Conc. Trid. Sess. iv. Decr. de Canonicis Scripturis. “Ecclesia [Tridentina Synodus] orthodoxorum Patrum exempla secuta, omnes libros tam Veteris quam Novi Testamenti, cum utriusque unus Deus sit auctor, necnon traditiones ipsas, tum ad fidem, tum ad mores pertinentes, tanquam vel ore tenus a Christo, vel a Sp. Sancto dictatas, et continua successione in ecclesia catholica conservatas, pari pietatis affectu ac reverentia suscipit ac veneratur.” Conc. Hard. x. 22.]

n
that E.

2
[Bulla Pii IV. super Profess. Fidei, (containing what is commonly called Pope Pius’ Creed, to which all ecclesiastical persons must assent:) ibid. t. x. 201, a. “Sanctam catholicam et apostolicam Romanam ecclesiam, omnium ecclesiarum matrem et magistram agnosco; Romanoque Pontifici, beati Petri Apostolorum Principis successori, ac Jesu Christi vicario, veram obedientiam spondeo acjuro.”]

3
[Conc. Trid. Sess. xiii. Can. 2. “Si quis dixerit in sacrosancto Eucharistiæ sacramento remanere substantiam panis et vini una cum corpore et sanguine Dom. nostri J. C.; negaveritque mirabilem illam et singularem conversionemtotius substantiæ panis in corpus et totius substantiæ vini in sanguinem, manentibus duntaxat speciebus panis et vini, quam quidem conversionem catholica ecclesia aptissime transubstantiationem appellat; anathema sit.” t. x. 83.]

4
[Ibid. can. 6. “Si quis dixerit, in sancto Eucharistiæ sacramento Christum unigenitum Dei Filium non esse cultu latriæ, etiam externo, adorandum; atque ideo nec festiva peculiari celebritate venerandum, neque in processionibus . . . solenniter circumgestandum; vel non publice, ut adoretur, populo proponendum; et ejus adoratores esse idololatras; anathema sit.” x. 84.]

1
[Ibid. Sess. xxii. Decr. de Missa, cap. 2. “Quoniam in divino hoc sacrificio, quod in missa peragitur, idem ille Christus continetur, et incruente immolatur, qui in ara crucis semel seipsum cruente obtulit; docet sancta synodus, sacrificium istud vere propitiatorium esse . . . Non solum pro fidelium vivorum peccatis, pœnis, satisfactionibus, et aliis necessitatibus, sed et pro defunctis in Christo, nondum ad plenum purgatis, rite, juxta Apostolorum traditionem, offertur.” x. 127.]

2
[Bulla Pii IV. ubi supr. “Constanter teneo . . . sanctos una cum Christo regnantes, venerandos atque invocandos esse: eosque orationes Deo pro nobis offerre. . .Firmissime assero, imagines Christi et Deiparæ semper Virginis, nec non aliorum sanctorum, habendas et retinendas esse, atque eis debitum honorem ac venerationem impertiendam.” x. 200.]

n
the transubstantiation of the E.

o
also om. E.

p
the adoration E.

q
that E.

r
in om. E.

s
also om. E.

t
an actual E.

u
should speak E.

x
marked E.

y
in D.

y
them om. E.

z
heresy E.

a
of them om. D.

b
free and clear D.

c
heresies E. and them for it.

d
their om. E.

e
this E.

1
Ver. 22.

f
lay F. lap Gauden.

g
into E.

h
everlasting flaming E.

i
of E.

j
errors E.

k
which are in E.

l
those D.

m
that D.

n
them E.

o
these E.

p
to our fathers E.

q
the guides . . . . taught E.

r
worldly om. E.

s
guide as of the guided E.

t
promise E.

u
which . . . . man’s E.

x
that sinners can find to E.

y
he hardeneth E.

z
two om. E.

a
E. inserts In the third . . . . believers between the two verses.

1
John iii. 17.

b
own Son E.

2
John iii. 18.

3
Rev. ii. 21-23.

c
is, if they were not altogether faithless and impenitent E. so F.

d
are F.

e
either om. E.

f
any way om. E. (?)

g
with E.

h
byde D.

i
fire D.

k
therein E.

l
that om. E.

m
being E.

1
[Adv. Helvid. c. 19. t. ii. pars i. p. 226. ed. Vallarsii. Venet. 1767.]

2
They misinterpret, not only by making false and corrupt glosses upon the Scripture, but also by forcing the old vulgar translation as the only authentical: howbeit, they refuse no book which is canonical, though they admit sundry which are not*.

*
This note om. D.

n
these E.

o
name of foundation E.

3
1 Tim. iii. 16.

4
John i. 49; iv. 42.

p
raise D.

q
deny E.

r
with E.

s
the heathenish E.

t
and E.

u
were E.

x
mortal om. E. not F.

y
This sentence om. E.

z
you D.

1
Gal. v. 2.

a
the other E.

b
that they E.

c
in E.

2
Plainly in all men’s sight whose eyes God hath enlightened to behold his truth. For they which are in error are in darkness, and see not that which in light is plain. In that which they teach concerning the natures of Christ, they hold the same which Nestorius fully, the same which Eutyches about the proprieties of his nature*. [If taken in the full literal sense, it seems hardly possible that this note should be Hooker’s, considering on the one hand his unvarying acknowledgment that the church of Rome is orthodox regarding the doctrine of the Incarnation; on the other hand his express condemnation of Nestorius and Eutyches. Comp. (e. g.) b. iii. c. i. 10; with b. v. c. xlii. 13; lii. 3, 4. It should be remembered that this sermon was not prepared by the author for the press, and that the Dublin copy of it has no notes at all.]

*
Note om. D.

c
disputing with them urge E. not F.

d
one only E.

e
his mercy om. E.

1
The opinion of the Lutherans, though it be no direct denial of the foundation, may notwithstanding be damnable unto some; and I do not think but that in many respects it is less damnable, as at this day some maintain it, than it was in them which held it at first; as Luther and others, whom I had an eye unto in this speech. The question is not, whether an error with such and such circumstances; but simply, whether an error overthrowing the foundation, do exclude all possibility of salvation, if it be not recanted, and expressly repented of*.

*
Note om. D.

f
firmly E.

g
my D.

h
here E.

2
[Apoc. iii. 8.]

3
Luke xiii. 3.

i
thought or word E.

j
holden all sins and errors E., all om. F.

k
error D.

l
unless E.

1
[Ps. xix. 12.]

m
were om. F.

n
errors E.

o
end E.

2
Gal. v. 2. 4.

p
because om. E., that they received not the love of the truth, they might not be saved? (Hooker’s own words, not a quotation) F.

q
word of truth E.

3
2 Thess. ii. 10-12.

1
Apoc. xiii. 8.

r
of them om. E.

s
For om. E.

2
[Penry, “M. Some laid out in his colours,” &c. p. 29. “We hold, that to him which dieth a papist, let him do never so many good works, and build if it were possible ten thousand colleges or churches, the very gates and portcullis of God’s mercy are quite shut up, and all those his glorious works, how sweet soever they may be to others, will prove but wrack and misery to himself. And in this point if either M. Hooker, M. Some, or all the reverend bishops of the land, do stand against us, it shall little dismay us: we say with their own Doctor, (but yet not altogether as he,) ‘Instar mille,’ (he saith Platonis, we say,) ‘veritatis calculus.’ ”]

t
slender om. E.

u
as yet om. E.

x
general repentance E.

y
This clause om. E., or for all sinners om. F.

z
oversight D.

a
the faults E.

b
fall E.

c
which lived E.

d
foundations E.

e
at the least E.

f
for fear E.

g
although E.

h
live and om. E. not F.

i
with E.

k
all held E.

l
but only E.

m
interpretation E.

n
good works E.

1
For this is the only thing alleged to prove the impossibility of their salvation: The church of Rome joineth works with Christ, which is a denial of the foundation, and unless we hold the foundation, we cannot be saved*.

*
Note om. D.

o
through om. E.

p
which E.

q
as E.

r
Paragraph here in D.

s
the same E.

t
each the other E.

x
both om. E.

y
nothing to be more sound E.

z
as om. E.

a
No paragraph D.

b
be E.

c
and charity E.

d
he om. E.

e
which om. E.

f
the E.

g
whereof E.

h
fruits of works E.

i
the which E.

k
the Scriptures E.

l
Jesus Christ E.

m
hath D.

n
except E.

o
hath E.

p
hath faith comes after adoption E.

q
they E.

r
and E.

s
to the last E. not F.

t
in om. E.

u
thing om. E.

x
and not works of ours without faith E., and no work F.

y
that we E.

z
good om. E.

a
Fathers om. E.

b
that om. E.

1
[In Syntagm. Confess. pars ii. p. 106. Gen. 1654. “Docemus bona opera divinitus præcepta necessario facienda esse, et mereri gratuita Dei clementia sua quædam sive corporalia sive spiritualia præmia.” This confession was exhibited at the council of Trent, 1552, by the deputies of the Duke of Wirtemberg. It was drawn up by Brentius, (Sleidan, l. 22. p. 277. ed. Argent. 1559.) and had been approved by the Saxon protestants.]

c
Others om. E.

d
might E.

e
worst E.

f
may E.

g
own om. E.

h
the D.

i
the angels E.

k
these E.

l
merits is then E. not F.

m
or E.

n
Jesus Christ E.

o
which lived E.

p
superstition E.

1
They may cease to put any confidence in works, and yet never think, living in popish superstition, they did amiss. Pighius died popish, and yet denied popery in the article of justification by works long before his death. [See Bayle, art. Pighius. He died at Utrecht, December 26, 1542: having the same year published at Cologne, “Controversiarum præcipuarum in comitiis Ratisponensibus tractatarum, et quibus nunc potissimum exagitatur Christi fides et religio, diligens et luculenta replicatio.” In the 2nd Controversy, De Fide et Justificatione, Sign. G. ii. is the following: “In illo justificamur coram Deo, non in nobis; non nostra sed illius justitia, quæ nobis cum illo jam communicantibus imputatur. Propriæ ju stitiæ inopes, extra nos in illo docemur justitiam quærere . . . Non nostra, sed Dei justitia justi efficimur in Christo. Quo jure? amicitiæ, quæ communionem omnium inter amicos facit, juxta vetus et celebratissimum proverbium: Christo insertis, conglutinatis, et unitis, etiam sua nostra facit; suas divitias nobis communicat; suam justitiam inter Patris judicium et nostram injustitiam interponit, et sub ea, velut sub umbone et clypeo, a divina, quam commeruimus, ira, nos abscondit, tuetur, ac protegit; immo eandem nobis impertit, ac nostram facit, qua tecti ornatique audacter et secure divino nos sistamus tribunali ac judicio, justique non solum appareamus, sed etiam simus.” Sign. G. iii. “Justificat nos Deus Pater bonitate sua gratuita qua nos in Christo complectitur: dum eidem insertos innocentia et justitia Christi nos induit: quæ una ut vera et perfecta est, quæ Dei sustinere conspectum potest, ita unam pro nobis sisti oportet tribunali divini judicii, et velut causæ nostræ intercessorem eidem repræsentari.” Ibid. et G. iv. Dissimulare non possumus, hanc vel primam doctrinæ Christianæ partem obscuratam magis quam illustratam a scholasticis speciosis plerisque quæstionibus et definitionibus, secundum quas nonnulli, magno supercilio primam in omnibus auctoritatem sibi arrogantes, et de omnibus facile pronunciantes, fortassis etiam nostram hanc damnarent sententiam qua propriam et quæ ex suis operibus esset coram Deo justitiam derogamus omnibus Adæ filiis, et docuimus una Dei in Christo niti nos posse justitia, una illa, justos esse coram Deo, destitutos propria.” It appears that he was censured in his own church as having a tendency to the Calvinistic notion of justification: and accused of Pelagianism both by Calvin and the Jansenists.]

q
what om. D.

r
by E.

s
clear E.

t
that E.

u
should E.

x
my D.

y
and E.

z
that doctrine of laws E. F. Gaud.

a
acception E.

1
“Vocata ad concionem multitudine, quæ coalescere in populi unius corpus nulla re præterquam legibus poterat.” Liv. de Romulo, lib. i. [c. 8.]

b
ye E.

c
the laws E.

d
now om. E.

2
Ephes. i. 23; iv. 15.

e
only the E.

3
Ephes. ii. 20.

f
which E.

g
these E.

h
those words of E.

i
herself E. not F.

4
John vi. 68; 2 Tim. iii. 15.

k
of om. E. not F.

l
obtained E.

m
who E.

n
what they shall do D.

o
still om. E. not F.

p
laws E.

q
mercy E.

r
not om. E.

s
Yet E.

t
possessed D.

1
Acts xvi. 17; Heb. x. 20.

u
before the written E.

x
possessed E.

2
Gen. xlix.

3
Job xix.

y
likewise om. E.

z
so om. E.

a
before it E.

b
earth E.

c
shall E.

d
blessed E.

e
he is a name E.

f
but F.

4
Acts iv. 12.

g
doth put E.

h
very om. E.

i
Christ om. D.

1
Luke ii. 28.

2
1 Cor. iii. 11.

j
it om. F.

k
be better opened E.

l
first om. E.

m
and om. E.

n
which serve E.

o
wrapped D.

p
had E.

q
a colour E.

r
their E.

s
either om. E.

t
great E.

u
own om. E.

1
Acts xxv. 19.

w
the Jews D.

x
called E.

y
entitled E.

z
the E.

a
stand E.

b
the Christianity D.

c
have om. E.

d
denying om. E.

e
Christianity, not D.

f
being om. E.

g
justified truly E.

h
persuasion om. E.

i
the third E.

k
where E.

1
Octav. c. 34.

l
per E.

m
there be many E.

n
to be om. E.

o
so F. Gaud. and following edd. Oxf. 1793. receive, Keble (? a misprint).

p
into E. not F.

p
or E.

q
the E.

r
or E.

s
one om. E.

t
the spiritual E.

u
if we have read E.

1
Rom. viii. 10.

2
Phil. ii. 16.

3
Col. iii. 4.

x
through E.

4
1 Pet. i. 23.

5
Ephes. ii. 5.

6
1 John v. 12.

y
for E.

z
that D.

a
which have the Son E.

7
1 John v. 13. Perpetuity of faith; Rom. vi. 10.

b
is everlasting in the world to come E.

1
[Should not “but” be omitted?]

c
died E.

d
the om. E.

e
allied E.

2
John xiv. 19.

f
many and sundry E.

g
so secret E.

h
faith E.

3
1 Pet. i. 23.

4
1 John iii. 9.

5
Ephes. i. 14.

6
John xiv. 17.

i
he om. D.

7
[“Mr. Miller in his madness denied it, and yet died faithful as I hear.” Anonymous note in the Dublin MS. of this sermon.]

k
Object. (ob. F.) But E.

l
you E.

m
that is to-day E.

n
be made E.

n
Sol. The E.

o
cause E.

1
[“Bolton’s end in despair, after persecution suffered in Queen Mary’s time.” Ibid. Strype, Mem. iii. i. 576. “John Bolton, sometime of Reading, who lying in gaol for religion, grew mad, and in his raving fits railed upon Queen Mary; who thereupon was cruelly tormented in the said prison. Which Bolton becoming sober, and of a better mind, Thackham took pity upon the man, because he seemed to be of good religion, and . . when by reason of the time, his very friends durst not become surety for such a traitor and rank heretic, as Bolton was then thought to be, he desired the mayor to take him alone with Bolton, which the mayor gently granted. And so this poor man was set at liberty and departed. But when the sessions came, Bolton left Thackham to pay the forfeiture.” It seems by a letter among Strype’s documents, Mem. iii. ii. 427, that this Bolton recanted so far as to attend mass, and yet afterwards printed “a certain story of his own great trouble and another’s recanting.” He was a silk weaver in Long-lane, Smithfield.]

p
Which kingdom om. E.

q
it shall not E.

r
be otherwise E.

2
Col. i. 23.

3
1 Tim. ii. 15.

4
John x. 28.

s
he promised E.

t
whereby it is E.

u
irrecoverably E.

v
unto E.

x
wherewith E.

y
angeis F. Gaud.

z
goodness E.

a
but the least E.

b
and E.

c
highest E.

d
which E.

1
1 John iii. 9.

e
that are of E.

2
[Platin. Vitt. Pontiff. p. 39. Colon. Ubiorum. 1600. “Marcellinus pontifex” (ad 304) “ad sacrificia gentium ductus, cum minis instarent carnifices, ut thura diis exhiberet, metu perterritus, deos alienos adoravit. Habito deinde non ita multo post concilio clxxx. episcoporum in Sinuessa urbe Campaniæ, eo et Marcellinus squalidus et pulverulentus et cilicio indutus proficiscitur, petitque ut sibi pro inconstantia debita pœna tribuatur. Qui eum damnaret, in tanto concilio nemo unus inventus est, cum dicerent omnes ea ferme ratione Petrum peccasse, ac flendo peccati pœnam luisse. Rediit Romam Marcellinus iratus, Diocletianum adiit hominemque increpat, qui se impulerit diis gentium immolare. Ducitur ad martyrium Diocletiani jussu . . . . Inter eundum vero Marcellum presbyterum admonet, ne corpus suum sepulturæ traderet, quod diceret ob negatum Salvatorem se id nequaquam mereri.” This story is examined by Tillemont (among others) and shewn to be incredible, Mem. t. v. p. 613. The Donatists in Africa circulated such an account, which is mentioned by St. Augustin, contra Lit. Petil. lib. ii. § 202. t. ix. 276; and rejected by him as unworthy of notice, § 208. p. 280.]

f
after that they E.

g
may om. E.

h
words E.

i
is nevertheless E.

j
hath E.

k
but F.

l
so om. E.

m
it om. E.

n
him om. D.

o
either at om. E.

p
he preserveth E.

q
the love E.

r
very om. E.

s
afterwards om. E.

1
Howsoever men be changed, (for changed they may be, even the best amongst men,) if they that have received, as it seemeth some of the Galatians, which fell into error, had received, the gifts and graces of God, which are called ἀμεταμέλητα, such as faith, hope, and charity are, which God doth never take away from him to whom they are given, as if it repented him to have given them; if such might be so far changed by error, as that the very root of faith should be quite extinguished in them, and so their salvation utterly lost, it would shake the hearts of the strongest and stoutest of us all. See the contrary in Beza’s Observations upon the Harmony of Confessions*.

*
This note om. D.

s
time E.

t
but that I, E.

u
perilous even in them E.

v
that E.

w
maintain E.

x
it is less D.

y
the om E.

z
the stubborn E.

a
calls E.

b
thought F.

c
and om. D.

d
them E. (not) F.

e
theirs also E.

f
that om. E.

g
that om. E.

h
only should be preached E.

i
necks D.

k
inevitably E. (not) F.

1
[“They might err in freewill, yet not as Pelagius, who was enemy to the grace of God.” Anonymous note in D.]

l
these E.

1
[“How were they justified, when their faith was subverted?” Ibid.]

m
received E.

2
[“S. Paul saw they were turned to another gospel, therefore in a damnable state.” Note in D.]

3
Error convicted, and afterwards maintained, is more than error; for although opinion be the same it was, in which respect I still call it error, yet they are not now the same they were, when they are taught what the truth is, and plainly taught*.

*
Note om. D.

n
as om. E.

o
the E.

4
Acts xv. 5. [“Equivocally, as the priests, John xii.” Note in D.]

p
other Pharisees E. these Pharisees, and other Pharisees F.

q
These E.

r
no E.

s
no believers E.

5
Gal. iv. 9.

t
you E.

6
Ver. 24.

u
gotten F.

7
Ver. 31.

1
Ver. 10.

w
that notwithstanding E.

2
[“The elect among them all were to be reclaimed from that error.” Note in D.]

x
by D.

y
which om. D.

3
Gal. v. 2, 4.

z
you D.

a
ye om. E.

b
because that they E.

c
and in grace E.

d
with a plain D.

e
godly or learned E. learned or godly F.

4
Bucer. de Unit. Eccles. Servanda*. [The editor has not found in Bucer’s works any tract with this title, and suspects that the name is put erroneously for that of some other reformer.]

*
This reference om. D.

f
as D.

g
spent upon the Galatians in vain D.

h
ever upon E.

i
willingly E.

k
and resist om. E.

1
[“Not true.” Note in D.]

l
their om. E.

m
of D.

n
Hereupon E. not F.

o
a reply E.

p
to prove E.

q
their denial of the foundation E.

r
besides E.

s
except . . death om. E.

t
understanding E.

u
that the E.

x
the foundation of om. E.

y
opinion E.

1
[“Who be they?” Ibid.]

z
heresies and om. E.

a
grant E.

2
Calv. Ep. 104. [p. 126. ed. Gen. 1617. “Quod Ecclesiæ reliquias manere in Papatu dico, non restringo ad electos, qui illic dispersi sunt, sed ruinas dissipatæ Ecclesiæ illic extare intelligo. Ac ne mihi longis rationibus disputandum sit, nos Pauli auctoritate contentos esse decet, qui Antichristum in templo Dei sessurum pronunciat. Quanquam et hoc rationibus satis validis me probasse puto, Ecclesiam licet semiruptam, immo si lubet diruptam ac deformem, aliquam tamen manere in Papatu.” This is from a letter to Lælius Socinus, 9 Dec. 1549.]

b
these E.

c
permit at your D.

d
not om. E.

e
where E.

3
[“God’s house a den of thieves.” Note in D.]

f
heretofore D.

g
before hath been declared E.

h
unto such E.

i
took a pleasure D.

1
[“Giving Christian repentance, and knowledge of the truth necessary to salvation, 1 Tim. ii.” Note in D.]

2
[“They deny the sufficiency of the scriptures, which you make the foundation.” Ibid.]

k
at om. E.

l
plain E.

3
Morn. de Eccles. [c. 2. p. 32. ed. 1594. “Si de Christi officio, et quærenda in Christo salute agatur, quo, tanquam jugulo, corpori caput, Ecclesiæ Christus conjungitur: sic meritis hominum et sanctorum, indulgentiarum sordibus, et infinitis blasphemiarum machinis pars hæc doctrinæ labefactata est, ut jam e tenui filo vita ecclesiæ penderet, eoque mox abrumpendo, (quæ fuit Antichristi in agendo sedulitas,) nisi tempore Dominus, qui eum compescerent, servos suos emisisset. Quamdiu vel tenui illud filum reliquum manet, Ecclesiæ nomen non denegamus, ut nec ei qui morbo contabescit nomen hominis quamdiu vivit.” The author of this work was the Breton nobleman, Philip Mornay du Plessis, leader of the more serious party among the French protestants: it was first published ad 1577.]

m
the E.

n
it om. E.

o
many E.

p
little om. E.

4
Zanch. Præfat. de Relig. [“Nescio quo singulari Dei beneficio, hoc adhuc boni in Rom. ecclesia servari nemo non videt, nisi qui videre non vult; quod nimirum, sicut semper, sic nunc etiam constans et firma in vera de Deo, deque personæ D. N. Jesu Christi doctrina persistit; et baptizat in nomine Patris et Filii et Spiritus Sancti; Christumque agnoscit ac prædicat pro unico mundi redemptore, futuroque vivorum et mortuorum judice, qui veros fideles secum in æternam vitam recepturus, incredulos autem et impios in æternum ignem cum Diabolo et angelis ejus ejecturus sit: quæ causa est, cur ecclesiam hanc pro ecclesia Christi etiamnum agnoscam: sed quali? qualis et ab Osea aliisque Prophetis ecclesia Israelis sub Jeroboamo, et deinceps, fuisse describitur: nunquam enim resipuit a suis fornicationibus.” Ad calc. Operis de Sacra Scriptura. t. viii. ed. 1605.]

q
did E.

r
God . . . . Jesus om. F.

s
to be E.

1
[“Not true altogether.” Note in D.]

t
world E.

u
hold E.

v
servants E.

2
[Blackstone’s Commentaries, vol. ii. p. 60. ed. Coleridge. “If the King granted a manor to A, and he granted a portion of the land to B, . . . the King was styled Lord Paramount, A was both tenant and lord, or was a mesne lord; and B was called tenant paravail, or the lowest tenant; being he who was supposed to make avail, or profit, of the land. 2 Inst. 296.”]*

*
[From the French, par amount, (ad montem,) above; par aval, (ad vallem,) below. v. Skeat’s Etym. Dict.] 1886.

3
[“Ambiguous, if they hold Christ’s redemption without works to be insufficient.” Note in D. Dr. Todd states that “this note is not easily decyphered, and ends imperfect, as if it had never been finished. It seems to be, ‘Ambiguous, if they hold workes, Christ’s Redemption without works thus to be unsufficient or—’ ”]

x
to the foundation om. E.

y
water E.

z
that D.

a
wholly are E.

b
within E.

c
be om. D.

d
too manifest E.

e
you E.

f
I did directly E.

g
which E.

1
[2 Thess. ii. 13.]

h
added E.

i
by the sequel E.

k
granteth E.

l
of being E.

1
Rom. xi. 6.

m
adding of works E.

2
I deny not but that the church of Rome requireth some kinds of works which she ought not to require at men’s hands. But our question is general about the adding of good works, not whether such or such works be good. In this comparison it is enough to touch so much on the matter in question between St. Paul and the Galatians, as inferreth those conclusions, “Ye are fallen from grace; Christ can profit you nothing:” which conclusions will follow upon circumcision and rites of the law ceremonial, if they be required as things necessary to salvation. This only was alleged against me: and need I touch more than was alleged*?

*
Note om. D.

n
and put away E.

3
[“But to justify us by faith without the merit of good works.” Note in D.]

4
[“The keeping of circumcision hindereth not salvation, but the opinion of the necessity thereof.” Ibid.]

5
[“Ambiguous.” Ibid.]

o
their om. D.

p
even om. E.

6
[The words “web” and “spun” in D. are underlined, and upon them written “tapinosis.”]

q
you D.

7
Matt. v. 20.

8
Luke xi. 39.

r
tithing E. not F.

s
things om. E.

9
Matt. v. 21.

1
[“The merit of works is most repugnant.” Note in D.]

2
[“Sophistry.” Note in D.]

x
as om. E.

y
our om. D.

z
by this speech we E.

a
or E.

b
very om. E.

c
out om. E.

d
as E.

e
case E.

f
are not excluded E.

g
it is not one meaning E. not F.

h
ado? E.

i
This sentence om. E. not F.

1
[“A base phrase.” Note in D.]

j
are of necessity required E.

k
unto E. (?)

2
Eph. i. 11.

l
acception E. not F.

m
to us E.

n
Jesus Christ E.

o
hands E.

p
with D.

q
that om. E.

r
it importeth E. it om. F.

s
are F.

t
yea E.

1
[“All the controversy brought to a term.” Note in D.]

2
[“All men take not ‘directly’ as he doth.” Ibid.]

u
necessary om. E.

3
[“How is the sanctification of infants accomplished?” Ibid.]

w
the power E.

x
to om. D.

4
[“Merit addeth to the very foundation, as the papists themselves will confess.” Ibid.]

y
by this addition consequently E.

z
be negatively E.

a
works E.

b
Not E.

c
that E.

d
who D.

e
straightwaies D.

1
“Hæc ratio ecclesiastici sacramenti et Catholicæ Fidei est, ut qui partem divini sacramenti negat, partem non valeat confiteri. Ita enim sibi connexa et concorporata sunt omnia, ut aliud sine alio stare non possit, et qui unum ex omnibus denegaverit, alia ei omnia credidisse non prosit.” Cassian. lib. vi. de Incarnat. Dom. [c. 17*.]

*
Note om. E.

f
thereupon E.

g
bere or bare D.

2
Acts xxvi. 23.

h
the E.

i
of om. E.

j
such as are E.

3
[“I think he was bishop of Constantinople.” Note in D.]

k
who E.

l
the Christian E.

4
Lib. vi. de Incar. Dom. cap. 17. [“Si negas Deum Dominum Jesum Christum, necesse est ut Filium Dei denegans etiam Patrem neges. Quia juxta ipsius Patris vocem, ‘Qui non habet Filium, nec Patrem habet; qui autem habet Filium et Patrem habet.’ Negans ergo genitum, etiam genitorem negas. Negans quoque Filium Dei in carne natum, consequens est, ut etiam in spiritu natum neges, quia idem natus in carne, qui prius natus in spiritu. Non credens ergo in carne editum, necesse est etiam passum esse non credas. Non credens autem illius passionem, quid reliquum est, nisi ut etiam resurrectionem neges? Quia fides suscitati ex fide mortui est. Nec stare potest ratio resurrectionis, nisi fides mortis ante præcesserit. Negans ergo passum et mortuum, negas quoque ab inferis resurgentem. Consequens utique est, ut neges etiam ascendentem: quia ascensio sine resurrectione esse non potuit. Et qui resurrexisse non creditur, necesse est nec ascendisse credatur: dicente Apostolo, ‘Qui enim descendit, ipse est qui ascendit.’ Ergo, quantum in te est, Dominus Jesus Christus neque ab inferis resurrexit, neque cœlum ascendit, neque ad dexteram Dei Patris sedet, neque ad illum qui expectatur examinationis ultimæ diem veniet, nec vivos nec mortuos judicabit.” c. 18. “Intelligis itaque, O infelix et furiosa perversitas, evacuasse te penitus omnem symboli fidem, omnem spei sacramentique virtutem? Et in ecclesia insuper stare ausus es, et esse te sacerdotem putas, cum illa omnia denegaveris, per quæ sacerdos esse cœpisti?” in Bibl. Patr. Colon. t. v. pars ii. p. 80.]

m
to be God om. E.

n
which E.

o
again om. E.

p
from death D.

q
affirmeth D. so 1618, 1676.

r
death D.

s
dost E.

t
the third E.

u
be E.

1
[v. Arist. Rhet. 1. 2. p. 8. Bekk. 1831.]

x
of E.

y
or E.

z
not D.

a
good works E.

b
one om. E.

c
directly deny E.

d
hold with D.

e
that E.

f
that E.

g
be E.

h
hath D.

i
are to come E.

k
for so much E.

l
means E.

1
[The word “justification” is not either in the original or the English translation.]

m
satisfaction E.

n
doctrine om. E.

o
behooful D.

2
Lewis of Granada, Medit. ch. last. 3. [“Of Prayer and Meditation. Wherein are contained fourteen devout Meditations for the seven days of the week, both for the Mornings and Evenings. And in them is treated of the consideration of the principal Mysteries of our Faith. Written first in the Spanish tongue by the famous religious Father, F. Lewis de Granada, Provincial of the holy order of Preachers in the Province of Portugal.” Paris, 1582. fol. 317. The writer was one of the most distinguished ascetic and devotional writers of Spain. He was confessor to the Queen Regent of Portugal, and died 1588. Biog. Univ.]

p
sin E. not 1618.

q
himself up E.

r
No stop in D.

s
of D.

3
Panigarola, lett. 11. [“Disceptationes Calvinicæ a Joanne Tonso Mediolan. Patritio in Latinum conversæ.” Milan 1594. Discept. xi. p. 272, 3. “ ‘Omnis gratia data est nobis per Christum Jesum:’ verum; at per applicationem. ‘Ipse est propitiatio pro peccatis nostris:’ verum; at applica. ‘Livore ejus sanati sumus:’ at applica. ‘Pro nobis se obtulit:’ at applica . . . Omnem enim satisfactionem in sanguine locamus; sed applicationem hac in re tribuimus, quibus ipse tribuit Christus, pœnalibus scilicet nostris operibus.” Francis Panigarola was one of the most distinguished preachers of Italy in the 16th century. The work from which the above is translated was a course of lectures against the Calvinists, addressed to Charles Emanuel, Duke of Savoy, by whom Panigarola was made Bishop of Asti, ad 1587: he died in 1594. See Tiraboschi, Storia della Letteratura Italiana, t. vii. part. i. lib. iii. c. 6. N°. 12-14.]

1
Annot. in 1 John i. [v. 7. “Whether sins be remitted by prayers, by fasting, by alms, by faith, by charity, by sacrifice, by sacraments, or by the priests, (as the holy Scriptures do plainly attribute remission to every of these,) yet none of all these do otherwise remit, but in the force, by the virtue and merit, of Christ’s blood: these being but the appointed means and instruments, by which Christ will have his holy blood to work effectually in us.”]

t
fastings E.

u
seyns D.

x
in this doctrine om. E.

2
[Apoc. ii. 24.]

y
unto E.

z
all the ability E.

b
its E. his F.

c
lists E.

d
also they be E.

e
own om. E.

f
unperfitt D.

g
do God no good E.

1
In his Book of Consolation, [i. 11. Works, p. 1153, ed. 1557.]

h
the greatest E.

i
and judgment om. E.

k
as well on the one side as on the other E.

l
work E.

m
mere om. E.

n
which om. E.

o
the worthiness of him E.

2
Panigarola, p. 264. [“Gratiam mereri non possumus, (alioqui gratia non esset gratia, sed præmium: nam meritum sequeretur); gloriam autem possumus. Præterea, (et in hoc summa consistit), opera nostra tanquam nostra, nunquam spiritalia et æterna bona merentur; cum vero merentur, id contingit, quoniam ab anima proficiscuntur, quæ gratiam habet. Simili nempe ratione, qua ex aquæ puræ rivulo nullus odor afflatur; sed si per odoratum canalem ea defluat, odorifera fit. Quare priusquam in gratia simus, res hujusmodi mereri non possumus; sed ante justificationem a gratia longe absumus; igitur ante justificationem non meremur: quo probato, consequens est, ut primam justificationem non mereamur, et primam gratiam mereri nunquam possimus. Atque ut rem absolvamus: opera nostra triplicia sunt, vel tribus modis considerare opera nostra possumus: tanquam quæ idoneos reddunt, quæ merentur, et quæ satisfaciunt. Antequam simus in gratia, quod ad eam attinet, opera nostra ad eam obtinendam nos præparant, et reddunt idoneos. Postquam vero gratiam obtinuimus, si pœnam tempore definitam, quæ reliqua est, spectemus, satisfaciunt; si vero gloriam, merentur.”]

p
sort om. E.

q
the om. E.

r
towards E.

s
grounds D.

t
sacred om. E.

1
[“S. Paul commandeth us so to do.” Note in D.]

u
such om. F.

w
of E.

x
abuse E. F.

y
and annihilate E. adnichillate D.

z
these E.

a
everlasting om. E.

b
of their om. D.

c
punishment D.

d
to D.

e
saith E.

2
[This has not been found in S. Aug. It was previously however quoted from him by Bp. Jewel, (iii. p. 210, Parker Soc. ed.) but with non possum for nolo (E. M.): also by Pilkington, p. 610, and Whitgift, i. 8. ii. 539. iii. 460, Parker Soc. ed.] 1886.

f
in E.

g
wise E.

h
which om. E.

i
and E.

k
and om. E.

l
that E.

m
pope or cardinal E.

n
a little E.

o
indifferently be E.

p
This sentence in marg. E.

q
as om. E.

r
popes E.

s
feet E.

t
stripped E.

u
and om. F. Antichrist converted &c. F.

w
foot E.

x
shall E.

y
and shall E.

z
or of a E.

a
John Stile D. John Style F.

1
[“John a Stile is not the son of perdition.” Note in D.]

b
the D.

c
that Paul or an angel should D.

d
a om. E.

e
the E.

f
the hope E.

g
if E.

h
truly E.

i
by E.

k
means E.

l
by which E.

m
to think om. E.

1
[“In these things whereof the truth is necessary to be known.” Note in D.]

n
never E.

o
doubted E. it om. F.

p
in the world om. E.

q
hindereth E.

r
doth E.

s
sin om. F.

t
I om. E.

u
evil E.

x
savour E.

2
[“The Apostle hath not deceived you, if [oft?] you mistake his meaning.” Note in D.]

y
said E.

z
obtain F.

3
[“The Apostle obtained mercy to show the truth which he persecuted in ignorance.” Note in D.]

a
you E.

b
I spake (speake F.) no otherwise, I meant no otherwise than he did E.

c
are commonly E.

d
examples E.

e
these E.

1
[S. John xxi. 22.]

f
as om. E.

g
my D.

h
and E.

i
so far as E.

k
satisfying of minds E.

l
this D.

m
state D.

n
myself into E.

o
have om. E.

p
helped E.

q
helped E.

1
Judges v. 23.

q
helped E.

s
so F. poisonous 1676, Keble.

t
taught E. teached F.

u
our om. E.

w
glory and praise D.

2
Rom. iii. 7, 8.

x
am I then . . . . as D.

y
besides E.

z
upon D.

a
He E.

b
for you om. E.

1
[Ecclus. v. 10.]

c
great om. E.

2
[Job xv. 2.]

d
do neither E.

e
must E.

f
that E.

g
whiles E.

h
I hope om. E.

i
if om. and a period at injury E.

j
indeed E.

k
speech E.

3
[Gal. iv. 12.]

l
and I do wish om. E.

m
of my E.

4
[Contr. Joan. Ierosolym. § 2. t. ii. 409 C. ed. Vallars. “Nolo in suspicione hæreseos quenquam esse patientem; ne apud eos qui ignorant innocentiam ejus, dissimulatio conscientia judicetur, si taceat.”]

n
of E.

o
cause E.

p
deny D.

q
case om. E.

1
James ii. 1.

r
Christ om. E.

s
man om. E.

t
indifferent E. different F.

u
judgments E.

x
that F.

y
you E.

1
[This Supplication, with Hooker’s Reply, was first printed at Oxford, by Barnes, 1612: under the superintendence of Henry Jackson, of C.C.C., to whom Hooker’s papers had been entrusted by Dr. Spenser, to be prepared for the press. It was reprinted in 1617, [1618, for Henry Fetherstone] in fol. with the Sermons, (except that on Matt. vii. 7, 8,) and subjoined to the five Books of Ecclesiastical Polity, by Stansby: who prefixed the following address to the reader:

“The pleasures of thy spacious walks in Master Hooker’s Templegarden (not unfitly so called, both for the Temple whereof he was Master, and the subject, Ecclesiastical Polity) do promise acceptance to these flowers, planted and watered by the same hand, and, for thy sake, composed into this posy. Sufficiently are they commended by their fragrant smell in the dogmatical truth; by their beautiful colours, in the accurate style; by their medicinable virtue, against some diseases in our neighbour churches, now proving epidemical, and threatening farther infection; by their straight feature and spreading nature, growing from the root of faith, (which, as here is proved, can never be rooted up,) and extending the branches of charity to the covering of Noah’s nakedness; opening the windows of hope to men’s misty conceits of their bemisted forefathers. Thus, and more than thus, do the works commend themselves: the workman needs a better workman to commend him, (Alexander’s picture requires Apelles’s pencil;) nay, he needs it not, His own works commend him in the gates; and, being dead he yet speaketh; the syllables of that memorable name, Master Richard Hooker, proclaiming more, than if I should here style him a painful student, a profound scholar, a judicious writer, with other due titles of his honour. Receive then this posthume orphan for his own, yea, for thine own sake; and if the printer hath, with over-much haste, like Mephibosheth’s nurse, lamed the child with slips and falls, yet be thou of David’s mind, shew kindness to him for his father Jonathan’s sake. God grant, that the rest of his brethren be not more than lamed, and that as Saul’s three sons died the same day with him, so those three promised to perfect his Polity, with other issues of that learned brain, be not buried in the grave with their renowned father. Farewell. W. S.”

For an account of the occasion of this petition, see Walton’s Life of Hooker, supr. i. 52. The Supplication and Reply are in MS. in the Bodleian library, Mus. 55. 20, 21. (B.) collated for this publication by the editor: both evidently transcripts, and by no very accurate hand.]

2
[This seems to have been soon after Easter, 1586, 27 Eliz.]

a
many B.

3
[That is, in recommending him for the mastership of the Temple. See Burghley’s Letter to Whitgift, Sept. 17, 1584, in Strype’s Additions to Walton’s Life of Hooker, vol. i. p. 29. And Life of Whitgift, b. iii. c. 9. “The Queen had asked the lord treasurer, what he thought of Travers to be master of the Temple. Who answered, that at the request of Dr. Alvey in his sick ness, and of a number of honest gentlemen of the Temple, he had yielded his allowance of him to the place, so as he would shew himself conformable to the orders of the Church.”]

1
[Fuller, Ch. Hist. b. ix. p. 217. “As for Travers his silencing, many which were well pleased with the deed done were offended at the manner of doing it. For all the congregation on a sabbath in the afternoon were assembled together, their attention prepared, the cloth, as I may say, and napkins were laid, the guests set, and their knives drawn for their spiritual repast, when suddenly as Mr. Travers was going up into the pulpit, a sorry fellow served him with a letter, prohibiting him to preach any more.”]

2
[S. Mark xv. 43.]

3
John vii. 51.

b
allow F.

c
those F.

d
those [rules om.] E. inserted in brackets F.

1
1 Tim. v. 21.

e
it F.

1
[He alludes doubtless to that which was one of the standing grievances of the puritans: viz. that too much favour comparatively was shewn to those clergymen who were charged with saying mass, or with otherwise betraying a tendency to popery.]

f
committed of E. convicted for F.

g
beneficence E. beneficency F.

h
letters E. F.

i
I rest.—I this answer: F.

2
[Fuller, C. H. ix. 215, gives the instrument of Travers’ ordination at Antwerp, 8 May, 1578.]

1
[Compare a letter of Travers to Lord Burghley, Strype, Whitg. b. iii. App. N°. xii; in which the same reasons are alleged for his not consenting to reordination, as Burghley and others wished, to qualify himself as Master of the Temple. See Life of Whitg. i. 346.]

j
in om. F.

1
[Cap. 12. “An act for the ministers of the Church to be of sound religion . . . . Every person under the degree of a bishop, which doth or shall pretend to be a priest or minister of God’s holy word and sacraments, by reason of any other form of institution, consecration, or ordering, than the form set forth by parliament in the time of the late king Edward VI. or now used; shall in the presence of the bishop or guardian of the spiritualities of some one diocese where he hath or shall have ecclesiastical living, declare his assent and subscribe to all the articles of religion, which only concern the confession of the true Christian faith and the doctrine of the sacraments.”]

k
case F.

2
[William Whittingham, born in Chester about 1524: an Oxford man, and student of Christ-church. He was exiled in queen Mary’s time, and one of the leaders of the Presbyterian party among the English at Frankfort and Geneva: one also of the most active in the translation called the Geneva Bible. Dr. M’Crie says, that he married Calvin’s sister. But Wood contradicts this. Being afterwards chaplain to the earl of Warwick at Havre, he was by his recommendation, or that of his brother of Leicester, promoted to the deanery of Durham; where he died in 1579. Wood, Ath. Oxon. i. 446; M’Crie, Life of Knox, i. 419. Whittingham was supposed author of the preface to Christopher Goodman’s violent book, “How Superior Powers ought to be obeyed, 1558.” Strype, Ann. I. i. 545; Mem. iii. ii. 131. Bancroft, Dangerous Positions, b. ii. c. 1. p. 38. ed. 1640, says that he was “unworthily dean of Durham;” an expression which Wood justifies by an account of the liberties he took with the graves, and consecrated things, in his cathedral. See also MS. of Mr. Carte in Collectanea Curiosa, ii. 105.]

3
[By Archbishop Sandys, who at his visitation, 1578, summoned him “to shew his orders, or rather no orders that he had received at Geneva:” see Strype, Ann. II. ii. 168. 620. The archbishop wrote to lord Burghley, 4 April, 1579, “If his ministry without authority of God or man, without law, order, or example of any church, may be current, take heed to the sequel. Who seeth not what is intended? God deliver his Church from it. I will never be guilty of it.” Whittingham seems to have produced such imperfect certificates, as might make it doubtful whether he had ever been ordained at all. The proceedings in the inquiry were purposely delayed by the influence of the earl of Huntingdon, lord president of the north, and within half a year Whittingham died. Travers’ argument from the case of a popish priest seems to have been taken from lord Huntingdon’s remonstrance, which is quoted in Strype.]

l
church B. had charge F.

1
[Strype, Whitg. b. iii. c. 2. ad 1583. “Very many preachers there were now started up, that would do nothing but preach, and neither read the Liturgy nor administer the sacraments, as disliking the manner and form thereof practised in our Communion Book. And some of these undertook to preach, that were either not ordained ministers at all, or ordained differently from the English book.”]

m
determine of it according to the order of the realm B.

n
when as B.

1
[The three articles set forth in Sept. 1583: affirming the Queen’s supremacy, the lawfulness of the Common Prayer, and the orthodoxy of the Thirty-nine Articles. Strype, Whitg. b. iii. c. 2.]

o
had B. F.

p
bond B.

2
[See in the pedigree of the Hooker family, vol. i. preface, of the marriage of a niece of Hooker with a person of the name of Travers.]

q
charge F.

r
brought F.

s
his F.

t
comfort of both our lives, and that the contrary would hinder the service of God and his Church, and tend to the disprofit and discomfort of both of us B.

u
so om. F.

x
discovered F.

y
for other F.

1
[Strype, Grind. 185. “There was now” (1568; see proof of this date in Wood’s Ath. Oxon. i. 579.) “in London one Corranus,” (or Anthony de Corro,) “a Spaniard, a native of Seville, preacher to an assembly of Spanish protestants,” (as he states in his Dedication of his Dialogue to the Students of the Temple,) “though he himself was a member of the Italian congregation, to which one Hieronymus was preacher: . . . . a man of good learning, as Grindal testified of him, but of a hasty, and somewhat contentious spirit . . . . had caused a Table, entitled De Operibus Dei, to be printed . . . . wavering, as it seems, somewhat from the opinions of Calvin.” And this is hinted in a letter from Beza to him (Ep. 59. p. 277); to whom the whole matter being referred, he begged Grindal to undertake it (Ep. 58); and it ended (1570) in Corranus’ suspension. (Strype, Gr. 217.) But in 1571 he was made reader of divinity in the Temple, by the interest either of Leicester or bishop Sandys, in which office Alvey being master, he continued about three years, in much disquiet, and then came with letters of recommendation from Leicester to the university of Oxford (Wood): which received him after some scruple about his Pelagianism on the part of Reynolds, and others. There he remained as student in Ch. Ch. and Divinity reader in several halls, at least until 1582; and died 1591, in London. In 1574 he published an abstract of his lectures on the Romans, in the form of a dialogue between St.Paul and a disciple; in which, and in certain articles of faith subjoined, he disavows (and apparently with good faith) all the heresies with which he was charged.]

z
his F.

a
to understand om. F.

b
followed in B.

c
more om. F.

d
Notwithstanding such warning, he replying B. replying F. comma after matter.

1
[See the sermon “Of the Certainty and Perpetuity of Faith in the Elect.”]

e
assureth B. F.

f
excuseth F.

2
1 Tim. i. 13.

g
expounded F.

h
of E.F.

i
concerning E.F.

1
Apoc. xviii. 4; Gal. v. 2-5.

k
in the faith of the Church of Rome B.

l
added F.

m
not F.

n
not om. F.

1
[1 Tim. v. 20.]

o
Apostles F.

2
[Gal. ii. 14. οὐκ ὀρθοποδου̑σι πρὸς τὴν ἀλήθειαν του̑ εὐαγγελίου.]

3
[Covel, Just and Temperate Defence of the five Books of Eccl. Policy, p. 80. “The rest of the discourse, which is sometimes two or three hours long—a time too long for most preachers to speak pertinently—&c.” Herbert, Country Parson, c. 7. “The parson exceeds not an hour in preaching, because all ages have thought that a competency.”]

1
[Sess. v. Decret. de Peccat. Orig. c. 5. “Declarat sancta synodus, non esse suæ intentionis comprehendere in hoc decreto, ubi de peccato originali agitur, beatam et immaculatam virginem Mariam, Dei genitricem, sed observandas esse constitutiones felicis recordationis Xisti Papæ IV, sub pœnis in eis constitutionibus contentis, quas innovat.” The decree of Sixtus IV. is in t. ix. 1495, and ordains that neither opinion on this shall be counted heretical, “cum nondum sit a Romana ecclesia et apostolica sede decisum.” ad 1483. Conc. Trid. Sess. vi. De Justificatione, can. 24. “Siquis dixerit, hominem justificatum posse in tota vita peccata omnia etiam venialia vitare, nisi ex speciali Dei privilegio, quemadmodum de beata Virgine tenet Ecclesia; anathema sit.”]

2
[This statement may have arisen from hastily reading such passages as the following: S. Tho. Aquin. Opusc. x. Art. 28. “An Christus venit tollere nisi peccatum originale principaliter, seu principalius inter omnia peccata quæ tollere venit. Ad quod dicendum, quod Christus quantum est in se venit tollere omnia peccata. . . . Tanto autem principalius contra aliquod peccatum venit, quanto est majus peccatum autem quod originaliter contrahitur, licet sit minus gravitate et reatu pœnæ, est tamen maximum communitate; secundum illud Apostoli, In quo omnes peccarunt. Et quantum ad hoc, potest dici quod Christus principaliter venit tollere originale.”]

3
[Vid. Catharin. Summ. Doctrinæ de Peccat. Orig. Romæ 1550. fol. 47, 54: and compare Hooker’s statement below, Answer to Travers, § 14. Ambrosio Catharino of Sienna was a Dominican, and much distinguished as a disputant at the council of Trent. See in Fra Paolo, ii. 65, his theory on original sin; in § 76, on works done before justification; in § 80, on assurance and predestination; in § 81, on the divine right of episcopacy; in § 86, on the necessity of serious intention in the administrator to the validity of sacraments. He was afterwards Bishop of Minori, then archbishop of Conza; and died in 1553.—Biog. Univ.

p
in F.

1
Mark iii. 17.

2
2 Sam. vii. 2-5.

3
Gal. ii. 11, 14.

q
either offence to him, or to such as F.

r
some F.

s
it F.

1
Ezek. xxii. 30; xxxiii. 6.

t
for answering the F.

u
unwilling F.

x
works om. F.

y
of F.

z
any F.

1
[See Philipp. ii. 17. εἰ καὶ σπένδομαι.]

a
but om. F.

b
Suppliant F.

c
Word of God F.

1
[It is observable that whereas Travers had supplicated the whole council, Hooker’s reply is addressed to the archbishop only.]

2
[I. e. three sermons on three successive Sundays: see Travers’ Supplic. p. 560, &c.]

a
others E.

b
now om. E.

3
[Wordsworth, E. B. iv. 118. “Mr. Gilpin took down the glove, and put it up in his bosom.” Ibid. iii. 490. “With that he (Cranmer) pulled out of his bosom their two letters.” Ibid. 601. “Putting his hand into his bosom, he drew forth his prayer.”]

c
so heavily E.

d
it is expedient D.F.

1
[See E. P. V. lxxvii. 14.]

e
the E. (?D.)

f
in the charge . . . with him om. E.F. 1676.

g
my D.

h
grieved E.F.

1
A mere formality it had been to me in that place, where as no man had ever used it before me, so it could neither further me if I did use it, nor hinder me if I did not.

2
[Bishop Saunderson, Pref. to Ussher on the Power of the Prince, § 19. “The ministers of that party, in their prayers before and after sermon, do not usually shew themselves over studious of brevity.” In the Geneva Prayer Book are forms or specimens of prayers to be used after the sermon, all of greater length than any before the sermon. See Phœnix, ii. 217, 20, 24.]

i
that E.

k
deposed F.

l
heard me then D.

m
occurrence D.

n
and sometimes D.

o
conference D.

1
[Archbishop Parker’s Advertisements, 1564; in Strype, Park. iii. 88. “Item, That all communycantes do receve kneeling. . . . , and not sittinge, or standinge.”]

2
[“I know not how,—our carriage, a many of us, is so loose; covered we sit; sitting we pray; standing, or walking, or as it takes us in the head, we receive.” Bp. Andrewes’ Sermons, fol. 549, preached on Easter Day, 1621.]

p
other D.

q
called openly E.F.

r
that they E.

s
but an order E.B.

t
good customs B.

u
granting E.F.

x
Alvey’s E.D.F.

y
each house B.D.

z
were E.F.

a
seemed E.F.

b
that om. E.F.

c
any way E.F.

d
it om. D.

e
much E.F.

f
this B.

1
[Bishop Aylmer: consecrated March 24, 1576-7; died June 3, 1594. Strype.]

g
the om. E.F.

h
persuasions B.

i
would D.B.

k
experiences D.

l
were D.

m
displeasing E.F.

n
towste D.

o
setting down B.

p
an om. D.B.

1
His words be* these: “The next Sabbath-day after this, Mr. Hooker kept the way he had entered into before, and bestowed his whole hour and more only upon the question† he had moved and maintained. Wherein he so set out the agreement of the church of Rome with us, and their disagreement from us, as if we had consented in the greatest and weightiest points, and differed only in certain smaller matters. Which agreement noted by him in two chief points, is not such as he would have made men believe: the one, in‡ that he said, they acknowledge all men sinners, even the Blessed Virgin, though some of them freed her from sin: for the council of Trent holdeth that she was free from sin: another in that he said, they teach Christ’s righteousness to be the only meritorious cause of taking away sin, and differ from us only in the applying of it. For Thomas Aquinas, their chief schoolman, and Archbishop Catharinus, teach, that Christ took away only original sin, and that the rest are to be taken away by ourselves: yea the council of Trent teacheth that the righteousness whereby we are righteous in God’s sight is inherent righteousness; which must needs be of our own works, and cannot be understood of the righteousness inherent only in Christ’s person, and accounted unto us.”

*
be om. D.

†
questions E.

‡
in om. D.

q
of om. E.F.

r
only of justification E.F.

1
[In 3 Sent. d. iii. art. i. qu. 2. “Aliorum positio est, quod sanctificatio Virginis subsecuta est originalis peccati contractionem; et hoc quia nullus immunis fuit a culpaoriginalis peccati, nisi solum filius Virginis;” quoting Rom. iii. . . . “Hic autem modus dicendi communior est, et rationabilior, et securior. Communior, inquam, quia omnes fere illud tenent, quod beata Virgo habuerit originale, cum illud appareat ex multiplici ipsius pœnalitate . . . Rationabilior . . . . quia esse naturæ præcedit esse gratiæ vel tempore vel natura . . . . Securior, quia magis consonat fidei pietati, et sanctorum auctoritati.” t. v. 36. ed. Rom. 1596.]

s
acknowledged E.F.

t
that by this E.D.F.

u
possible D.

x
myself B.D.

y
had freed E.B.F.

z
her original E.B.F.

1
[3 Summ. Theol. qu. xxvii. art. 1, 2.]

2
This doth much trouble Thomas, holding her conception stained with the natural blemish inherent in mortal seed. [“Si nunquam anima beatæ Virginis fuisset contagio originalis peccati inquinata, hoc derogaret dignitati Christi, secundum quam est universalis omnium salvator . . . . Sed beata Virgo contraxit quidem originale peccatum, sed ab eo fuit mundata antequam ex utero nasceretur.”] And therefore he putteth it off with two answers; the one that the church of Rome doth not allow but tolerate the feast; [“Licet Romana ecclesia conceptionem B. V. non celebret, tolerat tamen consuetudinem aliquarum ecclesiarum illud festum celebrantium. Unde talis celebritas non est totaliter reprobanda:”] which answer now will not serve: the other that being sure she was sanctified before birth, but unsure how long a while after her conception, therefore, under the name of her conception-day, they honour the time of her sanctification. So that besides this, they have now no soder to make the certain allowance of their feast, and their uncertain sentence concerning her sin to cleave together. [“Nec tamen per hoc, quod Festum Conceptionis celebratur, datur intelligi quod in sua conceptione fuerit sancta: sed quia quo tempore sanctificata fuerit ignoratur, celebratur Festum Sanctificationis ejus potius quam Conceptionis in die conceptionis ejus.”] Thomas, iii. part. quæst. 27, art. 2. ad 2m. et 3m. [t. xii. 101, 102.]*

*
Note om. D.

a
quote E.F.

b
pronounced D.B.

1
[Sess. v. Decret. de Peccato Originali, ad fin.] 1886.

2
Annot. in Rom. v. sect. 9. [v. 14. “ ‘Death reigned from Adam to Moses,’ not in them only which actually sinned, as Adam did, but in infants which never did actually offend, but only were born and conceived in sin, i. e. by having their natures defiled, destitute of justice, and averted from God in Adam, and by their descent from him. Christ only excepted, being conceived without man’s seed: and his mother, for his honour and by his special protection (as many godly devout men judge) preserved from the same.”]

c
Is it their wont to speak nicely E.

d
that D.B.

e
their om. E.F.

f
the D.

g
Andradens D. and so whenever the name occurs.

1
Lib. v. Defens. Trid. Fidei. [“Defen. Fidei Catholicæ et integerrimæ, quinque libris comprehensa, adversus hæreticorum detestabiles calumnias, et præsertim Martini Kemnitii Germani: autore illustri et R. D. Dieguo Payva d’ Andrada, Lusitano, insigni S. Theol. Doctore.” Ingolstadt, 1580. lib. v. pars iii. p. 487. “Mirari sane nemo debet, si in re, quæ nullis est vel Scripturæ sacræ apertis testimoniis, vel Patrum traditione, vel Ecclesiæ definitione constituta, variæ sint piorum et doctorum hominum sententiæ: suntque profecto nimium morosi, qui vel Deiparæ Virginis splendorem ita amplificant, ut illis succenseant, qui cum eam negent sine peccato fuisse conceptam, pro Christi se dignitate pugnare arbitrantur, vel qui Christi prærogativam sine aliqua sacrosanctæ Virginis macula retineri posse desperant.” 489. “Tridentini Patres non quid esset in hac quæstione certum, sed nihil esse in ea adhuc certum et exploratum, cum Xisto Roman. Pontif. pronunciarunt.” Andrada died 1575. See the part which he took in the council of Trent, Fra Paolo, vi. 30, 44. His “Defensio” is dedicated to Gregory XIII.]

h
the Scripture D.

i
men’s D.

k
do om. E.

l
or D.

1
Orthod. Except. lib. iii. [“Orthodoxarum Explicationum Libri Decem,” Cologne, 1564; against Chemnitz; lib. iii. p. 241. “Si vera sunt Christi verba, quibus salutem sempiternamque vitam iis solis pollicetur qui crediderint et baptizati fuerint, felicitatis autem aditum sola peccata occludere, Christi vero sola merita recludere possunt; annon satis constat, Baptismatis sacramentum Christi passionem et merita ita habere colligata, ut animam peccatis expunctis sanctificet, et beatitudinis aditum aperiat?”]

m
do D.F.

n
unto E.B.F.

2
In 4 Sent. dist. 1. quæst. 4. [3.] art. 6. [“Supponitur tam veteres quam nos, immo universos post peccatum Adæ per solam Christi passionem obtinere salutem . . . . cum autem nulla causa valeat nisi per ejus applicationem suum effectum assequi, nemo nisi per applicationem ejusdem passionis salutem consequitur.” p. 39. Douay, 1613.]

3
[Sess. vi. Decr. de Justif. c. 7. “Hujus justificationis causæ sunt; finalis quidem gloria Dei et Christi, ac vita æterna: efficiens vero, misericors Deus, qui gratuito abluit et sanctificat, signans et ungens Spiritu promissionis Sancto, qui est pignus hæreditatis nostræ: meritoria autem dilectissimus Unigenitus suus Dominus noster Jesus Christus, qui cum essemus inimici, propter nimiam caritatem, qua dilexit nos, sua sanctissima passione in ligno crucis nobis justificationem meruit, et pro nobis Deo Patri satisfecit: instrumentalis item sacramentum Baptismi, quod est sacramentum fidei, sine qua nulli unquam contigit justificatio.”]

1
[E. g. Dialog. de Justif. fol. 74, “Hæret. Vos ergo negatis imputari vobis justitiam Christi? Cathol. Nova hæc verba sunt et vestra, quæ tamen si ad verum sensum trahantur, non gravaremur recipere. Sic enim imputatur nobis justitia Christi, ut per ejus meritum solvamur a præcedentibus delictis, et induamur nova vera justitia dono Dei superno, per quam vere justi efficiamur coram Deo.”]

2
“Nemo Catholicorum unquam sic docuit; sed credimus et profitemur Christum in cruce pro omnibus omnino peccatis satisfecisse, tam originalibus quam actualibus*.” Bellarm. Judic. de Lib. Concor. Mendac. 18.† [He is protesting against the following statement in the “Concordia” of the Lutherans, 1581. “Accessit opinio, quod Christus satisfecerit sua passione pro peccato originis: et instituerit Missam, in qua fieret oblatio pro quotidianis delictis mortalibus et venialibus.” This which seems meant to describe an ill effect of the Romish doctrine, Bellarmine understood as descriptive of the doctrine itself: and he stigmatizes it accordingly. “Impudenti mendacio tribuitur Catholicis doctoribus illa divisio, quod Christus passione sua satisfecerit solum pro peccato originis, pro actualibus autem instituerit Missam. Nemo enim Catholicorum,” &c. Opp. t. vii. col. 604. Colon. 1617. The Dublin Copy of the Answer to Travers has here the following note: “Vide Bellarminum eodem capite, pag. 89. Ipse (opinor) secum in hac re pugnat.”]

*
Quotation om. D.

†
Robert Bellarius Judic. de Libr. Concor. D.

o
which om. D.

1
[Arist. Ethic. i. 2. ἕκαστος δὲ κρίνει καλω̑ς ἃ γινώσκει, καὶ τούτων ἐστὶν ἀγαθὸς κριτής.]

p
traduced D.

q
discovered and discoursed of B.

r
would be made B. would have E.F.

1
Calv. Inst. l. i. c. 16. sect. 9. [“Videmus, non temere in scholis inventas fuisse distinctiones, de necessitate secundum quid, et absoluta; item consequentis et consequentiæ.”]

s
harder sentences B. a harder D. if he wrote this in earnest B.

t
things I spoke D.

u
come D.

x
judgment E.F.

y
himself om. E.F.

1
In the Advertisements published in the seventh year of her Majesty’s reign: “If any Preacher, or Parson, Vicar, or Curate so licensed, shall fortune to preach any matter tending to dissension, or to derogation* of the religion and doctrine received, that the hearers denounce the same to the Ordinary, or the next Bishop of the same place, but not† openly to contrary or to impugn the same speech so disorderly uttered, whereby may grow offence and disquiet of the people, but shall be convinced and reproved by the Ordinary after such agreeable order as shall beseem‡ to him, according to the gravity of the offence: and that it be presented within one month after the words spoken.” [This is found, with some verbal differences, in Strype, (Park. iii. 86.) as one of the Ordinances accorded by the Archbishop of Canterbury. But the preamble states, that “the Queen’s Majesty. . . hath by the assent of the metropolitan and with certain other her commissioners in causes ecclesiastical decreed certain Rules and Orders to be used, as hereafter followeth. . . .as constitutions mere ecclesiastical.” This preamble was afterwards altered, in consequence of the Queen’s sanction being refused through Leicester’s influence: “whereat the Archbishop was greatly displeased.” (Ibid. 314, 15.)]

*
the derogation D.

†
no man D. note om. B.

‡
be seen to E.F.

2
[Strype, Park. iii. 65. Queen Elizabeth addressed her letters to Archbishop Parker, dated Jan. 25, 1564, requiring him to confer with the bishops of his province on the best mode of repressing the disorders of nonconformists. Ibid. 313—20. “The Archbishop and some other bishops of the ecclesiastical commission proceeded to compile certain articles, . . . . which were printed with a preface this year 1564. . . . . and entitled Advertisements. .because the book wanted the Queen’s authority . . . so prevalent was that party in the council that disliked it. . .At length. . . these ecclesiastical rules recovered their first names of Articles and Ordinances.” See them in Sparrow’s Collection, p. 123.]

z
ymproving D.

a
discover first E.F.

b
willing D.

c
conceive F.

d
should E.B.

e
conceiveth D.

f
fact D.

g
cases D.

h
a secret D.B.

i
t E.F.

k
served D.

l
wherefore E.B.F.

m
his om. B.

n
by om. D.

o
speeches D.

p
whereon D.

q
to do D.

r
uncharitably D.

s
considerate advice E.F.

t
good om. D.

1
[v. Ecclus. xix. 11, 12.] 1886.

u
desired D.

x
hear D.B.

y
and heard D.

z
true om. E.

a
doth E.

2
[v. Petri Martyris Locorum Communium Cl. I. Cap. ii. §§ 11, 12. Cl. II. Cap. iii. §§ 9, 10. Ed. 1583.] 1886.

b
alwaie D.

c
setteth D.

d
in the deeps D.

e
how D.

f
to om. E.P.F.

1
[See before, App. to E. P. b. v. in vol. ii. 564-576.]

g
authors D.

h
heathens E.F.

i
sentences D.

k
it E.F.

l
make om. D.

m
to om. E.F.

n
a reason D. reasons E.F.

o
my E.F.

p
without F.

q
cannot E.F.

r
dark E.B.F.

s
many man’s E.F.

t
this D.

u
nousled E. nozled F.

x
as D.B.

y
so that this E.B.F.

z
the om. E.B.F.

1
[Sozom. i. 17; Theod. i. 11.] 1886.

a
the length E.

b
so great E.

1
[Collated with the first Edition, printed by Joseph Barnes, Oxford, 1612, 4to.] 1887.

a
Abac. 1612, F. 1622.

b
is, commonly vulgar and trivial. 1612, F. 1622.

c
happiness of om. D.

d
nurces, 1612, F. 1622.

e
ends 1612, E.F. 1622.

1
[Heb. xiii. 14.]

f
kam om. E. cam 1612, F. 1676.

2
[See Coriolanus, act iii. sc. 4.

Sicin. “This is clean kamme.”

Brut. “Merely awry.”]

g
to om. E. not F, or 1612.

h
also hath 1612, E.F. 1622.

i
shall 1612, E.F. 1622.

k
heathens E. not F.

1
Rom. [ii. 14, 15.]

l
mo 1612, moe F. 1622, more Keble (1676).

m
all in D.

n
habilitie 1612, D.F. 1622.

1
Rom. [ii. 14, 15.]

o
hability E.F. habilitie 1612, 1622.

p
their not E. not F.

q
om. not 1612, F. 1622.

r
thought 1612, F. 1622, 1676, v. p. 650.

s
those hideous 1612, E.F. 1622.

t
ougly 1612, F.

1
[Rev. ii. 19, 20.]

2
[Acts viii. 21.]

u
eminent 1612, E.F. 1622.

x
the 1612, E.F. 1622.

y
no stop after “by” 1612, F. 1622.

z
inevitable 1612, F. 1622.

a
ways 1612, D.F. 1622.

b
few men 1612,E.F. 1622.

c
most wretched 1612, E.F. 1622.

d
or D.

e
takes 1612, 1622.

f
except 1612, F. 1622.

g
excellencies E. excellencie F.

h
you D.

1
Luke xxii. [25, 26.]

1
[Matt. xx. 21.]

i
a full stop here, 1612, 1622.

k
fansifull 1612, fansiefull F. 1622.

2
Luke xxii. 28, [30.]

l
is no 1612, E.F. 1622.

m
are seemlier 1612, E.F. 1622.

n
callings E.F.

o
uglie F.

p
exempled D.

q
specially 1612, D.F. 1622.

r
percingly 1612, F.

1
[Ver. 13.]

s
Corah 1612, F. 1622.

t
mutinies, repining 1612, E.F. 1622.

u
in a 1612, E.F. 1622.

x
estate D. state 1612, F. 1622.

y
how that when 1612, E.F. 1622.

z
the om. 1612, F. 1622.

a
[appale 1612, F. 1622, 1676, v. p. 481.]

b
praise 1612, F. 1622.

c
harm by strife D.

d
the cause 1612, E.F. 1622.

e
and the 1612, E.F. 1622.

f
helps 1612, E.F. 1622.

g
which tend 1612, E.F. 1622.

1
[1 Cor. iv. 8.]

2
[Isai. i. 6.]

h
unrighteous D.

3
[“De meritis meis non præsumo . . . sæpe enim nostra justitia, ad examen divinæ justitiæ deducta, injustitia est; et sordet in districtione Judicis, quod in æstimatione fulget operantis.” Greg. Moral. in Job. 5. 21, 56 in Alulfus’ Compilation on 1 Cor. iv. 3.—Greg. Opp. iv. p. 817.] 1887.

i
the E.

k
habilitie D.

l
works 1612, F. 1622.

m
to a voluntary 1612, E.F. 1622.

1
Annot. Rhem. in 1 Cor. iii. [8. “Every one shall receive his own reward according to his own labour.” “A most plain proof that men by their labours, and by the diversities thereof, shall be diversely rewarded in heaven; and therefore that by their works proceeding of grace they do deserve or merit heaven, and the more or less joy in the same. For though the holy Scripture commonly use not this word merit, yet in places innumerable of the Old and New Testament the very true sense of merit is contained, and so often as the word merces, and the like be used, they be ever understood as correlatives, or correspondent unto it. For if the joy of heaven be retribution, repayment, hire, wages, for works, (as in infinite places of holy Scripture,) then the works can be no other but the value, desert, price, worth, and merit of the same. And indeed this word reward, which in our English tongue may signify a voluntary or bountiful gift, doth not so well express the nature of the Latin word, or the Greek, which are rather the very stipend that the hired workman or journeyman covenanteth to have of him whose work he doth, and is a thing equally and justly answering to the time and weight of his travels and works, (in which sense the Scripture saith, ‘Dignus est operarius mercede sua,’) rather than a free gift.”]

n
so 1612, E.F. 1622.

1
[Psalm cxix. 71.]

2
[2 Cor. xii. 7.]

o
happily 1612, D.

p
medicinalle D.

3
[De Civ. Dei, xiv. 13. “Audeo dicere, superbis esse utile cadere in aliquod apertum manifestumque peccatum, unde sibi displiceant, qui jam sibi placendo ceciderant. Salubrius enim Petrus sibi displicuit, quando flevit, quam sibi placuit, quando præsumsit.” t. vii. 366 B.

q
Christall 1612, F. Chrystall, 1622.

1
[“Hucusque excusum exemplar: sequentia in eo non habentur.” Note in MS. D. The ed. of 1618 (F.) stops here. Cp. vol. i. liii. So also the ed. of 1612.]

2
[Psalm xliv. 23-27.]

r
ever?

3
2 Cor. vi. 9.

1
Deut. xxx. 19.

2
Psalm xxxvi. 7.

3
John vi. [v. 26.]

1
Gal. iv. [19.]

1
Psalm xxxiv.

1
[Arist. Rhet. i. 9. δι’ ἣν τὰ αὑτω̑ν ἕκαστοι ἔχουσι, καὶ ὡς νόμος.]

1
[XII Tab. Fragm. ad calc. Cod. Justin. ed. Gothofred. tit. 27. p. 91. “Interfici indemnatum quemcunque hominem, etiam XII Tabularum decreta vetuerunt. Hæc Salvianus Episc. Massiliensis; 8. de Judicio et Providentia.”]

2
[Ibid. tit. 26. “Servius ad illos versus Virgilii 6 Æn. 609. ‘fraus innexa clienti;’ ‘Ex lege,’ inquit, ‘XII Tabularum venit, in quibus scriptum est, Patronus, &c.’ ”]

3
Deut. xiii.

1
[Xenoph. Hist. Græc. lib. vi. The rest is a mere case put by Hooker for argument’s sake.]

1
Psalm cxv. 3.

2
Matth. xi. 25.

3
Ephes. i. 11.

1
Ephes. i. 6.

1
[“Soon or syne,” as Archdeacon Cotton has pointed out to the editor, is the Scottish expression for “soon or late.” See Jamieson’s Scottish Dictionary, voc. Syne. He quotes Baillie’s Letters, i. 355. “What I know I shall ever give you an account of soon or syne.”]

2
Esay xxxviii.

3
Psalm lxxxvi.

4
Esay lxvi.

5
Jer. xxxi.

6
Hos. xiv.

1
Gen. xviii. 25.

1
1 Reg. iii. 14.

2
Ps. lv. 23.

1
Ps. xxv. 13.

2
Esai xxx. 23.

3
Deut. xxviii.

1
[Valerius Maximus, II. x. 7.] E. M.

1
[Heb. xi. 25.]

1
[Aulus Gellius, XII. xi. 7. “Alius quidam veterum poëtarum, cujus nomen mihi nunc memoriæ non est, Veritatem Temporis filiam esse dixit.”] E.M. qu. Bacon, N.O. i. 84.

2
[The reading of the MS. here is doubtful.]

3
[De Prov. qu. in Parsons, Christian Directory, c. ii. § 2.] E.M.

1
Psalm lxxiii.

1
[Vid. S. Tho. Aquin. 2 Summ. Theol. pars i. qu. 86. “De Macula Peccati;” et qu. 87. “De reatu pœnæ,” especially art. 6. respons.]

1
[S. Tho. Aquin. in 4. Sent. dist. xiv. qu. ii. art. 1; et dist. xviii. qu. i. art. 3. “Pœna est duplex; scil. exterminans hostes;—et talis pœna ex reconciliatione ipsa removetur:—alia pœna est quæ corrigit civem et filium, vel amicum, et debitum ejus potest remanere reconciliatione jam facta: et ideo simul cum peccatum remittitur quoad maculam, remittitur quoad pœnam æternam quæ est exterminans, sed non quoad pœnam temporalem quæ est corrigens.”]

2
[Id. dist. xx. qu. i. art. 3. “Quidam dicunt, quod indulgentiæ non valent ad absolvendum a reatu pœnæ, quam quis in purgatorio secundum judicium Dei meretur, sed valent ad absolutionem ab obligatione, qua sacerdos obligavit pœnitentem ad pœnam aliquam, vel ad quam etiam ordinatur ex canonum statutis. Sed hæc opinio videtur non vera . . . quia . . . Ecclesia hujusmodi indulgentias largiens seu dans magis damnificaret quam adjuvaret, quia remitteret ad graviores pœnas, scil. purgatorii, absolvendo a pœnitentiis injunctis.” et dist. xxi. qu. i. art. 1.]

3
1 Cor. xi.

1
[S. Tho. Aquin. in 4 Sent. d. xxi. qu. i. art. 1. “Ad tertiam quæstionem dicendum, quod in purgatorio erit duplex pœna. Una damni, in quantum sc. retardantur a divina visione: alia sensus, secundum quod ab igne corporali punientur; et quantum ad utrumque pœna purgatorii minima excedit maximam pœnam hujus vitæ.”]

2
[Ibid. “Sancti Patres ante adventum Christi fuerunt in loco digniori quam sit locus in quo purgantur animæ post mortem, quia non erat ibi aliqua pœna sensibilis: sed locus ille erat conjunctus inferno, vel idem quod infernus, alias Christus ad limbum descendens non diceretur ad inferos descendisse: ergo et purgatorium est in eodem loco, vel juxta infernum.”]

3
[The words “pœnam damni” are changed into “predestination,” in an old transcript of this sermon: Library, Trin. Coll. Dubl. (MS. B. I. 13). Contractions, in some measure, account for the mistake. So Mr. Gibbings informs the Editor.]

1
[Ibid. “De loco purgatorii non invenitur aliquid expresse determinatum in scriptura, nec rationes possunt ad hoc efficaces induci: tamen probabiliter et secundum quod consonat magis sanctorum dictis, et revelationi factæ multis, locus purgatorii est duplex . . Secundum legem communem locus purgatorii est locus inferior inferno conjunctus, ita quod idem ignis sit qui damnatos cruciat in inferno, et qui justos in purgatorio purgat, quamvis damnati secundum quod sunt inferiores merito, et loco inferiores ordinandi sunt. Alius est locus purgatorii secundum dispensationem, et sic quandoque in diversis locis puniti leguntur, vel ad vivorum instructionem, vel ad mortuorum subventionem, ut viventibus eorum pœna innotescens, per suffragia ecclesiæ mitigaretur.”]

2
[Ibid. art. 3. “Culpa venialis in eo qui cum gratia decedit, post hanc vitam dimittitur per ignem purgatorium, quia pœna illa aliqualiter voluntaria virtute gratiæ habebit vim expiandi culpam omnem quæ simul cum gratia stare potest.”]

1
[At this point unfortunately the Dublin MS. of this sermon breaks off: and no other copy has been found to supply the deficiency. On a leaf at the end of it is written, apparently by three different scribes:

“Abacuc 2. 4.”

“The first part printed: The rest not.”

“Mr. Hooker, in his own hand.”]

1
[Luke xxiii. 28.]

1
Psalm lxxiii. 3.

2
[Isa. xlviii. 22.]

3
Eccles. vi. 2.

1
[Job xx. 16.]

2
[“To” omitted Ed. [1618, and] 1622: which has been collated with the first edition, as having possibly been inspected by H. Jackson.]

1
[An allusion doubtless to the strewing of rushes along the passages leading to banqueting rooms. See (e. g.) Taming of the Shrew, act iv. sc. 1. “Is supper ready, the house trimmed, rushes strewed . . . the carpets laid, and every thing in order?”]

1
[Num. xxiii. 10.]

1
[Rev. xiv. 13.]

2
[There seems to be a mistake in this reference: the only scriptural passage corresponding to it being Ecclesiasticus xxi. 2.]

3
[“Add” in the editions of 1612, 1618, 1622.]

1
[“thought” cp. p. 601, ed. 1618, 1622.]

2
[Dan. v. 6.]

3
[“The twenty-one (21.) of the Revelation.” ed. 1622 and 1618.]

4
Heb. v. 7.

1
[Babylonians 1618.]

2
[the om. 1618.]

3
Rev. xviii. 7.

4
[Psalm lxxxviii. 15.]

5
[Psalm ix. 20.]

6
[“hath” edit. 1612; “have” edit. 1618.]

1
[Matt. xxvi. 39.]

2
[Job xiii. 15.]

3
[2 Tim. i. 12.]

4
[Prov. xxii. 3.]

5
Isa. xxvi. 20.

1
Lib. iv. c. 6. de Doct. Chr. [“Audeo dicere, omnes qui recte intelligunt quod illi loquuntur, simul intelligere non eos aliter loqui debuisse.” iii. 67. The rest of the sentence (though printed by Jackson as a quotation) does not appear in St. Augustine.]

a
Castellio his, 1614, 1618, 1622.

2
[In his translation, published at Basle, 1546, and severely censured by Beza in the notes to his own version of the New Testament, 1556, for sacrificing accuracy and propriety to grammatical purity.]

b
Alphonsus X, 1614.

3
Rod. Tolet. lib. iv. c. 5. [ap. Script. Rer. Hispan. t. i. p. 377. Francof. 1579.]

1
2 Pet. i. 19.

2
Præf. in Orat. D. Rainold.

3
Parsons in 3 Convers. [“The third part of a Treatise entitled, ‘of Three Conversions of England;’ containing an Examen of the Calendar or Catalogue of Protestant Saints, Martyrs and Confessors, devised by Fox, and prefixed before his huge volume of Acts and Monuments; with a parallel or comparison thereof to the Catholic Roman Calendar, and Saints therein contained. The last six months.” 1604. p. 215.]

c
using of, 1614, 1618, 1622.

4
Mal. ii. 7.

5
Canus, Locor. lib. xi. c. 6. [“Nec ego . . . auctorem excuso . . . historiæ ejus, quæ Legenda Aurea nominatur . . . Hanc homo scripsit ferrei oris, plumbei cordis, animi certe parum severi ac prudentis.” p. 658. Lovan. 1569.] Vives, lib. ii. de corrupt. Art. [De Causis corruptarum Artium, ii. t. i. 371. Opera, Basil. 1555. “Quam indigna est divis et hominibus Christianis illa Sanctorum historia, quæ Legenda Aurea nominatur, &c.”] Hard. lib. iv. [“A Detection of sundry foul Errors, Lies, Slanders, &c. uttered and practised by M. Jewel, in ‘A Defence of the Apology,’ &c. By Tho. Harding, D.D. Lovan. 1568.” b. 4. fol. 251. “There is an old moth-eaten book, wherein saints’ lives are said to be contained. Sometimes it is called Legenda Aurea, sometimes Speculum Sanctorum, &c. . . . It shall not greatly skill who was the author of it: certain it is that among some true stories there be many vain fables written.”]

d
Mr. 1614.

1
Pag. 1903. edit. 1570. [p. 1731. ed. 1583.]

2
In the third part of the Three Conversions of England: in the Examination of Fox’s Saints, c. 14. sect. 53, 54. p. 215.

3
Sect. 55.

4
Plutarch. in Demosthen. [c. 23.]

5
Liv. Dec. i. lib. ii. an. U. C. 60. [c. 32.]

1
1 Tim. ii. 8.

2
Annal. tom. i. an. 57. n. 109, 110. et tom. ii. an. 132. n. 5.

3
“S. Paulus de sua salute incertus;” Richeom. Jesuit. lib. ii.c. 12. Idololat. Huguenot. p. 119. in Marg. edit. Lat. Mogunt. 1613. interpret. Marcel. Bomper. Jesuita. [“S. Paul craint de n’estre pas sauvé.” Œuvres du R. Pere Louis Richeome, Provençale Religieux de la Compagnie de Jesus. Paris, 1628. t. i. p. 627.]

4
Witness the verses of Horatius, a Jesuit, recited by Possev. Biblioth. Select. part. 2. lib. xvii. c. 19. [27. p. 449. Colon. 1607.]

Exue Franciscum tunica laceroque cucullo:
Qui Franciscus erat, jam tibi Christus erit.
Francisci exuviis (si qua licet) indue Christum:
Jam Franciscus erit, qui modo Christus erat.
[Horatius Torsellinus was the author of this epigram: which is entitled, “De S. Francisci Stigmatibus;” and concludes with the following couplet:

“Quid cœlestis amor non audes? fingis amantes
Arte nova, effigies ut sit amantis amans.”]
The like hath Bencius, another Jesuit. [Francisci Bencii, e soc. Jesu, Carm. lib. iv. 25. Ad Effigiem S. Francisci.

“Tene ego Francisce aspicio vel Christe? tabella
Una tuam effigiem reddit, et una tuam.
Quod si Francisci, nostræ monumenta salutis
Vulnera cur palmas, corda, pedesque premunt?
Si Christus; quid vult onerosa lacerna, nec altam
Dulce crucem pondus pendula membra gravant?
Induit, ut video, Christi Franciscus, et ille
Francisci vultus: alter utrumque refert.
Quid non mortales cogat divinus? in unum,
Non animos tantum, corpora conflat amor.”
P. 203. Ingoldstadt. 1599.]
1
[2 Cor. v. 1.]

1
[Collated, not with the first edition of 1613, which the editor has not been able to obtain a sight of, but with that of 1622.] (Now collated with the first edition, in which the Dedication bears the date Jan. 13i 1613, and the title-page the date 1614: and also with the edition of 1618.) 1887.

2
[so 1614 and 1618. “occasions” ed. 1622.]

a
enimies, 1614.

1
[Ver. 18, 19.]

2
[2 Pet. i. 19, 20.]

1
Job xv. 2, 3.

2
Wisd. ix. 6.

3
[Job xlii. 3.]

4
[2 Macc. xv. 38.]

5
Isa. xlix. 3.

6
[1 Cor. ii. 12, 13.]

1
Ezek. iii. 2, 3.

b
eate, 1614.

2
[Ezek. iii. 14.]

3
[Rom. i. 16.]

4
[Matt. xxvii. 46.]

1
[Luke x. 16.]

2
James ii. 2.

3
Acts xii. 22; [xvii. 18.]

4
James ii. 5.

1
[De Trin. ii. 12. 795. ed. Bened. “Superest de inenarrabili generatione Filii adhuc aliquid, imo aliquid illud adhuc totum est. Æstuo, differor, hebesco, et unde incipiam nescio . . . Quem imprecer? quem implorem? ex quibus libris ad tantarum difficultatum enarrationem verba præsumam? Evolvam omnem Græciæ scholam? Sed legi, Ubi sapiens? ubi conquisitor sæculi? In hoc ergo sophistæ mundi et sapientes muti sunt, sapientiam enim Dei reprobaverunt. Scribam ergo legis consulam? Sed ignorat; quia ei crux Christi scandalum est. Hortabor forte vos connivere et tacere, quia ad venerationem satis sit ejus qui prædicatur, leprosos emundatos fuisse, surdos audisse, claudos cucurrisse, paralyticos constitisse, cæcos lumen recepisse, cæcum ab utero oculos consecutum, dæmonas fugatos, ægrotos revaluisse, mortuos vixisse? Sed hæc hæretici confitentur, et pereunt.” § 13. “Expectate itaque nihil minus claudorum cursu, cæcorum visu, fuga dæmonum, vita mortuorum. Consistit enim mecum, in patrocinium editarum superius difficultatum, piscator egens, ignotus, indoctus, manibus lino occupatus, veste uvida, pedibus limo oblitus, totus e navi. Quærite et intelligite, utrum mirabilius fuerit mortuos excitasse, an imperito scientiam doctrinæ istius intimasse. Ait enim, In principio erat Verbum.”]

c
Solomon, 1614.

d
ear, 1614, 1618.

1
[So in “The Device for Alteration of Religion, in the first year of Queen Elizabeth,” ap. Strype, A. i. 2. p. 394. Oxford, 1824. “A cloaked papistry, or a mingle mangle.” And in a Letter of certain Puritans to the Bp. of Norwich, (Parkhurst) A. ii. 2. p. 455. “You think God may be served with a mingle-mangle.”]

2
[S. Jerome, Adv. Helvid. c. 19. v. supra, p. 501.]. E. M.

1
[In Ps. cxxxii. Col. 463. ed. Ben.] E. M.

1
[E. P. V. lxxvii. 3. “They which have once received this power may not think to put it off and on like a cloak as the weather serveth.”]

2
[Gen. xxi. 12.]

1
[Job xxxviii. 11.]

2
[Acts ii. 37.]

3
[Acts xxvi. 28.]

4
[Rom. x. 2.]

5
Acts xvii. 22.

e
staid, 1614.

1
[1 John ii. 19.]

2
[So 1614, 1618: “hath” ed. 1622.]

1
[1 John iii. 14.]

2
[2 Cor. xiii. 5.]

f
enimies, 1614.

3
Coloss. i. 21-23.

4
Ver. 24.

g
wained, 1614, 1618: weyned, 1622.

h
Solomon, 1614.

5
[Job xxi. 11.]

1
[Psal. cxix. 14.]

2
[John vi. 33.]

3
[Rev. xv. 2, 3.]

4
[Ver. 39.]

5
[Gal. iv. 6.]

6
[John xvii. 25, 26.]

1
[There are but two dates from which these twenty-four years may be reckoned: the accession of Queen Elizabeth, 1558, or the publication of the bull of Pius V. against her, 1570. This sentence was therefore probably written either in 1582 or 1594. The latter date is perhaps the preferable one, as a book is afterwards quoted which was published in 1583. See note 4. p. 677.]

2
Acts xxiv. 14.

3
Rev. xviii. 4.

4
2 Chron. xiii. 5.

i
Solomon, 1614.

5
Cant. viii. 11.

1
Acts xx. 28.

2
[John xxi. 22, 23.]

3
Matt. xvi. 18.

k
Solomon, 1614.

l
Jeroboam, 1614 & 1618.

1
[Luke xxii. 25, 26.]

2
Conc. de Lector. Cardin. [de Electione Cardinalium: e. g. Constant. ad 1418. Sess. xliii. t. viii. 876. “De numero et qualitate dominorum Cardinalium. . . . . Sint viri in scientia, moribus et rerum experientia excellentes: doctores in theologia, aut in jure canonico vel civili: præter admodum paucos, qui de stirpe regia, aut ducali, aut magni principis oriundi exsistant, in quibus competens literatura sufficiat.” Comp. Conc. Basil. 1436. Sess. xxiii. § 4. t. viii. 1207.]

m
cousenage, 1614; coozenage, 1618, 1622.

3
Laurent. Surius Com. de reb. gest. a Pio V. [“Commentarius brevis rerum in orbe gestarum ab ad 1500 usque ad ad 1574.” Colon. 1574. p. 486. “Est Romæ Academia, quam vulgo Sapientiam vocant. Ea jam collapsa erat, et ejus liberales annuos proventus quidam sibi vindicarant. At Pontifex eos jussit restitui, ut iis alantur scientiarum omnium egregii professores.”]

4
[Ibid. p. 485. “Impudicas mulieres, publico proposito edicto, ab urbe expelli jussit. Erant tum Romanorum plerique qui se vehementer opponerent, dicerentque si meretrices ab urbe excluderentur magnum id Reip. annui quæstus dispendium allaturum. Pontifex tandem ita eis silentium imposuit, ut asseveraret se cum tota curia sua alio migraturum, nisi illa hominum fæx profligaretur. Ita illæ coactæ sunt discedere, interim tamen nonnullæ in vicis ignobilibus ob pejora vitanda relictæ sunt, cum prius in viis publicis et splendidis ædibus magno numero habitarant.” P. 486. “Meretricibus, quas in unum urbis angulum rejecit, severiter præcepit, ne per urbem vagentur: contra facientes jussit publice flagellari. Duo vero aut tria designavit templa, quæ sacrificii et concionis auscultandæ causa petant. Ex iis pleræque nupserunt, aliæ complures cupiunt, honesta reperta conditione, ex ea se turpitudine extrahere; nec Pontifex se defuturum illis dixit, quas paupertas a sectanda pietate remoraretur. Eas autem, quæ sine sacramentis in iis sordibus decederent, in sterquiliniis vult sepeliri, quod multis calcar addat ad meditandam resipiscentiam.”] Francisc. Sansovin. de Gubern. Regnor. et Rerumpubl. t. xi. [xii.] cap. de Jud. Marescal. et Soldan. [“Del Governo et Amministratione di diversi Regni,” &c. Vinegia. 1583, fol. 89. “Le meretrici della città pagano ogni anno un certo censo che essi chiamano tributo.”]

n
he, 1614, 1618.

1
2 Chr. xiii. 8, 9.

o
ougly, 1614, 1618, 1622.

p
have, 1614.

1
[Bulla Pii V. 26 Apr. 1570. “Missæ sacrificium, preces, jejunia, ciborum delectum, cælibatum, ritusque Catholicos abolevit.” ap. Cherubini, Bullarium. tit. ii. 229. Rom. 1638.]

2
[Strype, Park. ii. 392. “Anno 1574. popish books imported. Motives to the Catholic Faith by Richard Bristow, Priest, Licentiate in Divinity. Imprinted at Antwerp 1574.” Ann. II. i. 498. A book of great vogue with the papists, which Dr. Fulk of Cambridge now answered in a treatise called The Retentive. In the year 1599 it was published again at Antwerp. And again the next year, 1600, one Dr. Hill put it forth at Antwerp, entitled then ‘Reasons for the Catholic Religion,’ . . . as a new book of his own . . . which was fully and learnedly answered by Geo. Abbot, D.D., Master of University College, Oxford, afterwards archbishop of Canterbury. And in our time came out Bristow’s Motives again, with a new name, viz. The Touchstone of the new Gospel: which Dr. Simon Patrick, afterwards bishop of Ely, briefly and effectually answered.” The title of Bristow’s book is, “A brief treatise of divers plain and sure ways to find out the truth in this doubtful and dangerous time of heresy, containing sundry Motives unto the Catholic Faith: or, Considerations to move a man to believe the Catholics, and not the heretics.” He was born at Worcester, and bred in Oxford, where he was made Petreian Fellow of Exeter College, 1567. Two years afterwards he conformed to the church of Rome and went over to Douay, where and at Rheims he read lectures in divinity; and died in England, 1582. Wood, Ath. Oxon. i. 482; who adds, that “he collected, and for the most part wrote, the notes to the Rhemish Testament.”]

3
[Motives to the Catholic Faith, fol. 151, ed. 1599. This book has prefixed the testimony of Cardinal Allen, “that it is in all points catholic, learned, and worthy to be read and printed.”

1
2 Chron. xiii. 10, 11.

2
2 Chron. xiii. 12.

a
loath, 1614.

1
[Psal. cv. 15; Rev. vii. 3.]

2
[Ver. 13, 14.]

3
[Ver. 10, 11.]

4
[Ver. 10, 11, 13.]

1
Gen. vi. 3, 13; ver. 8, 18.

2
Gen. xix. 12.

3
Ver. 15.

4
Ver. 16.

5
Ver. 18-20.

1
Ver. 21, 22.

b
om. of, 1614.

2
[Ver. 31, 34.]

3
[Ver. 35.]

1
Matt. vi. 20; 1 Tim. ii. 9, 10.

2
[“shamefacedness,” Keble, but not the early edd. Cf. Spenser, F. Q. v. iii. 23, and Skeat’s Etym. Dict.]

3
[De Cult. Fœmin. ad fin. p. 161. Paris, 1664.]

c
civit, 1614, 1618, 1622.

4
[vii. 6.]

d
om. the, 1614.

e
Solomon, 1614.

5
[Luke x. 41, 42.]

1
[2 Cor. vi. 16.]

2
[John xiv. 23.]

f
lay, 1614.

3
[“setled,” 1614, 1622: “settled,” 1618.]

4
[“times,” 1614, 1618, 1622.]

1
[“We profess ourselves to have a due comfort, if truely; and if in hypocrisy, then woe worth us. Therefore ere, &c.” Edit. 1622, not 1618. “We profess ourselves to have; a due comfort, if truly, and if in hypocrisy, then woe worth us. Therefore ere,” &c. 1614.]

2
Lam. ii. 13.

g
name-sake, 1614.

1
[1 Pet. iii. 7.]

h
et passim, Sampson, 1614.

2
Ephes. ii. 19-22.

3
[1 John iv. 15.]

i
enimie, 1614.

1
1 John v. 4, 5.

2
Matt. vii. 25.

3
[Ver. 4.]

k
him, 1614, 1618, 1622.

l
reverent, trembling, 1614.

4
Rom. xi. 18.

1
1 John v. 12.

2
[Heb. xi. 33, 34.]

3
[Heb. xi. 6.]

4
John vi. 28, 29.

1
[Psalm lxxx. 8-11.]

2
Psalm lxix. 22, 23; Rom. xi. 9, 10.

m
sithens, 1614, 1618, 1622.

3
Psalm lxxx. 14.

4
Rom. xi. 20, 22.

1
Hosea i. 9. “not my people.”

2
Verse 6. “not obtaining mercy.”

3
Rom. xi. 21.

1
[De Pudicitia, c. xxii. “Sufficiat martyri propria delicta purgasse.” p. 575.]

2
[“Safe-conducting the rebels from their ships.” Richard III. iv. 4. 483.]

1
[Psalm xxxii. 1.]

n
Solomon, 1614.

2
[Rom. ix. 31-33.]

1
[Phil. iii. 8, 9.]

1
[1 Pet. v. 2.]

o
So 1614;thorow, 1618, 1622.

p
so 1622: eternall, 1614, 1618.

1
[Hosea iv. 17.]

2
Careless [margin, early edd.].

q
loathsome, 1614.

1
Amos viii. 11, 12.

2
1 Pet. iv. 17.

3
[noondays Ed. 1614, 1618, 1622.]

4
[“in middest” 1614, 1618, 1622.]

5
[Hagg. ii. 2, 3.]

6
[The ed. 1614 puts no stop; 1618 and 1622 put a comma after “nothing.”]

1
[Edd. 1614, 1618, 1622 put a full stop at “gone.”]

2
[That is, “to reprove such as gainsaid the truth.”]

r
Niniveh, 1614, 1618, 1622.

3
[Moral. in Job. lib. xxv. § 34; t. i. 807. A. ed. Bened.]

4
Jer. iii. 14, 15.

1
[Psalm cxlvii. 9.]

1
Psalm xxii. 9.

2
Psalm l. 15.

3
Prov. xxiii. 26.

1
[2 Cor. ix. 7.]

2
Luke xviii. 10-14.

1
Matt. xx. 23.

2
Matt. xix. 16, 17.

3
2 Tim. iii. 7.

4
ii. 37.

1
iii. 1.

1
Gen. xlix. 14, 15.

1
John i. 29.

2
Psalm cxvi. 4-8.

1
John xxi. 22.

2
Habak. i. 46.

3
[In Cantica, Serm. lxxxiv. 2.] 1887.

4
Rom. xi. 33.

1
Mark vi. 23; Esther vii. 2.

2
John xvi. 23.

3
James iv. 3.

4
Isa. lv. 6.

5
Psalm xxi. 3, 4.

1
James i. 5.
