\chapter*[The First Book]{THE FIRST BOOK. 
CONCERNING LAWS AND THEIR SEVERAL KINDS IN GENERAL.}
\label{chap:book1}
\addcontentsline{toc}{chapter}{THE FIRST BOOK}

THE MATTER CONTAINED IN THIS FIRST BOOK.

I. The cause of writing this general Discourse concerning Laws.

II. Of that Law which God from before the beginning hath set for himself to do all things by.

III. The Law which natural agents observe, and their necessary manner of keeping it.

IV. The Law which the Angels of God obey.

V. The Law whereby man is in his actions directed to the imitation of God.

VI. Men’s first beginning to understand that Law.

VII. Of Man’s Will, which is the first thing that Laws of action are made to guide.

VIII. Of the natural finding out of Laws by the light of Reason, to guide the Will unto that which is good.

IX. Of the benefit of keeping that Law which Reason teacheth.

X. How Reason doth lead men unto the making of human Laws, whereby politic Societies are governed, and to agreement about Laws whereby the fellowship or communion of independent Societies standeth.

XI. Wherefore God hath by Scripture further made known such supernatural Laws as do serve for men’s direction.

XII. The cause why so many natural or rational Laws are set down in Holy Scripture.

XIII. The benefit of having divine Laws written.

XIV. The sufficiency of Scripture unto the end for which it was instituted.

XV. Of Laws positive contained in Scripture, the mutability of certain of them, and the general use of Scripture.

XVI. A Conclusion, shewing how all this belongeth to the cause in question.

\PRLsep

\section*{The cause of writing this general Discourse concerning Laws.}

I. HE that goeth about to persuade a multitude, that they are not so well governed as they ought to be, shall never want attentive and favourable hearers; because they know the manifold defects whereunto every kind of regiment is subject, but the secret lets and difficulties, which in public proceedings are innumerable and inevitable, they have not ordinarily the judgment to consider. And because such as openly reprove supposed disorders of state are taken for principal friends to the common benefit of all, and for men that carry singular freedom of mind; under this fair and plausible colour whatsoever they utter passeth for good and current. That which wanteth in the weight of their speech, is supplied by the aptness of men’s minds to accept and believe it. Whereas on the other side, if we maintain things that are established, we have not only to strive with a number of heavy prejudices deeply rooted in the hearts of men, who think that herein we serve the time, and speak in favour of the present state, because thereby we either hold or seek preferment; but also to bear such exceptions as minds so averted beforehand usually take against that which they are loth should be poured into them.

[2.]Albeit therefore much of that we are to speak in this present cause may seem to a number perhaps tedious, perhaps obscure, dark, and intricate; (for many talk of the truth, which never sounded the depth from whence it springeth; and therefore when they are led thereunto they are soon weary, as men drawn from those beaten paths wherewith they have been inured;) yet this may not so far prevail as to cut off that which the matter itself requireth, howsoever the nice humour of some be therewith pleased or no. They unto whom we shall seem tedious are in no wise injured by us, because it is in their own hands to spare that labour which they are not willing to endure. And if any complain of obscurity, they must consider, that in these matters it cometh no otherwise to pass than in sundry the works both of art and also of nature, where that which hath greatest force in the very things we see is notwithstanding itself oftentimes not seen. The stateliness of houses, the goodliness of trees, when we behold them delighteth the eye;  but that foundation which beareth up the one, that root which ministereth unto the other nourishment and life, is in the bosom of the earth concealed; and if there be at any time occasion to search into it, such labour is then more necessary than pleasant, both to them which undertake it and for the lookers-on. In like manner, the use and benefit of good laws all that live under them may enjoy with delight and comfort, albeit the grounds and first original causes from whence they have sprung be unknown, as to the greatest part of men they are. But when they who withdraw their obedience pretend that the laws which they should obey are corrupt and vicious; for better examination of their quality, it behoveth the very foundation and root, the highest wellspring and fountain of them to be discovered. Which because we are not oftentimes accustomed to do, when we do it the pains we take are more needful a great deal than acceptable, and the matters which we handle seem by reason of newness (till the mind grow better acquainted with them) dark, intricate, and unfamiliar. For as much help whereof as may be in this case, I have endeavoured throughout the body of this whole discourse, that every former part might give strength unto all that follow, and every later bring some light unto all before. So that if the judgments of men do but hold themselves in suspense as touching these first more general meditations, till in order they have perused the rest that ensue; what may seem dark at the first will afterwards be found more plain, even as the later particular decisions will appear I doubt not more strong, when the other have been read before.

[3.]The Laws of the Church, whereby for so many ages together we have been guided in the exercise of Christian religion and the service of the true God, our rites, customs, and orders of ecclesiastical government, are called in question: we are accused as men that will not have Christ Jesus to rule over them, but have wilfully cast his statutes behind their backs, hating to be reformed and made subject unto the sceptre of his discipline. Behold therefore we offer the laws whereby we live unto the general trial and judgment of the whole world; heartily beseeching Almighty God, whom we desire to serve according to his own will, that both we  and others (all kind of partial affection being clean laid aside) may have eyes to see and hearts to embrace the things that in his sight are most acceptable.

And because the point about which we strive is the quality of our laws, our first entrance hereinto cannot better be made, than with consideration of the nature of law in general, and of that law which giveth life unto all the rest, which are commendable, just, and good; namely the law whereby the Eternal himself doth work. Proceeding from hence to the law, first of Nature, then of Scripture, we shall have the easier access unto those things which come after to be debated, concerning the particular cause and question which we have in hand.

\section*{Of that Law which God from before the beginning hath set for himself to do all things by.}

II. All things that are, have some operation not violent or casual. Neither doth any thing ever begin to exercise the same, without some fore-conceived end for which it worketh. And the end which it worketh for is not obtained, unless the work be also fit to obtain it by. For unto every end every operation will not serve. That which doth assign unto each thing the kind, that which doth moderate the force and power, that which doth appoint the form and measure, of working, the same we term a Law. So that no certain end could ever be attained, unless the actions whereby it is attained were regular; that is to say, made suitable, fit and correspondent unto their end, by some canon, rule or law. Which thing doth first take place in the works even of God himself.

[2.]All things therefore do work after a sort, according to law: all other things according to a law, whereof some superior, unto whom they are subject, is author; only the works and operations of God have Him both for their worker, and for the law whereby they are wrought. The being of God is a kind of law to his working: for that perfection which God is, giveth perfection to that he doth. Those natural, necessary, and internal operations of God, the Generation of the Son, the Proceeding of the Spirit, are without the compass of my present intent: which is to touch only such operations as have their beginning and being by a voluntary purpose, wherewith God hath eternally decreed when and how they should be. Which eternal decree is that we term an eternal law.


Dangerous it were for the feeble brain of man to wade far into the doings of the Most High; whom although to know be life, and joy to make mention of his name; yet our soundest knowledge is to know that we know him not as indeed he is, neither can know him: and our safest eloquence concerning him is our silence, when we confess without confession that his glory is inexplicable, his greatness above our capacity and reach,He is above, and we upon earth; therefore it behoveth our words to be wary and few.

Our God is one, or rather very Oneness, and mere unity, having nothing but itself in itself, and not consisting (as all things do besides God) of many things. In which essential Unity of God a Trinity personal nevertheless subsisteth, after a manner far exceeding the possibility of man’s conceit. The works which outwardly are of God, they are in such sort of Him being one, that each Person hath in them somewhat peculiar and proper. For being Three, and they all subsisting in the essence of one Deity; from the Father, by the Son, through the Spirit, all things are. That which the Son doth hear of the Father, and which the Spirit doth receive of the Father and the Son, the same we have at the hands of the Spirit as being the last, and therefore the nearest unto us in order, although in power the same with the second and the first.

[3.]The wise and learned among the very heathens themselves have all acknowledged some First Cause, whereupon originally the being of all things dependeth. Neither have they otherwise spoken of that cause than as an Agent, which knowing what and why it worketh, observeth in working a most exact order or law. Thus much is signified by that which Homer mentioneth, Διὸς δ’ ἐτελείετο βουλή. Thus  much acknowledged by Mercurius Trismegistus, Τὸν πάντα κόσμον ἐποίησεν ὁ δημιουργὸς οὐ χερσὶν ἀλλὰ λόγῳ,Thus much confest by Anaxagoras and Plato, terming the Maker of the world an intellectual Worker. Finally the Stoics, although imagining the first cause of all things to be fire, held nevertheless, that the same fire having art, did ὁδῳ̑ βαδίζειν ἐπὶ γενέσει κόσμου. They all confess therefore in the working of that first cause, that Counsel is used, Reason followed, a Way observed; that is to say, constant Order and Law is kept; whereof itself must needs be author unto itself. Otherwise it should have some worthier and higher to direct it, and so could not itself be the first. Being the first, it can have no other than itself to be the author of that law which it willingly worketh by.

God therefore is a law both to himself, and to all other things besides. To himself he is a law in all those things, whereof our Saviour speaketh, saying, “My Father worketh as yet, so I.” God worketh nothing without cause. All those things which are done by him have some end for which they are done; and the end for which they are done is a reason of his will to do them. His will had not inclined to create woman, but that he saw it could not be well if she were not created. Non est bonum, “It is not good man should be alone; therefore let us make a helper for him.” That and nothing else is done by God, which to leave undone were not so good.

If therefore it be demanded, why God having power and ability infinite, the effects notwithstanding of that power are all so limited as we see they are: the reason hereof is the end which he hath proposed, and the law whereby his wisdom hath stinted the effects of his power in such sort, that it doth not work infinitely, but correspondently unto that end for which it worketh, even “all things χρηστω̑ς,  in most decent and comely sort,” all things in “Measure, Number, and Weight.”

[4.]The general end of God’s external working is the exercise of his most glorious and most abundant virtue. Which abundance doth shew itself in variety, and for that cause this variety is oftentimes in Scripture exprest by the name of riches,“The Lord hath made all things for his own sake.” Not that any thing is made to be beneficial unto him, but all things for him to shew beneficence and grace in them.

The particular drift of every act proceeding externally from God we are not able to discern, and therefore cannot always give the proper and certain reason of his works. Howbeit undoubtedly a proper and certain reason there is of every finite work of God, inasmuch as there is a law imposed upon it; which if there were not, it should be infinite, even as the worker himself is.

[5.]They err therefore who think that of the will of God to do this or that there is no reason besides his will. Many times no reason known to us; but that there is no reason thereof I judge it most unreasonable to imagine, inasmuch as he worketh all things κατὰ τὴν βουλὴν του̑ θελήματος αὐτου̑, not only according to his own will, but “the Counsel of his own will.” And whatsoever is done with counsel or wise resolution hath of necessity some reason why it should be done, albeit that reason be to us in some things so secret, that it forceth the wit of man to stand, as the blessed Apostle himself doth, amazed thereat: “O the depth of the riches both of the wisdom and knowledge of God! how unsearchable are his judgments,” \&c. That law eternal which God himself hath made to himself, and thereby worketh all things whereof he is the cause and author; that law in the admirable frame whereof shineth with most perfect beauty the countenance of that wisdom which hath testified concerning herself, “The Lord possessed me in the beginning of his way, even before his works of old I was set up;” that law, which hath been the pattern to make, and is the card  to guide the world by; that law which hath been of God and with God everlastingly; that law, the author and observer whereof is one only God to be blessed for ever: how should either men or angels be able perfectly to behold? The book of this law we are neither able nor worthy to open and look into. That little thereof which we darkly apprehend we admire, the rest with religious ignorance we humbly and meekly adore.

[6.]Seeing therefore that according to this law He worketh, “of whom, through whom, and for whom, are all things;” although there seem unto us confusion and disorder in the affairs of this present world: “Tamen quoniam bonus mundum rector temperat, recte fieri cuncta ne dubites:” “let no man doubt but that every thing is well done, because the world is ruled by so good a guide,” as transgresseth not His own law, than which nothing can be more absolute, perfect, and just.

The law whereby He worketh is eternal, and therefore can have no show or colour of mutability: for which cause, a part of that law being opened in the promises which God hath made (because his promises are nothing else but declarations what God will do for the good of men) touching those promises the Apostle hath witnessed, that God may as possibly “deny himself” and not be God, as fail to perform them. And concerning the counsel of God, he termeth it likewise a thing “unchangeable;” the counsel of God, and that law of God whereof now we speak, being one.

Nor is the freedom of the will of God any whit abated, let or hindered, by means of this; because the imposition of this law upon himself is his own free and voluntary act.

This law therefore we may name eternal, being “that order which God before all ages hath set down with himself, for himself to do all things by.”

\section*{The Law which natural agents observe, and their necessary manner of keeping it.}

III. I am not ignorant that by “law eternal” the learned for the most part do understand the order, not which God hath eternally purposed himself in all his works to observe,  but rather that which with himself he hath set down as expedient to be kept by all his creatures, according to the several condition wherewith he hath endued them. They who thus are accustomed to speak apply the name of Law unto that only rule of working which superior authority imposeth; whereas we somewhat more enlarging the sense thereof term any kind of rule or canon, whereby actions are framed, a law. Now that law which, as it is laid up in the bosom of God, they call Eternal, receiveth according unto the different kinds of things which are subject unto it different and sundry kinds of names. That part of it which ordereth natural agents we call usually Nature’s law; that which Angels do clearly behold and without any swerving observe is a law Celestial and heavenly; the law of Reason, that which bindeth creatures reasonable in this world, and with which by reason they may most plainly perceive themselves bound; that which bindeth them, and is not known but by special revelation from God, Divine law; Human law, that which out of the law either of reason or of God men probably gathering to be expedient, they make it a law. All things therefore, which are as they ought to be, are conformed unto this second law eternal; and even those things which to this eternal law are not conformable are notwithstanding in some sort ordered by the first eternal law. For what good or evil is there under the sun, what action correspondent or repugnant unto the law which God hath imposed upon his creatures, but in or upon it God doth work according to the law which himself hath eternally purposed to keep; that is to say, the first law eternal? So that a twofold law eternal being thus made, it is not hard to conceive how they both take place in all things.

[2.] Wherefore to come to the law of nature: albeit thereby we sometimes mean that manner of working which God hath set for each created thing to keep; yet forasmuch as those things are termed most properly natural agents, which keep the law of their kind unwittingly, as the heavens and elements of the world, which can do no otherwise than they do; and forasmuch as we give unto intellectual natures the name of Voluntary agents, that so we may distinguish them from the other; expedient it will be, that we sever the law of nature observed by the one from that which the other is tied unto. Touching the former, their strict keeping of one tenure, statute, and law, is spoken of by all, but hath in it more than men have as yet attained to know, or perhaps ever shall attain, seeing the travail of wading herein is given of God to the sons of men, that perceiving how much the least thing in the world hath in it more than the wisest are able to reach unto, they may by this means learn humility. Moses, in describing the work of creation, attributeth speech unto God: “God said, Let there be light: let there be a firmament: let the waters under the heaven be gathered together into one place: let the earth bring forth: let there be lights in the firmament of heaven.” Was this only the intent of Moses, to signify the infinite greatness of God’s power by the easiness of his accomplishing such effects, without travail, pain, or labour? Surely it seemeth that Moses had herein besides this a further purpose, namely, first to teach that God did not work as a  necessary but a voluntary agent, intending beforehand and decreeing with himself that which did outwardly proceed from him: secondly, to shew that God did then institute a law natural to be observed by creatures, and therefore according to the manner of laws, the institution thereof is described, as being established by solemn injunction. His commanding those things to be which are, and to be in such sort as they are, to keep that tenure and course which they do, importeth the establishment of nature’s law. This world’s first creation, and the preservation since of things created, what is it but only so far forth a manifestation by execution, what the eternal law of God is concerning things natural? And as it cometh to pass in a kingdom rightly ordered, that after a law is once published, it presently takes effect far and wide, all states framing themselves thereunto; even so let us think it fareth in the natural course of the world: since the time that God did first proclaim the edicts of his law upon it, heaven and earth have hearkened unto his voice, and their labour hath been to do his will: He “made a law for the rain;” He gave his “decree unto the sea, that the waters should not pass his commandment.” Now if nature should intermit her course, and leave altogether though it were but for a while the observation of her own laws; if those principal and mother elements of the world, whereof all things in this lower world are made, should lose the qualities which now they have; if the frame of that heavenly arch erected over our heads should loosen and dissolve itself; if celestial spheres should forget their wonted motions, and by irregular volubility turn themselves any way as it might happen; if the prince of the lights of heaven, which now as a giant doth run his unwearied course, should as it were through a languishing faintness begin to stand and to rest himself; if the moon should wander from her beaten way, the times and seasons of the year blend themselves by disordered and confused mixture, the winds breathe out their last gasp, the clouds yield no rain, the earth be defeated of heavenly influence, the fruits of the earth pine away as children at the withered breasts of their mother no longer  able to yield them relief: what would become of man himself, whom these things now do all serve? See we not plainly that obedience of creatures unto the law of nature is the stay of the whole world?

[3.]Notwithstanding with nature it cometh sometimes to pass as with art. Let Phidias have rude and obstinate stuff to carve, though his art do that it should, his work will lack that beauty which otherwise in fitter matter it might have had. He that striketh an instrument with skill may cause notwithstanding a very unpleasant sound, if the string whereon he striketh chance to be uncapable of harmony. In the matter whereof things natural consist, that of Theophrastus taketh place, Πολὺ τὸ οὐχ ὑπακου̑ον οὐδὲ δεχόμενον τὸ εὐ̑. “Much of it is oftentimes such as will by no means yield to receive that impression which were best and most perfect.” Which defect in the matter of things natural, they who gave themselves unto the contemplation of nature amongst the heathen observed often: but the true original cause thereof, divine malediction, laid for the sin of man upon these creatures which God had made for the use of man, this being an article of that saving truth which God hath revealed unto his Church, was above the reach of their merely natural  capacity and understanding. But howsoever these swervings are now and then incident into the course of nature, nevertheless so constantly the laws of nature are by natural agents observed, that no man denieth but those things which nature worketh are wrought, either always or for the most part, after one and the same manner.

[4.]If here it be demanded what that is which keepeth nature in obedience to her own law, we must have recourse to that higher law whereof we have already spoken, and because all other laws do thereon depend, from thence we must borrow so much as shall need for brief resolution in this point. Although we are not of opinion therefore, as some are, that nature in working hath before her certain exemplary draughts or patterns, which subsisting in the bosom of the Highest, and being thence discovered, she fixeth her eye upon them, as travellers by sea upon the pole-star of the world, and that according thereunto she guideth her hand to work by imitation: although we rather embrace the oracle of Hippocrates, that “each thing both in small and in great fulfilleth the task which destiny hath set down;” and concerning the manner of executing and fulfilling the same, “what they do they know not, yet is it in show and appearance as though they did know what they do; and the truth is they do not discern the things which they look on:” nevertheless, forasmuch as the works of nature are no less exact, than if she did both behold and study how to express some absolute shape or mirror always present before her; yea, such her dexterity and skill appeareth, that no intellectual creature in the world were able by capacity to do that which nature doth without capacity and knowledge; it cannot be but nature hath some director of infinite knowledge to guide her in all her ways. Who the guide of nature, but only the God of nature? “In him we live, move, and are.” Those things which nature is said to do, are by divine art performed,  using nature as an instrument; nor is there any such art or knowledge divine in nature herself working, but in the Guide of nature’s work.

Whereas therefore things natural which are not in the number of voluntary agents, (for of such only we now speak, and of no other,) do so necessarily observe their certain laws, that as long as they keep those forms which give them their being, they cannot possibly be apt or inclinable to do otherwise than they do; seeing the kinds of their operations are both constantly and exactly framed according to the several ends for which they serve, they themselves in the meanwhile, though doing that which is fit, yet knowing neither what they do, nor why: it followeth that all which they do in this sort proceedeth originally from some such agent, as knoweth, appointeth, holdeth up, and even actually frameth the same.

The manner of this divine efficiency, being far above us, we are no more able to conceive by our reason than creatures unreasonable by their sense are able to apprehend after what manner we dispose and order the course of our affairs. Only thus much is discerned, that the natural generation and process of all things receiveth order of proceeding from the settled stability of divine understanding. This appointeth unto them their kinds of working; the disposition whereof in the purity of God’s own knowledge and will is rightly termed by the name of Providence. The same being referred unto the things themselves here disposed by it, was wont by the ancient to be called natural Destiny. That law, the performance whereof we behold in things natural, is as it were an authentical or an original draught written in the bosom of God himself; whose Spirit being to execute the same useth every particular nature, every mere natural agent, only as an instrument created at the beginning, and ever since the beginning used, to work his own will and pleasure withal. Nature therefore is nothing else but God’s instrument: in the course whereof Dionysius perceiving  some sudden disturbance is said to have cried out, “Aut Deus naturæ patitur, aut mundi machina dissolvetur:” “either God doth suffer impediment, and is by a greater than himself hindered; or if that be impossible, then hath he determined to make a present dissolution of the world; the execution of that law beginning now to stand still, without which the world cannot stand.”

This workman, whose servitor nature is, being in truth but only one, the
heathens imagining to be more, gave him in the sky the name of Jupiter, in the air the name of Juno, in the water the name of Neptune, in the earth the name of Vesta and sometimes of Ceres, the name of Apollo in the sun, in the moon the name of Diana, the name of Æolus and divers other in the winds; and to conclude, even so many guides of nature they dreamed of, as they saw there were kinds of things natural in the world. These they honoured, as having power to work or cease accordingly as men deserved of them. But unto us there is one only Guide of all agents natural, and he both the Creator and the Worker of all in all, alone to be blessed, adored and honoured by all for ever.

[5.]That which hitherto hath been spoken concerneth natural agents considered in themselves. But we must further remember also, (which thing to touch in a word shall suffice,) that as in this respect they have their law, which law directeth them in the means whereby they tend to their own perfection: so likewise another law there is, which toucheth them as they are sociable parts united into one body; a law which bindeth them each to serve unto other’s good, and all to prefer the good of the whole before whatsoever their own particular; as we plainly see they do, when things natural in that regard forget their ordinary natural wont; that which is heavy mounting sometime upwards of  it own accord, and forsaking the centre of the earth which to itself is most natural, even as if it did hear itself commanded to let go the good it privately wisheth, and to relieve the present distress of nature in common.

\section*{The Law which the Angels of God obey.}

IV. But now that we may lift up our eyes (as it were) from the footstool to the throne of God, and leaving these natural, consider a little the state of heavenly and divine creatures: touching Angels, which are spirits immaterial and intellectual, the glorious inhabitants of those sacred palaces, where nothing but light and blessed immortality, no shadow of matter for tears, discontentments, griefs, and uncomfortable passions to work upon, but all joy, tranquillity, and peace, even for ever and ever doth dwell: as in number and order they are huge, mighty, and royal armies, so likewise in perfection of obedience unto that law, which the Highest, whom they adore, love, and imitate, hath imposed upon them, such observants they are thereof, that our Saviour himself being to set down the perfect idea of that which we are to pray and wish for on earth, did not teach to pray or wish for more than only that here it might be with us, as with them it is in heaven. God which moveth mere natural agents as an efficient only, doth otherwise move intellectual creatures, and especially his holy angels: for beholding the face of God, in admiration of so great excellency they all adore him; and being rapt with the love of his beauty, they cleave inseparably for ever unto him. Desire to resemble him in goodness maketh them unweariable and even unsatiable in their longing to do by all means all manner good unto all the creatures of God, but especially unto the children of  men: in the countenance of whose nature, looking downward, they behold themselves beneath themselves; even as upward, in God, beneath whom themselves are, they see that character which is no where but in themselves and us resembled. Thus far even the paynims have approached; thus far they have seen into the doings of the angels of God; Orpheus confessing, that “the fiery throne of God is attended on by those most industrious angels, careful how all things are performed amongst men;” and the Mirror of human wisdom plainly teaching, that God moveth angels, even as that thing doth stir man’s heart, which is thereunto presented amiable. Angelical actions may therefore be reduced unto these three general kinds: first, most delectable love arising from the visible apprehension of the purity, glory, and beauty of God, invisible saving only unto spirits that are pure: secondly, adoration grounded upon the evidence of the greatness of God, on whom they see how all things depend; thirdly, imitation, bred by the presence of his exemplary goodness, who ceaseth not before them daily to fill heaven and earth with the rich treasures of most free and undeserved grace.

[2.]Of angels, we are not to consider only what they are and do in regard of their own being, but that also which concerneth them as they are linked into a kind of corporation amongst themselves, and of society or fellowship with men. Consider angels each of them severally in himself, and their law is that which the prophet David mentioneth, “All ye his angels praise him.” Consider the angels of God associated, and their law is that which disposeth them as an army, one in order and degree above another. Consider finally the angels as having with us that communion which the apostle to the Hebrews noteth, and in regard whereof  angels have not disdained to profess themselves our “fellow-servants;” from hence there springeth up a third law, which bindeth them to works of ministerial employment. Every of which their several functions are by them performed with joy.

[3.]A part of the angels of God notwithstanding (we know) have fallen, and that their fall hath been through the voluntary breach of that law, which did require at their hands continuance in the exercise of their high and admirable virtue. Impossible it was that ever their will should change or incline to remit any part of their duty, without some object having force to avert their conceit from God, and to draw it another way; and that before they attained that high perfection of bliss, wherein now the elect angels are without possibility of falling. Of any thing more than of God they could not by any means like, as long as whatsoever they knew besides God they apprehended it not in itself without dependency upon God; because so long God must needs seem infinitely better than any thing which they so could apprehend. Things beneath them could not in such sort be presented unto their eyes, but that therein they must needs see always how those things did depend on God. It seemeth therefore that there was no other way for angels to sin, but by reflex of their understanding upon themselves; when being held with admiration of their own sublimity and honour, the memory of their subordination unto God and their dependency on him was drowned in this conceit; whereupon their adoration, love, and imitation of God could not choose but be also interrupted. The fall of angels therefore was pride. Since their fall, their practices have been the clean contrary unto those before mentioned. For being dispersed,  some in the air, some on the earth, some in the water, some among the minerals, dens, and caves, that are under the earth; they have by all means laboured to effect an universal rebellion against the laws, and as far as in them lieth utter destruction of the works of God. These wicked spirits the heathens honoured instead of gods, both generally under the name of Dii inferi, “gods infernal;” and particularly, some in oracles, some in idols, some as household gods, some as nymphs: in a word, no foul and wicked spirit which was not one way or other honoured of men as God, till such time as light appeared in the world and dissolved the works of the devil. Thus much therefore may suffice for angels, the next unto whom in degree are men.

\section*{The Law whereby man is in his actions directed to the imitation of God.}

V. God alone excepted, who actually and everlastingly is whatsoever he may be, and which cannot hereafter be that which now he is not; all other things besides are somewhat in possibility, which as yet they are not in act. And for this cause there is in all things an appetite or desire, whereby they incline to something which they may be; and when they are it, they shall be perfecter than now they are. All which perfections are contained under the general name of Goodness. And because there is not in the world any thing whereby another may not some way be made the perfecter, therefore all things that are, are good.

[2.]Again, sith there can be no goodness desired which proceedeth not from God himself, as from the supreme cause of all things; and every effect doth after a sort contain, at leastwise resemble, the cause from which it proceedeth: all things in the world are said in some sort to seek the highest, and to covet more or less the participation of God himself. Yet this doth no where so much appear as it doth in man, because there are so many kinds of perfections which man seeketh. The first degree of goodness is that general perfection which all things do seek, in desiring the continuance of their being. All things therefore coveting as much as may be to be like unto God in being ever, that which cannot hereunto  attain personally doth seek to continue itself another way, that is by offspring and propagation. The next degree of goodness is that which each thing coveteth by affecting resemblance with God in the constancy and excellency of those operations which belong unto their kind. The immutability of God they strive unto, by working either always or for the most part after one and the same manner; his absolute exactness they imitate, by tending unto that which is most exquisite in every particular. Hence have risen a number of axioms in philosophy, showing how “the works of nature do always aim at that which cannot be bettered.”

[3.]These two kinds of goodness rehearsed are so nearly united to the things themselves which desire them, that we scarcely perceive the appetite to stir in reaching forth her hand towards them. But the desire of those perfections which grow externally is more apparent; especially of such as are not expressly desired unless they be first known, or such as are not for any other cause than for knowledge itself desired. Concerning perfections in this kind; that by proceeding in the knowledge of truth, and by growing in the exercise of virtue, man amongst the creatures of this inferior world aspireth to the greatest conformity with God; this is not only known unto us, whom he himself hath so instructed, but even they do acknowledge, who amongst men are not judged the nearest unto him. With Plato what one thing more usual, than to excite men unto the love of wisdom, by shewing how much wise men are thereby exalted above men; how knowledge doth raise them up into heaven; how it maketh them, though not gods, yet as gods, high, admirable, and divine? And Mercurius Trismegistus speaking of the virtues of a righteous soul, “Such spirits” (saith he) “are never cloyed with praising and speaking well of all men, with doing good unto every one by word and deed, because they study to frame themselves according to the pattern of the Father of spirits.”

\section*{Men’s first beginning to grow to the knowledge of that Law which they are to observe.}

VI. In the matter of knowledge, there is between the angels of God and the children of men this difference: angels already have full and complete knowledge in the highest degree that can be imparted unto them; men, if we view them in their spring, are at the first without understanding or knowledge at all. Nevertheless from this utter vacuity they grow by degrees, till they come at length to be even as the angels themselves are. That which agreeth to the one now, the other shall attain unto in the end; they are not so far disjoined and severed, but that they come at length to meet. The soul of man being therefore at the first as a book, wherein nothing is and yet all things may be imprinted; we are to search by what steps and degrees it riseth unto perfection of knowledge.

[2.]Unto that which hath been already set down concerning natural agents this we must add, that albeit therein we have comprised as well creatures living as void of life, if they be in degree of nature beneath men; nevertheless a difference we must observe between those natural agents that work altogether unwittingly, and those which have though weak yet some understanding what they do, as fishes, fowls, and beasts have. Beasts are in sensible capacity as ripe even as men themselves, perhaps more ripe. For as stones, though in dignity of nature inferior unto plants, yet exceed them in firmness of strength or durability of being; and plants, though beneath the excellency of creatures endued with sense, yet exceed them in the faculty of vegetation and of fertility: so beasts, though otherwise behind men, may notwithstanding in actions of sense and fancy go beyond them; because the endeavours of nature, when it hath a higher perfection to seek, are in lower the more remiss, not esteeming thereof so much as those things do, which have no better proposed unto them.

[3.]The soul of man therefore being capable of a more divine perfection, hath (besides the faculties of growing unto sensible knowledge which is common unto us with beasts) a further ability, whereof in them there is no show at all, the ability of reaching higher than unto sensible things. Till  we grow to some ripeness of years, the soul of man doth only store itself with conceits of things of inferior and more open quality, which afterwards do serve as instruments unto that which is greater; in the meanwhile above the reach of meaner creatures it ascendeth not. When once it comprehendeth any thing above this, as the differences of time, affirmations, negations, and contradictions in speech, we then count it to have some use of natural reason. Whereunto if afterwards there might be added the right helps of true art and learning (which helps, I must plainly confess, this age of the world, carrying the name of a learned age, doth neither much know nor greatly regard), there would undoubtedly be almost as great difference in maturity of judgment between men therewith inured, and that which now men are, as between men that are now and innocents. Which speech if any condemn, as being over hyperbolical, let them consider but this one thing. No art is at the first finding out so perfect as industry may after make it. Yet the very first man that to any purpose knew the way we speak of and followed it, hath alone thereby performed more very near in all parts of natural knowledge, than sithence in any one part thereof the whole world besides hath done.

[4.]In the poverty of that other new devised aid two  things there are notwithstanding singular. Of marvellous quick despatch it is, and doth shew them that have it as much almost in three days, as if it dwell threescore years with them. Again, because the curiosity of man’s wit doth many times with peril wade farther in the search of things than were convenient; the same is thereby restrained unto such generalities as every where offering themselves are apparent unto men of the weakest conceit that need be. So as following the rules and precepts thereof, we may define it to be, an Art which teacheth the way of speedy discourse, and restraineth the mind of man that it may not wax over-wise.

[5.]Education and instruction are the means, the one by use, the other by precept, to make our natural faculty of reason both the better and the sooner able to judge rightly between truth and error, good and evil. But at what time a man may be said to have attained so far forth the use of reason, as sufficeth to make him capable of those Laws, whereby he is then bound to guide his actions; this is a great deal more easy for common sense to discern, than for any man by skill and learning to determine; even as it is not in philosophers, who best know the nature both of fire and of gold, to teach what degree of the one will serve to purify the other, so well as the artisan, who doth this by fire, discerneth by sense when the fire hath that degree of heat which sufficeth for his purpose.

\section*{Of Man’s Will, which is the first thing that Laws of action are made to guide.}

VII. By reason man attaineth unto the knowledge of things that are and are not sensible. It resteth therefore that we search how man attaineth unto the knowledge of such things unsensible as are to be known that they may be done. Seeing then that nothing can move unless there be  some end, the desire whereof provoketh unto motion; how should that divine power of the soul, that “spirit of our mind,” as the apostle termeth it, ever stir itself unto action, unless it have also the like spur? The end for which we are moved to work, is sometimes the goodness which we conceive of the very working itself, without any further respect at all; and the cause that procureth action is the mere desire of action, no other good besides being thereby intended. Of certain turbulent wits it is said, “Illis quieta movere magna merces videbatur:” they thought the very disturbance of things established an hire sufficient to set them on work. Sometimes that which we do is referred to a further end, without the desire whereof we would leave the same undone; as in their actions that gave alms to purchase thereby the praise of men.

[2.]Man in perfection of nature being made according to the likeness of his Maker resembleth him also in the manner of working: so that whatsoever we work as men, the same we do wittingly work and freely; neither are we according to the manner of natural agents any way so tied, but that it is in our power to leave the things we do undone. The good which either is gotten by doing, or which consisteth in the very doing itself, causeth not action, unless apprehending it as good we so like and desire it: that we do unto any such end, the same we choose and prefer before the leaving of it undone. Choice there is not, unless the thing which we take be so in our power that we might have refused and left it. If fire consume the stubble, it chooseth not so to do, because the nature thereof is such that it can do no other. To choose is to will one thing before another. And to will is to bend our souls to the having or doing of that which they see to be good. Goodness is seen with the eye of the understanding. And the light of that eye, is reason. So that two principal fountains there are of human action, Knowledge and Will; which Will, in things tending towards any end, is termed Choice. Concerning Knowledge, “Behold, (saith Moses,) I have set before you this day good and evil, life and death.” Concerning Will, he addeth  immediately, “Choose life;” that is to say, the things that tend unto life, them choose.

[3.]But of one thing we must have special care, as being a matter of no small moment; and that is, how the Will, properly and strictly taken, as it is of things which are referred unto the end that man desireth, differeth greatly from that inferior natural desire which we call Appetite. The object of Appetite is whatsoever sensible good may be wished for; the object of Will is that good which Reason doth lead us to seek. Affections, as joy, and grief, and fear, and anger, with such like, being as it were the sundry fashions and forms of Appetite, can neither rise at the conceit of a thing indifferent, nor yet choose but rise at the sight of some things. Wherefore it is not altogether in our power, whether we will be stirred with affections or no: whereas actions which issue from the disposition of the Will are in the power thereof to be performed or stayed. Finally, Appetite is the Will’s solicitor, and the Will is Appetite’s controller; what we covet according to the one by the other we often reject; neither is any other desire termed properly Will, but that where Reason and Understanding, or the show of Reason, prescribeth the thing desired.

It may be therefore a question, whether those operations of men are to be counted voluntary, wherein that good which is sensible provoketh Appetite, and Appetite causeth action, Reason being never called to counsel; as when we eat or drink, and betake ourselves unto rest, and such like. The truth is, that such actions in men having attained to the use of Reason are voluntary. For as the authority of higher powers hath force even in those things, which are done without their privity, and are of so mean reckoning that to acquaint them therewith it needeth not; in like sort, voluntarily we are said to do that also, which the Will if it listed might hinder from being done, although about the doing thereof we do not expressly use our reason or understanding, and so immediately apply our wills thereunto. In cases therefore of such facility, the Will doth yield her assent as it were with a kind of silence, by not dissenting; in which respect her force is not so apparent as in express mandates or prohibitions, especially upon advice and consultation going before.


[4.] Where understanding therefore needeth, in those things Reason is the director of man’s Will by discovering in action what is good. For the Laws of well-doing are the dictates of right Reason. Children, which are not as yet come unto those years whereat they may have; again, innocents, which are excluded by natural defect from ever having; thirdly, madmen, which for the present cannot possibly have the use of right Reason to guide themselves, have for their guide the Reason that guideth other men, which are tutors over them to seek and to procure their good for them. In the rest there is that light of Reason, whereby good may be known from evil, and which discovering the same rightly is termed right.

[5.]The Will notwithstanding doth not incline to have or do that which Reason teacheth to be good, unless the same do also teach it to be possible. For albeit the Appetite, being more general, may wish any thing which seemeth good, be it never so impossible; yet for such things the reasonable Will of man doth never seek. Let Reason teach impossibility in any thing, and the Will of man doth let it go; a thing impossible it doth not affect, the impossibility thereof being manifest.

[6.]There is in the Will of man naturally that freedom, whereby it is apt to take or refuse any particular object whatsoever being presented unto it. Whereupon it followeth,  that there is no particular object so good, but it may have the shew of some difficulty or unpleasant quality annexed to it, in respect whereof the Will may shrink and decline it; contrariwise (for so things are blended) there is no particular evil which hath not some appearance of goodness whereby to insinuate itself. For evil as evil cannot be desired: if that be desired which is evil, the cause is the goodness which is or seemeth to be joined with it. Goodness doth not move by being, but by being apparent; and therefore many things are neglected which are most precious, only because the value of them lieth hid. Sensible Goodness is most apparent, near, and present; which causeth the Appetite to be therewith strongly provoked. Now pursuit and refusal in the Will do follow, the one the affirmation the other the negation of goodness, which the understanding apprehendeth, grounding itself upon sense, unless some higher Reason do chance to teach the contrary. And if Reason have taught it rightly to be good, yet not so apparently that the mind receiveth it with utter impossibility of being otherwise, still there is place left for the Will to take or leave. Whereas therefore amongst so many things as are to be done, there are so few, the goodness whereof Reason in such sort doth or easily can discover, we are not to marvel at the choice of evil even then when the contrary is probably known. Hereby it cometh to pass that custom inuring the mind by long practice, and so leaving there a sensible impression, prevaileth more than reasonable  persuasion what way soever. Reason therefore may rightly discern the thing which is good, and yet the Will of man not incline itself thereunto, as oft as the prejudice of sensible experience doth oversway.

[7.]Nor let any man think that this doth make any thing for the just excuse of iniquity. For there was never sin committed, wherein a less good was not preferred before a greater, and that wilfully; which cannot be done without the singular disgrace of Nature, and the utter disturbance of that divine order, whereby the preeminence of chiefest acceptation is by the best things worthily challenged. There is not that good which concerneth us, but it hath evidence enough for itself, if Reason were diligent to search it out. Through neglect thereof, abused we are with the show of that which is not; sometimes the subtilty of Satan inveigling us as it did Eve, sometimes the hastiness of our Wills preventing the more considerate advice of sound Reason, as in the Apostles, when they no sooner saw what they liked not, but they forthwith were desirous of fire from heaven; sometimes the very custom of evil making the heart obdurate against whatsoever instructions to the contrary, as in them over whom our Saviour spake weeping, “O Jerusalem, how often, and thou wouldest not!” Still therefore that wherewith we stand blameable, and can no way excuse it, is, In doing evil, we prefer a less good before a greater, the greatness whereof is by reason investigable and may be known. The search of knowledge is a thing painful; and the painfulness of knowledge is that which maketh the Will so hardly inclinable thereunto. The root hereof, divine malediction; whereby the instruments being weakened wherewithal the soul (especially in reasoning) doth work, it preferreth rest in ignorance before wearisome labour to know. For a spur of diligence therefore we have a natural thirst after knowledge ingrafted in us. But by reason of that original weakness in the instruments, without which the understanding part is not  able in this world by discourse to work, the very conceit of painfulness is as a bridle to stay us. For which cause the Apostle, who knew right well that the weariness of the flesh is an heavy clog to the Will, striketh mightily upon this key, “Awake thou that sleepest; Cast off all which presseth down; Watch; Labour; Strive to go forward, and to grow in knowledge.”

\section*{Of the natural finding out of Laws by the light of Reason, to guide the Will unto that which is good.}

VIII. Wherefore to return to our former intent of discovering the natural way, whereby rules have been found out concerning that goodness wherewith the Will of man ought to be moved in human actions; as every thing naturally and necessarily doth desire the utmost good and greatest perfection whereof Nature hath made it capable, even so man. Our felicity therefore being the object and accomplishment of our desire, we cannot choose but wish and covet it. All particular things which are subject unto action, the Will doth so far forth incline unto, as Reason judgeth them the better for us, and consequently the more available to our bliss. If Reason err, we fall into evil, and are so far forth deprived of the general perfection we seek. Seeing therefore that for the framing of men’s actions the knowledge of good from evil is necessary, it only resteth that we search how this may be had. Neither must we suppose that there needeth one rule to know the good and another the evil by. For he that knoweth what is straight doth even thereby discern what is crooked, because the absence of straightness in bodies capable thereof is crookedness. Goodness in actions is like unto straightness; wherefore that which is done well we term right. For as the straight way is most acceptable to him that travelleth, because by it he cometh soonest to his journey’s end; so in action, that which doth lie the evenest between us and the end we desire must needs be the fittest for our use. Besides which fitness for use, there is also in rectitude, beauty; as contrariwise in obliquity, deformity. And that which is good in the actions of men, doth not only delight as profitable, but as amiable also. In which consideration the Grecians most divinely have given to the active perfection of  men a name expressing both beauty and goodness, because goodness in ordinary speech is for the most part applied only to that which is beneficial. But we in the name of goodness do here imply both.

[2.]And of discerning goodness there are but these two ways; the one the knowledge of the causes whereby it is made such; the other the observation of those signs and tokens, which being annexed always unto goodness, argue that where they are found, there also goodness is, although we know not the cause by force whereof it is there. The former of these is the most sure and infallible way, but so hard that all shun it, and had rather walk as men do in the dark by haphazard, than tread so long and intricate mazes for knowledge’ sake. As therefore physicians are many times forced to leave such methods of curing as themselves know to be the fittest, and being overruled by their patients’ impatiency are fain to try the best they can, in taking that way of cure which the cured will yield unto; in like sort, considering how the case doth stand with this present age full of tongue and weak of brain, behold we yield to the stream thereof; into the causes of goodness we will not make any curious or deep inquiry; to touch them now and then it shall be sufficient, when they are so near at hand that easily they may be conceived without any far-removed discourse: that way we are contented to prove, which being the worse in itself, is notwithstanding now by reason of common imbecility the fitter and likelier to be brooked.

[3.]Signs and tokens to know good by are of sundry kinds; some more certain and some less. The most certain token of evident goodness is, if the general persuasion of all men do so account it. And therefore a common received error is never utterly overthrown, till such time as we go from signs unto causes, and shew some manifest root or fountain thereof common unto all, whereby it may clearly appear how it hath come to pass that so many have been overseen. In which case surmises and slight probabilities will not serve, because the universal consent of men is the perfectest and strongest in this kind, which comprehendeth  only the signs and tokens of goodness. Things casual do vary, and that which a man doth but chance to think well of cannot still have the like hap. Wherefore although we know not the cause, yet thus much we may know; that some necessary cause there is, whensoever the judgments of all men generally or for the most part run one and the same way, especially in matters of natural discourse. For of things necessarily and naturally done there is no more affirmed but this, “They keep either always or for the most part one tenure.” The general and perpetual voice of men is as the sentence of God himself. For that which all men have at all times learned, Nature herself must needs have taught; and God being the author of Nature, her voice is but his instrument. By her from Him we receive whatsoever in such sort we learn. Infinite duties there are, the goodness whereof is by this rule sufficiently manifested, although we had no other warrant besides to approve them. The Apostle St. Paul having speech concerning the heathen saith of them, “They are a law unto themselves.” His meaning is, that by force of the light of Reason, wherewith God illuminateth every one which cometh into the world, men being enabled to know truth from falsehood,  and good from evil, do thereby learn in many things what the will of God is; which will himself not revealing by any extraordinary means unto them, but they by natural discourse attaining the knowledge thereof, seem the makers of those Laws which indeed are his, and they but only the finders of them out.

[4.]A law therefore generally taken, is a directive rule unto goodness of operation. The rule of divine operations outward, is the definitive appointment of God’s own wisdom set down within himself. The rule of natural agents that work by simple necessity, is the determination of the wisdom of God, known to God himself the principal director of them, but not unto them that are directed to execute the same. The rule of natural agents which work after a sort of their own accord, as the beasts do, is the judgment of common sense or fancy concerning the sensible goodness of those objects wherewith they are moved. The rule of ghostly or immaterial natures, as spirits and angels, is their intuitive intellectual judgment concerning the amiable beauty and high goodness of that object, which with unspeakable joy and delight doth set them on work. The rule of voluntary agents on earth is the sentence that Reason giveth concerning the goodness of those things which they are to do. And the sentences which Reason giveth are some more some less general, before it come to define in particular actions what is good.

[5.]The main principles of Reason are in themselves apparent. For to make nothing evident of itself unto man’s understanding were to take away all possibility of knowing any thing. And herein that of Theophrastus is true, “They that seek a reason of all things do utterly overthrow Reason.” In every kind of knowledge some such grounds there are, as that being proposed the mind doth presently embrace them as free from all possibility of error, clear and manifest without proof. In which kind axioms or principles more general are such as this, “that the greater good is to be chosen before the less.” If therefore it should be demanded what reason there is, why the Will of Man, which doth necessarily shun harm and covet whatsoever  is pleasant and sweet, should be commanded to count the pleasures of sin gall, and notwithstanding the bitter accidents wherewith virtuous actions are compassed, yet still to rejoice and delight in them: surely this could never stand with Reason, but that wisdom thus prescribing groundeth her laws upon an infallible rule of comparison; which is, “That small difficulties, when exceeding great good is sure to ensue, and on the other side momentany benefits, when the hurt which they draw after them is unspeakable, are not at all to be respected.” This rule is the ground whereupon the wisdom of the Apostle buildeth a law, enjoining patience unto himself; “The present lightness of our affliction worketh unto us even with abundance upon abundance an eternal weight of glory; while we look not on the things which are seen, but on the things which are not seen: for the things which are seen are temporal, but the things which are not seen are eternal:” therefore Christianity to be embraced, whatsoever calamities in those times it was accompanied withal. Upon the same ground our Saviour proveth the law most reasonable, that doth forbid those crimes which men for gain’s sake fall into. “For a man to win the world if it be with the loss of his soul, what benefit or good is it?” Axioms less general, yet so manifest that they need no further proof, are such as these, “God to be worshipped;” “parents to be honoured;” “others to be used by us as we ourselves would by them.” Such things, as soon as they are alleged, all men acknowledge to be good; they require no proof or further discourse to be assured of their goodness.

Notwithstanding whatsoever such principle there is, it was at the first found out by discourse, and drawn from out of the very bowels of heaven and earth. For we are to note, that things in the world are to us discernible, not only so far forth as serveth for our vital preservation, but further also in a twofold higher respect. For first if all other uses were utterly taken away, yet the mind of man being by nature speculative and delighted with contemplation in itself, they were to be known even for mere knowledge and understanding’s sake. Yea further besides this, the knowledge of every the least  thing in the whole world hath in it a second peculiar benefit unto us, inasmuch as it serveth to minister rules, canons, and laws, for men to direct those actions by, which we properly term human. This did the very heathens themselves obscurely insinuate, by making Themis, which we call Jus, or Right, to be the daughter of heaven and earth.

[6.]We know things either as they are in themselves, or as they are in mutual relation one to another. The knowledge of that which man is in reference unto himself, and other things in relation unto man, I may justly term the mother of all those principles, which are as it were edicts, statutes, and decrees, in that Law of Nature, whereby human actions are framed. First therefore having observed that the best things, where they are not hindered, do still produce the best operations, (for which cause, where many things are to concur unto one effect, the best is in all congruity of reason to guide the residue, that it prevailing most, the work principally done by it may have greatest perfection:) when hereupon we come to observe in ourselves, of what excellency our souls are in comparison of our bodies, and the diviner part in relation unto the baser of our souls; seeing that all these concur in producing human actions, it cannot be well unless the chiefest do command and direct the rest. The soul then ought to conduct the body, and the spirit of our minds the soul. This is therefore the first Law, whereby the highest power of the mind requireth general obedience at the hands of all the rest concurring with it unto action.

[7.]Touching the several grand mandates, which being imposed by the understanding faculty of the mind must be obeyed by the Will of Man, they are by the same method found out, whether they import our duty towards God or towards man.

Touching the one, I may not here stand to open, by what degrees of discourse the minds even of mere natural men have attained to know, not only that there is a God, but also what power, force, wisdom, and other properties that God hath, and how all things depend on him. This being therefore presupposed, from that known relation which God hath  unto us as unto children, and unto all good things as unto effects whereof himself is the principal cause, these axioms and laws natural concerning our duty have arisen, “that in all things we go about his aid is by prayer to be craved:” “that he cannot have sufficient honour done unto him, but the utmost of that we can do to honour him we must;” which is in effect the same that we read, “Thou shalt love the Lord thy God with all thy heart, with all thy soul, and with all thy mind:” which Law our Saviour doth term “The first and the great commandment.”

Touching the next, which as our Saviour addeth is “like unto this,” (he meaneth in amplitude and largeness, inasmuch as it is the root out of which all Laws of duty to menward have grown, as out of the former all offices of religion towards God,) the like natural inducement hath brought men to know that it is their duty no less to love others than themselves. For seeing those things which are equal must needs all have one measure; if I cannot but wish to receive all good, even as much at every man’s hand as any man can wish unto his own soul, how should I look to have any part of my desire herein satisfied, unless myself be careful to satisfy the like desire which is undoubtedly in other men, we all being of one and the same nature? To have any thing offered them repugnant to this desire must needs in all respects grieve them as much as me: so that if I do harm I must look to suffer; there being no reason that others should shew greater measure of love to me than they have by me shewed unto them. My desire therefore to be loved of my equals in nature as much as possible may be, imposeth upon me a natural duty of bearing to them-ward fully the like affection. From which relation of equality between ourselves and them that are as ourselves, what several rules and canons natural Reason hath drawn for direction of life no man is ignorant; as namely, “That because we would take no harm, we must  therefore do none;” “That sith we would not be in any thing extremely dealt with, we must ourselves avoid all extremity in our dealings;” “That from all violence and wrong we are utterly to abstain;” with such like; which further to wade in would be tedious, and to our present purpose not altogether so necessary, seeing that on these two general heads already mentioned all other specialities are dependent.

[8.]Wherefore the natural measure whereby to judge our doings, is the sentence of Reason, determining and setting down what is good to be done. Which sentence is either mandatory, shewing what must be done; or else permissive, declaring only what may be done; or thirdly admonitory, opening what is the most convenient for us to do. The first taketh place, where the comparison doth stand altogether between doing and not doing of one thing which in itself is absolutely good or evil; as it had been for Joseph to yield or not to yield to the impotent desire of his lewd mistress, the one evil the other good simply. The second is, when of divers things evil, all being not evitable, we are permitted to take one; which one saving only in case of so great urgency were not otherwise to be taken; as in the matter of divorce amongst the Jews. The last, when of divers things good, one is principal and most eminent; as in their act who sold their possessions and laid the price at the Apostles’ feet; which possessions they might have retained unto themselves without sin: again, in the Apostle St. Paul’s own choice to maintain himself by his own labour; whereas in living by the Church’s maintenance, as others did, there had been no offence committed. In Goodness therefore there is a latitude or extent, whereby it cometh to pass that even of good actions some are better than other some; whereas  otherwise one man could not excel another, but all should be either absolutely good, as hitting jump that indivisible point or centre wherein goodness consisteth; or else missing it they should be excluded out of the number of well-doers. Degrees of well-doing there could be none, except perhaps in the seldomness and oftenness of doing well. But the nature of Goodness being thus ample, a Law is properly that which Reason in such sort defineth to be good that it must be done. And the Law of Reason or human Nature is that which men by discourse of natural Reason have rightly found out themselves to be all for ever bound unto in their actions.

[9.]Laws of Reason have these marks to be known by. Such as keep them resemble most lively in their voluntary actions that very manner of working which Nature herself doth necessarily observe in the course of the whole world. The works of Nature are all behoveful, beautiful, without superfluity or defect; even so theirs, if they be framed according to that which the Law of Reason teacheth. Secondly, those Laws are investigable by Reason, without the help of Revelation supernatural and divine. Finally, in such sort they are investigable, that the knowledge of them is general, the world hath always been acquainted with them; according to that which one in Sophocles observeth concerning a branch of this Law, “It is no child of to-day’s or yesterday’s birth, but hath been no man knoweth how long sithence.” It is not agreed upon by one, or two, or few, but by all. Which we may not so understand, as if every particular man in the whole world did know and confess whatsoever the Law of Reason doth contain; but this Law is such that being proposed no man can reject it as unreasonable and unjust. Again, there is nothing in it but any man (having natural perfection of wit and ripeness of judgment) may by labour and travail find out. And to conclude, the general principles thereof are such, as it is not easy to find men ignorant of them, Law rational therefore, which men commonly use to call the Law of Nature, meaning thereby the Law which human Nature knoweth itself in reason universally bound unto, which also  for that cause may be termed most fitly the Law of Reason; this Law, I say, comprehendeth all those things which men by the light of their natural understanding evidently know, or at leastwise may know, to be beseeming or unbeseeming, virtuous or vicious, good or evil for them to do.

[10.]Now although it be true, which some have said, that “whatsoever is done amiss, the Law of Nature and Reason thereby is transgressed,” because even those offences which are by their special qualities breaches of supernatural laws, do also, for that they are generally evil, violate in general that principle of Reason, which willeth universally to fly from evil: yet do we not therefore so far extend the Law of Reason, as to contain in it all manner laws whereunto reasonable creatures are bound, but (as hath been shewed) we restrain it to those only duties, which all men by force of natural wit either do or might understand to be such duties as concern all men. “Certain half-waking men there are” (as Saint Augustine noteth), “who neither altogether asleep in folly, nor yet throughly awake in the light of true understanding, have thought that there is not at all any thing just and righteous in itself; but look, wherewith nations are inured, the same they take to be right and just. Whereupon their conclusion is, that seeing each sort of people hath a different kind of right from other, and that which is right of its own nature must be everywhere one and the same, therefore in itself there is nothing right. These good folk,” saith he, (“that I may not trouble their wits with rehearsal of too many things,) have not looked so far into the world as to perceive that, ‘Do as thou wouldest be done unto,’ is a sentence which all nations  under heaven are agreed upon. Refer this sentence to the love of God, and it extinguisheth all heinous crimes; refer it to the love of thy neighbour, and all grievous wrongs it banisheth out of the world.” Wherefore as touching the Law of Reason, this was (it seemeth) Saint Augustine’s judgment: namely, that there are in it some things which stand as principles universally agreed upon; and that out of those principles, which are in themselves evident, the greatest moral duties we owe towards God or man may without any great difficulty be concluded.

[11.]If then it be here demanded, by what means it should come to pass (the greatest part of the Law moral being so easy for all men to know) that so many thousands of men notwithstanding have been ignorant even of principal moral duties, not imagining the breach of them to be sin: I deny not but lewd and wicked custom, beginning perhaps at the first amongst few, afterwards spreading into greater multitudes, and so continuing from time to time, may be of force even in plain things to smother the light of natural understanding; because men will not bend their wits to examine whether things wherewith they have been accustomed be good or evil. For example’s sake, that grosser kind of heathenish idolatry, whereby they worshipped the very works of their own hands, was an absurdity to reason so palpable, that the Prophet David comparing idols and idolaters together maketh almost no odds between them, but the one in a manner as much without wit and sense as the other; “They that make them are like unto them, and so are all that trust in them.” That wherein an idolater doth seem so absurd and foolish is by the Wise Man thus exprest, “He is not ashamed to speak unto that which hath no life, he calleth on him that is weak for health, he prayeth for life unto him which is dead, of him which hath no experience he requireth help, for his journey he sueth to him which is not able to go, for gain and work and success in his affairs he seeketh furtherance of him that hath no manner of power.” The cause of which senseless stupidity is afterwards imputed to custom. “When a father mourned grievously for his son that was taken away suddenly, he  made an image for him that was once dead, whom now he worshippeth as a god, ordaining to his servants ceremonies and sacrifices. Thus by process of time this wicked custom prevailed, and was kept as a law;” the authority of rulers, the ambition of craftsmen, and such like means thrusting forward the ignorant, and increasing their superstition.

Unto this which the Wise Man hath spoken somewhat besides may be added. For whatsoever we have hitherto taught, or shall hereafter, concerning the force of man’s natural understanding, this we always desire withal to be understood; that there is no kind of faculty or power in man or any other creature, which can rightly perform the functions allotted to it, without perpetual aid and concurrence of that Supreme Cause of all things. The benefit whereof as oft as we cause God in his justice to withdraw, there can no other thing follow than that which the Apostle noteth, even men endued with the light of reason to walk notwithstanding “in the vanity of their mind, having their cogitations darkened, and being strangers from the life of God through the ignorance which is in them, because of the hardness of their hearts.” And this cause is mentioned by the prophet Esay, speaking of the ignorance of idolaters, who see not how the manifest Law of Reason condemneth their gross iniquity and sin. “They have not in them,” saith he, “so much wit as to think, ‘Shall I bow to the stock of a tree?’ All knowledge and understanding is taken from them; for God hath shut their eyes that they cannot see.”

That which we say in this case of idolatry serveth for all other things, wherein the like kind of general blindness hath prevailed against the manifest Laws of Reason. Within the compass of which laws we do not only comprehend whatsoever may be easily known to belong to the duty of all men, but even whatsoever may possibly be known to be of that quality, so that the same be by necessary consequence deduced out of clear and manifest principles. For if once we descend unto probable collections what is convenient for men, we are then in the territory where free and arbitrary determinations, the territory where Human Laws take place; which laws are after to be considered.

\section*{Of the benefit of keeping that Law which Reason teacheth.}

IX. Now the due observation of this Law which Reason teacheth us cannot but be effectual unto their great good that observe the same. For we see the whole world and each part thereof so compacted, that as long as each thing performeth only that work which is natural unto it, it thereby preserveth both other things and also itself. Contrariwise, let any principal thing, as the sun, the moon, any one of the heavens or elements, but once cease or fail, or swerve, and who doth not easily conceive that the sequel thereof would be ruin both to itself and whatsoever dependeth on it? And is it possible, that Man being not only the noblest creature in the world, but even a very world in himself, his transgressing the Law of his Nature should draw no manner of harm after it? Yes, “tribulation and anguish unto every soul that doeth evil.” Good doth follow unto all things by observing the course of their nature, and on the contrary side evil by not observing it; but not unto natural agents that good which we call Reward, not that evil which we properly term Punishment. The reason whereof is, because amongst creatures in this world, only Man’s observation of the Law of his Nature is Righteousness, only Man’s transgression Sin. And the reason of this is the difference in his manner of observing or transgressing the Law of his Nature. He doth not otherwise than voluntarily the one or the other. What we do against our wills, or constrainedly, we are not properly said to do it, because the motive cause of doing it is not in ourselves, but carrieth us, as if the wind should drive a feather in the air, we no whit furthering that whereby we are driven. In such cases therefore the evil which is done moveth compassion; men are pitied for it, as being rather miserable in such respect than culpable. Some things are likewise done by man, though not through outward force and impulsion, though not against, yet without their wills; as in alienation of mind, or any the like inevitable utter absence of wit and judgment. For which cause, no man did ever think the hurtful actions of furious men and innocents to be punishable. Again, some things we do neither against nor without, and yet not simply and merely with our wills, but with our wills in such sort moved, that  albeit there be no impossibility but that we might, nevertheless we are not so easily able to do otherwise. In this consideration one evil deed is made more pardonable than another. Finally, that which we do being evil, is notwithstanding by so much more pardonable, by how much the exigence of so doing or the difficulty of doing otherwise is greater; unless this necessity or difficulty have originally risen from ourselves. It is no excuse therefore unto him, who being drunk committeth incest, and allegeth that his wits were not his own; inasmuch as himself might have chosen whether his wits should by that mean have been taken from him. Now rewards and punishments do always presuppose something willingly done well or ill; without which respect though we may sometimes receive good or harm, yet then the one is only a benefit and not a reward, the other simply an hurt not a punishment. From the sundry dispositions of man’s Will, which is the root of all his actions, there groweth variety in the sequel of rewards and punishments, which are by these and the like rules measured: “Take away the will, and all acts are equal: That which we do not, and would do, is commonly accepted as done.” By these and the like rules men’s actions are determined of and judged, whether they be in their own nature rewardable or punishable.

[2.]Rewards and punishments are not received, but at the hands of such as being above us have power to examine and judge our deeds. How men come to have this authority one over another in external actions, we shall more diligently examine in that which followeth. But for this present, so much all do acknowledge, that sith every man’s heart and conscience doth in good or evil, even secretly committed and known to none but itself, either like or disallow itself, and accordingly either rejoice, very nature exulting (as it were) in certain hope of reward, or else grieve (as it were) in a sense of future punishment; neither of which can in this case be looked for from any other, saving only from Him who discerneth and judgeth the very secrets of all hearts:  therefore He is the only rewarder and revenger of all such actions; although not of such actions only, but of all whereby the Law of Nature is broken whereof Himself is author. For which cause, the Roman laws, called The Laws of the Twelve Tables, requiring offices of inward affection which the eye of man cannot reach unto, threaten the neglecters of them with none but divine punishment.

\section*{How Reason doth lead men unto the making of human Laws, whereby politic Societies are governed, and to agreement about Laws whereby the fellowship or communion of independent Societies standeth.}

X. That which hitherto we have set down is (I hope) sufficient to shew their brutishness, which imagine that religion and virtue are only as men will account of them; that we might make as much account, if we would, of the contrary, without any harm unto ourselves, and that in nature they are as indifferent one as the other. We see then how nature itself teacheth laws and statutes to live by. The laws which have been hitherto mentioned do bind men absolutely even as they are men, although they have never any settled fellowship, never any solemn agreement amongst themselves what to do or not to do. But forasmuch as we are not by ourselves sufficient to furnish ourselves with competent store of things needful for such a life as our nature doth desire, a life fit for the dignity of man; therefore to supply those defects and imperfections which are in us living single and solely by ourselves, we are naturally induced to seek communion and fellowship with others. This was the cause of men’s uniting themselves at the first in politic Societies, which societies could not be without Government, nor Government without a distinct kind of Law from that which hath been already declared. Two foundations there are which bear up public societies; the one, a natural inclination, whereby all men desire sociable life and fellowship; the other, an order expressly or secretly agreed upon touching the manner of their union in living together. The latter is that which we call the Law of a Commonweal, the very soul of a politic body, the parts whereof are by law animated, held together, and set on work in such actions, as the common good requireth. Laws politic, ordained for external order and regiment amongst men, are never framed as they  should be, unless presuming the will of man to be inwardly obstinate, rebellious, and averse from all obedience unto the sacred laws of his nature; in a word, unless presuming man to be in regard of his depraved mind little better than a wild beast, they do accordingly provide notwithstanding so to frame his outward actions, that they be no hindrance unto the common good for which societies are instituted: unless they do this, they are not perfect. It resteth therefore that we consider how nature findeth out such laws of government as serve to direct even nature depraved to a right end.

[2.]All men desire to lead in this world a happy life. That life is led most happily, wherein all virtue is exercised without impediment or let. The Apostle, in exhorting men to contentment although they have in this world no more than very bare food and raiment, giveth us thereby to understand that those are even the lowest of things necessary; that if we should be stripped of all those things without which we might possibly be, yet these must be left; that destitution in these is such an impediment, as till it be removed suffereth not the mind of man to admit any other care. For this cause, first God assigned Adam maintenance of life, and then appointed him a law to observe. For this cause, after men began to grow to a number, the first thing we read they gave themselves unto was the tilling of the earth and the feeding of cattle. Having by this mean whereon to live, the principal actions of their life afterward are noted by the exercise of their religion. True it is, that the kingdom of God must be the first thing in our purposes and desires. But inasmuch as righteous life presupposeth life; inasmuch as to live virtuously it is impossible except we live; therefore the first impediment, which naturally we endeavour to remove, is penury and want of things without which we cannot live. Unto life many implements are necessary; more, if we seek (as all men naturally do) such a life as hath in it joy, comfort, delight, and pleasure. To this end we see how quickly sundry arts mechanical were found out, in the very prime of the world. As things of greatest  necessity are always first provided for, so things of greatest dignity are most accounted of by all such as judge rightly. Although therefore riches be a thing which every man wisheth, yet no man of judgment can esteem it better to be rich, than wise, virtuous, and religious. If we be both or either of these, it is not because we are so born. For into the world we come as empty of the one as of the other, as naked in mind as we are in body. Both which necessities of man had at the first no other helps and supplies than only domestical; such as that which the Prophet implieth, saying, “Can a mother forget her child?” such as that which the Apostle mentioneth, saying, “He that careth not for his own is worse than an infidel;” such as that concerning Abraham, “Abraham will command his sons and his household after him, that they keep the way of the Lord.”

[3.]But neither that which we learn of ourselves nor that which others teach us can prevail, where wickedness and malice have taken deep root. If therefore when there was but as yet one only family in the world, no means of instruction human or divine could prevent effusion of blood; how could it be chosen but that when families were multiplied and increased upon earth, after separation each providing for itself, envy, strife, contention and violence must grow amongst them? For hath not Nature furnished man with wit and valour, as it were with armour, which may be used as well unto extreme evil as good? Yea, were they not used by the rest of the world unto evil; unto the contrary only by Seth, Enoch, and those few the rest in that line? We all make complaint of the iniquity of our times: not unjustly; for the days are evil. But compare them with those times wherein there were no civil societies, with those times wherein there was as yet no manner of public regiment established, with those times wherein there were not above eight persons righteous living upon the face of the earth; and we have surely good cause to think that God hath blessed us exceedingly, and hath made us behold most happy days.

[4.]To take away all such mutual grievances, injuries, and wrongs, there was no way but only by growing unto composition  and agreement amongst themselves, by ordaining some kind of government public, and by yielding themselves subject thereunto; that unto whom they granted authority to rule and govern, by them the peace, tranquillity, and happy estate of the rest might be procured. Men always knew that when force and injury was offered they might be defenders of themselves; they knew that howsoever men may seek their own commodity, yet if this were done with injury unto others it was not to be suffered, but by all men and by all good means to be withstood; finally they knew that no man might in reason take upon him to determine his own right, and according to his own determination proceed in maintenance thereof, inasmuch as every man is towards himself and them whom he greatly affecteth partial; and therefore that strifes and troubles would be endless, except they gave their common consent all to be ordered by some whom they should agree upon: without which consent there were no reason that one man should take upon him to be lord or judge over another; because, although there be according to the opinion of some very great and judicious men a kind of natural right in the noble, wise, and virtuous, to govern them which are of servile disposition; nevertheless for manifestation of this their right, and men’s more peaceable contentment on both sides, the assent of them who are to be governed seemeth necessary.

To fathers within their private families Nature hath given a supreme power; for which cause we see throughout the world even from the foundation thereof, all men have ever been taken as lords and lawful kings in their own houses. Howbeit over a whole grand multitude having no such dependency upon any one, and consisting of so many families as every politic society in the world doth, impossible it is that any should have complete lawful power, but by consent of men, or immediate appointment of God; because not having the natural superiority of fathers, their power must needs be either usurped, and then unlawful; or, if lawful, then either granted or consented unto by them over whom they exercise the same, or else given extraordinarily from God, unto whom all the world is subject. It is no improbable opinion therefore which the arch-philosopher was of, that as the chiefest person  in every household was always as it were a king, so when numbers of households joined themselves in civil society together, kings were the first kind of governors amongst them. Which is also (as it seemeth) the reason why the name of Father continued still in them, who of fathers were made rulers; as also the ancient custom of governors to do as Melchisedec, and being kings to exercise the office of priests, which fathers did at the first, grew perhaps by the same occasion.

Howbeit not this the only kind of regiment that hath been received in the world. The inconveniences of one kind have caused sundry other to be devised. So that in a word all public regiment of what kind soever seemeth evidently to have risen from deliberate advice, consultation, and composition between men, judging it convenient and behoveful; there being no impossibility in nature considered by itself, but that men might have lived without any public regiment. Howbeit, the corruption of our nature being presupposed, we may not deny but that the Law of Nature doth now require of necessity some kind of regiment, so that to bring things unto the first course they were in, and utterly to take away all kind of public government in the world, were apparently to overturn the whole world.

[5.]The case of man’s nature standing therefore as it doth, some kind of regiment the Law of Nature doth require; yet the kinds thereof being many, Nature tieth not to any one, but leaveth the choice as a thing arbitrary. At the first when some certain kind of regiment was once approved, it may be that nothing was then further thought upon for the manner of governing, but all permitted unto their wisdom and discretion which were to rule; till by experience they found this for all parts very inconvenient, so as the thing which they had devised for a remedy did indeed but increase the sore which it should have cured. They saw that to live by one man’s will became the cause of all men’s misery. This constrained  them to come unto laws, wherein all men might see their duties beforehand, and know the penalties of transgressing them. If things be simply good or evil, and withal universally so acknowledged, there needs no new law to be made for such things. The first kind therefore of things appointed by laws human containeth whatsoever being in itself naturally good or evil, is notwithstanding more secret than that it can be discerned by every man’s present conceit, without some deeper discourse and judgment. In which discourse because there is difficulty and possibility many ways to err, unless such things were set down by laws, many would be ignorant of their duties which now are not, and many that know what they should do would nevertheless dissemble it, and to excuse themselves pretend ignorance and simplicity, which now they cannot.

[6.]And because the greatest part of men are such as prefer their own private good before all things, even that good which is sensual before whatsoever is most divine; and for that the labour of doing good, together with the pleasure arising from the contrary, doth make men for the most part slower to the one and proner to the other, than that duty prescribed them by law can prevail sufficiently with them: therefore unto laws that men do make for the benefit of men it hath seemed always needful to add rewards, which may more allure unto good than any hardness deterreth from it, and punishments, which may more deter from evil than any sweetness thereto allureth. Wherein as the generality is natural, virtue rewardable and vice punishable; so the particular determination of the reward or punishment belongeth unto them by whom laws are made. Theft is naturally punishable, but the kind of punishment is positive, and such lawful as men shall think with discretion convenient by law to appoint.

[7.]In laws, that which is natural bindeth universally, that which is positive not so. To let go those kind of positive  laws which men impose upon themselves, as by vow unto God, contract with men, or such like; somewhat it will make unto our purpose, a little more fully to consider what things are incident into the making of the positive laws for the government of them that live united in public society. Laws do not only teach what is good, but they enjoin it, they have in them a certain constraining force. And to constrain men unto any thing inconvenient doth seem unreasonable. Most requisite therefore it is that to devise laws which all men shall be forced to obey none but wise men be admitted. Laws are matters of principal consequence; men of common capacity and but ordinary judgment are not able (for how should they?) to discern what things are fittest for each kind and state of regiment. We cannot be ignorant how much our obedience unto laws dependeth upon this point. Let a man though never so justly oppose himself unto them that are disordered in their ways, and what one amongst them commonly doth not stomach at such contradiction, storm at reproof, and hate such as would reform them? Notwithstanding even they which brook it worst that men should tell them of their duties, when they are told the same by a law, think very well and reasonably of it. For why? They presume that the law doth speak with all indifferency; that the law hath no side-respect to their persons; that the law is as it were an oracle proceeded from wisdom and understanding.

[8.]Howbeit laws do not take their constraining force from the quality of such as devise them, but from that power which doth give them the strength of laws. That which we spake before concerning the power of government must here be applied unto the power of making laws whereby to govern; which power God hath over all: and by the natural law, whereunto he hath made all subject, the lawful power of making laws to command whole politic societies of men belongeth so properly unto the same entire societies, that for any prince or potentate of what kind soever upon earth to exercise the same of himself, and not either by express commission immediately and personally received from God, or else by authority derived at the first from  their consent upon whose persons they impose laws, it is no better than mere tyranny.

Laws they are not therefore which public approbation hath not made so. But approbation not only they give who personally declare their assent by voice sign or act, but also when others do it in their names by right originally at the least derived from them. As in parliaments, councils, and the like assemblies, although we be not personally ourselves present, notwithstanding our assent is by reason of others agents there in our behalf. And what we do by others, no reason but that it should stand as our deed, no less effectually to bind us than if ourselves had done it in person. In many things assent is given, they that give it not imagining they do so, because the manner of their assenting is not apparent. As for example, when an absolute monarch commandeth his subjects that which seemeth good in his own discretion, hath not his edict the force of a law whether they approve or dislike it? Again, that which hath been received long sithence and is by custom now established, we keep as a law which we may not transgress; yet what consent was ever thereunto sought or required at our hands?

Of this point therefore we are to note, that sith men naturally have no full and perfect power to command whole politic multitudes of men, therefore utterly without our consent we could in such sort be at no man’s commandment living. And to be commanded we do consent, when that society whereof we are part hath at any time before consented, without revoking the same after by the like universal agreement. Wherefore as any man’s deed past is good as long as himself continueth; so the act of a public society of men done five hundred years sithence standeth as theirs who presently are of the same societies, because corporations are immortal; we were then alive in our predecessors, and they in their successors do live still. Laws therefore human, of what kind soever, are available by consent.

[9.]If here it be demanded how it cometh to pass that this being common unto all laws which are made, there should be found even in good laws so great variety as there  is; we must note the reason hereof to be the sundry particular ends, whereunto the different disposition of that subject or matter, for which laws are provided, causeth them to have especial respect in making laws. A law there is mentioned amongst the Grecians whereof Pittacus is reported to have been author; and by that law it was agreed, that he which being overcome with drink did then strike any man, should suffer punishment double as much as if he had done the same being sober. No man could ever have thought this reasonable, that had intended thereby only to punish the injury committed according to the gravity of the fact: for who knoweth not that harm advisedly done is naturally less pardonable, and therefore worthy of the sharper punishment? But forasmuch as none did so usually this way offend as men in that case, which they wittingly fell into, even because they would be so much the more freely outrageous; it was for their public good where such disorder was grown to frame a positive law for remedy thereof accordingly. To this appertain those known laws of making laws; as that law-makers must have an eye to the place where, and to the men amongst whom; that one kind of laws cannot serve for all kinds of regiment; that where the multitude beareth sway, laws that shall tend unto preservation of that state must make common smaller offices to go by lot, for fear of strife and division likely to arise; by reason that ordinary qualities sufficing for discharge of such offices, they could not but by many be desired, and so with danger contended for, and not missed without grudge and discontentment, whereas at an uncertain lot none can find themselves grieved, on whomsoever it lighteth; contrariwise the greatest, whereof but few are capable, to pass by popular election, that neither the people may envy such as have those honours, inasmuch as themselves bestow them, and that the chiefest may be kindled with desire to exercise all parts of rare and beneficial virtue, knowing they shall not lose their labour by growing in fame and estimation amongst the people: if the helm of chief government be in the hands of a few of the wealthiest, that then laws providing for continuance thereof must make the punishment of contumely and wrong offered  unto any of the common sort sharp and grievous, that so the evil may be prevented whereby the rich are most likely to bring themselves into hatred with the people, who are not wont to take so great offence when they are excluded from honours and offices, as when their persons are contumeliously trodden upon. In other kinds of regiment the like is observed concerning the difference of positive laws, which to be every where the same is impossible and against their nature.

[10.]Now as the learned in the laws of this land observe, that our statutes sometimes are only the affirmation or ratification of that which by common law was held before; so here it is not to be omitted that generally all laws human, which are made for the ordering of politic societies, be either such as establish some duty whereunto all men by the law of reason did before stand bound; or else such as make that a duty now which before was none. The one sort we may for distinction’s sake call “mixedly,” and the other “merely” human. That which plain or necessary reason bindeth men unto may be in sundry considerations expedient to be ratified by human law. For example, if confusion of blood in marriage, the liberty of having many wives at once, or any other the like corrupt and unreasonable custom doth happen to have prevailed far, and to have gotten the upper hand of right reason with the greatest part; so that no way is left to rectify such foul disorder without prescribing by law the same things which reason necessarily doth enforce but is not perceived that so it doth; or if many be grown unto that which the Apostle did lament in some, concerning whom he writeth, saying, that “even what things they naturally know, in those very things as beasts void of reason they corrupted themselves;” or if there be no such special accident, yet forasmuch as the common sort are led by the sway of their  sensual desires, and therefore do more shun sin for the sensible evils which follow it amongst men, than for any kind of sentence which reason doth pronounce against it: this very thing is cause sufficient why duties belonging unto each kind of virtue, albeit the Law of Reason teach them, should notwithstanding be prescribed even by human law. Which law in this case we term mixed, because the matter whereunto it bindeth is the same which reason necessarily doth require at our hands, and from the Law of Reason it differeth in the manner of binding only. For whereas men before stood bound in conscience to do as the Law of Reason teacheth, they are now by virtue of human law become constrainable, and if they outwardly transgress, punishable. As for laws which are merely human, the matter of them is any thing which reason doth but probably teach to be fit and convenient; so that till such time as law hath passed amongst men about it, of itself it bindeth no man. One example whereof may be this. Lands are by human law in some places after the owner’s decease divided unto all his children, in some all descendeth to the eldest son. If the Law of Reason did necessarily require but the one of these two to be done, they which by law have received the other should be subject to that heavy sentence, which denounceth against all that decree wicked, unjust, and unreasonable things, woe. Whereas now whichsoever be received there is no Law of Reason transgressed; because there is probable reason why either of them may be expedient, and for either of them more than probable reason there is not to be found.

[11.]Laws whether mixedly or merely human are made by politic societies: some, only as those societies are civilly united; some, as they are spiritually joined and make such a body as we call the Church. Of laws human in this latter kind we are to speak in the third book following. Let it therefore suffice thus far to have touched the force wherewith Almighty God hath graciously endued our nature, and thereby enabled the same to find out both those laws which all men generally are for ever bound to observe, and also such  as are most fit for their behoof, who lead their lives in any ordered state of government.

[12.]Now besides that law which simply concerneth men as men, and that which belongeth unto them as they are men linked with others in some form of politic society, there is a third kind of law which toucheth all such several bodies politic, so far forth as one of them hath public commerce with another. And this third is the Law of Nations. Between men and beasts there is no possibility of sociable communion, because the well-spring of that communion is a natural delight which man hath to transfuse from himself into others, and to receive from others into himself especially those things wherein the excellency of his kind doth most consist. The chiefest instrument of human communion therefore is speech, because thereby we impart mutually one to another the conceits of our reasonable understanding. And for that cause seeing beasts are not hereof capable, forasmuch as with them we can use no such conference, they being in degree, although above other creatures on earth to whom nature hath denied sense, yet lower than to be sociable companions of man to whom nature hath given reason; it is of Adam said that amongst the beasts “he found not for himself any meet companion.” Civil society doth more content the nature of man than any private kind of solitary living, because in society this good of mutual participation is so much larger than otherwise. Herewith notwithstanding we are not satisfied, but we covet (if it might be) to have a kind of society and fellowship even with all mankind. Which thing Socrates intending to signify professed himself a citizen, not of this or that commonwealth, but of the world. And an effect of that very natural desire in us (a manifest token that we wish after a sort an universal fellowship with all men) appeareth by the wonderful delight men have, some to visit foreign countries, some to discover nations not heard of in former ages, we all to know the affairs and dealings of other people, yea to be in league of amity with them: and this not only for traffick’s sake, or to the end that when many are confederated each may make other the more strong, but  for such cause also as moved the Queen of Saba to visit Salomon; and in a word, because nature doth presume that how many men there are in the world, so many gods as it were there are, or at leastwise such they should be towards men.

[13.]Touching laws which are to serve men in this behalf; even as those Laws of Reason, which (man retaining his original integrity) had been sufficient to direct each particular person in all his affairs and duties, are not sufficient but require the access of other laws, now that man and his offspring are grown thus corrupt and sinful; again, as those laws of polity and regiment, which would have served men living in public society together with that harmless disposition which then they should have had, are not able now to serve, when men’s iniquity is so hardly restrained within any tolerable bounds: in like manner, the national laws of mutual commerce between societies of that former and better quality might have been other than now, when nations are so prone to offer violence, injury, and wrong. Hereupon hath grown in every of these three kinds that distinction between Primary and Secondary laws; the one grounded upon sincere, the other built upon depraved nature. Primary laws of nations are such as concern embassage, such as belong to the courteous entertainment of foreigners and strangers, such as serve for commodious traffick, and the like. Secondary laws in the same kind are such as this present unquiet world is most familiarly acquainted with; I mean laws of arms, which yet are much better known than kept. But what matter the Law of Nations doth contain I omit to search.

The strength and virtue of that law is such that no particular nation can lawfully prejudice the same by any their several laws and ordinances, more than a man by his private resolutions the law of the whole commonwealth or state wherein he liveth. For as civil law, being the act of a whole body politic, doth therefore overrule each several part of the same body; so there is no reason that any one commonwealth of itself should to the prejudice of another  annihilate that whereupon the whole world hath agreed. For which cause, the Lacedæmonians forbidding all access of strangers into their coasts, are in that respect both by Josephus and Theodoret deservedly blamed, as being enemies to that hospitality which for common humanity’s sake all the nations on earth should embrace.

[14.]Now as there is great cause of communion, and consequently of laws for the maintenance of communion, amongst nations; so amongst nations Christian the like in regard even of Christianity hath been always judged needful.

And in this kind of correspondence amongst nations the force of general councils doth stand. For as one and the same law divine, whereof in the next place we are to speak, is unto all Christian churches a rule for the chiefest things; by means whereof they all in that respect make one church, as having all but “one Lord, one faith, and one baptism:” so the urgent necessity of mutual communion for preservation of our unity in these things, as also for order in some other things convenient to be every where uniformly kept, maketh it requisite that the Church of God here on earth have her laws of spiritual commerce between Christian nations; laws by virtue whereof all churches may enjoy freely the use of those reverend, religious, and sacred consultations, which are termed Councils General. A thing whereof God’s own blessed Spirit was the author; a thing practised by the holy Apostles themselves; a thing always afterwards kept and observed throughout the world; a thing never otherwise than most highly esteemed of, till pride, ambition, and tyranny began by factious and vile endeavours to abuse that divine invention unto the furtherance of wicked purposes. But as the just authority of civil courts and parliaments is not therefore to be abolished, because sometime there is cunning used to frame them according to the private intents of men over potent in the commonwealth; so the grievous abuse which hath been of councils should rather cause men to study how so gracious a thing may again be reduced to that first perfection, than in regard of stains and blemishes sithence growing be held for ever in extreme disgrace.


To speak of this matter as the cause requireth would require very long discourse. All I will presently say is this: whether it be for the finding out of any thing whereunto divine law bindeth us, but yet in such sort that men are not thereof on all sides resolved; or for the setting down of some uniform judgment to stand touching such things, as being neither way matters of necessity, are notwithstanding offensive and scandalous when there is open opposition about them; be it for the ending of strifes, touching matters of Christian belief, wherein the one part may seem to have probable cause of dissenting from the other; or be it concerning matters of polity, order, and regiment in the church; I nothing doubt but that Christian men should much better frame themselves to those heavenly precepts, which our Lord and Saviour with so great instancy gave as concerning peace and unity, if we did all concur in desire to have the use of ancient councils again renewed, rather than these proceedings continued, which either make all contentions endless, or bring them to one only determination, and that of all other the worst, which is by sword.

[15.]It followeth therefore that a new foundation being laid, we now adjoin hereunto that which cometh in the next place to be spoken of; namely, wherefore God hath himself by Scripture made known such laws as serve for direction of men.

\section*{Wherefore God hath by Scripture further made known such supernatural Laws as do serve for men’s direction.}

XI. All things, (God only excepted,) besides the nature which they have in themselves, receive externally some perfection from other things, as hath been shewed. Insomuch as there is in the whole world no one thing great or small, but either in respect of knowledge or of use it may unto our perfection add somewhat. And whatsoever such perfection there is which our nature may acquire, the same we properly term our Good; our Sovereign Good or Blessedness, that wherein the highest degree of all our perfection consisteth, that which being once attained unto there can rest nothing further to be desired; and therefore with it our souls are fully content and satisfied, in that they have they rejoice, and thirst for no more. Wherefore of good things desired some are such that for themselves we covet them not, but only because they serve as instruments unto that for which we are  to seek: of this sort are riches. Another kind there is, which although we desire for itself, as health, and virtue, and knowledge, nevertheless they are not the last mark whereat we aim, but have their further end whereunto they are referred, so as in them we are not satisfied as having attained the utmost we may, but our desires do still proceed. These things are linked and as it were chained one to another; we labour to eat, and we eat to live, and we live to do good, and the good which we do is as seed sown with reference to a future harvest. But we must come at length to some pause. For, if every thing were to be desired for some other without any stint, there could be no certain end proposed unto our actions, we should go on we know not whither; yea, whatsoever we do were in vain, or rather nothing at all were possible to be done. For as to take away the first efficient of our being were to annihilate utterly our persons, so we cannot remove the last final cause of our working, but we shall cause whatsoever we work to cease. Therefore something there must be desired for itself simply and for no other. That is simply for itself desirable, unto the nature whereof it is opposite and repugnant to be desired with relation unto any other. The ox and the ass desire their food, neither propose they unto themselves any end wherefore; so that of them this is desired for itself; but why? By reason of their imperfection which cannot otherwise desire it; whereas that which is desired simply for itself, the excellency thereof is such as permitteth it not in any sort to be referred to a further end.

[2.]Now that which man doth desire with reference to a further end, the same he desireth in such measure as is unto that end convenient; but what he coveteth as good in itself, towards that his desire is ever infinite. So that unless the last good of all, which is desired altogether for itself, be also infinite, we do evil in making it our end; even as they who placed their felicity in wealth or honour or pleasure or any thing here attained; because in desiring any thing as our final perfection which is not so, we do amiss. Nothing  may be infinitely desired but that good which indeed is infinite; for the better the more desirable; that therefore most desirable wherein there is infinity of goodness: so that if any thing desirable may be infinite, that must needs be the highest of all things that are desired. No good is infinite but only God; therefore he our felicity and bliss. Moreover, desire tendeth unto union with that it desireth. If then in Him we be blessed, it is by force of participation and conjunction with Him. Again, it is not the possession of any good thing can make them happy which have it, unless they enjoy the thing wherewith they are possessed. Then are we happy therefore when fully we enjoy God, as an object wherein the powers of our souls are satisfied even with everlasting delight; so that although we be men, yet by being unto God united we live as it were the life of God.

[3.]Happiness therefore is that estate whereby we attain, so far as possibly may be attained, the full possession of that which simply for itself is to be desired, and containeth in it after an eminent sort the contentation of our desires, the highest degree of all our perfection. Of such perfection capable we are not in this life. For while we are in the world, subject we are unto sundry imperfections, griefs of body, defects of mind; yea the best things we do are painful, and the exercise of them grievous, being continued without intermission; so as in those very actions whereby we are especially perfected in this life we are not able to persist; forced we are with very weariness, and that often, to interrupt them: which tediousness cannot fall into those operations that are in the state of bliss, when our union with God is complete. Complete union with him must be according unto every power and faculty of our minds apt to receive so glorious an object. Capable we are of God both by understanding and will: by understanding, as He is that sovereign Truth which comprehendeth the rich treasures of all wisdom; by will, as He is that sea of Goodness whereof whoso tasteth  shall thirst no more. As the will doth now work upon that object by desire, which is as it were a motion towards the end as yet unobtained; so likewise upon the same hereafter received it shall work also by love. “Appetitus inhiantis fit amor fruentis,” saith St. Augustine: “The longing disposition of them that thirst is changed into the sweet affection of them that taste and are replenished.” Whereas we now love the thing that is good, but good especially in respect of benefit unto us; we shall then love the thing that is good, only or principally for the goodness of beauty in itself. The soul being in this sort, as it is active, perfected by love of that infinite good, shall, as it is receptive, be also perfected with those supernatural passions of joy, peace, and delight. All this endless and everlasting. Which perpetuity, in regard whereof our blessedness is termed “a crown which withereth not,” doth neither depend upon the nature of the thing itself, nor proceed from any natural necessity that our souls should so exercise themselves for ever in beholding and loving God, but from the will of God, which doth both freely perfect our nature in so high a degree, and continue it so perfected. Under Man, no creature in the world is capable of felicity and bliss. First, because their chiefest perfection consisteth in that which is best for them, but not in that which is simply best, as ours doth. Secondly, because whatsoever external perfection they tend unto, it is not better than themselves, as ours is. How just occasion have we therefore even in this respect with the Prophet to admire the goodness of God! “Lord, what is man, that thou shouldst exalt him above the works of thy hands,” so far as to make thyself the inheritance of his rest and the substance of his felicity?

[4.]Now if men had not naturally this desire to be happy, how were it possible that all men should have it? All men have. Therefore this desire in man is natural. It is not in our power not to do the same; how should it then be in our power to do it coldly or remissly? So that our desire being  natural is also in that degree of earnestness whereunto nothing can be added. And is it probable that God should frame the hearts of all men so desirous of that which no man may obtain? It is an axiom of nature that natural desire cannot utterly be frustrate. This desire of ours being natural should be frustrate, if that which may satisfy the same were a thing impossible for man to aspire unto. Man doth seek a triple perfection: first a sensual, consisting in those things which very life itself requireth either as necessary supplements, or as beauties and ornaments thereof; then an intellectual, consisting in those things which none underneath man is either capable of or acquainted with; lastly a spiritual and divine, consisting in those things whereunto we tend by supernatural means here, but cannot here attain unto them. They that make the first of these three the scope of their whole life, are said by the Apostle to have no god but only their belly, to be earthly-minded men. Unto the second they bend themselves, who seek especially to excel in all such knowledge and virtue as doth most commend men. To this branch belongeth the law of moral and civil perfection. That there is somewhat higher than either of these two, no other proof doth need than the very process of man’s desire, which being natural should be frustrate, if there were not some farther thing wherein it might rest at the length contented, which in the former it cannot do. For man doth not seem to rest satisfied, either with fruition of that wherewith his life is preserved, or with performance of such actions as advance him most deservedly in estimation; but doth further covet, yea oftentimes manifestly pursue with great sedulity and earnestness, that which cannot stand him in any stead for vital use; that which exceedeth the reach of sense; yea somewhat above capacity of reason, somewhat divine and heavenly, which with hidden exultation it rather surmiseth than conceiveth; somewhat it seeketh, and what that is directly it knoweth not, yet very intentive desire thereof doth so incite it, that all other known delights and pleasures are  laid aside, they give place to the search of this but only suspected desire. If the soul of man did serve only to give him being in this life, then things appertaining unto this life would content him, as we see they do other creatures; which creatures enjoying what they live by seek no further, but in this contentation do shew a kind of acknowledgment that there is no higher good which doth any way belong unto them. With us it is otherwise. For although the beauties, riches, honours, sciences, virtues, and perfections of all men living, were in the present possession of one; yet somewhat beyond and above all this there would still be sought and earnestly thirsted for. So that Nature even in this life doth plainly claim and call for a more divine perfection than either of these two that have been mentioned.

[5.]This last and highest estate of perfection whereof we speak is received of men in the nature of a Reward. Rewards do always presuppose such duties performed as are rewardable. Our natural means therefore unto blessedness are our works; nor is it possible that Nature should ever find any other way to salvation than only this. But examine the works which we do, and since the first foundation of the world what one can say, My ways are pure? Seeing then all flesh is guilty of that for which God hath threatened eternally to punish, what possibility is there this way to be saved? There resteth therefore either no way unto salvation, or if any, then surely a way which is supernatural, a way which could never have entered into the heart of man as much as once to conceive or imagine, if God himself had not revealed it extraordinarily. For which cause we term it the Mystery or secret way of salvation. And therefore St. Ambrose in this matter appealeth justly from man to God, “Cœli mysterium doceat me Deus qui condidit, non homo qui seipsum ignoravit:—Let God himself that made me, let not man that knows not himself, be my instructor concerning the mystical way to heaven.” “When men of excellent wit,” saith Lactantius, “had wholly betaken themselves unto study, after farewell bidden unto all kind as well of private as public action, they spared no labour that might be spent in the  search of truth; holding it a thing of much more price to seek and to find out the reason of all affairs as well divine as human, than to stick fast in the toil of piling up riches and gathering together heaps of honours. Howbeit, they both did fail of their purpose, and got not as much as to quite their charges; because truth which is the secret of the Most High God, whose proper handy-work all things are, cannot be compassed with that wit and those senses which are our own. For God and man should be very near neighbours, if man’s cogitations were able to take a survey of the counsels and appointments of that Majesty everlasting. Which being utterly impossible, that the eye of man by itself should look into the bosom of divine Reason; God did not suffer him being desirous of the light of wisdom to stray any longer up and down, and with bootless expense of travail to wander in darkness that had no passage to get out by. His eyes at the length God did open, and bestow upon him the knowledge of the truth by way of Donative, to the end that man might both be clearly convicted of folly, and being through error out of the way, have the path that leadeth unto immortality laid plain before him.” Thus far Lactantius Firmianus, to shew that God himself is the teacher of the truth, whereby is made known the supernatural way of salvation and law for them to live in that shall be saved. In the natural path of everlasting life the first beginning is that  ability of doing good, which God in the day of man’s creation endued him with; from hence obedience unto the will of his Creator, absolute righteousness and integrity in all his actions; and last of all the justice of God rewarding the worthiness of his deserts with the crown of eternal glory. Had Adam continued in his first estate, this had been the way of life unto him and all his posterity. Wherein I confess notwithstanding with the wittiest of the school-divines, “That if we speak of strict justice, God could no way have been bound to requite man’s labours in so large and ample a manner as human felicity doth import; inasmuch as the dignity of this exceedeth so far the other’s value. But be it that God of his great liberality had determined in lieu of man’s endeavours to bestow the same by the rule of that justice which best beseemeth him, namely, the justice of one that requiteth nothing mincingly, but all with pressed and heaped and even over-enlarged measure; yet could it never hereupon necessarily be gathered, that such justice should add to the nature of that reward the property of everlasting continuance; sith possession of bliss, though it should be but for a moment, were an abundant retribution.” But we are not now to enter into this consideration, how gracious and bountiful our good God might still appear in so rewarding the sons of men, albeit they should exactly perform whatsoever duty their nature bindeth them unto. Howsoever God did propose this reward, we that were to be rewarded must have done that which is required at our hands; we failing in the one, it were in nature an impossibility that the other should be looked for. The light of nature is never able to find out any way of obtaining the reward of bliss, but by performing exactly the duties and works of righteousness.

[6.]From salvation therefore and life all flesh being  excluded this way, behold how the wisdom of God hath revealed a way mystical and supernatural, a way directing unto the same end of life by a course which groundeth itself upon the guiltiness of sin, and through sin desert of condemnation and death. For in this way the first thing is the tender compassion of God respecting us drowned and swallowed up in misery; the next is redemption out of the same by the precious death and merit of a mighty Saviour, which hath witnessed of himself, saying, “I am the way,” the way that leadeth us from misery into bliss. This supernatural way had God in himself prepared before all worlds. The way of supernatural duty which to us he hath prescribed, our Saviour in the Gospel of St. John doth note, terming it by an excellency, The Work of God, “This is the work of God, that ye believe in him whom he hath sent.” Not that God doth require nothing unto happiness at the hands of men saving only a naked belief (for hope and charity we may not exclude); but that without belief all other things are as nothing, and it the ground of those other divine virtues.

Concerning Faith, the principal object whereof is that eternal Verity which hath discovered the treasures of hidden wisdom in Christ; concerning Hope, the highest object whereof is that everlasting Goodness which in Christ doth quicken the dead; concerning Charity, the final object whereof is that incomprehensible Beauty which shineth in the countenance of Christ the Son of the living God: concerning these virtues, the first of which beginning here with a weak apprehension of things not seen, endeth with the intuitive vision of God in the world to come; the second beginning here with a trembling expectation of things far removed and as yet but only heard of, endeth with real and actual fruition of that which no tongue can express; the third beginning here with a weak inclination of heart towards him unto whom we are not able to approach, endeth with endless union, the  mystery whereof is higher than the reach of the thoughts of men; concerning that Faith, Hope, and Charity, without which there can be no salvation, was there ever any mention made saving only in that law which God himself hath from heaven revealed? There is not in the world a syllable muttered with certain truth concerning any of these three, more than hath been supernaturally received from the mouth of the eternal God.

Laws therefore concerning these things are supernatural, both in respect of the manner of delivering them, which is divine; and also in regard of the things delivered, which are such as have not in nature any cause from which they flow, but were by the voluntary appointment of God ordained besides the course of nature, to rectify nature’s obliquity withal.

\section*{The cause why so many natural or rational Laws are set down in Holy Scripture.}

XII. When supernatural duties are necessarily exacted, natural are not rejected as needless. The law of God therefore is, though principally delivered for instruction in the one, yet fraught with precepts of the other also. The Scripture is fraught even with laws of Nature; insomuch that Gratian defining Natural Right, (whereby is meant the right which exacteth those general duties that concern men naturally even as they are men,) termeth “Natural Right, that which the Books of the Law and the Gospel do contain.” Neither is it vain that the Scripture aboundeth with so great store of laws in this kind: for they are either such as we of ourselves could not easily have found out, and then the benefit is not small to have them readily set down to our hands; or if they be so clear and manifest that no man endued with reason can lightly be ignorant of them, yet the Spirit as it were borrowing them from the school of Nature, as serving to prove things less manifest, and to induce a persuasion of somewhat which were in itself more hard and dark, unless it should in such sort be cleared, the very applying of them unto cases particular is not without most singular use and profit many ways for men’s instruction. Besides, be they plain of themselves or obscure, the evidence of God’s own testimony added to the natural assent of reason concerning the certainty of them, doth not a little comfort and confirm the same.


[2.]Wherefore inasmuch as our actions are conversant about things beset with many circumstances, which cause men of sundry wits to be also of sundry judgments concerning that which ought to be done; requisite it cannot but seem the rule of divine law should herein help our imbecility, that we might the more infallibly understand what is good and what evil. The first principles of the Law of Nature are easy; hard it were to find men ignorant of them. But concerning the duty which Nature’s law doth require at the hands of men in a number of things particular, so far hath the natural understanding even of sundry whole nations been darkened, that they have not discerned no not gross iniquity to be sin. Again, being so prone as we are to fawn upon ourselves, and to be ignorant as much as may be of our own deformities, without the feeling sense whereof we are most wretched, even so much the more, because not knowing them we cannot so much as desire to have them taken away: how should our festered sores be cured, but that God hath delivered a law as sharp as the two-edged sword, piercing the very closest and most unsearchable corners of the heart, which the Law of Nature can hardly, human laws by no means possible, reach unto? Hereby we know even secret concupiscence to be sin, and are made fearful to offend though it be but in a wandering cogitation. Finally, of those things which are for direction of all the parts of our life needful, and not impossible to be discerned by the  light of Nature itself; are there not many which few men’s natural capacity, and some which no man’s, hath been able to find out? They are, saith St. Augustine, but a few, and they endued with great ripeness of wit and judgment, free from all such affairs as might trouble their meditations, instructed in the sharpest and the subtlest points of learning, who have, and that very hardly, been able to find out but only the immortality of the soul. The resurrection of the flesh what man did ever at any time dream of, having not heard it otherwise than from the school of Nature? Whereby it appeareth how much we are bound to yield unto our Creator, the Father of all mercy, eternal thanks, for that he hath delivered his law unto the world, a law wherein so many things are laid open, clear, and manifest, as a light which otherwise would have been buried in darkness, not without the hazard, or rather not with the hazard but with the certain loss, of infinite thousands of souls most undoubtedly now saved.

[3.]We see, therefore, that our sovereign good is desired naturally; that God the author of that natural desire had appointed natural means whereby to fulfil it; that man having utterly disabled his nature unto those means hath had other revealed from God, and hath received from heaven a law to teach him how that which is desired naturally must now supernaturally be attained. Finally, we see that because those latter exclude not the former quite and clean as unnecessary, therefore together with such supernatural duties as could not possibly have been otherwise known to the world, the same law that teacheth them, teacheth also with them such natural duties as could not by light of Nature easily have been known.

\section*{The benefit of having divine Laws written.}

XIII. In the first age of the world God gave laws unto our fathers, and by reason of the number of their days their memories served instead of books; whereof the manifold imperfections and defects being known to God, he mercifully relieved the same by often putting them in mind of that whereof it behoved them to be specially mindful. In which respect we see how many times one thing hath been iterated unto sundry even of the best and wisest amongst them. After  that the lives of men were shortened, means more durable to preserve the laws of God from oblivion and corruption grew in use, not without precise direction from God himself. First therefore of Moyses it is said, that he “wrote all the words of God;” not by his own private motion and device: for God taketh this act to himself, “I have written.” Furthermore, were not the Prophets following commanded also to do the like? Unto the holy evangelist St. John, how often express charge is given, “Scribe,” “Write these things.” Concerning the rest of our Lord’s disciples, the words of St. Augustine are, “Quicquid ille de suis factis et dictis nos legere voluit, hoc scribendum illis tanquam suis manibus imperavit.”

[2.]Now, although we do not deny it to be a matter merely accidental unto the law of God to be written; although writing be not that which addeth authority and strength thereunto; finally, though his laws do require at our hands the same obedience howsoever they be delivered; his providence, notwithstanding, which hath made principal choice of this way to deliver them, who seeth not what cause we have to admire and magnify? The singular benefit that hath grown unto the world, by receiving the laws of God even by his own appointment committed unto writing, we are not able to esteem as the value thereof deserveth. When the question therefore is, whether we be now to seek for any revealed law of God otherwhere than only in the sacred Scripture; whether we do now stand bound in the sight of God to yield to traditions urged by the Church of Rome the same obedience and reverence we do to his written law, honouring equally and adoring both as divine: our answer is, No. They that so earnestly plead for the authority of tradition, as if nothing were more safely conveyed than that which spreadeth itself by report, and descendeth by relation of former generations unto the ages that succeed, are not all of them (surely a miracle it were if they should be) so simple as thus to persuade themselves; howsoever, if the simple  were so persuaded, they could be content perhaps very well to enjoy the benefit, as they account it, of that common error. What hazard the truth is in when it passeth through the hands of report, how maimed and deformed it becometh, they are not, they cannot possibly be ignorant. Let them that are indeed of this mind consider but only that little of things divine, which the heathen have in such sort received. How miserable had the state of the Church of God been long ere this, if wanting the sacred Scripture we had no record of his laws, but only the memory of man receiving the same by report and relation from his predecessors?

[3.]By Scripture it hath in the wisdom of God seemed meet to deliver unto the world much but personally expedient to be practised of certain men; many deep and profound points of doctrine, as being the main original ground whereupon the precepts of duty depend; many prophecies, the clear performance whereof might confirm the world in belief of things unseen; many histories to serve as looking-glasses to behold the mercy, the truth, the righteousness of God towards all that faithfully serve, obey, and honour him; yea many entire meditations of piety, to be as patterns and precedents in cases of like nature; many things needful for explication, many for application unto particular occasions, such as the providence of God from time to time hath taken to have the several books of his holy ordinance written. Be it then that together with the principal necessary laws of God there are sundry other things written, whereof we might haply be ignorant and yet be saved: what? shall we hereupon think them needless? shall we esteem them as riotous branches wherewith we sometimes behold most pleasant vines overgrown? Surely no more than we judge our hands or our eyes superfluous, or what part soever, which if our bodies did want, we might notwithstanding any such defect retain still the complete being of men. As therefore a complete  man is neither destitute of any part necessary, and hath some parts whereof though the want could not deprive him of his essence, yet to have them standeth him in singular stead in respect of the special uses for which they serve; in like sort all those writings which contain in them the Law of God, all those venerable books of Scripture, all those sacred tomes and volumes of Holy Writ, they are with such absolute perfection framed, that in them there neither wanteth any thing the lack whereof might deprive us of life, nor any thing in such wise aboundeth, that as being superfluous, unfruitful, and altogether needless, we should think it no loss or danger at all if we did want it.

\section*{The sufficiency of Scripture unto the end for which it was instituted.}

XIV. Although the Scripture of God therefore be stored with infinite variety of matter in all kinds, although it abound with all sorts of laws, yet the principal intent of Scripture is to deliver the laws of duties supernatural. Oftentimes it hath been in very solemn manner disputed, whether all things necessary unto salvation be necessarily set down in the Holy Scriptures or no. If we define that necessary unto salvation, whereby the way to salvation is in any sort made more plain, apparent, and easy to be known; then is there no part of true philosophy, no art of account, no kind of science rightly so called, but the Scripture must contain it. If only those things be necessary, as surely none else are, without the knowledge and practice whereof it is not the will and pleasure of God to make any ordinary grant of salvation; it may be notwithstanding and oftentimes hath been demanded, how the books of Holy Scripture contain in them all necessary things, when of things necessary the very chiefest is to know what books we are bound to esteem holy; which point is confessed impossible for the Scripture itself to teach. Whereunto we may answer with truth, that there is not in the world any art or science, which proposing unto itself an end (as every one doth some end or other) hath been therefore thought defective, if it have not delivered simply whatsoever is needful to the same end; but all kinds of knowledge have their certain bounds and limits; each  of them presupposeth many necessary things learned in other sciences and known beforehand. He that should take upon him to teach men how to be eloquent in pleading causes, must needs deliver unto them whatsoever precepts are requisite unto that end; otherwise he doth not the thing which he taketh upon him. Seeing then no man can plead eloquently unless he be able first to speak; it followeth that ability of speech is in this case a thing most necessary. Notwithstanding every man would think it ridiculous, that he which undertaketh by writing to instruct an orator should therefore deliver all the precepts of grammar; because his profession is to deliver precepts necessary unto eloquent speech, yet so that they which are to receive them be taught beforehand so much of that which is thereunto necessary, as comprehendeth the skill of speaking. In like sort, albeit Scripture do profess to contain in it all things that are necessary unto salvation; yet the meaning cannot be simply of all things which are necessary, but all things that are necessary in some certain kind or form; as all things which are necessary, and either could not at all or could not easily be known by the light of natural discourse; all things which are necessary to be known that we may be saved, but known with presupposal of knowledge concerning certain principles whereof it receiveth us already persuaded, and then instructeth us in all the residue that are necessary. In the number of these principles one is the sacred authority of Scripture. Being therefore persuaded by other means that these Scriptures are the oracles of God, themselves do then teach us the rest, and lay before us all the duties which God requireth at our hands as necessary unto salvation.

[2.]Further, there hath been some doubt likewise, whether containing in Scripture do import express setting down in plain terms, or else comprehending in such sort that by reason we may from thence conclude all things which are necessary. Against the former of these two constructions instance hath sundry ways been given. For our belief in the Trinity, the co-eternity of the Son of God with his Father, the proceeding of the Spirit from the Father and the Son, the duty of baptizing infants: these with such  other principal points, the necessity whereof is by none denied, are notwithstanding in Scripture nowhere to be found by express literal mention, only deduced they are out of Scripture by collection. This kind of comprehension in Scripture being therefore received, still there is doubt how far we are to proceed by collection, before the full and complete measure of things necessary be made up. For let us not think that as long as the world doth endure the wit of man shall be able to sound the bottom of that which may be concluded out of the Scripture; especially if “things contained by collection” do so far extend, as to draw in whatsoever may be at any time out of Scripture but probably and conjecturally surmised. But let necessary collection be made requisite, and we may boldly deny, that of all those things which at this day are with so great necessity urged upon this church under the name of reformed church-discipline, there is any one which their books hitherto have made manifest to be contained in the Scripture. Let them, if they can, allege but one properly belonging to their cause, and not common to them and us, and shew the deduction thereof out of Scripture to be necessary.

[3.]It hath been already shewed, how all things necessary unto salvation in such sort as before we have maintained must needs be possible for men to know; and that many things are in such sort necessary, the knowledge whereof is by the light of Nature impossible to be attained. Whereupon it followeth that either all flesh is excluded from possibility of salvation, which to think were most barbarous; or else that God hath by supernatural means revealed the way of life so far forth as doth suffice. For this cause God hath so many times and ways spoken to the sons of men. Neither hath he by speech only, but by writing also, instructed and taught his Church. The cause of writing hath been to the end that things by him revealed unto the world might have the longer continuance, and the greater certainty of assurance, by how much that which standeth on record hath in both those respects preeminence above that which passeth from hand to hand, and hath no pens but the tongues, no books but the ears of men to record it. The several books of Scripture having had each some several occasion and particular purpose which  caused them to be written, the contents thereof are according to the exigence of that special end whereunto they are intended. Hereupon it groweth that every book of Holy Scripture doth take out of all kinds of truth, natural, historical, foreign, supernatural, so much as the matter handled requireth.

Now forasmuch as there hath been reason alleged sufficient to conclude, that all things necessary unto salvation must be made known, and that God himself hath therefore revealed his will, because otherwise men could not have known so much as is necessary; his surceasing to speak to the world, since the publishing of the Gospel of Jesus Christ and the delivery of the same in writing, is unto us a manifest token that the way of salvation is now sufficiently opened, and that we need no other means for our full instruction than God hath already furnished us withal.

[4.]The main drift of the whole New Testament is that which St. John setteth down as the purpose of his own history; “These things are written, that ye might believe that Jesus is Christ the Son of God, and that in believing ye might have life through his name.” The drift of the Old that which the Apostle mentioneth to Timothy, “The Holy Scriptures are able to make thee wise unto salvation.” So that the general end both of Old and New is one; the difference between them consisting in this, that the Old did make wise by teaching salvation through Christ that should come, the New by teaching that Christ the Saviour is come, and that Jesus whom the Jews did crucify, and whom God did raise again from the dead, is he. When the Apostle therefore affirmeth unto Timothy, that the Old was able to make him wise to salvation, it was not his meaning that the Old alone can do this unto us which live sithence the publication of the New. For he speaketh with presupposal of the doctrine of Christ known also unto Timothy; and therefore first it is said, “Continue thou in those things which thou hast learned and art persuaded, knowing of whom thou hast been taught them.” Again, those Scriptures  he granteth were able to make him wise to salvation; but he addeth, “through the faith which is in Christ.” Wherefore without the doctrine of the New Testament teaching that Christ hath wrought the redemption of the world, which redemption the Old did foreshew he should work, it is not the former alone which can on our behalf perform so much as the Apostle doth avouch, who presupposeth this when he magnifieth that so highly. And as his words concerning the books of ancient Scripture do not take place but with presupposal of the Gospel of Christ embraced; so our own words also, when we extol the complete sufficiency of the whole entire body of the Scripture, must in like sort be understood with this caution, that the benefit of nature’s light be not thought excluded as unnecessary, because the necessity of a diviner light is magnified.

[5.]There is in Scripture therefore no defect, but that any man, what place or calling soever he hold in the Church of God, may have thereby the light of his natural understanding so perfected, that the one being relieved by the other, there can want no part of needful instruction unto any good work which God himself requireth, be it natural or supernatural, belonging simply unto men as men, or unto men as they are united in whatsoever kind of society. It sufficeth therefore that Nature and Scripture do serve in such full sort, that they both jointly, and not severally either of them, be so complete, that unto everlasting felicity we need not the knowledge of any thing more than these two may easily furnish our minds with on all sides; and therefore they which add traditions, as a part of supernatural necessary truth, have not the truth, but are in error. For they only plead, that whatsoever God revealeth as necessary for all  Christian men to do or believe, the same we ought to embrace, whether we have received it by writing or otherwise; which no man denieth: when that which they should confirm, who claim so great reverence unto traditions, is, that the same traditions are necessarily to be acknowledged divine and holy. For we do not reject them only because they are not in the Scripture, but because they are neither in Scripture, nor can otherwise sufficiently by any reason be proved to be of God. That which is of God, and may be evidently proved to be so, we deny not but it hath in his kind, although unwritten, yet the selfsame force and authority with the written laws of God. It is by ours acknowledged, “that the Apostles did in every church institute and ordain some rites and customs serving for the seemliness of church-regiment, which rites and customs they have not committed unto writing.” Those rites and customs being known to be apostolical, and having the nature of things changeable, were no less to be accounted of in the Church than other things of the like degree; that is to say, capable in like sort of alteration, although set down in the Apostles’ writings. For both being known to be apostolical, it is not the manner of delivering them unto the Church, but the author from whom they proceed; which doth give them their force and credit.

\section*{Of Laws positive contained in Scripture, the mutability of certain of them, and the general use of Scripture.}

XV. Laws being imposed either by each man upon himself, or by a public society upon the particulars thereof, or by all the nations of men upon every several society, or by the Lord himself upon any or every of these; there is not amongst these four kinds any one but containeth sundry both natural and positive laws. Impossible it is but that they should fall into a number of gross errors, who only take such laws for positive as have been made or invented of men, and holding this position hold also, that all positive and none but positive laws are mutable. Laws natural do always bind; laws positive not so, but only after they have been expressly and  wittingly imposed. Laws positive there are in every of those kinds before mentioned. As in the first kind the promises which we have passed unto men, and the vows we have made unto God; for these are laws which we tie ourselves unto, and till we have so tied ourselves they bind us not. Laws positive in the second kind are such as the civil constitutions peculiar unto each particular commonweal. In the third kind the law of Heraldry in war is positive: and in the last all the judicials which God gave unto the people of Israel to observe. And although no laws but positive be mutable, yet all are not mutable which be positive. Positive laws are either permanent or else changeable, according as the matter itself is concerning which they were first made. Whether God or man be the maker of them, alteration they so far forth admit, as the matter doth exact.

[2.]Laws that concern supernatural duties are all positive, and either concern men supernaturally as men, or else as parts of a supernatural society, which society we call the Church. To concern men as men supernaturally is to concern them as duties which belong of necessity to all, and yet could not have been known by any to belong unto them, unless God had opened them himself, inasmuch as they do not depend upon any natural ground at all out of which they may be deduced, but are appointed of God to supply the defect of those natural ways of salvation, by which we are not now able to attain thereunto. The Church being a supernatural society doth differ from natural societies in this, that the persons unto whom we associate ourselves, in the one are men simply considered as men, but they to whom we be joined in the other, are God, Angels, and holy men. Again the Church being both a society and a society supernatural, although as it is a society it have the selfsame original  grounds which other politic societies have, namely, the natural inclination which all men have unto sociable life, and consent to some certain bond of association, which bond is the law that appointeth what kind of order they shall be associated in: yet unto the Church as it is a society supernatural this is peculiar, that part of the bond of their association which belong to the Church of God must be a law supernatural, which God himself hath revealed concerning that kind of worship which his people shall do unto him. The substance of the service of God therefore, so far forth as it hath in it any thing more than the Law of Reason doth teach, may not be invented of men, as it is amongst the heathens, but must be received from God himself, as always it hath been in the Church, saving only when the Church hath been forgetful of her duty.

[3.]Wherefore to end with a general rule concerning all the laws which God hath tied men unto: those laws divine that belong, whether naturally or supernaturally, either to men as men, or to men as they live in politic society, or to men as they are of that politic society which is the Church, without any further respect had unto any such variable accident as the state of men and of societies of men and of the Church itself in this world is subject unto; all laws that so belong unto men, they belong for ever, yea although they be Positive Laws, unless being positive God himself which made them alter them. The reason is, because the subject or matter of laws in general is thus far forth constant: which matter is that for the ordering whereof laws were instituted, and being instituted are not changeable without cause, neither can they have cause of change, when that which gave them their first institution remaineth for ever one and the same. On the other side, laws that were made for men or societies or churches, in regard of their being such as they do not always continue, but may perhaps be clean otherwise a while after, and so may require to be otherwise ordered than before; the laws of God himself which are of this nature, no man endued with common sense will ever deny to be of a different constitution from the former, in respect of the one’s  constancy and the mutability of the other. And this doth seem to have been the very cause why St. John doth so peculiarly term the doctrine that teacheth salvation by Jesus Christ, Evangelium æternum, “an eternal Gospel;” because there can be no reason wherefore the publishing thereof should be taken away, and any other instead of it proclaimed, as long as the world doth continue: whereas the whole law of rites and ceremonies, although delivered with so great solemnity, is notwithstanding clean abrogated, inasmuch as it had but temporary cause of God’s ordaining it.

[4.]But that we may at the length conclude this first general introduction unto the nature and original birth, as of all other laws, so likewise of those which the sacred Scripture containeth, concerning the Author whereof even infidels have confessed that He can neither err nor deceive: albeit about things easy and manifest unto all men by common sense there needeth no higher consultation; because as a man whose wisdom is in weighty affairs admired would take it in some disdain to have his counsel solemnly asked about a toy, so the meanness of some things is such, that to search the Scripture of God for the ordering of them were to derogate from the reverend authority and dignity of the Scripture, no less than they do by whom Scriptures are in ordinary talk very idly applied unto vain and childish trifles: yet better it were to be superstitious than profane; to take from thence our direction even in all things great or small, than to wade through matters of principal weight and moment, without ever caring what the law of God hath either for or against our designs. Concerning the custom of the very Painims, thus much Strabe witnesseth: “Men that are civil do lead their lives after one common law appointing them what to do. For that otherwise a multitude should with harmony amongst themselves concur in the doing of one thing, (for this is civilly to live,) or that they should in any sort manage community of life, it is not possible. Now laws or statutes are of two sorts. For they are either received from gods, or else from men.  And our ancient predecessors did surely most honour and reverence that which was from the gods; for which cause consultation with oracles was a thing very usual and frequent in their times.” Did they make so much account of the voice of their gods, which in truth were no gods; and shall we neglect the precious benefit of conference with those oracles of the true and living God, whereof so great store is left to the Church, and whereunto there is so free, so plain, and so easy access for all men? “By thy commandments” (this was David’s confession unto God) “thou hast made me wiser than mine enemies.” Again, “I have had more understanding than all my teachers, because thy testimonies are my meditations.” What pains would not they have bestowed in the study of these books, who travelled sea and land to gain the treasure of some few days’ talk with men whose wisdom the world did make any reckoning of? That little which some of the heathens did chance to hear, concerning such matter as the sacred Scripture plentifully containeth, they did in wonderful sort affect; their speeches as oft as they make mention thereof are strange, and such as themselves could not utter as they did other things, but still acknowledged that their wits, which did every where else conquer hardness, were with profoundness here over-matched. Wherefore seeing that God hath endued us with sense, to the end that we might perceive such things as this present life doth need; and with reason, lest that which sense cannot reach unto, being both now and also in regard of a future estate hereafter necessary to be known, should lie obscure; finally, with the heavenly support of prophetical revelation, which doth open those hidden mysteries that reason could never have been able to find out, or to have known the  necessity of them unto our everlasting good: use we the precious gifts of God unto his glory and honour that gave them, seeking by all means to know what the will of our God is; what righteous before him; in his sight what holy, perfect, and good, that we may truly and faithfully do it.

\section*{A Conclusion, shewing how all this belongeth to the cause in question.}

XVI. Thus far therefore we have endeavoured in part to open, of what nature and force laws are, according unto their several kinds; the law which God with himself hath eternally set down to follow in his own works; the law which he hath made for his creatures to keep; the law of natural and necessary agents; the law which angels in heaven obey; the law whereunto by the light of reason men find themselves bound in that they are men; the law which they make by composition for multitudes and politic societies of men to be guided by; the law which belongeth unto each nation; the law that concerneth the fellowship of all; and lastly the law which God himself hath supernaturally revealed. It might peradventure have been more popular and more plausible to vulgar ears, if this first discourse had been spent in extolling the force of laws, in shewing the great necessity of them when they are good, and in aggravating their offence by whom public laws are injuriously traduced. But forasmuch as with such kind of matter the passions of men are rather stirred one way or other, than their knowledge any way set forward unto the trial of that whereof there is doubt made; I have therefore turned aside from that beaten path, and chosen though a less easy yet a more profitable way in regard of the end we propose. Lest therefore any man should marvel whereunto all these things tend, the drift and purpose of all is this, even to shew in what manner, as every good and perfect gift, so this very gift of good and perfect laws is derived from the Father of lights; to teach men a reason why just and reasonable laws are of so great force, of so great use in the world; and to inform their minds with some method of reducing the laws whereof there is present controversy unto their first original causes, that so it may be in every particular ordinance thereby the better discerned, whether the same be reasonable, just, and righteous, or no. Is there any thing which can either be throughly understood or soundly judged of, till the very first causes and principles from which originally  it springeth be made manifest? If all parts of knowledge have been thought by wise men to be then most orderly delivered and proceeded in, when they are drawn to their first original; seeing that our whole question concerneth the quality of ecclesiastical laws, let it not seem a labour superfluous that in the entrance thereunto all these several kinds of laws have been considered, inasmuch as they all concur as principles, they all have their forcible operations therein, although not all in like apparent and manifest manner. By means whereof it cometh to pass that the force which they have is not observed of many.

[2.]Easier a great deal it is for men by law to be taught what they ought to do, than instructed how to judge as they should do of law: the one being a thing which belongeth generally unto all, the other such as none but the wiser and more judicious sort can perform. Yea, the wisest are always touching this point the readiest to acknowledge, that soundly to judge of a law is the weightiest thing which any man can take upon him. But if we will give judgment of the laws under which we live; first let that law eternal be always before our eyes, as being of principal force and moment to breed in religious minds a dutiful estimation of all laws, the use and benefit whereof we see; because there can be no doubt but that laws apparently good are (as it were) things copied out of the very tables of that high everlasting law; even as the book of that law hath said concerning itself, “By me kings reign, and” by me “princes decree justice.” Not as if men did behold that book and accordingly frame their laws; but because it worketh in them, because it discovereth and (as it were) readeth itself to the world by them, when the laws which they make are righteous. Furthermore, although we perceive not the goodness of laws made, nevertheless sith things in themselves may have that which we peradventure discern not, should not this breed a fear in our hearts, how we speak or judge in the worse part concerning that, the unadvised  disgrace whereof may be no mean dishonour to Him, towards whom we profess all submission and awe? Surely there must be very manifest iniquity in laws, against which we shall be able to justify our contumelious invectives. The chiefest root whereof, when we use them without cause, is ignorance how laws inferior are derived from that supreme or highest law.

[3.]The first that receive impression from thence are natural agents. The law of whose operations might be haply thought less pertinent, when the question is about laws for human actions, but that in those very actions which most spiritually and supernaturally concern men, the rules and axioms of natural operations have their force. What can be more immediate to our salvation than our persuasion concerning the law of Christ towards his Church? What greater assurance of love towards his Church, than the knowledge of that mystical union, whereby the Church is become as near unto Christ as any one part of his flesh is unto other? That the Church being in such sort his he must needs protect it, what proof more strong than if a manifest law so require, which law it is not possible for Christ to violate? And what other law doth the Apostle for this allege, but such as is both common unto Christ with us, and unto us with other things natural; “No man hateth his own flesh, but doth love and cherish it?” The axioms of that law therefore, whereby natural agents are guided, have their use in the moral, yea, even in the spiritual actions of men, and consequently in all laws belonging unto men howsoever.

[4.]Neither are the Angels themselves so far severed from us in their kind and manner of working, but that between the law of their heavenly operations and the actions of men in this our state of mortality such correspondence there is, as maketh it expedient to know in some sort the one, for the other’s more perfect direction. Would Angels acknowledge  themselves “fellow-servants” with the sons of men, but that both having one Lord, there must be some kind of law which is one and the same to both, whereunto their obedience being perfecter is to our weaker both a pattern and a spur? Or would the Apostles, speaking of that which belongeth unto saints as they are linked together in the bond of spiritual society, so often make mention how Angels therewith are delighted, if in things publicly done by the Church we are not somewhat to respect what the Angels of heaven do? Yea, so far hath the Apostle Saint Paul proceeded, as to signify, that even about the outward orders of the Church which serve but for comeliness, some regard is to be had of Angels, who best like us when we are most like unto them in all parts of decent demeanour. So that the law of Angels we cannot judge altogether impertinent unto the affairs of the Church of God.

[5.]Our largeness of speech how men do find out what things reason bindeth them of necessity to observe, and what it guideth them to choose in things which are left as arbitrary; the care we have had to declare the different nature of laws which severally concern all men, from such as belong unto men either civilly or spiritually associated, such as pertain to the fellowship which nations, or which Christian nations, have amongst themselves, and in the last place such as concerning every or any of these God himself hath revealed by his Holy Word: all serveth but to make manifest, that as the actions of men are of sundry distinct kinds, so the laws thereof must accordingly be distinguished. There are in men operations, some natural, some rational, some supernatural, some politic, some finally ecclesiastical: which if we measure not each by his own proper law, whereas the things themselves are so different, there will be in our understanding and judgment of them confusion.

As that first error sheweth, whereon our opposites in this cause have grounded themselves. For as they rightly maintain that God must be glorified in all things, and that the actions of men cannot tend unto his glory unless they be framed after his law; so it is their error to think that the only law which God hath appointed unto men in that behalf  is the sacred Scripture. By that which we work naturally, as when we breathe, sleep, move, we set forth the glory of God as natural agents do, albeit we have no express purpose to make that our end, nor any advised determination therein to follow a law, but do that we do (for the most part) not as much as thinking thereon. In reasonable and moral actions another law taketh place; a law by the observation whereof we glorify God in such sort, as no creature else under man is able to do; because other creatures have not judgment to examine the quality of that which is done by them, and therefore in that they do they neither can accuse nor approve themselves. Men do both, as the Apostle teacheth; yea, those men which have no written law of God to shew what is good or evil, carry written in their hearts the universal law of mankind, the Law of Reason, whereby they judge as by a rule which God hath given unto all men for that purpose. The law of reason doth somewhat direct men how to honour God as their Creator; but how to glorify God in such sort as is required, to the end he may be an everlasting Saviour, this we are taught by divine law, which law both ascertaineth the truth and supplieth unto us the want of that other law. So that in moral actions, divine law helpeth exceedingly the law of reason to guide man’s life; but in supernatural it alone guideth.

Proceed we further; let us place man in some public society with others, whether civil or spiritual; and in this case there is no remedy but we must add yet a further law. For although even here likewise the laws of nature and reason be of necessary use, yet somewhat over and besides them is necessary, namely human and positive law, together with that law which is of commerce between grand societies, the law of nations, and of nations Christian. For which cause the law of God hath likewise said, “Let every soul be subject to the higher powers.” The public power of all societies is above every soul contained in the same societies. And the principal use of that power is to give laws unto all that are under it; which laws in such case we must obey, unless there be reason shewed which may necessarily enforce that the law of Reason or of God doth enjoin the contrary.  Because except our own private and but probable resolutions be by the law of public determinations overruled, we take away all possibility of sociable life in the world. A plainer example whereof than ourselves we cannot have. How cometh it to pass that we are at this present day so rent with mutual contentions, and that the Church is so much troubled about the polity of the Church? No doubt if men had been willing to learn how many laws their actions in this life are subject unto, and what the true force of each law is, all these controversies might have died the very day they were first brought forth.

[6.]It is both commonly said, and truly, that the best men otherwise are not always the best in regard of society. The reason whereof is, for that the law of men’s actions is one, if they be respected only as men; and another, when they are considered as parts of a politic body. Many men there are, than whom nothing is more commendable when they are singled; and yet in society with others none less fit to answer the duties which are looked for at their hands. Yea, I am persuaded, that of them with whom in this cause we strive, there are whose betters amongst men would be hardly found, if they did not live amongst men, but in some wilderness by themselves. The cause of which their disposition so unframable unto societies wherein they live, is, for that they discern not aright what place and force these several kinds of laws ought to have in all their actions. Is there question either concerning the regiment of the Church in general, or about conformity between one church and another, or of ceremonies, offices, powers, jurisdictions in our own church? Of all these things they judge by that rule which they frame to themselves with some show of probability, and what seemeth in that sort convenient, the same they think themselves bound to practise; the same by all means they labour mightily to uphold; whatsoever any law of man to the contrary hath determined they weigh it not. Thus by following the law of private reason, where the law of public should take place, they breed disturbance.

[7.]For the better inuring therefore of men’s minds with  the true distinction of laws, and of their several force according to the different kind and quality of our actions, it shall not peradventure be amiss to shew in some one example how they all take place. To seek no further, let but that be considered, than which there is not any thing more familiar unto us, our food.

What things are food and what are not we judge naturally by sense; neither need we any other law to be our director in that behalf than the selfsame which is common unto us with beasts.

But when we come to consider of food, as of a benefit which God of his bounteous goodness hath provided for all things living; the law of Reason doth here require the duty of thankfulness at our hands, towards him at whose hands we have it. And lest appetite in the use of food should lead us beyond that which is meet, we owe in this case obedience to that law of Reason, which teacheth mediocrity in meats and drinks. The same things divine law teacheth also, as at large we have shewed it doth all parts of moral duty, whereunto we all of necessity stand bound, in regard of the life to come.


But of certain kinds of food the Jews sometime had, and we ourselves likewise have, a mystical, religious, and supernatural use, they of their paschal lamb and oblations, we of our bread and wine in the Eucharist; which use none but divine law could institute.

Now as we live in civil society, the state of the commonwealth wherein we live both may and doth require certain laws concerning food; which laws, saving only that we are members of the commonwealth where they are of force, we should not need to respect as rules of action, whereas now in their place and kind they must be respected and obeyed.

Yea, the selfsame matter is also a subject wherein sometime ecclesiastical laws have place; so that unless we will be  authors of confusion in the Church, our private discretion, which otherwise might guide us a contrary way, must here submit itself to be that way guided, which the public judgment of the Church hath thought better. In which case that of Zonaras concerning fasts may be remembered. “Fastings are good, but let good things be done in good and convenient manner. He that transgresseth in his fasting the orders of the holy fathers,” the positive laws of the Church of Christ, must be plainly told, “that good things do lose the grace of their goodness, when in good sort they are not performed.”

And as here men’s private fancies must give place to the higher judgment of that Church which is in authority a mother over them; so the very actions of whole churches have, in regard of commerce and fellowship with other churches, been subject to laws concerning food, the contrary unto which laws had else been thought more convenient for them to observe; as by that order of abstinence from strangled and blood may appear; an order grounded upon that fellowship which the churches of the Gentiles had with the Jews.

Thus we see how even one and the selfsame thing is under divers considerations conveyed through many laws; and that to measure by any one kind of law all the actions of men were to confound the admirable order, wherein God hath disposed all laws, each as in nature, so in degree, distinct from other.

[8.]Wherefore that here we may briefly end: of Law there can be no less acknowledged, than that her seat is the bosom of God, her voice the harmony of the world: all things in heaven and earth do her homage, the very least as feeling her care, and the greatest as not exempted from her power, both Angels and men and creatures of what condition soever, though each in different sort and manner, yet all with uniform consent, admiring her as the mother of their peace and joy.
