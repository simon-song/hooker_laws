\chapter*[The Second Book]{THE SECOND BOOK. 
CONCERNING THEIR FIRST POSITION WHO URGE REFORMATION IN THE CHURCH OF ENGLAND: NAMELY, THAT SCRIPTURE IS THE ONLY RULE OF ALL THINGS WHICH IN THIS LIFE MAY BE DONE BY MEN.}
\label{chap:book2}
\addcontentsline{toc}{chapter}{THE SECOND BOOK}

THE MATTER CONTAINED IN THIS SECOND BOOK.

I. An answer to their first proof brought out of Scripture, Prov. ii. 9.

II. To their second, 1 Cor. x. 31.

III. To their third, 1 Tim. iv. 5.

IV. To their fourth, Rom. xiv. 23.

V. To their proofs out of Fathers, who dispute negatively from authority of Holy Scripture.

VI. To their proof by the Scripture’s custom of disputing from divine authority negatively.

VII. An examination of their opinion concerning the force of arguments taken from human authority for the ordering of men’s actions and persuasions.

VIII. A declaration what the truth is in this matter.

\PRLsep

\section*{An answer to their first proof brought out of Scripture, Prov. ii. 9.}

AS that which in the title hath been proposed for the matter whereof we treat, is only the ecclesiastical law whereby we are governed; so neither is it my purpose to maintain any other thing than that which therein truth and reason shall approve. For concerning the dealings of men who administer government, and unto whom the execution of that law belongeth; they have their Judge who sitteth in heaven, and before whose tribunal-seat they are accountable for whatsoever abuse or corruption, which (being worthily misliked in this church) the want either of care or of conscience in them hath bred. We are no patrons of those things therefore, the best defence whereof is speedy redress and amendment. That which is of God we defend, to the uttermost of that ability which he hath given; that which is otherwise, let it wither even in the root from whence it hath sprung. Wherefore all these abuses being severed and set apart,  which rise from the corruption of men and not from the laws themselves; come we to those things which in the very whole entire form of our church polity have been (as we persuade ourselves) injuriously blamed by them, who endeavour to overthrow the same, and instead thereof to establish a much worse; only through a strong misconceit they have, that the same is grounded on divine authority.

Now whether it be that through an earnest longing desire to see things brought to a peaceable end, I do but imagine the matters whereof we contend to be fewer than indeed they are; or else for that in truth they are fewer when they come to be discussed by reason, than otherwise they seem when by heat of contention they are divided into many slips, and of every branch an heap is made: surely, as now we have drawn them together, choosing out those things which are requisite to be severally all discussed, and omitting such mean specialties as are likely (without any great labour) to fall afterwards of themselves; I know no cause why either the number or the length of these controversies should diminish our hope of seeing them end with concord and love on all sides; which of his infinite love and goodness the Father of all peace and unity grant.

[2.]Unto which scope that our endeavour may the more directly tend, it seemeth fittest that first those things be examined, which are as seeds from whence the rest that ensue have grown. And of such the most general is that wherewith we are here to make our entrance: a question not moved (I think) any where in other churches, and therefore in ours the more likely to be soon (I trust) determined. The rather, for that it hath grown from no other root, than only a desire to enlarge the necessary use of the Word of God; which desire hath begotten an error enlarging it further than (as we are persuaded) soundness of truth will bear. For whereas God hath left sundry kinds of laws unto men, and by all those laws the actions of men are in some sort directed; they hold that one only law, the Scripture, must be the rule to direct in all things, even so far as to the “taking up of a rush or straw.” About which point there should not need any  question to grow, and that which is grown might presently end, if they did yield but to these two restraints: the first is, not to extend the actions whereof they speak so low as that instance doth import of taking up a straw, but rather keep themselves at the least within the compass of moral actions, actions which have in them vice or virtue: the second, not to exact at our hands for every action the knowledge of some place of Scripture out of which we stand bound to deduce it, as by divers testimonies they seek to enforce; but rather as the truth is, so to acknowledge, that it sufficeth if such actions be framed according to the law of Reason; the general axioms, rules, and principles of which law being so frequent in Holy Scripture, there is no let but in that regard even out of Scripture such duties may be deduced by some kind of consequence, (as by long circuit of deduction it may be that even all truth out of any truth may be concluded,) howbeit no man bound in such sort to deduce all his actions out of Scripture, as if either the place be to him unknown whereon they may be concluded, or the reference unto that place not presently considered of, the action shall in that respect be condemned as unlawful. In this we dissent, and this we are presently to examine.

[3.]In all parts of knowledge rightly so termed things most general are most strong. Thus it must be, inasmuch as the certainty of our persuasion touching particulars dependeth altogether upon the credit of those generalities out of which they grow. Albeit therefore every cause admit not such infallible evidence of proof, as leaveth no possibility of doubt or scruple behind it; yet they who claim the general assent  of the whole world unto that which they teach, and do not fear to give very hard and heavy sentence upon as many as refuse to embrace the same, must have special regard that their first foundations and grounds be more than slender probabilities. This whole question which hath been moved about the kind of church regiment, we could not but for our own resolution’s sake endeavour to unrip and sift; following therein as near as we might the conduct of that judicial method which serveth best for invention of truth. By means whereof, having found this the head theorem of all their discourses, who plead for the change of ecclesiastical government in England, namely, “That the Scripture of God is in such sort the rule of human actions, that simply whatsoever we do and are not by it directed thereunto, the same is sin;” we hold it necessary that the proofs hereof be weighed. Be they of weight sufficient or otherwise, it is not ours to judge and determine; only what difficulties there are which as yet withhold our assent, till we be further and better satisfied, I hope no indifferent amongst them will scorn or refuse to hear.

[4.]First therefore whereas they allege, “That Wisdom” doth teach men “every good way;” and have thereupon inferred that no way is good in any kind of action unless wisdom do by Scripture lead unto it; see they not plainly how they restrain the manifold ways which wisdom hath to teach men by, unto one only way of teaching, which is by Scripture? The bounds of wisdom are large, and within them much is contained. Wisdom was Adam’s instructor in Paradise; wisdom endued the fathers who lived before the law with the knowledge of holy things; by the wisdom of the law of God David attained to excel others in understanding; and Salomon likewise to excel David by the selfsame wisdom of God teaching him many things besides the law. The ways of well-doing are in number even as  many as are the kinds of voluntary actions; so that whatsoever we do in this world and may do it ill, we shew ourselves therein by well-doing to be wise. Now if wisdom did teach men by Scripture not only all the ways that are right and good in some certain kind, according to that of St. Paul concerning the use of Scripture, but did simply without any manner of exception, restraint, or distinction, teach every way of doing well; there is no art, but Scripture should teach it, because every art doth teach the way how to do something or other well. To teach men therefore wisdom professeth, and to teach them every good way; but not every good way by one way of teaching. Whatsoever either men on earth or the Angels of heaven do know, it is as a drop of that unemptiable fountain of wisdom; which wisdom hath diversely imparted her treasures unto the world. As her ways are of sundry kinds, so her manner of teaching is not merely one and the same. Some things she openeth by the sacred books of Scripture; some things by the glorious works of Nature: with some things she inspireth them from above by spiritual influence; in some things she leadeth and traineth them only by worldly experience and practice. We may not so in any one special kind admire her, that we disgrace her in any other; but let all her ways be according unto their place and degree adored.

\section*{To their second, 1 Cor. x. 31.}

II. That “all things be done to the glory of God,” the blessed Apostle (it is true) exhorteth. The glory of God is the admirable excellency of that virtue divine, which being made manifest, causeth men and Angels to extol his greatness,  and in regard thereof to fear him. By “being glorified” it is not meant that he doth receive any augmentation of glory at our hands, but his name we glorify when we testify our acknowledgment of his glory. Which albeit we most effectually do by the virtue of obedience; nevertheless it may be perhaps a question, whether St. Paul did mean that we sin as oft as ever we go about any thing, without an express intent and purpose to obey God therein. He saith of himself, “I do in all things please all men, seeking not mine own commodity but” rather the good “of many, that they may be saved.” Shall it hereupon be thought that St. Paul did not move either hand or foot, but with express intent even thereby to further the common salvation of men? We move, we sleep, we take the cup at the hand of our friend, a number of things we oftentimes do, only to satisfy some natural desire, without present, express, and actual reference unto any commandment of God. Unto his glory even these things are done which we naturally perform, and not only that which morally and spiritually we do. For by every effect proceeding from the most concealed instincts of nature His power is made manifest. But it doth not therefore follow that of necessity we shall sin, unless we expressly intend this in every such particular.

[2.]But be it a thing which requireth no more than only our general presupposed willingness to please God in all things, or be it a matter wherein we cannot so glorify the name of God as we should without an actual intent to do him in that particular some special obedience; yet for any thing there is in this sentence alleged to the contrary, God may be glorified by obedience, and obeyed by performance of his will, and his will be performed with an actual intelligent desire to fulfil that law which maketh known what his will is, although no special clause or sentence of Scripture be in every such action set before men’s eyes to warrant it. For Scripture is not the only law whereby God hath opened his will touching all things that may be done, but there are other kinds of laws which notify the will of God, as in the former book hath been proved at large: nor is there any law of God, whereunto he doth not account our obedience his glory. “Do therefore all  things unto the glory of God (saith the Apostle), be inoffensive both to Jews and Grecians and the Church of God; even as I please all men in all things, not seeking mine own commodity, but many’s, that they may be saved.” In the least thing done disobediently towards God, or offensively against the good of men, whose benefit we ought to seek for as for our own, we plainly shew that we do not acknowledge God to be such as indeed he is, and consequently that we glorify him not. This the blessed Apostle teacheth; but doth any Apostle teach, that we cannot glorify God otherwise, than only in doing what we find that God in Scripture commandeth us to do?

[3.]The churches dispersed amongst the heathen in the east part of the world are by the Apostle St. Peter exhorted to have their “conversation honest amongst the Gentiles, that they which spake evil of them as of evil-doers might by the good works which they should see glorify God in the day of visitation.” As long as that which Christians did was good, and no way subject unto just reproof, their virtuous conversation was a mean to work the heathen’s conversion unto Christ. Seeing therefore this had been a thing altogether impossible, but that infidels themselves did discern, in matters of life and conversation, when believers did well and when otherwise, when they glorified their heavenly Father and when not; it followeth that some things wherein God is glorified may be some other way known than only by the sacred Scripture; of which Scripture the Gentiles being utterly ignorant did notwithstanding judge rightly of the quality of Christian men’s actions. Most certain it is that nothing but only sin doth dishonour God. So that to glorify him in all things is to do nothing whereby the name of God may be blasphemed; nothing whereby the salvation of Jew or Grecian or any in the Church of Christ may be let or hindered; nothing whereby his law is transgressed. But the question is, whether only Scripture do shew whatsoever God is glorified in?

\section*{To their third, 1 Tim. iv. 5.}

III. And though meats and drinks be said to be sanctified by the word of God and by prayer, yet neither is this a  reason sufficient to prove, that by Scripture we must of necessity be directed in every light and common thing which is incident into any part of man’s life. Only it sheweth that unto us the word, that is to say the Gospel of Christ, having not delivered any such difference of things clean and unclean, as the Law of Moses did unto the Jews, there is no cause but that we may use indifferently all things, as long as we do not (like swine) take the benefit of them without a thankful acknowledgment of His liberality and goodness by whose providence they are enjoyed. And therefore the Apostle gave warning beforehand to take heed of such as should enjoin to “abstain from meats, which God hath created to be received with thanksgiving by them which believe and know the truth. For every creature of God is good, and nothing to be refused, if it be received with thanksgiving, because it is sanctified by the Word of God and prayer.” The Gospel, by not making many things unclean, as the Law did, hath sanctified those things generally to all, which particularly each man unto himself must sanctify by a reverend and holy use. Which will hardly be drawn so far as to serve their purpose, who have imagined the Word in such sort to sanctify all things, that neither food can be tasted, nor raiment put on, nor in the world any thing done, but this deed must needs be sin in them which do not first know it appointed unto them by Scripture before they do it.

\section*{To their fourth, Rom. xiv. 23.}

IV. But to come unto that which of all other things in Scripture is most stood upon; that place of St. Paul they say is “of all other most clear, where speaking of those things which are called indifferent, in the end he concludeth, That ‘whatsoever is not of faith is sin.’ But faith is not but in respect of the Word of God. Therefore whatsoever is not done by the Word of God is sin.” Whereunto we answer, that albeit the name of Faith being properly and strictly taken, it must needs have reference unto some uttered word as the object of belief: nevertheless sith the ground of credit is the credibility of things credited; and things are made credible, either by the known condition and quality of  the utterer, or by the manifest likelihood of truth which they have in themselves; hereupon it riseth that whatsoever we are persuaded of, the same we are generally said to believe. In which generality the object of faith may not so narrowly be restrained, as if the same did extend no further than to the only Scriptures of God. “Though,” saith our Saviour, “ye believe not me, believe my works, that ye may know and believe that the Father is in me and I in him.” “The other disciples said unto Thomas, We have seen the Lord;” but his answer unto them was, “Except I see in his hands the print of the nails, and put my finger into them, I will not believe.” Can there be any thing more plain than that which by these two sentences appeareth, namely, that there may be a certain belief grounded upon other assurance than Scripture: any thing more clear, than that we are said not only to believe the things which we know by another’s relation, but even whatsoever we are certainly persuaded of, whether it be by reason or by sense?

[2.]Forasmuch therefore as it is granted that St. Paul doth mean nothing else by Faith, but only “a full persuasion that that which we do is well done;” against which kind of faith or persuasion as St. Paul doth count it sin to enterprise any thing, so likewise “some of the very heathen have taught, as Tully, ‘That nothing ought to be done whereof thou doubtest whether it be right or wrong;’ whereby it appeareth that even those which had no knowledge  of the word of God did see much of the equity of this which the Apostle requireth of a Christian man;” I hope we shall not seem altogether unnecessarily to doubt of the soundness of their opinion, who think simply that nothing but only the word of God can give us assurance in any thing we are to do, and resolve us that we do well. For might not the Jews have been fully persuaded that they did well to think (if they had so thought) that in Christ God the Father was, although the only ground of this their faith had been the wonderful works they saw him do? Might not, yea, did not Thomas fully in the end persuade himself, that he did well to think that body which now was raised to be the same which had been crucified? That which gave Thomas this assurance was his sense; “Thomas, because thou hast seen, thou believest,” saith our Saviour. What Scripture had Tully for this assurance? Yet I nothing doubt but that they who allege him think he did well to set down in writing a thing so consonant unto truth. Finally, we all believe that the Scriptures of God are sacred, and that they have proceeded from God; ourselves we assure that we do right well in so believing. We have for this point a demonstration sound and infallible. But it is not the word of God which doth or possibly can assure us, that we do well to think it his word. For if any one book of Scripture did give testimony to all, yet still that Scripture which giveth credit to the rest would require another Scripture to give credit unto it, neither could we ever come unto any pause whereon to rest our assurance this way; so that unless beside Scripture there were something which might assure us that we do well, we could not think we do well, no not in being assured that Scripture is a sacred and holy rule of well-doing.

[3.]On which determination we might be contented to stay ourselves without further proceeding herein, but that we are drawn on into larger speech by reason of their so great earnestness, who beat more and more upon these last alleged words, as being of all other most pregnant.

Whereas therefore they still argue, “That wheresoever faith is wanting, there is sin;” and, “in every action not  commanded faith is wanting;” ergo, “in every action not commanded, there is sin:” I would demand of them first, forasmuch as the nature of things indifferent is neither to be commanded nor forbidden, but left free and arbitrary; how there can be any thing indifferent, if for want of faith sin be committed when any thing not commanded is done. So that of necessity they must add somewhat, and at leastwise thus set it down: in every action not commanded of God or permitted with approbation, faith is wanting, and for want of faith there is sin.

[4.]The next thing we are to inquire is, What those things be which God permitteth with approbation, and how we may know them to be so permitted. When there are unto one end sundry means; as for example, for the sustenance of our bodies many kinds of food, many sorts of raiment to clothe our nakedness, and so in other things of like condition: here the end itself being necessary, but not so any one mean thereunto; necessary that our bodies should be both fed and clothed, howbeit no one kind of food or raiment necessary; therefore we hold these things free in their own nature and indifferent. The choice is left to our own discretion, except a principal bond of some higher duty remove the indifferency that such things have in themselves. Their indifferency is removed, if either we take away our own liberty, as Ananias did, for whom to have sold or held his possessions it was indifferent, till his solemn vow and promise unto God had strictly bound him one only way; or if God himself have precisely abridged the same, by restraining us unto or by barring us from some one or more things of many, which otherwise were in themselves altogether indifferent. Many fashions of priestly attire there were, whereof Aaron and his sons might have had their free choice without sin, but that God expressly tied them unto one. All meats indifferent unto the Jew, were it not that God by name excepted some, as swine’s flesh. Impossible therefore it is we should otherwise think, than that what things God doth neither command nor forbid, the same he permitteth with approbation either to be done or left undone.  “All things are lawful unto me,” saith the Apostle, speaking as it seemeth in the person of the Christian Gentile for maintenance of liberty in things indifferent; whereunto his answer is, that nevertheless “all things are not expedient;” in things indifferent there is a choice, they are not always equally expedient.

[5.]Now in things although not commanded of God yet lawful because they are permitted, the question is, what light shall shew us the conveniency which one hath above another. For answer, their final determination is, that “Whereas the Heathen did send men for the difference of good and evil to the light of Reason, in such things the Apostle sendeth us to the school of Christ in his word, which only is able through faith to give us assurance and resolution in our doings.” Which word only, is utterly without possibility of ever being proved. For what if it were true concerning things indifferent, that unless the word of the Lord had determined of the free use of them, there could have been no lawful use of them at all: which notwithstanding is untrue; because it is not the Scripture’s setting down such things as indifferent, but their not setting down as necessary, that doth make them to be indifferent: yet this to our present purpose serveth nothing at all. We inquire not now, whether any thing be free to be used which Scripture hath not set down as free: but concerning things known and acknowledged to be indifferent, whether particularly in choosing any one of them before another we sin, if any thing but Scripture direct us in this our choice. When many meats are set before me, all are indifferent, none unlawful, I take one as most convenient. If Scripture require me so to do, then is not the thing indifferent, because I must do what Scripture requireth. They are all indifferent, I might take any, Scripture doth not require of me to make any special choice of one: I do notwithstanding make choice of one, my discretion teaching me so to do. A hard case, that hereupon I should be justly condemned of sin. Nor let any man think that following the judgment of natural discretion in such cases we can have no assurance that we please God. For to the Author and God of our nature, how shall any  operation proceeding in natural sort be in that respect unacceptable? The nature which himself hath given to work by he cannot but be delighted with, when we exercise the same any way without commandment of his to the contrary.

[6.]My desire is to make this cause so manifest, that if it were possible, no doubt or scruple concerning the same might remain in any man’s cogitation. Some truths there are, the verity whereof time doth alter: as it is now true that Christ is risen from the dead; which thing was not true at such time as Christ was living on earth, and had not suffered. It would be known therefore, whether this which they teach concerning the sinful stain of all actions not commanded of God, be a truth that doth now appertain unto us only, or a perpetual truth, in such sort that from the first beginning of the world unto the last consummation thereof, it neither hath been nor can be otherwise. I see not how they can restrain this unto any particular time, how they can think it true now and not always true, that in every action not commanded there is for want of faith sin. Then let them cast back their eyes unto former generations of men, and mark what was done in the prime of the world. Seth, Enoch, Noah, Sem, Abraham, Job, and the rest that lived before any syllable of the law of God was written, did they not sin as much as we do in every action not commanded? That which God is unto us by his sacred word, the same he was unto them by such like means as Eliphaz in Job describeth. If therefore we sin in every action which the Scripture commandeth us not, it followeth that they did the like in all such actions as were not by revelation from Heaven exacted at their hands. Unless God from heaven did by vision still shew them what to do, they might do nothing, not eat, not drink, not sleep, not move.

[7.]Yea, but even as in darkness candlelight may serve to guide men’s steps, which to use in the day were madness; so when God had once delivered his law in writing, it may be they are of opinion that then it must needs be sin for men to do any thing which was not there commanded them to do,  whatsoever they might do before. Let this be granted, and it shall hereupon plainly ensue, either that the light of Scripture once shining in the world, all other light of Nature is therewith in such sort drowned, that now we need it not, neither may we longer use it; or if it stand us in any stead, yet as Aristotle speaketh of men whom Nature hath framed for the state of servitude, saying, “They have reason so far forth as to conceive when others direct them, but little or none in directing themselves by themselves;” so likewise our natural capacity and judgment must serve us only for the right understanding of that which the sacred Scripture teacheth. Had the Prophets who succeeded Moses, or the blessed Apostles which followed them, been settled in this persuasion, never would they have taken so great pains in gathering together natural arguments, thereby to teach the faithful their duties. To use unto them any other motive than Scriptum est, “Thus it is written,” had been to teach them other grounds of their actions than Scripture; which I grant they allege commonly, but not only. Only Scripture they should have alleged, had they been thus persuaded, that so far forth we do sin as we do any thing otherwise directed than by Scripture. St. Augustine was resolute in points of Christianity to credit none, how godly and learned soever he were, unless he confirmed his sentence by the Scriptures, or by some reason not contrary to them. Let them therefore with St. Augustine reject and condemn that which is not grounded either on the Scripture, or on some reason not contrary to Scripture, and we are ready to give them our hands in token of friendly consent with them.

\section*{To their proofs out of Fathers, who dispute negatively from authority of Holy Scripture.}

V. But against this it may be objected, and is, That the Fathers do nothing more usually in their books, than draw  arguments from the Scripture negatively in reproof of that which is evil; “Scriptures teach it not, avoid it therefore:” these disputes with the Fathers are ordinary, neither is it hard to shew that the Prophets themselves have so reasoned. Which arguments being sound and good, it should seem that it cannot be unsound or evil to hold still the same assertion against which hitherto we have disputed. For if it stand with reason thus to argue, “such a thing is not taught us in Scripture, therefore we may not receive or allow it;” how should it seem unreasonable to think, that whatsoever we may lawfully do, the Scripture by commanding it must make it lawful? But how far such arguments do reach, it shall the better appear by considering the matter wherein they have been urged.

[2.]First therefore this we constantly deny, that of so many testimonies as they are able to produce for the strength of negative arguments, any one doth generally (which is the point in question) condemn either all opinions as false, or all actions as unlawful, which the Scripture teacheth us not. The most that can be collected out of them is only that in some cases a negative argument taken from Scripture is strong, whereof no man endued with judgment can doubt. But doth the strength of some negative argument prove this kind of negative argument strong, by force whereof all things are denied which Scripture affirmeth not, or all things which Scripture prescribeth not condemned? The question between us is concerning matter of action, what things are lawful or unlawful for men to do. The sentences alleged out of the Fathers are as peremptory and as large in every respect for matter of opinion as of action: which argueth that in truth they never meant any otherwise to tie the one than the other unto Scripture, both being thereunto equally tied, as far as each is required in the same kind of necessity unto salvation. If therefore it be not unlawful to know and with full persuasion to believe much more than Scripture alone doth teach; if it be against all sense and reason to condemn the knowledge of so many arts and sciences as are otherwise learned than in Holy Scripture, notwithstanding the manifest speeches of ancient Catholic Fathers, which seem to close up within the bosom thereof all manner good and lawful knowledge; wherefore  should their words be thought more effectual to shew that we may not in deeds and practice, than they are to prove that in speculation and knowledge we ought not to go any farther than the Scripture? Which Scripture being given to teach matters of belief no less than of action, the Fathers must needs be and are even as plain against credit besides the relation, as against practice without the injunction of the Scripture.

[3.]St. Augustine hath said, “Whether it be question of Christ, or whether it be question of his Church, or of what thing soever the question be; I say not, if we, but if an angel from heaven shall tell us any thing beside that you have received in the Scripture under the Law and the Gospel, let him be accursed.” In like sort Tertullian, “We may not give ourselves this liberty to bring in any thing of our will, nor choose any thing that other men bring in of their will; we have the Apostles themselves for authors, which themselves brought nothing of their own will, but the discipline which they received of Christ they delivered faithfully unto the people.” In which place the name of Discipline importeth not as they who allege it would fain have it construed, but as any man who noteth the circumstance of the place and the occasion of uttering the words will easily acknowledge, even the selfsame thing it signifieth which the name of Doctrine doth, and as well might the one as the other there have been used. To help them farther, doth not St. Jerome after the selfsame manner dispute, “We believe it  not, because we read it not?” Yea, “We ought not so much as to know the things which the Book of the Law containeth not,” saith St. Hilary. Shall we hereupon then conclude, that we may not take knowledge of or give credit unto any thing, which sense or experience or report or art doth propose, unless we find the same in Scripture? No; it is too plain that so far to extend their speeches is to wrest them against their true intent and meaning. To urge any thing upon the Church, requiring thereunto that religious assent of Christian belief, wherewith the words of the holy prophets are received; to urge any thing as part of that supernatural and celestially revealed truth which God hath taught, and not to shew it in Scripture; this did the ancient Fathers evermore think unlawful, impious, execrable. And thus, as their speeches were meant, so by us they must be restrained.

[4.]As for those alleged words of Cyprian, “The Christian Religion shall find, that out of this Scripture rules of all doctrines have sprung, and that from hence doth spring and hither doth return whatsoever the ecclesiastical discipline doth contain:” surely this place would never have been brought forth in this cause, if it had been but once read over in the author himself out of whom it is cited. For the words are uttered concerning that one principal commandment of love; in the honour whereof he speaketh after this sort: “Surely this commandment containeth the law and  the Prophets, and in this one word is the abridgment of all the volumes of Scripture. This nature and reason and the authority of thy word, O Lord, doth proclaim; this we have heard out of thy mouth; herein the perfection of all religion doth consist. This is the first commandment and the last: this being written in the Book of Life is (as it were) an everlasting lesson both to Men and Angels. Let Christian religion read this one word, and meditate upon this commandment, and out of this Scripture it shall find the rules of all learning to have sprung, and from hence to have risen and hither to return whatsoever the ecclesiastical discipline containeth, and that in all things it is vain and bootless which charity confirmeth not.” Was this a sentence (trow you) of so great force to prove that Scripture is the only rule of all the actions of men? Might they not hereby even as well prove, that one commandment of Scripture is the only rule of all things, and so exclude the rest of the Scripture, as now they do all means beside Scripture? But thus it fareth, when too much desire of contradiction causeth our speech rather to pass by number than to stay for weight.

[5.] Well, but Tertullian doth in this case speak yet more plainly: “The Scripture,” saith he, “denieth what it noteth not;” which are indeed the words of Tertullian. But what? the Scripture reckoneth up the kings of Israel, and amongst those kings David; the Scripture reckoneth up the sons of David, and amongst those sons Salomon. To prove that amongst the kings of Israel there was no David but only one, no Salomon but one in the sons of David; Tertullian’s argument will fitly prove. For inasmuch as the Scripture did propose to reckon up all, if there were more it would have named them. In this case “the Scripture doth deny the thing it noteth not.” Howbeit I could not but think that man to do me some piece of manifest injury, which would hereby fasten upon me a general opinion, as if I did think the Scripture to deny the very reign of King Henry the Eighth, because it nowhere noteth that any such King did reign. Tertullian’s speech is probable concerning such matter as he there speaketh of. “There was,” saith Tertullian, “no second Lamech like to him that had two wives; the Scripture denieth what it noteth not.” As therefore it noteth one such to have been in that age of the world; so had there been more, it would by likelihood as well have noted many as one. What infer we now hereupon? “There was no second Lamech; the Scripture denieth what it noteth not.” Were it consonant unto reason to divorce these two sentences, the former of which doth shew how the later is restrained, and not marking the former to conclude by the later of them, that simply whatsoever any man at this day doth think true is by the Scripture denied, unless it be there affirmed to be true? I wonder that a cause so weak and feeble hath been so much persisted in.

[6.]But to come unto those their sentences wherein matters of action are more apparently touched: the name of Tertullian is as before so here again pretended; who writing unto his wife two books, and exhorting her in the one to live a widow, in case God before her should take him unto his mercy; and in the other, if she did marry, yet not to join herself to an infidel, as in those times some widows Christian had done for the advancement of their estate in this present world, he urged very earnestly St. Paul’s words, “only in the Lord:”  whereupon he demandeth of them that think they may do the contrary, what Scripture they can shew where God hath dispensed and granted license to do against that which the blessed Apostle so strictly doth enjoin. And because in defence it might perhaps be replied, “Seeing God doth will that couples which are married when both are infidels, if either party chance to be after converted unto Christianity, this should not make separation between them, as long as the unconverted was willing to retain the other on whom the grace of Christ had shined; wherefore then should that let the making of marriage, which doth not dissolve marriage being made?” after great reasons shewed why God doth in converts being married allow continuance with infidels, and yet disallow that the faithful when they are free should enter into bonds of wedlock with such, [he] concludeth in the end concerning those women that so marry, “They that please not the Lord do even thereby offend the Lord; they do even thereby throw themselves into evil;” that is to say, while they please him not by marrying in him, they do that whereby they incur his displeasure; they make an offer of themselves into the service of that enemy with whose servants they link themselves in so near a bond. What one syllable is there in all this prejudicial any way to that which we hold? For the words of Tertullian as they are by them alleged are two ways misunderstood; both in the former part, where that is extended generally to “all things” in the neuter gender, which he speaketh in the feminine gender of women’s persons; and in the latter, where “received with hurt” is put instead of “wilful incurring that which is evil.” And so in sum Tertullian doth neither mean nor say as is pretended, “Whatsoever pleaseth not the Lord displeaseth him, and with hurt is received;” but, “Those women that please not the Lord” by their kind of marrying “do even thereby offend the Lord, they do even thereby throw themselves into evil.”

[7.]Somewhat more show there is in a second place of Tertullian, which notwithstanding when we have examined it  will be found as the rest are. The Roman emperor’s custom was at certain solemn times to bestow on his soldiers a donative; which donative they received wearing garlands upon their heads. There were in the time of the emperors Severus and Antoninus many, who being soldiers had been converted unto Christ, and notwithstanding continued still in that military course of life. In which number, one man there was amongst all the rest, who at such a time coming to the tribune of the army to receive his donative, came but with a garland in his hand, and not in such sort as others did. The tribune offended hereat demandeth what this great singularity should mean. To whom the soldier, Christianus sum, “I am a Christian.” Many there were so besides him which yet did otherwise at that time; whereupon grew a question, whether a Christian soldier might herein do as the unchristian did, and wear as they wore. Many of them which were very sound in Christian belief did rather commend the zeal of this man than approve his action.

Tertullian was at the same time a Montanist, and an enemy unto the church for condemning that prophetical spirit which Montanus and his followers did boast they had received, as if in them Christ had performed his last promise; as if to them he had sent the Spirit that should be their perfecter and final instructor in the mysteries of Christian truth. Which exulceration of mind made him apt to take all occasions of contradiction. Wherefore in honour of that action, and to gall their minds who did not so much commend it, he wrote his book De Corona Militis, not dissembling the stomach wherewith  he wrote it. For first, the man he commendeth as “one more constant than the rest of his brethren, who presumed,” saith he, “that they might well enough serve two Lords.” Afterwards choler somewhat more rising with him, he addeth, “It doth even remain that they should also devise how to rid themselves of his martyrdoms, towards the prophecies of whose Holy Spirit they have already shewed their disdain. They mutter that their good and long peace is now in hazard. I doubt not but some of them send the Scriptures before, truss up bag and baggage, make themselves in a readiness that they may fly from city to city. For that is the only point of the Gospel which they are careful not to forget. I know even their pastors very well what men they are; in peace lions, harts in time of trouble and fear.” Now these men, saith Tertullian, “they must be answered, where we do find it written in Scripture that a Christian man may not wear a garland.”

And as men’s speeches uttered in heat of distempered affection have oftentimes much more eagerness than weight, so he that shall mark the proofs alleged and the answers to things objected in that book will now and then perhaps espy the like imbecility. Such is that argument whereby they that wore on their heads garlands are charged as transgressors of nature’s law, and guilty of sacrilege against God the Lord of nature, inasmuch as flowers in such sort worn can neither be smelt nor seen well by those that wear them; and God made flowers sweet and beautiful, that being seen and smelt  unto they might so delight. Neither doth Tertullian bewray this weakness in striking only, but also in repelling their strokes with whom he contendeth. They ask, saith he, “What Scripture is there which doth teach that we should not be crowned? And what Scripture is there which doth teach that we should? For in requiring on the contrary part the aid of Scripture, they do give sentence beforehand that their part ought also by Scripture to be aided.” Which answer is of no great force. There is no necessity, that if I confess I ought not to do that which the Scripture forbiddeth me, I should thereby acknowledge myself bound to do nothing which the Scripture commandeth me not. For many inducements besides Scripture may lead me to that, which if Scripture be against, they all give place and are of no value, yet otherwise are strong and effectual to persuade.

Which thing himself well enough understanding, and being not ignorant that Scripture in many things doth neither command nor forbid, but use silence; his resolution in fine is, that in the church a number of things are strictly observed, whereof no law of Scripture maketh mention one way or other; that of things once received and confirmed by use, long usage is a law sufficient; that in civil affairs, when there is no other law, custom itself doth stand for law; that inasmuch as law doth stand upon reason, to allege reason serveth as well as to cite Scripture; that whatsoever is reasonable, the same is lawful whosoever is author of it; that the authority  of custom is great; finally, that the custom of Christians was then and had been a long time not to wear garlands, and therefore that undoubtedly they did offend who presumed to violate such a custom by not observing that thing, the very inveterate observation whereof was a law sufficient to bind all men to observe it, unless they could shew some higher law, some law of Scripture, to the contrary. This presupposed, it may stand then very well with strength and soundness of reason, even thus to answer, “Whereas they ask what Scripture forbiddeth them to wear a garland; we are in this case rather to demand what Scripture commandeth them. They cannot here allege that it is permitted which is not forbidden them: no, that is forbidden them which is not permitted.” For long-received custom forbidding them to do as they did, (if so be it did forbid them,) there was no excuse in the world to justify their act, unless in the Scripture they could shew some law, that did license them thus to break a received custom.

Now whereas in all the books of Tertullian besides there is not so much found as in that one, to prove not only that we may do, but that we ought to do, sundry things which the Scripture commandeth not; out of that very book these sentences are brought to make us believe that Tertullian was of a clean contrary mind. We cannot therefore hereupon yield; we cannot grant, that hereby is made manifest the argument of Scripture negatively to be of force, not only in doctrine and ecclesiastical discipline, but even in matters arbitrary. For Tertullian doth plainly hold even in that book, that neither the matter which he intreateth of was arbitrary but necessary, inasmuch as the received custom of the Church  did tie and bind them not to wear garlands as the heathens did; yea, and further also he reckoneth up particularly a number of things, whereof he expressly concludeth, “Harum et aliarum ejusmodi disciplinarum si legem expostules Scripturarum, nullam invenies;” which is as much as if he had said in express words, “Many things there are which concern the discipline of the Church and the duties of men, which to abrogate and take away the Scripture negatively urged may not in any case persuade us, but they must be observed, yea, although no Scripture be found which requireth any such thing.” Tertullian therefore undoubtedly doth not in this book shew himself to be of the same mind with them by whom his name is pretended.

\section*{To their proof by the Scripture’s custom of disputing from divine authority negatively.}

VI. But sith the sacred Scriptures themselves afford oftentimes such arguments as are taken from divine authority both one way and other; “The Lord hath commanded, therefore it must be;” and again in like sort, “He hath not, therefore it must not be;” some certainty concerning this point seemeth requisite to be set down.

God himself can neither possibly err, nor lead into error.  For this cause his testimonies, whatsoever he affirmeth, are always truth and most infallible certainty.

Yea further, because the things that proceed from him are perfect without any manner of defect or maim; it cannot be but that the words of his mouth are absolute, and lack nothing which they should have for performance of that thing whereunto they tend. Whereupon it followeth, that the end being known whereunto he directeth his speech, the argument even negatively is evermore strong and forcible concerning those things that are apparently requisite unto the same end. As for example: God intending to set down sundry times that which in Angels is most excellent, hath not any where spoken so highly of them as he hath of our Lord and Saviour Jesus Christ; therefore they are not in dignity equal unto him. It is the Apostle St. Paul’s argument.

[2.]The purpose of God was to teach his people, both unto whom they should offer sacrifice, and what sacrifice was to be offered. To burn their sons in fire unto Baal he did not command them, he spake no such thing, neither came it into his mind; therefore this they ought not to have done. Which argument the Prophet Jeremy useth more than once, as being so effectual and strong, that although the thing he reproveth were not only not commanded but forbidden them, and that expressly; yet the Prophet chooseth rather to charge them with the fault of making a law unto themselves, than with the crime of transgressing a law which God had made. For when the Lord hath once himself precisely set down a form of executing that wherein we are to serve him; the fault appeareth greater to do that which we are not, than not to do that which we are commanded. In this we seem to charge the law of God with hardness only, in that with foolishness; in this we shew ourselves weak and unapt to be doers of his will, in that we take upon us to be controllers of his wisdom; in this we fail to perform the thing which God seeth meet,  convenient, and good, in that we presume to see what is meet and convenient better than God himself. In those actions therefore the whole form whereof God hath of purpose set down to be observed, we may not otherwise do than exactly as he hath prescribed; in such things negative arguments are strong.

[3.]Again, with a negative argument David is pressed concerning the purpose he had to build a temple unto the Lord; “Thus saith the Lord, Thou shalt not build me a house to dwell in. Wheresoever I have walked with all Israel, spake I one word to any of the judges of Israel, whom I commanded to feed my people, saying, Why have ye not built me an house?” The Jews urged with a negative argument touching the aid which they sought at the hands of the King of Egypt; “Woe to those rebellious children, saith the Lord, which walk forth to go down into Egypt, and have not asked counsel at my mouth; to strengthen themselves with the strength of Pharao.” Finally, the league of Joshua with the Gabeonites is likewise with a negative argument touched. It was not as it should be: and why? the Lord gave them not that advice; “They sought not counsel at the mouth of the Lord.”

By the virtue of which examples if any man shall suppose the force of negative arguments approved, when they are taken from Scripture in such sort as we in this question are pressed therewith, they greatly deceive themselves. For unto which of all these was it said that they had done amiss, in purposing to do or in doing any thing at all which “the Scripture” commanded them not? Our question is, Whether all be sin which is done without direction by Scripture, and not, Whether the Israelites did at any time amiss by following their own minds without asking counsel of God. No, it was that people’s singular privilege, a favour which God vouchsafed them above the rest of the world, that in the affairs of their estate which were not determinable one way or other by the Scripture, himself gave them extraordinarily direction and counsel as oft as they sought it at his hands. Thus God did first by speech unto Moses, after by Urim and Thummim unto priests, lastly by dreams and visions unto prophets, from whom in such cases they were to receive the answer of God.


Concerning Josua therefore, thus spake the Lord unto Moses, saying, “He shall stand before Eleazar the priest, who shall ask counsel for him by the judgment of Urim before the Lord;” whereof had Josua been mindful, the fraud of the Gabeonites could not so smoothly have passed unespied till there was no help.

The Jews had prophets to have resolved them from the mouth of God himself whether Egyptian aids should profit them, yea or no; but they thought themselves wise enough, and him unworthy to be of their counsel. In this respect therefore was their reproof though sharp yet just, albeit there had been no charge precisely given them that they should always take heed of Egypt.

But as for David, to think that he did evil in determining to build God a temple, because there was in Scripture no commandment that he should build it, were very injurious: the purpose of his heart was religious and godly, the act most worthy of honour and renown; neither could Nathan choose but admire his virtuous intent, exhort him to go forward, and beseech God to prosper him therein. But God saw the endless troubles which David should be subject unto during the whole time of his regiment, and therefore gave charge to defer so good a work to the days of tranquillity and peace, wherein it might without interruption be performed. David supposed that it could not stand with the duty which he owed unto God, to set himself in a house of cedar-trees, and to behold the ark of the Lord’s covenant unsettled. This opinion the Lord abateth, by causing Nathan to shew him plainly, that it should be no more imputed unto him for a fault than it had been unto the Judges of Israel before him, his case being the same which theirs was, their times not more unquiet than his, not more unfit for such an action.

Wherefore concerning the force of negative arguments so taken from the authority of Scripture as by us they are denied, there is in all this less than nothing.

[4.]And touching that which unto this purpose is borrowed from the controversy sometime handled between M. Harding  and the worthiest divine that Christendom hath bred for the space of some hundreds of years, who being brought up together in one University, it fell out in them which was spoken of two others, “They learned in the same that which in contrary camps they did practise:” of these two the one objecting that with us arguments taken from authority negatively are over common, the Bishop’s answer hereunto is, that “This kind of argument is thought to be good, whensoever proof is taken of God’s word; and is used not only by us, but also by St. Paul, and by many of the Catholic Fathers. St. Paul saith, God said not unto Abraham, ‘In thy seeds all the nations of the earth shall be blessed:’ but, ‘In thy seed, which is Christ:’ and thereof he thought he made a good argument. Likewise, saith Origen, ‘The bread which the Lord gave unto his disciples, saying unto  them, Take and eat, he deferred not, nor commanded to be reserved till the next day.’ Such arguments Origen and other learned Fathers thought to stand for good, whatsoever misliking Master Harding hath found in them. This kind of proof is thought to hold in God’s commandments, for that they be full and perfect: and God hath specially charged us, that we should neither put to them nor take from them; and therefore it seemeth good unto them that have learned of Christ, Unus est Magister vester, Christus, and have heard the voice of God the Father from heaven, Ipsum audite. But unto them that add to the word of God what them listeth, and make God’s will subject unto their will, and break God’s commandments for their own tradition’s sake, unto them it seemeth not good.”

Again, the English Apology alleging the example of the Greeks, how they have neither private masses, nor mangled sacraments, nor purgatories, nor pardons; it pleaseth Master Harding to jest out the matter, to use the help of his wits where strength of truth failed him, and to answer with scoffing at negatives. The Bishop’s defence in this case is, “The ancient learned Fathers having to deal with impudent heretics, that in defence of their errors avouched the judgment of all the old bishops and doctors that had been before them, and the general consent of the primitive and whole universal Church, and that with as good regard of truth and as faithfully as you do now; the better to discover the shameless boldness and nakedness of their doctrine, were oftentimes likewise forced to use the negative, and so to drive the same heretics, as we do you, to prove their affirmatives, which thing to do it was never possible. The ancient father Irenæus thus stayed himself, as we do, by the negative, ‘Hoc neque Prophetæ prædicaverunt, neque Dominus docuit, neque Apostoli tradiderunt;’ ‘This thing neither did the Prophets publish, nor our Lord teach, nor the Apostles deliver.’ By a like negative Chrysostom saith,  ‘This tree neither Paul planted, nor Apollos watered, nor God increased.’ In like sort Leo saith, ‘What needeth it to believe that thing that neither the Law hath taught, nor the Prophets have spoken, nor the Gospel hath preached, nor the Apostles have delivered?’ And again, ‘How are the new devices brought in that our Fathers never knew?’ St. Augustine, having reckoned up a great number of the Bishops of Rome, by a general negative saith thus; ‘In all this order of succession of bishops there is not one bishop found that was a Donatist.’ St. Gregory being himself a Bishop of Rome, and writing against the title of Universal Bishop, saith thus, ‘None of all my predecessors ever consented to use this ungodly title; no Bishop of Rome ever took upon him this name of singularity.’ By such negatives, M. Harding, we reprove the vanity and novelty of your religion; we tell you, none of the catholic ancient learned Fathers either Greek or Latin, ever used either your private mass, or your half communion, or your barbarous unknown prayers. Paul never planted them, Apollos never watered them, God never increased them; they are of yourselves, they are not of God.”

In all this there is not a syllable which any way crosseth us. For concerning arguments negative even taken from human authority, they are here proved to be in some cases very strong and forcible. They are not in our estimation idle reproofs, when the authors of needless innovations are opposed with such negatives as that of Leo, “How are these new devices brought in which our Fathers never knew?” When their grave and reverend superiors do reckon up unto them as Augustine did unto the Donatists, large catalogues of Fathers wondered at for their wisdom, piety, and learning, amongst  whom for so many ages before us no one did ever so think of the Church’s affairs as now the world doth begin to be persuaded; surely by us they are not taught to take exception hereat, because such arguments are negative. Much less when the like are taken from the sacred authority of Scripture, if the matter itself do bear them. For in truth the question is not, whether an argument from Scripture negatively may be good, but whether it be so generally good, that in all actions men may urge it. The Fathers I grant do use very general and large terms, even as Hiero the king did in speaking of Archimedes, “From henceforward, whatsoever Archimedes speaketh, it must be believed.” His meaning was not that Archimedes could simply in nothing be deceived, but that he had in such sort approved his skill, that he seemed worthy of credit for ever after in matters appertaining unto the science he was skilful in. In speaking thus largely it is presumed that men’s speeches will be taken according to the matter whereof they speak. Let any man therefore that carrieth indifferency of judgment peruse the bishop’s speeches, and consider well of those negatives concerning Scripture, which he produceth out of Irenæus, Chrysostom and Leo;  which three are chosen from amongst the residue, because the sentences of the others (even as one of theirs also) do make for defence of negative arguments taken from human authority, and not from divine only.They mention no more restraint in the one than in the other; yet I think themselves will not hereby judge, that the Fathers took both to be strong, without restraint unto any special kind of matter wherein they held such arguments forcible. Nor doth the bishop either say or prove any more, than that an argument in some kinds of matter may be good, although taken negatively from Scripture.

\section*{An examination of their opinion concerning the force of arguments taken from human authority for the ordering of men’s actions and persuasions.}

VII. An earnest desire to draw all things unto the determination of bare and naked Scripture hath caused here much pains to be taken in abating the estimation and credit of man. Which if we labour to maintain as far as truth and reason will bear, let not any think that we travail about a matter not greatly needful. For the scope of all their pleading against man’s authority is, to overthrow such orders, laws, and constitutions in the Church, as depending thereupon if they should therefore be taken away, would peradventure leave neither face nor memory of Church to continue long in the world, the world especially being such as now it is. That which they have in this case spoken I would for brevity’s sake let pass, but that the drift of their speech being so dangerous, their words are not to be neglected.

[2.]Wherefore to say that simply an argument taken from man’s authority doth hold no way, “neither affirmatively nor negatively,” is hard. By a man’s authority we here understand the force which his word hath for the assurance of another’s mind that buildeth upon it; as the Apostle somewhat did upon their report of the house of Chloe; and the  Samaritans in a matter of far greater moment upon the report of a simple woman. For so it is said in St. John’s Gospel, “Many of the Samaritans of that city believed in him for the saying of the woman, which testified, He hath told me all things that ever I did.”

The strength of man’s authority is affirmatively such that the weightiest affairs in the world depend thereon. In judgment and justice are not hereupon proceedings grounded? Saith not the Law that “in the mouth of two or three witnesses every word shall be confirmed?” This the law of God would not say, if there were in a man’s testimony no force at all to prove any thing.

And if it be admitted that in matter of fact there is some credit to be given to the testimony of man, but not in matter of opinion and judgment; we see the contrary both acknowledged and universally practised also throughout the world. The sentences of wise and expert men were never but highly esteemed. Let the title of a man’s right be called in question; are we not bold to rely and build upon the judgment of such as are famous for their skill in the laws of this land? In matter of state the weight many times of some one man’s authority is thought reason sufficient, even to sway over whole nations.

And this not only “with the simpler sort;” but the learneder and wiser we are, the more such arguments in some cases prevail with us. The reason why the simpler sort are moved with authority is the conscience of their own ignorance; whereby it cometh to pass that having learned men in admiration, they rather fear to dislike them than know wherefore they should allow and follow their judgments. Contrariwise with them that are skilful authority is much more strong and forcible; because they only are able to discern how just cause there is why to some men’s authority so much should be attributed. For which cause the name of Hippocrates (no doubt) were more effectual to persuade even such men as Galen himself, than to move a silly empiric. So that the very selfsame argument in this kind which doth but induce the vulgar sort to like, may constrain the wiser to yield. And therefore not orators only with the people, but even the very  profoundest disputers in all faculties have hereby often with the best learned prevailed most.

As for arguments taken from human authority and that negatively; for example sake, if we should think the assembling of the people of God together by the sound of a bell, the presenting of infants at the holy font by such as commonly we call their godfathers, or any other the like received custom, to be impious, because some men of whom we think very reverently have in their books and writings nowhere mentioned or taught that such things should be in the Church; this reasoning were subject unto just reproof, it were but feeble, weak, and unsound. Notwithstanding even negatively an argument from human authority may be strong, as namely thus: The Chronicles of England mention no more than only six kings bearing the name of Edward since the time of the last conquest; therefore it cannot be there should be more. So that if the question be of the authority of a man’s testimony, we cannot simply avouch either that affirmatively it doth not any way hold; or that it hath only force to induce the simpler sort, and not to constrain men of understanding and ripe judgment to yield assent; or that negatively it hath in it no strength at all. For unto every of these the contrary is most plain.

[3.]Neither doth that which is alleged concerning the infirmity of men overthrow or disprove this. Men are blinded with ignorance and error; many things may escape them, and in many things they may be deceived; yea, those things which they do know they may either forget, or upon sundry indirect considerations let pass; and although themselves do not err, yet may they through malice or vanity even of purpose deceive others. Howbeit infinite cases there are wherein all these impediments and lets are so manifestly excluded, that there is no show or colour whereby any such exception may be taken, but that the testimony of man will stand as a ground of infallible assurance. That there is a city of Rome, that Pius Quintus and Gregory the Thirteenth and others have been Popes of Rome, I suppose we are certainly enough persuaded. The ground of our persuasion, who never saw the place nor persons beforenamed, can be nothing but man’s testimony. Will any man here notwithstanding allege those  mentioned human infirmities, as reasons why these things should be mistrusted or doubted of?

Yea, that which is more, utterly to infringe the force and strength of man’s testimony were to shake the very fortress of God’s truth. For whatsoever we believe concerning salvation by Christ, although the Scripture be therein the ground of our belief; yet the authority of man is, if we mark it, the key which openeth the door of entrance into the knowledge of the Scripture. The Scripture could not teach us the things that are of God, unless we did credit men who have taught us that the words of Scripture do signify those things. Some way therefore, notwithstanding man’s infirmity, yet his authority may enforce assent.

[4.]Upon better advice and deliberation so much is perceived, and at the length confest; that arguments taken from the authority of men may not only so far forth as hath been declared, but further also be of some force in “human sciences;” which force be it never so small, doth shew that they are not utterly naught. But in “matters divine” it is still maintained stiffly, that they have no manner force at all. Howbeit, the very selfsame reason, which causeth to yield that they are of some force in the one, will at the length constrain also to acknowledge that they are not in the other altogether unforcible. For if the natural strength of man’s wit may by experience and study attain unto such ripeness in the knowledge of things human, that men in this respect may  presume to build somewhat upon their judgment; what reason have we to think but that even in matters divine, the like wits furnished with necessary helps, exercised in Scripture with like diligence, and assisted with the grace of Almighty God, may grow unto so much perfection of knowledge, that men shall have just cause, when any thing pertinent unto faith and religion is doubted of, the more willingly to incline their minds towards that which the sentence of so grave, wise, and learned in that faculty shall judge most sound? For the controversy is of the weight of such men’s judgments. Let it therefore be suspected; let it be taken as gross, corrupt, repugnant unto the truth, whatsoever concerning things divine above nature shall at any time be spoken as out of the mouths of mere natural men, which have not the eyes wherewith heavenly things are discerned. For this we contend not. But whom God hath endued with principal gifts to aspire unto knowledge by; whose exercises, labours, and divine studies he hath so blessed that the world for their great and rare skill that way hath them in singular admiration; may we reject even their judgment likewise, as being utterly of no moment? For mine own part, I dare not so lightly esteem of the Church, and of the principal pillars therein.

[5.]The truth is, that the mind of man desireth evermore to know the truth according to the most infallible certainty which the nature of things can yield. The greatest assurance generally with all men is that which we have by plain aspect and intuitive beholding. Where we cannot attain unto this, there what appeareth to be true by strong and invincible demonstration, such as wherein it is not by any way possible to be deceived, thereunto the mind doth necessarily assent, neither is it in the choice thereof to do otherwise. And in case these both do fail, then which way greatest probability leadeth, thither the mind doth evermore incline. Scripture with Christian men being received as the Word of God; that for which we have probable, yea, that which we have necessary reason for, yea, that which we see with our eyes, is not thought so sure as that which the Scripture of God teacheth; because we hold that his speech revealeth there what himself seeth, and therefore the strongest proof of all, and the most necessarily assented unto by us (which do thus receive the Scripture) is  the Scripture. Now it is not required or can be exacted at our hands, that we should yield unto any thing other assent, than such as doth answer the evidence which is to be had of that we assent unto. For which cause even in matters divine, concerning some things we may lawfully doubt and suspend our judgment, inclining neither to one side nor other; as namely touching the time of the fall both of man and angels: of some things we may very well retain an opinion that they are probable and not unlikely to be true, as when we hold that men have their souls rather by creation than propagation, or that the Mother of our Lord lived always in the state of virginity as well after his birth as before (for of these two the one, her virginity before, is a thing which of necessity we must believe; the other, her continuance in the same state always, hath more likelihood of truth than the contrary); finally in all things then are our consciences best resolved, and in most agreeable sort unto God and nature settled, when they are so far persuaded as those grounds of persuasion which are to be had will bear.

Which thing I do so much the rather set down, for that I see how a number of souls are for want of right information in this point oftentimes grievously vexed. When bare and unbuilded conclusions are put into their minds, they finding not themselves to have thereof any great certainty, imagine that this proceedeth only from lack of faith, and that the Spirit of God doth not work in them as it doth in true believers; by this means their hearts are much troubled, they fall into anguish and perplexity: whereas the truth is, that how bold and confident soever we may be in words, when it cometh to the point of trial, such as the evidence is which the truth hath either in itself or through proof, such is the heart’s assent thereunto; neither can it be stronger, being grounded as it should be.

I grant that proof derived from the authority of man’s judgment is not able to work that assurance which doth grow by a stronger proof; and therefore although ten thousand general councils would set down one and the same definitive sentence concerning any point of religion whatsoever, yet one demonstrative reason alleged, or one manifest testimony cited from the mouth of God himself to the contrary, could not  choose but overweigh them all; inasmuch as for them to have been deceived it is not impossible; it is, that demonstrative reason or testimony divine should deceive. Howbeit in defect of proof infallible, because the mind doth rather follow probable persuasions than approve the things that have in them no likelihood of truth at all; surely if a question concerning matter of doctrine were proposed, and on the one side no kind of proof appearing, there should on the other be alleged and shewed that so a number of the learnedest divines in the world have ever thought; although it did not appear what reason or what Scripture led them to be of that judgment, yet to their very bare judgment somewhat a reasonable man would attribute, notwithstanding the common imbecilities which are incident into our nature.

[6.]And whereas it is thought, that especially with “the Church, and those that are called and persuaded of the authority of the Word of God, man’s authority” with them especially “should not prevail;” it must and doth prevail even with them, yea with them especially, as far as equity requireth; and farther we maintain it not. For men to be  tied and led by authority, as it were with a kind of captivity of judgment, and though there be reason to the contrary not to listen unto it, but to follow like beasts the first in the herd, they know not nor care not whither, this were brutish. Again, that authority of men should prevail with men either against or above Reason, is no part of our belief. “Companies of learned men” be they never so great and reverend, are to yield unto Reason; the weight whereof is no whit prejudiced by the simplicity of his person which doth allege it, but being found to be sound and good, the bare opinion of men to the contrary must of necessity stoop and give place.

Irenæus, writing against Marcion, which held one God author of the Old Testament and another of the New, to prove that the Apostles preached the same God which was known before to the Jews, he copiously allegeth sundry their sermons and speeches uttered concerning that matter and recorded in Scripture. And lest any should be wearied with such store of allegations, in the end he concludeth, “While we labour for these demonstrations out of Scripture, and do summarily declare the things which many ways have been  spoken, be contented quietly to hear, and do not think my speech tedious: Quoniam ostensiones quæ sunt in Scripturis non possunt ostendi nisi ex ipsis Scripturis; Because demonstrations that are in Scripture may not otherwise be shewed than by citing them out of the Scriptures themselves where they are.” Which words make so little unto the purpose, that they seem as it were offended at him which hath called them thus solemnly forth to say nothing.

And concerning the verdict of Jerome; if no man, be he never so well learned, have after the Apostles any authority to publish new doctrine as from heaven, and to require the world’s assent as unto truth received by prophetical revelation; doth this prejudice the credit of learned men’s judgments in opening that truth, which by being conversant in the Apostles’ writings they have themselves from thence learned?

St. Augustine exhorteth not to hear men, but to hearken what God speaketh. His purpose is not (I think) that we should stop our ears against his own exhortation, and therefore he cannot mean simply that audience should altogether be denied unto men, but either that if men speak one thing and God himself teach another, then he not they to be obeyed; or if they both speak the same thing, yet then also man’s speech unworthy of hearing, not simply, but in comparison of that which proceedeth from the mouth of God.

“Yea, but we doubt what the will of God is.” Are we in this case forbidden to hear what men of judgment think it to be? If not, then this allegation also might very well have been spared.

In that ancient strife which was between the catholic Fathers and Arians, Donatists, and others of like perverse and froward disposition, as long as to Fathers or councils alleged on the one side the like by the contrary side were opposed, impossible it was that ever the question should by this means grow unto any issue or end. The Scripture they both believed: the Scripture they knew could not give  sentence on both sides; by Scripture the controversy between them was such as might be determined. In this case what madness was it with such kinds of proofs to nourish their contention, when there were such effectual means to end all controversy that was between them! Hereby therefore it doth not as yet appear, that an argument of authority of man affirmatively is in matters divine nothing worth.

Which opinion being once inserted into the minds of the vulgar sort, what it may grow unto God knoweth. Thus much we see, it hath already made thousands so headstrong even in gross and palpable errors, that a man whose capacity will scarce serve him to utter five words in sensible manner blusheth not in any doubt concerning matter of Scripture to think his own bare Yea as good as the Nay of all the wise, grave, and learned judgments that are in the whole world: which insolency must be repressed, or it will be the very bane of Christian religion.

[7.]Our Lord’s disciples marking what speech he uttered unto them, and at the same time calling to mind a common opinion held by the Scribes, between which opinion and the words of their Master it seemed unto them that there was some contradiction, which they could not themselves answer with full satisfaction of their own minds; the doubt they propose to our Saviour, saying, “Why then say the Scribes that Elias must first come?” They knew that the Scribes did err greatly, and that many ways even in matters of their own profession. They notwithstanding thought the judgment of the very Scribes in matters divine to be of some value; some probability they thought there was that Elias should come, inasmuch as the Scribes said it. Now no truth can contradict any truth; desirous therefore they were to be taught how both might stand together; that which they knew could not be false, because Christ spake it; and this which to them did seem true, only because the Scribes had said it. For the Scripture, from whence the Scribes did gather it, was not then in their heads. We do not find that our Saviour reproved them of error, for thinking the judgment of the Scribes to be worth the objecting, for esteeming it to be of any moment or value in matters concerning God.


[8.]We cannot therefore be persuaded that the will of God is, we should so far reject the authority of men as to reckon it nothing. No, it may be a question, whether they that urge us unto this be themselves so persuaded indeed. Men do sometimes bewray that by deeds, which to confess they are hardly drawn. Mark then if this be not general with all men for the most part. When the judgments of learned men are alleged against them, what do they but either elevate their credit, or oppose unto them the judgments of others as learned? Which thing doth argue that all men acknowledge in them some force and weight, for which they are loath the cause they maintain should be so much weakened as their testimony is available. Again, what reason is there why alleging testimonies as proofs, men give them some title of credit, honour, and estimation, whom they allege, unless beforehand it be sufficiently known who they are; what reason hereof but only a common ingrafted persuasion, that in some men there may be found such qualities as are able to countervail those exceptions which might be taken against them, and that such men’s authority is not lightly to be shaken off?

[9.]Shall I add further, that the force of arguments drawn from the authority of Scripture itself, as Scriptures commonly are alleged, shall (being sifted) be found to depend upon the strength of this so much despised and debased authority of man? Surely it doth, and that oftener than we are aware of. For although Scripture be of God, and therefore the proof which is taken from thence must needs be of all other most invincible; yet this strength it hath not, unless it avouch the selfsame thing for which it is brought. If there be either undeniable appearance that so it doth, or reason such as cannot deceive, then Scripture-proof (no  doubt) in strength and value exceedeth all.But for the most part, even such as are readiest to cite for one thing five hundred sentences of holy Scripture; what warrant have they, that any one of them doth mean the thing for which it is alleged? Is not their surest ground most commonly, either some probable conjecture of their own, or the judgment of others taking those Scriptures as they do? Which notwithstanding to mean otherwise than they take them, it is not still altogether impossible. So that now and then they ground themselves on human authority, even when they most pretend divine. Thus it fareth even clean throughout the whole controversy about that discipline which is so earnestly urged and laboured for. Scriptures are plentifully alleged to prove that the whole Christian world for ever ought to embrace it. Hereupon men term it The discipline of God. Howbeit examine, sift and resolve their alleged proofs, till you come to the very root from whence they spring, the heart wherein their strength lieth; and it shall clearly appear unto any man of judgment, that the most which can be inferred upon such plenty of divine testimonies is only this, That some things which they maintain, as far as some men can probably conjecture, do seem to have been out of Scripture not absurdly gathered. Is this a warrant sufficient for any man’s conscience to build such proceedings upon, as have been and are put in ure for the stablishment of that cause?

[10.]But to conclude, I would gladly understand how it cometh to pass, that they which so peremptorily do maintain that human authority is nothing worth are in the cause which they favour so careful to have the common sort of men persuaded, that the wisest, the godliest and the best learned in all Christendom are that way given, seeing they judge this to make nothing in the world for them. Again how cometh it to pass they cannot abide that authority should be alleged on the other side, if there be no force at all in authorities on one side or other? Wherefore labour they to strip their adversaries of such furniture as doth not help? Why take they such needless pains to furnish also their own cause with the like? If it be void and to no purpose that the names of men are so frequent in their books,  what did move them to bring them in, or doth to suffer them there remaining? Ignorant I am not how this is salved, “They do it not but after the truth made manifest first by reason or by Scripture: they do it not but to control the enemies of the truth, who bear themselves bold upon human authority making not for them but against them rather.” Which answers are nothing: for in what place or upon what consideration soever it be they do it, were it in their own opinion of no force being done, they would undoubtedly refrain to do it.

\section*{A declaration what the truth is in this matter.}

VIII. But to the end it may more plainly appear what we are to judge of their sentences, and of the cause itself wherein they are alleged: first it may not well be denied, that all actions of men endued with the use of reason are generally either good or evil. For although it be granted that no action is properly termed good or evil unless it be voluntary; yet this can be no let to our former assertion, That all actions of men endued with the use of reason are generally either good or evil; because even those things are done voluntarily by us which other creatures do naturally, inasmuch as we might stay our doing of them if we would. Beasts naturally do take their food and rest when it offereth itself unto them. If men did so too, and could not do otherwise of themselves, there were no place for any such reproof as that of our Saviour Christ unto his disciples, “Could ye not watch with me one hour?” That which is voluntarily performed in things tending to the end, if it be well done, must needs be done with deliberate consideration of some reasonable cause wherefore we rather should do it than not. Whereupon it seemeth, that in such actions only those are said to be good or evil which are capable of deliberation: so that many things being hourly done by men, wherein they need not use with themselves any manner of consultation at all, it may perhaps hereby seem that well or ill-doing belongeth only to our  weightier affairs, and to those deeds which are of so great importance that they require advice.But thus to determine were perilous, and peradventure unsound also. I do rather incline to think, that seeing all the unforced actions of men are voluntary, and all voluntary actions tending to the end have choice, and all choice presupposeth the knowledge of some cause wherefore we make it: where the reasonable cause of such actions so readily offereth itself that it needeth not to be sought for; in those things though we do not deliberate, yet they are of their nature apt to be deliberated on, in regard of the will, which may incline either way, and would not any one way bend itself, if there were not some apparent motive to lead it. Deliberation actual we use, when there is doubt what we should incline our wills unto. Where no doubt is, deliberation is not excluded as impertinent unto the thing, but as needless in regard of the agent, which seeth already what to resolve upon. It hath no apparent absurdity therefore in it to think, that all actions of men endued with the use of reason are generally either good or evil.

[2.]Whatsoever is good, the same is also approved of God: and according unto the sundry degrees of goodness, the kinds of divine approbation are in like sort multiplied. Some things are good, yet in so mean a degree of goodness, that men are only not disproved nor disallowed of God for them. “No man hateth his own flesh.” “If ye do good unto them that do so to you, the very publicans themselves do as much.” “They are worse than infidels that have no care to provide for their own.” In actions of this sort, the very light of Nature alone may discover that which is so far forth in the sight of God allowable.

[3.]Some things in such sort are allowed, that they be also required as necessary unto salvation, by way of direct immediate and proper necessity final; so that without performance of them we cannot by ordinary course be saved, nor by any means be excluded from life observing them. In actions of this kind our chiefest direction is from Scripture, for Nature is no sufficient teacher what we should do that  we may attain unto life everlasting.The unsufficiency of the light of Nature is by the light of Scripture so fully and so perfectly herein supplied, that further light than this hath added there doth not need unto that end.

[4.]Finally some things, although not so required of necessity that to leave them undone excludeth from salvation, are notwithstanding of so great dignity and acceptation with God, that most ample reward in heaven is laid up for them. Hereof we have no commandment either in Nature or Scripture which doth exact them at our hands; yet those motives there are in both which draw most effectually our minds unto them. In this kind there is not the least action but it doth somewhat make to the accessory augmentation of our bliss. For which cause our Saviour doth plainly witness, that there shall not be as much as a cup of cold water bestowed for his sake without reward. Hereupon dependeth whatsoever difference there is between the states of saints in glory; hither we refer whatsoever belongeth unto the highest perfection of man by way of service towards God; hereunto that fervour and first love of Christians did bend itself, causing them to sell their possessions, and lay down the price at the blessed Apostles’ feet. Hereat St. Paul undoubtedly did aim in so far abridging his own liberty, and exceeding that which the bond of necessary and enjoined duty tied him unto.

[5.]Wherefore seeing that in all these several kinds of actions there can be nothing possibly evil which God approveth; and that he approveth much more than he doth command; and that his very commandments in some kind,  as namely his precepts comprehended in the law of nature, may be otherwise known than only by Scripture; and that to do them, howsoever we know them, must needs be acceptable in his sight: let them with whom we have hitherto disputed consider well, how it can stand with reason to make the bare mandate of sacred Scripture the only rule of all good and evil in the actions of mortal men. The testimonies of God are true, the testimonies of God are perfect, the testimonies of God are all sufficient unto that end for which they were given. Therefore accordingly we do receive them, we do not think that in them God hath omitted any thing needful unto his purpose, and left his intent to be accomplished by our devisings. What the Scripture purposeth, the same in all points it doth perform.

Howbeit that here we swerve not in judgment, one thing especially we must observe, namely that the absolute perfection of Scripture is seen by relation unto that end whereto it tendeth. And even hereby it cometh to pass, that first such as imagine the general and main drift of the body of sacred Scripture not to be so large as it is, nor that God did  thereby intend to deliver, as in truth he doth, a full instruction in all things unto salvation necessary, the knowledge whereof man by nature could not otherwise in this life attain unto: they are by this very mean induced either still to look for new revelations from heaven, or else dangerously to add to the word of God uncertain tradition, that so the doctrine of man’s salvation may be complete; which doctrine, we constantly hold in all respects without any such thing added to be so complete, that we utterly refuse as much as once to acquaint ourselves with any thing further. Whatsoever to make up the doctrine of man’s salvation is added, as in supply of the Scripture’s unsufficiency, we reject it. Scripture purposing this, hath perfectly and fully done it.

Again the scope and purpose of God in delivering the Holy Scripture such as do take more largely than behoveth, they on the contrary side, racking and stretching it further than by him was meant, are drawn into sundry as great inconveniences. These pretending the Scripture’s perfection infer thereupon, that in Scripture all things lawful to be done must needs be contained. We count those things perfect which want nothing requisite for the end whereto they were instituted. As therefore God created every part and particle of man exactly perfect, that is to say in all points sufficient unto that use for which he appointed it; so the Scripture, yea, every sentence thereof, is perfect, and wanteth nothing requisite unto that purpose for which God delivered the same. So that if hereupon we conclude, that because the Scripture is perfect, therefore all things lawful to be done are comprehended in the Scripture; we may even as well conclude so of every sentence, as of the whole sum and body thereof, unless we first of all prove that it was the drift, scope, and purpose of Almighty God in Holy Scripture to comprise all things which man may practise.

[6.]But admit this, and mark, I beseech you, what would follow. God in delivering Scripture to his Church should clean have abrogated amongst them the law of nature; which is an infallible knowledge imprinted in the minds of all the children of men, whereby both general principles for directing of human actions are comprehended, and conclusions derived from them; upon which conclusions groweth in particularity  the choice of good and evil in the daily affairs of this life. Admit this, and what shall the Scripture be but a snare and a torment to weak consciences, filling them with infinite perplexities, scrupulosities, doubts insoluble, and extreme despairs? Not that the Scripture itself doth cause any such thing, (for it tendeth to the clean contrary, and the fruit thereof is resolute assurance and certainty in that it teacheth,) but the necessities of this life urging men to do that which the light of nature, common discretion and judgment of itself directeth them unto; on the other side, this doctrine teaching them that so to do were to sin against their own souls, and that they put forth their hands to iniquity whatsoever they go about and have not first the sacred Scripture of God for direction; how can it choose but bring the simple a thousand times to their wits’ end? how can it choose but vex and amaze them? For in every action of common life to find out some sentence clearly and infallibly setting before our eyes what we ought to do, (seem we in Scripture never so expert,) would trouble us more than we are aware. In weak and tender minds we little know what misery this strict opinion would breed, besides the stops it would make in the whole course of all men’s lives and actions. Make all things sin which we do by direction of nature’s light, and by the rule of common discretion, without thinking at all upon Scripture; admit this position, and parents shall cause their children to sin, as oft as they cause them to do any thing, before they come to years of capacity and be ripe for knowledge in the Scripture: admit this, and it shall not be with masters as it was with him in the Gospel, but servants being commanded to go shall stand still, till they have their errand warranted unto them by Scripture. Which as it standeth with Christian duty in some cases, so in common affairs to require it were most unfit.

[7.]Two opinions therefore there are concerning sufficiency of Holy Scripture, each extremely opposite unto the other, and both repugnant unto truth. The schools of Rome teach  Scripture to be so unsufficient, as if, except traditions were added, it did not contain all revealed and supernatural truth, which absolutely is necessary for the children of men in this life to know that they may in the next be saved. Others justly condemning this opinion grow likewise unto a dangerous extremity, as if Scripture did not only contain all things in that kind necessary, but all things simply, and in such sort that to do any thing according to any other law were not only unnecessary but even opposite unto salvation, unlawful and sinful. Whatsoever is spoken of God or things appertaining to God otherwise than as the truth is, though it seem an honour it is an injury. And as incredible praises given unto men do often abate and impair the credit of their deserved commendation; so we must likewise take great heed, lest in attributing unto Scripture more than it can have, the incredibility of that do cause even those things which indeed it hath most abundantly to be less reverently esteemed. I therefore leave it to themselves to consider, whether they have in this first point or not overshot themselves; which God doth know is quickly done, even when our meaning is most sincere, as I am verily persuaded theirs in this case was.

