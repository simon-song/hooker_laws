\chapter*[The Seventh Book]{THE SEVENTH BOOK. 
THEIR SIXTH ASSERTION, THAT THERE OUGHT NOT TO BE IN THE CHURCH, BISHOPS ENDUED WITH SUCH AUTHORITY AND HONOUR AS OURS ARE.}
\label{chap:book7}
\addcontentsline{toc}{chapter}{THE SEVENTH BOOK}

THE MATTER CONTAINED IN THIS SEVENTH BOOK.

I. The state of Bishops, although some time oppugned, and that by such as therein would most seem to please God, yet by his providence upheld hitherto, whose glory it is to maintain that whereof himself is the author.

II. What a Bishop is, what his name doth import, and what doth belong unto his office as he is a Bishop.

III. In Bishops two things traduced; of which two, the one their authority; and in it the first thing condemned, their superiority over other ministers: what kind of superiority in ministers it is which the one part holdeth, and the other denieth lawful.

IV. From whence it hath grown that the Church is governed by Bishops.

V. The time and cause of instituting every where Bishops with restraint.

VI. What manner of power Bishops from the first beginning have had.

VII. After what sort Bishops, together with presbyters, have used to govern the churches which were under them.

VIII. How far the power of Bishops hath reached from the beginning in respect of territory, or local compass.

IX. In what respects episcopal regiment hath been gainsaid of old by Aërius.

X. In what respect episcopal regiment is gainsaid by the authors of pretended reformation at this day.

XI. Their arguments in disgrace of regiment by Bishops, as being a mere invention of man, and not found in Scripture, answered.

XII. Their arguments to prove there was no necessity of instituting Bishops in the Church.

XIII. The fore-alleged arguments answered.

XIV. An answer unto those things which are objected concerning the difference between that power which Bishops now have, and that which ancient Bishops had more than other presbyters.

XV. Concerning the civil power and authority which our Bishops have.

XVI. The arguments answered, whereby they would prove that the law of God, and the judgment of the best in all ages condemneth the ruling superiority of one minister over another. 

XVII. The second malicious thing wherein the state of Bishops suffereth obloquy, is their honour.

XVIII. What good doth publicly grow from the Prelacy.

XIX. What kinds of honour be due unto Bishops.

XX. Honour in Title, Place, Ornament, Attendance, and Privilege.

XXI. Honour by Endowment with Lands and Livings.

XXII. That of ecclesiastical Goods, and consequently of the Lands and Livings which Bishops enjoy, the propriety belongs unto God alone.

XXIII. That ecclesiastical persons are receivers of God’s rents, and that the honour of Prelates is to be thereof his chief receivers, not without liberty from him granted of converting the same unto their own use, even in large manner.

XXIV. That for their unworthiness to deprive both them and their successors of such goods, and to convey the same unto men of secular callings, now [were?] extreme sacrilegious injustice.

\PRLsep

\section*{The state of Bishops although sometime oppugned, and that by such as therein would most seem to please God, yet by his providence upheld hitherto, whose glory it is to maintain that whereof himself is the author.} 

I. I HAVE heard that a famous kingdom in the world being solicited to reform such disorders as all men saw the Church exceedingly burdened with, when of each degree great multitudes thereunto inclined, and the number of them did every day so increase that this intended work was likely to take no other effect than all good men did wish and labour for;a principal actor herein (for zeal and boldness of spirit) thought it good to shew them betimes what it was which must be effected, or else that there could be no work of perfect reformation accomplished. To this purpose, in a solemn sermon, and in a great assembly, he described unto them the present quality of their public estate by the parable of a tree, huge and goodly to look upon, but without that fruit which it should and might bring forth; affirming that the only way of redress was a full and perfect establishment of Christ’s discipline (for so their manner is to entitle a thing hammered out upon the forge of their own invention), and that to make way of entrance for it, there must be three great limbs cut off from the body of that stately tree of the kingdom: those three limbs were three sorts of men; nobles, whose high estate would make them otherwise disdain to put their necks under  that yoke; lawyers, whose courts being not pulled down, the new church consistories were not like to flourish; finally, prelates, whose ancient dignity, and the simplicity of their intended church discipline, could not possibly stand together.

 The proposition of which device being plausible to active spirits, restless through desire of innovation, whom commonly nothing doth more offend than a change which goeth fearfully on by slow and suspicious paces; the heavier and more experienced sort began presently thereat to pull back their feet again, and exceedingly to fear the stratagem of reformation for ever after. Whereupon ensued those extreme conflicts of the one part with the other, which continuing and increasing to this very day, have now made the state of that flourishing kingdom even such, as whereunto we may most fitly apply those words of the Prophet Jeremiah, “Thy breach is great like the sea, who can heal thee?”

[2]Whether this were done in truth, according to the constant affirmation of some avouching the same, I take not upon me to examine; that which I note therein is, how with us that policy hath been corrected. For to the authors of pretended reformation with us, it hath not seemed expedient to offer the edge of the axe to all three boughs at once, but rather to single them, and strike at the weakest first, making show that the lop of that one shall draw the more abundance of sap to the other two, that they may thereby the better prosper.

All prosperity, felicity and peace we wish multiplied on each estate, as far as their own hearts’ desire is: but let men know that there is a God, whose eye beholdeth them in all their ways; a God, the usual and ordinary course of whose justice is to return upon the head of malice the same devices which it contriveth against others. The foul practices which have been used for the overthrow of bishops, may perhaps wax bold in process of time to give the like assault even there, from whence at this present they are most seconded.

[3]Nor let it over dismay them who suffer such things at the hands of this most unkind world, to see that heavenly estate and dignity thus conculcated, in regard whereof so many their predecessors were no less esteemed than if they had not been men, but angels amongst men. With  former bishops it was as with Job in the days of that prosperity which at large he describeth, saying,
 “Unto me men gave ear, they waited and held their tongue at my counsel; after my words they replied not; I appointed out their way and did sit as chief; I dwelt as it had been a king in an army.” At this day the case is otherwise with them; and yet no otherwise than with the selfsame Job at what time the alteration of his estate wrested these contrary speeches from him, “But now they that are younger than I mock at me, the children of fools, and offspring of slaves, creatures more base than the earth they tread on, such as if they did shew their heads, young and old would shout at them and chase them through the streets with a cry, their song I am, I am a theme for them to talk on.” An injury less grievous if it were not offered by them whom Satan hath through his fraud and subtilty so far beguiled as to make them imagine herein they do unto God a part of most faithful service. Whereas the lord in truth, whom they serve herein, is as St. Cyprian telleth them, like, not Christ, (for he it is that doth appoint and protect bishops,) but rather Christ’s adversary and enemy of his Church.

[4]A thousand five hundred years and upward the Church of Christ hath now continued under the sacred regiment of bishops. Neither for so long hath Christianity been ever planted in any kingdom throughout the world but with this kind of government alone; which to have been ordained of God, I am for mine own part even as resolutely persuaded, as that any other kind of government in the world whatsoever is of God. In this realm of England, before Normans, yea before Saxons, there being Christians, the chief pastors of their souls were bishops. This order from about the first establishment of Christian religion, which was publicly begun through the virtuous disposition of King Lucie not fully two hundred years after Christ, continued till the coming in of the Saxons; by whom Paganism being every where else replanted, only one part of the island, whereinto the ancient  natural inhabitants the Britons were driven, retained constantly the faith of Christ, together with the same form of spiritual regiment, which their fathers had before received. Wherefore in the histories of the Church we find very ancient mention made of our own bishops. At the council of Ariminum, about the year three hundred and fifty-nine, Britain had three of her bishops present. At the arrival of Augustine the monk, whom Gregory sent hither to reclaim the Saxons from Gentility about six hundred years after Christ, the Britons he found observers still of the selfsame government by bishops over the rest of the clergy; under this form Christianity took root again, where it had been exiled. Under the selfsame form it remained till the days of the Norman conqueror. By him and his successors thereunto sworn, it hath from that time till now by the space of five hundred years more been upheld.

O nation utterly without knowledge, without sense! We are not through error of mind deceived, but some wicked thing hath undoubtedly bewitched us, if we forsake that government, the use whereof universal experience hath for so many years approved, and betake ourselves unto a regiment neither appointed of God himself, as they who favour it pretend, nor till yesterday ever heard of among men. By the Jews Festus was much complained of, as being a governor  marvellous corrupt, and almost intolerable:
 such notwithstanding were they who came after him, that men which thought the public condition most afflicted under Festus, began to wish they had him again, and to esteem him a ruler commendable. Great things are hoped for at the hands of these new presidents, whom reformation would bring in: notwithstanding the time may come, when bishops whose regiment doth now seem a yoke so heavy to bear, will be longed for again even by them that are the readiest to have it taken off their necks.

But in the hands of Divine Providence we leave the ordering of all such events, and come now to the question itself which is raised concerning bishops. For the better understanding whereof we must beforehand set down what is meant, when in this question we name a bishop.

\section*{What a Bishop is, what his name doth import, and what doth belong to his office as he is a Bishop.}

II. For whatsoever we bring from antiquity, by way of defence in this cause of bishops, it is cast off as impertinent matter, all is wiped away with an odd kind of shifting answer, “That the bishops which now are, be not like unto them which were.” We therefore beseech all indifferent judges to weigh sincerely with themselves how the case doth stand. If it should be at this day a controversy whether kingly regiment were lawful or no, peradventure in defence thereof, the long continuance which it hath had sithence the first beginning might be alleged; mention perhaps might be made what kings there were of old even in Abraham’s time, what sovereign princes both before and after. Suppose that herein some man purposely bending his wit against sovereignty, should think to elude all such allegations by making ample discovery through a number of particularities, wherein the kings that are do differ from those that have been, and should therefore in the end conclude, that such ancient examples are no convenient proofs of that royalty which is now in use. Surely for decision of truth in this case there were no remedy, but only to shew the nature of sovereignty, to sever it from accidental properties, make it clear that ancient and present regality are one and the same in substance, how great odds soever otherwise may seem to be between them. In like manner, whereas a question of late hath grown, whether ecclesiastical regiment by bishops be lawful in the Church of  Christ or no:
 in which question, they that hold the negative, being pressed with that general received order, according whereunto the most renowned lights of the Christian world have governed the same in every age as bishops; seeing their manner is to reply, that such bishops as those ancient were, ours are not; there is no remedy but to shew, that to be a bishop is now the selfsame thing which it hath been; that one definition agreeth fully and truly as well to those elder, as to these latter bishops. Sundry dissimilitudes we grant there are, which notwithstanding are not such that they cause any equivocation in the name, whereby we should think a bishop in those times to have had a clean other definition than doth rightly agree unto bishops as they are now. Many things there are in the state of bishops, which the times have changed; many a parsonage at this day is larger than some ancient bishoprics were; many an ancient bishop poorer than at this day sundry under them in degree. The simple hereupon lacking judgment and knowledge to discern between the nature of things which changeth not, and these outward variable accidents, are made believe that a bishop heretofore and now are things in their very nature so distinct that they cannot be judged the same. Yet to men that have any part of skill, what more evident and plain in bishops, than that augmentation or diminution in their precincts, allowances, privileges, and such like, do make a difference indeed, but no essential difference between one bishop and another? As for those things in regard whereof we use properly to term them bishops, those things whereby they essentially differ from other pastors, those things which the natural definition of a bishop must contain; what one of them is there more or less appliable unto bishops now than of old?

[2]The name Bishop hath been borrowed from the Grecians, with whom it signifieth one which hath principal charge to guide and oversee others. The same word in ecclesiastical  writings being applied unto church governors, at the first unto all and not unto the chiefest only, grew in short time peculiar and proper to signify such episcopal authority alone, as the chiefest governors exercised over the rest. For with all names this is usual, that inasmuch as they are not given till the things whereunto they are given have been sometime first observed, therefore generally things are ancienter than the names whereby they are called.

Again, sith the first things that grow into general observation, and do thereby give men occasion to find names for them, are those which being in many subjects, are thereby the easier, the oftener, and the more universally noted; it followeth that names imposed to signify common qualities or operations are ancienter, than is the restraint of those names, to note an excellency of such qualities and operations in some one or few amongst others. For example, the name disciple being invented to signify generally a learner, it cannot choose but in that signification be more ancient than when it signifieth as it were by a kind of appropriation, those learners who being taught of Christ were in that respect termed disciples by an excellency. The like is to be seen in the name Apostle, the use whereof to signify a messenger must needs be more ancient than that use which restraineth it unto messengers sent concerning evangelical affairs; yea this use more ancient than that whereby the same word is yet restrained further to signify only those whom our Saviour himself immediately did send. After the same manner the title or name of a Bishop having been used of old to signify both an ecclesiastical overseer in general, and more particularly also a principal ecclesiastical overseer; it followeth, that this latter restrained signification is not so ancient as the former, being more common. Yet because the things themselves are always ancienter than their names; therefore that thing which the restrained use of the word doth import,  is likewise ancienter than the restraint of the word is,
 and consequently that power of chief ecclesiastical overseers, which the term of a bishop importeth, was before the restrained use of the name which doth import it. Wherefore a lame and an impotent kind of reasoning it is, when men go about to prove that in the Apostles’ times there was no such thing as the restrained name of a bishop doth now signify, because in their writings there is found no restraint of that name, but only a general use whereby it reacheth unto all spiritual governors and overseers.

[3]But to let go the name, and come to the very nature of that thing which is thereby signified. In all kinds of regiment whether ecclesiastical or civil, as there are sundry operations public, so likewise great inequality there is in the same operations, some being of principal respect, and therefore not fit to be dealt in by every one to whom public actions, and those of good importance, are notwithstanding well and fitly enough committed. From hence have grown those different degrees of magistrates or public persons, even ecclesiastical as well as civil. Amongst ecclesiastical persons therefore bishops being chief ones, a bishop’s function must be defined by that wherein his chiefty consisteth.

A Bishop is a minister of God, unto whom with permanent continuance there is given not only power of administering the Word and Sacraments, which power other Presbyters have; but also a further power to ordain ecclesiastical persons, and a power of chiefty in government over Presbyters as well as Laymen, a power to be by way of jurisdiction a Pastor even to Pastors themselves. So that this office, as he is a Presbyter or Pastor, consisteth in those things which are common unto him with other pastors, as in ministering the Word and Sacraments: but those things incident unto his office, which do properly make him a Bishop, cannot be common unto him with other Pastors.

Now even as pastors, so likewise bishops being principal pastors, are either at large or else with restraint: at large, when the subject of their regiment is indefinite, and not tied  to any certain place;
 bishops with restraint are they whose regiment over the Church is contained within some definite, local compass, beyond which compass their jurisdiction reacheth not. Such therefore we always mean when we speak of that regiment by bishops which we hold a thing most lawful, divine and holy in the Church of Christ.

\section*{In Bishops two things traduced; of which two the one their authority; and in it the first thing condemned, their superiority over other ministers: what kind of superiority in ministers it is which the one part holdeth and the other denieth lawful.}

III. In our present regiment by bishops two things there are complained of, the one their great authority, and the other their great honour. Touching the authority of our bishops, the first thing which therein displeaseth their adversaries, is their superiority which bishops have over other ministers. They which cannot brook the superiority which bishops have, do notwithstanding themselves admit that some kind of difference and inequality there may be lawfully amongst ministers. Inequality as touching gifts and graces they grant, because this is so plain that no mist in the world can be cast before men’s eyes so thick, but that they needs must discern through it, that one minister of the gospel may be more learneder, holier, and wiser, better able to instruct, more apt to rule and guide them than another: unless thus much were confessed, those men should lose their fame and glory whom they themselves do entitle the lights and grand worthies of this present age. Again, a priority of order they deny not but that there may be, yea such a priority as maketh one man amongst many a principal actor in those things whereunto sundry of them must necessarily concur, so that the same be admitted only during the time of such actions and no longer; that is to say, just so much superiority, and neither more nor less may be liked of, than it hath pleased them in their own kind of regiment to set down. The inequality which they complain of is, “That one minister of the word and sacraments should have a permanent superiority above another, or in any sort a superiority of power mandatory, judicial, and coercive over other ministers.” By us on the contrary side, “inequality, even such inequality as unto bishops being ministers of the word and sacraments granteth a superiority permanent above ministers, yea a permanent superiority of power mandatory, judicial and coercive over them,” is maintained a thing allowable, lawful and good.

For superiority of power may be either above them or upon them, in regard of whom it is termed superiority. One pastor hath superiority of power above another, when either some are authorized to do things worthier than are permitted unto all, [or] some are preferred to be principal agents, the rest agents with dependency and subordination. The former of these two kinds of superiority is such as the high-priest had above other priests of the law, in being appointed to enter once a year the holy place, which the rest of the priests might not do. The latter superiority, such as presidents have in those actions which are done by others with them, they nevertheless being principal and chief therein.

One pastor hath superiority of power, not only above but upon another, when some are subject unto others’ commandment and judicial controlment by virtue of public jurisdiction.

Superiority in this last kind is utterly denied to be allowable; in the rest it is only denied that the lasting continuance and settled permanency thereof is lawful. So that if we prove at all the lawfulness of superiority in this last kind, where the same is simply denied, and of permanent superiority in the rest where some kind of superiority is granted, but with restraint to the term and continuance of certain actions, with which the same must, as they say, expire and cease; if we can shew these two things maintainable, we bear up sufficiently that which the adverse party endeavoureth to overthrow. Our desire therefore is, that this issue may be strictly observed, and those things accordingly judged of, which we are to allege. This we boldly therefore set down as a most infallible truth, “That the Church of Christ is at this day lawfully, and so hath been sithence the first beginning, governed by Bishops, having permanent superiority, and ruling power over other ministers of the word and sacraments.”

[2]For the plainer explication whereof, let us briefly declare first, the birth and original of the same power, whence and by what occasion it grew. Secondly, what manner of power antiquity doth witness bishops to have had more than presbyters which were no bishops. Thirdly, after what sort bishops together with presbyters have used to govern the  churches under them,
 according to the like testimonial evidence of antiquity. Fourthly, how far the same episcopal power hath usually extended, unto what number of persons it hath reached, what bounds and limits of place it hath had. This done, we may afterwards descend unto those by whom the same either hath been heretofore, or is at this present hour gainsaid.

\section*{From whence it hath grown that the Church is governed by Bishops.}

IV. The first Bishops in the Church of Christ were his blessed Apostles; for the office whereunto Matthias was chosen the sacred history doth term ἐπισκοπὴν, an episcopal office. Which being spoken expressly of one, agreeth no less unto them all than unto him. For which cause St. Cyprian speaking generally of them all doth call them Bishops. They which were termed Apostles, as being sent of Christ to publish his gospel throughout the world, and were named likewise Bishops, in that the care of government was also committed unto them, did no less perform the offices of their episcopal authority by governing, than of their apostolical by teaching. The word ἐπισκοπὴ, expressing that part of their office which did consist in regiment, proveth not (I grant) their chiefty in regiment over others, because as then that name was common unto the function of their inferiors, and not peculiar unto theirs. But the history of their actions sheweth plainly enough how the thing itself which that name appropriated importeth, that is to say, even such spiritual chiefty as we have already defined to be properly episcopal, was in the holy Apostles of Christ. Bishops therefore they were at large.

[2]But was it lawful for any of them to be a bishop with restraint? True it is their charge was indefinite; yet so, that in case they did all whether severally or jointly discharge the office of proclaiming every where the gospel and of guiding the Church of Christ, none of them casting off his part in their burden which was laid upon them, there doth appear no impediment but that they having received their common charge indefinitely might in the execution thereof notwithstanding  restrain themselves, or at leastwise be restrained by the after commandment of the Spirit,
 without contradiction or repugnancy unto that charge more indefinite and general before given them: especially if it seemed at any time requisite, and for the greater good of the Church, that they should in such sort tie themselves unto some special part of the flock of Jesus Christ, guiding the same in several as bishops. For first, notwithstanding our Saviour’s commandment unto them all to go and preach unto all nations; yet some restraint we see there was made, when by agreement between Paul and Peter, moved with those effects of their labours which the providence of God brought forth, the one betook himself unto the Gentiles, the other unto the Jews, for the exercise of that office of every where preaching. A further restraint of their apostolic labours as yet there was also made, when they divided themselves into several parts of the world; John for his charge taking Asia, and so the residue other quarters to labour in. If nevertheless it seem very hard that we should admit a restraint so particular, as after that general charge received to make any Apostle notwithstanding the bishop of some one church; what think we of the bishop of Jerusalem, James, whose consecration unto that mother see  of the world,
 because it was not meet that it should at any time be left void of some Apostle, doth seem to have been the very cause of St. Paul’s miraculous vocation, to make up the number of the twelve again, for the gathering of nations abroad, even as the martyrdom of the other James, the reason why Barnabas in his stead was called.

Finally, Apostles, whether they did settle in any one certain place, as James, or else did otherwise, as the Apostle Paul, episcopal authority either at large or with restraint they had and exercised. Their episcopal power they sometimes gave unto others to exercise as agents only in their stead, and as it were by commission from them. Thus Titus, and thus Timothy, at the first, though afterwards endued with apostolical power of their own.

[3]For in process of time the Apostles gave episcopal authority, and that to continue always with them which had it. “We are able to number up them,” saith Irenæus, “who by the Apostles were made bishops.” In Rome he affirmeth that the Apostles themselves made Linus the first bishop. Again of Polycarp he saith likewise, that the Apostles made him bishop of the church of Smyrna. Of Antioch they made Evodius bishop, as Ignatius witnesseth, exhorting that church to tread in his holy steps, and to follow his virtuous example.

The Apostles therefore were the first which had such authority, and all others who have it after them in orderly sort are their lawful successors, whether they succeed in any particular church, where before them some Apostle hath been  seated,
 as Simon succeeded James in Jerusalem; or else be otherwise endued with the same kind of bishoply power, although it be not where any Apostle before hath been. For to succeed them, is after them to have that episcopal kind of power which was first given to them. “All bishops are,” saith Jerome, “the Apostles’ successors.” In like sort Cyprian doth term bishops, “Præpositos qui Apostolis vicaria ordinatione succedunt.” From hence it may haply seem to have grown, that they whom we now call Bishops were usually termed at the first Apostles, and so did carry their very names in whose rooms of spiritual authority they succeeded.

[4]Such as deny Apostles to have any successors at all in the office of their apostleship, may hold that opinion without contradiction to this of ours, if they well explain themselves in declaring what truly and properly apostleship is. In some things every presbyter, in some things only bishops, in some things neither the one nor the other are the Apostles’ successors. The Apostles were sent as special chosen eyewitnesses of Jesus Christ, from whom immediately they received their whole embassage, and their commission to be the principal first founders of an house of God, consisting as well of Gentiles as of Jews. In this there are not after them any other like unto them; and yet the Apostles have now their successors upon earth, their true successors, if not in the largeness,  surely in the kind of that episcopal function,
 whereby they had power to sit as spiritual ordinary judges, both over laity and over clergy, where churches Christian were established.

\section*{The time and cause of instituting everywhere Bishops with restraint.}

V. The Apostles of our Lord did according unto those directions which were given them from above, erect churches in all such cities as received the word of truth, the gospel of God. All churches by them erected received from them the same faith, the same sacraments, the same form of public regiment. The form of regiment by them established at first was, that the laity or people should be subject unto a college of ecclesiastical persons, which were in every such city appointed for that purpose. These in their writings they term sometime presbyters, sometime bishops. To take one church out of a number for a pattern what the rest were; the presbyters of Ephesus, as it is in the history of their departure from the Apostle Paul at Miletum, are said to have wept abundantly all, which speech doth shew them to have been many. And by the Apostle’s exhortation it may appear that they had not each his several flock to feed, but were in common appointed to feed that one flock, the church of Ephesus; for which cause the phrase of his speech is this, Attendite gregi, “Look all to that one flock over which the Holy Ghost hath made you bishops.” These persons ecclesiastical being termed as then, presbyters and bishops both, were all subject unto Paul as to an higher governor appointed of God to be over them.

[2]But forasmuch as the Apostles could not themselves be present in all churches, and as the Apostle St. Paul foretold the presbyters of the Ephesians that there would “rise up from amongst their ownselves, men speaking perverse things to draw disciples after them;” there did grow in short time amongst the governors of each church those emulations, strifes, and contentions, whereof there could be no sufficient remedy provided, except according unto the order of Jerusalem already begun, some one were endued with episcopal authority over the rest, which one being resident might keep them in order, and have preeminence or principality in those things wherein the equality of many agents was the cause of disorder and trouble. This one president or governor amongst the rest had his known authority established a long time before that settled difference of name and title took place, whereby such alone were named bishops. And therefore in the book of St. John’s Revelation we find that they are entitled angels.

It will perhaps be answered, that the angels of those churches were only in every church a minister of the word and sacraments. But then we ask, is it probable that in every of these churches, even in Ephesus itself, where many such ministers were long before, as hath been proved, there was but one such when John directed his speech to the angel of that church? If there were many, surely St. John in naming but only one of them an angel, did behold in that one somewhat above the rest.

Nor was this order peculiar unto some few churches, but the whole world universally became subject thereunto; insomuch as they did not account it to be a church which was not subject unto a bishop. It was the general received persuasion of the ancient Christian world, that Ecclesia est in Episcopo, “the outward being of a church consisteth in the having of a bishop.” That where colleges of presbyters were, there was at the first equality amongst them, St. Jerome  thinketh it a matter clear;
 but when the rest were thus equal, so that no one of them could command any other as inferior unto him, they all were controllable by the Apostles, who had that episcopal authority abiding at the first in themselves, which they afterwards derived unto others.

The cause wherefore they under themselves appointed such bishops as were not every where at the first, is said to have been those strifes and contentions, for remedy whereof, whether the Apostles alone did conclude of such a regiment, or else they together with the whole Church judging it a fit and a needful policy did agree to receive it for a custom; no doubt but being established by them on whom the Holy Ghost was poured in so abundant measure for the ordering of Christ’s Church, it had either divine appointment beforehand, or divine approbation afterwards, and is in that respect to be acknowledged the ordinance of God, no less than that ancient Jewish regiment, whereof though Jethro were the deviser, yet after that God had allowed it, all men were subject unto it, as to the polity of God, and not of Jethro.

[3]That so the ancient Fathers did think of episcopal regiment; that they held this order as a thing received from the blessed Apostles themselves, and authorized even from heaven, we may perhaps more easily prove, than obtain that they all shall grant it who see it proved. St. Augustine setteth it down for a principle, that whatsoever positive order the whole Church every where doth observe, the same it must needs have received from the very Apostles themselves, unless perhaps some general council were the authors of it. And he saw that the ruling superiority of bishops was a thing universally established, not by the force of any council (for councils do all presuppose bishops, nor can there any council be named so ancient, either general, or as much as provincial, sithence the Apostles’ own times, but we can shew that bishops had their authority before it, and not from it). Wherefore St. Augustine knowing this, could not choose but reverence the  authority of bishops, as a thing to him apparently and most clearly apostolical.


[4]But it will be perhaps objected that regiment by bishops was not so universal nor ancient as we pretend; and that an argument hereof may be Jerome’s own testimony, who, living at the very same time with St. Augustine, noted this kind of regiment as being no where ancient, saving only in Alexandria; his words are these: “It was for a remedy of schism that one was afterwards chosen to be placed above the rest; lest every man’s pulling unto himself should rend asunder the Church of Christ. For (that which also may serve for an argument or token hereof), at Alexandria, from Mark the Evangelist, unto Heraclas and Dionysius, the presbyters always chose one of themselves, whom they placed in higher degree, and gave unto him the title of bishop.” Now St. Jerome they say would never have picked out that one church from amongst so many, and have noted that in it there had been bishops from the time that St. Mark lived, if so be the selfsame order were of like antiquity every where; his words therefore must be thus scholied: in the church of Alexandria, presbyters indeed had even from the time of St. Mark the Evangelist always a bishop to rule over them, for a remedy against divisions, factions, and schisms. Not so in other churches, neither in that very church any longer than usque ad Heraclam et Dionysium, “till Heraclas and his successor Dionysius were bishops.”

[5]But this construction doth bereave the words construed, partly of wit, and partly of truth; it maketh them both absurd and false. For, if the meaning be that episcopal government in that church was then expired, it must have expired with the end of some one, and not of two several bishops’ days, unless perhaps it fell sick under Heraclas, and with Dionysius gave up the ghost.

Besides, it is clearly untrue that the presbyters of that church did then cease to be under a bishop. Who doth not  know that after Dionysius,
 Maximus was bishop of Alexandria, after him Theonas, after him Peter, after him Achillas, after him Alexander: of whom Socrates in this sort writeth: “it fortuned on a certain time that this Alexander in the presence of the presbyters which were under him, and of the rest of the clergy there, discoursed somewhat curiously and subtilly of the holy Trinity, bringing high philosophical proofs, that there is in the Trinity an Unity. Whereupon Arius, one of the presbyters which were placed in that degree under Alexander, opposed eagerly himself against those things which were uttered by the bishop.” So that thus long bishops continued even in the church of Alexandria. Nor did their regiment here cease, but these also had others their successors till St. Jerome’s own time, who living long after Heraclas and Dionysius had ended their days, did not yet live himself to see the presbyters of Alexandria otherwise than subject unto a bishop. So that we cannot with any truth so interpret his words as to mean, that in the church of Alexandria there had been bishops endued with superiority over presbyters from St. Mark’s time only till the time of Heraclas and of Dionysius.

[6]Wherefore that St. Jerome may receive a more probable interpretation than this, we answer, that generally of regiment by bishops, and what term of continuance it had in the church of Alexandria, it was no part of his mind to speak, but to note one only circumstance belonging to the manner of their election, which circumstance is, that in Alexandria they use to choose their bishops altogether out of the college of their own presbyters, and neither from abroad nor out of any other inferior order of the clergy; whereas oftentimes elsewhere the use was to choose as well from abroad as at home, as well inferior unto presbyters as presbyters when they saw occasion. This custom,  saith he, the Church of Alexandria did always keep, till in Heraclas and Dionysius they began to do otherwise. These two were the very first not chosen out of their college of presbyters.

The drift and purpose of St. Jerome’s speech doth plainly shew what his meaning was: for whereas some did over extol the office of the deacon in the church of Rome, where deacons being grown great, through wealth, challenged place above presbyters; St. Jerome to abate this insolency, writing to Evagrius diminisheth by all means the deacon’s estimation, and lifteth up presbyters as far as possible the truth might bear. “An attendant,” saith he, “upon tables and widows proudly to exalt himself above them at whose prayers is made the Body and Blood of Christ; above them, between whom and bishops there was at the first for a time no difference neither in authority nor in title. And whereas afterward schisms and contentions made it necessary that some one should be placed over them, by which occasion the title of bishop became proper unto that one, yet was that one chosen out of the presbyters, as being the chiefest, the highest, the worthiest degree of the clergy, and not out of deacons: in which consideration also it seemeth that in  Alexandria even from St. Mark to Heraclas and Dionysius bishops there, the presbyters evermore have chosen one of themselves, and not a deacon at any time, to be their bishop. Nor let any man think that Christ hath one church in Rome and another in the rest of the world; that in Rome he alloweth deacons to be honoured above presbyters, and otherwise will have them to be in the next degree to the bishop. If it be deemed that abroad where bishops are poorer, the presbyters under them may be the next unto them in honour, but at Rome where the bishop hath ample revenues, the deacons whose estate is nearest for wealth, may be also for estimation the next unto him: we must know that a bishop in the meanest city is no less a bishop than he who is seated in the greatest; the countenance of a rich and the meanness of a poor estate doth make no odds between bishops: and therefore, if a presbyter at Eugubium be the next in degree to a bishop, surely, even at Rome it ought in reason to be so likewise, and not a deacon for wealth’s sake only to be above, who by order should be, and elsewhere is, underneath a presbyter. But ye will say that according to the custom of Rome a deacon presenteth unto the bishop him which standeth to be ordained presbyter, and upon the deacon’s testimony given concerning his fitness, he receiveth at the Bishop’s hands ordination: so that in Rome the deacon having this special preeminence, the presbyter ought there to give place unto him. Wherefore is the custom of one city brought against the practice of the whole world? The paucity of deacons in the church of Rome hath gotten the [them?] credit; as unto presbyters their multitude hath been cause of contempt: howbeit even in the Church of Rome, presbyters sit, and deacons stand; an argument as strong against the superiority of deacons, as the fore-alleged reason doth seem for it. Besides, whosoever is promoted must needs be raised from a lower degree to an higher; wherefore either let him which is presbyter be made a deacon, that so the deacon may appear to be the greater; or if of deacons presbyters be made, let them know themselves to be in regard of deacons, though below in gain, yet above in office. And to the end we may understand that those apostolical orders are taken out of the Old Testament, what Aaron  and his sons and the Levites were in the temple,
 the same in the Church may bishops and presbyters and deacons challenge unto themselves.”

[7]This is the very drift and substance, this the true construction and sense of St. Jerome’s whole discourse in that epistle: which I have therefore endeavoured the more at large to explain, because no one thing is less effectual or more usual to be alleged against the ancient authority of bishops; concerning whose government St. Jerome’s own words otherwhere are sufficient to shew his opinion, that this order was not only in Alexandria so ancient, but even as ancient in other churches. We have before alleged his testimony touching James the bishop of Jerusalem. As for bishops in other churches, on the first of the Epistle to Titus thus he speaketh, “Till through instinct of the Devil there grew in the Church factions, and among the people it began to be professed, I am of Paul, I of Apollos, and I of Cephas, churches were governed by the common advice  of presbyters;
 but when every one began to reckon those whom himself had baptized his own and not Christ’s, it was decreed in the whole world that one chosen out of the presbyters should be placed above the rest, to whom all care of the Church should belong, and so the seeds of schism be removed.” If it be so, that by St. Jerome’s own confession this order was not then begun when people in the apostles’ absence began to be divided into factions by their teachers, and to rehearse, “I am of Paul,” but that even at the very first appointment thereof [it] was agreed upon and received throughout the world; how shall a man be persuaded that the same Jerome thought it so ancient no where saving in Alexandria, one only church of the whole world?

[8]A sentence there is indeed of St. Jerome’s, which being not thoroughly considered and weighed may cause his meaning so to be taken, as if he judged episcopal regiment to have been the Church’s invention long after, and not the apostles’ own institution; as namely, when he admonisheth bishops in this manner: “As therefore presbyters do know that the custom of the Church makes them subject to the Bishop which is set over them; so let bishops know that custom, rather than the truth of any ordinance of the Lord’s maketh  them greater than the rest, and that with common advice they ought to govern the Church.”

To clear the sense of these words therefore, as we have done already the former: laws which the Church from the beginning universally hath observed were some delivered by Christ himself, with a charge to keep them to the world’s end, as the law of baptizing and administering the holy eucharist; some brought in afterwards by the apostles, yet not without the special direction of the Holy Ghost, as occasions did arise. Of this sort are those apostolical orders and laws whereby deacons, widows, virgins, were first appointed in the Church. *[This answer to St. Jerome seemeth dangerous; I have qualified it as I may by addition of some words of restraint: yet I satisfy not myself, in my judgment it would be altered.] “Now whereas Jerome doth term the government of bishops by restraint an apostolical tradition, acknowledging thereby the same to have been of the apostles’ own institution, it may be demanded how these two will stand together; namely, that the apostles by divine instinct should be, as Jerome confesseth, the authors of that regiment; and yet the custom of the Church be accounted (for so by Jerome it may seem to be in this place accounted) the chiefest prop that upholdeth the same? To this we answer, That forasmuch as the whole body of the Church hath power to alter, with general consent and upon necessary occasions, even the positive laws of the apostles, if there be no command to the contrary, and it manifestly appears to her, that change of times have clearly taken away the very reasons of God’s first institution; as by sundry examples may be most clearly proved: what laws the universal Church might  change, and doth not, if they have long continued without any alteration, it seemeth that St. Jerome ascribeth the continuance of such positive laws, though instituted by God himself, to the judgment of the Church. For they which might abrogate a law and do not, are properly said to uphold, to establish it, and to give it being. The regiment therefore whereof Jerome speaketh being positive, and consequently not absolutely necessary, but of a changeable nature, because there is no divine voice which in express words forbiddeth it to be changed; he might imagine both that it came by the apostles by very divine appointment at the first, and notwithstanding be, after a sort, said to stand in force, rather by the custom of the Church, choosing to continue in it, than by the necessary constraint of any commandment from the word, requiring perpetual continuance thereof.” So that St. Jerome’s admonition is reasonable, sensible, and plain, being contrived to this effect: The ruling superiority of one bishop over many presbyters in each church, is an order descended from Christ to the Apostles, who were themselves bishops at large, and from the Apostles to those whom they in their steads appointed bishops over particular countries and cities; and even from those ancient times, universally established, thus many years it hath continued throughout the world; for which cause presbyters must not grudge to continue subject unto their bishops, unless they will proudly oppose themselves against that which God himself ordained by his apostles, and the whole Church of Christ approveth and judgeth most convenient. On the other side bishops, albeit they may avouch with conformity of truth that their authority hath thus descended even from the very apostles themselves, yet the absolute and everlasting continuance of it they cannot say that any commandment of the Lord doth enjoin; and therefore must acknowledge that the Church hath power by universal consent upon urgent cause to take it away, if thereunto she be constrained through the proud, tyrannical, and unreformable dealings of her bishops, whose regiment she hath thus long delighted in, because she hath found it good and requisite to be so governed. Wherefore lest bishops forget themselves, as if none on earth had authority to touch their states, let them continually bear in mind, that it is  rather the force of custom,
 whereby the Church having so long found it good to continue under the regiment of her virtuous bishops, doth still uphold, maintain, and honour them in that respect, than that any such true and heavenly law can be shewed, by the evidence whereof it may of a truth appear that the Lord himself hath appointed presbyters for ever to be under the regiment of bishops, in what sort soever they behave themselves. Let this consideration be a bridle unto them, let it teach them not to disdain the advice of their presbyters, but to use their authority with so much the greater humility and moderation, as a sword which the Church hath power to take from them. In all this there is no let why St. Jerome might not think the authors of episcopal regiment to have been the very blessed apostles themselves, directed therein by the special motion of the Holy Ghost, which the ancients all before and besides him and himself also elsewhere being known to hold, we are not without better evidence than this to think him in judgment divided both from himself and from them.

[9]Another argument that the regiment of churches by one Bishop over many presbyters hath been always held apostolical, may be this. We find that throughout all those cities where the apostles did plant Christianity, the history of times hath noted succession of pastors in the seat of one, not of many (there being in every such Church evermore many pastors), and the first one in every rank of succession we find to have been, if not some Apostle, yet some Apostle’s disciple. By Epiphanius the bishops of Jerusalem are reckoned down from James to Hilarion then Bishop. Of them which boasted that they held the same things which they received of such as lived with the apostles themselves, Tertullian speaketh after this sort: “Let them therefore  shew the beginnings of their churches,
 let them recite their bishops one by one, each in such sort succeeding other, that the first bishop of them have had for his author and predecessor some Apostle, or at least some apostolical person who persevered with the apostles. For so apostolical churches are wont to bring forth the evidence of their estates. So doth the Church of Smyrna, having Polycarp whom John did consecrate.” Catalogues of bishops in a number of other churches, *(bishops, and succeeding one another) from the very apostles’ times, are by Eusebius and Socrates collected; whereby it appeareth so clear, as nothing in the world more, that under them and by their appointment this order began, which maketh many presbyters subject unto the regiment of some one bishop. For as in Rome while the civil ordering of the commonwealth was jointly and equally in the hands of two consuls, historical records concerning them did evermore mention them both, and note which two as colleagues succeeded from time to time; so there is no doubt but ecclesiastical antiquity had done the very like, had not one pastor’s place and calling been always so eminent above the rest in the same church.

[10]And what need we to seek far for proofs that the apostles, who began this order of regiment of bishops, did it not but by divine instinct, when without such direction things of far less weight and moment they attempted not? Paul and Barnabas did not open their mouths to the Gentiles, till the Spirit had said, “Separate me Paul and Barnabas for the work whereunto I have sent them.” The eunuch by Philip was neither baptized nor instructed before the angel of God was sent to give him notice that so it pleased the Most High. In Asia, Paul and the rest were silent, because the Spirit forbade them to speak. When they intended to have seen Bithynia they stayed their journey, the Spirit not giving them leave to go. Before Timothy was employed in those episcopal affairs of the Church, about which the Apostle St. Paul used him, the Holy Ghost gave special charge for his ordination, and prophetical intelligence  more than once,
 what success the same would have. And shall we think that James was made bishop of Jerusalem, Evodius bishop of the church of Antioch, the Angels in the churches of Asia bishops, that bishops every where were appointed to take away factions, contentions, and schisms, without some like divine instigation and direction of the Holy Ghost? Wherefore let us not fear to be herein bold and peremptory, that if any thing in the Church’s government, surely the first institution of bishops was from heaven, was even of God, the Holy Ghost was the author of it.

\section*{What manner of power Bishops from the first beginning have had.}

VI. “A Bishop,” saith St. Augustine, “is a Presbyter’s superior:” but the question is now, wherein that superiority did consist. The Bishop’s preeminence we say therefore was twofold. First he excelled in latitude of the power of order, secondly in that kind of power which belongeth unto jurisdiction. Priests in the law had authority and power to do greater things than Levites, the high-priest greater than inferior priests might do; therefore Levites were beneath priests, and priests inferior to the high-priest, by reason of the very degree of dignity, and of worthiness in the nature of those functions which they did execute, and not only for that the one had power to command and control the other. In like sort presbyters having a weightier and a worthier charge than deacons had, the deacon was in this sort the presbyter’s inferior; and where we say that a bishop was likewise ever accounted a presbyter’s superior, even according unto his very power of order, we must of necessity declare what principal duties belonging unto that kind of power a bishop might perform, and not a presbyter.

[2]The custom of the primitive Church in consecrating holy virgins and widows unto the service of God and his Church, is a thing not obscure, but easy to be known, both  by that which St. Paul himself concerning them hath,
 and by the latter consonant evidence of other men’s writings. Now a part of the preeminence which bishops had in their power of order, was that by them only such were consecrated.

[3]Again, the power of ordaining both deacons and presbyters, the power to give the power of order unto others, this also hath been always peculiar unto bishops. It hath not been heard of, that inferior presbyters were ever authorized to ordain. And concerning ordination, so great force and dignity it hath, that whereas presbyters, by such power as they have received for administration of the sacraments, are able only to beget children unto God; bishops having power to ordain, do by virtue thereof create fathers to the people of God, as Epiphanius fitly disputeth. There are which hold that between a bishop and a presbyter, touching power of order, there is no difference. The reason of which conceit is, for that they see presbyters no less than bishops authorized to offer up the prayers of the Church, to preach the gospel, to baptize, to administer the holy Eucharist; but they considered not withal as they should, that the presbyter’s authority to do these things is derived from the bishop which doth ordain him thereunto, so that even in those things which are common unto both, yet the power of the one is as it were a certain light borrowed from the others’ lamp. The apostles being bishops at large, ordained every where presbyters. Titus and Timothy having received episcopal power, as apostolic ambassadors or legates, the one in Greece [Crete], the other in Ephesus, they both did by virtue thereof likewise ordain throughout all churches deacons and presbyters within the circuits allotted unto them. As for bishops by restraint, their power this way incommunicable unto presbyters which of the ancients do not acknowledge?


[4]I make not confirmation any part of that power which hath always belonged only unto bishops, because in some places the custom was that presbyters might also confirm in the absence of a bishop; albeit for the most part none but only bishops were thereof the allowed ministers.

[5]Here it will perhaps be objected that the power of ordination itself was not every where peculiar and proper unto bishops, as may be seen by a council of Carthage, which sheweth their church’s order to have been, that presbyters should together with the bishop lay hands upon the ordained. But the answer hereunto is easy; for doth it hereupon follow that the power of ordination was not principally and originally in the bishop? Our Saviour hath said unto his Apostles, “With me ye shall sit and judge the twelve tribes of Israel;” yet we know that to him alone it belongeth to judge the world, and that to him all judgment is given. With us even at this day presbyters are licensed to do as much as that council speaketh of, if any be present. Yet will not any man thereby conclude that in this church others than bishops are allowed to ordain. The association of presbyters is no sufficient proof that the power of ordination is in them; but rather that it never was in them we may hereby understand, for that no man is able to shew either deacon or presbyter ordained by presbyters only, and his ordination accounted lawful in any ancient part of the Church; every where examples being found both of deacons and of presbyters ordained by bishops alone oftentimes, neither ever in that respect thought unsufficient.

[6]Touching that other chiefty, which is of jurisdiction; amongst the Jews he which was highest through the worthiness of peculiar duties incident unto his function in the legal service of God, did bear always in ecclesiastical jurisdiction the chiefest sway. As long as the glory of the temple of God did last, there were in it sundry orders of men consecrated  unto the service thereof,
 one sort of them inferior unto another in dignity and degree; the Nathiners subordinate unto the Levites, the Levites unto the Priests, the rest of the priests to those twenty-four which were chief priests, and they all to the High Priest. If any man surmise that the difference between them was only by distinction in the former kind of power, and not in this latter of jurisdiction, are not the words of the law manifest which make Eleazar the son of Aaron the priest chief captain of the Levites, and overseer of them unto whom the charge of the sanctuary was committed? Again, at the commandment of Aaron and his sons are not the Gersonites themselves required to do all their service in the whole charge belonging unto the Gersonites, being inferior priests as Aaron and his sons were high priests? Did not Jehoshaphat appoint Amarias the priest to be chief over them who were judges for the cause of the Lord in Jerusalem? “Priests,” saith Josephus, “worship God continually, and the eldest of the stock are governors over the rest. He doth sacrifice unto God before others, he hath care of the laws, judgeth controversies, correcteth offenders, and whosoever obeyeth him not is convict of impiety against God.”

[7]But unto this they answer, that the reason thereof was because the high priest did prefigure Christ, and represent to the people that chiefty of our Saviour which was to come; so that Christ being now come there is no cause why such preeminence should be given unto any one. Which fancy pleaseth so well the humour of all sorts of rebellious spirits, that they all seek to shroud themselves under it. Tell the Anabaptist, which holdeth the use of the sword unlawful for a Christian man, that God himself did allow his people to  make wars; they have their answer round and ready, “Those ancient wars were figures of the spiritual wars of Christ.” Tell the Barrowist what sway David and others the kings of Israel did bear in the ordering of spiritual affairs, the same answer again serveth, namely, “That David and the rest of the kings of Israel prefigured Christ.” Tell the Martinist of the high priest’s great authority and jurisdiction amongst the Jews, what other thing doth serve his turn but the selfsame shift; “By the power of the high priest the universal supreme authority of our Lord Jesus Christ was shadowed.”

The thing is true, that indeed high priests were figures of Christ, yet this was in things belonging unto their power of order; they figured Christ by entering into the holy place, by offering for the sins of all the people once a year, and by other the like duties: but that to govern and to maintain order amongst those that were subject to them, is an office figurative and abrogated by Christ’s coming in the ministry; that their exercise of jurisdiction was figurative, yea figurative in such sort, that it had no other cause of being instituted, but only to serve as a representation of somewhat to come, and that herein the Church of Christ ought not to follow them; this article is such as must be confirmed, if any way, by miracle, otherwise it will hardly enter into the heads of reasonable men, why the high priest should more figure Christ in being a Judge than in being whatsoever he might be besides. St. Cyprian deemed it no wresting of Scripture  to challenge as much for Christian bishops as was given to the high priest among the Jews,
 and to urge the law of Moses as being most effectual to prove it. St. Jerome likewise thought it an argument sufficient to ground the authority of bishops upon. “To the end,” saith he, “we may understand Apostolical traditions to have been taken from the Old Testament; that which Aaron and his sons and the Levites were in the temple, Bishops and Presbyters and Deacons in the Church may lawfully challenge to themselves.”

[8]In the office of a Bishop Ignatius observeth these two functions, ἱερατεύειν καὶ ἄρχειν: concerning the one, such is a [the?] preeminence of a bishop, that he only hath the heavenly mysteries of God committed originally unto him, so that otherwise than by his ordination, and by authority received from him, others besides him are not licensed therein to deal as ordinary ministers of God’s church. And touching the other part of their sacred function, wherein the power of their jurisdiction doth appear, first how the Apostles themselves, and secondly how Titus and Timothy had rule and jurisdiction over presbyters, no man is ignorant. And had not Christian bishops afterwards the like power? Ignatius bishop of Antioch being ready by blessed martyrdom to end his life, writeth unto his presbyters, the pastors under him, in this sort: Οἱ πρεσβύτεροι, ποιμάνατε τὸ ἐν ὑμι̑ν ποιμνίον, ἕως ἀναδείξῃ ὁ Θεὸς τὸν μέλλοντα ἄρχειν ὑμω̑ν. Ἐγὼ γὰρ ἤδη σπένδομαι. After the death of Fabian bishop of Rome, there growing some trouble about the receiving of such persons into the Church as had fallen away in persecution, and did now repent their fall, the presbyters and deacons of the same church advertised St. Cyprian thereof, signifying, “That they  must of necessity defer to deal in that cause till God did send them a new bishop which might moderate all things.” Much we read of extraordinary fasting usually in the Church. And in this appeareth also somewhat concerning the chiefty of bishops. “The custom is,” saith Tertullian, “that bishops do appoint when the people shall all fast.” “Yea, it is not a matter left to our own free choice whether bishops shall rule or no, but the will of our Lord and Saviour is,” saith Cyprian, “that every act of the Church be governed by her bishops.” An argument it is of the bishop’s high preeminence, rule and government over all the rest of the clergy, even that the sword of persecution did strike, especially, always at the bishop as at the head, the rest by reason of their lower estate being more secure, as the selfsame Cyprian noteth; the very manner of whose speech unto his own both deacons and presbyters who remained safe, when himself then bishop was driven into exile, argueth likewise his eminent authority and rule over them. “By these letters,” saith he, “I both exhort and command that ye whose presence there is not envied at, nor so much beset with dangers, supply my room in doing those things which the exercise of religion doth require.” Unto the same purpose serve most directly those comparisons, than which nothing is more familiar in the books  of the ancient Fathers,
 who as oft as they speak of the several degrees in God’s clergy, if they chance to compare presbyters with Levitical priests of the law, the bishop they compare unto Aaron the high priest; if they compare the one with the Apostles, the other they compare (although in a lower proportion) sometime to Christ, and sometime to God himself, evermore shewing that they placed the bishop in an eminent degree of ruling authority and power above other presbyters. Ignatius comparing bishops with deacons, and with such ministers of the word and sacraments as were but presbyters, and had no authority over presbyters; “What is,” saith he, “the bishop, but one which hath all principality and power over all, so far forth as man may have it, being to his power a follower even of God’s own Christ?”

[9]Mr. Calvin himself, though an enemy unto regiment by bishops, doth notwithstanding confess, that in old time the ministers which had charge to teach, chose of their company one in every city, to whom they appropriated the title of bishop, lest equality should breed dissension. He added farther, that look, what duty the Roman consuls did execute in proposing matters unto the senate, in asking their opinions, in directing them by advice, admonition, exhortation, in guiding actions by their authority, and in seeing that performed which was with common consent agreed on, the like charge had the bishop in the assembly of other ministers. Thus much Calvin being forced by the evidence of truth to grant, doth  yet deny the bishops to have been so in authority at the first as to bear rule over other ministers:
 wherein what rule he doth mean, I know not. But if the bishops were so far in dignity above other ministers, as the consuls of Rome for their year above other senators, it is as much as we require. And undoubtedly if as the consuls of Rome, so the bishops in the Church of Christ had such authority, as both to direct other ministers, and to see that every of them should observe that which their common consent had agreed on, how this could be done by the bishop not bearing rule over them, for mine own part I must acknowledge that my poor conceit is not able to comprehend.

[10]One objection there is of some force to make against that which we have hitherto endeavoured to prove, if they mistake it not who allege it. St. Jerome, comparing other presbyters with him unto whom the name of bishop was then appropriate, asketh, “What a bishop by virtue of his place and calling may do more than a presbyter, except it be only to ordain?” In like sort Chrysostom having moved a question, wherefore St. Paul should give Timothy precept concerning the quality of bishops, and descend from them to deacons, omitting the order of presbyters between, he maketh thereunto this answer, “What things he spake concerning bishops, the same are also meet for presbyters, whom bishops seem not to excel in any thing but only in the power of ordination.” Wherefore seeing this doth import no ruling superiority, it follows that bishops were as then no rulers over that part of the clergy of God.

Whereunto we answer, that both St. Jerome and St. Chrysostom  had in those their speeches an eye no further than only to that function for which presbyters and bishops were consecrated unto God. Now we know that their consecration had reference to nothing but only that which they did by force and virtue of the power of order, wherein sith bishops received their charge, only by that one degree, to speak of, more ample than presbyters did theirs, it might be well enough said that presbyters were that way authorized to do, in a manner, even as much as bishops could do, if we consider what each of them did by virtue of solemn consecration: for as concerning power of regiment and jurisdiction, it was a thing withal added unto bishops for the necessary use of such certain persons and people, as should be thereunto subject in those particular churches whereof they were bishops, and belonged to them only as bishops of such or such a church; whereas the other kind of power had relation indefinitely unto any of the whole society of Christian men, on whom they should chance to exercise the same, and belonged to them absolutely, as they were bishops wheresoever they lived. St. Jerome’s conclusion thereof is, “That seeing in the one kind of power there is no greater difference between a presbyter and a bishop, bishops should not because of their preeminence in the other too much lift up themselves above the presbyters under them.” St. Chrysostom’s collection, “That whereas the Apostle doth set down the qualities whereof regard should be had in the consecration of bishops, there was no need to make a several discourse how presbyters ought to be qualified when they are ordained; because there being so little difference in the functions, whereunto the one and the other receive ordination, the same precepts might well serve for both; at leastwise by the virtues required in the greater, what should need in the less might be easily understood. As for the difference of jurisdiction, the truth is, the Apostles yet living, and themselves where they were resident exercising the jurisdiction in their own persons, it was not every where established in bishops.” When the Apostles prescribed those laws, and when Chrysostom thus spake concerning them, it was not by him at all respected, but his eye  was the same way with Jerome’s; his cogitation was wholly fixed on that power which by consecration is given to bishops more than to presbyters, and not on that which they have over presbyters by force of their particular accessary jurisdiction.

Wherein if any man suppose that Jerome and Chrysostom knew no difference at all between a presbyter and a bishop, let him weigh but one or two of their sentences. The pride of insolent bishops hath not a sharper enemy than Jerome, for which cause he taketh often occasions most severely to inveigh against them, sometimes for shewing disdain and contempt of the clergy under them; sometime for not suffering themselves to be told of their faults, and admonished of their duty by inferiors; sometime for not admitting their presbyters to teach, if so be themselves were in presence; sometimes for not vouchsafing to use any conference with them, or to take any counsel of them. Howbeit never doth he in such wise bend himself against their disorders, as to deny their rule and authority over presbyters. Of Vigilantius being a presbyter, he thus writeth: “Miror sanctum episcopum in cujus parochia presbyter esse dicitur, acquiescere furori ejus, et non virga apostolica virgaque ferrea confringere vas inutile:” “I marvel that the holy bishop under whom Vigilantius is said to be a presbyter, doth yield to his fury, and not break that unprofitable vessel with his apostolic and iron rod.” With this agreeth most fitly the grave advice he giveth to Nepotian: “Be  thou subject unto thy bishop, and receive him as the father of thy soul. This also I say, that bishops should know themselves to be priests and not lords; that they ought to honour the clergy as beseemeth the clergy to be honoured, to the end their clergy may yield them the honour which as bishops they ought to have. That of the orator Domitius is famous: ‘Wherefore should I esteem of thee as of a prince, when thou makest not of me that reckoning which should in reason be made of a senator?’ Let us know the bishop and his presbyters to be the same which Aaron sometime and his sons were.” Finally writing against the heretics which were named Luciferians, “The very safety of the Church,” saith he, “dependeth on the dignity of the chief priest, to whom unless men grant an exceeding and an eminent power, there will grow in churches even as many schisms as there are persons which have authority.”

Touching Chrysostom, to shew that by him there was also acknowledged a ruling superiority of bishops over presbyters, both then usual, and in no respect unlawful, what need we allege his words and sentences, when the history of his own episcopal actions in that very kind is till this day extant for all men to read that will? For St. Chrysostom of a presbyter in Antioch, grew to be afterwards bishop of Constantinople; and in process of time when the emperor’s heavy displeasure had through the practice of a powerful faction against him effected his banishment, Innocent the bishop of Rome understanding thereof wrote his letters unto the clergy of that Church, “That no successor ought to be chosen in Chrysostom’s room: nec ejus Clerum alii parere Pontifici, nor his clergy obey any other bishop than him.” A fond kind of speech, if so be there had been as then in bishops no ruling  superiority over presbyters.
 When two of Chrysostom’s presbyters had joined themselves to the faction of his mortal enemy Theophilus, Patriarch in the Church of Alexandria, the same Theophilus and other bishops which were of his conventicle, having sent those two amongst others to cite Chrysostom their lawful bishop, and to bring him into public judgment, he taketh against this one thing special exception, as being contrary to all order, that those presbyters should come as messengers and call him to judgment, who were a part of that clergy whereof himself was ruler and judge. So that bishops to have had in those times a ruling superiority over presbyters, neither could Jerome nor Chrysostom be ignorant; and therefore hereupon it were superfluous that we should any longer stand.

\section*{After what sort Bishops together with presbyters have used to govern the churches which were under them.}

VII. Touching the next point, how bishops together with presbyters have used to govern the churches which were under them: it is by Zonaras somewhat plainly and at large declared, that the bishop had his seat on high in the church above the residue which were present; that a number of presbyters did always there assist him; and that in the oversight of the people those presbyters were after a sort the bishop’s coadjutors. The bishops [bishop?] and presbyters who together with him governed the Church, are for the most part by Ignatius jointly mentioned. In the epistle to them of Trallis, he saith of presbyters that they are σύμβουλοι καὶ συνέδρευται του̑ ἐπισκόπου, “counsellors and assistants of the bishop;” and  concludeth in the end, “He that should disobey these were a plain atheist, and an irreligious person, and one that did set Christ himself and his own ordinances at nought.” Which order making presbyters or priests the bishop’s assistants doth not import that they were of equal authority with him, but rather so adjoined that they also were subject, as hath been proved. In the writings of St. Cyprian nothing is more usual than to make mention of the college of presbyters subject unto the bishop, although in handling the common affairs of the Church they assisted him. But of all other places which open the ancient order of episcopal presbyters the most clear is that epistle of Cyprian unto Cornelius, concerning certain Novatian heretics received again upon their conversion into the unity of the Church. “After that Urbanus and Sidonius, confessors, had come and signified unto our presbyters, that Maximus a confessor and presbyter did together with them desire to return into the Church, it seemed meet to hear from their own mouths and confessions that which by message they had delivered. When they were come, and had been called to account by the presbyters touching those things they had committed, their answer was, that they had been deceived, and did request that such things as there they were charged with might be forgotten. It being brought unto me what was done, I took order that the presbytery might be assembled. There were also present five bishops, that upon settled advice it might be with consent of all determined what should be done about their persons.”  Thus far St. Cyprian. Wherein it may be peradventure demanded, whether he and other bishops did thus proceed with advice of their presbyters in all such public affairs of the Church, as being thereunto bound by ecclesiastical canons, or else that they voluntarily so did, because they judged it in discretion as then most convenient. Surely the words of Cyprian are plain, that of his own accord he chose this way of proceeding, “1Unto that,” saith he, “which Donatus, and Fortunatus, and Novatus, and Gordius, our com-presbyters, have written, I could by myself alone make no answer, forasmuch as at the very first entrance into my bishoprick I resolutely determined not to do any thing of mine own private judgment, without your counsel and the people’s consent.” The reason whereof he rendereth in the same epistle, saying, “When by the grace of God myself shall come unto you,” (for St. Cyprian was now in exile,) “of things which either have been or must be done we will consider, sicut honor mutuus poscit, as the law of courtesy which one doth owe to another of us requireth.” And at this very mark doth St. Jerome evermore aim in telling bishops that presbyters were at the first their equals, that in some churches for a long time no bishop was made but only such as the presbyters did choose out amongst themselves, and therefore no cause why the bishop should disdain to consult with them, and in weighty affairs of the Church to use their advice. Sometime to countenance their own actions, or to repress the boldness of proud and insolent spirits, that which bishops had in themselves sufficient authority and power to have done, notwithstanding they would not do alone, but craved therein the aid and assistance of other bishops, as in the case of those Novatian heretics, before alleged, Cyprian himself did. And in Cyprian we find of others the like practice. Rogatian a bishop, having been used contumeliously by a deacon of his own church, wrote thereof his complaint unto Cyprian and other bishops.  In which case their answer was, “That although in his own cause he did of humility rather shew his grievance, than himself take revenge, which by the vigour of his apostolical office and the authority of his chair he might have presently done, without any further delay;” yet if the party should do again as before, their judgments were, “fungaris circa eum potestate honoris tui, et eum vel deponas vel abstineas;”—“use on him that power which the honour of thy place giveth thee, either to depose him or exclude him from access unto holy things.”

[2]The bishop for his assistance and ease had under him, to guide and direct deacons in their charge, his archdeacon, so termed in respect of care over deacons, albeit himself were not deacon but presbyter. For the guidance of presbyters in their function the bishop had likewise under him one of the selfsame order with them, but above them in authority, one whom the ancients termed usually an arch-presbyter, we at this day name him dean. For most certain truth it is that churches cathedral and the bishops of them are as glasses, wherein the face and very countenance of apostolical antiquity remaineth even as yet to be seen, notwithstanding the alterations which tract of time and the course of the world hath brought. For defence and maintenance of them we are most earnestly bound to strive, even as the Jews were for their temple and the high priest of God therein: the overthrow and ruin of the one, if ever the sacrilegious avarice of Atheists should prevail so far, which God of his infinite mercy forbid, ought no otherwise to move us than the people of God were moved, when having beheld the sack and combustion of his sanctuary in most lamentable manner flaming before their eyes, they uttered from the bottom of their grieved spirits those voices of doleful supplication, “Exsurge Domine et miserearis Sion; Servi tui diligunt lapides ejus, pulveris ejus miseret eos.”

\section*{How far the power of Bishops hath reached from the beginning in respect of territory or local compass.}

VIII. How far the power which bishops had did reach, what number of persons was subject unto them at the first,  and how large their territories were, it is not for the question we have in hand a thing very greatly material to know. For if we prove that bishops have lawfully of old ruled over other ministers, it is enough, how few soever those ministers have been, how small soever the circuit of place which hath contained them. Yet hereof somewhat, to the end we may so far forth illustrate church antiquities.

[2]A law imperial there is,
 which sheweth that there was great care had to provide for every Christian city a bishop as near as might be, and that each city had some territory belonging unto it, which territory was also under the bishop of the same city; that because it was not universally thus, but in some countries one bishop had subject unto him many cities and their territories, the law which provided for establishment of the other orders, should not prejudice those churches wherein this contrary custom had before prevailed. Unto the bishop of every such city, not only the presbyters of the same city, but also of the territory thereunto belonging, were from the first beginning subject. For we must note that when as yet there were in cities no parish churches, but only colleges of presbyters under their bishop’s regiment, yet smaller congregations and churches there were even then abroad, in which churches there was but some one only presbyter to perform among them divine duties. Towns and villages abroad receiving the faith of Christ from cities whereunto they were adjacent, did as spiritual and heavenly colonies by their subjection honour those ancient mother churches out of which they grew. And in the Christian cities themselves, when the mighty increase of believers made it necessary to have them divided into certain several companies, and over every of those companies one only pastor to be appointed for the ministry of  holy things;
 between the first and the rest after it there could not but be a natural inequality, even as between the temple and synagogues in Jerusalem. The clergy of cities were termed urbici, to shew a difference between them and the clergies of the towns, of villages, of castles abroad. And how many soever these parishes or congregations were in number, which did depend on any one principal city church, unto the bishop of that one church they and their several sole presbyters were all subject.

[3]For if so be, as some imagine, every petty congregation or hamlet had had his own particular bishop, what sense could there be in those words of Jerome concerning castles, villages, and other places abroad, which having only presbyters to teach them and to minister unto them the sacraments, were resorted unto by bishops for the administration of that wherewith their presbyters were not licensed to meddle. To note a difference of that one church where the bishop hath his seat, and the rest which depend upon it, that one hath usually been termed cathedral, according to the same sense wherein Ignatius speaking of the Church of Antioch termeth it his throne; and Cyprian making mention of Evaristus, who had been bishop and was now deposed, termeth him cathedræ extorrem, one that was thrust besides his chair. The church where the bishop is set with his college of presbyters about him we call a see; the local compass of his authority we term a diocess. Unto a bishop within the compass of his own both see and diocess, it hath by right of his place evermore appertained to ordain presbyters,  to make deacons,
 and with judgment to dispose of all things of weight. The apostle St. Paul had episcopal authority, but so at large that we cannot assign unto him any one certain diocess. His positive orders and constitutions churches every where did obey. Yea, “a charge and a care,” saith he, “I have even of all the churches.” The walks of Titus and Timothy were limited within the bounds of a narrow precinct. As for other bishops, that which Chrysostom hath concerning them, if they be evil, could not possibly agree unto them, unless their authority had reached farther than to some one only congregation. “The danger being so great as it is, to him that scandalizeth one soul, what shall he,” saith Chrysostom, speaking of a bishop, “what shall he deserve, by whom so many souls, yea, even whole cities and peoples, men, women, and children, citizens, peasants, inhabitants, both of his own city, and of other towns subject unto it, are offended?” A thing so unusual it was for a bishop not to have ample jurisdiction, that Theophilus, patriarch of Alexandria, for making one a bishop of a small town, is noted as a proud despiser of the commendable orders of the Church with this censure: “Such novelties Theophilus presumed every where to begin, taking upon him, as it had been, another Moses.”

[4]Whereby is discovered also their error, who think that such as in ecclesiastical writings they find termed Chorepiscopos were the same in the country which the bishop was in the city: whereas the old Chorepiscopi are they that were appointed of the bishopa to have, as his vicegerentsb, some oversight of those churches abroad, which were subject unto his  see;
 in which churches they had also power to make subdeacons, readers, and such like petty church officers. With which power so stinted, they not contenting themselves, but adventuring at the length to ordain even deacons and presbyters also, as the bishop himself did, their presumption herein was controlled and stayed by the ancient edict of councils. For example that of Antioch, “It hath seemed good to the holy synod that such in towns and countries as are called Chorepiscopi do know their limits and govern the churches under them, contenting themselves with the charge thereof, and with authority to make readers, sub-deacons, exorcists, and to be leaders or guiders of them; but not to meddle with the ordination either of a presbyter or of a deacon, without the bishop of that city, whereunto the Chorepiscopus and his territory also is subject.” The same synod appointed likewise that those Chorepiscopi shall be made by none but the bishop of that city under which they are. Much might hereunto be added, if it were further needful to prove that the local compass of a bishop’s authority and power was never so straitly listed, as some men would have the world to imagine.

[5]But to go forward; degrees there are and have been of old even amongst bishops also themselves; one sort of bishops being superiors unto presbyters only, another sort having preeminence also above bishops. It cometh here to be considered in what respect inequality of bishops was thought at the first a thing expedient for the Church, and what odds there hath been between them, by how much the power of one hath been larger, higher, and greater than of another. Touching the causes for which it hath been esteemed meet that bishops themselves should not every way be equals; they are the same for which the wisdom both of God and man hath evermore approved it as most requisite, that where many governors must of necessity concur for the ordering of the same affairs, of what  nature soever they be, one should have some kind of sway or stroke more than all the residue. For where number is, there must be order, or else of force there will be confusion. Let there be divers agents, of whom each hath his private inducements with resolute purpose to follow them (as each may have); unless in this case some had preeminence above the rest, a chance it were if ever any thing should be either begun, proceeded in, or brought unto any conclusion by them; deliberations and counsels would seldom go forward, their meetings would always be in danger to break up with jars and contradictions. In an army a number of captains, all of equal power, without some higher to oversway them; what good would they do? In all nations where a number are to draw any one way, there must be some one principal mover.

Let the practice of our very adversaries themselves herein be considered; are the presbyters able to determine of church affairs, unless their pastors do strike the chiefest stroke and have power above the rest? Can their pastoral synod do any thing, unless they have some president amongst them? In synods they are forced to give one pastor preeminence and superiority above the rest. But they answer, that he who being a pastor according to the order of their discipline is for the time some little deal mightier than his brethren, doth not continue so longer than only during the synod. Which answer serveth not to help them out of the briers; for by their practice they confirm our principle touching the necessity of one man’s preeminence wheresoever a concurrency of many is required unto any one solemn action: this nature teacheth, and this they cannot choose but acknowledge. As for the change of his person to whom they give this preeminence, if they think it expedient to make for every synod a new superior,  there is no law of God which bindeth them so to [do]c;
 neither any that telleth them that they might [not?] suffer one and the same man being made president even to continue so during life, and to leave his preeminence unto his successors after him, as by the ancient order of the Church, archbishops, presidents amongst bishops, have used to do.

[6]The ground therefore of their preeminence above bishops is the necessity of often concurrency of many bishops about the public affairs of the Church, as consecrations of bishops, consultations of remedy of general disorders, audience judicial, when the actions of any bishop should be called in question, or appeals are made from his sentence by such as think themselves wronged. These and the like affairs usually requiring that many bishops should orderly assemble, begin, and conclude somewhat; it hath seemed in the eyes of reverend antiquity a thing most requisite, that the Church should not only have bishops, but even amongst bishops some to be in authority chiefest.

[7]Unto which purpose, the very state of the whole world, immediately before Christianity took place, doth seem by the special providence of God to have been prepared. For we must know, that the countries where the Gospel was first planted, were for the most part subject to the Roman empire. The Romans’ use was commonly, when by war they had subdued foreign nations, to make them provinces, that is, to place over them Roman governors, such as might order them according to the laws and customs of Rome. And, to the end that all things might be the more easily and orderly done, a whole country being divided into sundry parts, there was in each part some one city, whereinto they about did resort for justice. Every such part was termed a diocess. Howbeit, the name diocess is sometime so generally taken, that it containeth  not only mored such parts of a province,
 but even more provinces also than one; as the diocess of Asia contained eight, the diocess of Africa seven. Touching diocesses according unto a stricter sense, whereby they are taken for a part of a province, the words of Livy do plainly shew what order the Romans did observe in them. For at what time they had brought the Macedonians into subjection, the Roman governor, by order from the senate of Rome, gave charge that Macedonia should be divided into four regions or diocesses. “Capita regionum ubi concilia fierent, primæ sedis Amphipolim, secundæ Thessalonicen, tertiæ Pellam, quartæ Pelagoniam fecit. Eo concilia suæ cujusque regionis indici, pecuniam conferri, ibi magistratus creari jussit.” This being before the days of the emperors, by their appointment Thessalonica was afterwards the chiefest, and in it the highest governor of Macedonia had his seat. Whereupon the other three diocesses were in that respect inferior unto it, as daughters unto a mother city; for not unto every town of justice was that title given, but was peculiar unto those cities wherein principal courts were kept. Thus in Macedonia the mother city was Thessalonica; in Asia, Ephesus; in Africa, Carthage; for so Justinian in his time made it. The governors, officers, and inhabitants of these mother cities were termed for difference’ sake metropolites, that is to say, mother city men; than which nothing could possibly have been devised more fit to suit with the nature of that form of spiritual regiment under which afterward the Church should live.

Wherefore if the prophet saw cause to acknowledge unto  the Lord that the light of his gracious providence did shine no where more apparently to the eye than in preparing the land of Canaan to be [a]e receptacle for that Church which was of old,
 “Thou hast brought a vine out of Egypt, thou hast cast out the heathen and planted it, thou madest room for it, and when it had taken root it filled the land:” how much more ought we to wonder at the handy-work of Almighty God who to settle the kingdom of his dear Son did not cast out any one people, but directed in such sort the politic counsels of them who ruled far and wide over all, that they throughout all nations, people and countries upon earth, should unwittingly prepare the field wherein the vine which God did intend, that is to say, the Church of his dearly-beloved Son was to take root? For unto nothing else can we attribute it, saving only unto the very incomprehensible force of Divine providence, that the world was in so marvellous fit sort divided, levelled and laid out before-hand. Whose work could it be but his alone to make such provision for the direct implantation of his Church?

[8]Wherefore inequality of Bishops being found a thing convenient for the Church of God, in such consideration as hath been shewed, when it came secondly in question which bishops should be higher and which lower, it seemed herein not to the civil monarch only, but to the most, expedient that the dignity and celebrity of mother cities should be respected. They which dream that if civil authority had not given such preeminence unto one city more than another, there had never grown an inequality amongst bishops, are deceived: superiority of one bishop over another would be requisite in the Church although that civil distinction were abolished: other causes having made it necessary even amongst bishops to have some in degree higher than the rest, the civil dignity of place was considered only as a reason wherefore this bishop should be preferred before that: which deliberation had been likely enough to have raised no small trouble,  but that such was the circumstance of place,
 as being followed in that choice, besides the manifest conveniency thereof, took away all show of partiality, prevented secret emulations, and gave no man occasion to think his person disgraced in that another was preferred before him.

[9]Thus we see upon what occasion metropolitan bishops became archbishops. Now while the whole Christian world in a manner still continued under one civil government, there being oftentimes within some one more large territory divers and sundry mother churches, the metropolitans whereof were archbishops; as for order’s sake it grew hereupon expedient there should be a difference also amongst them, so no way seemed in those times more fit than to give preeminence unto them whose metropolitan sees were of special desert or dignity: for which cause these as being bishops in the chiefest mother churches were termed primates, and at the length by way of excellency, patriarchs. For ignorant we are not, how sometimes the title of patriarch is generally given to all metropolitan bishops.

They are mightily therefore to blame which are so bold and confident, as to affirm that for the space of above four hundred and thirty years after Christ, all metropolitan bishops were in every respect equals, till the second council of Constantinople  exalted certain metropolitans above the rest. True it is, they were equals as touching the exercise of spiritual power within their diocesses, when they dealt with their own flock. For what is it that one of them might do within the compass of his own precinct, but another within his might do the same? But that there was no subordination at all of one of them unto another; that when they all, or sundry of them, were to deal in the same causes, there was no difference of first and second in degree, no distinction of higher and lower in authority acknowledged amongst them; is most untrue.

The great council of Nice was after our Saviour Christ but three hundred twenty-four years, and in that council certain metropolitans are said even then to have had an ancient preeminence and dignity above the rest; namely the primate of Alexandria, of Rome, and of Antioch. Threescore years after this there were synods under the emperor Theodosius; which synod was the first at Constantinople, whereat one hundred and fifty bishops were assembled: at which council it was decreed that the bishop of Constantinople should not only be added unto the former primates, but also that his place should be second amongst them, the next to the bishop of Rome in dignity. The same decree again renewed concerning Constantinople, and the reason thereof laid open in the council of Chalcedon: at the length came that second  of Constantinople,
 whereat were six hundred and thirty bishops, for a third confirmation thereof. Laws imperial there are likewise extant to the same effect. Herewith the bishop of Constantinople being overmuch puffed up, not only could not endure that see to be in estimation higher, whereunto his own had preferment to be the next, but he challenged more than ever any Christian bishop in the world before either had, or with reason could have. What he challenged, and was therein as then refused by the bishop of Rome, the same the bishop of Rome in process of time obtained for himself, and having gotten it by bad means, hath both upheld and augmented it, and upholdeth it by acts and practices much worse.

[10]But primates, according to their first institution, were all, in relation unto archbishops, the same by prerogative which archbishops were being compared unto bishops. Before  the council of Nice, albeit there were both metropolitans and primates, yet could not this be a means forcible enough to procure the peace of the Church, but all things were wonderful tumultuous and troublesome, by reason of one special practice common unto the heretics of those times; which was, that when they had been condemned and cast out of the Church by the sentence of their own bishops, they contrary to the ancient received orders of the Church, had a custom to wander up and down, and to insinuate themselves into favour where they were not known, imagining themselves to be safe enough, and not to be clean cut off from the body of the Church, if they could any where find a bishop which was content to communicate with them; whereupon ensued, as in that case there needs must, every day quarrels and jars unappeasable amongst bishops. The Nicene council for redress hereof considered the bounds of every archbishop’s ecclesiastical jurisdiction, what they had been in former times, and accordingly appointed unto each grand part of the Christian world some one primate, from whose judgment no man living within his territory might appeal, unless it were to a council general of all bishops. The drift and purpose of which order was, that neither any man oppressed by his own particular bishop might be destitute of a remedy through appeal unto the more indifferent sentence of some other ordinary judge; nor yet every man be left at such liberty as before, to shift himself out of their hands for whom it was most meet to have the hearing and determining of his cause. The evil, for remedy whereof this order was taken, annoyed at that present especially the church of Alexandria in Egypt, where Arianism begun. For which cause the state of that church is in the Nicene canons concerning this matter mentioned before the rest. The words of their sacred edict are these: “Let those customs remain in force which have been of old, the  customs of Egypt and Libya,
 and Pentapolis; by which customs the bishop of Alexandria hath authority over all these; the rather for that this hath also been the use of the bishop of Rome, yea the same hath been kept in Antioch and in other provinces.” Now because the custom likewise had been that great honour should be done to the bishop of Ælia or Jerusalem, therefore lest their decree concerning the primate of Antioch should any whit prejudice the dignity and honour of that see, special provision is made, that although it were inferior in degree, not only unto Antioch the chief of the East, but even unto Cæsarea too, yet such preeminence it should retain as belonged to a mother city, and enjoy whatsoever special prerogative or privilege it had besides. Let men therefore hereby judge of what continuance this order which upholdeth degrees of bishops must needs have been, when a general council of three hundred and eighteen bishops living themselves within three hundred years after Christ doth reverence the same for antiquity’s sake, as a thing which had been even then of old observed in the most renowned parts of the Christian world.

[11]Wherefore needless altogether are those vain and wanton demands, “No mention of an archbishop in Theophilus bishop of Antioch? None in Ignatius? None in Clemens of Alexandria? None in Justin Martyr, Irenæus, Tertullian, Cyprian? None in all those old historiographers, out of which Eusebius gathereth his story? None till the time of the council of Nice, three hundred and twenty years after Christ?” As if the mention which is thereof made in that very council, where so many bishops acknowledge  archiepiscopal dignity even then ancient, were not of far more weight and value than if every of those Fathers had written large discourses thereof. But what is it which they will blush at, who dare so confidently set it down, that in the council of Nice some bishops being termed metropolitans, no more difference is thereby meant to have been between one bishop and another, than is shewed between one minister and another, when we say such a one is a minister in the city of London, and such a one minister in the town of Newington? So that to be termed a metropolitan bishop did in their conceit import no [moref] preeminence above other bishops, than we mean that a girdler hath over others of the same trade, if we term him which doth inhabit some mother city for difference’ sake a metropolitan girdler.

But the truth is too manifest to be so deluded; a bishop at that time had power in his own diocess over all other ministers there, and a metropolitan bishop sundry preeminences above other bishops, one of which preeminences was in the ordination of bishops, to have κυ̑ρος τω̑ν γινομένων, the chief power of ordering all things done. Which preeminence that council itself doth mention, as also a greater belonging unto the patriarch or primate of Alexandria, concerning whom it is there likewise said, that to him did belong ἐξουσία, authority and power over all Egypt, Pentapolis, and Libya: within which compass sundry metropolitan sees to have been,  there is no man ignorant,
 which in those antiquities have [hath?] any knowledge.

[12]Certain prerogatives there are wherein metropolitans excelled other bishops, certain also wherein primates excelled other metropolitans. Archiepiscopal or metropolitan prerogatives are those mentioned in old imperial constitutions, to convocate the holy bishops under them within the compass of their own provinces, when need required their meeting together for inquisition and redress of public disorders; to grant unto bishops under them leave and faculty of absence from their own diocesses, when it seemed necessary that they should otherwhere converse for some reasonable while; to give notice unto bishops under them of things commanded by supreme authority; to have the hearing and first determining of such causes as any man had against a bishop; to receive the appeals of the inferior clergy, in case they found themselves overborne by the bishop their immediate judge. And lest haply it should be imagined that canons ecclesiastical we want to make the selfsame thing manifest; in the council of Antioch  it was thus decreed:
 “The bishops in every province must know, that he which is bishop in the mother city hath not only charge of his own parish or diocess, but even of the whole province also.” Again: “It hath seemed good that other bishops without him should do nothing more than only that which concerns each one’s parish and the places underneath it.” Further by the selfsame council all councils provincial are reckoned void and frustrate, unless the bishop of the mother city within that province where such councils should be, were present at them. So that the want of his presence, and in canons for church-government, want of his approbation also, did disannul them: not so the want of any others. Finally, concerning elections of bishops, the council of Nice hath this general rule, that the chief ordering of all things here, is in every province committed to the metropolitan.

[13]Touching them, who amongst metropolitans were also primates, and had of sundry united provinces the chiefest metropolitan see, of such that canon in the council of Carthage was eminent, whereby a bishop is forbidden to go beyond seas without the license of the highest chair within the same bishop’s own country; and of such which beareth the name of apostolical, is that ancient canon likewise, which chargeth the bishops of each nation, to know him which is first amongst them, and to esteem of him as an head, and to do no extraordinary thing but with his leave. The chief primates of the Christian world were the bishops of Rome, Alexandria, and Antioch. To whom the bishop of Constantinople being afterwards added, St. Chrysostom the bishop of that see is in  that respect said to have had the care and charge not only of the city of Constantinople,
 “sed etiam totius Thraciæ, quæ sex præfecturis est divisa, et Asiæ totius, quæ ab undecim præsidibus regitur.” The rest of the East was under Antioch, the South under Alexandria, and the West under Rome. Whereas therefore John the bishop of Jerusalem being noted of heresy, had written an apology for himself unto the bishop of Alexandria, named Theophilus; St. Jerome reproveth his breach of the order of the Church herein, saying, “Tu qui regulas quæris ecclesiasticas, et Niceni concilii canonibus uteris, responde mihi, ad Alexandrinum episcopum Palæstina quid pertinet? Ni fallor, hoc ibi decernitur, ut Palæstinæ metropolis Cæsarea sit, et totius Orientis Antiochia. Aut igitur ad Cæsariensem episcopum referre debueras; aut si procul expetendum judicium erat, Antiochiam potius literæ dirigendæ.” Thus much concerning that Local Compass which was anciently set out to bishops; within the bounds and limits whereof we find that they did accordingly exercise that episcopal authority and power which they had over the Church of Christ.

\section*{In what respects episcopal regiment hath been gainsaid of old by Aërius.}

IX. The first whom we read to have bent themselves against the superiority of bishops were Aërius and his followers. Aërius seeking to be made a bishop, could not brook that Eustathius was thereunto preferred before him. Whereas therefore he saw himself unable to rise to that greatness which his ambitious pride did affect, his way of revenge was to try what wit being sharpened with envy and malice could do in raising a new seditious opinion, that the superiority which bishops had was a thing which they should not have, that a bishop might not ordain, and that a bishop ought not any way to be distinguished from a presbyter. For so doth St. Augustine deliver the opinion of Aërius:  Epiphanius not so plainly nor so directly,
 but after a more rhetorical sort. “His speech was rather furious than convenient for man to use: What is,” saith he, “a bishop more than a presbyter? The one doth differ from the other nothing. For their order is one, their honour one, one their dignity. A bishop imposeth his hands, so doth a presbyter. A bishop baptizeth, the like doth a presbyter. The bishop is a minister of divine service, a presbyter is the same. The bishop sitteth as judge in a throne, even the presbyter sitteth also.” A presbyter therefore doing thus far the selfsame thing which a bishop did, it was by Aërius enforced that they ought not in any thing to differ.

[2]Are we to think Aërius had wrong in being judged an heretic for holding this opinion? Surely if heresy be an error falsely fathered upon Scriptures, but indeed repugnant to the truth of the Word of God, and by the consent of the universal Church, in the councils, or in her contrary uniform practice throughout the whole world, declared to be such; and the opinion of Aërius in this point be a plain error of that nature: there is no remedy, but Aërius, so schismatically and stiffly maintaining it, must even stand where Epiphanius and Augustine have placed him. An error repugnant unto the truth of the Word of God is held by them, whosoever they be, that stand in defence of any conclusion drawn erroneously out of Scripture, and untruly thereon fathered. The opinion of Aërius therefore being falsely collected out of Scripture, must needs be acknowledged an error repugnant unto the truth of the word of God. His opinion was that there ought not to be any difference between a bishop and a presbyter. His grounds and reasons for this opinion were sentences of Scripture. Under pretence of which sentences, whereby it seemed that bishops and presbyters at the first did  not differ,
 it was concluded by Aërius that the Church did ill in permitting any difference to be made.

[3]The answer which Epiphanius maketh unto some part of the proofs by Aërius alleged, was not greatly studied or laboured; for through a contempt of so base an error (for this himself did perceive and profess) yieldeth he thereof expressly this reason: Men that have wit do evidently see that all this is mere foolishness. But how vain and ridiculous soever his opinion seemed unto wise men, with it Aërius deceived many; for which cause somewhat was convenient to be said against it. And in that very extemporal slightness which Epiphanius there useth, albeit the answer made to Aërius be in part but raw, yet ought not hereby the truth to find any less favour than in other causes it doth, where we do not therefore judge heresy to have the better, because now and then it allegeth that for itself, which defenders of the truth do not always so fully answer. Let it therefore suffice, that Aërius did bring nothing unanswerable. The weak solutions which the one doth give, are to us no prejudice against the cause, as long as the other’s oppositions are of no greater strength and validity. Did not Aërius, trow you, deserve to be esteemed as a new Apollos, mighty and powerful in the word, which could for maintenance of his cause bring forth so plain divine authorities, to prove by the Apostles’ own writings that bishops ought not in any thing to differ from other presbyters? For example, where it is said that presbyters made Timothy bishop, is it not clear that a bishop should not differ from a presbyter, by having power of ordination? Again, if a bishop might by order be distinguished  from a presbyter,
 x. . would the Apostle have given as he doth unto presbyters the title of bishops? These were the invincible demonstrations wherewith Aërius did so fiercely assault bishops.

[4]But the sentence of Aërius perhaps was only, that the difference between a bishop and a presbyter hath grown by the order and custom of the Church, the word of God not appointing that any such difference should be. Well, let Aërius then find the favour to have his sentence so construed; yet his fault in condemning the order of the Church, his not submitting himself unto that order, the schism which he caused in the Church about it, who can excuse? No, the truth is, that these things did even necessarily ensue, by force of the very opinion which he and his followers did hold. His conclusion was, that there ought to be no difference between a presbyter and a bishop. His proofs, those Scripture sentences which make mention of bishops and presbyters without any such distinction or difference. So that if between his conclusion and the proofs whereby he laboured to strengthen the same, there be any show of coherence at all, we must of necessity confess, that when Aërius did plead, There is by the Word of God no difference between a presbyter and a bishop, his meaning was not only, that the Word of God itself appointeth not, but that it enforceth on us the duty of not appointing nor allowing that any such difference should be made.

\section*{In what respects episcopal regiment is gainsaid by the authors of pretended reformation at this day.}

X. And of the selfsame mind are the enemies of government by bishops, even at this present day. They hold as Aërius did, that if Christ and his Apostles were obeyed, a bishop should not be permitted to ordain; that between a presbyter and a bishop the word of God alloweth not any inequality or difference to be made; that their order, their authority, their power, ought to be one; that it is but by usurpation and corruption that the one sort are suffered to have rule of the other, or to be any way superior unto them. Which opinion having now so many defenders, shall never  be able while the world doth stand to find in some [so many?],
 xi. . believing antiquity, as much as one which hath given it countenance, or borne any friendly affection towards it.

[2]Touching these men therefore, whose desire is to have all equal, three ways there are whereby they usually oppugn the received order of the Church of Christ. First, by disgracing the inequality of pastors, as a new and mere human invention, a thing which was never drawn out of Scripture, where all pastors are found (they say) to have one and the same power both of order and jurisdiction: Secondly, by gathering together the differences between that power which we give to bishops, and that which was given them of old in the Church; so that albeit even the ancient took more than was warrantable, yet so far they swerved not as ours have done: Thirdly, by endeavouring to prove, that the Scripture directly forbiddeth, and that the judgment of the wisest, the holiest, the best in all ages, condemneth utterly the inequality which we allow.

\section*{Their arguments in disgrace of regiment by Bishops, as being a mere invention of man, and not found in Scripture, answered.}

XI. That inequality of pastors is a mere human invention, a thing not found in the word of God, they prove thus:

i. “All the places of Scripture where the word Bishop is used, or any other derived of that name, signify an oversight in respect of some particular congregation only, and never in regard of pastors committed unto his oversight. For which cause the names of bishops, and presbyters, or pastoral elders, are used indifferently, to signify one and the selfsame thing. Which so indifferent and common use of these words for one and the selfsame office, so constantly and perpetually in all places, declareth that the word Bishop in the Apostles’ writing importeth not a pastor of higher power and authority over other pastors.”

ii. “All pastors are called to their office by the same means of proceeding; the Scripture maketh no difference in the manner of their trial, election, ordination: which proveth their office and power to be by Scripture all one.”

iii. “The Apostles were all of equal power, and all pastors do alike succeed the Apostles in their ministry and power, the commission and authority whereby they succeed being  in Scripture but one and the same that was committed to the Apostles, without any difference of committing to one pastor more, or to another less.”


iv. “The power of the censures and keys of the Church, and of ordaining and ordering ministers (in which two points especially this superiority is challenged), is not committed to any one pastor of the Church more than to another; but the same is committed as a thing to be carried equally in the guidance of the Church. Whereby it appeareth, that Scripture maketh all pastors, not only in the ministry of the word and sacraments, but also in all ecclesiastical jurisdiction and authority, equal.”

v. “The council of Nice doth attribute this difference, not unto any ordination of God, but to an ancient custom used in former times, which judgment is also followed afterwards by other councils: Concil. Antioch. cap. 9.”

vi. Upon these premises, their summary collection and conclusion is, “That the ministry of the Gospel, and the functions thereof, ought to be from heaven and of God (John i. 23); that if they be of God, and from heaven, then are they set down in the word of God; that if they be not in the word of God, (as by the premises it doth appear, they say, that our kind of bishops are not,) it followeth, they are invented by the brain of men, and are of the earth, and that consequently they can do no good in the Church of Christ, but harm.”

Answer.[2]Our answer hereunto is, first, that their proofs are unavailable to shew that Scripture affordeth no evidence for the inequality of pastors: Secondly, that albeit the Scripture did no way insinuate the same to be God’s ordinance, and  the Apostles to have brought it in,
 albeit the Church were acknowledged by all men to have been the first beginner thereof a long time after the Apostles were gone; yet is not the authority of bishops hereby disannulled, it is not hereby proved unfit or unprofitable for the Church.

[3]First, that the word of God doth acknowledge no inequality of power amongst pastors of the Church, neither doth it appear by the signification of this word bishop, nor by the indifferent use thereof.

For concerning signification, first it is clearly untrue, that no other thing is thereby signified, but only an oversight in respect of a particular church and congregation. For, I beseech you, of what parish or particular congregation was Matthias bishop? his office Scripture doth term episcopal: which being no other than was common unto all the Apostles of Christ, forasmuch as in that number there is not any to whom the oversight of many pastors did not belong by force and virtue of that office; it followeth that the very word doth sometimes even in Scripture signify an oversight, such as includeth charge over pastors themselves.

And if we look to the use of the word, being applied with reference unto some one church, as Ephesus, Philippi, and such like, albeit the guides of those churches be interchangeably in Scripture termed sometime bishops, sometime presbyters, to signify men having oversight and charge, without relation at all unto other than the Christian laity alone; yet this doth not hinder, but that Scripture may in some place have other names, whereby certain of those presbyters or bishops are noted to have the oversight and charge of pastors, as out of all peradventure they had whom St. John doth entitle angels.

[4]Secondly, as for those things which the Apostle hath set down concerning trial, election, and ordination of pastors, that he maketh no difference in the manner of their calling, this also is but a silly argument to prove their office and their power equal by the Scripture. The form of admitting each sort unto their offices, needed no particular instruction: there was no fear, but that such matters of course would easily enough be observed. The Apostle therefore toucheth those  things wherein judgment,
 wisdom and conscience is required, he carefully admonisheth of what quality ecclesiastical persons should be, that their dealing might not be scandalous in the Church. And forasmuch as those things are general, we see that of deacons there are delivered in a manner the selfsame precepts which are given concerning pastors, so far as concerneth their trial, election, and ordination. Yet who doth hereby collect that Scripture maketh deacons and pastors equal?

If notwithstanding it be yet demanded, “Wherefore he which teacheth what kind of persons deacons and presbyters should be, hath nothing in particular about the quality of chief presbyters, whom we call bishops?” I answer briefly, that there it was no fit place for any such discourse to be made, inasmuch as the Apostle wrote unto Timothy and Titus, who having by commission episcopal authority, were to exercise the same in ordaining, not bishops (the apostles themselves yet living, and retaining that power in their own hands) but presbyters, such as the apostles at the first did create throughout all churches. Bishops by restraint (only James at Jerusalem excepted) were not yet in being.

[5]Thirdly, about equality amongst the apostles there is by us no controversy moved. If in the rooms of the apostles, which were of equal authority, all pastors do by Scripture succeed alike, where shall we find a commission in Scripture which they speak of, which appointed all to succeed in the selfsame equality of power, except that commission which doth authorize to preach and baptize should be alleged, which maketh nothing to the purpose, for in such things all pastors are still equal. We must, I fear me, wait very long before any other will be shewed. For howsoever the Apostles were equals amongst themselves, all other pastors were not equals with the Apostles while they lived, neither are they any where appointed to be afterward each other’s equal. Apostles had, as we know, authority over all such as were no Apostles; by force of which their authority they might both command and judge. It was for the singular good and benefit of those disciples whom Christ left behind him, and of the pastors which were afterwards chosen; for the great  good,
 I say, of all sorts, that the Apostles were in power above them. Every day brought forth somewhat wherein they saw by experience, how much it stood them in stead to be under controlment of those superiors and higher governors of God’s house. Was it a thing so behoveful that pastors should be subject unto pastors in the Apostles’ own times? and is there any commandment that this subjection should cease with them, and that the pastors of the succeeding ages should be all equals? No, no, this strange and absurd conceit of equality amongst pastors (the mother of schism and of confusion) is but a dream newly brought forth, and seen never in the Church before.

[6]Fourthly, power of censure and ordination appeareth even by Scripture marvellous probable to have been derived from Christ to his Church, without this surmised equality in them to whom he hath committed the same. For I would know whether Timothy and Titus were commanded by St. Paul to do any thing more than Christ hath authorized pastors to do? And to the one it is Scripture which saith, “Against a presbyter receive thou no accusation, saving under two or three witnesses;” Scripture which likewise hath said to the other, “For this very cause left I thee in Crete, that thou shouldest redress the things that remain, and shouldest ordain presbyters in every city, as I appointed thee.” In the former place the power of censure is spoken of, and the power of ordination in the latter. Will they say that every pastor there was equal to Timothy and Titus in these things? If they do, the Apostle himself is against it, who saith that of their two very persons he had made choice, and appointed in those places them, for performances of those duties: whereas if the same had belonged unto others no less than to them, and not principally unto them above others, it had been fit for the Apostle accordingly to have directed his letters concerning these things in general unto them all which had equal interest in them; even as it had been likewise fit to have written those epistles in St. John’s Revelation unto whole ecclesiastical senates, rather than only unto the angels of each church, had not some one been above the rest in authority to order the affairs of the church. Scripture therefore doth most probably make for the  inequality of pastors, even in all ecclesiastical affairs,
 and by very express mention as well in censures as ordinations.

[7]Fifthly, In the Nicene council there are confirmed certain prerogatives and dignities belonging unto primates or archbishops, and of them it is said that the ancient custom of the Church had been to give them such preeminence, but no syllable whereby any man should conjecture that those fathers did not honour [did honour?] the superiority which bishops had over other pastors only upon ancient custom, and not as a true apostolical, heavenly, and divine ordinance.

[8]Sixthly, Now although we should leave the general received persuasion held from the first beginning, that the Apostles themselves left bishops invested with power above other pastors; although, I say, we should give over this opinion, and embrace that other conjecture which so many have thought good to follow, and which myself did  sometimes judge a great deal more probable than now I do,
 merely that after the Apostles were deceased, churches did  agree amongst themselves for preservation of peace and order, to make one presbyter in each city chief over the rest, and to translate into him that power by force and virtue whereof the Apostles, while they were alive,
 did preserve and uphold order in the Church, exercising spiritual jurisdiction partly by themselves and partly by evangelists, because they could not always every where themselves be present: this order taken by the Church itself (for so let us suppose that the Apostles did neither by word nor deed appoint it) were notwithstanding more warrantable than that it should give place and be abrogated, because the ministry of the Gospel and the functions thereof ought to be from heaven.

[9]There came chief priests and elders unto our Saviour Christ as he was teaching in the temple, and the question which they moved unto him was this, “By what authority doest thou these things, and who gave thee this authority?” Their question he repelled with a counter-demand, “The baptism of John, whence was it, from heaven, or of men?” Hereat they paused, secretly disputing within themselves, “If we shall say, From heaven, he will ask, Wherefore did ye not then believe him? and if we say, Of men, we fear the people, for all hold John a prophet.” What is it now which hereupon these men would infer? That all functions ecclesiastical ought in such sort to be from heaven, as the function of John was? No such matter here contained. Nay, doth not the contrary rather appear most plainly by that which is here set down? For when our Saviour doth ask concerning the baptism, that is to say the whole spiritual function, of John, whether it were “from heaven, or of men,” he giveth clear to understand that men give authority unto  some,
 and some God himself from heaven doth authorize. Nor is it said, or in any sort signified, that none have lawful authority which have it not in such manner as John, from heaven. Again when the priests and elders were loth to say that John had his calling from men, the reason was not because they thought that so John should not have had any good or lawful calling, but because they saw that by this means they should somewhat embase the calling of John; whom all men knew to have been sent from God, according to the manner of prophets, by a mere celestial vocation. So that out of the evidence here alleged, these things we may directly conclude: first that whoso doth exercise any kind of function in the Church, he cannot lawfully so do except authority be given him; secondly that if authority be not given him from men, as the authority of teaching was given unto Scribes and Pharisees, it must be given him from heaven, as authority was given unto Christ, Elias, John Baptist, and the prophets. For these two only ways there are to have authority. But a strange conclusion it is, God himself did from heaven authorize John to bear witness of the light, to prepare a way for the promised Messias, to publish the nearness of the kingdom of God, to preach repentance, and to baptize (for by this part, which was in the function of John most noted, all the rest are together signified), therefore the Church of God hath no power upon new occurrences to appoint, to ordain an ecclesiastical function, as Moses did upon Jethro’s advice devise a civil.

[10]All things we grant which are in the Church ought to be of God. But forasmuch as they may be two ways accounted such, one if they be of his own institution and not of ours, another if they be of ours, and yet with his approbation: this latter way there is no impediment but that the same thing which is of men may be also justly and truly said to be of God, the same thing from heaven which is from earth. Of all good things God himself is author, and consequently an approver of them. The rule to discern when the actions of men are good, when they are such as they ought to be, is more ample and large than the law which God hath set particular down in his holy word; the Scripture is but a part of that rule, as hath been heretofore at  large declared. If therefore all things be of God which are well done, and if all things be well done which are according to the rule of well-doing, and if the rule of well-doing be more ample than the Scripture: what necessity is there, that every thing which is of God should be set down in holy Scripture? True it is in things of some one kind; true it is that what we are now of necessity for ever bound to believe or observe in the special mysteries of salvation, Scripture must needs give notice of it unto the world; yet true it cannot be, touching all things that are of God. Sufficient it is for the proof of lawfulness in any thing done, if we can shew that God approveth it. And of his approbation the evidence is sufficient, if either himself have by revelation in his word warranted it, or we by some discourse of reason find it good of itself, and unrepugnant unto any of his revealed laws and ordinances. Wherefore injurious we are unto God, the author and giver of human capacity, judgment, and wit, when because of some things wherein he precisely forbiddeth men to use their own inventions, we take occasion to disauthorize and disgrace the works which he doth produce by the hand either of nature or of grace in them. We offer contumely even unto him, when we scornfully reject what we list, without any other exception than this, “The brain of man hath devised it.” Whether we look into the church or commonweal, as well in the one as in the other, both the ordination of officers, and the very institution of their offices may be truly derived from God, and approved of him, although they be not always of him in such sort as those things are which are in Scripture. Doth not the Apostle term the law of nature, even as the evangelist doth the law of Scripture, δικαίωμα του̑ Θεου̑, God’s own righteous ordinance? The law of nature then being his law, that must needs be of him which it hath directed men unto. Great odds I grant there is between things devised by men, although agreeable with the law of nature, and things in Scripture set down by the finger of the Holy Ghost. Howbeit the dignity of these is no hinderance, but that those be also reverently accounted of in their place.


[11]Thus much they very well saw, who although not living themselves under this kind of church polity, yet being through some experience more moderate, grave and circumspect in their judgment, have given hereof their sounder and better advised sentence. “That which the holy Fathers,” saith Zanchius, “have by common consent without contradiction of Scripture received, for my part I neither will nor dare with good conscience disallow. And what more certain than that the ordering of ecclesiastical persons, one in authority above another, was received into the church by the common consent of the Christian world? What am I that I should take upon me to control the whole Church of Christ in that which is so well known to have been lawfully, religiously, and to notable purpose instituted?”

Calvin making mention even of primates that have authority above bishops: “It was,” saith he, “the institution of the ancient church, to the end that the bishops might by this bond of concord continue the faster linked amongst themselves.” And lest any man should think that as well he might allow the papacy itself, to prevent this he addeth, “Aliud est moderatum gerere honorem, quam totum terrarum orbem immenso imperio complecti.”


These things standing as they do, we may conclude, that albeit the offices which bishops execute had been committed unto them only by the Church, and that the superiority which they have over other pastors were not first by Christ himself given to the Apostles, and from them descended to others, but afterwards in such consideration brought in and agreed upon as is pretended; yet could not this be a just or lawful exception against it.

\section*{Their arguments to prove there was no necessity of instituting Bishops in the Church.}

XII. But they will say, “There was no necessity of instituting bishops; the Church might have stood well enough without them; they are as those superfluous things, which neither while they continue do good, nor do harm when they are removed, because there is not any profitable use whereunto they should serve. For first, in the primitive Church their pastors were all equal, the bishops of those days were the very same which pastors of parish churches at this day are with us, no one at commandment or controlment by any other’s authority amongst them. The Church therefore may stand and flourish without bishops. If they be necessary, wherefore were they not sooner instituted?

“Again, if any such thing were needful for the Church, Christ would have set it down in Scripture, as he did all kind of officers needful for Jewish regiment. He which prescribed unto the Jews so particularly the least thing pertinent unto their temple, would not have left so weighty offices undetermined of in Scripture, but that he knew the Church could never have any profitable use of them.”

“Furthermore, it is the judgment of Cyprian, that equity requireth every man’s cause to be heard, where the fault he is charged with was committed: and the reason he allegeth is, forasmuch as there they may have both accusers and witnesses in their cause. Sith therefore every man’s cause is meetest to be handled at home by the judges of his own parish, to what purpose serveth their device, which  have appointed bishops unto whom such causes may be brought,
 and archbishops to whom they may be also from thence removed?”

\section*{The forealleged arguments answered.}

XIII. What things have necessary use in the Church, they of all others are the most unfit to judge, who bend themselves purposely against whatsoever the Church useth, except it please themselves to give it the grace and countenance of their favourable approbation; which they willingly do not yield unto any part of church polity, in the forehead whereof there is not the mark of that new-devised stamp. But howsoever men like or dislike, whether they judge things necessary or needless in the house of God, a conscience they should have, touching that which they boldly affirm or deny.

[2](.) “In the primitive Church no bishops, no pastors having power over other pastors, but all equals, every man supreme commander and ruler within the kingdom of his own congregation or parish? The bishops that are spoken of in the time of the primitive Church, all such as parsons or rectors of parishes are with us?” If thus it have been in the prime of the Church, the question is, how far they will have that prime to extend? and where the latter spring of that new supposed disorder to begin? That primitive Church, wherein they hold that amongst the Fathers all which had pastoral charge were equal, they must of necessity so far enlarge as to contain some hundred of years, because for proof hereof they allege boldly and confidently St. Cyprian, who suffered martyrdom about two hundred and threescore years after our blessed Lord’s incarnation. A bishop, they say, such as Cyprian doth speak of, had only a church or congregation, such as the ministers and pastors with us, which are appointed unto several towns. Every bishop in Cyprian’s time was pastor of one only congregation, assembled in one place to be taught of one man.

A thing impertinent, although it were true. For the  question is about personal inequality amongst governors of the Church.
 Now to shew there was no such thing in the Church at such time as Cyprian lived, what bring they forth? Forsooth that bishops had then but a small circuit of place for the exercise of their authority. Be it supposed, that no one bishop had more than one only town to govern, one only congregation to rule: doth it by Cyprian appear, that in any such town or congregation being under the care and charge of some one bishop, there were not besides that one bishop others also ministers of the word and sacraments, yet subject to the power of the same bishop? If this appear not, how can Cyprian be alleged for a witness that in those times there were no bishops which did differ from other ministers, as being above them in degree of ecclesiastical power?

But a gross and a palpable untruth it is, that “bishops with Cyprian were as ministers are with us in parish churches; and that each of them did guide some parish without any other pastors under him.” St. Cyprian’s own person may serve for a manifest disproof hereof. Pontius being deacon under Cyprian noteth, that his admirable virtues caused him to be bishop with the soonest; which advancement therefore himself endeavoured for a while to avoid. It seemed in his own eyes too soon for him to take the title of so great honour, in regard whereof a bishop is termed Pontifex, Sacerdos, Antistes Dei. Yet such was his quality, that whereas others did hardly perform that duty whereunto the discipline of their order, together with the religion of the oath they took at their entrance into the office, even constrained them; him the chair did not make but receive such a one as behoved that a bishop should be. But soon after followed that proscription, whereby being driven into exile, and continuing in that estate for the space of some two years, he ceased not by letters to deal with his clergy, and to direct  them about the public affairs of the Church.
 They unto whom those epistles were written, he commonly entitleth the presbyters and deacons of that church. If any man doubt whether those presbyters of Carthage were ministers of the word and sacraments or no, let him consider but that one only place of Cyprian, where he giveth them his careful advice, how to deal with circumspection in the perilous times of the Church, that neither they which were for the truth’s sake imprisoned might want those ghostly comforts which they ought to have, nor the Church by ministering the same unto them incur unnecessary danger and peril. In which epistle it doth expressly appear, that the presbyters of whom he speaketh did offer, that is to say, administer the Eucharist; and that many there were of them in the Church of Carthage, so as they might have every day change for performance of that duty. Nor will any man of sound judgment I think deny, that Cyprian was in authority and power above the clergy of that church, above those presbyters unto whom he gave direction. It is apparently therefore untrue, that in Cyprian’s time ministers of the word and sacraments were all equal, and that no one of them had either title more excellent than the rest, or authority and government over the rest. Cyprian being bishop of Carthage was clearly superior unto all other ministers there: yea Cyprian was by reason of the dignity of his see an archbishop, and so consequently superior unto bishops.

[3]Bishops we say there have been always, even as long as the Church of Christ itself hath been. The Apostles who planted it, did themselves rule as bishops over it; neither could they so well have kept things in order during their own times, but that episcopal authority was given them from  above,
 to exercise far and wide over all other guides and pastors of God’s Church. The Church indeed for a time continued without bishops by restraint, every where established in Christian cities. But shall we thereby conclude that the Church hath no use of them, that without them it may stand and flourish? No, the cause wherefore they were so soon universally appointed was, for that it plainly appeared that without them the Church could not have continued long. It was by the special providence of God no doubt so disposed, that the evil whereof this did serve for remedy might first be felt, and so the reverend authority of bishops be made by so much the more effectual, when our general experience had taught men what it was for churches to want them. Good laws are never esteemed so good, nor acknowledged so necessary, as when precedent crimes are as seeds out of which they grow. Episcopal authority was even in a manner sanctified unto the Church of Christ by that little better [bitter?]g experience which it first had of the pestilent evil of schisms. Again, when this very thing was proposed as a remedy, yet a more suspicious and fearful acceptance it must needs have found, if the selfsame provident wisdom of Almighty God had not also given beforehand sufficient trial thereof in the regiment of Jerusalem, a mother church, which having received the same order even at the first, was by it most peaceably governed, when other churches without it had trouble. So that by all means the necessary use of episcopal government is confirmed, yea strengthened it is and ratified, even by the not establishment thereof in all churches every where at the first.

[4](.) When they further dispute, “That if any such thing were needful, Christ would in Scripture have set down particular statutes and laws, appointing that bishops should be made, and prescribing in what order, even as the law doth for all kind of officers which were needful in the Jewish regiment;” might not a man that would bend his wit to maintain the fury of the Petrobrusian heretics, in  pulling down oratories, use the selfsame argument with as much countenance of reason? “If it were needful that we should assemble ourselves in churches, would that God which taught the Jews so exactly the frame of their sumptuous temple, leave us no particular instructions in writing, no not so much as which way to lay any one stone?” Surely such kind of argumentation doth not so strengthen the sinews of their cause, as weaken the credit of their judgment which are led therewith.

[5](.) And whereas thirdly, in disproof [of]h that useh which episcopal authority hath in judgment of spiritual causes, they bring forth the verdict of Cyprian, who saith, that “equity requireth every man’s cause to be heard, where the fault he was charged with was committed, forasmuch as there they may have both accusers and witnesses in the cause;” this argument grounding itself on principles no less true in civil than in ecclesiastical causes, unless it be qualified with some exceptions or limitations, overturneth the highest tribunal seats both in Church and commonwealth; it taketh utterly away all appeals; it secretly condemneth even the blessed Apostle himself, as having transgressed the law of equity, by his appeal from the court of Judæa unto those higher which were in Rome. The generality of such kind of axioms deceiveth, unless it be construed with such cautions as the matter whereunto they are appliable doth require. An usual and ordinary transportation of causes out of Africa into Italy, out of one kingdom into another, as discontented persons list, which was the thing that Cyprian disalloweth, may be unequal and unmeet; and yet not therefore a thing unnecessary to have the courts erected in higher places, and judgment committed unto greater persons, to whom the meaner may bring their causes either by way of appeal or otherwise, to be determined according to the order of justice; which hath been always observed every where in civil states, and is no less requisite also for the state of the  Church of God. The reasons which teach it to be expedient for the one, will shew it to be for the other at leastwise not unnecessary.

Inequality of pastors is an ordinance both divine and profitable: their exceptions against it in these two respects we have shewed to be altogether causeless, unreasonable, and unjust.

\section*{An answer unto those things which are objected, concerning the difference between that power which Bishops now have, and that which ancient Bishops had, more than other presbyters.}

XIV. The next thing which they upbraid us with, is the difference between that inequality of pastors which hath been of old, and which now is. For at length they grant, that “the superiority of bishops and of archbishops is somewhat ancient, but no such kind of superiority as ours have.” By the laws of our discipline a bishop may ordain without asking the people’s consent, a bishop may excommunicate and release alone, a bishop may imprison, a bishop may bear civil office in the realm, a bishop may be a counsellor of state; these things ancient bishops neither did nor might do. Be it granted that ordinarily neither in elections nor deprivations, neither in excommunicating nor in releasing the excommunicate, in none of the weighty affairs of government, bishops of old were wont to do any thing without consultation with their clergy and consent of the people under them. Be it granted that the same bishops did neither touch any man with corporal punishment, nor meddle with secular affairs and offices, the whole clergy of God being then tied by the strict and severe canons of the Church to use no other than ghostly power, to attend no other business than heavenly. Tarquinius was in the Roman commonwealth deservedly hated, of whose unorderly proceedings the history speaketh thus: “Hic regum primus traditum a prioribus morem de omnibus senatum consulendi solvit; domesticis consiliis rempub. administravit; bellum, pacem, fœdera, societates, per seipsum, cum quibus voluit, injussu populi ac senatus, fecit diremitque.” Against bishops the like is objected, “That they are invaders of other men’s rights, and by intolerable usurpation take upon them to do that alone, wherein ancient laws have appointed that others, not they only, should bear sway.”

[2]Let the case of bishops be put, not in such sort as it  is, but even as their very heaviest adversaries would devise it. Suppose that bishops at the first had encroached upon the Church; that by sleights and cunning practices they had appropriated ecclesiastical, as Augustus did imperial power; that they had taken the advantage of men’s inclinable affections, which did not suffer them for revenue’s sake to be suspected of ambition; that in the meanwhile their usurpation had gone forward by certain easy and unsensible degrees; that being not discerned in the growth, when it was thus far grown as we now see it hath proceeded, the world at length perceiving there was just cause of complaint, but no place of remedy left, had assented unto it by a general secret agreement to bear it now as a helpless evil; all this supposed for certain and true, yet surely a thing of this nature, as for the superior to do that alone unto which of right the consent of some other inferiors should have been required by them; though it had an indirect entrance at the first, must needs, through continuance of so many ages as this hath stood, be made now a thing more natural to the Church, than that it should be oppressed with the mention of contrary orders worn so many ages since quite and clean out of ure.

[3]But with bishops the case is otherwise; for in doing that by themselves which others together with them have been accustomed to do, they do not any thing but that whereunto they have been upon just occasions authorized by orderly means. All things natural have in them naturally more or less the power of providing for their own safety: and as each particular man hath this power, so every politic society of men must needs have the same, that thereby the whole may provide for the good of all parts therein. For other benefit we have not any by sorting ourselves into politic societies, saving only that by this mean each part hath that relief which the virtue of the whole is able to yield it. The Church therefore being a politic society or body, cannot possibly want the power of providing for itself; and the chiefest part of that power consisteth in the authority of making laws. Now forasmuch as corporations are perpetual, the laws of the ancienter Church cannot choose but bind the latter, while they are in force. But we must note withal, that because the body of the Church  continueth the same, it hath the same authority still, and may abrogate old laws, or make new, as need shall require. Wherefore vainly are the ancient canons and constitutions objected as laws, when once they are either let secretly to die by disusage, or are openly abrogated by contrary laws.

[4]The ancient had cause to do no otherwise than they did; and yet so strictly they judged not themselves in conscience bound to observe those orders, but that in sundry cases they easily dispensed therewith, which I suppose they would never have done, had they esteemed them as things whereunto everlasting, immutable, and undispensable observation did belong. The bishop usually promoted none which were not first allowed as fit, by conference had with the rest of his clergy and with the people: notwithstanding, in the case of Aurelius, St. Cyprian did otherwise. In matters of deliberation and counsel, for disposing of that which belongeth generally to the whole body of the Church, or which being more particular, is nevertheless of so great consequence, that it needeth the force of many judgments conferred; in such things the common saying must necessarily take place, “An eye cannot see that which eyes can.” As for clerical ordinations, there are no such reasons alleged against the order which is, but that it may be esteemed as good in every respect as that which hath been; and in some considerations better; at leastwise (which is sufficient to our purpose) it may be held in the Church of Christ without transgressing any law, either ancient or late, divine or human, which we ought to observe and keep.

[5]The form of making ecclesiastical officers hath sundry parts, neither are they all of equal moment.

When Deacons having not been before in the Church of  Christ, the Apostles saw it needful to have such ordained, they first, assemble the multitude, and shew them how needful it is that deacons be made: secondly, they name unto them what number they judge convenient, what quality the men must be of, and to the people they commit the care of finding such out: thirdly, the people hereunto assenting, make their choice of Stephen and the rest; those chosen men they bring and present before the Apostles: howbeit, all this doth not endue them with any ecclesiastical power. But when so much was done, the Apostles finding no cause to take exception, did with prayer and imposition of hands make them deacons. This was it which gave them their very being; all other things besides were only preparations unto this.

[6]Touching the form of making Presbyters, although it be not wholly of purpose any where set down in the Apostles’ writings, yet sundry speeches there are which insinuate the chiefest things that belong unto that action: as when Paul and Barnabas are said to have fasted, prayed, and made presbyters: when Timothy is willed to “lay hands suddenly on no man,” for fear of participating with other men’s sins. For this cause the order of the primitive Church was, between choice and ordination to have some space for such probation and trial as the Apostle doth mention in deacons, saying, “Let them first be proved, and then minister, if so be they be found blameless.”

Alexander Severus beholding in his time how careful the Church of Christ was, especially for this point; how after the choice of their pastors they used to publish the names of the parties chosen, and not to give them the final act of approbation till they saw whether any let or impediment would be alleged; he gave commandment that the like should also be done in his own imperial elections, adding this as a reason  wherefore he so required, namely, “For that both Christians and Jews being so wary about the ordination of their priests, it seemed very unequal for him not to be in like sort circumspect, to whom he committed the government of provinces, containing power over men’s both estates and lives.” This the canon itself doth provide for, requiring before ordination scrutiny: “Let them diligently be examined three days together before the Sabbath, and on the Sabbath [i.e. Saturday2] let them be presented unto the bishop.” And even this in effect also is the very use of the church of England, at all solemn ordaining of ministers; and if all ordaining were solemn, I must confess it were much the better.

[7]The pretended disorder of the church of England is, that bishops ordain them to whose election the people give no voices, and so the bishops make them alone; that is to say, they give ordination without popular election going before, which ancient bishops neither did nor might do. Now in very truth, if the multitude have hereunto a right, which right can never be translated from them for any cause, then is there no remedy but we must yield, that unto the lawful making of ministers the voice of the people is required; and that according to the adverse party’s assertion, such as make ministers without asking the people’s consent, do but exercise a certain tyranny.

At the first erection of the commonwealth of Rome, the  people (for so it was then fittest) determined of all affairs: afterwards this growing troublesome, their senators did that for them which themselves before had done: in the end all came to one man’s hands, and the emperor alone was instead of many senators.

In these things the experience of time may breed both civil and ecclesiastical change from that which hath been before received, neither do latter things always violently exclude former, but the one growing less convenient than it hath been, giveth place to that which is now become more. That which was fit for the people themselves to do at the first, might afterwards be more convenient for them to do by some other: which other is not thereby proved a tyrant, because he alone doth that which a multitude were wont to do, unless by violence he take that authority upon him, against the order of law, and without any public appointment; as with us if any did, it should (I suppose) not long be safe for him so to do.

[8]This answer (I hope) will seem to be so much the more reasonable, in that themselves, who stand against us, have furnished us therewith. For whereas against the making of ministers by bishops alone, their use hath been to object, what sway the people did bear when Stephen and the rest were ordained deacons; they begin to espy how their own platform swerveth not a little from that example wherewith they control the practice of others. For touching the form of the people’s concurrence in that action, they observe it not; no, they plainly profess that they are not in this point bound to be followers of the Apostles. The Apostles ordained whom the people had first chosen. They hold, that their ecclesiastical senate ought both to choose, and also to ordain. Do not themselves then take away that which the Apostles gave the people, namely, the privilege of choosing ecclesiastical officers? They do. But behold in what sort they answer it. “By the sixth and the fourteenth of the Acts1” (say they) “it doth appear that the people had the chiefest power of choosing. Howbeit that, as unto me it seemeth, was done upon special cause which doth not so much concern us, neither ought it  to be drawn unto the ordinary and perpetual form of governing the Church. For as in establishing commonweals, not only if they be popular, but even being such as are ordered by the power of a few the chiefest, or as by the sole authority of one, till the same be established, the whole sway is in the people’s hands, who voluntarily appoint those magistrates by whose authority they may be governed; so that afterward not the multitude itself, but those magistrates which are chosen by the multitude, have the ordering of public affairs: after the selfsame manner it fared in establishing also the Church; when there was not as yet any placed over the people, all authority was in them all; but when they all had chosen certain to whom the regiment of the Church was committed, this power is not now any longer in the hands of the whole multitude, but wholly in theirs who are appointed guides of the Church. Besides, in the choice of deacons, there was also another special cause wherefore the whole Church at that time should choose them. For inasmuch as the Grecians murmured against the Hebrews, and complained that in the daily distribution which was made for relief of the poor, they were not indifferently respected, nor such regard had of their widows as was meet; this made it necessary that they all should have to deal in the choice of those unto whom that care was afterwards to be committed, to the end that all occasion of jealousies and complaints might be removed. Wherefore that which was done by the people for certain causes, before the Church was fully settled, may not be drawn out and applied unto a constant and perpetual form of ordering the Church.”

[9]Let them cast the discipline of the church of England into the same scales where they weigh their own, let them give us the same measure which here they take, and our strifes shall soon be brought to a quiet end. When they urge the Apostles as precedents; when they condemn us of tyranny, because we do not in making ministers the same which the Apostles did; when they plead, “That with us one alone doth ordain, and that our ordinations are without the people’s knowledge, contrary to that example which the blessed Apostles gave:” we do not request at their  hands allowance as much as of one word we speak in our own defence, if that which we speak be of our own; but that which themselves speak, they must be contented to listen unto. To exempt themselves from being over far pressed with the Apostles’ example, they can answer, “That which was done by the people once upon special causes, when the Church was not yet established, is not to be made a rule for the constant and continual ordering of the Church.” In defence of their own election, although they do not therein depend on the people so much as the Apostles in the choice of deacons, they think it a very sufficient apology, that there were special considerations why deacons at that time should be chosen by the whole Church, but not so now. In excuse of dissimilitudes between their own and the Apostles’ discipline, they are contented to use this answer, “That many things were done in the Apostles’ times, before the settling of the Church, which afterward the Church was not tied to observe.” For countenance of their own proceedings, wherein their governors do more than the Apostles, and their people less than under the Apostles the first Churches are found to have done, at the making of ecclesiastical officers, they deem it a marvellous reasonable kind of pleading to some [say?] “That even as in commonweals, when the multitude have once chosen many or one to rule over them, the right which was at the first in the whole body of the people is now derived into those many or that one which is so chosen; and that this being done, it is not the whole multitude, to whom the administration of such public affairs any longer appertaineth, but that which they did, their rulers may now do lawfully without them: after the selfsame manner it standeth with the Church also.”

How easy and plain might we make our defence, how clear and allowable even unto them, if we could but obtain of them to admit the same things consonant unto equity in our mouths, which they require to be so taken from their own! If that which is truth, being uttered in maintenance of Scotland and Geneva, do not cease to be truth when the church of England once allegeth it, this great crime of tyranny wherewith we are charged hath a plain and an easy defence.

[10]“Yea, but we do not at all ask the people’s approbation,  which they do, whereby they shew themselves more indifferent and more free from taking away the people’s right.” Indeed, when their lay-elders have chosen whom they think good, the people’s consent thereunto is asked, and if they give their approbation, the thing standeth warranted for sound and good. But if not, is the former choice overthrown? No, but the people is to yield to reason; and if they which have made the choice, do so like the people’s reason, as to reverse their own deed at the hearing of it, then a new election to be made; otherwise the former to stand, notwithstanding the people’s negative and dislike. What is this else but to deal with the people, as those nurses do with infants, whose mouths they besmear with the backside of the spoon, as though they had fed them, when they themselves devour the food? They cry in the ears of the people, that all men’s consent should be had unto that which concerns all; they make the people believe we wrong them, and deprive them of their right in making ministers, whereas with us the people have commonly far more sway and force than with them. For inasmuch as there are but two main things observed in every ecclesiastical function, Power to exercise the duty itself, and some charge of People whereon to exercise the same; the former of these is received at the hands of the whole visible catholic Church. For it is not any one particular multitude that can give power, the force whereof may reach far and wide indefinitely, as the power of order doth, which whoso hath once received, there is no action which belongeth thereunto but he may exercise effectually the same in any part of the world without iterated  ordination. They whom the whole Church hath from the beginning used as her agents in conferring this power, are not either one or more of the laity, and therefore it hath not been heard of that ever any such were allowed to ordain ministers: only persons ecclesiastical, and they, in place of calling, superiors both unto deacons and unto presbyters; only such persons ecclesiastical have been authorized to ordain both, and to give them the power of order, in the name of the whole Church. Such were the Apostles, such was Timothy, such was Titus, such are bishops. Not that there is between these no difference, but that they all agree in preeminence of place above both presbyters and deacons, whom they otherwise might not ordain.

[11]Now whereas hereupon some do infer, that no ordination can stand but only such as is made by bishops, which have had their ordination likewise by other bishops before them, till we come to the very Apostles of Christ themselves; in which respect it was demanded of Beza at Poissie, “By what authority he could administer the holy sacraments, being not thereunto ordained by any other than Calvin, or by such as to whom the power of ordination did not belong, according to the ancient orders and customs of the Church; sith Calvin and they who joined with him in that action were no bishops:” and Athanasius maintaineth the fact of Macarius a presbyter, which overthrew the holy table  whereat one Ischyras would have ministered the blessed Sacrament, having not been consecrated thereunto by laying on of some bishop’s hands, according to the ecclesiastical canons; as also Epiphanius inveigheth sharply against divers for doing the like, when they had not episcopal ordination: to this we answer, that there may be sometimes very just and sufficient reason to allow ordination made without a bishop.

The whole Church visible being the true original subject of all power, it hath not ordinarily allowed any other than bishops alone to ordain: howbeit, as the ordinary course is ordinarily in all things to be observed, so it may be in some cases not unnecessary that we decline from the ordinary ways.

Men may be extraordinarily, yet allowably, two ways admitted unto spiritual functions in the Church. One is, when God himself doth of himself raise up any, whose labour he useth without requiring that men should authorize them; but then he doth ratify their calling by manifest signs and tokens himself from heaven: and thus even such as believed not our Saviour’s teaching, did yet acknowledge him a lawful teacher sent from God: “Thou art a teacher sent from God, otherwise none could do those things which thou doest.” Luther did but reasonably therefore, in declaring that the senate of Mulheuse should do well to ask of Muncer, from whence he received power to teach, who it was that had called him; and if his answer were that God had given him his charge, then to require at his hands some evident sign thereof for men’s satisfaction: because so God is wont, when he himself is the author of any extraordinary calling.

Another extraordinary kind of vocation is, when the exigence of necessity doth constrain to leave the usual ways of the Church, which otherwise we would willingly keep: where  the church must needs have some ordained, and neither hath nor can have possibly a bishop to ordain; in case of such necessity, the ordinary institution of God hath given oftentimes, and may give, place. And therefore we are not simply without exception to urge a lineal descent of power from the Apostles by continued succession of bishops in every effectual ordination. These cases of inevitable necessity excepted, none may ordain but only bishops: by the imposition of their hands it is, that the Church giveth power of order, both unto presbyters and deacons.

[12]Now when that power so received is once to have any certain subject whereon it may work, and whereunto it is to be tied, here cometh in the people’s consent, and not before. The power of order I may lawfully receive, without asking leave of any multitude; but that power I cannot exercise upon any one certain people utterly against their wills; neither is there in the church of England any man, by order of law, possessed with pastoral charge over any parish, but the people in effect do choose him thereunto. For albeit they choose not by giving every man personally his particular voice, yet can they not say that they have their pastors violently obtruded upon them, inasmuch as their ancient and original interest therein hath been by orderly means derived into the patron who chooseth for them. And if any man be desirous to know how patrons came to have such interest, we are to consider, that at the first erection of churches, it seemed but reasonable in the eyes of the whole Christian world to pass that right to them and their successors, on whose soil and at whose charge the same were founded. This all men gladly and willingly did, both in honour of so great piety, and for encouragement of many others unto the like, who peradventure else would have been as slow to erect churches or to endow them, as we are forward both to spoil them and to pull them down.

It is no true assertion therefore in such sort as the pretended reformers mean it, “That all ministers of God’s word  ought to be made by consent of many, that is to say, by the people’s suffrages; that ancient bishops neither did nor might ordain otherwise; and that ours do herein usurp a far greater power than was, or than lawfully could have been granted unto bishops which were of old.”

[13]Furthermore, as touching spiritual jurisdiction, our bishops, they say, do that which of all things is most intolerable, and which the ancient never did. “Our bishops excommunicate and release alone, whereas the censures of the Church neither ought, nor were wont to be administered otherwise than by consent of many.” Their meaning here, when they speak of many, is not as before it was; when they hold that ministers should be made with consent of many, they understand by many, the multitude, or common people; but in requiring that many should evermore join with the bishop in the administration of church censures, they mean by many, a few lay-elders chosen out of the rest of the people to that purpose. This they say is ratified by ancient councils, by ancient bishops this was practised.  And the reason hereof, as Beza supposeth, was, “Because if the power of ecclesiastical censures did belong unto any one, there would this great inconveniency follow, ecclesiastical  regiment should be changed into mere tyranny, or else into a civil royalty: therefore no one, either bishop or presbyter, should or can alone exercise that power, but with his ecclesiastical consistory he ought to do it, as may appear by the old discipline.” And is it possible, that one so grave and judicious should think it in earnest tyranny for a bishop to excommunicate, whom law and order hath authorized so to do? or be persuaded that ecclesiastical regiment degenerateth into civil regality, when one is allowed to do that which hath been at any time the deed of more? Surely, far meaner witted men than the world accounteth Mr. Beza do easily perceive, that tyranny is power violently exercised against order, against law; and that the difference of these two regiments, ecclesiastical and civil, consisteth in the matter about which the actions of each are conversant; and not in this, that civil royalty admitteth but one, ecclesiastical government requireth many supreme correctors. Which allegation, were it true, would prove no more than only that some certain number is necessary for the assistance of the bishop; but that a number of such as they do require is necessary, how doth it prove? Wherefore albeit bishops should now do the very same which the ancients did, using the college of presbyters under them as their assistants when they administer church-censures, yet should they still swerve utterly from that which these men so busily labour for, because the agents whom they require to assist in those cases are a sort of lay-elders, such as no ancient bishop ever was assisted with.

Shall these fruitless jars and janglings never cease? shall we never see end of them? How much happier were the world if those eager taskmasters whose eyes are so curious and sharp in discerning what should be done by many and what by few, were all changed into painful doers of that which every good Christian man ought either only or chiefly to do, and to be found therein doing when that great and glorious Judge of all men’s both deeds and words shall appear? In the meanwhile, be it one that hath this charge, or be they many that be his assistants, let there be careful provision that justice may be administered, and in this shall our God be glorified more than by such contentious disputes.

\section*{Concerning the civil power and authority which our Bishops have.}

XV. Of which nature that also is, wherein Bishops are over and besides all this accused “to have much more excessive power than the ancient, inasmuch as unto their ecclesiastical authority, the civil magistrate for the better repressing of such as contemn ecclesiastical censures,
hath for divers ages annexed civil. The crime of bishops herein is divided into these two several branches; the one, that in causes ecclesiastical they strike with the sword of secular punishments; the other, that offices are granted them, by virtue whereof they meddle with civil affairs.”

[2]Touching the one, it reacheth no farther than only unto restraint of liberty by imprisonment (which yet is not done but by the laws of the land, and by virtue of authority derived from the prince). A thing which being allowable in priests amongst the Jews, must needs have received some strange alteration in nature since, if it be now so pernicious and venomous to be coupled with a spiritual vocation in any man which beareth office in the Church of Christ. Shemaiah writing to the college of priests which were in Jerusalem, and to Zephaniah the principal of them, told them they were appointed of God, “that they might be officers in the house of the Lord, for every man which raved, and did make himself a prophet,” to the end that they might by the force of this their authority “put such in prison and in the stocks.” His malice is reproved, for that he provoketh them to shew their power against the innocent. But surely, when any man justly punishable had been brought before them, it could be no unjust thing for them even in such sort then to have punished.

[3]As for offices by virtue whereof bishops have to deal in civil affairs, we must consider that civil affairs are of divers  kinds, and as they be not all fit for ecclesiastical persons to meddle with, so neither is it necessary, nor at this day haply convenient, that from meddling with any such thing at all they all should without exception be secluded. I will therefore set down some few causes, wherein it cannot but clearly appear unto reasonable men that civil and ecclesiastical functions may be lawfully united in one and the same person.

First therefore, in case a Christian society be planted amongst their professed enemies, or by toleration do live under some certain state whereinto they are not incorporated, whom shall we judge the meetest man to have the hearing and determining of such mere civil controversies as are every day wont to grow between man and man? Such being the state of the church of Corinth, the Apostle giveth them this direction, “Dare any of you having business against another be judged by the unjust, and not under saints? Do ye not know that the saints shall judge the world? If the world then shall be judged by you, are ye unworthy to judge the smallest matters? Know ye not that we shall judge the angels? how much more things that appertain to this life? If then ye have judgment of things pertaining to this life, set up them which are least esteemed in the Church. I speak it to your shame; is it so that there is not a wise man amongst you? no not one that can judge between his brethren, but a brother goeth to law with a brother and that under the infidels? Now therefore there is utterly a fault among you, because ye go to law one with another; why rather suffer ye not wrong, why rather sustain ye not harm?” In which speech there are these degrees: better to suffer and to put up injuries, than to contend; better to end contention by arbitrement, than by judgment; better by judgment before the wisest of their own, than before the simpler; better before the simplest of their own, than the wisest of them without; So that if judgment of secular affairs should be committed unto wise men, unto men of chiefest credit and account amongst them, when the pastors of their souls are such, who more fit to be also their judges for the ending of strifes? The wisest in things divine may  be also in things human the most skilful. At leastwise they are by likelihood commonly more able to know right from wrong than the common unlettered sort.

And what St. Augustine did hereby gather, his own words do sufficiently shew. “I call God to witness upon my soul,” saith he, “that according to the order which is kept in well-ordered monasteries, I could wish to have every day my hours of labouring with my hands, my hours of reading and of praying, rather than to endure these most tumultuous perplexities of other men’s causes, which I am forced to bear while I travel in secular businesses, either by judging to discuss them, or to cut them off by entreaty: unto which toils that Apostle, who himself sustained them not, for any thing we read, hath notwithstanding tied us not of his own accord, but being thereunto directed by that Spirit which speaks in him. His own apostleship which  drew him to travel up and down, suffered him not to be any where settled to this purpose; wherefore the wise, faithful and holy men which were seated here and there, and not them which travelled up and down to preach, he made examiners of such businesses. Whereupon of him it is no where written, that he had leisure to attend these things, from which we cannot excuse ourselves although we be simple: because even such he requireth, if wise men cannot be had, rather than the affairs of Christians should be brought into public judgment. Howbeit not without comfort in our Lord are these travels undertaken by us, for the hope’s sake of eternal life, to the end that with patience we may reap fruit.” So far is St. Augustine from thinking it unlawful for pastors in such sort to judge civil causes, that he plainly collecteth out of the Apostle’s words a necessity to undertake that duty; yea himself he comforteth with the hope of a blessed reward, in lieu of travel that way sustained.

[4]Again, even where whole Christian kingdoms are, how troublesome were it for universities and other greater collegiate societies, erected to serve as nurseries unto the Church of Christ, if every thing which civilly doth concern them were to be carried from their own peculiar governors, because for the most part they are (as fittest it is they should be) persons of ecclesiastical calling? It was by the wisdom of our famous predecessors foreseen how unfit this would be, and hereupon provided by grant of special charters that it might be as now it is in the universities; where their vice-chancellors, being for the most part professors of divinity, are nevertheless civil judges over them in the most of their ordinary causes.

[5]And to go yet some degrees further; a thing impossible it is not, neither altogether unusual, for some who are of royal blood to be consecrated unto the ministry of Jesus Christ, and so to be nurses of God’s Church, not only as the Prophet did foretell, but also as the Apostle St. Paul was. Now in case the crown should by this mean descend unto such persons, perhaps when they are the very last, or perhaps the very best of their race, so that a greater benefit they are not able to bestow upon a kingdom than by accepting  their right therein: shall the sanctity of their order deprive them of that honour whereunto they have right by blood? or shall it be a bar to shut out the public good that may grow by their virtuous regiment? If not, then must they cast off the office which they received by divine imposition of hands; or, if they carry a more religious opinion concerning that heavenly function, it followeth, that being invested as well with the one as the other, they remain God’s lawfully anointed both ways. With men of skill and mature judgment there is of this so little doubt, that concerning such as at this day are under the archbishops of Mentz, Colen, and Trevers, being both archbishops and princes of the empire; yea such as live within the Pope’s own civil territories, there is no cause why any should deny to yield them civil obedience in any thing which they command, not repugnant to Christian piety; yea, even that civilly for such as are under them not to obey them, were but the part of seditious persons. Howbeit for persons ecclesiastical  thus to exercise civil dominion of their own, is more than when they only sustain some public office, or deal in some business civil, being thereunto even by supreme authority required.

[6]As nature doth not any thing in vain, so neither grace. Wherefore if it please God to bless some principal attendants on his own sanctuary, and to endue them with extraordinary parts of excellency, some in one kind, some in another, surely a great derogation it were to the very honour of him who bestowed so precious graces, except they on whom he hath bestowed them should accordingly be employed, that the fruit of those heavenly gifts might extend itself unto the body of the commonwealth wherein they live; which being of purpose instituted (for so all commonwealths are) to the end that all might enjoy whatsoever good it pleaseth the Almighty to endue each one man with, must needs suffer loss, when it hath not the gain which eminent civil ability in ecclesiastical persons is now and then found apt to afford. Shall we then discommend the people of Milan for using Ambrose their bishop as an ambassador about their public and politic affairs; the Jews for electing their priests sometimes to be leaders in war; David for making the high-priest his chiefest counsellor of state: finally, all Christian kings and princes which have appointed unto like services bishops or other of the clergy under them? No, they have done in this respect that which most sincere and religious wisdom alloweth.

[7]Neither is it allowable only, when either a kind of necessity doth cast civil offices upon them, or when they are thereunto preferred in regard of some extraordinary fitness; but further also when there are even of right annexed unto some of their places, or of course imposed upon certain of their persons, functions of dignity and account in the commonwealth; albeit no other consideration be had therein save this, that their credit and countenance may by such means be augmented. A thing if ever to be respected, surely most of all now, when God himself is for his own sake generally no where honoured, religion almost no where, no where religiously  adored, the ministry of the word and sacraments of Christ a very cause of disgrace in the eyes both of high and low, where it hath not somewhat besides itself to be countenanced with. For unto this very pass things are come, that the glory of God is constrained even to stand upon borrowed credit, which yet were somewhat the more tolerable, if there were not that dissuade to lend it him.

No practice so vile, but pretended holiness is made sometime as a cloak to hide it. The French king Philip Valois in his time made an ordinance that all prelates and bishops should be clean excluded from parliaments where the affairs of the kingdom were handled; pretending that a king with good conscience cannot draw pastors, having cure of souls, from so weighty a business, to trouble their heads with consultations of state. But irreligious intents are not able to hide themselves, no not when holiness is made their cloak. This is plain and simple truth, that the councils of wicked men hate always the presence of them, whose virtue, though it should not be able to prevail against their purposes, would notwithstanding be unto their minds a secret corrosive:  and therefore, till either by one shift or another they can bring all things to their own hands alone, they are not secure.

[8]Ordinances holier and better there stand as yet in force by the grace of Almighty God, and the works of his providence amongst us. Let not envy so far prevail, as to make us account that a blemish, which if there be in us any spark of sound judgment, or of religious conscience, we must of necessity acknowledge to be one of the chiefest ornaments unto this land: by the ancient laws whereof, the clergy being held for the chief of those three estates, which together make up the entire body of this commonwealth, under one supreme head and governor, it hath all this time ever borne a sway proportionable in the weighty affairs of the land; wise and virtuous kings condescending most willingly thereunto, even of reverence to the Most High; with the flower of whose sanctified inheritance, as it were with a kind of Divine presence, unless their chiefest civil assemblies were so far forth beautified as might be without any notable impediment unto their heavenly functions, they could not satisfy themselves as having shewed towards God an affection most dutiful.

Thus, first, in defect of other civil magistrates; secondly, for the ease and quietness of scholastical societies; thirdly, by way of political necessity; fourthly, in regard of quality, care, and extraordinancy; fifthly, for countenance unto the ministry; and lastly, even of devotion and reverence towards God himself: there may be admitted at leastwise in some particulars well and lawfully enough a conjunction of civil and ecclesiastical power, except there be some such law or reason to the contrary, as may prove it to be a thing simply in itself naught.

[9]Against it many things are objected, as first, “That the matters which are noted in the holy Scriptures to have belonged to the ordinary office of any ministers of God’s holy word and sacraments, are these which follow, with such like, and no other; namely, the watch of the sanctuary, the business of God, the ministry of the word and sacraments, oversight of the house of God, watching over his flock, prophecy, prayer, dispensations of the mysteries of  God, charge and care of men’s souls.” If a man would shew what the offices and duties of a chirurgeon or physician are, I suppose it were not his part so much as to mention any thing belonging to the one or the other, in case either should be also a soldier or a merchant, or an housekeeper, or a magistrate: because the functions of these are different from those of the former, albeit one and the same man may haply be both. The case is like, when the Scripture teacheth what duties are required in an ecclesiastical minister; in describing of whose office, to touch any other thing than such as properly and directly toucheth his office that way, were impertinent.

[10]Yea, “but in the Old Testament the two powers civil and ecclesiastical were distinguished, not only in nature, but also in person; the one committed unto Moses, and the magistrates joined with him; the other to Aaron and his sons. Jehoshaphat in his reformation doth not only distinguish causes ecclesiastical from civil, and  erecteth divers courts for them, but appointeth also divers judges.”

With the Jews these two powers were not so distinguished, but that sometimes they might and did concur in one and the same person. Was not Eli both priest and judge? after their return from captivity, Esdras a priest, and the same their chief governor even in civil affairs also?

These men which urge the necessity of making always a personal distinction of these two powers, as if by Jehoshaphat’s example the same person ought not to deal in both causes, yet are not scrupulous to make men of civil place and calling presbyters and ministers of spiritual jurisdiction in their own spiritual consistories. If it be against the Jewish precedents for us to give civil power unto such as have ecclesiastical; is it not as much against the same for them to give ecclesiastical power unto such as have civil? They will answer perhaps, that their position is only against conjunction of ecclesiastical power of order, and the power of civil jurisdiction in one person. But this answer will not stand with their proofs, which make no less against the power of civil and ecclesiastical jurisdiction in one person; for of these two powers Jehoshaphat’s example is: besides, the contrary example [examples?] of Eli and of Ezra, by us alleged, do plainly shew, that amongst the Jews even the power of order ecclesiastical and civil jurisdiction were sometimes lawfully united in one and the same person.


[11]Pressed further we are with our Lord and Saviour’s example, who “denieth his kingdom to be of this world, and therefore, as not standing with his calling, refused to be made a king, to give sentence in a criminal cause of adultery, and in a civil of dividing an inheritance.” The Jews imagining that their Messiah should be a potent monarch upon earth, no marvel, though when they did otherwise wonder at Christ’s greatness, they sought forthwith to have him invested with that kind of dignity, to the end he might presently begin to reign. Others of the Jews, which likewise had the same imagination of the Messiah, and did somewhat incline to think that peradventure this might be he, thought good to try whether he would take upon him that which he might do, being a king, such as they supposed their true Messiah should be. But Christ refused to be a king over them, because it was no part of the office of their Messiah, as they did falsely conceive; and to intermeddle in those acts of civil judgment he refused also, because he had no such jurisdiction in that commonwealth, being in regard of his civil person a man of mean and low calling. As for repugnancy between ecclesiastical and civil power, or any inconvenience that these two powers should be united, it doth not appear that this was the cause of his resistance either to reign or else to judge.

[12]What say we then to the blessed Apostles, who teach,  “that soldiers entangle not themselves with the business of this life, but leave them, to the end they may please him who hath chosen them to serve, and that so the good soldiers of Christ ought to do.”

“The Apostles which taught this, did never take upon them any place or office of civil power. No, they gave over the ecclesiastical care of the poor, that they might wholly attend upon the word and prayer.”

St. Paul indeed doth exhort Timothy after this manner: “Suffer thou evil as a noble soldier of Jesus Christ: no man warring is entangled with the affairs of life, because he must serve such as have pressed him unto warfare.” The sense and meaning whereof is plain, that soldiers may not be nice and tender, that they must be able to endure hardness, that no man betaking himself unto wars continueth entangled with such kind of businesses as tend only unto the ease and quiet felicity of this life, but if the service of him who hath taken them under his banner require the hazard, yea the loss of their lives, to please him they must be content and willing with any difficulty, any peril, be it never so much against the natural desire which they have to live in safety. And at this point the clergy of God must always stand; thus it behoveth them to be affected as oft as their Lord and captain leadeth them into the field, whatsoever conflicts, perils, or evils they are to endure. Which duty being not such, but that therewith the civil dignities which ecclesiastical persons amongst us do enjoy may enough stand; the exhortation of Paul to Timothy is but a slender allegation against them.

As well might we gather out of this place, that men having children or wives are not fit to be ministers, (which also hath been collected, and that by sundry of the ancient3), and that it is requisite the clergy be utterly forbidden marriage: for as  the burden of civil regiment doth make them who bear it the less able to attend their ecclesiastical charge; even so St. Paul doth say, that the married are careful for the world, the unmarried freer to give themselves wholly to the service of God. Howbeit, both experience hath found it safer, that the clergy should bear the cares of honest marriage, than be subject to the inconveniences which single life imposed upon them would draw after it: and as many as are of sound judgment know it to be far better for this present age, that the detriment be borne which haply may grow through the lessening of some few men’s spiritual labours, than that the clergy and commonwealth should lack the benefit which both the one and the other may reap through their dealing in civil affairs. In which consideration, that men consecrated unto the spiritual service of God be licensed so far forth to meddle with the secular affairs of the world, as doth seem for some special good cause requisite, and may be without any grievous prejudice unto the Church, surely there is not in the Apostles being rightly understood, any let.

[13]That no Apostle did ever bear office, may it not be a wonder, considering the great devotion of the age wherein they lived, and the zeal of Herod, of Nero the great commander of the known world, and of other kings of the earth at that time, to advance by all means Christian religion.

Their deriving unto others that smaller charge of distributing of the goods which were laid at their feet, and of making provision for the poor, which charge, being in part civil, themselves had before (as I suppose lawfully) undertaken, and their following of that which was weightier, may serve as a marvellous good example for the dividing of one man’s office into divers slips, and the subordinating of inferiors to discharge some part of the same, when by reason of multitude increasing that labour waxeth great and troublesome which before was easy and light; but very small force it hath to infer a perpetual divorce between ecclesiastical and civil power in the same persons.

[14]The most that can be said in this case is, “That sundry eminent canons, bearing the name of apostolical, and divers councils likewise there are, which have forbidden the  clergy to bear any secular office; and have enjoined them to attend altogether upon reading, preaching, and prayer: whereupon the most of the ancient fathers have shewed great dislikes that these two powers should be united in one person.”


For a full and final answer whereunto, I would first demand, whether the commixtion and separation of these two powers be a matter of mere positive law, or else a thing simply with or against the law immutable of God and nature? That which is simply against this latter law can at no time be allowable in any person, more than adultery, blasphemy, sacrilege, and the like. But conjunction of power ecclesiastical and civil, what law is there which hath not at some time or other allowed as a thing convenient and meet? In the law of God we have examples sundry, whereby it doth most manifestly appear how of him the same hath oftentimes been approved. No kingdom or nation in the world, but hath been thereunto accustomed without inconvenience and hurt. In the prime of the world, kings and civil rulers were priests for the most part all. The Romans note it as a thing beneficial in their own commonwealth, and even to them apparently forcible for the strengthening of the Jews’ regiment under Moses and Samuel.

I deny not but sometime there may be, and hath been perhaps, just cause to ordain otherwise. Wherefore we are not so to urge those things which heretofore have been either ordered or done, as thereby to prejudice those orders, which upon contrary occasion and the exigence of the present time by like authority have been established. For what is there which doth let but that from contrary occasions contrary laws may grow, and each be reasoned and disputed for by such as are subject thereunto, during the time they are in force; and yet neither so opposite to other, but that both may laudably continue, as long as the ages which keep them do see no  necessary cause which may draw them unto alteration? Wherefore in these things, canons, constitutions, and laws, which have been at one time meet, do not prove that the Church should always be bound to follow them. Ecclesiastical persons were by ancient order forbidden to be executors of any man’s testament, or to undertake the wardship of children. Bishops by the imperial law are forbidden to bequeath by testament or otherwise to alienate any thing grown unto them after they were made bishops. Is there no remedy but that these or the like orders must therefore every where still be observed?

[15]The reason is not always evident, why former orders have been repealed and other established in their room. Herein therefore we must remember the axiom used in the civil laws, “That the prince is always presumed to do that with reason, which is not against reason being done, although no reason of his deed be expressed.” Which being in every respect as true of the Church, and her divine authority in making laws, it should be some bridle unto those malapert and proud spirits, whose wits not conceiving the reason of laws that are established, they adore their own private fancy as the supreme law of all, and accordingly take upon them to judge that whereby they should be judged.

But why labour we thus in vain? For even to change that which now is, and to establish instead thereof that which themselves would acknowledge the very selfsame which hath been, to what purpose were it, sith they protest, “that  they utterly condemn as well that which hath been as that which is; as well the ancient as the present superiority, authority and power of ecclesiastical persons.”

\section*{The arguments answered, whereby they would prove that the law of God and the judgment of the best in all ages condemneth the ruling superiority of one minister over another.}

XVI. Now where they lastly allege, “That the law of our Lord Jesus Christ, and the judgment of the best in all ages, condemn all ruling superiority of ministers over ministers;” they are in this, as in the rest, more bold to affirm, than able to prove the things which they bring for support of their weak and feeble cause. “The bearing of dominion or the exercising of authority (they say2), is that wherein the civil magistrate is severed from the ecclesiastical officer, according to the words of our Lord and Saviour, ‘Kings of nations bear rule over them, but it shall not be so with you:’ therefore bearing of dominion doth not agree to one minister over another.”

[2]This place hath been, and still is, although most falsely, yet with far greater show and likelihood of truth, brought forth by the anabaptists, to prove that the Church of Christ ought to have no civil magistrates, but [to be] ordered only by Christ. Wherefore they urge the opposition between heathens and them unto whom our Saviour speaketh. For sith the Apostles were opposite to heathens, not in that they were Apostles, but in that they were Christians, the  anabaptists’ inference is, “that Christ doth here give a law, to be for ever observed by all true Christian men, between whom and heathens there must be always this difference, that whereas heathens have their kings and princes to rule, Christians ought not in this thing to be like unto them.” Wherein their construction hath the more show, because that which Christ doth speak to his Apostles is not found always agreeable unto them as Apostles, or as pastors of men’s souls, but oftentimes it toucheth them in generality, as they are Christians; so that Christianity being common unto them with all believers, such speeches must be so taken that they may be applied unto all, and not only unto them.

[3]They which consent with us, in rejecting such collections as the anabaptist maketh with more probability, must give us leave to reject such as themselves have made with less: for a great deal less likely it is, that our Lord should here establish an everlasting difference, not between his Church and pagans, but between the pastors of his Church and civil governors. For if herein they must always differ, that the one may not bear rule, the other may; how did the Apostles themselves observe this difference, the exercise of whose authority, both in commanding and in controlling others, the Scripture hath made so manifest that no gloss can overshadow it? Again, it being, as they would have it, our Saviour’s purpose to withhold his Apostles and in them all other pastors from bearing rule, why should kingly dominion be mentioned, which occasions men to gather, that not all dominion and rule, but this one only form was prohibited, and that authority was permitted them, so it were not regal? Furthermore, in case it had been his purpose to withhold pastors altogether from bearing rule, why should kings of nations be mentioned, as if they were not forbidden to exercise, no not regal dominion itself, but only such regal dominion as heathen kings do exercise?

[4]The very truth is, our Lord and Saviour did aim at a far other mark than these men seem to observe. The end of his speech was to reform their particular mispersuasion to whom he spake: and their mispersuasion was, that which was also the common fancy of the Jews at that time, that their  Lord being the Messias of the world, should restore unto Israel that kingdom, whereof the Romans had as then bereaved them; they imagined that he should not only deliver the state of Israel, but himself reign as king in the throne of David with all secular pomp and dignity; that he should subdue the rest of the world, and make Jerusalem the seat of an universal monarchy. Seeing therefore they had forsaken all to follow him, being now in so mean condition, they did not think but that together with him they also should rise in state; that they should be the first and the most advanced by him. Of this conceit it came that the mother of the sons of Zebedee sued for her children’s preferment; of this conceit it grew, that the Apostles began to question amongst themselves which of them should be greatest; and in controlment of this conceit it was that our Lord so plainly told them, that the thoughts of their hearts were vain:” the kings of nations have indeed their large and ample dominions, they reign far and wide, and their servants they advance unto honour in the world; they bestow upon them large and ample secular preferments, in which respect they are also termed many of them benefactors, because of the liberal hand which they use in rewarding such as have done them service: but was it the meaning of the ancient prophets of God that the Messias the king of Israel should be like unto these kings, and his retinue grow in such sort as theirs? “Wherefore ye are not to look for at my hands such preferment as kings of nations are wont to bestow upon their attendants, ‘With you not so.’ Your reward in heaven shall be most ample, on earth your chiefest honour must be to suffer persecution for righteousness’ sake; submission, humility and meekness are things fitter for you to inure your minds withal, than these aspiring cogitations; if any amongst you be greater than other, let him shew himself greatest in being lowliest, let him be above them in being under them, even as a servant for their good. These are affections which you must put on; as for degrees of preferment and honour in this world, if ye expect any such thing at my hands ye deceive yourselves, for in the world your portion is rather the clear contrary.”

[5]Wherefore they who allege this place against episcopal  authority abuse it, they many ways deprave and wrest it, clean from the true understanding wherein our Saviour himself did utter it.

For first, whereas he by way of mere negation had said, “With you it shall not be so,” foretelling them only that it should not so come to pass as they vainly surmised; these men take his words in the plain nature of a prohibition, as if Christ had thereby forbidden all inequality of ecclesiastical power. Secondly, whereas he did but cut off their idle hope of secular advancements; all standing superiority amongst persons ecclesiastical these men would rase off with the edge of his speech. Thirdly, whereas he in abating their hope even of secular advancements spake but only with relation unto himself, informing them that he would be no such munificent Lord unto them in their temporal dignity and honour, as they did erroneously suppose; so that any Apostle might afterwards have grown by means of others to be even emperor of Rome, for any thing in those words to the contrary: these men removing quite and clean the hedge of all such restraints, enlarge so far the bounds of his meaning, as if his very precise intent and purpose had been not to reform the error of his Apostles conceived as touching him, and to teach what himself would not be towards them, but to prescribe a special law both to them and their successors for ever; a law determining what they should not be in relation of one to another, a law forbidding that any such title should be given to any minister as might import or argue in him a superiority over other ministers.

[6]Being thus defeated of that succour which they thought their cause might have had out of the words of our Saviour Christ, they try their adventure in seeking what  aid man’s testimony will yield them: “Cyprian objecteth it to Florentinus as a proud thing, that by believing evil reports, and misjudging of Cyprian, he made himself bishop of a bishop, and judge over him whom God had for the time appointed to be judge.” “The endeavour of godly men to strike at these insolent names may appear in the council of Carthage: where it was decreed, that the bishop of the chief see should not be entitled the exarch of priests, or the highest priest, or any other thing of like sense, but only the bishop of the chiefest see; whereby are shut out the name of archbishop, and all other such haughty titles.” In these allegations it fareth, as in broken reports snatched out of the author’s mouth, and broached before they be half either told on the one part, or on the other understood. The matter which Cyprian complaineth of in Florentinus was thus: Novatus misliking the easiness of Cyprian to admit men into the fellowship of believers after they had fallen away from the bold and constant confession of Christian faith, took thereby occasion to separate himself from the Church, and being united with certain excommunicate persons, they joined their wits together, and drew out against Cyprian their lawful bishop sundry grievous accusations; the crimes such, as being true, had made him uncapable of that office whereof he was six years as then possessed; they went to Rome, and to other places, accusing him every where as guilty of those faults of which themselves had lewdly condemned him, pretending that twenty-five African bishops (a thing most false) had heard and examined his  cause in a solemn assembly, and that they all had given their sentence against him, holding his election by the canons of the church void. The same factious and seditious persons coming also unto Florentinus, who was at that time a man imprisoned for the testimony of Jesus Christ, but yet a favourer of the error of Novatus, their malicious accusations he over-willingly hearkened unto, gave them credit, concurred with them, and unto Cyprian in fine wrote his letters against Cyprian: which letters he justly taketh in marvellous evil part, and therefore severely controlleth his so great presumption in making himself a judge of a judge, and, as it were, a bishop’s bishop, to receive accusations against him, as one that had been his ordinary. “2What height of pride is this (saith Cyprian), what arrogancy of spirit, what a puffing up of mind, to call guides and priests to be examined and sifted before him! So that unless we shall be cleared in your court, and absolved by your sentence, behold for these six years’ space neither shall the brotherhood have had a bishop, nor the people a guide, nor the flock a shepherd, nor the Church a governor, nor Christ a prelate, nor God a priest.” This is the pride which Cyprian condemneth in Florentinus, and not the title or name of archbishop; about which matter there was not at that time so much as the dream of any controversy at all between them. A silly collection it is, that because Cyprian reproveth Florentinus for lightness of belief and presumptuous rashness of judgment, therefore he held the title of archbishop to be a vain and a proud name.

[7]Archbishops were chief amongst bishops, yet archbishops had not over bishops that full authority which every bishop had over his own particular clergy: bishops were not  subject unto their archbishop as an ordinary, by whom at all times they were to be judged, according to the manner of inferior pastors, within the compass of each diocess. A bishop might suspend, excommunicate, depose, such as were of his own clergy without any other bishops assistants; not so an archbishop the bishops that were in his own province, above whom divers prerogatives were given him, howbeit no such authority and power as alone to be judge over them. For as a bishop could not be ordained, so neither might he be judged by any one only bishop, albeit that bishop were his metropolitan. Wherefore Cyprian, concerning the liberty and freedom which every bishop had, spake in the council of Carthage, whereat fourscore and seven bishops were present, saying, “It resteth that every of us declare what we think of this matter, neither judging nor severing from the right of communion any that shall think otherwise: for of us there is not any which maketh himself a bishop of bishops, or with tyrannical fear constraineth his colleagues unto the necessity of obedience, inasmuch as every bishop, according to the reach of his liberty and power, hath his own free judgment, and can no more have another his judge, than himself be judge to another.” Whereby it appeareth, that amongst the African bishops none did use such authority over any as the bishop of Rome did afterwards claim over all, forcing upon them opinions by main and absolute power. Wherefore unto the bishop of Rome the same Cyprian also writeth concerning his opinion about baptism: “These  things we present unto your conscience, most dear brother, as well for common honour’s sake, as of single and sincere love, trusting that as you are truly yourself religious and faithful, so those things which agree with religion and faith will be acceptable unto you: howbeit we know, that what some have over drunk in, they will not let go, neither easily change their mind, but with care of preserving whole amongst their brethren the bond of peace and concord, retaining still to themselves certain their own opinions wherewith they have been inured; wherein we neither use force, nor prescribe a law unto any, knowing that in the government of the Church every ruler hath his own voluntary free judgment, and of that which he doth shall render unto the Lord himself an account.”

[8]As for the council of Carthage, doth not the very first canon thereof establish with most effectual terms all things which were before agreed on in the council of Nice? and that the council of Nice did ratify the preeminence of metropolitan bishops, who is ignorant? The name of an archbishop importeth only a bishop having chiefty of certain prerogatives above his brethren of the same order. Which thing, sith the council of Nice doth allow, it cannot be that the other of Carthage should condemn it, inasmuch as this doth yield unto that a Christian unrestrained approbation. The thing provided for by the synod of Carthage can be no other therefore, than only that the chiefest metropolitan, where many archbishops were within any greater province, should not be termed by those names, as to import the power of an ordinary jurisdiction belonging in such degree and manner unto him over the rest of the bishops and archbishops, as did belong unto every bishop over other pastors under him.

But much more absurd it is to affirm, that both Cyprian  and the council of Carthage condemn even such superiority also of bishops themselves over pastors their inferiors, as the words of Ignatius imply, in terming the bishop “a prince of priests.” Bishops to be termed arch-priests, in regard of their superiority over priests, is in the writings of the ancient Fathers a thing so usual and familiar, as almost no one thing more. At the council of Nice, saith Theodoret, three hundred and eighteen arch-priests were present. Were it the meaning of the council of Carthage, that the title of chief priests and such like ought not in any sort at all to be given unto any Christian Bishop, what excuse should we make for so many ancient both Fathers, and synods of Fathers, as have generally applied the title of arch-priest unto every bishop’s office?

[9]High time I think it is, to give over the obstinate defence of this most miserable forsaken cause; in the favour whereof neither God, nor amongst so many wise and virtuous men as antiquity hath brought forth, any one can be found to have hitherto directly spoken. Irksome confusion must of  necessity be the end whereunto all such vain and ungrounded confidence doth bring, as hath nothing to bear it out but only an excessive measure of bold and peremptory words, holpen by the start of a little time, before they came to be examined. In the writings of the ancient Fathers, there is not any thing with more serious asseveration inculcated, than that it is God which maketh bishops, that their authority hath divine allowance, that the bishop is the priest of God, that he is judge in Christ’s stead, that according to God’s own law the whole Christian fraternity standeth bound to obey him. Of this there was not in the Christian world of old any doubt or controversy made, it was a thing universally every where agreed upon. What should move men to judge that now so unlawful and naught, which then was so reverendly esteemed? Surely no other cause but this, men were in those times meek, lowly, tractable, willing to live in dutiful awe and subjection unto the pastor of their souls: now we imagine ourselves so able every man to teach and direct all others, that none of us can brook it to have superiors; and for a mask to hide our pride, we pretend falsely the law of Christ, as if we did seek the execution of his will, when in truth we labour for the mere satisfaction of our own against his.

\section*{The second malicious thing wherein the state of Bishops suffereth obloquy is their honour.}

XVII. The chiefest cause of disdain and murmur against bishops in the Church of England is that evil-affected eye wherewith the world looked upon them, since the time that irreligious profaneness, beholding the due and just advancements of God’s clergy, hath under pretence of enmity unto ambition and pride proceeded so far, that the contumely of old offered unto Aaron in the like quarrel may seem very moderate and quiet dealing, if we compare it with the fury of our own times. The ground and original of both their proceedings one and the same; in declaration of their grievances they differ not; the complaints as well of the one as the other are, “Wherefore lift ye up yourselves” thus far “above the congregation of the Lord? It is too much which you take upon you;” too much power, and too much honour. Wherefore as we have shewed that there is not in their power any thing unjust or unlawful, so it resteth that in their honour also the like be done. The labour we take  unto this purpose is by so much the harder, in that we are forced to wrestle with the stream of obstinate affection, mightily carried by a wilful prejudice, the dominion whereof is so powerful over them in whom it reigneth, that it giveth them no leave, no not so much as patiently to hearken unto any speech which doth not profess to feed them in this their bitter humour. Notwithstanding, forasmuch as I am persuaded that against God they will not strive, if they perceive once that in truth it is he against whom they open their mouths, my hope is their own confession will be at the length, “Behold, we have done exceeding foolishly; it was the Lord, and we knew it not; him in his ministers we have despised, we have in their honour impugned his.” But the alteration of men’s hearts must be his good and gracious work, whose most omnipotent power framed them.

[2]Wherefore to come to our present purpose, honour is no where due, saving only unto such as have in them that whereby they are found, or at the least presumed, voluntarily beneficial unto them of whom they are honoured. Wheresoever nature seeth the countenance of a man, it still presumeth that there is in him a mind willing to do good, if need require, inasmuch as by nature so it should be; for which cause men unto men do honour, even for very humanity’s sake: and unto whom we deny all honour, we seem plainly to take from them all opinion of human dignity, to make no account or reckoning of them, to think them so utterly without virtue, as if no good thing in the world could be looked for at their hands. Seeing therefore it seemeth hard that we should so hardly think of any man, the precept of St. Peter is, “Honour all men.”

Which duty of every man towards all doth vary according to the several degrees whereby they are more or less beneficial, whom we do honour. “Honour the physician,” saith the wise man: the reason why, because for necessities’ sake God created him. Again, “Thou shalt rise up before the hoary head, and honour the person of the aged:” the reason why, because the younger sort have great benefit by their gravity, experience, and wisdom; for which cause,  these things the wise man termeth the crown or diadem of the aged. Honour due to parents: the reason why, because we have our beginning from them; “Obey the father that hath begotten thee, the mother that bare thee despise thou not.” Honour due unto kings and governors: the reason why, because God hath set them “for the punishment of evil doers, and for the praise of them that do well.” Thus we see by every of these particulars, that there is always some kind of virtue beneficial, wherein they excel who receive honour; and that degrees of honour are distinguished according to the value of those effects which the same beneficial virtue doth produce.

[3]Nor is honour only an inward estimation, whereby they are reverenced and well thought of in the minds of men; but honour whereof we now speak, is defined to be an external sign, by which we give a sensible testification that we acknowledge the beneficial virtue of others. Sarah honoured her husband Abraham; this appeareth by the title she gave him. The brethren of Joseph did him honour in the land of Egypt; their lowly and humble gesture sheweth it. Parents will hardly persuade themselves that this intentional honour, which reacheth no farther than to the inward conception only, is the honour which their children owe them. Touching that honour which mystically agreeing unto Christ, was yielded literally and really unto Solomon, the words of the Psalmist concerning it are, “Unto him they shall give of the gold of Sheba, they shall pray for him continually, and daily bless him.”

[4]Weigh these things in themselves, titles, gestures, presents, other the like external signs wherein honour doth consist, and they are matters of no great moment. Howbeit, take them away, let them cease to be required, and they are not things of small importance, which that surcease were likely to draw after it. Let the lord mayor of London, or any other unto whose office honour belongeth, be deprived but of that title which in itself is a matter of nothing; and suppose we that it would be a small maim unto the credit, force, and countenance of his office? It hath not without the  singular wisdom of God been provided, that the ordinary outward tokens of honour should for the most part be in themselves things of mean account; for to the end they might easily follow as faithful testimonies of that beneficial virtue whereunto they are due, it behoved them to be of such nature, that to himself no man might over-eagerly challenge them, without blushing; nor any man where they are due withhold them, but with manifest appearance of too great malice or pride.

[5]Now forasmuch as according to the ancient orders and customs of this land, as of the kingdom of Israel, and of all Christian kingdoms through the world, the next in degree of honour unto the chief sovereign are the chief prelates of God’s Church; what the reason hereof may be, it resteth next to be inquired.

\section*{What good doth publicly grow from the Prelacy.}

XVIII. Other reason there is not any, wherefore such honour hath been judged due, saving only that public good which the prelates of God’s clergy are authors of. For I would know which of these things it is whereof we make any question, either that the favour of God is the chiefest pillar to bear up kingdoms and states; or that true religion publicly exercised is the principal mean to retain the favour of God; or that the prelates of the Church are they without whom the exercise of true religion cannot well and long continue. If these three be granted, then cannot the public benefit of prelacy be dissembled.

[2]And of the first or second of these I look not for any professed denial; the world at this will blush, not to grant at the leastwise in word as much as heathens themselves have of old with most earnest asseveration acknowledged, concerning the force of divine grace in upholding kingdoms. Again, though his mercy doth so far strive with men’s ingratitude, that all kind of public iniquities deserving his indignation, their safety is through his gracious providence many times nevertheless continued to the end that amendment might  if it were possible avert their envy; so that as well commonweals as particular persons both may and do endure much longer, when they are careful, as they should be, to use the most effectual means of procuring his favour on whom their continuance principally dependeth: yet this point no man will stand to argue, no man will openly arm himself to enter into set disputation against the emperors Theodosius and Valentinian, for making unto their laws concerning religion this preface, “Decere arbitramur nostrum imperium, subditos nostros de religione commonefacere. Ita enim et pleniorem acquiri Dei ac Salvatoris nostri Jesu Christi benignitatem possibile esse existimamus, si quando et nos pro viribus ipsi placere studuerimus, et nostros subditos ad eam rem instituerimus:” or against the emperor Justinian, for that he also maketh the like profession: “Per sanctissimas ecclesias et nostrum imperium sustineri, et communes res clementissimi Dei gratia muniri credimus.” And in another place, “Certissime credimus, quia Sacerdotum puritas et decus, et ad Dominum Deum ac salvatorem nostrum Jesum Christum fervor, et ab ipsis missæ perpetuæ preces, multum favorem nostræ reipublicæ et incrementum præbent.”

[3]Wherefore only the last point is that which men will boldly require us to prove; for no man feareth now to make it a question, “whether the prelacy of the Church be any thing available or no to effect the good and long continuance of true religion?” Amongst the principal blessings wherewith God enriched Israel, the prophet in the Psalm acknowledgeth especially this for one, “Thou didst lead thy people like sheep by the hands of Moses and Aaron.” That which sheep are if pastors be wanting, the same are the people of God if so be they want governors: and that which the principal civil governors are in comparison of regents under them, the same are the prelates of the Church being compared with the rest of God’s clergy. Wherefore inasmuch as amongst the Jews the benefit of civil government grew principally from Moses, he being their principal civil  governor; even so the benefit of spiritual regiment grew from Aaron principally, he being in the other kind their principal rector, although even herein subject to the sovereign dominion of Moses. For which cause, these two alone are named as the heads and well-springs of all. As for the good which others did in service either of the commonwealth or of the sanctuary, the chiefest glory thereof did belong to the chiefest governors of the one sort and of the other, whose vigilant care and oversight kept them in their due order. Bishops are now as high priests were then, in regard of power over other priests: and in respect of subjection unto high priests, what priests were then, the same now presbyters are by way of their place under bishops. The one’s authority therefore being so profitable, how should the other’s be thought unnecessary? Is there any man professing Christian religion which holdeth it not as a maxim, that the Church of Jesus Christ did reap a singular benefit by apostolical regiment, not only for other respects, but even in regard of that prelacy whereby they had and exercised power of jurisdiction over lower guides of the Church? Prelates are herein the Apostles’ successors, as hath been proved.

[4]Thus we see that prelacy must needs be acknowledged exceedingly beneficial in the Church; and yet for more perspicuity’s sake, it shall not be pains superfluously taken, if the manner how be also declared at large. For this one thing not understood by the vulgar sort, causeth all contempt to be offered unto higher powers, not only ecclesiastical, but civil: whom when proud men have disgraced, and are therefore reproved by such as carry some dutiful affection of mind, the usual apologies which they make for themselves are these: “What more virtue in these great ones than in others? We see no such eminent good which they do above other men.”

We grant indeed, that the good which higher governors do is not so immediate and near unto every of us, as many times the meaner labours of others under them, and this doth make it to be less esteemed. But we must note, that it is  in this case as in a ship; he that sitteth at the stern is quiet, he moveth not, he seemeth in a manner to do little or nothing in comparison of them that sweat about other toil, yet that which he doth is in value and force more than all the labours of the residue laid together. The influence of the heavens above worketh infinitely more to our good, and yet appeareth not half so sensible as the force doth of things below. We consider not what it is which we reap by the authority of our chiefest spiritual governors, nor are likely to enter into any consideration thereof, till we want them; and that is the cause why they are at our hands so unthankfully rewarded.

[5]Authority is a constraining power, which power were needless if we were all such as we should be, willing to do the things we ought to do without constraint. But because generally we are otherwise, therefore we all reap singular benefit by that authority which permitteth no men, though they would, to slack their duty. It doth not suffice, that the lord of an household appoint labourers what they should do, unless he set over them some chief workmen to see they do it. Constitutions and canons made for the ordering of church affairs are dead taskmasters. The due execution of laws spiritual dependeth most upon the vigilant care of the chiefest spiritual governors, whose charge is to see that such laws be kept by the clergy and people under them: with those duties which the law of God and the ecclesiastical canons require in the clergy, lay governors are neither for the most part so well acquainted, nor so deeply and nearly touched. Requisite therefore it is, that ecclesiastical persons have authority in such things; which kind of authority maketh them that have it prelates. If then it be a thing confessed, as by all good men it needs must be, to have prayers read in all churches, to have the sacraments of God administered, to have the mysteries of salvation painfully taught, to have God every where devoutly worshipped, and all this perpetually, and with quietness, bringeth unto the whole Church, and unto every member thereof, inestimable good; how can that authority which hath been proved the ordinance of God for preservation of these duties in the Church, how can it choose but deserve to be held a thing publicly most beneficial?


[6]It were to be wished, and is to be laboured for, as much as can be, that they who are set in such rooms may be furnished with honourable qualities and graces, every way fit for their calling: but be they otherwise, howsoever, so long as they are in authority, all men reap some good by them, albeit not so much good as if they were abler men. There is not any amongst us all, but is a great deal more apt to exact another man’s duty, than the best of us is to discharge exactly his own; and therefore prelates, although neglecting many ways their duty unto God and men, do notwithstanding by their authority great good, in that they keep others at the leastwise in some awe under them. It is our duty therefore in this consideration, to honour them that rule as prelates; which office if they discharge well, the Apostle’s own verdict is, that the honour they have they be worthy of, yea though it were double. And if their government be otherwise, the judgment of sage men hath ever been this, that albeit the dealings of governors be culpable, yet honourable they must be, in respect of that Authority by which they govern. Great caution must be used that we neither be emboldened to follow them in evil, whom for authority’s sake we honour; nor induced in authority to dishonour them, whom as examples we may not follow. In a word, not to dislike sin, though it should be in the highest, were unrighteous meekness; and proud righteousness it is to contemn or dishonour highness, though it should be in the sinfullest men that live.

[7]But so hard it is to obtain at our hands, especially as now things stand, the yielding of honour to whom honour in this case belongeth, that by a brief declaration only what the duties of men are towards the principal guides and pastors of their souls, we cannot greatly hope to prevail, partly for the malice of their open adversaries, and partly for the cunning of such as in a sacrilegious intent work their dishonour under covert, by more mystical and secret means. Wherefore requisite and in a manner necessary it is, that by particular instances we make it even palpably manifest what singular benefit and use public the nature of prelates is apt to yield.


First, no man doubteth but that unto the happy condition of commonweals it is a principal help and furtherance, when in the eye of foreign states their estimation and credit is great. In which respect, the Lord himself commending his own laws unto his people, mentioneth this as a thing not meanly to be accounted of, that their careful obedience yielded thereunto should purchase them a great good opinion abroad, and make them every where famous for wisdom. Fame and reputation groweth especially by the virtue, not of common ordinary persons, but of them which are in each estate most eminent by occasion of their higher place and calling. The mean man’s actions, be they good or evil, they reach not far, they are not greatly inquired into, except perhaps by such as dwell at the next door: whereas men of more ample dignity are as cities on the tops of hills, their lives are viewed afar off; so that the more there are which observe aloof what they do, the greater glory by their well-doing they purchase, both unto God whom they serve, and to the state wherein they live. Wherefore if the clergy be a beautifying unto the body of this commonweal in the eyes of foreign beholders, and if in the clergy the prelacy be most exposed unto the world’s eye, what public benefit doth grow from that order, in regard of reputation thereby gotten to the land from abroad, we may soon conjecture. Amongst the Jews (their kings excepted) who so renowned throughout the world as their high priest? Who so much or so often spoken of as their prelates?

[8](.) Which order is not for the present only the most in sight, but for that very cause also the most commended unto posterity. For if we search those records wherein there hath descended from age to age whatsoever notice and intelligence we have of those things which were before us, is there any thing almost else, surely not any thing so much, kept in memory, as the successions, doings, sufferings, and affairs of prelates. So that either there is not any public use of that light which the Church doth receive from antiquity; or if this be absurd to think, then must we necessarily acknowledge ourselves beholding more unto prelates than unto others their  inferiors, for that good of direction which ecclesiastical actions recorded do always bring.

[9]Thirdly, But to call home our cogitations, and more inwardly to weigh with ourselves what principal commodity that order yieldeth, or at leastwise is of its own disposition and nature apt to yield: kings and princes, partly for information of their own consciences, partly for instruction what they have to do in a number of most weighty affairs, entangled with the cause of religion, having, as all men know, so usual occasion of often consultations and conferences with their clergy; suppose we, that no public detriment would follow upon the want of honourable personages ecclesiastical to be used in those cases? It will be haply said, “That the highest might learn to stoop, and not to disdain the advice of some circumspect, wise and virtuous minister of God, albeit the ministry were not by such degrees distinguished.” What princes in that case might or should do, it is not material. Such difference being presupposed therefore, as we have proved already to have been the ordinance of God, there is no judicious man will ever make any question or doubt, but that fit and direct it is for the highest and chiefest order in God’s clergy to be employed before others about so near and necessary offices as the sacred estate of the greatest on earth doth require. For this cause Joshua had Eleazar; David, Abiathar; Constantine, Hosius, bishop of Corduba; other emperors and kings their prelates, by whom in private (for with princes this is the most effectual way of doing good) to be admonished, counselled, comforted, and if need were, reproved. Whensoever sovereign rulers are willing to admit these so necessary private conferences for their spiritual and ghostly good, inasmuch as they do for the time while they take advice grant a kind of superiority unto them of whom they receive it, albeit haply they can be contented even so far to bend to the gravest and chiefest persons in the order of God’s clergy; yet this of the very best being rarely and hardly obtained, now that there are whose greater and higher callings do somewhat more proportion them unto that ample conceit and spirit wherewith the minds of so powerable persons are possessed; what should we look for, in case God himself not authorizing any by miraculous means, as of old he did his prophets, the  equal meanness of all did leave, in respect of calling, no more place of decency for one than for another to be admitted? Let unexperienced wits imagine what pleaseth them, in having to deal with so great personages these personal differences are so necessary that there must be regard had of them.

[10]Fourthly, Kingdoms being principally (next unto God’s Almightiness, and the sovereignty of the highest under God) upheld by wisdom and by valour, as by the chiefest human means to cause continuance in safety with honour (for the labours of them who attend the service of God, we reckon as means divine, to procure our protection from heaven); from hence it riseth, that men excelling in either of these, or descending from such as for excellency either way have been ennobled, or possessing howsoever the rooms of such as should be in politic wisdom or in martial prowess eminent, are had in singular recommendation. Notwithstanding, because they are by the state of nobility great, but not thereby made inclinable to good things; such they oftentimes prove even under the best princes, as under David certain of the Jewish nobility were. In polity and counsel the world had not Achitophel’s equal, nor hell his equal in deadly malice. Joab the general of the host of Israel, valiant, industrious, fortunate in war, but withal headstrong, cruel, treacherous, void of piety towards God; in a word, so conditioned, that easy it is not to define, whether it were for David harder to miss the benefit of his warlike ability, or to bear the enormity of his other crimes. As well for the cherishing of those virtues therefore, wherein if nobility do chance to flourish, they are both an ornament and a stay to the commonwealth wherein they live; as also for the bridling of those disorders, which if they loosely run into, they are by reason of their greatness dangerous; what help could there ever have been invented more divine, than the sorting of the clergy into such degrees, that the chiefest of the prelacy being matched in a kind of equal yoke, as it were, with the higher, the next with the lower degree of nobility, the reverend authority of the one might be to the other as a courteous bridle, a mean to keep them lovingly in awe that are exorbitant, and to correct such excesses in them, as whereunto their courage, state and dignity maketh them over-prone? O that there were for encouragement of prelates herein, that  inclination of all Christian kings and princes towards them, which sometime a famous king of this land either had, or pretended to have, for the countenancing of a principal prelate under him in the actions of spiritual authority! “Let my lord archbishop know,” saith he, “that if a bishop, or earl, or any other great person, yea if my own chosen son, shall presume to withstand or to hinder his will and disposition, whereby he may be withheld from performing the work of the embassage committed unto him; such a one shall find, that of his contempt I will shew myself no less a persecutor and revenger, than if treason were committed against mine own very crown and dignity.” Sith therefore by the fathers and first founders of this commonweal it hath upon great experience and forecast being judged most for the good of all sorts, that as the whole body politic wherein we live should be for strength’s sake a threefold cable, consisting of the king as a supreme head over all, of peers and nobles under him, and of the people under them; so likewise, that in this conjunction of states, the second wreath of that cable should, for important respects, consist as well of lords spiritual as temporal: nobility and prelacy being by this mean twined together, how can it possibly be avoided, but that the tearing away of the one must needs exceedingly weaken the other, and by consequent impair greatly the good of all?

[11](Fifthly.) The force of which detriment there is no doubt but that the common sort of men would feel to their helpless woe, how goodly a thing soever they now surmise it to be, that themselves and their godly teachers did all alone without controlment of their prelate. For if the manifold jeopardies whereto a people destitute of pastors is subject be unavoidable without government; and if the benefit of government,  whether it be ecclesiastical or civil, do grow principally from them who are principal therein, as hath been proved out of the prophet, who albeit the people of Israel had sundry inferior governors, ascribeth not unto them the public benefit of government, but maketh mention of Moses and Aaron only, the chief prince and chief prelate, because they were the wellspring of all the good which others under them did: may we not boldly conclude, that to take from the people their prelate is to leave them in effect without guides, as leastwise without those guides which are the strongest hands that God doth direct them by? “Thou didst lead thy people like sheep,” saith the Prophet, “by the hands of Moses and Aaron.”

If now there arise any matter of grievance between the pastor and the people that are under him, they have their ordinary, a judge indifferent to determine their causes, and to end their strife. But in case there were no such appointed to sit and to hear both, what would then be the end of their quarrels? They will answer perhaps, “That for such purposes their synods shall serve.” Which is as if in the commonwealth the higher magistrates being removed, every township should be a state, altogether free and independent; and the controversies which they cannot end speedily within themselves, to the contentment of both parties, should be all determined by solemn parliaments. Merciful God! where is the light of wit and judgment, which this age doth so much vaunt of and glory in, when unto these such odd imaginations so great not only assent, but also applause is yielded?

[12](Sixthly.) As for those in the clergy whose place and calling is lower, were it not that their eyes are blinded lest they should see the thing that of all others is for their good most effectual, somewhat they might consider the benefit which they enjoy by having such in authority over them as are of the selfsame profession, society and body with them; such as have trodden the same steps before; such as know by their own experience the manifold intolerable contempts and indignities which faithful pastors, intermingled with the multitude, are constrained every day to suffer in the exercise of their spiritual charge and function, unless their superiors, taking their causes even to heart, be by a kind of sympathy drawn to  relieve and aid them in their virtuous proceedings, no less effectually than loving parents their dear children.

Thus therefore prelacy, being unto all sorts so beneficial, ought accordingly to receive honour at the hands of all; but we have just cause exceedingly to fear that those miserable times of confusion are drawing on, wherein “the people shall be oppressed one of another;” inasmuch as already that which prepareth the way thereunto is come to pass, “children presume against the ancient, and the vile against the honourable:” Prelacy, the temperature of excesses in all estates, the glue and soder of the public weal, the ligament which tieth and connecteth the limbs of this body politic each to other, hath instead of deserved honour, all extremity of disgrace. The foolish every where plead, that unto the wise in heart they owe neither service, subjection, nor honour.

\section*{What kinds of honour be due unto Bishops.}

XIX. Now that we have laid open the causes for which honour is due unto prelates, the next thing we are to consider is, what kinds of honour be due. The good government either of the Church or the commonwealth dependeth scarcely on any one external thing so much as on the public marks and tokens, whereby the estimation that governors are in is made manifest to the eyes of men. True it is, that governors are to be esteemed according to the excellency of their virtues; the more virtuous they are, the more they ought to be honoured, if respect be had unto that which every man should voluntarily perform unto his superiors. But the question is now, of that honour which public order doth appoint unto church-governors, in that they are governors; the end whereof is, to give open sensible testimony, that the place which they hold is judged publickly in such degree beneficial, as the marks of their excellency, the honours appointed to be done unto them do import. Wherefore this honour we are to do them, without presuming ourselves to examine how worthy they are, and withdrawing it if by us they be thought unworthy. It is a note of that public judgment which is given of them; and therefore not tolerable that men in private should by refusal to do them such honour reverse as much as in them lieth the public judgment. If it deserve such grievous punishment, when any particular person adventureth to deface those marks whereby is signified what  value some small piece of coin is publickly esteemed at; is it sufferable that honours, the characters of that estimation which publickly is had of public estates and callings in the Church or commonwealth, should at every man’s pleasure be cancelled?

[2]Let us not think that without most necessary cause the same have been thought expedient. The first authors thereof were wise and judicious men; they knew it a thing altogether impossible, for each particular in the multitude to judge what benefit doth grow unto them from their prelates, and thereupon uniformly to yield them convenient honour. Wherefore that all sorts might be kept in obedience and awe, doing that unto their superiors of every degree, not which every man’s special fancy should think meet, but which being beforehand agreed upon as meet, by public sentence and decision, might afterwards stand as a rule for each in particular to follow; they found that nothing was more necessary, than to allot unto all degrees their certain honour, as marks of public judgment concerning the dignity of their places; which mark when the multitude should behold, they might be thereby given to know, that of such or such estimation their governors are, and in token thereof do carry those notes of excellency. Hence it groweth, that the different notes and signs of honour do leave a correspondent impression in the minds of common beholders. Let the people be asked who are the chiefest in any kind of calling? who most to be listened unto? who of greatest account and reputation? and see if the very discourse of their minds lead them not unto those sensible marks, according to the difference whereof they give their suitable judgment, esteeming them the worthiest persons who carry the principal note and public mark of worthiness. If therefore they see in other estates a number of tokens sensible, whereby testimony is given what account there is publickly made of them, but no such thing in the clergy; what will they hereby, or what can they else conclude, but that where they behold this, surely in that commonwealth, religion and they that are conversant about it are not esteemed greatly beneficial? Whereupon in time the open contempt of God and godliness must needs ensue: “Qui bona fide Deum colit, amat et sacerdotes,” saith Papinius. In vain doth that kingdom or  commonwealth pretend zeal to the honour of God, which doth not provide that his clergy also may have honour. xx. .

[3]Now if all that are employed in the service of God should have one kind of honour, what more confused, absurd, and unseemly? Wherefore in the honour which hath been allotted unto God’s clergy, we are to observe, how not only the kinds thereof, but also in every particular kind, the degrees do differ. The honour which the clergy of God hath hitherto enjoyed, consisteth especially in the preeminence of Title, Place, Ornament, Attendance, Privilege, Endowment. In every of which it hath been evermore judged meet, that there should be no small odds between prelates and the inferior clergy.

\section*{Honour in Title, Place, Ornament, Attendancy, and Privilege.}

XX. Concerning title, albeit even as under the law all they whom God hath severed to offer him sacrifice were generally termed priests, so likewise the name of pastor or presbyter be now common unto all that serve him in the ministry of the gospel of Jesus Christ; yet both then and now the higher orders, as well of the one sort as of the other, have by one and the same congruity of reason their different titles of honour, wherewith we find them in the phrase of ordinary speech exalted above others. Thus the heads of the twenty-four companies of priests are in Scripture termed arch-priests; Aaron and the successors of Aaron being above those arch-priests, themselves are in that respect further entitled high and great. After what sort antiquity hath used to style Christian bishops, and to yield them in that kind honour more than were meet for inferior pastors, I may the better omit to declare, both because others have sufficiently done it already, and in so slight a thing it were but a loss of time to bestow further travel. The allegation of Christ’s prerogative to be named an arch-pastor simply, in regard of  his absolute excellency over all, is no impediment but that the like title in an unlike signification may be granted unto others besides him, to note a more limited superiority, whereof men are capable enough without derogation from his glory, than which nothing is more sovereign. To quarrel at syllables, and to take so poor exceptions at the first four letters in the name of an archbishop, as if they were manifestly stolen goods whereof restitution ought to be made to the civil magistrate toucheth no more the prelates that now are, than it doth the very blessed Apostle, who giveth unto himself the title of an archbuilder.

As for our Saviour’s words alleged against the title of lordship and grace, we have before sufficiently opened how far they are drawn from their natural meaning, to bolster up a cause which they nothing at all concern. Bishops Theodoret entitleth “most honourable.” Emperors writing unto bishops, have not disdained to give them their appellations of honour, “Your holiness,” “Your blessedness,” “Your amplitude,” “Your highness,” and the like: such as purposely have done otherwise are noted of insolent singularity and pride.

[2]Honour done by giving preeminence of place unto one sort before another, is for decency, order, and quietness’ sake so needful, that both imperial laws and canons ecclesiastical have made their special provisions for it. Our  Saviour’s invective against the vain affectation of superiority, whether in title or in place, may not hinder these seemly differences usual in giving and taking honour, either according to the one or the other.

[3]Something there is even in the ornaments of honour also; otherwise idle it had been for the wise man speaking of Aaron, to stand so much upon the circumstance of his priestly attire, and to urge it as an argument of such dignity and greatness in him: “An everlasting covenant God made with Aaron, and gave him the priesthood among the people, and made him blessed through his comely ornament, and clothed him with the garment of honour.” The robes of a judge do not add to his virtue; the chiefest ornament of kings is justice; holiness and purity of conversation do much more adorn a bishop, than his peculiar form of clothing. Notwithstanding, both judges, through the garments of judicial authority, and through the ornaments of sovereignty, princes; yea bishops through the very attire of bishops, are made blessed, that is to say, marked and manifested they are to be such as God hath poured his blessing upon, by advancing them above others, and placing them where they may do him principal good service. Thus to be called is to be blessed, and therefore to be honoured with the signs of such a calling must needs be in part a blessing also; for of good things even the signs are good.

[4]Of honour, another part is attendancy; and therefore in the visions of the glory of God, angels are spoken of as his attendants. In setting out the honour of that mystical queen, the prophet mentioneth the virgin ladies which waited on her. Amongst the tokens of Solomon’s honourable condition, his servants and waiters the sacred history omitteth not.

This doth prove attendants a part of honour: but this as yet doth not shew with what attendancy prelates are to be honoured. Of the high-priest’s retinue amongst the Jews, somewhat the Gospel itself doth intimate. And albeit our Saviour came to minister, and not, as the Jews did imagine  their Messias should, to be ministered unto in this world, yet attended on he was by his blessed Apostles, who followed him not only as scholars, but even as servants about him. After that he had sent them, as himself was sent of God, in the midst of that hatred and extreme contempt which they sustained at the world’s hands, by saints and believers this part of honour was most plentifully done unto them. Attendants they had provided in all places where they went; which custom of the Church was still continued in bishops their successors, as by Ignatius it is plain to be seen. And from hence no doubt those Acolythes took their beginning, of whom so frequent mention is made; the bishop’s attendants, his followers they were: in regard of which service the name of Acolythes seemeth plainly to have been given. The custom for bishops to be attended upon by many is as Justinian doth shew ancient: the affairs of regiment, wherein prelates are employed, make it necessary that they always have many about them whom they may command, although no such thing did by way of honour belong unto them.

Some men’s judgment is, that if clerks, students, and religious persons were more, common serving men and lay retainers fewer than they are in bishops’ palaces, the use and the honour thereof would be much more suitable than now. But these things, concerning the number and quality of persons fit to attend on prelates, either for necessity, or for  honour’s sake, are rather in particular discretion to be ordered, than to be argued of by disputes.

[5]As for the vain imagination of some, who teach the original hereof to have been a preposterous imagination of Maximinus the emperor, who being addicted unto idolatry, chose of the choicest magistrates to be priests, and to the end they might be in great estimation, gave unto each of them a train of followers; and that Christian emperors, thinking the same would promote Christianity which promoted superstition, endeavoured to make their bishops encounter and match with those idolatrous priests: such frivolous conceits, having no other ground than conceit, we weigh not so much as to frame any answer unto them; our declaration of the true original of ancient attendancy on bishops being sufficient. Now if that which the light of sound reason doth teach to be fit, have upon like inducements reasonable, allowable, and good, approved itself in such wise as to be accepted, not only of us, but of pagans and infidels also; doth conformity with them that are evil in that which is good, make that thing which is good evil? We have not herein followed the heathens, nor the heathens us, but both we and they one and the selfsame divine rule, the light of a true and sound understanding,  which sheweth what honour is fit for prelates, and what attendancy convenient to be a part of their honour.

Touching privileges granted for honour’s sake, partly in general unto the clergy, and partly unto prelates the chiefest persons ecclesiastical in particular; of such quality and number they are, that to make but rehearsal of them we scarce think it safe, lest the very entrails of some of our godly brethren, as they term themselves, should thereat haply burst in sunder.

\section*{Honour by endowment with Lands and Livings.}

XXI. And yet of all these things rehearsed, it may be there never would have grown any question, had bishops been honoured only thus far forth. But the honouring of the clergy with wealth, this is in the eyes of them which pretend to seek nothing but mere reformation of abuses, a sin that can never be remitted.

How soon, O how soon might the Church be perfect, even without any spot or wrinkle, if public authority would at the length say Amen unto the holy and devout requests of those godly brethren, who as yet with outstretched necks groan in the pangs of their zeal to see the houses of bishops rifled, and their so long desired livings gloriously divided amongst the righteous! But there is an impediment, a let, which somewhat hindereth those good men’s prayers from taking effect: they in whose hands the sovereignty of power and dominion over this Church doth rest, are persuaded there is a God; for undoubtedly either the name of Godhead is but a feigned thing, or if in heaven there be a God, the sacrilegious intention  of Church robbers, xxii. . which lurketh under this plausible name of Reformation, is in his sight a thousand times more hateful than the plain professed malice of those very miscreants, who threw their vomit in the open face of our blessed Saviour.

[2]They are not words of persuasion by which true men can hold their own, when they are over beset with thieves. And therefore to speak in this cause at all were but labour lost, saving only in respect of them, who being as yet unjoined unto this conspiracy, may be haply somewhat stayed, when they shall know betimes what it is to see thieves and to run on with them, as the Prophet in the Psalm speaketh; “When thou sawest a thief, then thou consentedst with him, and hast been partaker with adulterers.”

For the better information therefore of men which carry true, honest and indifferent minds, these things we will endeavour to make most clearly manifest: First, That in goods and livings of the Church none hath propriety but God himself. Secondly, That the honour which the clergy therein hath, is to be, as it were, God’s Receivers; the honour of prelates, to be his chief and principal Receivers. Thirdly, That from him they have right, not only to receive, but also to use such goods, the lower sort in smaller, and the higher in larger measure. Fourthly, That in case they be thought, yea, or found to abuse the same, yet may not such honour be therefore lawfully taken from them, and be given away unto persons of other calling.

\section*{That of ecclesiastical goods, and consequently of the lands and livings which Bishops enjoy, the propriety belongeth unto God alone.}

XXII. Possessions, lands and livings spiritual, the wealth of the clergy, the goods of the Church, are in such sort the Lord’s own, that man can challenge no propriety in them. His they are, and not ours; all things are his, in that from him they have their being. “My corn, and my wine, and mine oil,” saith the Lord. All things his, in that he hath absolute power to dispose of them at his pleasure. “Mine (saith he3) are the sheep and oxen of a thousand hills.” All things his, in that when we have them, we may say with Job, “God hath given;” and when we are deprived of them, “The Lord,” whose they are, hath likewise “taken  them away” again. But these sacred possessions are his by another tenure; his, because those men who first received them from him have unto him returned them again by way of religious gift or oblation: and in this respect it is, that the Lord doth term those houses wherein such gifts and oblations were laid, “his treasuries.”

[2]The ground whereupon men have resigned their own interests in things temporal, and given over the same unto God, is that precept which Solomon borroweth from the law of nature, “Honour the Lord out of thy substance, and of the chiefest of all thy revenue: so shall thy barns be filled with plenty, and with new wine the fat of thy press shall overflow.” For although it be by one most fitly spoken against those superstitious persons that only are scrupulous in external rites; “Wilt thou win the favour of God? be virtuous: they best worship him that are his followers:” it is not the bowing of your knees, but of your hearts; it is not the number of your oblations, but the integrity of your lives; not your incense, but your obedience, which God is delighted to be honoured by: nevertheless, we must beware, lest simply understanding this, which comparatively is meant; that is to say, whereas the meaning is, that God doth chiefly respect the inward disposition of the heart; we must take heed we do not hereupon so worship him in spirit, that outwardly we take all worship, reverence and honour from him.

[3]Our God will be glorified both of us himself, and for us by others: to others because our hearts are [not?] known, and yet our example is required for their good, therefore it is not sufficient to carry religion in our hearts, as fire is carried in flint-stones, but we are outwardly, visibly, apparently, to serve and honour the living God; yea to employ that way, as not only our souls but our bodies, so not only our bodies but our goods, yea, the choice, the flower, the chiefest of all thy revenue, saith Solomon. If thou hast any thing in all thy possessions of more value and price than other, to what use shouldest thou convert it, rather than this? Samuel was dear unto Hannah his mother: the child that  Hannah did so much esteem, she could not but greatly wish to advance; and her religious conceit was, that the honouring of God with it was the advancing of it unto honour. The chiefest of the offspring of men are the males which be first born: and for this cause, in the ancient world they all were by right of their birth priests to the Most High. By these and the like precedents, it plainly enough appeareth, that in what heart soever doth dwell unfeigned religion, in the same there resteth also a willingness to bestow upon God that soonest which is most dear. Amongst us the law is, that sith gold is the chiefest of metals, if it be any where found in the bowels of the earth, it belongeth in right of honour, as all men know, to the King: whence hath this custom grown but only from a natural persuasion, whereby men judge it decent for the highest persons always to be honoured with the choicest things? “If ye offer unto God the blind,” saith the Prophet Malachi, “it is not evil; if the lame and sick, it is good enough. Present it unto thy prince, and see if he will content himself, or accept thy person, saith the Lord of hosts.” When Abel presented God with an offering, it was the fattest of all the lambs in his whole flock; he honoured God not only out of his substance, but out of the very chiefest therein; whereby we may somewhat judge how religiously they stand affected towards God, who grudge that any thing worth the having should be his.

[4]Long it were to reckon up particularly what God was owner of under the Law: for of this sort was all which they spent in legal sacrifices; of this sort their usual oblations and offerings; of this sort tithes and firstfruits; of this sort that which by extraordinary occasions they vowed unto God; of this sort all that they gave to the building of the tabernacle; of this sort all that which was gathered amongst them for the erecting of the temple, and the adorning of it erected; of this sort whatsoever their Corban contained, wherein that blessed widow’s deodate was laid up. Now either this kind of honour was prefiguratively altogether ceremonial, and then  our Saviour accepteth it not; or if we find that to him also it hath been done, and that with divine approbation given for encouragement of the world, to shew by such kind of service their dutiful hearts towards Christ, there will be no place left for men to make any question at all whether herein they do well or no.

[5]Wherefore to descend from the synagogue unto the Church of Christ: albeit sacrifices, wherewith sometimes God was highly honoured, be not accepted as heretofore at the hands of men; yet forasmuch as “Honour God with thy riches” is an edict of the unseparable law of nature, so far forth as men are therein required by such kind of homage to testify their thankful minds, this sacrifice God doth accept still. Wherefore as it was said of Christ, that “all kings should worship him, and all nations do him service;” so this very kind of worship or service was likewise mentioned, lest we should think that our Lord and Saviour would allow of no such thing. “The kings of Tarshish and of the isles shall bring presents; the kings of Sheba and Seba shall bring gifts.” And as it maketh not a little to the praise of those sages mentioned in the Gospel, that the first amongst men which did solemnly honour our Saviour on earth were they; so it soundeth no less to the dignity of this particular kind, that the rest by it were prevented; “They fell down and worshipped him, and opened their treasures, and presented unto him gifts; gold, and incense, and myrrh.” Of all those things which were done to the honour of Christ in his lifetime there is not one whereof he spake in such sort, as when Mary to testify the largeness of her affection, seemed to waste away a gift upon him, the price of which gift might, as they thought who saw it, much better have been spent in works of mercy towards the poor: “Verily I say unto you, Wheresoever this Gospel shall be preached throughout all the world, there shall also this that she hath done be spoken of for memorial of her.”

[6]Of service to God, the best works are they which continue longest: and for permanency what like Donation, whereby things are unto him for ever dedicated? That the  ancient lands and livings of the Church were all in such sort given into the hands of God by the just lords and owners of them, that unto him they passed over their whole interest and right therein, the form of sundry the said donations as yet extant most plainly sheweth. And where time hath left no such evidence as now remaining to be seen, yet the same intention is presumed in all donors, unless the contrary be apparent. But to the end it may yet more plainly appear unto all men under what title the several kinds of ecclesiastical possessions are held, “Our Lord himself,” saith St. Augustine, “had coffers to keep those things which the faithful offered unto him. Then was the form of the church treasury first instituted, to the end that withal we might understand that in forbidding to be careful for tomorrow, his purpose was not to bar his saints from keeping money, but to withdraw them from doing God service for wealth’s sake, and from forsaking righteousness through fear of losing their wealth.” The first gifts consecrated unto Christ after his departure out of the world were sums of money, in process of time other moveables were added, and at length goods unmoveable, churches and oratories hallowed to the honour of his glorious name, houses and lands for perpetuity conveyed unto him, inheritance given to remain his as long as the world should endure. “The Apostles,” saith Melchiades, “they foresaw that God would have his Church amongst the Gentiles, and for that cause in Judea they took no lands but price of lands sold.”  This he conjectureth to have been the cause why the Apostles did that which the history reporteth of them. The truth is, that so the state of those times did require, as well otherwhere as in Judea. Wherefore when afterwards it did appear much more commodious for the Church to dedicate such inheritances, than the value and price of them being sold; the former custom was changed for this, as for the better. The devotion of Constantine herein all the world even till this very day admireth. They that lived in the prime of the Christian world thought no testament Christianly made, nor any thing therein well bequeathed, unless something were thereby added unto Christ’s patrimony.

[7]Touching which men, what judgment the world doth now give I know not; perhaps we deem them to have been herein but blind and superstitious persons. Nay, we in these cogitations are blind; they contrariwise did with Solomon plainly know and persuade themselves, that thus to diminish their wealth was not to diminish but to augment it, according to that which God doth promise to his own people by the Prophet Malachi, and which they by their own particular experience found true. If Wickliff therefore were of that  opinion which his adversaries ascribe unto him (whether truly or of purpose to make him odious I cannot tell, for in his writings I do not find it) namely, “That Constantine and others following his steps did evil, as having no sufficient ground whereby they might gather that such donations are acceptable to Jesus Christ;” it was in Wickliff a palpable error. I will use but one only argument to stand in the stead of many. Jacob taking his journey unto Haran made in this sort his solemn vow: “If God will be with me, and will keep me in this journey which I go, and will give me bread to eat, and clothes to put on, so that I come again to my father’s house in safety; then shall the Lord be my God, and this stone which I have set up a pillar shall be the house of God, and of all that thou shalt give me will I give the tenth unto thee.” May a Christian man desire as great things as Jacob did at the hands of God? may he desire them in as earnest manner? may he promise as great thankfulness  in acknowledging the goodness of God? may he vow any certain kind of public acknowledgment beforehand; or though he vow it not, perform it after in such sort that men may see he is persuaded how the Lord hath been his God? Are these particular kind of testifying thankfulness to God, the erecting of oratories, the dedicating of lands and goods to maintain them, forbidden any where? Let any mortal man living shew but one reason wherefore in this point to follow Jacob’s example should not be a thing both acceptable unto God, and in the eyes of the world for ever most highly commendable. Concerning goods of this nature, goods whereof when we speak we term them τὰ τῳ̑ Θεῳ̑ ἀϕιερωθέντα, the goods that are consecrated unto God, and as Tertullian speaketh, deposita pietatis, things which piety and devotion hath laid up as it were in the bosom of God; touching such goods, the law civil following mere light of nature defineth them to be no man’s, because no mortal man, or community of men, hath right of propriety in them.

\section*{That ecclesiastical persons are receivers of God’s rents; and that the honour of Prelates is, to be thereof his chief receivers; not without liberty from him granted, of converting the same unto their own use, even in large manner.}

XXIII. Persons ecclesiastical are God’s stewards, not only for that he hath set them over his family, as the ministers of ghostly food, but even for this very cause also, that they are to receive and dispose his temporal revenues, the gifts and oblations which men bring him. Of the Jews it is plain that their tithes they offered unto the Lord, and those offerings the Lord bestowed upon the Levites. When the Levites gave the tenth of their tithes, this their gift the Law doth term the Lord’s heave-offering, and appoint that the high-priest should receive the same. Of spoils taken in war, that part which they were accustomed to separate unto God, they brought it before the priest of the Lord, by whom it was laid up in the tabernacle of the congregation, for a memorial of their thankfulness towards God, and his goodness towards them in fighting for them against their enemies. As therefore the Apostle magnifieth the honour of Melchisedec, in that he being an high-priest, did receive at the hands of Abraham the tithes which Abraham did honour God with; so it argueth in the Apostles themselves great honour, that at their feet  the price of those possessions was laid, which men thought good to bestow on Christ. St. Paul commending the churches which were in Macedonia for their exceeding liberality this way, saith of them that he himself would bear record, they had declared their forward minds according to their power, yea, beyond their power, and had so much exceeded his expectation of them, that “they seemed as it were even to give away themselves first to the Lord,” saith the Apostle, “and then by the will of God unto us:” to him, as the owner of such gifts; to us, as his appointed receivers and dispensers. The gift of the Church of Antioch, bestowed unto the use of distressed brethren which were in Judea, Paul and Barnabas did deliver unto the presbyters of Jerusalem; and the head of those presbyters was James, he therefore the chiefest disposer thereof. Amongst those canons which are entitled Apostolical, one is this, “We appoint that the Bishop have care of those things which belong to the Church;” the meaning is, of church goods, as the reason following sheweth: “For if the precious souls of men must be committed unto him of trust, much more it behoveth the charge of money to be given him, that by his authority the presbyters and deacons may administer all things to them that stand in need.” So that he which hath done them the honour to be, as it were, his treasurers, hath left them also authority and power to use these treasures, both otherwise, and for the maintenance even of their own estate: the lower sort of the clergy according unto a meaner, the higher after a larger proportion.

[2]The use of spiritual goods and possessions hath been a matter much disputed of; grievous complaints there are usually made against the evil and unlawful usage of them, but  with no certain determination hitherto, on what things and persons, with what proportion and measure they being bestowed, do retain their lawful use. Some men condemn it as idle, superfluous, and altogether vain, that any part of the treasure of God should be spent upon costly ornaments appertaining unto his service: who being best worshipped, when he is served in spirit and truth, hath not for want of pomp and magnificence rejected at any time those who with faithful hearts have adored him. Whereupon the heretics, termed Henriciani and Petrobrusiani, threw down temples and houses of prayer erected with marvellous great charge, as being in that respect not fit for Christ by us to be honoured in.

[3]We deny not, but that they who sometime wandered as pilgrims on earth, and had no temples, but made caves and dens to pray in, did God such honour as was most acceptable in his sight: God did not reject them for their poverty and nakedness’ sake; their sacraments were not abhorred for want of vessels of gold.

Howbeit, let them who thus delight to plead, answer me: when Moses first, and afterwards David, exhorted the people of Israel unto matter of charge about the service of God; suppose we it had been allowable in them to have thus pleaded: “Our fathers in Egypt served God devoutly, God was with them in all their afflictions, he heard their prayers, pitied their case, and delivered them from the tyranny of their oppressors; what house, tabernacle, or temple had they?” Such argumentations are childish and fond; God doth not refuse to be honoured at all where there lacketh wealth; but where abundance and store is, he there requireth the flower thereof, being bestowed on him, to be employed even unto the ornament of his service. In Egypt the state of his people was servitude, and therefore his service was accordingly. In the desert they had no sooner aught of their own, but a tabernacle is required; and in the land of Canaan a temple. In the eyes of David it seemed a thing not fit, a thing not decent, that himself should be more richly seated than God.

[4]But concerning the use of ecclesiastical goods bestowed  this way, there is not so much contention amongst us, as what measure of allowance is fit for ecclesiastical persons to be maintained with. A better rule in this case to judge things by we cannot possibly have than the wisdom of God himself: by considering what he thought meet for each degree of the clergy to enjoy in time of the Law, what for Levites, what for priests, and what for high priests, somewhat we shall be the more able to discern rightly what may be fit, convenient, and right for the Christian clergy likewise. Priests for their maintenance had those first fruits of cattle, corn, wine, oil, and other commodities of the earth, which the Jews were accustomed yearly to present God with. They had the price which was appointed for men to pay in lieu of the first-born of their children, and the price of the first-born also amongst cattle which were unclean: they had the vowed gifts of the people, or the prices, if they were redeemable by the donors after vow, as some things were: they had the free and unvowed oblations of men: they had the remainder of things sacrificed: with tithes the Levites were maintained; and with the tithe of their tithes the high-priest. In a word, if the quality of that which God did assign to his clergy be considered, and their manner of receiving it without labour, expense, or charge, it will appear that the tribe of Levi, being but the twelfth part of Israel, had in effect as good as four twelfth parts of all such goods as the holy land did yield: so that their worldly estate was four times as good as any other tribe’s in Israel besides. But the high-priests’ condition, how ample! to whom belonged the tenth of all the tithe of this land, especially the law providing also, that as the people did bring the best of all things unto the priests and Levites, so the Levites should deliver the choice and flower of all their commodities to the high-priest, and so his tenth part by that mean be made the very best part amongst ten: by which proportion, if the Levites were ordinarily in all not above thirty thousand men, (whereas when David numbered them10, he found almost thirty-eight thousand above the age of thirty years,) the high-priest, after this  very reckoning, had as much as three or four thousand others of the clergy to live upon.

Over and besides all this, lest the priests of Egypt, holding lands, should seem in that respect better provided for than the priests of the true God, it pleased him further to appoint unto them forty and eight whole cities with territories of land adjoining, to hold as their own free inheritance for ever. For to the end they might have all kind of encouragement, not only to do what they ought, but to take pleasure in that they did; albeit they were expressly forbidden to have any part of the land of Canaan laid out whole to themselves, by themselves, in such sort as the rest of the tribes had; forasmuch as the will of God was rather that they should throughout all tribes be dispersed, for the easier access of the people unto knowledge; yet were they not barred altogether to hold a land [hold land?]i, nor yet otherwise the worse provided for, in respect of that former restraint; for God by way of special preeminence undertook to feed them at his own table, and out of his own proper treasury to maintain them, that want and penury they might never feel, except God himself did first receive injury.

[5]A thing most worthy our consideration is the wisdom of God herein; for the common sort being prone unto envy and murmur, little considereth of what necessity, use and importance the sacred duties of the clergy are, and for that cause hardly yieldeth them any such honour without repining and grudging thereat; they cannot brook it, that when they have laboured and come to reap, there should so great a portion go out of the fruit of their labours, and be yielded up unto such as sweat not for it. But when the Lord doth challenge this as his own due, and require it to be done by way of homage unto him, whose mere liberality and goodness had raised them from a poor and servile estate, to place them where they had all those ample and rich possessions; they must be worse than brute beasts if they would storm at any thing which he did receive at their hands. And for him to bestow his own on his own servants (which liberty is not denied unto the meanest of men), what man liveth that can think it other than most  reasonable? Wherefore no cause there was, why that which the clergy had should in any man’s eye seem too much, unless God himself were thought to be of an over-having disposition. 1This is the mark whereat all those speeches drive, “Levi hath no part nor inheritance with his brethren, the Lord is his inheritance;” again, “2To the tribe of Levi he gave no inheritance, the sacrifices of the Lord God of Israel an inheritance of Levi;” again, “3The tithes of the which they shall offer as an offering unto the Lord, I have given the Levites for an inheritance;” and again, “4All the heave offerings of the holy things which the children of Israel shall offer unto the Lord, I have given thee, and thy sons and thy daughters with thee, to be a duty for ever; it is a perpetual covenant of salt before the Lord.”

[6]Now that if such provision be possible to be made, the Christian clergy ought not herein to be inferior unto the Jewish, what sounder proof than the Apostle’s own kind of argument? “5Do ye not know that they which minister about the holy things eat of the things of the temple? and they which wait at the altar are partakers with the altar? so, even so, hath the Lord ordained that they which preach the gospel should live of the gospel.” Upon which words I thus conclude, that if the people of God do abound, and abounding can so far forth find in their hearts to shew themselves towards Christ their Saviour thankful as to honour him with their riches (which no law of God or nature forbiddeth) no less than the ancient Jewish people did honour God; the plain ordinance of Christ appointeth as large and as ample proportion out of his own treasure unto them that serve him in the gospel as ever the priests of the law did enjoy. What further proof can we desire? It is the blessed Apostle’s testimony, That “even so the Lord hath ordained.” Yea, I know not whether it be sound to interpret the Apostle otherwise than that, whereas he judgeth the presbyters “which rule well in the Church of Christ to be worthy of double honour,” he means double unto that which the priests of the law received; “7For  if that ministry which was of the letter were so glorious, how shall not the ministry of the spirit be more glorious?” If the teachers of the Law of Moses, which God delivered written with letters in tables of stone, were thought worthy of so great honour, how shall not the teachers of the gospel of Christ be in his sight most worthy, the Holy Ghost being sent from heaven to engrave the gospel on their hearts who first taught it, and whose successors they that teach it at this day are? So that according to the ordinance of God himself, their estate for worldly maintenance ought to be no worse than is granted unto other sorts of men, each according to that degree they were placed in.

[7]Neither are we so to judge of their worldly condition, as if they were servants of men, and at men’s hands did receive those earthly benefits by way of stipend in lieu of pains whereunto they are hired; nay, that which is paid unto them is homage and tribute due unto the Lord Christ. His servants they are, and from him they receive such goods by way of stipend. Not so from men: for at the hands of men, he himself being honoured with such things, hath appointed his servants therewith according to their several degrees and places to be maintained. And for their greater encouragement who are his labourers he hath to their comfort assured them for ever, that they are in his estimation “worthy the hire” which he alloweth them; and therefore if men should withdraw from him the store which those his servants that labour in his work are maintained with, yet he in his word shall be found everlastingly true, their labour in the Lord shall not be forgotten; the hire he accounteth them worthy of, they shall surely have either one way or other answered.

[8]In the prime of the Christian world, that which was  brought and laid down at the Apostles’ feet, they disposed of by distribution according to the exigence of each man’s need. Neither can we think that they who out of Christ’s treasury made provision for all others, were careless to furnish the clergy with all things fit and convenient for their estate: and as themselves were chiefest in place of authority and calling, so no man doubteth but that proportionably they had power to use the same for their own decent maintenance. The Apostles with the rest of the clergy in Jerusalem lived at that time according to the manner of a fellowship or collegiate society, maintaining themselves and the poor of the Church with a common purse, the rest of the faithful keeping that purse continually stored. And in that sense it is that the sacred history saith, “All which believed were in one place, and had all things common.” In the histories of the Church, and in the writings of the ancient Fathers for some hundreds of years after, we find no other way for the maintenance of the clergy but only this, the treasury of Jesus Christ furnished through men’s devotion, bestowing sometimes goods, sometimes lands that way, and out of his treasury the charge of the service of God was defrayed, the bishop and the clergy under him maintained, the poor in their necessity ministered unto. For which purpose, every bishop had some one of the presbyters under him to be 3treasurer of the church, to receive, keep, and deliver all; which office in churches cathedral remaineth even till this day, albeit the use thereof be not altogether so large now as heretofore.

[9]The disposition of these goods was by the appointment of the bishop. Wherefore Prosper 4speaking of the bishop’s care herein saith, “It was necessary for one to be troubled therewith, to the end that the rest under him  might be the freer to attend quietly their spiritual businesses.” And lest any man should imagine that bishops by this means were hindered themselves from attending the service of God, “Even herein,” saith he, “they do God service; for if those things which are bestowed on the Church be God’s, he doth the work of God, who not of a covetous mind, but with purpose of most faithful administration, taketh care of things consecrated unto God.”

And forasmuch as the presbyters of every church could not all live with the bishop, partly for that their number was great, and partly because the people being once divided into parishes, such presbyters as had severally charge of them were by that mean more conveniently to live in the midst each of his own particular flock, therefore a competent number being fed at the same table with the bishop, the rest had their whole allowance apart, which several allowances were called sportulæ, and they who received them, sportulantes fratres.

Touching the bishop, as his place and estate was higher, so likewise the proportion of his charges about himself being for that cause in all equity and reason greater, yet forasmuch as his stint herein was no other than it pleased himself to set, the rest (as the manner of inferiors is to think that they which are over them always have too much) grudged many times at the measure of the bishop’s private expense, perhaps not without cause. Howsoever, by this occasion there grew amongst them great heart-burning, quarrel and strife: where the bishops were found culpable, as eating too much beyond their tether, and drawing more to their own private maintenance than the proportion of Christ’s patrimony being not greatly abundant could bear, sundry constitutions hereupon were made to moderate the same, according to the Church’s condition in  those times. Some before they were made bishops having been owners of ample possessions, sold them and gave them away to the poor: thus did Paulinus, Hilary, Cyprian, and sundry others. Hereupon they who entering into the same spiritual and high function held their secular possessions still were hardly thought of: and even when the case was fully resolved, that so to do was not unlawful, yet it grew a question, “whether they lawfully might then take any thing out of the public treasury of Christ:” a question, “whether bishops, holding by civil title sufficient to live of their own, were bound in conscience to leave the goods of the Church altogether to the use of others.” Of contentions about these matters there was no end, neither appeared there any possible way for quietness, otherwise than by making partition of church-revenues, according to the several ends and uses for which they did serve, that so the bishop’s part might be certain. Such partition being made, the bishop enjoyed  his portion several to himself; the rest of the clergy likewise theirs; a third part was severed to the furnishing and upholding of the church; a fourth to the erection and maintenance of houses wherein the poor might have relief. After which separation made, lands and livings began every day to be dedicated unto each use severally, by means whereof every of them became in short time much greater than they had been for worldly maintenance, the fervent devotion of men being glad that this new opportunity was given of shewing zeal to the house of God in more certain order.

[10]By these things it plainly appeareth what proportion of maintenance hath been ever thought reasonable for a bishop; sith in that very partition agreed on to bring him unto his certain stint, as much is allowed unto him alone as unto all the clergy under him, namely, a fourth part of the whole yearly rents and revenues of the church. Nor is it likely, that before those temporalities which now are such eyesores were added unto the honour of bishops, their state was so mean as some imagine: for if we had no other evidence than the covetous and ambitious humour of heretics, whose impotent desires of aspiring thereunto, and extreme discontentment as oft as they were defeated, even this doth shew that the state of bishops was not a few degrees advanced above the rest. Wherefore of grand apostates which were in the very prime of the primitive Church, thus Lactantius above thirteen hundred years sithence testified, “Men of a slippery faith  they were, who feigning that they knew and worshipped God, but seeking only that they might grow in wealth and honour, affected the place of the highest priesthood; whereunto when their betters were chosen before them, they thought it better to leave the Church, and to draw their favourers with them, than to endure those men their governors, whom themselves desired to govern.”

[11]Now whereas against the present estate of bishops, and the greatness of their port, and the largeness of their expenses at this day, there is not any thing more commonly objected than those ancient canons, whereby they are restrained unto a far more sparing life, their houses, their retinue, their diet limited within a far more narrow compass than is now kept; we must know, that those laws and orders were made when bishops lived of the same purse which served as well for a number of others as them, and yet all at their disposing. So that convenient it was to provide that there might be a moderate stint appointed to measure their expenses by, lest others should be injured by their wastefulness. Contrariwise there is now no cause wherefore any such law should be urged, when bishops live only of that which hath been peculiarly allotted unto them. They having therefore temporalities and other revenues to bestow for their own private use, according to that which their state requireth, and no  other having with them any such common interest therein, their own discretion is to be their law for this matter; neither are they to be pressed with the rigour of such ancient canons as were framed for other times, much less so odiously to be upbraided with unconformity unto the pattern of our Lord and Saviour’s estate, in such circumstances as himself did never mind to require that the rest of the world should of necessity be like him. Thus against the wealth of the clergy they allege how meanly Christ himself was provided for; against bishops’ palaces, his want of a hole to hide his head in; against the service done unto them, that “he came to minister, not to be ministered unto in the world.” Which things, as they are not unfit to control covetous, proud or ambitious desires of the ministers of Christ, and even of all Christians, whatsoever they be; and to teach men contentment of mind, how mean soever their estate is, considering that they are but servants to him, whose condition was far more abased than theirs is, or can be; so to prove such difference in state between us and him unlawful, they are of no force or strength at all. If one convented before their consistories, when he standeth to make his answer, should break out into invectives against their authority, and tell them that Christ, when he was on earth, did not sit to judge, but stand to be judged; would they hereupon think it requisite  to dissolve their eldership, and to permit no tribunals, no judges at all, for fear of swerving from our Saviour’s example? If those men, who have nothing in their mouths more usual than the Poverty of Jesus Christ and his Apostles, allege not this as Julian sometime did Beati pauperes unto Christians, when his meaning was to spoil them of that they had; our hope is then, that as they seriously and sincerely wish that our Saviour Christ in this point may be followed, and to that end only propose his blessed example; so at our hands again they will be content to hear with like willingness the holy Apostle’s exhortation made unto them of the laity also, “Be ye followers of us, even as we are of Christ; let us be your example, even as the Lord Jesus Christ is ours, that we may all proceed by one and the same rule.”

\section*{That for their unworthiness to deprive both them and their successors of such goods, and to convey the same unto men of secular calling, were extreme sacrilegious injustice.}

XXIV. But beware we of following Christ as thieves follow true men, to take their goods by violence from them. Be it that bishops were all unworthy, not only of living, but even of life, yet what hath our Lord Jesus Christ deserved, for which men should judge him worthy to have the things that are his given away from him unto others that have no right unto them? For at this mark it is that the head lay-reformers do all aim. Must these unworthy prelates give place? What then? Shall better succeed in their rooms? Is this desired, to the end that others may enjoy their honours, which shall do Christ more faithful service than they have done? Bishops are the worst men living upon earth; therefore let their sanctified possessions be divided: amongst whom? O blessed reformation! O happy men, that put to their helping hands for the furtherance of so good and glorious a work!

[2]Wherefore albeit the whole world at this day do already perceive, and posterity be like hereafter a great deal more plainly to discern, not that the clergy of God is thus heaved at because they are wicked, but that means are used to put it into the heads of the simple multitude that they are such indeed, to the end that those who thirst for the spoil of spiritual possessions may till such time as they have their  purpose be thought to covet nothing but only the just extinguishment of unreformable persons; so that in regard of such men’s intentions, practices, and machinations against them, the part that suffereth these things may most fitly pray with David, “Judge thou me, O Lord, according to my righteousness, and according unto mine innocency: O let the malice of the wicked come to an end, and be thou the guide of the just:” notwithstanding, forasmuch as it doth not stand with Christian humility otherwise to think, than that this violent outrage of men is a rod in the ireful hands of the Lord our God, the smart whereof we deserve to feel; let it not seem grievous in the eyes of my reverend lords the Bishops, if to their good consideration I offer a view of those sores which are in the kind of their heavenly function most apt to breed, and which being not in time cured, may procure at the length that which God of his infinite mercy avert.

[3]Of bishops in his time St. Jerome complaineth, that they took it in great disdain to have any fault great or small found with them. Epiphanius likewise before Jerome noteth their impatiency this way to have been the very cause of a schism in the Church of Christ; at what time one Audius, a man of great integrity of life, full of faith and zeal towards God, beholding those things which were corruptly done in the Church, told the bishops and presbyters their faults in such sort as those men are wont, who love the truth from their  hearts, and walk in the paths of a most exact life. Whether it were covetousness or sensuality in their lives, absurdity or error in their teaching; any breach of the laws and canons of the Church wherein he espied them faulty, certain and sure they were to be thereof most plainly told. Which thing they whose dealings were justly culpable could not bear; but instead of amending their faults bent their hatred against him who sought their amendment, till at length they drove him by extremity of infestation, through weariness of striving against their injuries, to leave both them and with them the Church.

Amongst the manifold accusations, either generally intended against the bishops of this our Church, or laid particularly to the charge of any of them, I cannot find that hitherto their spitefullest adversaries have been able to say justly, that any man for telling them their personal faults in good and Christian sort hath sustained in that respect much persecution. Wherefore notwithstanding mine own inferior estate and calling in God’s Church, the consideration whereof assureth me, that in this kind the sweetest sacrifice which I can offer unto Christ is meek obedience, reverence and awe unto the prelates which he hath placed in seats of higher authority over me, emboldened I am, so far as may conveniently stand with that duty of humble subjection, meekly to crave, my good lords, your favourable pardon, if it shall seem a fault thus far to presume; or if otherwise, your wonted courteous acceptation.

—“Sine me hæc haud mollia fatu
“Sublatis aperire dolis.”
Æneid. lib. xii. 

[4]First, In government, be it of what kind soever, but especially if it be such kind of government as prelates have over the Church, there is not one thing publicly more hurtful than that an hard opinion should be conceived of governors at the first: and a good opinion how should the world ever conceive of them for their after-proceedings in regiment, whose first access and entrance thereunto giveth just occasion to think them corrupt men, which fear not that God in whose name they are to rule? Wherefore a scandalous thing it is to the Church of God, and to the actors themselves dangerous, to have aspired unto rooms of prelacy by wicked means. We are not at this day troubled much with that tumultuous kind  of ambition wherewith the elections of Damasus in St. Jerome’s age, and of Maximus in Gregory’s time, and of others, were long sithence stained. Our greatest fear is rather the evil which Leo and Anthemius did by imperial constitution endeavour as much as in them lay to prevent. He which granteth, or he which receiveth the office and dignity of a bishop, otherwise than beseemeth a thing divine and most holy; he which bestoweth, and he which obtaineth it after any other sort than were honest and lawful to use, if our Lord Jesus Christ were present himself on earth to bestow it even with his own hands, sinneth a sin by so much more grievous than the sin of Belshazzar, by how much offices and functions heavenly are more precious than the meanest ornaments or implements which thereunto appertain. If it be as the Apostle saith, that the Holy Ghost doth make bishops, and that the whole action of making them is God’s own deed, men being therein but his agents; what spark of the fear of God can there possibly remain in their hearts, who representing the person of God in naming worthy men to ecclesiastical charge, do sell that which in his name they are to bestow; or who standing as it were at the throne of the living God do bargain for that which at his hands they are to receive? Woe worth such impious and irreligious profanations! The Church of Christ hath been hereby made, not “a den of thieves,” but in a manner the very dwelling-place of foul spirits; for undoubtedly such a  number of them have been in all ages who thus have climbed into seat of episcopal regiment.

[5]Secondly, Men may by orderly means be invested with spiritual authority and yet do harm by reason of ignorance how to use it to the good of the Church. “It is,” saith Chrysostom, “πολλου̑ μὲν ἀξιώματος, δύσκολον δὲ, ἐπισκοπει̑ν; a thing highly to be accounted of, but an hard thing, to be that which a bishop should be.” Yea a hard and a toilsome thing it is for a bishop to know the things that belong unto a bishop. A right good man may be a very unfit magistrate. And for discharge of a bishop’s office, to be well-minded is not enough, no not to be well learned also. Skill to instruct is a thing necessary, skill to govern much more necessary in a bishop. It is not safe for the Church of Christ, when bishops learn what belongeth unto government, as empirics learn physic by killing of the sick. Bishops were wont to be men of great learning in the laws both civil and of the Church; and while they were so, the wisest men in the land for counsel and government were bishops.

[6]Thirdly, Know we never so well what belongeth unto a charge of so great moment, yet can we not therein proceed but with hazard of public detriment, if we rely on ourselves alone, and use not the benefit of conference with others. A singular mean to unity and concord amongst themselves, a marvellous help unto uniformity in their dealings, no small addition of weight and credit unto that which they do, a strong bridle unto such as watch for occasions to stir against them, finally, a very great stay unto all that are under their government, it could not choose but be soon found, if bishops did often and seriously use the help of mutual consultation.

[7]These three rehearsed are things only preparatory unto the course of episcopal proceedings. But the hurt is more manifestly seen which doth grow to the Church of God by faults inherent in their several actions, as when they carelessly ordain, when they institute negligently, when corruptly they bestow church-livings, benefices, prebends, and rooms especially of jurisdiction, when they visit for gain’s sake rather than with serious intent to do good, when their courts erected for the maintenance of good order, are disordered, when they regard not the clergy under them, when neither clergy nor laity are kept in that awe for which this authority should serve,  when any thing appeareth in them rather than a fatherly affection towards the flock of Christ, when they have no respect to posterity, and finally when they neglect the true and requisite means whereby their authority should be upheld. Surely the hurt which groweth out of these defects must needs be exceeding great. In a minister, ignorance and disability to teach is a maim; nor is it held a thing allowable to ordain such, were it not for the avoiding of a greater evil which the church must needs sustain, if in so great scarcity of able men, and unsufficiency of most parishes throughout the land to maintain them, both public prayer and the administration of sacraments should rather want, than any man thereunto be admitted lacking dexterity and skill to perform that which otherwise was most requisite. Wherefore the necessity of ordaining such is no excuse for the rash and careless ordaining of every one that hath but a friend to bestow some two or three words of ordinary commendation in his behalf. By reason whereof the Church groweth burdened with silly creatures more than need, whose noted baseness and insufficiency bringeth their very order itself into contempt.

It may be that the fear of a Quare impedit doth cause institutions to pass more easily than otherwise they would.  And to speak plainly the very truth, it may be that writs of Quare non impedit were for these times most necessary in the other’s place: yet where law will not suffer men to follow their own judgment, to shew their judgment they are not hindered. And I doubt not but that even conscienceless and wicked patrons, of which sort the swarms are too great in the church of England, are the more emboldened to present unto bishops any refuse, by finding so easy acceptation thereof. Somewhat they might redress this sore, notwithstanding so strong impediments, if it did plainly appear that they took it indeed to heart, and were not in a manner contented with it.

[8]Shall we look for care in admitting whom others present, if that which some of yourselves confer be at any time corruptly bestowed? A foul and an ugly kind of deformity it hath, if a man do but think what it is for a bishop to draw commodity and gain from those things whereof he is left a free bestower, and that in trust, without any other obligation than his sacred order only, and that religious integrity which hath been presumed on in him. Simoniacal corruption I may not for honour’s sake suspect to be amongst men of so great place. So often they do not I trust offend by sale, as by unadvised gift of such preferments, wherein that ancient canon should specially be remembered, which forbiddeth a bishop to be led by human affection in bestowing the things of God. A fault no where so hurtful, as in bestowing places of jurisdiction, and in furnishing cathedral churches, the prebendaries and other dignities whereof are the very true successors of those ancient presbyters which  were at the first as counsellors unto bishops. A foul abuse it is, that any one man should be loaded as some are with livings in this kind, yea some even of them who condemn utterly the granting of any two benefices unto the same man, whereas the other is in truth a matter of far greater sequel, as experience would soon shew, if churches cathedral being furnished with the residence of a competent number of virtuous, grave, wise and learned divines, the rest of the prebends of every such church were given within the diocess unto men of worthiest desert, for their better encouragement unto industry and travel; unless it seem also convenient to extend the benefit of them unto the learned in universities, and men of special employment otherwise in the affairs of the Church of God. But howsoever, surely with the public good of the Church it will hardly stand, that in any one person such favours be more multiplied than law permitteth in those livings which are with cure.

[9]Touching bishops’ visitations, the first institution of them was profitable, to the end that the state and condition of churches being known, there might be for evils growing convenient remedies provided in due time. The observation of church laws, the correction of faults in the service of God and manners of men, these are things that visitors should seek. When these things are inquired of formally, and but for custom’s sake, fees and pensions being the only thing which is sought, and little else done by visitations; we are not to marvel if the baseness of the end doth make the action itself loathsome. The good which bishops may do not only by these visitations belonging ordinarily to their office, but also in respect of that power which the founders of colleges have given them of special trust, charging even fearfully their consciences therewith: the good, I say, which they might do by this their authority, both within their own diocess, and in the well-springs themselves, the universities, is plainly such as cannot choose but add weight to their heavy accounts in that dreadful day if they do it not.

[10]In their courts, where nothing but singular integrity and justice should prevail, if palpable and gross corruptions be found, by reason of offices so often granted unto men who seek nothing but their own gain, and make no account what disgrace doth grow by their unjust dealings unto them under  whom they deal, the evil hereof shall work more than they which procure it do perhaps imagine.

[11]At the hands of a bishop the first thing looked for is a care of the clergy under him, a care that in doing good they may have whatsoever comforts and encouragements his countenance, authority and place may yield. Otherwise what heart shall they have to proceed in their painful course, all sorts of men besides being so ready to malign, despise and every way oppress them? Let them find nothing but disdain in bishops; in the enemies of present government, if that way they list to betake themselves, all kind of favourable and friendly helps; unto which part think we it likely that men having wit, courage and stomach, will incline?

As great a fault is the want of severity when need requireth, as of kindness and courtesy in bishops. But touching this, what with ill usage of their power amongst the meaner, and what with disusage amongst the higher sort, they are in the eyes of both sorts as bees that have lost their sting. It is a long time sithence any great one hath felt, or almost any one much feared the edge of that ecclesiastical severity, which sometime held lords and dukes in a more religious awe than now the meanest are able to be kept.

[12]A bishop, in whom there did plainly appear the marks and tokens of a fatherly affection towards them that are under his charge, what good might he do ten thousand ways more than any man knows how to set down? But the souls of men are not loved, that which Christ shed his blood for is not esteemed precious. This is the very root, the fountain of all negligence in church-government.

[13]Most wretched are the terms of men’s estate when once they are at a point of wretchedness so extreme, that they bend not their wits any further than only to shift out the present time, never regarding what shall become of their successors after them. Had our predecessors so loosely cast off from them all care and respect to posterity, a Church Christian there had not been about the regiment whereof we should need at this day to strive. It was the barbarous affection of Nero, that the ruin of his own imperial seat he could have been well enough contented to see, in case he  might also have seen it accompanied with the fall of the whole world: an affection not more intolerable than theirs, who care not to overthrow all posterity, so they may purchase a few days of ignominious safety unto themselves and their present estates; if it may be termed a safety which tendeth so fast unto their very overthrow that are the purchasers of it in so vile and base manner. Men whom it standeth upon to uphold a reverend estimation of themselves in the minds of others, without which the very best things they do are hardly able to escape disgrace, must before it be over late remember how much easier it is to retain credit once gotten, than to recover it being lost. The executors of bishops are sued if their mansion-house be suffered to go to decay: but whom shall their successors sue for the dilapidations which they make of that credit, the unrepaired diminutions whereof will in time bring to pass, that they which would most do good in that calling shall not be able, by reason of prejudice generally settled in the minds of all sorts against them?

[14]By what means their estimation hath hitherto decayed, it is no hard thing to discern. Herod and Archelaus are noted to have sought out purposely the dullest and most ignoble that could be found amongst the people, preferring such to the high priest’s office, thereby to abate the great opinion which the multitude had of that order, and to procure a more expedite course for their own wicked counsels, whereunto they saw the high priests were no small impediment, as long as the common sort did much depend upon them. It may be there hath been partly some show and just suspicion of like practice in some, in procuring the undeserved preferments of some unworthy persons, the very cause of  whose advancement hath been principally their unworthiness to be advanced. But neither could this be done altogether without the inexcusable fault of some preferred before, and so oft we cannot imagine it to have been done, that either only or chiefly from thence this decay of their estimation may be thought to grow. Somewhat it is that the malice of their cunning adversaries, but much more which themselves have effected against themselves.

[15]A bishop’s estimation doth grow from the excellency of virtues suitable unto his place. Unto the place of a bishop those high divine virtues are judged suitable, which virtues being not easily found in other sorts of great men, do make him appear so much the greater in whom they are found. Devotion and the feeling sense of religion are not usual in the noblest, wisest, and chiefest personages of state, by reason their wits are so much employed another way, and their minds so seldom conversant in heavenly things. If therefore wherein themselves are defective they see that bishops do blessedly excel, it frameth secretly their hearts to a stooping kind of disposition, clean opposite to contempt. The very countenance of Moses was glorious after that God had conferred with him. And where bishops are, the powers and faculties of whose souls God hath possessed, those very actions, the kind whereof is common unto them with other men, have notwithstanding in them a more high and heavenly form, which draweth correspondent estimation unto it, by virtue of that celestial impression, which deep meditation of holy things, and as it were conversation with God doth leave in their minds. So that bishops which will be esteemed of as they ought, must frame themselves to that very pattern from whence those Asian bishops unto whom St. John writeth were denominated, even so far forth as this our frailty will permit; shine they must as angels of God in the midst of perverse men. They are not to look that the world should always carry the affection of Constantine, to bury that  which might derogate from them, and to cover their imbecilities. More than high time it is that they bethink themselves of the Apostle’s admonition, Attende tibi, “Have a vigilant eye to thyself.” They err if they do not persuade themselves that wheresoever they walk or sit, be it in their churches or in their consistories, abroad and at home, at their tables or in their closets, they are in the midst of snares laid for them. Wherefore as they are with the prophet every one of them to make it their hourly prayer unto God, “Lead me O Lord in thy righteousness, because of enemies;” so it is not safe for them, no not for a moment, to slacken their industry in seeking every way that estimation which may further their labours unto the Church’s good. Absurdity, though but in words, must needs be this way a maim, where nothing but wisdom, gravity and judgment is looked for. That which the son of Sirach hath concerning the writings of the old sages, “Wise sentences are found in them,” should be the proper mark and character of bishops’ speeches, whose lips, as doors, are not to be opened, but for egress of instruction and sound knowledge. If base servility and dejection of mind be ever espied in them, how should men esteem them as worthy the rooms of the great ambassadors of God? A wretched desire to gain by bad and unseemly means standeth not with a mean man’s credit, much less with that reputation which Fathers of the Church should be in. But if besides all this there be also coldness in works of piety and charity, utter contempt even of learning itself, no care to further it by any such helps as they easily might and ought to afford, no not as much as that due respect unto their very families about them, which all men that are of account do order as near as they can in such sort that no grievous offensive deformity be therein noted; if there still continue in that most reverend order such as, by so many engines, work day and night to pull down the whole frame of their own  estimation amongst men, some of the rest secretly also permitting others their industrious opposites every day more and more to seduce the multitude; how should the Church of God hope for great good at their hands?

[16]What we have spoken concerning these things, let not malicious accusers think themselves therewith justified, no more than Shimei was by his sovereign’s most humble and meek acknowledgment even of that very crime which so impudent a caitiff’s tongue upbraided him withal; the one in the virulent rancour of a cankered affection, took that delight for the present, which in the end did turn to his own more tormenting woe; the other in the contrite patience even of deserved malediction had yet this comfort, “It may be the Lord will look on mine affliction, and do me good for his cursing this day.” As for us over whom Christ hath placed them to be the chiefest guides and pastors of our souls, our common fault is, that we look for much more in our governors than a tolerable sufficiency can yield, and bear much less than humanity and reason do require we should. Too much perfection over rigorously exacted in them, cannot but breed in us perpetual discontentment, and on both parts cause all things to be unpleasant. It is exceedingly worth the nothing, which Plato hath about the means whereby men fall into an utter dislike of all men with whom they converse: “This sourness of mind which maketh every man’s dealings unsavoury in our taste, entereth by an unskilful overweening, which at the first we have of one, and so of another, in whom we afterwards find ourselves to have been deceived, they declaring themselves in the end to be frail men, whom we judged demigods. When we have oftentimes been thus beguiled, and that far besides expectation, we grow at the length to this plain conclusion, that there is  nothing at all sound in any man. Which bitter conceit is unseemly, and plain to have risen from lack of mature judgment in human affairs; which if so be we did handle with art, we would not enter into dealings with men, otherwise than being beforehand grounded in this persuasion, that the number of persons notably good or bad is but very small; that the most part of good have some evil, and of evil men some good in them.” So true our experience doth find those aphorisms of Mercurius Trismegistus, Ἀδύνατον τὸ ἀγαθὸν ἐνθάδε καθαρεύειν τη̑ς κακίας, “to purge goodness quite and clean from all mixture of evil here is a thing impossible.” Again, Τὸ μὴ λίαν κακὸν ἐνθάδε τὸ ἀγαθόν ἐστι, “when in this world we term a thing good, we cannot by exact construction have any other true meaning, than that the said thing so termed is not noted to be a thing exceedingly evil.” And again, Μόνον, ὡ̑ Ἀσκλήπιε, τὸ ὄνομα του̑ ἀγαθου̑ ἐν ἀνθρώποις, τὸ δὲ ἔργον οὐδαμου̑, “Amongst men, O Æsculapius, the name of that which is good we find, but no where the very true thing itself.” When we censure the deeds and dealings of our superiors, to bring with us a fore-conceit thus qualified, shall be as well on our part as theirs a thing available unto quietness.

[17]But howsoever the case doth stand with men’s either good or bad quality, the verdict which our Lord and Saviour hath given, should continue for ever sure; “Quæ Dei sunt, Deo;” let men bear the burden of their own iniquity; as for those things which are God’s, let not God be deprived of them. For if only to withhold that which should be given be no better than to rob God, if to withdraw any mite of that which is but in propose [purpose?]k only bequeathed, though as yet undelivered into the sacred treasure of God, be a sin for which Ananias and Sapphira felt so heavily the dreadful hand of divine revenge; quite and clean to take that away which we never gave, and that after God hath for so many ages therewith been possessed, and that without any other shew of cause, saving only that it seemeth in their eyes who seek it to be too much for them which have it in their  hands, can we term it or think it less than most impious injustice, most heinous sacrilege? Such was the religious affection of Joseph, that it suffered him not to take that advantage, no not against the very idolatrous priests of Egypt, which he took for the purchasing of other men’s lands to the king; but he considered, that albeit their idolatry deserved hatred, yet for the honour’s sake due unto priesthood, better it was the king himself should yield them relief in public extremity, than permit that the same necessity should constrain also them to do as the rest of the people did.

[18]But it may be men have now found out, that God hath proposed the Christian clergy as a prey for all men freely to seize upon; that God hath left them as the fishes of the sea, which every man that listeth to gather into his net may; or that there is no God in heaven to pity them, and to regard the injuries which man doth lay upon them: yet the public good of this church and commonwealth doth, I hope, weigh somewhat in the hearts of all honestly disposed men. Unto the public good no one thing is more directly available, than that such as are in place, whether it be of civil or of ecclesiastical authority, be so much the more largely furnished even with external helps and ornaments of this life, [by?] how much the more highly they are in power and calling advanced above others. For nature is not contented with bare sufficiency unto the sustenance of man, but doth evermore covet a decency proportionable unto the place which man hath in the body or society of others. For according unto the greatness of men’s calling, the measure of all their actions doth grow in every man’s secret expectation, so that great men do always know that great things are at their hands expected. In a bishop great liberality, great hospitality, actions in every kind great are looked for: and for actions which must be great, mean instruments will not serve. Men are but men, what room soever amongst men they hold. If therefore the measure of their worldly abilities be beneath that proportion which their calling doth make to be looked for at their hands, a stronger inducement it is than perhaps men are aware of unto evil and corrupt dealings for supply of that defect. For which cause we must needs think it a thing necessary unto the  common good of the Church, that great jurisdiction being granted unto bishops over others, a state of wealth proportionable should likewise be provided for them. Where wealth is had in so great admiration, as generally in this golden age it is, that without it angelical perfections are not able to deliver from extreme contempt, surely to make bishops poorer than they are, were to make them of less account and estimation than they should be. Wherefore if detriment and dishonour do grow to religion, to God, to his Church, when the public account which is made of the chief of the clergy decayeth, how should it be but in this respect for the good of religion, of God, of his Church, that the wealth of bishops be carefully preserved from further diminution?

The travels and crosses wherewith prelacy is never unaccompanied, they which feel them know how heavy and how great they are. Unless such difficulties therefore annexed unto that estate be tempered by co-annexing thereunto things esteemed of in this world, how should we hope that the minds of men, shunning naturally the burdens of each function, will be drawn to undertake the burden of episcopal care and labour in the Church of Christ? Wherefore if long we desire to enjoy the peace, quietness, order and stability of religion, which prelacy (as hath been declared) causeth, then must we necessarily, even in favour of the public good, uphold those things, the hope whereof being taken away, it is not the mere goodness of the charge, and the divine acceptation thereof, that will be able to invite many thereunto.

[19]What shall become of that commonwealth or church in the end, which hath not the eye of learning to beautify, guide and direct it? At the length what shall become of that learning, which hath not wherewith any more to encourage her industrious followers? And finally, what shall become of that courage to follow learning, which hath already so much failed through the only diminution of her chiefest rewards, bishoprics? Surely wheresoever this wicked intendment of overthrowing cathedral churches, or of taking away those livings, lands and possessions which bishops hitherto have enjoyed, shall once prevail, the handmaids attending thereupon will be paganism and extreme barbarity.

[20]In the Law of Moses, how careful provision is made  that goods of this kind might remain to the Church for ever: “Ye shall not make common the holy things of the children of Israel, lest ye die, saith the Lord.” Touching the fields annexed unto Levitical cities, the law was plain, they might not be sold; and the reason of the law, this, “for it was their possession for ever:” He which was Lord and owner of it, his will and pleasure was, that from the Levites it should never pass to be enjoyed by any other. The Lord’s own portion, without his own commission and grant, how should any man justly hold? They which hold it by his appointment had it plainly with this condition, “They shall not sell of it, neither change it, nor alienate the first-fruits of the land; for it is holy unto the Lord.” It falleth sometimes out, as the prophet Habakkuk noteth, that the very “prey of savage beasts becometh dreadful unto themselves.” It did so in Judas, Achan, Nebuchadnezzar; their evil-purchased goods were their snare, and their prey their own terror; a thing no where so likely to follow, as in those goods and possessions, which being laid where they should not rest, have by the Lord’s own testimony his most bitter curse their undividable companion.

[21]These persuasions we use for other men’s cause, not for theirs with whom God and religion are parts of the abrogated law of ceremonies. Wherefore not to continue longer in the cure of a sore desperate, there was a time when the clergy had almost as little as these good people wish. But the kings of this realm and others whom God had blest, considered devoutly with themselves, as David in like case sometimes had done, “Is it meet that we at the hands of God should enjoy all kinds of abundance, and God’s clergy suffer want?” They considered that of Solomon, “6Honour God with thy substance, and the chiefest of all thy revenue; so shall thy barns be filled with corn, and thy vessels shall run over with new wine.” They considered how the care which Jehosaphat had, in providing that the Levites might have encouragement to do the work of the Lord cheerfully, was left of God as a fit pattern to be followed in the Church for  ever. They considered what promise our Lord and Saviour had made unto them, at whose hands his prophets should receive but the least part of the meanest kind of friendliness, though it were but a draught of water; which promise seemeth lnot [now?] to be taken, as if Christ had made them of any higher courtesy uncapable, and had promised reward not unto such as give them but that, but unto such as leave them but that. They considered how earnest the Apostle is, that if the ministers of the law were so amply provided for, less care then ought not to be had of them, who under the gospel of Jesus Christ possess correspondent rooms in the Church. They considered how needful it is that they who provoke all others unto works of mercy and charity should especially have wherewith to be examples of such things, and by such means to win them, with whom other means without those do commonly take very small effect. In these and the like considerations, the Church revenues were in ancient times augmented, our Lord thereby performing manifestly the promise made to his servants, that they which did “leave either father, or mother, or lands, or goods, for his sake, should receive even in this world an hundred fold.” For some hundreds of years together, they which joined themselves to the Church were fain to relinquish all worldly emoluments and to endure the hardness of an afflicted estate. Afterward the Lord gave rest to his Church, kings and princes became as fathers thereunto, the hearts of all men inclined towards it, and by his providence there grew unto it every day earthly possessions in more and more abundance, till the greatness thereof bred envy, which no diminutions are able to satisfy.

[22]For as those ancient nursing Fathers thought they did never bestow enough; even so in the eye of this present age, as long as any thing remaineth, it seemeth to be too much. Our fathers we imitate in perversum, as Tertullian speaketh; like them we are, by being in equal degree the contrary unto that which they were. Unto those earthly blessings which God as then did with so great abundance pour down upon the ecclesiastical state, we may in regard of most near resemblance  apply the selfsame words which the prophet hath, “God blessed them exceedingly, and by this very mean turned the hearts of their own brethren to hate them, and to deal politicly with his servants.” Computations are made, and there are huge sums set down, for princes to see how much they may amplify and enlarge their own treasure; how many public burdens they may ease; what present means they may have to reward their servants about them, if they please but to grant their assent, and to accept of the spoil of bishops, by whom church goods are but abused unto pomp and vanity. Thus albeit they deal with one whose princely virtue giveth them small hope to prevail in impious and sacrilegious motions, yet shame they not to move her royal majesty even with a suit not much unlike unto that wherewith the Jewish high priest [priests?] tried Judas, whom they solicited unto treason against his Master, and proposed unto him a number of silver pence in lieu of so virtuous and honest a service. But her sacred majesty disposed to be always like herself, her heart so far estranged from willingness to gain by pillage of that estate, the only awe whereof under God she hath been unto this present hour, as of all other parts of this noble commonwealth, whereof she hath vowed herself a protector till the end of her days on earth, which if nature could permit, we wish, as good cause we have, endless: this her gracious inclination is more than a seven times sealed warrant, upon the same assurance whereof, touchingm any action so dishonourable as this, we are on her part most secure, not doubting but that unto all posterity it shall for ever appear, that from the first to the very last of her sovereign proceedings there hath not been one authorized deed other than consonant with that Symmachus saith, “Fiscus bonorum principum, non sacerdotum damnis, sed hostium spoliis augeatur:” consonant with that imperial law, “Ea quæ ad beatissimæ ecclesiæ jura pertinent, tanquam ipsam sacrosanctam et religiosam ecclesiam, intacta convenit  venerabiliter custodiri; ut sicut ipsa religionis et fidei mater perpetua est, ita ejus patrimonium jugiter servetur illæsum.”

[23]As for the case of public burdens, let any politician living make it appear, that by confiscation of bishops’ livings, and their utter dissolution at once, the commonwealth shall ever have half that relief and ease which it receiveth by their continuance as now they are, and it shall give us some cause to think, that albeit we see they are impiously and irreligiously minded, yet we may esteem them at least to be tolerable commonwealth’s-men. But the case is too clear and manifest, the world doth but too plainly see it that no one order of subjects whatsoever within this land doth bear the seventh part of that proportion which the clergy beareth in the burdens of the commonwealth. No revenue of the crown like unto it, either for certainty or for greatness. Let the good which this way hath grown to the commonwealth by the dissolution of religious houses, teach men what ease unto public burdens there is like to grow by the overthrow of the clergy. My meaning is not hereby to make the state of bishoprick and of those dissolved companies alike, the one no less unlawful to be removed than the other. For those religious persons were men which followed only a special kind of contemplative life in the commonwealth, they were properly no portion of God’s clergy (only such amongst them excepted as were also priests), their goods (that excepted which they unjustly held through the pope’s usurped power of appropriating ecclesiastical livings unto them) may in part seem to be of the nature of civil possessions, held by other kinds of corporations, such as the city of London hath divers. Wherefore as their institution was human, and their end for the most part superstitious, they had not therein merely that holy and divine interest which belongeth unto bishops, who being employed by Christ in the principal service of his Church, are receivers and disposers of his patrimony, as hath been shewed, which whosoever shall withhold or withdraw at any time from them, he undoubtedly robbeth God himself.

[24]If they abuse the goods of the Church unto pomp and vanity, such faults we do not excuse in them. Only we wish it to be considered whether such faults be verily in them, or  else but objected against them by such as gape after spoil, and therefore are no competent judges what is moderate and what excessive in them, whom under this pretence they would spoil. But the accusation may be just. In plenty and fulness it may be we are of God more forgetful than were requisite. Notwithstanding men should remember how not to the clergy alone it was said by Moses in Deuteronomy, “Ne cum manducaveris et biberis et domos optimas ædificaveris.” If the remedy prescribed for this disease be good, let it unpartially be applied. “Interest reipub. ut re sua quisque bene utatur.” Let all states be put to their moderate pensions, let their livings and lands be taken away from them whosoever they be, in whom such ample possessions are found to have been matters of grievous abuse: were this just? would noble families think this reasonable? The title which bishops have to their livings is as good as the title of any sort of men unto whatsoever we account to be most justly held by them; yea in this one thing the claim of bishops hath preeminence above all secular titles of right, in that God’s own interest is the tenure whereby they hold, even as also it was to the priests of the law an assurance of their spiritual goods and possessions, whereupon, though they many times abused greatly the goods of the Church, yet was not God’s patrimony therefore taken away from them, and made saleable unto other tribes. To rob God, to ransack the Church, to overthrow the whole order of Christian bishops, and to turn them out of land and living, out of house and home, what man of common honesty can think it for any manner of abuse to be a remedy lawful or just? We must confess that God is righteous in taking away that which men abuse: but doth that excuse the violence of thieves and robbers?

[25]Complain we will not with St. Jerome, “That the hands of men are so straitly tied, and their liberal minds so much bridled and held back from doing good by augmentation of the Church patrimony.” For we confess that herein  mediocrity may be and hath been sometime exceeded. There did want heretofore a Moses to temper men’s liberality, to say unto them who enriched the Church, Sufficit, Stay your hands, lest fervour of zeal do cause you to empty yourselves too far. It may be the largeness of men’s hearts being then more moderate, had been after more durable; and one state by too much overgrowing the rest, had not given occasion unto the rest to undermine it. That evil is now sufficiently cured: the Church treasury, if then it were over full, hath since been reasonable [reasonably?] well emptied. That which Moses spake unto givers, we must now inculcate unto takers away from the Church, Let there be some stay, some stint in spoiling. If “grape-gatherers came unto them,” saith the prophet, “would they not leave some remnant behind?” But it hath fared with the wealth of the Church as with a tower, which being built at the first with the highest, overthroweth itself after by its own greatness; neither doth the ruin thereof cease with the only fall of that which hath exceeded mediocrity, but one part beareth down another, till the whole be laid prostrate. For although the state ecclesiastical, both others and even bishops themselves, be now fallen to so low an ebb, as all the world at this day doth see; yet because there remaineth still somewhat which unsatiable minds can thirst for, therefore we seem not to have been hitherto sufficiently wronged. Touching that which hath been taken from the Church in appropriations known to amount to the value of one hundred twenty-six thousand pounds yearly, we rest contentedly and quietly without it, till it shall please God to touch the hearts of men, of their own voluntary accord, to restore it to him again; judging thereof no otherwise than some others did of those goods which were by Sylla taken away from the citizens of Rome, that albeit they were in truth male capta, unconscionably taken away from the right owners at the first, nevertheless, seeing that such as were after possessed of them held them not without some title, which law did after a sort made good, repetitio eorum proculdubio labefactabat compositam civitatem. What hath been taken away as dedicated unto uses superstitious, and consequently not given unto God,  or at the leastwise not so rightly given, we repine not thereat. That which hath gone by means secret and indirect, through corrupt compositions or compacts, we cannot help. What the hardness of men’s hearts doth make them loth to have exacted, though being due by law, even thereof the want we do also bear. Out of that which after all these deductions cometh clearly unto our hands, I hope it will not be said that towards the public charge we disburse nothing. And doth the residue seem yet excessive? The ways whereby temporal men provide for themselves and their families are fore-closed unto us. All that we have to sustain our miserable life with, is but a remnant of God’s own treasure, so far already diminished and clipped, that if there were any sense of common humanity left in this hard-hearted world, the impoverished estate of the clergy of God would at the length even of very commiseration be spared. The mean gentleman that hath but an hundred pound land to live on, would not be hasty to change his worldly estate and condition with many of these so over abounding prelates; a common artisan or tradesman of the city, with ordinary pastors of the Church.

[26]It is our hard and heavy lot, that no other sort of men being grudged at, how little benefit soever the public weal reap by them, no state complained of for holding that which hath grown unto them by lawful means; only the governors of our souls, they that study night and day so to guide us, that both in this world we may have comfort and in the world to come endless felicity and joy (for even such is the very scope of all their endeavours, this they wish, for this they labour, how hardly soever we use to construe of their intents): hard, that only they should be thus continually lifted at for possessing but that whereunto they have by law both of God and man most just title. If there should be no other remedy but that the violence of men in the end must needs bereave them of all succour, further than the inclination of others shall vouchsafe to cast upon them, as it were by way of alms for their relief but from hour to hour; better they are not than their fathers, which have been contented with as hard a portion at the world’s hands: let the light of the sun and moon, the common benefit of heaven and earth be taken from bishops, if the question were whether God should lose his glory, and the safety of his  Church be hazarded, or they relinquish the right and interest which they have in the things of this world. But sith the question in truth is whether Levi shall be deprived of the portion of God or no, to the end that Simeon or Reuben may devour it as their spoil, the comfort of the one in sustaining the injuries which the other would offer, must be that prayer poured out by Moses the prince of prophets, in most tender affection to Levi, “Bless, O Lord, his substance, accept thou the work of his hands; smite through the loins of them that rise up against him, and of them which hate him, that they rise no more.”

