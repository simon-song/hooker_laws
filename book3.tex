\chapter*[The Third Book]{THE THIRD BOOK. 
CONCERNING THEIR SECOND ASSERTION, THAT IN SCRIPTURE THERE MUST BE OF NECESSITY CONTAINED A FORM OF CHURCH POLITY, THE LAWS WHEREOF MAY IN NOWISE BE ALTERED.}
\label{chap:book3}
\addcontentsline{toc}{chapter}{THE THIRD BOOK}

THE MATTER CONTAINED IN THIS THIRD BOOK.

I. What the Church is, and in what respect Laws of Polity are thereunto necessarily required.

II. Whether it be necessary that some particular Form of Church Polity be set down in Scripture, sith the things that belong particularly to any such Form are not of necessity to Salvation.

III. That matters of Church Polity are different from matters of Faith and Salvation, and that they themselves so teach which are our reprovers for so teaching.

IV. That hereby we take not from Scripture any thing which thereunto with the soundness of truth may be given.

V. Their meaning who first urged against the Polity of the Church of England, that nothing ought to be established in the Church more than is commanded by the Word of God.

VI. How great injury men by so thinking should offer unto all the Churches of God.

VII. A shift notwithstanding to maintain it, by interpreting commanded, as though it were meant that greater things only ought to be found set down in Scripture particularly, and lesser framed by the general rules of Scripture.

VIII. Another device to defend the same, by expounding commanded, as if it did signify grounded on Scripture, and were opposed to things found out by light of natural reason only.

IX. How Laws for the Polity of the Church may be made by the advice of men, and how those Laws being not repugnant to the Word of God are approved in his sight.

X. That neither God’s being the Author of Laws, nor yet his committing of them to Scripture, is any reason sufficient to prove that they admit no addition or change.

XI. Whether Christ must needs intend Laws unchangeable altogether, or have forbidden any where to make any other Law than himself did deliver.

\PRLsep

\section*{What the Church is, and in what respect Laws of Polity are thereunto necessarily required.}

I. ALBEIT the substance of those controversies whereinto we have begun to wade be rather of outward things appertaining to the Church of Christ, than of any thing wherein the nature and being of the Church consisteth, yet because the subject or matter which this position concerneth is,A Form of Church Government or Church Polity, it therefore behoveth us so far forth to consider the nature of the Church, as is requisite for men’s more clear and plain understanding in what respect Laws of Polity or Government are necessary thereunto.

[2.]That Church of Christ, which we properly term his body mystical, can be but one; neither can that one be sensibly discerned by any man, inasmuch as the parts thereof are some in heaven already with Christ, and the rest that are on earth (albeit their natural persons be visible) we do not discern under this property, whereby they are truly and infallibly of that body. Only our minds by intellectual conceit are able to apprehend, that such a real body there is, a body collective, because it containeth an huge multitude; a body mystical, because the mystery of their conjunction is removed altogether from sense. Whatsoever we read in Scripture concerning the endless love and the saving mercy which God sheweth towards his Church, the only proper subject thereof is this Church. Concerning this flock it is that our Lord and Saviour hath promised, “I give unto them eternal life, and they shall never perish, neither shall any pluck them out of my hands.” They who are of this society have such marks and notes of distinction from all others, as are not object unto our sense; only unto God, who seeth their hearts and understandeth all their secret cogitations, unto him they are clear and manifest. All men knew Nathanael to be an Israelite. But our Saviour piercing deeper giveth further testimony of him than men could have done with such certainty as he did, “Behold indeed an Israelite in whom is no guile.” If we profess, as Peter did, that we love the Lord, and profess it in the hearing of men, charity is prone to believe all things, and therefore charitable men are likely to think we do so, as long as they see no proof to the contrary.  But that our love is sound and sincere, that it cometh from “a pure heart and a good conscience and a faith unfeigned,” who can pronounce, saving only the Searcher of all men’s hearts, who alone intuitively doth know in this kind who are His?

[3.]And as those everlasting promises of love, mercy, and blessedness belong to the mystical Church; even so on the other side when we read of any duty which the Church of God is bound unto, the Church whom this doth concern is a sensibly known company. And this visible Church in like sort is but one, continued from the first beginning of the world to the last end. Which company being divided into two moieties, the one before, the other since the coming of Christ; that part, which since the coming of Christ partly hath embraced and partly shall hereafter embrace the Christian Religion, we term as by a more proper name the Church of Christ. And therefore the Apostle affirmeth plainly of all men Christian, that be they Jews or Gentiles, bond or free, they are all incorporated into one company, they all make but one body. The unity of which visible body and Church of Christ consisteth in that uniformity which all several persons thereunto belonging have, by reason of that one Lord whose servants they all profess themselves, that one Faith which they all acknowledge, that one Baptism wherewith they are all initiated.

[4.]The visible Church of Jesus Christ is therefore one, in outward profession of those things, which supernaturally appertain to the very essence of Christianity, and are necessarily required in every particular Christian man. “Let all the house of Israel know for certainty,” saith Peter, “that God hath made him both Lord and Christ, even this Jesus whom you have crucified.” Christians therefore they are not, which call not him their Master and Lord. And from hence it came that first at Antioch, and afterwards throughout the whole world, all that are of the Church visible were  called Christians even amongst the heathen. Which name unto them was precious and glorious, but in the estimation of the rest of the world even Christ Jesus himself was execrable; for whose sake all men were so likewise which did acknowledge him to be their Lord. This himself did foresee, and therefore armed his Church, to the end they might sustain it without discomfort. “All these things they will do unto you for my name’s sake; yea, the time shall come, that whosoever killeth you will think that he doth God good service2.” “These things I tell you, that when the hour shall come, ye may then call to mind how I told you beforehand of them.”

[5.]But our naming of Jesus Christ the Lord is not enough to prove us Christians, unless we also embrace that faith, which Christ hath published unto the world. To shew that the angel of Pergamus continued in Christianity, behold how the Spirit of Christ speaketh, “Thou keepest my name, and thou hast not denied my faith.” Concerning which faith, “the rule thereof,” saith Tertullian, “is one alone, immovable, and no way possible to be better framed anew.” What rule that is he sheweth by rehearsing those few articles of Christian belief. And before Tertullian, Ireney; “The Church though scattered through the whole world unto the utmost borders of the earth, hath from the Apostles and their disciples received belief.” The  parts of which belief he also reciteth, in substance the very same with Tertullian, and thereupon inferreth, “This faith the Church being spread far and wide preserveth as if one house did contain them: these things it equally embraceth, as though it had even one soul, one heart, and no more: it publisheth, teacheth and delivereth these things with uniform consent, as if God had given it but one only tongue wherewith to speak. He which amongst the guides of the Church is best able to speak uttereth no more than this, and less than this the most simple doth not utter,” when they make profession of their faith.

[6.]Now although we know the Christian faith and allow of it, yet in this respect we are but entering; entered we are not into the visible Church before our admittance by the door of Baptism. Wherefore immediately upon the acknowledgment of Christian faith, the Eunuch (we see) was baptized by Philip, Paul by Ananias, by Peter an huge multitude containing three thousand souls, which being once baptized were reckoned in the number of souls added to the visible Church.

[7.]As for those virtues that belong unto moral righteousness and honesty of life, we do not mention them, because they are not proper unto Christian men, as they are Christian, but do concern them as they are men. True it is, the want of these virtues excludeth from salvation. So doth much  more the absence of inward belief of heart; so doth despair and lack of hope; so emptiness of Christian love and charity. But we speak now of the visible Church, whose children are signed with this mark, “One Lord, one Faith, one Baptism.” In whomsoever these things are, the Church doth acknowledge them for her children; them only she holdeth for aliens and strangers, in whom these things are not found. For want of these it is that Saracens, Jews, and Infidels are excluded out of the bounds of the Church. Others we may not deny to be of the visible Church, as long as these things are not wanting in them. For apparent it is, that all men are of necessity either Christians or not Christians. If by external profession they be Christians, then are they of the visible Church of Christ: and Christians by external profession they are all, whose mark of recognizance hath in it those things which we have mentioned, yea, although they be impious idolaters, wicked heretics, persons excommunicable, yea, and cast out for notorious improbity. Such withal we deny not to be the imps and limbs of Satan, even as long as they continue such.

[8.]Is it then possible, that the selfsame men should belong both to the synagogue of Satan and to the Church of Jesus Christ? Unto that Church which is his mystical body, not possible; because that body consisteth of none but only true Israelites, true sons of Abraham, true servants and saints of God. Howbeit of the visible body and Church of Jesus Christ those may be and oftentimes are, in respect of the main parts of their outward profession, who in regard of their inward disposition of mind, yea, of external conversation, yea, even of some parts of their very profession, are most worthily both hateful in the sight of God himself, and in the eyes of the sounder parts of the visible Church most execrable. Our Saviour therefore compareth the kingdom of  heaven to a net, whereunto all which cometh neither is nor seemeth fish: his Church he compareth unto a field, where tares manifestly known and seen by all men do grow intermingled with good corn, and even so shall continue till the final consummation of the world. God hath had ever and ever shall have some Church visible upon earth. When the people of God worshipped the calf in the wilderness; when they adored the brazen serpent; when they served the gods of nations; when they bowed their knees to Baal; when they burnt incense and offered sacrifice unto idols: true it is, the wrath of God was most fiercely inflamed against them, their prophets justly condemned them, as an adulterous seed and a wicked generation of miscreants, which had forsaken the living God, and of him were likewise forsaken, in respect of that singular mercy wherewith he kindly and lovingly embraceth his faithful children. Howbeit retaining the law of God and the holy seal of his covenant, the sheep of his visible flock they continued even in the depth of their disobedience and rebellion. Wherefore not only amongst them God always had his Church, because he had thousands which never bowed their knees to Baal; but whose knees were bowed unto Baal, even they were also of the visible Church of God. Nor did the Prophet so complain, as if that Church had been quite and clean extinguished; but he took it as though there had not been remaining in the world any besides himself, that carried a true and an upright heart towards God with care to serve him according unto his holy will.

[9.]For lack of diligent observing the difference, first between the Church of God mystical and visible, then between the visible sound and corrupted, sometimes more, sometimes less, the oversights are neither few nor light that have been committed. This deceiveth them, and nothing else, who think that in the time of the first world the family of Noah did contain all that were of the visible Church of God.  From hence it grew, and from no other cause in the world, that the African bishops in the council of Carthage, knowing how the administration of baptism belongeth only to the Church of Christ, and supposing that heretics which were apparently severed from the sound believing Church could not possibly be of the Church of Jesus Christ, thought it utterly against reason, that baptism administered by men of corrupt belief should be accounted as a sacrament. And therefore in maintenance of rebaptization their arguments are built upon the fore-alleged ground, “That heretics are not at all any part of the Church of Christ. Our Saviour founded his Church on a rock, and not upon heresy. Power of baptizing he gave to his Apostles, unto heretics he gave it not. Wherefore they that are without the Church, and oppose themselves against Christ, do but scatter His sheep and flock, without the Church baptize they cannot.” Again, “Are heretics Christians or are they not? If they be Christians, wherefore remain they not in God’s Church? If they be no Christians, how make they Christians? Or to what purpose shall those words of the Lord serve: ‘He which is not with me is against me;’ and, ‘He which gathereth not with me scattereth?’ Wherefore evident it is, that upon misbegotten children and the brood of Antichrist without rebaptization the Holy Ghost cannot descend.” But none in this case so earnest as Cyprian: “I know no baptism but one, and that in the  Church only; none without the Church, where he that doth cast out the devil hath the devil: he doth examine about belief whose lips and words do breathe forth a canker; the faithless doth offer the articles of faith; a wicked creature forgiveth wickedness; in the name of Christ Antichrist signeth; he which is cursed of God blesseth; a dead carrion promiseth life; a man unpeaceable giveth peace; a blasphemer calleth upon the name of God; a profane person doth exercise priesthood; a sacrilegious wretch doth prepare the altar; and in the neck of all these that evil also cometh, the Eucharist a very bishop of the devil doth presume to consecrate.” All this was true, but not sufficient to prove that heretics were in no sort any part of the visible church of Christ, and consequently their baptism no baptism. This opinion therefore was afterwards both condemned by a better advised council, and also revoked by the chiefest of the authors thereof themselves.

[10.]What is it but only the selfsame error and misconceit,  wherewith others being at this day likewise possessed, they ask us where our Church did lurk, in what cave of the earth it slept for so many hundreds of years together before the birth of Martin Luther? As if we were of opinion that Luther did erect a New Church of Christ. No, the Church of Christ which was from the beginning is and continueth unto the end: of which Church all parts have not been always equally sincere and sound. In the days of Abia it plainly appeareth that Judah was by many degrees more free from pollution than Israel, as that solemn oration sheweth wherein he pleadeth for the one against the other in this wise: “O Jeroboam and all Israel hear you me: have ye not driven away the priests of the Lord, the sons of Aaron and the Levites, and have made you priests like the people of nations? Whosoever cometh to consecrate with a young bullock and seven rams, the same may be a priest of them that are no gods. But we belong unto the Lord our God, and have not forsaken him; and the priests the sons of Aaron minister unto the Lord every morning and every evening burnt-offerings and sweet incense, and the bread is set in order upon the pure table, and the candlestick of gold with the lamps thereof to burn every evening; for we keep the watch of the Lord our God, but ye have forsaken him.” In St. Paul’s time the integrity of Rome was famous; Corinth many ways reproved; they of Galatia much more out of square. In St. John’s time Ephesus and Smyrna in far better state than Thyatira and Pergamus were. We hope therefore that to reform ourselves, if at any time we have done amiss, is not to sever ourselves from the Church we were of  before. In the Church we were, and we are so still. Other difference between our estate before and now we know none but only such as we see in Juda; which having sometime been idolatrous became afterwards more soundly religious by renouncing idolatry and superstition. If Ephraim “be joined unto idols,” the counsel of the Prophet is, “Let him alone.” “If Israel play the harlot, let not Juda sin.” “If it seem evil unto you,” saith Josua, “to serve the Lord, choose you this day whom ye will serve; whether the gods whom your fathers served beyond the flood, or the gods of the Amorites in whose land ye dwell: but I and mine house will serve the Lord.” The indisposition therefore of the Church of Rome to reform herself must be no stay unto us from performing our duty to God; even as desire of retaining conformity with them could be no excuse if we did not perform that duty.

Notwithstanding so far as lawfully we may, we have held and do hold fellowship with them. For even as the Apostle doth say of Israel that they are in one respect enemies but in another beloved of God; in like sort with Rome we dare not communicate concerning sundry her gross and grievous abominations, yet touching those main parts of Christian truth wherein they constantly still persist, we gladly acknowledge them to be of the family of Jesus Christ; and our hearty prayer unto God Almighty is, that being conjoined so far forth with them, they may at the length (if it be his will) so yield to frame and reform themselves, that no distraction remain in any thing, but that we “all may with one heart and one mouth glorify God the Father of our Lord and Saviour,” whose Church we are.

As there are which make the Church of Rome utterly no Church at all, by reason of so many, so grievous errors in their doctrines; so we have them amongst us, who under pretence of imagined corruptions in our discipline do give even as hard a judgment of the Church of England itself.

[11.]But whatsoever either the one sort or the other teach, we must acknowledge even heretics themselves to be, though a maimed part, yet a part of the visible Church. If an infidel  should pursue to death an heretic professing Christianity, only for Christian profession’s sake, could we deny unto him the honour of martyrdom? Yet this honour all men know to be proper unto the Church. Heretics therefore are not utterly cut off from the visible Church of Christ.

If the Fathers do any where, as oftentimes they do, make the true visible Church of Christ and heretical companies opposite; they are to be construed as separating heretics, not altogether from the company of believers, but from the fellowship of sound believers. For where professed unbelief is, there can be no visible Church of Christ; there may be, where sound belief wanteth. Infidels being clean without the Church deny directly and utterly reject the very principles of Christianity; which heretics embrace, and err only by misconstruction: whereupon their opinions, although repugnant indeed to the principles of Christian faith, are notwithstanding by them held otherwise, and maintained as most consonant thereunto. Wherefore being Christians in regard of the general truth of Christ which they openly profess, yet they are by the Fathers every where spoken of as men clean excluded out of the right believing Church, by reason of their particular errors, for which all that are of a sound belief must needs condemn them.

[12.]In this consideration, the answer of Calvin unto Farel concerning the children of Popish parents doth seem crazed. “Whereas,” saith he, “you ask our judgment about a matter, whereof there is doubt amongst you, whether ministers of our order professing the pure doctrine of the Gospel may lawfully admit unto baptism an infant whose father is a stranger unto our Churches, and whose mother hath fallen from us unto the Papacy, so that both the parents are popish: thus we have thought good to answer; namely, that it is an absurd thing for us to baptize them which cannot be reckoned members of our body. And sith Papists’  children are such, we see not how it should be lawful to minister baptism unto them.” Sounder a great deal is the answer of the ecclesiastical college of Geneva unto Knox, who having signified unto them, that himself did not think it lawful to baptize bastards or the children of idolaters (he meaneth Papists) or of persons excommunicate, till either the parents had by repentance submitted themselves unto the Church, or else their children being grown unto the years of understanding should come and sue for their own baptism: “For thus thinking,” saith he, “I am thought to be over-severe, and that not only by them which are popish, but even in their judgments also who think themselves maintainers of the truth.” Master Knox’s oversight herein they controlled. Their sentence was, “Wheresoever the profession of Christianity hath not utterly perished and been extinct, infants are beguiled of their right, if the common seal be denied them.” Which conclusion in itself is sound, although it seemeth the ground is but weak whereupon they built it. For the reason which they yield of their sentence, is this; “The promise which God doth make to the faithful concerning their seed reacheth unto a thousand generations; it resteth not only in the first degree of descent. Infants therefore whose great-grandfathers have been holy and godly, do in that respect belong to the body of the church, although the fathers and grandfathers of whom they descend have been apostates: because the tenure of the grace of God which did adopt them three hundred years ago or more in their ancient predecessors, cannot with justice be defeated and broken off by their parents’ impiety coming between.”  By which reason of theirs although it seem that all the world may be baptized, inasmuch as no man living is a thousand descents removed from Adam himself, yet we mean not at this time either to uphold or to overthrow it: only their alleged conclusion we embrace, so it be construed in this sort; “That forasmuch as men remain in the visible Church, till they utterly renounce the profession of Christianity, we may not deny unto infants their right by withholding from them the public sign of holy baptism, if they be born where the outward acknowledgment of Christianity is not clean gone and extinguished.” For being in such sort born, their parents are within the Church, and therefore their birth doth give them interest and right in baptism.

[13.]Albeit not every error and fault, yet heresies and crimes which are not actually repented of and forsaken, exclude quite and clean from that salvation which belongeth unto the mystical body of Christ; yea, they also make a separation from the visible sound Church of Christ; altogether from the visible Church neither the one nor the other doth sever. As for the act of excommunication, it neither shutteth out from the mystical, nor clean from the visible, but only from fellowship with the visible in holy duties. With what congruity then doth the Church of Rome deny, that her enemies, whom she holdeth always for heretics, do at all appertain to the Church of Christ; when her own do freely grant, that albeit the Pope (as they say) cannot teach heresy nor propound error, he may notwithstanding himself worship idols, think amiss concerning matters of faith, yea, give himself unto acts diabolical, even being Pope? How exclude they us from being any part of the Church of Christ under the colour and pretence of heresy, when they cannot but grant it possible even for him to be as touching his own personal persuasion  heretical, who in their opinion not only is of the Church, but holdeth the chiefest place of authority over the same? But of these things we are not now to dispute. That which already we have set down, is for our present purpose sufficient.

[14.]By the Church therefore in this question we understand no other than only the visible Church. For preservation of Christianity there is not any thing more needful, than that such as are of the visible Church have mutual fellowship and society one with another. In which consideration, as the main body of the sea being one, yet within divers precincts hath divers names; so the Catholic Church is in like sort divided into a number of distinct Societies, every of which is termed a Church within itself. In this sense the Church is always a visible society of men; not an assembly, but a society. For although the name of the Church be given unto Christian assemblies, although any multitude of Christian men congregated may be termed by the name of a Church, yet assemblies properly are rather things that belong to a Church. Men are assembled for performance of public actions; which actions being ended, the assembly dissolveth itself and is no longer in being, whereas the Church which was assembled doth no less continue afterwards than before. “Where but three are, and they of the laity also (saith Tertullian), yet there is a Church:” that is to say, a Christian assembly. But a Church, as now we are to understand it, is a Society; that is, a number of men belonging unto some Christian fellowship, the place and limits whereof are certain. That wherein they have communion is the public exercise of such duties as those mentioned in the Apostles’ Acts, Instruction, Breaking of Bread, and Prayers. As therefore they that are of the mystical body of Christ have those inward graces and virtues,  whereby they differ from all others, which are not of the same body; again, whosoever appertain to the visible body of the Church, they have also the notes of external profession, whereby the world knoweth what they are: after the same manner even the several societies of Christian men, unto every of which the name of a Church is given with addition betokening severalty, as the Church of Rome, Corinth, Ephesus, England, and so the rest, must be endued with correspondent general properties belonging unto them as they are public Christian societies. And of such properties common unto all societies Christian, it may not be denied that one of the very chiefest is Ecclesiastical Polity.

Which word I therefore the rather use, because the name of Government, as commonly men understand it in ordinary speech, doth not comprise the largeness of that whereunto in this question it is applied. For when we speak of Government, what doth the greatest part conceive thereby, but only the exercise of superiority peculiar unto rulers and guides of others? To our purpose therefore the name of Church-Polity will better serve, because it containeth both government and also whatsoever besides belongeth to the ordering of the Church in public. Neither is any thing in this degree more necessary than Church-Polity, which is a form of ordering the public spiritual affairs of the Church of God.

\section*{Whether it be necessary that some particular Form of Church Polity be set down in Scripture, sith the things that belong particularly to any such Form are not of necessity to Salvation.}

II. But we must note, that he which affirmeth speech to be necessary amongst all men throughout the world, doth not thereby import that all men must necessarily speak one kind of language. Even so the necessity of polity and regiment in all Churches may be held without holding any one certain form to be necessary in them all. Nor is it possible that any form of polity, much less of polity ecclesiastical, should be good, unless God himself be author of it. “Those things that are not of God” (saith Tertullian), “they can have no other than God’s adversary for their author.” Be it whatsoever in the Church of God, if it be not of God, we hate it. Of God it must be; either as those things sometime were, which God supernaturally revealed, and so delivered them unto Moses for government of the commonwealth of Israel; or else as those things which men find  out by help of that light which God hath given them unto that end. The very Law of Nature itself, which no man can deny but God hath instituted, is not of God, unless that be of God, whereof God is the author as well this later way as the former. But forasmuch as no form of Church-Polity is thought by them to be lawful, or to be of God, unless God be so the author of it that it be also set down in Scripture; they should tell us plainly, whether their meaning be that it must be there set down in whole or in part. For if wholly, let them shew what one form of Polity ever was so. Their own to be so taken out of Scripture they will not affirm; neither deny they that in part even this which they so much oppugn is also from thence taken. Again they should tell us, whether only that be taken out of Scripture which is actually and particularly there set down; or else that also which the general principles and rules of Scripture potentially contain. The one way they cannot as much as pretend, that all the parts of their own discipline are in Scripture: and the other way their mouths are stopped, when they would plead against all other forms besides their own; seeing the general principles are such as do not particularly prescribe any one, but sundry may equally be consonant unto the general axioms of the Scripture.

[2.]But to give them some larger scope and not to close them up in these straits: let their allegations be considered, wherewith they earnestly bend themselves against all which deny it necessary that any one complete form of Church-Polity should be in Scripture. First therefore whereas it hath been told them that matters of faith, and in general matters necessary unto salvation, are of a different nature from ceremonies, order, and the kind of church government; and that the one is necessary to be expressly contained in the word of God, or else manifestly collected out of the same, the other not so; that it is necessary not to receive the one, unless there be something in Scripture for them; the other free, if nothing against them may thence be alleged; although there do not appear any just or reasonable cause to reject  or dislike of this, nevertheless as it is not easy to speak to the contentation of minds exulcerated in themselves, but that somewhat there will be always which displeaseth; so herein for two things we are reproved. The first is misdistinguishing, because matters of discipline and church government are (as they say) “matters necessary to salvation and of faith,” whereas we put a difference between the one and the other. Our second fault is, injurious dealing with the Scripture of God, as if it contained only “the principal points of religion, some rude and unfashioned matter of building the Church, but had left out that which belongeth unto the form and fashion of it; as if there were in the Scripture no more than only to cover the Church’s nakedness, and not chains, bracelets, rings, jewels, to adorn her; sufficient to quench her thirst, to kill her hunger, but not to minister a more liberal, and (as it were) a more delicious and dainty diet.” In which case our apology shall not need to be very long.

\section*{That matters of Church Polity are different from matters of Faith and Salvation, and that they themselves so teach which are our reprovers for so teaching.}

III. The mixture of those things by speech which by nature are divided, is the mother of all error. To take away therefore that error which confusion breedeth, distinction is requisite. Rightly to distinguish is by conceit of mind to sever things different in nature, and to discern wherein they differ. So that if we imagine a difference where there is none,That matters of discipline are different from matters of faith and salvation; and that they themselves so teach which are our reprovers. because we distinguish where we should not, it may not be denied that we misdistinguish. The only trial whether we do so, yea or no, dependeth upon comparison between our conceit and the nature of things conceived.

[2.]Touching matters belonging unto the Church of Christ this we conceive, that they are not of one suit. Some things are merely of faith, which things it doth suffice that we know and believe; some things not only to be known but done, because they concern the actions of men. Articles about the Trinity are matters of mere faith, and must be believed. Precepts concerning the works of charity are matters of action; which to know, unless they be practised, is not enough. This being so clear to all men’s understanding, I somewhat marvel that they especially should think it absurd to oppose Church-government, a plain matter of action, unto matters of faith, who know that themselves divide the Gospel into Doctrine and Discipline. For if matters of discipline be rightly by them distinguished from matters of doctrine, why not matters of government by us as reasonably set against matters of faith? Do not they under doctrine comprehend the same which we intend by matter of faith? Do not they under discipline comprise the regiment of the Church? When they blame that in us which themselves follow, they give men great cause to doubt that some other thing than judgment doth guide their speech.

[3.]What the Church of God standeth bound to know or do, the same in part nature teacheth. And because nature can teach them but only in part, neither so fully as is requisite for man’s salvation, nor so easily as to make the way plain and expedite enough that many may come to the knowledge  of it, and so be saved; therefore in Scripture hath God both collected the most necessary things that the school of nature teacheth unto that end, and revealeth also whatsoever we neither could with safety be ignorant of, nor at all be instructed in but by supernatural revelation from him. So that Scripture containing all things that are in this kind any way needful for the Church, and the principal of the other sort, this is the next thing wherewith we are charged as with an error: we teach that whatsoever is unto salvation termed necessary by way of excellency, whatsoever it standeth all men upon to know or do that they may be saved, whatsoever there is whereof it may truly be said, “This not to believe is eternal death and damnation,” or, “This every soul that will live must duly observe;” of which sort the articles of Christian faith and the sacraments of the Church of Christ are: all such things if Scripture did not comprehend, the Church of God should not be able to measure out the length and the breadth of that way wherein for ever she is to walk, heretics and schismatics never ceasing some to abridge, some to enlarge, all to pervert and obscure the same. But as for those things that are accessory hereunto, those things that so belong to the way of salvation, as to alter them is no otherwise to change that way, than a path is changed by altering only the uppermost face thereof; which be it laid with gravel, or set with grass, or paved with stone, remaineth still the same path; in such things because discretion may teach the Church what is convenient, we hold not the Church further tied herein unto Scripture, than that against Scripture nothing be admitted in the Church, lest that path which ought always to be kept even, do thereby come to be overgrown with brambles and thorns.

[4.]If this be unsound, wherein doth the point of unsoundness lie? It is not that we make some things necessary, some things accessory and appendent only: for our Lord and Saviour himself doth make that difference, by terming judgment and mercy and fidelity with other things of like nature, “the greater and weightier matters of the law.” Is it then in that we account ceremonies, (wherein we do not comprise sacraments, or any other the like substantial duties in the  exercise of religion, but only such external rites as are usually annexed unto Church actions,) is it an oversight that we reckon these things and matters of government in the number of things accessory, not things necessary in such sort as hath been declared? Let them which therefore think us blameable consider well their own words. Do they not plainly compare the one unto garments which cover the body of the Church; the other unto rings, bracelets, and jewels, that only adorn it; the one to that food which the Church doth live by, the other to that which maketh her diet liberal, “dainty,” and more “delicious”? Is dainty fare a thing necessary to the sustenance, or to the clothing of the body rich attire? If not, how can they urge the necessity of that which themselves resemble by things not necessary? or by what construction shall any man living be able to make those comparisons true, holding that distinction untrue, which putteth a difference between things of external regiment in the Church and things necessary unto salvation?

\section*{That hereby we take not from Scripture any thing which thereunto with the soundness of truth may be given.}

IV. Now as it can be to nature no injury that of her we say the same which diligent beholders of her works have observed; namely, that she provideth for all living creatures nourishment which may suffice; that she bringeth forth no kind of creature whereto she is wanting in that which is needful: although we do not so far magnify her exceeding bounty, as to affirm that she bringeth into the world the sons of men  adorned with gorgeous attire, or maketh costly buildings to spring up out of the earth for them: so I trust that to mention what the Scripture of God leaveth unto the Church’s discretion in some things, is not in any thing to impair the honour which the Church of God yieldeth to the sacred Scripture’s perfection. Wherein seeing that no more is by us maintained, than only that Scripture must needs teach the Church whatsoever is in such sort necessary as hath been set down; and that it is no more disgrace for Scripture to have left a number of other things free to be ordered at the discretion of the Church, than for nature to have left it unto the wit of man to devise his own attire, and not to look for it as the beasts of the field have theirs: if neither this can import, nor any other proof sufficient be brought forth, that we either will at any time or ever did affirm the sacred Scripture to comprehend no more than only those bare necessaries; if we acknowledge that as well for particular application to special occasions, as also in other manifold respects, infinite treasures of wisdom are over and besides abundantly to be found in the Holy Scripture; yea, that scarcely there is any noble part of knowledge, worthy the mind of man, but from thence it may have some direction and light; yea, that although there be no necessity it should of purpose prescribe any one particular form of church government, yet touching the manner of governing in general the precepts that Scripture setteth down are not few, and the examples many which it proposeth for all church governors even in particularities to follow; yea, that those things finally which are of principal weight in the very particular form of church polity (although not that form which they imagine, but that which we against them uphold) are in the selfsame Scriptures contained: if all this be willingly granted by us which are accused “to pin the word of God in so narrow room, as that it should be able to direct us but in principal points of our religion; or as though the substance of religion or some rude and unfashioned matter of building the Church were uttered in them, and those things left out that should pertain to the form and fashion of it;” let the cause of the accused be referred to the accusers’ own conscience, and let that judge whether this accusation be deserved where it hath been laid.

\section*{Their meaning who first urged against the Polity of the Church of England, that nothing ought to be established in the Church more than is commanded by the Word of God.}

V. But so easy it is for every man living to err, and so hard to wrest from any man’s mouth the plain acknowledgment of error, that what hath been once inconsiderately defended, the same is commonly persisted in, as long as wit by whetting itself is able to find out any shift, be it never so slight, whereby to escape out of the hands of present contradiction. So that it cometh herein to pass with men unadvisedly fallen into error, as with them whose state hath no ground to uphold it, but only the help which by subtle conveyance they draw out of casual events arising from day to day, till at length they be clean spent. They which first gave out, that “nothing ought to be established in the Church which is not commanded by the word of God,” thought this principle plainly warranted by the manifest words of the Law, “Ye shall put nothing unto the word which I command you, neither shall you take aught therefrom, that ye may keep the commandments of the Lord your God, which I command you.” Wherefore having an eye to a number of rites and orders in the Church of England, as marrying with a ring, crossing in the one sacrament, kneeling at the other, observing of festival days more than only that which is called the Lord’s day, enjoining abstinence at certain times from some kinds of meat, churching of women after childbirth, degrees taken by divines in universities, sundry church offices, dignities, and callings, for which they found no commandment in the Holy Scripture, they thought by the one only stroke of that axiom to have cut them off. But that which they took for an oracle being sifted was repelled. True it is concerning the word of God, whether it be by misconstruction of the sense or by falsification of the words, wittingly to endeavour that any thing may seem divine which is not, or any thing not seem which is, were plainly to abuse, and even to falsify divine evidence; which injury offered but unto men, is most worthily counted heinous. Which point I wish they did well observe, with whom nothing is more familiar than to plead in these causes, “the law of God,” “the word of the Lord;” who notwithstanding when  they come to allege what word and what law they mean, their common ordinary practice is to quote by-speeches in some historical narration or other, and to urge them as if they were written in most exact form of law. What is to add to the law of God if this be not? When that which the word of God doth but deliver historically, we construe without any warrant as if it were legally meant, and so urge it further than we can prove that it was intended; do we not add to the laws of God, and make them in number seem more than they are? It standeth us upon to be careful in this case. For the sentence of God is heavy against them that wittingly shall presume thus to use the Scripture.

\section*{How great injury men by so thinking should offer unto all the Churches of God.}

VI. But let that which they do hereby intend be granted them; let it once stand as consonant to reason, that because we are forbidden to add to the law of God any thing, or to take aught from it, therefore we may not for matters of the Church make any law more than is already set down in Scripture: who seeth not what sentence it shall enforce us to give against all Churches in the world, inasmuch as there is not one, but hath had many things established in it, which though the Scripture did never command, yet for us to condemn were rashness? Let the Church of God even in the time of our Saviour Christ serve for example unto all the rest. In their domestical celebration of the passover, which supper they divided (as it were) into two courses; what Scripture did give commandment that between the first and the second he that was chief should put off the residue of his garments, and keeping on his feast-robe only wash the feet of them that were with him? What Scripture did command them never to lift up their hands unwashed in prayer unto God? which custom Aristeas (be the credit of the author more or less) sheweth wherefore they did so religiously observe. What Scripture did command the Jews every festival-day to fast till the sixth hour? the custom both  mentioned by Josephus in the history of his own life, and by the words of Peter signified. Tedious it were to rip up all such things as were in that church established, yea by Christ himself and by his Apostle observed, though not commanded any where in Scripture.

\section*{A shift notwithstanding to maintain it, by interpreting commanded, as though it were meant that greater things only ought to be found set down in Scripture particularly, and lesser framed by the general rules of Scripture.}

VII. Well, yet a gloss there is to colour that paradox, and notwithstanding all this, still to make it appear in show not to be altogether unreasonable. And therefore till further reply come, the cause is held by a feeble distinction; that the commandments of God being either general or special, although there be no express word for every thing in specialty, yet there are general commandments for all things, to the end, that even such cases as are not in Scripture particularly mentioned, might not be left to any to order at their pleasure only with caution, that nothing be done against the word of God: and that for this cause the Apostle hath set down in Scripture four general rules, requiring such things alone to be received in the Church as do best and nearest agree with the same rules, that so all things in the Church may be appointed, not only not against, but by and according to the word of God. The rules are these, “Nothing scandalous or offensive unto any, especially unto the Church of God;” “All things in order and with seemliness;” “All unto edification;” finally, “All to the glory of God.” Of which kind how many might be gathered out of the Scripture, if it were necessary to take so much pains? Which rules they that urge, minding thereby to prove that nothing may be done in the Church but what Scripture commandeth, must needs hold that they tie the Church of Christ no otherwise than only because we find them there set down by the finger of the Holy Ghost. So that unless the Apostle by writing had delivered those rules to the Church, we should by observing them have sinned, as now by not observing them.

[2.]In the Church of the Jews is it not granted, that the appointment of the hour for daily sacrifices; the building of synagogues throughout the land to hear the word of God and  to pray in, when they came not up to Jerusalem, the erecting of pulpits and chairs to teach in, the order of burial, the rites of marriage, with such-like, being matters appertaining to the Church, yet are not any where prescribed in the law, but were by the Church’s discretion instituted? What then shall we think? Did they hereby add to the law, and so displease God by that which they did? None so hardly persuaded of them. Doth their law deliver unto them the selfsame general rules of the Apostle, that framing thereby their orders they might in that respect clear themselves from doing amiss? St. Paul would then of likelihood have cited them out of the Law, which we see he doth not. The truth is, they are rules and canons of that law which is written in all men’s hearts; the Church had for ever no less than now stood bound to observe them, whether the Apostles had mentioned them or no.

Seeing therefore those canons do bind as they are edicts of nature, which the Jews observing as yet unwritten, and thereby framing such church orders as in their law were not prescribed, are notwithstanding in that respect unculpable: it followeth that sundry things may be lawfully done in the Church, so as they be not done against the Scripture, although no Scripture do command them, but the Church only following the light of reason judge them to be in discretion meet.

[3.]Secondly, unto our purpose and for the question in hand, whether the commandments of God in Scripture be general or special, it skilleth not: for if being particularly applied they have in regard of such particulars a force constraining us to take some one certain thing of many, and to leave the rest; whereby it would come to pass, that any other particular but that one being established, the general rules themselves in that case would be broken; then is it utterly impossible that God should leave any thing great or small free for the Church to establish or not.

[4.]Thirdly, if so be they shall grant, as they cannot otherwise do, that these rules are no such laws as require any one particular thing to be done, but serve rather to direct the Church in all things which she doth; so that free and lawful it is to devise any ceremony, to receive any order, and to authorize any kind of regiment, no special commandment  being thereby violated, and the same being thought such by them, to whom the judgment thereof appertaineth, as that it is not scandalous, but decent, tending unto edification, and setting forth the glory of God; that is to say, agreeable unto the general rules of Holy Scripture: this doth them no good in the world for the furtherance of their purpose. That which should make for them must prove that men ought not to make laws for church regiment, but only keep those laws which in Scripture they find made. The plain intent of the Book of Ecclesiastical Discipline is to shew that men may not devise laws of church government, but are bound for ever to use and to execute only those which God himself hath already devised and delivered in the Scripture. The selfsame drift the Admonitioners also had, in urging that nothing ought to be done in the Church according unto any law of man’s devising, but all according to that which God in his word hath commanded. Which not remembering, they gather out of Scripture general rules to be followed in making laws; and so in effect they plainly grant that we ourselves may lawfully make laws for the Church, and are not bound out of Scripture only to take laws already made, as they meant who first alleged that principle whereof we speak. One particular platform it is which they respected, and which they laboured thereby to force upon all Churches; whereas these general rules do not let but that there may well enough be sundry. It is the particular order established in the Church of England, which thereby they did intend to alter, as being not commanded of God; whereas unto those general rules they know we do not defend that we may hold any thing unconformable. Obscure it is not what meaning they had, who first gave out that grand axiom; and according unto that meaning it doth prevail far and wide with the favourers of that part. Demand of them, wherefore they conform not themselves unto the order of our Church, and in every particular their answer for the most part is, “We find no such thing commanded in the word:” whereby they plainly require some special commandment for that which is exacted at their hands; neither are they content  to have matters of the Church examined by general rules and canons. 

[5.]As therefore in controversies between us and the Church of Rome, that which they practise is many times even according to the very grossness of that which the vulgar sort conceiveth; when that which they teach to maintain it is so nice and subtle that hold can very hardly be taken thereupon; in which cases we should do the Church of God small benefit by disputing with them according unto the finest points of their dark conveyances, and suffering that sense of their doctrine to go uncontrolled, wherein by the common sort it is ordinarily received and practised: so considering what disturbance hath grown in the Church amongst ourselves, and how the authors thereof do commonly build altogether on this as a sure foundation, “Nothing ought to be established in the Church which in the word of God is not commanded;” were it reason that we should suffer the same to pass without controlment in that current meaning whereby every where it prevaileth, and stay till some strange construction were made thereof, which no man would lightly have thought on but being driven thereunto for a shift?

\section*{Another device to defend the same, by expounding commanded, as if it did signify grounded on Scripture, and were opposed to things found out by light of natural reason only.}

VIII. The last refuge in maintaining this position is thus to construe it, “Nothing ought to be established in the Church, but that which is commanded in the word of God;” that is to say, all Church orders must be “grounded upon the word of God;” in such sort grounded upon the word, not that being found out by some “star, or light of reason, or learning, or other help,” they may be received, so they be not against the word of God; but according at leastwise unto the general rules of Scripture they must be made. Which is in effect as much as to say, “We know not what to say well in defence of this position; and therefore lest we should say it is false, there is no remedy but to say that in some sense or other it may be true, if we could tell how.”

[2.]First, that scholy had need of a very favourable reader and a tractable, that should think it plain construction, when to be commanded in the word and grounded upon the word are made all one. If when a man may live in the state of matrimony, seeking that good thereby which nature principally  desireth, he make rather choice of a contrary life in regard of St. Paul’s judgment; that which he doth is manifestly grounded upon the word of God, yet not commanded in his word, because without breach of any commandment he might do otherwise.

[3.]Secondly, whereas no man in justice and reason can be reproved for those actions which are framed according unto that known will of God, whereby they are to be judged; and the will of God which we are to judge our actions by, no sound divine in the world ever denied to be in part made manifest even by light of nature, and not by Scripture alone: if the Church being directed by the former of these two (which God hath given who gave the other, that man might in different sort be guided by them both), if the Church I say do approve and establish that which thereby it judgeth meet, and findeth not repugnant to any word or syllable of holy Scripture; who shall warrant our presumptuous boldness controlling herein the Church of Christ?

[4.]But so it is, the name of the light of nature is made hateful with men; the “star of reason and learning,” and all other such like helps, beginneth no otherwise to be thought of than if it were an unlucky comet; or as if God had so accursed it, that it should never shine or give light in things concerning our duty any way towards him, but be esteemed as that star in the Revelation called Wormwood, which being fallen from heaven, maketh rivers and waters in which it falleth so bitter, that men tasting them die thereof. A number there are, who think they cannot admire as they ought the power and authority of the word of God, if in things divine they should attribute any force to man’s reason. For which cause they never use reason so willingly as to disgrace reason. Their usual and common discourses are unto this effect. First, “the natural man perceiveth not the things of the Spirit of God: for they are foolishness unto him: neither can he know them, because they are spiritually discerned.” Secondly, it is not for nothing that St. Paul giveth charge to “beware of philosophy,” that is to say, such knowledge as men by natural reason attain unto. Thirdly, consider them  that have from time to time opposed themselves against the Gospel of Christ, and most troubled the Church with heresy. Have they not always been great admirers of human reason? Hath their deep and profound skill in secular learning made them the more obedient to the truth, and not armed them rather against it? Fourthly, they that fear God will remember how heavy his sentences are in this case: “I will destroy the wisdom of the wise, and will cast away the understanding of the prudent. Where is the wise? where is the scribe? where is the disputer of this world? hath not God made the wisdom of this world foolishness? Seeing the world by wisdom knew not God in the wisdom of God, it pleased God by the foolishness of preaching to save believers.” Fifthly, the word of God in itself is absolute, exact and perfect. The word of God is a two-edged sword; as for the weapons of natural reason, they are as the armour of Saul, rather cumbersome about the soldier of Christ than needful. They are not of force to do that which the Apostles of Christ did by the power of the Holy Ghost: “My preaching,” therefore saith Paul, “hath not been in the enticing speech of man’s wisdom, but in plain evidence of the Spirit and of power, that your faith might not be in the wisdom of men, but in the power of God.” Sixthly, if I believe the Gospel, there needeth no reasoning about it to persuade me; if I do not believe, it must be the Spirit of God and not the reason of man that shall convert my heart unto him. By these and the like disputes an opinion hath spread itself very far in the world, as if the way to be ripe in faith were to be raw in wit and judgment; as if reason were an enemy unto religion, childish simplicity the mother of ghostly and divine wisdom.

[5.]The cause why such declamations prevail so greatly, is, for that men suffer themselves in two respects to be deluded; one is, that the wisdom of man being debased either in comparison with that of God, or in regard of some special thing exceeding the reach and compass thereof, it seemeth to them (not marking so much) as if simply it were condemned: another, that learning, knowledge or wisdom, falsely so termed, usurping a name whereof they are not worthy, and being  under that name controlled; their reproof is by so much the more easily misapplied, and through equivocation wrested against those things whereunto so precious names do properly and of right belong. This, duly observed, doth to the former allegations itself make sufficient answer. Howbeit, for all men’s plainer and fuller satisfaction:

[6.]First, Concerning the inability of reason to search out and to judge of things divine, if they be such as those properties of God and those duties of men towards him, which may be conceived by attentive consideration of heaven and earth; we know that of mere natural men the Apostle testifieth, how they knew both God, and the Law of God. Other things of God there be which are neither so found, nor though they be shewed can never be approved without the special operation of God’s good grace and Spirit. Of such things sometime spake the Apostle St. Paul, declaring how Christ had called him to be a witness of his death and resurrection from the dead, according to that which the Prophets and Moses had foreshewed. Festus, a mere natural man, an infidel, a Roman, one whose ears were unacquainted with such matter, heard him, but could not reach unto that whereof he spake; the suffering and the rising of Christ from the dead he rejecteth as idle superstitious fancies not worth the hearing. The Apostle that knew them by the Spirit, and spake of them with power of the Holy Ghost, seemed in his eyes but learnedly mad. Which example maketh manifest what elsewhere the same Apostle teacheth, namely, that nature hath need of grace, whereunto I hope we are not opposite, by holding that grace hath use of nature.

[7.]Secondly, Philosophy we are warned to take heed of: not that philosophy, which is true and sound knowledge attained by natural discourse of reason; but that philosophy, which to bolster heresy or error casteth a fraudulent show of reason upon things which are indeed unreasonable, and by that mean as by a stratagem spoileth the simple which are not able to withstand such cunning. “Take heed lest any spoil you through philosophy and vain deceit.” He that exhorteth to beware of an enemy’s policy doth not give  counsel to be impolitic, but rather to use all provident foresight and circumspection, lest our simplicity be overreached by cunning sleights. The way not to be inveigled by them that are so guileful through skill, is thoroughly to be instructed in that which maketh skilful against guile, and to be armed with that true and sincere philosophy, which doth teach, against that deceitful and vain, which spoileth.

[8.]Thirdly, But many great philosophers have been very unsound in belief. And many sound in belief, have been also great philosophers. Could secular knowledge bring the one sort unto the love of Christian faith? Nor Christian faith the other sort out of love with secular knowledge. The harm that heretics did, they did it unto such as were unable to discern between sound and deceitful reasoning; and the remedy against it was ever the skill which the ancient Fathers had to descry and discover such deceit. Insomuch that Cresconius the heretic complained greatly of St. Augustine, as being too full of logical subtilties. Heresy prevaileth only by a counterfeit show of reason; whereby notwithstanding it becometh invincible, unless it be convicted of fraud by manifest remonstrance clearly true and unable to be withstood. When therefore the Apostle requireth ability to convict heretics, can we think he judgeth it a thing unlawful, and not rather needful, to use the principal instrument of their conviction, the light of reason? It may not be denied but that in the Fathers’ writings there are sundry sharp invectives against heretics, even for their very philosophical reasonings. The cause whereof Tertullian confesseth not to have been any dislike conceived against the kind of such reasonings, but the end. “We may,” saith he, “even in matters of God  be made wiser by reasons drawn from the public persuasions, which are grafted in men’s minds: so they be used to further the truth, not to bolster error; so they make with, not against, that which God hath determined. For there are some things even known by nature, as the immortality of the soul unto many, our God unto all. I will therefore myself also use the sentence of some such as Plato, pronouncing every soul immortal. I myself too will use the secret acknowledgment of the commonalty, bearing record of the God of gods. But when I hear men allege, ‘That which is dead is dead;’ and, ‘While thou art alive be alive;’ and, ‘After death an end of all, even of death itself;’ then will I call to mind both that the heart of the people with God is accounted dust, and that the very wisdom of the world is pronounced folly. If then an heretic fly also unto such vicious popular and secular conceits, my answer unto him shall be, ‘Thou heretic, avoid the heathen; although in this ye be one, that ye both belie God, yet thou that doest this under the name of Christ, differest from the heathen, in that thou seemest to thyself a Christian. Leave him therefore his conceits, seeing that neither will he learn thine. Why dost thou having sight trust to a blind guide; thou which hast put on Christ take raiment of him that is naked? If the Apostle have armed thee, why dost thou borrow a stranger’s shield? Let him rather learn of thee to acknowledge, than thou of him to renounce the resurrection of the flesh.’ ” In a word, the Catholic Fathers did good  unto all by that knowledge, whereby heretics hindering the truth in many, might have furthered therewith themselves, but that obstinately following their own ambitious or otherwise corrupted affections, instead of framing their wills to maintain that which reason taught, they bent their wits to find how reason might seem to teach that which their wills were set to maintain. For which cause the Apostle saith of them justly, that they are for the most part αὐτοκατάκριτοι, men condemned even in and of themselves. For though they be not all persuaded that it is truth which they withstand, yet that to be error which they uphold they might undoubtedly the sooner a great deal attain to know, but that their study is more to defend what once they have stood in, than to find out sincerely and simply what truth they ought to persist in for ever.

[9.]Fourthly, There is in the world no kind of knowledge, whereby any part of truth is seen, but we justly account it precious; yea, that principal truth, in comparison whereof all other knowledge is vile, may receive from it some kind of light; whether it be that Egyptian and Chaldean wisdom mathematical, wherewith Moses and Daniel were furnished; or that natural, moral, and civil wisdom, wherein Salomon excelled all men; or that rational and oratorial wisdom of the Grecians, which the Apostle St. Paul brought from Tarsus; or that Judaical, which he learned in Jerusalem sitting at the feet of Gamaliel: to detract from the dignity thereof were to injury even God himself, who being that light which none can approach unto, hath sent out these lights whereof we are capable, even as so many sparkles resembling the bright fountain from which they rise.

But there are that bear the title of wise men and scribes and great disputers of the world, and are nothing in deed less than what in show they most appear. These being wholly addicted unto their own wills, use their wit, their learning, and all the wisdom they have, to maintain that which their  obstinate hearts are delighted with, esteeming in the frantic error of their minds the greatest madness in the world to be wisdom, and the highest wisdom foolishness. Such were both Jews and Grecians, which professed the one sort legal, and the other secular skill, neither enduring to be taught the mystery of Christ: unto the glory of whose most blessed name, whoso study to use both their reason and all other gifts, as well which nature as which grace hath endued them with, let them never doubt but that the same God who is to destroy and confound utterly that wisdom falsely so named in others, doth make reckoning of them as of true Scribes, Scribes by wisdom instructed to the kingdom of heaven, not Scribes against that kingdom hardened in a vain opinion of wisdom; which in the end being proved folly, must needs perish, true understanding, knowledge, judgment and reason continuing for evermore.

[10.]Fifthly, Unto the word of God, being in respect of that end for which God ordained it perfect, exact, and absolute in itself, we do not add reason as a supplement of any maim or defect therein, but as a necessary instrument, without which we could not reap by the Scripture’s perfection that fruit and benefit which it yieldeth. “The word of God is a twoedged sword,” but in the hands of reasonable men; and reason as the weapon that slew Goliath, if they be as David was that use it. Touching the Apostles, He which gave them from above such power for miraculous confirmation of that which they taught, endued them also with wisdom from above to teach that which they so did confirm. Our Saviour made choice of twelve simple and unlearned men, that the greater their lack of natural wisdom was, the more admirable that might appear which God supernaturally endued them with from heaven. Such therefore as knew the poor and silly estate wherein they had lived, could not but wonder to hear the wisdom of their speech, and be so much the more attentive unto their teaching. They studied for no tongue, they spake with all; of themselves they were rude, and knew not so much as how to premeditate; the Spirit gave them speech and eloquent utterance.

But because with St. Paul it was otherwise than with the  rest, inasmuch as he never conversed with Christ upon earth as they did; and his education had been scholastical altogether, which theirs was not; hereby occasion was taken by certain malignants, secretly to undermine his great authority in the Church of Christ, as though the gospel had been taught him by others than by Christ himself, and as if the cause of the Gentiles’ conversion and belief through his means had been the learning and skill which he had by being conversant in their books; which thing made them so willing to hear him, and him so able to persuade them; whereas the rest of the Apostles prevailed, because God was with them, and by miracle from heaven confirmed his word in their mouths. They were mighty in deeds: as for him, being absent, his writings had some force; in presence, his power not like unto theirs. In sum, concerning his preaching, their very byword was, λόγος ἐξουθενημένος, addle speech, empty talk: his writings full of great words, but in the power of miraculous operations his presence not like the rest of the Apostles.

Hereupon it riseth that St. Paul was so often driven to make his apologies. Hereupon it riseth that whatsoever time he had spent in the study of human learning, he maketh earnest protestation to them of Corinth, that the gospel which he had preached amongst them did not by other means prevail with them, than with others the same gospel taught by the rest of the Apostles of Christ. “My preaching,” saith he, “hath not been in the persuasive speeches of human wisdom, but in demonstration of the Spirit and of power: that your faith may not be in the wisdom of men, but in the power of God.” What is it which the Apostle doth here deny? Is it denied that his speech amongst them had been persuasive? No: for of him the sacred history plainly testifieth, that for the space of a year and a half he spake in their synagogue every Sabbath, and persuaded both Jews and Grecians. How then is the speech of men made persuasive? Surely there can be but two ways to bring this to pass, the one human, the other divine. Either St. Paul did only by art and natural industry cause his own speech to be credited; or else God by  miracle did authorize it, and so bring credit thereunto, as to the speech of the rest of the Apostles. Of which two, the former he utterly denieth. For why? if the preaching of the rest had been effectual by miracle, his only by force of his own learning; so great inequality between him and the other Apostles in this thing had been enough to subvert their faith. For might they not with reason have thought, that if he were sent of God as well as they, God would not have furnished them and not him with the power of the Holy Ghost? Might not a great part of them being simple haply have feared, lest their assent had been cunningly gotten unto his doctrine, rather through the weakness of their own wits than the certainty of that truth which he had taught them? How unequal had it been that all believers through the preaching of other Apostles should have their faith strongly built upon the evidence of God’s own miraculous approbation, and they whom he had converted should have their persuasion built only upon his skill and wisdom who persuaded them?

As therefore calling from men may authorize us to teach, although it could not authorize him to teach as other Apostles did: so although the wisdom of man had not been sufficient to enable him such a teacher as the rest of the apostles were, unless God’s miracles had strengthened both the one and the other’s doctrine; yet unto our ability both of teaching and learning the truth of Christ, as we are but mere Christian men, it is not a little which the wisdom of man may add.


[11.]Sixthly, Yea, whatsoever our hearts be to God and to his truth, believe we or be we as yet faithless, for our conversion or confirmation the force of natural reason is great. The force whereof unto those effects is nothing without grace. What then? To our purpose it is sufficient, that whosoever doth serve, honour, and obey God, whosoever believeth in Him, that man would no more do this than innocents and infants do, but for the light of natural reason that shineth in him, and maketh him apt to apprehend those things of God, which being by grace discovered, are effectual to persuade reasonable minds and none other, that honour, obedience, and credit, belong of right unto God. No man cometh unto God to offer him sacrifice, to pour out supplications and prayers before him, or to do him any service, which doth not first believe him both to be, and to be a rewarder of them who in such sort seek unto him. Let men be taught this either by revelation from heaven, or by instruction upon earth; by labour, study, and meditation, or by the only secret inspiration of the Holy Ghost; whatsoever the mean be they know it by, if the knowledge thereof were possible without discourse of natural reason, why should none be found capable thereof but only men; nor men till such time as they come unto ripe and full ability to work by reasonable understanding? The whole drift of the Scripture of God, what is it but only to teach Theology? Theology, what is it but the science of things divine? What science can be attained unto without the help of natural discourse and reason? “Judge you of that which I speak,” saith the Apostle. In vain it were to speak any thing of God, but that by reason men are able somewhat to judge of that they hear, and by discourse to discern how consonant it is to truth.

[12.]Scripture indeed teacheth things above nature, things  which our reason by itself could not reach unto. Yet those things also we believe, knowing by reason that the Scripture is the word of God. In the presence of Festus a Roman, and of King Agrippa a Jew, St. Paul omitting the one, who neither knew the Jews’ religion nor the books whereby they were taught it, speaketh unto the other of things foreshewed by Moses and the Prophets and performed in Jesus Christ; intending thereby to prove himself so unjustly accused, that unless his judges did condemn both Moses and the Prophets, him they could not choose but acquit, who taught only that fulfilled, which they so long since had foretold. His cause was easy to be discerned; what was done their eyes were witnesses; what Moses and the Prophets did speak their books could quickly shew; it was no hard thing for him to compare them, which knew the one, and believed the other. “King Agrippa, believest thou the Prophets? I know thou dost.” The question is how the books of the Prophets came to be credited of King Agrippa. For what with him did authorize the Prophets, the like with us doth cause the rest of the Scripture of God to be of credit.

[13.]Because we maintain that in Scripture we are taught all things necessary unto salvation; hereupon very childishly it is by some demanded, what Scripture can teach us the sacred authority of the Scripture, upon the knowledge whereof our whole faith and salvation dependeth? As though there were any kind of science in the world which leadeth men into knowledge without presupposing a number of things already known. No science doth make known the first principles whereon it buildeth, but they are always either taken as plain and manifest in themselves, or as proved and granted already, some former knowledge having made them evident. Scripture teacheth all supernatural revealed truth, without the knowledge whereof salvation cannot be attained. The main principle whereupon our belief of all things therein contained dependeth, is, that the Scriptures are the oracles of God himself. This in itself we cannot say is evident. For then all men that hear it would acknowledge it in heart, as they do when they hear that “every whole is more than any part of that whole,” because this in itself is evident. The  other we know that all do not acknowledge when they hear it. There must be therefore some former knowledge presupposed which doth herein assure the hearts of all believers. Scripture teacheth us that saving truth which God hath discovered unto the world by revelation, and it presumeth us taught otherwise that itself is divine and sacred.

[14.]The question then being by what means we are taught this; some answer that to learn it we have no other way than only tradition; as namely that so we believe because both we from our predecessors and they from theirs have so received. But is this enough? That which all men’s experience teacheth them may not in any wise be denied. And by experience we all know, that the first outward motive leading men so to esteem of the Scripture is the authority of God’s Church. For when we know the whole Church of God hath that opinion of the Scripture, we judge it even at the first an impudent thing for any man bred and brought up in the Church to be of a contrary mind without cause. Afterwards the more we bestow our labour in reading or hearing the mysteries thereof, the more we find that the thing itself doth answer our received opinion concerning it. So that the former inducement prevailing somewhat with us before, doth now much more prevail, when the very thing hath ministered farther reason. If infidels or atheists  chance at any time to call it in question, this giveth us occasion to sift what reason there is, whereby the testimony of the Church concerning Scripture, and our own persuasion which Scripture itself hath confirmed, may be proved a truth infallible. In which case the ancient Fathers being often constrained to shew, what warrant they had so much to rely upon the Scriptures, endeavoured still to maintain the authority of the books of God by arguments such as unbelievers themselves must needs think reasonable, if they judged thereof as they should. Neither is it a thing impossible or greatly hard, even by such kind of proofs so to manifest and clear that point, that no man living shall be able to deny it, without denying some apparent principle such as all men acknowledge to be true.

Wherefore if I believe the Gospel, yet is reason of singular use, for that it confirmeth me in this my belief the more: if I do not as yet believe, nevertheless to bring me to the number of believers except reason did somewhat help, and were an instrument which God doth use unto such purposes, what should it boot to dispute with infidels or godless persons for their conversion and persuasion in that point?

[15.]Neither can I think that when grave and learned men do sometime hold, that of this principle there is no proof but by the testimony of the Spirit, which assureth our hearts therein, it is their meaning to exclude utterly all force which any kind of reason may have in that behalf; but I rather incline to interpret such their speeches, as if they had more expressly set down, that other motives and inducements, be they never so strong and consonant unto reason, are notwithstanding uneffectual of themselves to work faith concerning this principle, if the special grace of the Holy Ghost concur not to the enlightening of our minds. For otherwise I doubt not but men of wisdom and judgment will grant, that the Church, in this point especially, is furnished with reason, to stop the mouths of her impious adversaries; and that as it were altogether bootless to allege against them what the Spirit hath taught us, so likewise that even to our ownselves it needeth caution and explication how the testimony of the Spirit may be discerned, by what means it may be known; lest men think that the Spirit of God doth testify those things  which the Spirit of error suggesteth. The operations of the Spirit, especially these ordinary which be common unto all true Christian men, are as we know things secret and undiscernible even to the very soul where they are, because their nature is of another and an higher kind than that they can be by us perceived in this life. Wherefore albeit the Spirit lead us into all truth and direct us in all goodness, yet because these workings of the Spirit in us are so privy and secret, we therefore stand on a plainer ground, when we gather by reason from the quality of things believed or done, that the Spirit of God hath directed us in both, than if we settle ourselves to believe or to do any certain particular thing, as being moved thereto by the Spirit.

[16.]But of this enough. To go from the books of Scripture to the sense and meaning thereof: because the sentences which are by the Apostles recited out of the Psalms, to prove the resurrection of Jesus Christ, did not prove it, if so be the Prophet David meant them of himself; this exposition therefore they plainly disprove, and shew by manifest reason, that of David the words of David could not possibly be meant. Exclude the use of natural reasoning about the sense of Holy Scripture concerning the articles of our faith, and then that the Scripture doth concern the articles of our faith who can assure us? That, which by right exposition buildeth up Christian faith, being misconstrued breedeth error: between true and false construction, the difference reason must shew. Can Christian men perform that which Peter requireth at their hands; is it possible they should both believe and be able, without the use of reason, to render “a reason of their belief,” a reason sound and sufficient to answer them that demand it, be they of the same faith with us or enemies thereunto? may we cause our faith without reason to appear reasonable in the eyes of men? This being required even of learners in the school of Christ, the duty of their teachers in bringing them unto such ripeness must needs be somewhat more, than only to read the sentences of Scripture, and then paraphrastically to scholy them: to vary them with sundry forms of speech, without arguing or disputing about any thing which they contain. This method of teaching may  commend itself unto the world by that easiness and facility which is in it: but a law or a pattern it is not, as some do imagine, for all men to follow that will do good in the Church of Christ.

[17.]Our Lord and Saviour himself did hope by disputation to do some good, yea by disputation not only of but against, the truth, albeit with purpose for the truth. That Christ should be the son of David was truth; yet against this truth our Lord in the gospel objecteth, “If Christ be the son of David, how doth David call him Lord?” There is as yet no way known how to dispute, or to determine of things disputed, without the use of natural reason.

If we please to add unto Christ their example, who followed him as near in all things as they could; the sermon of Paul and Barnabas set down in the Acts, where the people would have offered unto them sacrifice; in that sermon what is there but only natural reason to disprove their act? “O men, why do you these things? We are men even subject to the selfsame passions with you: we preach unto you to leave these vanities and to turn to the living God, the God that hath not left himself without witness, in that he hath done good to the world, giving rain and fruitful seasons, filling our heart with joy and gladness.”

Neither did they only use reason in winning such unto Christian belief as were yet thereto unconverted, but with believers themselves they followed the selfsame course. In that great and solemn assembly of believing Jews how doth Peter prove that the Gentiles were partakers of the grace of God as well as they, but by reason drawn from those effects, which were apparently known amongst them? “God which knoweth hearts hath borne them witness in giving unto them the Holy Ghost as unto us.”

The light therefore, which the “star of natural reason” and wisdom casteth, is too bright to be obscured by the mist of a word or two uttered to diminish that opinion which justly hath been received concerning the force and virtue thereof, even in matters that touch most nearly the principal duties of men and the glory of the eternal God.

[18.]In all which hitherto hath been spoken touching the  force and use of man’s reason in things divine, I must crave that I be not so understood or construed, as if any such thing by virtue thereof could be done without the aid and assistance of God’s most blessed Spirit. The thing we have handled according to the question moved about it; which question is, whether the light of reason be so pernicious, that in devising laws for the Church men ought not by it to search what may be fit and convenient. For this cause therefore we have endeavoured to make it appear, how in the nature of reason itself there is no impediment, but that the selfsame Spirit, which revealeth the things that God hath set down in his law, may also be thought to aid and direct men in finding out by the light of reason what laws are expedient to be made for the guiding of his Church, over and besides them that are in Scripture. Herein therefore we agree with those men, by whom human laws are defined to be ordinances, which such as have lawful authority given them for that purpose do probably draw from the laws of nature and God, by discourse of reason aided with the influence of divine grace. And for that cause, it is not said amiss touching ecclesiastical canons, that “by instinct of the Holy Ghost they have been made, and consecrated by the reverend acceptation of all the world.”

\section*{How Laws for the Polity of the Church may be made by the advice of men, and how those Laws being not repugnant to the Word of God are approved in his sight.}

IX. Laws for the Church are not made as they should be, unless the makers follow such direction as they ought to be guided by: wherein that Scripture standeth not the Church of God in any stead, or serveth nothing at all to direct, but may be let pass as needless to be consulted with, we judge it profane, impious, and irreligious to think. For although it were in vain to make laws which the Scripture hath already made, because what we are already there commanded to do, on our parts there resteth nothing but only that it be executed; yet because both in that which we are commanded, it concerneth the duty of the Church by law to provide, that the looseness and slackness of men may not cause the commandments of God to be unexecuted; and a number of things there are for which the Scripture hath not provided by any law,  but left them unto the careful discretion of the Church; we are to search how the Church in these cases may be well directed to make that provision by laws which is most convenient and fit. And what is so in these cases, partly Scripture and partly reason must teach to discern. Scripture comprehending examples and laws, laws some natural and some positive: examples there neither are for all cases which require laws to be made, and when there are, they can but direct as precedents only. Natural laws direct in such sort, that in all things we must for ever do according unto them; Positive so, that against them in no case we may do any thing, as long as the will of God is that they should remain in force. Howbeit when Scripture doth yield us precedents, how far forth they are to be followed; when it giveth natural laws, what particular order is thereunto most agreeable; when positive, which way to make laws unrepugnant unto them; yea though all these should want, yet what kind of ordinances would be most for that good of the Church which is aimed at, all this must be by reason found out. And therefore, “to refuse the conduct of the light of nature,” saith St. Augustine, “is not folly alone but accompanied with impiety.”

[2.]The greatest amongst the School-divines, studying how to set down by exact definition the nature of an human law, (of which nature all the Church’s constitutions are,) found not which way better to do it than in these words: “Out of the precepts of the law of nature, as out of certain common and undemonstrable principles, man’s reason doth necessarily proceed unto certain more particular determinations; which particular determinations being found out according unto the reason of man, they have the names of human laws, so that such other conditions be therein kept as the making of laws doth require,” that is, if they whose authority is thereunto required do establish and publish them as laws. And  the truth is, that all our controversy in this cause concerning the orders of the Church is, what particulars the Church may appoint. That which doth find them out is the force of man’s reason. That which doth guide and direct his reason is first the general law of nature; which law of nature and the moral law of Scripture are in the substance of law all one. But because there are also in Scripture a number of laws particular and positive, which being in force may not by any law of man be violated; we are in making laws to have thereunto an especial eye. As for example, it might perhaps seem reasonable unto the Church of God, following the general laws concerning the nature of marriage, to ordain in particular that cousin-germans shall not marry. Which law notwithstanding ought not to be received in the Church, if there should be in Scripture a law particular to the contrary, forbidding utterly the bonds of marriage to be so far forth abridged. The same Thomas therefore whose definition of human laws we mentioned before, doth add thereunto this caution concerning the rule and canon whereby to make them: human laws are measures in respect of men whose actions they must direct; howbeit such measures they are, as have also their higher rules to be measured by, which rules are two, the law of God, and the law of nature. So that laws human must be made according to the general laws of nature, and without contradiction unto any positive law in Scripture. Otherwise they are ill made.

[3.]Unto laws thus made and received by a whole church, they which live within the bosom of that church must not think it a matter indifferent either to yield or not to yield obedience. Is it a small offence to despise the Church of God? “My son keep thy father’s commandment,” saith Salomon, “and forget not thy mother’s instruction: bind them both always about thine heart.” It doth not stand with the duty which we owe to our heavenly Father, that to the ordinances of our mother the Church we should shew ourselves disobedient. Let us not say we keep the commandments of the one, when we break the law of the other: for  unless we observe both, we obey neither. And what doth let but that we may observe both, when they are not the one to the other in any sort repugnant? For of such laws only we speak, as being made in form and manner already declared, can have in them no contradiction unto the laws of Almighty God. Yea that which is more, the laws thus made God himself doth in such sort authorize, that to despise them is to despise in them Him. It is a loose and licentious opinion which the Anabaptists have embraced, holding that a Christian man’s liberty is lost, and the soul which Christ hath redeemed unto himself injuriously drawn into servitude under the yoke of human power, if any law be now imposed besides the Gospel of Jesus Christ: in obedience whereunto the Spirit of God and not the constraint of man is to lead us, according to that of the blessed Apostle, “Such as are led by the Spirit of God they are the sons of God,” and not such as live in thraldom unto men. Their judgment is therefore that the Church of Christ should admit no law-makers but the Evangelists. The author of that which causeth another thing to be, is author of that thing also which thereby is caused. The light of natural understanding, wit, and reason, is from God; he it is which thereby doth illuminate every man entering into the world. If there proceed from us any thing afterwards corrupt and naught, the mother thereof is our own darkness, neither doth it proceed from any such cause whereof God is the author. He is the author of all that we think or do by virtue of that light, which himself hath given. And therefore the laws which the very heathens did gather to direct their actions by, so far forth as they proceeded from the light of nature, God himself doth acknowledge to have proceeded even from himself, and that he was the writer of them in the tables of their hearts. How much more then he the author of those laws, which have been made by his saints, endued further with the heavenly grace of his Spirit, and directed as much as might be with such instructions as his sacred word doth yield! Surely if we have unto those laws that dutiful regard which their dignity doth require, it will not greatly need that we should be exhorted to live in obedience unto them. If they have God himself for their  author, contempt which is offered unto them cannot choose but redound unto him. The safest and unto God the most acceptable way of framing our lives therefore is, with all humility, lowliness, and singleness of heart, to study, which way our willing obedience both unto God and man may be yielded even to the utmost of that which is due.

\section*{That neither God’s being the Author of Laws, nor yet his committing of them to Scripture, is any reason sufficient to prove that they admit no addition or change.}

X. Touching the mutability of laws that concern the regiment and polity of the Church; changed they are, when either altogether abrogated, or in part repealed, or augmented with farther additions. Wherein we are to note, that this question about the changing of laws concerneth only such laws as are positive, and do make that now good or evil by being commanded or forbidden, which otherwise of itself were not simply the one or the other. Unto such laws it is expressly sometimes added, how long they are to continue in force. If this be nowhere exprest, then have we no light to direct our judgments concerning the changeableness or immutability of them, but by considering the nature and quality of such laws. The nature of every law must be judged of by the end for which it was made, and by the aptness of things therein prescribed unto the same end. It may so fall out that the reason why some laws of God were given is neither opened nor possible to be gathered by wit of man. As why God should forbid Adam that one tree, there was no way for Adam ever to have certainly understood. And at Adam’s ignorance of this point Satan took advantage, urging the more securely a false cause because the true was unto Adam unknown. Why the Jews were forbidden to plough their ground with an ox and an ass, why to clothe themselves with mingled attire of wool and linen, both it was unto them and to us it remaineth obscure. Such laws perhaps cannot be abrogated saving only by whom they were made: because the intent of them being known unto none but the author, he alone can judge how long it is requisite they should endure. But if the reason why things were instituted may be known, and being known do appear manifestly to be of perpetual necessity; then are those things also perpetual, unless they  cease to be effectual unto that purpose for which they were at the first instituted. Because when a thing doth cease to be available unto the end which gave it being, the continuance of it must then of necessity appear superfluous. And of this we cannot be ignorant, how sometimes that hath done great good, which afterwards, when time hath changed the ancient course of things, doth grow to be either very hurtful, or not so greatly profitable and necessary. If therefore the end for which a law provideth be perpetually necessary, and the way whereby it provideth perpetually also most apt, no doubt but that every such law ought for ever to remain unchangeable.

[2.]Whether God be the author of laws by authorizing that power of men whereby they are made, or by delivering them made immediately from himself, by word only, or in writing also, or howsoever; notwithstanding the authority of their Maker, the mutability of that end for which they are made doth also make them changeable. The law of ceremonies came from God: Moses had commandment to commit it unto the sacred records of Scripture, where it continueth even unto this very day and hour: in force still, as the Jew surmiseth, because God himself was author of it, and for us to abolish what he hath established were presumption most intolerable. But (that which they in the blindness of their obdurate hearts are not able to discern) sith the end for which that law was ordained is now fulfilled, past and gone; how should it but cease any longer to be, which hath no longer any cause of being in force as before? “That which necessity of some special time doth cause to be enjoined bindeth no longer than during that time, but doth afterwards become free.”

Which thing is also plain even by that law which the Apostles assembled at the council of Jerusalem did from thence deliver unto the Church of Christ, the preface whereof to authorize it was, “To the Holy Ghost and to us it hath seemed good:” which style they did not use as matching themselves in power with the Holy Ghost, but as testifying  the Holy Ghost to be the author, and themselves but only utterers of that decree. This law therefore to have proceeded from God as the author thereof no faithful man will deny. It was of God, not only because God gave them the power whereby they might make laws, but for that it proceeded even from the holy motion and suggestion of that secret divine Spirit, whose sentence they did but only pronounce. Notwithstanding, as the law of ceremonies delivered unto the Jews, so this very law which the Gentiles received from the mouth of the Holy Ghost, is in like respect abrogated by decease of the end for which it was given.

[3.]But such as do not stick at this point, such as grant that what hath been instituted upon any special cause needeth not to be observed, that cause ceasing, do notwithstanding herein fail; they judge the laws of God only by the author and main end for which they were made, so that for us to change that which he hath established, they hold it execrable pride and presumption, if so be the end and purpose for which God by that mean provideth be permanent. And upon this they ground those ample disputes concerning orders and offices, which being by him appointed for the government of his Church, if it be necessary always that the Church of Christ be governed, then doth the end for which God provided remain still; and therefore in those means which he by law did establish as being fittest unto that end, for us to alter any thing is to lift up ourselves against God, and as it were to countermand him. Wherein they mark not that laws are instruments to rule by, and that instruments are not only to be framed according unto the general end for which they are provided, but even according unto that very particular, which riseth out of the matter whereon they have to work.  The end wherefore laws were made may be permanent, and those laws nevertheless require some alteration, if there be any unfitness in the means which they prescribe as tending unto that end and purpose. As for example, a law that to bridle theft doth punish thieves with a quadruple restitution hath an end which will continue as long as the world itself continueth. Theft will be always, and will always need to be bridled. But that the mean which this law provideth for that end, namely the punishment of quadruple restitution, that this will be always sufficient to bridle and restrain that kind of enormity no man can warrant. Insufficiency of laws doth sometimes come by want of judgment in the makers. Which cause cannot fall into any law termed properly and immediately divine, as it may and doth into human laws often. But that which hath been once most sufficient may wax otherwise by alteration of time and place; that punishment which hath been sometime forcible to bridle sin may grow afterwards too weak and feeble.

[4.]In a word, we plainly perceive by the difference of those three laws which the Jews received at the hands of God, the moral, ceremonial, and judicial, that if the end for which and the matter according whereunto God maketh his laws continue always one and the same, his laws also do the like; for which cause the moral law cannot be altered: secondly, that whether the matter whereon laws are made continue or continue not, if their end have once ceased, they cease also to be of force; as in the law ceremonial it fareth: finally, that albeit the end continue, as in that law of theft specified and in a great part of those ancient judicials it doth; yet forasmuch as there is not in all respects the same subject or matter remaining for which they were first instituted, even this is sufficient cause of change: and therefore laws, though both ordained of God himself, and the end for which they were ordained continuing, may notwithstanding cease, if by alteration of persons or times they be found unsufficient to attain unto that end. In which respect why may we not presume that God doth even call for such change or alteration as the very condition of things themselves doth make necessary?

[5.]They which do therefore plead the authority of the law-maker as an argument, wherefore it should not be lawful to change that which he hath instituted, and will have this the cause why all the ordinances of our Saviour are immutable; they which urge the wisdom of God as a proof, that whatsoever laws he hath made they ought to stand, unless himself from heaven proclaim them disannulled, because it is not in man to correct the ordinance of God; may know, if it please them to take notice thereof, that we are far from presuming to think that men can better any thing which God hath done, even as we are from thinking that men should presume to undo some things of men, which God doth know they cannot better. God never ordained any thing that could be bettered. Yet many things he hath that have been changed, and that for the better. That which succeedeth as better now when change is requisite, had been worse when that which now is changed was instituted. Otherwise God had not then left this to choose that, neither would now reject that to choose this, were it not for some new-grown occasion making that which hath been better worse. In this case therefore men do not presume to change God’s ordinance, but they yield thereunto requiring itself to be changed.

[6.]Against this it is objected, that to abrogate or innovate the Gospel of Christ if men or angels should attempt, it were most heinous and cursed sacrilege. And the Gospel (as they say) containeth not only doctrine instructing men how they should believe, but also precepts concerning the regiment of the Church. Discipline therefore is “a part of the Gospel;” and God being the author of the whole Gospel, as well of discipline as of doctrine, it cannot be but that both of them “have a common cause.” So that as we are to believe for ever the articles of evangelical doctrine, so the precepts of discipline we are in like sort bound for ever to observe.

[7.]Touching points of doctrine, as for example, the Unity  of God, the Trinity of Persons, salvation by Christ, the resurrection of the body, life everlasting, the judgment to come, and such like, they have been since the first hour that there was a Church in the world, and till the last they must be believed. But as for matters of regiment, they are for the most part of another nature. To make new articles of faith and doctrine no man thinketh it lawful; new laws of government what commonwealth or church is there which maketh not either at one time or another? “The rule of faith,” saith Tertullian, “is but one, and that alone immoveable and impossible to be framed or cast anew.” The law of outward order and polity not so. There is no reason in the world wherefore we should esteem it as necessary always to do, as always to believe, the same things; seeing every man knoweth that the matter of faith is constant, the matter contrariwise of action daily changeable, especially the matter of action belonging unto church polity. Neither can I find that men of soundest judgment have any otherwise taught, than that articles of belief, and things which all men must of necessity do to the end they may be saved, are either expressly set down in Scripture, or else plainly thereby to be gathered. But touching things which belong to discipline and outward polity, the Church hath authority to make canons, laws, and decrees, even as we read that in the Apostles’ times it did. Which kind of laws (forasmuch as they are not in themselves necessary to salvation) may after they are made be also changed as the difference of times or places shall require. Yea, it is not denied I am sure by themselves, that certain things in discipline are of that nature, as they may be varied by times, places, persons, and other the like circumstances. Whereupon I demand, are those changeable points of discipline commanded in the word of God or no? If they be not commanded and yet may be  received in the Church, how can their former position stand, condemning all things in the Church which in the word are not commanded? If they be commanded and yet may suffer change, how can this latter stand, affirming all things immutable which are commanded of God? Their distinction touching matters of substance and of circumstance, though true, will not serve. For be they great things or be they small, if God have commanded them in the Gospel, and his commanding them in the Gospel do make them unchangeable, there is no reason we should more change the one than we may the other. If the authority of the maker do prove unchangeableness in the laws which God hath made, then must all laws which he hath made be necessarily for ever permanent, though they be but of circumstance only and not of substance. I therefore conclude, that neither God’s being author of laws for government of his Church, nor his committing them unto Scripture, is any reason sufficient wherefore all churches should for ever be bound to keep them without change.

[8.]But of one thing we are here to give them warning by the way. For whereas in this discourse we have oftentimes profest that many parts of discipline or church polity are delivered in Scripture, they may perhaps imagine that we are driven to confess their discipline to be delivered in Scripture, and that having no other means to avoid it, we are fain to argue for the changeableness of laws ordained even by God himself, as if otherwise theirs of necessity should take place, and that under which we live be abandoned. There is no remedy therefore but to abate this error in them, and directly to let them know, that if they fall into any such conceit, they do but a little flatter their own cause. As for us, we think in no respect so highly of it. Our persuasion is, that no age ever had knowledge of it but only ours; that they which defend it devised it; that neither Christ nor his Apostles at any time taught it, but the contrary. If therefore we did seek to maintain that which most advantageth our own cause, the very best way for us and the strongest against them were to hold even as they do, that in Scripture there must needs be found some particular form of church polity which God hath instituted, and which for that very  cause belongeth to all churches, to all times. But with any such partial eye to respect ourselves, and by cunning to make those things seem the truest which are the fittest to serve our purpose, is a thing which we neither like nor mean to follow. Wherefore that which we take to be generally true concerning the mutability of laws, the same we have plainly delivered, as being persuaded of nothing more than we are of this, that whether it be in matter of speculation or of practice, no untruth can possibly avail the patron and defender long, and that things most truly are likewise most behovefully spoken.

\section*{Whether Christ must needs intend Laws unchangeable altogether, or have forbidden any where to make any other Law than himself did deliver.}

XI. This we hold and grant for truth, that those very laws which of their own nature are changeable, be notwithstanding uncapable of change, if he which gave them, being of authority so to do, forbid absolutely to change them; neither may they admit alteration against the will of such a law-maker. Albeit therefore we do not find any cause why of right there should be necessarily an immutable form set down in holy Scripture; nevertheless if indeed there have been at any time a church polity so set down, the change whereof the sacred Scripture doth forbid, surely for men to alter those laws which God for perpetuity hath established were presumption most intolerable.

[2.]To prove therefore that the will of Christ was to establish laws so permanent and immutable that in any sort to alter them cannot but highly offend God, thus they reason. First, if Moses, being but a servant in the house of God,  did therein establish laws of government for perpetuity, laws which they that were of the household might not alter; shall we admit into our thoughts, that the Son of God hath in providing for this his household declared himself less faithful than Moses? Moses delivering unto the Jews such laws as were durable, if those be changeable which Christ hath delivered unto us, we are not able to avoid it, but (that which to think were heinous impiety) we of necessity must confess even the Son of God himself to have been less faithful than Moses. Which argument shall need no touchstone to try it by but some other of the like making. Moses erected in the wilderness a tabernacle which was moveable from place to place; Salomon a sumptuous and stately temple which was not moveable: therefore Salomon was faithfuller than Moses, which no man endued with reason will think. And yet by this reason it doth plainly follow.

He that will see how faithful the one or the other was, must compare the things which they both did unto the charge which God gave each of them. The Apostle in making comparison between our Saviour and Moses attributeth faithfulness unto both, and maketh this difference between them; Moses in, but Christ over the house of God; Moses in that house which was his by charge and commission, though to govern it, yet to govern it as a servant; but Christ over this house as being his own entire possession.

[3.]Our Lord and Saviour doth make protestation, “I have given unto them the words which thou gavest me.” Faithful therefore he was, and concealed not any part of his Father’s will. But did any part of that will require the immutability of laws concerning church polity? They answer, Yea. For else God should less favour us than the Jews. God would not have their church guided by any laws but his  own. And seeing this did so continue even till Christ, now to ease God of that care, or rather to deprive the Church of his patronage, what reason have we? Surely none to derogate any thing from the ancient love which God hath borne to his Church. An heathen philosopher there is, who considering how many things beasts have which men have not, how naked in comparison of them, how impotent, and how much less able we are to shift for ourselves a long time after we enter into this world, repiningly concluded hereupon, that nature being a careful mother for them, is towards us a hard-hearted stepdame. No, we may not measure the affection of our gracious God towards his by such differences. For even herein shineth his wisdom, that though the ways of his providence be many, yet the end which he bringeth all at the length unto is one and the selfsame.

[4.]But if such kind of reasoning were good, might we not even as directly conclude the very same concerning laws of secular regiment? Their own words are these: “In the ancient church of the Jews, God did command and Moses commit unto writing all things pertinent as well to the civil as to the ecclesiastical state.” God gave them laws of civil regiment, and would not permit their commonweal to be governed by any other laws than his own. Doth God less regard our temporal estate in this world, or provide for it  worse than for theirs? To us notwithstanding he hath not as to them delivered any particular form of temporal regiment, unless perhaps we think, as some do, that the grafting of the Gentiles and their incorporating into Israel doth import that we ought to be subject unto the rites and laws of their whole polity. We see then how weak such disputes are, and how smally they make to this purpose.

[5.]That Christ did not mean to set down particular positive laws for all things in such sort as Moses did, the very different manner of delivering the laws of Moses and the laws of Christ doth plainly shew. Moses had commandment to gather the ordinances of God together distinctly, and orderly to set them down according unto their several kinds, for each public duty and office the laws that belong thereto, as appeareth in the books themselves, written of purpose for that end. Contrariwise the laws of Christ we find rather mentioned by occasion in the writings of the Apostles, than any solemn thing directly written to comprehend them in legal sort.

[6.]Again, the positive laws which Moses gave, they were given for the greatest part with restraint to the land of Jewry: “Behold,” saith Moses, “I have taught you ordinances and laws, as the Lord my God commanded me, that ye should do even so within the land whither ye go to possess it.” Which laws and ordinances positive he plainly distinguisheth afterward from the laws of the Two Tables which were moral. “The Lord spake unto you out of the midst of the fire; ye heard the voice of the words, but saw no similitude, only a voice. Then he declared unto you his covenant which he commanded you to do, the Ten Commandments, and wrote them upon two tables of stone. And the Lord commanded me that same time, that I should teach you ordinances and laws which ye should observe in the land whither ye go to possess it.” The same difference is again set down in the next chapter following. For rehearsal being made of the Ten Commandments, it followeth immediately, “These words the Lord spake unto all your multitude in the mount out of the midst of the fire, the cloud, and the darkness, with a great voice, and added no  more; and wrote them upon two tables of stone, and delivered them unto me.” But concerning other laws, the people give their consent to receive them at the hands of Moses: “Go thou near, and hear all that the Lord our God saith, and declare thou unto us all that the Lord our God saith unto thee, and we will hear it and do it.” The people’s alacrity herein God highly commendeth with most effectual and hearty speech: “I have heard the voice of the words of this people; they have spoken well. O that there were such an heart in them to fear me, and to keep all my commandments always, that it might go well with them and with their children for ever! Go, say unto them, ‘Return you to your tents;’ but stand thou here with me, and I will tell thee all the commandments and the ordinances and the laws which thou shalt teach them, that they may do them in the land which I have given them to possess.” From this later kind the former are plainly distinguished in many things. They were not both at one time delivered, neither both after one sort, nor to one end. The former uttered by the voice of God himself in the hearing of six hundred thousand men; the former written with the finger of God; the former termed by the name of a Covenant; the former given to be kept without either mention of time how long, or of place where. On the other side, the later given after, and neither written by God himself, nor given unto the whole multitude immediately from God, but unto Moses, and from him to them both by word and writing; the later termed Ceremonies, Judgments, Ordinances, but no where Covenants; finally, the observation of the later restrained unto the land where God would establish them to inhabit.

The laws positive are not framed without regard had to the place and persons for which they are made. If therefore Almighty God in framing their laws had an eye unto the nature of that people, and to the country where they were to dwell; if these peculiar and proper considerations were respected in the making of their laws, and must be also regarded in the positive laws of all other nations besides: then seeing that nations are not all alike, surely the giving of one kind of positive laws unto one only people, without any liberty to  alter them, is but a slender proof, that therefore one kind should in like sort be given to serve everlastingly for all.

[7.]But that which most of all maketh for the clearing of this point is, that the Jews, who had laws so particularly determining and so fully instructing them in all affairs what to do, were notwithstanding continually inured with causes exorbitant, and such as their laws had not provided for. And in this point much more is granted us than we ask, namely, that for one thing which we have left to the order of the Church, they had twenty which were undecided by the express word of God; and that as their ceremonies and sacraments were multiplied above ours, even so grew the number of those cases which were not determined by any express word. So that if we may devise one law, they by this reason might devise twenty; and if their devising so many were not forbidden, shall their example prove us forbidden to devise as much as one law for the ordering of the Church? We might not devise no not one, if their example did prove that our Saviour had utterly forbidden all alteration of his laws; inasmuch as there can be no law devised, but needs it must either take away from his, or add thereunto more or less, and so make some kind of alteration. But of this so large a grant we are content not to take advantage. Men are oftentimes in a sudden passion more liberal than they would be if they had leisure to take advice. And therefore so bountiful words of course and frank speeches we are contented to let pass, without turning them unto advantage with too much rigour.

[8.]It may be they had rather be listened unto, when they commend the kings of Israel “which attempted nothing in the government of the Church without the express word of God;” and when they urge that God left nothing in his word “undescribed,” whether it concerned the worship of God or outward polity, nothing unset down, and therefore  charged them strictly to keep themselves unto that, without any alteration. Howbeit, seeing it cannot be denied, but that many things there did belong unto the course of their public affairs, wherein they had no express word at all to shew precisely what they should do; the difference between their condition and ours in these cases will bring some light unto the truth of this present controversy. Before the fact of the son of Shelomith, there was no law which did appoint any certain punishment for blasphemers. That wretched creature being therefore deprehended in that impiety, was held in ward, till the mind of the Lord were known concerning his case. The like practice is also mentioned upon occasion of a breach of the Sabbath day. They find a poor silly creature gathering sticks in the wilderness, they bring him unto Moses and Aaron and all the congregation, they lay him in hold, because it was not declared what should be done with him, till God had said unto Moses, “This man shall die the death.” The law required to keep the Sabbath; but for the breach of the Sabbath what punishment should be inflicted it did not appoint. Such occasions as these are rare. And for such things as do fall scarce once in many ages of men, it did suffice to take such order as was requisite when they fell. But if the case were such as being not already determined by law were notwithstanding likely oftentimes to come in question, it gave occasion of adding laws that were not before. Thus it fell out in the case of those men polluted, and of the daughters of Zelophehad, whose causes Moses having brought before the Lord, received laws to serve for the like in time to come. The Jews to this end had the Oracle of God, they had the Prophets: and by such means God himself instructed them from heaven what to do, in all things that did greatly concern their state and were not already set down in the Law. Shall we then hereupon argue even against our own experience and knowledge? Shall we seek to persuade men that of necessity it is with us as it was with them; that because God is ours in all respects as much as theirs, therefore either no such way of direction hath been at any time, or if it have been, it doth still continue in the Church; or if the same  do not continue, that yet it must be at the least supplied by some such mean as pleaseth us to account of equal force? A more dutiful and religious way for us were to admire the wisdom of God, which shineth in the beautiful variety of all things, but most in the manifold and yet harmonious dissimilitude of those ways, whereby his Church upon earth is guided from age to age, throughout all generations of men.

[9.]The Jews were necessarily to continue till the coming of Christ in the flesh, and the gathering of nations unto him. So much the promise made unto Abraham did import. So much the prophecy of Jacob at the hour of his death did foreshew. Upon the safety therefore of their very outward state and condition for so long, the after-good of the whole world and the salvation of all did depend. Unto their so long safety, for two things it was necessary to provide; namely, the preservation of their state against foreign resistance, and the continuance of their peace within themselves.

Touching the one, as they received the promise of God to be the rock of their defence, against which whoso did violently rush should but bruise and batter themselves; so likewise they had his commandment in all their affairs that way to seek direction and counsel from him. Men’s consultations are always perilous. And it falleth out many times that after long deliberation those things are by their wit even resolved on, which by trial are found most opposite to public safety. It is no impossible thing for states, be they never so well established, yet by oversight in some one act or treaty between them and their potent opposites utterly to cast away themselves for ever. Wherefore lest it should so fall out to them upon whom so much did depend, they were not permitted to enter into war, nor conclude any league of peace, nor to wade through any act of moment between them and foreign states, unless the Oracle of God or his Prophets were first consulted with.

And lest domestical disturbance should waste them within themselves, because there was nothing unto this purpose more effectual, than if the authority of their laws and governors were such, as none might presume to take exception against it, or to shew disobedience unto it, without incurring the  hatred and detestation of all men that had any spark of the fear of God; therefore he gave them even their positive laws from heaven, and as oft as occasion required chose in like sort rulers also to lead and govern them. Notwithstanding some desperately impious there were, which adventured to try what harm it could bring upon them, if they did attempt to be authors of confusion, and to resist both governors and laws. Against such monsters God maintained his own by fearful execution of extraordinary judgment upon them.

By which means it came to pass, that although they were a people infested and mightily hated of all others throughout the world, although by nature hard-hearted, querulous, wrathful, and impatient of rest and quietness; yet was there nothing of force either one way or other to work the ruin and subversion of their state, till the time before-mentioned was expired. Thus we see that there was not no cause of dissimilitude in these things between that one only people before Christ, and the kingdoms of the world since.

[10.]And whereas it is further alleged that albeit “in civil matters and things pertaining to this present life God hath used a greater particularity with them than amongst us, framing laws according to the quality of that people and country; yet the leaving of us at greater liberty in things civil is so far from proving the like liberty in things pertaining to the kingdom of heaven, that it rather proves a straiter bond. For even as when the Lord would have his favour more appear by temporal blessings of this life towards the people under the Law than towards us, he gave also politic laws most exactly, whereby they might both most easily come into and most steadfastly remain in possession of those earthly benefits: even so at this time, wherein he would not have his favour so much esteemed by those outward commodities, it is required, that as his care in prescribing laws for that purpose hath somewhat fallen in leaving them to men’s consultations which may be deceived, so his care for conduct and government of the life to come should (if it were possible) rise, in leaving less to the order of men than in times past.” These are but weak and feeble disputes for the inference of that conclusion which is intended. For  saving only in such consideration as hath been shewed, there is no cause wherefore we should think God more desirous to manifest his favour by temporal blessings towards them than towards us. Godliness had unto them, and it hath also unto us, the promises both of this life and the life to come. That the care of God hath fallen in earthly things, and therefore should rise as much in heavenly; that more is left unto men’s consultations in the one, and therefore less must be granted in the other; that God, having used a greater particularity with them than with us for matters pertaining unto this life, is to make us amends by the more exact delivery of laws for government of the life to come: these are proportions, whereof if there be any rule, we must plainly confess that which truth is, we know it not. God which spake unto them by his Prophets, hath unto us by his only-begotten Son; those mysteries of grace and salvation which were but darkly disclosed unto them, have unto us most clearly shined. Such differences between them and us the Apostles of Christ have well acquainted us withal. But as for matter belonging to the outward conduct or government of the Church, seeing that even in sense it is manifest that our Lord and Saviour hath not by positive laws descended so far into particularities with us as Moses with them, neither doth by extraordinary means, oracles, and prophets, direct us as them he did in those things which rising daily by new occasions are of necessity to be provided for; doth it not hereupon rather follow, that although not to them, yet to us there should be freedom and liberty granted to make laws?

[11.]Yea, but the Apostle St. Paul doth fearfully charge Timothy, even “in the sight of God who quickeneth all,  and of Jesus Christ who witnessed that famous confession before Pontius Pilate, to keep what was commanded him safe and sound till the appearance of our Lord Jesus Christ.” This doth exclude all liberty of changing the laws of Christ, whether by abrogation or addition, or howsoever. For in Timothy the whole Church of Christ receiveth charge concerning her duty; and that charge is to keep the Apostle’s commandment; and his commandment did contain the laws that concerned church government; and those laws he straitly requireth to be observed without breach or blame, till the appearance of our Lord Jesus Christ.

In Scripture we grant every one man’s lesson to be the common instruction of all men, so far forth as their cases are like; and that religiously to keep the Apostle’s commandments in whatsoever they may concern us we all stand bound. But touching that commandment which Timothy was charged with, we swerve undoubtedly from the Apostle’s precise meaning if we extend it so largely, that the arms thereof shall reach unto all things which were commanded him by the Apostle. The very words themselves do restrain themselves unto some one especial commandment among many. And therefore it is not said, “Keep the ordinances, laws, and constitutions, which thou hast received;” but τὴν ἐντολὴν, “that great commandment, which doth principally concern thee and thy calling;” that commandment which Christ did so often inculcate unto Peter; that commandment unto the careful discharge whereof they of Ephesus are exhorted, “Attend to yourselves, and to all the flock wherein the Holy Ghost hath placed you Bishops, to feed the Church of God, which he hath purchased by his own blood;” finally that commandment which unto the same Timothy is by the same Apostle even in the same form and manner afterwards again urged, “I charge thee in the sight of God and the Lord Jesus Christ, which will judge the quick and dead at his appearance and in his kingdom, preach the word of God.”  When Timothy was instituted into the office, then was the credit and trust of this duty committed unto his faithful care. The doctrine of the Gospel was then given him, “as the precious talent or treasure of Jesus Christ;” then received he for performance of this duty “the special gift of the Holy Ghost.” “To keep this commandment immaculate and blameless” was to teach the Gospel of Christ without mixture of corrupt and unsound doctrine, such as a number did even in those times intermingle with the mysteries of Christian belief. “Till the appearance of Christ to keep it so,” doth not import the time wherein it should be kept, but rather the time whereunto the final reward for keeping it was reserved: according to that of St. Paul concerning himself, “I have kept the faith; for the residue there is laid up for me a crown of righteousness, which the Lord the righteous shall in that day render unto me.” If they that labour in this harvest should respect but the present fruit of their painful travel, a poor encouragement it were unto them to continue therein all the days of their life. But their reward is great in heaven; the crown of righteousness which shall be given them in that day is honourable. The fruit of their industry then shall they reap with full contentment and satisfaction, but not till then. Wherein the greatness of their reward is abundantly sufficient to countervail the tediousness of their expectation. Wherefore till then, they that are in labour must rest in hope. “O Timothy, keep that which is committed unto thy charge; that great commandment which thou hast received keep, till the appearance of our Lord Jesus Christ.”

In which sense although we judge the Apostle’s words to have been uttered, yet hereunto we do not require them to yield, that think any other construction more sound. If therefore it be rejected, and theirs esteemed more probable which hold, that the last words do import perpetual observation of the Apostle’s commandment imposed necessarily for ever upon the militant Church of Christ; let them withal consider, that then his commandment cannot so largely be taken, as to comprehend whatsoever the Apostle did command Timothy. For themselves do not all bind the Church unto  some things whereof Timothy received charge, as namely unto that precept concerning the choice of widows. So as they cannot hereby maintain that all things positively commanded concerning the affairs of the Church were commanded for perpetuity. And we do not deny that certain things were commanded to be though positive yet perpetual in the Church.

[12.]They should not therefore urge against us places that seem to forbid change, but rather such as set down some measure of alteration, which measure if we have exceeded, then might they therewith charge us justly: whereas now they themselves both granting, and also using liberty to change, cannot in reason dispute absolutely against all change. Christ delivered no inconvenient or unmeet laws: sundry of ours they hold inconvenient: therefore such laws they cannot possibly hold to be Christ’s: being not his, they must of necessity grant them added unto his. Yet certain of those very laws so added they themselves do not judge unlawful; as they plainly confess both in matter of prescript attire and of rites appertaining to burial. Their own protestations are, that they plead against the inconvenience, not the unlawfulness of popish apparel; and against the inconvenience not the unlawfulness of ceremonies in burial. Therefore they hold it a thing not unlawful to add to the laws of Jesus Christ; and so consequently they yield that no law of Christ forbiddeth addition unto church laws.

[13.]The judgment of Calvin being alleged against them,  to whom of all men they attribute most; whereas his words be plain, that for ceremonies and external discipline the Church hath power to make laws: the answer which hereunto they make is, that indefinitely the speech is true, and that so it was meant by him; namely, that some things belonging unto external discipline and ceremonies are in the power and arbitrement of the Church; but neither was it meant, neither is it true generally, that all external discipline and all ceremonies are left to the order of the Church, inasmuch as the sacraments of Baptism and the Supper of the Lord are ceremonies, which yet the Church may not therefore abrogate. Again, Excommunication is a part of external discipline, which might also be cast away, if all external discipline were arbitrary and in the choice of the Church.

By which their answer it doth appear, that touching the names of ceremony and external discipline they gladly would have us so understood, as if we did herein contain a great deal more than we do. The fault which we find with them is, that they overmuch abridge the Church of her power in these things. Whereupon they recharge us, as if in these things we gave the Church a liberty which hath no limits or bounds; as if all things which the name of discipline containeth were of the Church’s free choice; so that we might either have church governors and government or want them, either retain or reject church censures as we list. They wonder at us, as at men which think it so indifferent what the Church doth in matter of ceremonies, that it may be feared lest we judge the very Sacraments themselves to be held at the Church’s pleasure.


No, the name of ceremonies we do not use in so large a meaning as to bring Sacraments within the compass and reach thereof, although things belonging unto the outward form and seemly administration of them are contained in that name, even as we use it. For the name of ceremonies we use as they themselves do, when they speak after this sort: “The doctrine and discipline of the Church, as the weightiest things, ought especially to be looked unto; but the ceremonies also, as mint and cummin, ought not to be neglected.” Besides, in the matter of external discipline or regiment itself, we do not deny but there are some things whereto the church is bound till the world’s end. So as the question is only how far the bounds of the Church’s liberty do reach. We hold, that the power which the Church hath lawfully to make laws and orders for itself doth extend unto sundry things of ecclesiastical jurisdiction, and such other matters, whereto their opinion is that the Church’s authority and power doth not reach. Whereas therefore in disputing against us about this point, they take their compass a great deal wider than the truth of things can afford; producing reasons and arguments by way of generality, to prove that Christ hath set down all things belonging any way unto the form of ordering his Church, and hath absolutely forbidden change by addition or diminution, great or small: (for so their manner of disputing is:) we are constrained to make our defence, by shewing that Christ hath not deprived his Church so far of all liberty in making orders and laws for itself, and that they themselves do not think he hath so done. For are they able to shew that all particular customs, rites, and orders of reformed churches have been appointed by Christ himself? No: they grant that in matter of circumstance they alter that which they have received, but in things of substance, they keep the laws of Christ without change. If we say the same in our own behalf (which surely we may do with a great deal more truth) then must they cancel all that hath been before alleged, and begin to inquire afresh, whether we retain the  laws that Christ hath delivered concerning matters of substance, yea or no. For our constant persuasion in this point is as theirs, that we have no where altered the laws of Christ farther than in such particularities only as have the nature of things changeable according to the difference of times, places, persons, and other the like circumstances. Christ hath commanded prayers to be made, sacraments to be ministered, his Church to be carefully taught and guided. Concerning every of these somewhat Christ hath commanded which must be kept till the world’s end. On the contrary side, in every of them somewhat there may be added, as the Church shall judge it expedient. So that if they will speak to purpose, all which hitherto hath been disputed of they must give over, and stand upon such particulars only as they can shew we have either added or abrogated otherwise than we ought, in the matter of church polity. Whatsoever Christ hath commanded for ever to be kept in his Church, the same we take not upon us to abrogate; and whatsoever our laws have thereunto added besides, of such quality we hope it is as no law of Christ doth any where condemn.

[14.]Wherefore that all may be laid together and gathered into a narrower room: First, so far forth as the Church is the mystical body of Christ and his invisible spouse, it needeth no external polity. That very part of the law divine which teacheth faith and works of righteousness is itself alone sufficient for the Church of God in that respect. But as the Church is a visible society and body politic, laws of polity it cannot want.

[15.]Secondly: Whereas therefore it cometh in the second place to be inquired, what laws are fittest and best for the Church; they who first embraced that rigorous and strict opinion, which depriveth the Church of liberty to make any kind of law for herself, inclined as it should seem thereunto, for that they imagined all things which the Church doth without commandment of Holy Scripture subject to that reproof which the Scripture itself useth in certain cases when divine authority ought alone to be followed. Hereupon they thought it enough for the cancelling of any kind of order whatsoever, to say, “The word of God teacheth it not, it is a device of  the brain of man, away with it therefore out of the Church.” St. Augustine was of another mind, who speaking of fasts on the Sunday saith, “That he which would choose out that day to fast on, should give thereby no small offence to the Church of God, which had received a contrary custom. For in these things, whereof the Scripture appointeth no certainty, the use of the people of God or the ordinances of our fathers must serve for a law. In which case if we will dispute, and condemn one sort by another’s custom, it will be but matter of endless contention; where, forasmuch as the labour of reasoning shall hardly beat into men’s heads any certain or necessary truth, surely it standeth us upon to take heed, lest with the tempest of strife the brightness of charity and love be darkened.”

If all things must be commanded of God which may be practised of his Church, I would know what commandment the Gileadites had to erect that altar which is spoken of in the Book of Josua. Did not congruity of reason induce them thereunto, and suffice for defence of their fact? I would know what commandment the women of Israel had yearly to mourn and lament in the memory of Jephtha’s daughter; what commandment the Jews had to celebrate their feast of Dedication, never spoken of in the law, yet solemnized even by our Saviour himself; what commandment finally they had for the ceremony of odours used about the bodies of the dead, after which custom notwithstanding (sith it was their custom) our Lord was contented that his own most precious body should be entombed. Wherefore to reject all orders of the Church which men have established, is to think worse of the laws of men in this respect, than either the judgment of wise men alloweth, or the law of God itself will bear.

[16.]Howbeit they which had once taken upon them to  condemn all things done in the Church and not commanded of God to be done, saw it was necessary for them (continuing in defence of this their opinion) to hold that needs there must be in Scripture set down a complete particular form of church polity, a form prescribing how all the affairs of the Church must be ordered, a form in no respect lawful to be altered by mortal men. For reformation of which oversight and error in them, there were that thought it a part of Christian love and charity to instruct them better, and to open unto them the difference between matters of perpetual necessity to all men’s salvation, and matters of ecclesiastical polity: the one both fully and plainly taught in holy Scripture, the other not necessary to be in such sort there prescribed; the one not capable of any diminution or augmentation at all by men, the other apt to admit both. Hereupon the authors of the former opinion were presently seconded by other wittier and better learned, who being loth that the form of church polity which they sought to bring in should be otherwise than in the highest degree accounted of, took first an exception against the difference between church polity and matters of necessity unto salvation; secondly, against the restraint of Scripture, which they say receiveth injury at our hands, when we teach that it teacheth not as well matters of polity as of faith and salvation. Thirdly, Constrained hereby we have been therefore both to maintain that distinction, as a thing not only true in itself, but by them likewise so acknowledged, though unawares; Fourthly, and to make manifest that from Scripture we offer not to derogate the least thing that truth thereunto doth claim, inasmuch as by us it is willingly confest, that the Scripture of God is a storehouse abounding with inestimable  treasures of wisdom and knowledge in many kinds, over and above things in this one kind barely necessary; yea, even that matters of ecclesiastical polity are not therein omitted, but taught also, albeit not so taught as those other things before mentioned. For so perfectly are those things taught, that nothing can ever need to be added, nothing ever cease to be necessary; these on the contrary side, as being of a far other nature and quality, not so strictly nor everlastingly commanded in Scripture, but that unto the complete form of church polity much may be requisite which the Scripture teacheth not, and much which it hath taught become unrequisite, sometime because we need not use it, sometime also because we cannot. In which respect for mine own part, although I see that certain reformed churches, the Scottish especially and French, have not that which best agreeth with the sacred Scripture, I mean the government that is by Bishops, inasmuch as both those churches are fallen under a different kind of regiment; which to remedy it is for the one altogether too late, and too soon for the other during their present affliction and trouble: this their defect and imperfection I had rather lament in such case than exagitate, considering that men oftentimes without any fault of their own may be driven to want that kind of polity or regiment which is best, and to content themselves with that, which either the irremediable error of former times, or the necessity of the present hath cast upon them.


[17.]Fifthly, Now because that position first-mentioned, which holdeth it necessary that all things which the Church may lawfully do in her own regiment be commanded in holy Scripture, hath by the later defenders thereof been greatly qualified; who, though perceiving it to be over extreme, are notwithstanding loth to acknowledge any oversight therein, and therefore labour what they may to salve it by construction; we have for the more perspicuity delivered what was thereby meant at the first: sixthly, how injurious a thing it were unto all the churches of God for men to hold it in that meaning: seventhly, and how imperfect their interpretations are who so much labour to help it, either by dividing commandments of Scripture into two kinds, and so defending that all things must be commanded, if not in special yet in general precepts; eighthly, or by taking it as meant, that in case the Church do devise any new order, she ought therein to follow the direction of Scripture only, and not any starlight of man’s reason. Ninthly, both which evasions being cut off, we have in the next place declared after what sort the Church may lawfully frame to herself laws of polity, and in what reckoning such positive laws both are with God and should be with men. Tenthly, furthermore, because to abridge the liberty of the Church in this behalf, it hath been made a thing very odious, that when God himself hath devised some certain laws and committed them to sacred Scripture, man by abrogation, addition, or any way, should presume to alter and change them; it was of necessity to be examined, whether the authority of God in making, or his care in committing those his laws unto Scripture, be sufficient arguments to prove that God doth in no case allow they should suffer any such kind of change. Eleventhly, the last refuge for proof that divine laws of Christian church polity may not be altered by extinguishment of any old or addition of new in that kind, is partly a marvellous strange discourse, that Christ (unless he should shew himself not so faithful as Moses, or not so wise as Lycurgus and Solon) must needs have set down in holy  Scripture some certain complete and unchangeable form of polity: and partly a coloured show of some evidence where change of that sort of laws may seem expressly forbidden, although in truth nothing less be done.

[18.]I might have added hereunto their more familiar and popular disputes, as, The Church is a city, yea the city of the great King; and the life of a city is polity: The Church is the house of the living God; and what house can there be without some order for the government of it? In the royal house of a prince there must be officers for government, such as not any servant in the house but the prince whose the house is shall judge convenient. So the house of God must have orders for the government of it, such as not any of the household but God himself hath appointed. It cannot stand with the love and wisdom of God to leave such order untaken as is necessary for the due government of his Church. The numbers, degrees, orders, and attire of Salomon’s servants, did shew his wisdom; therefore he which is greater than Salomon hath not failed to leave in his house such orders for government thereof, as may serve to be a looking-glass for his providence, care, and wisdom, to be seen in. That little spark of the light of nature which remaineth in us may serve us for the affairs of this life. “But as in all other matters concerning the kingdom of heaven, so principally in this which concerneth the very government of that kingdom, needful it is we should be taught of God. As long as men are persuaded of any order that it is only of men, they presume of their own understanding, and they think to devise another not only as good, but better than that which they  have received. By severity of punishment this presumption and curiosity may be restrained. But that cannot work such cheerful obedience as is yielded where the conscience hath respect to God as the author of laws and orders. This was it which countenanced the laws of Moses, made concerning outward polity for the administration of holy things. The like some lawgivers of the heathens did pretend, but falsely; yet wisely discerning the use of this persuasion. For the better obedience’ sake therefore it was expedient that God should be author of the polity of his Church.”

[19.]But to what issue doth all this come? A man would think that they which hold out with such discourses were of nothing more fully persuaded than of this, that the Scripture hath set down a complete form of church polity, universal, perpetual, altogether unchangeable. For so it would follow, if the premises were sound and strong to such effect as is pretended. Notwithstanding, they which have thus formally maintained argument in defence of the first oversight, are by the very evidence of truth themselves constrained to make this in effect their conclusion, that the Scripture of God hath many things concerning church polity; that of those many some are of greater weight, some of less; that what hath been urged as touching immutability of laws, it extendeth in truth no farther than only to laws wherein things of greater moment are prescribed. Now those things of greater moment, what are they? Forsooth, “doctors, pastors, lay-elders, elderships compounded of these three; synods, consisting of many elderships; deacons, women-church-servants or widows; free consent of the people unto actions of greatest moment, after they be by churches or synods orderly resolved.” All “this form” of polity (if yet we may term that a form of building, when men have laid a few rafters together, and those not all of the soundest neither) but howsoever, all this form they conclude is prescribed in such sort, that to add to it any thing as of like importance (for so I think they mean) or to abrogate of it any thing at all, is unlawful. In which resolution if they will firmly and constantly persist, I see not but that concerning the points which hitherto have been disputed of, they must agree that they have molested the Church  with needless opposition, and henceforward as we said before betake themselves wholly unto the trial of particulars, whether every of those things which they esteem as principal, be either so esteemed of, or at all established for perpetuity in holy Scripture; and whether any particular thing in our Church polity be received other than the Scripture alloweth of, either in greater things or in smaller.

[20.]The matters wherein Church polity is conversant are the public religious duties of the Church, as the administration of the word and sacraments, prayers, spiritual censures, and the like. To these the Church standeth always bound. Laws of polity, are laws which appoint in what manner these duties shall be performed.

In performance whereof because all that are of the Church cannot jointly and equally work, the first thing in polity required is a difference of persons in the Church, without which difference those functions cannot in orderly sort be executed. Hereupon we hold that God’s clergy are a state, which hath been and will be, as long as there is a Church upon earth, necessary by the plain word of God himself; a state whereunto the rest of God’s people must be subject as touching things that appertain to their souls’ health. For where polity is, it cannot but appoint some to be leaders of others, and some to be led by others. “If the blind lead the blind, they both perish.” It is with the clergy, if their persons be respected, even as it is with other men; their quality many times far beneath that which the dignity of their place requireth. Howbeit according to the order of polity, they being the “lights of the world,” others (though better and wiser) must that way be subject unto them.

Again, forasmuch as where the clergy are any great multitude, order doth necessarily require that by degrees they be distinguished; we hold there have ever been and ever ought to be in such case at leastwise two sorts of ecclesiastical persons, the one subordinate unto the other; as to the Apostles in the beginning, and to the Bishops always since, we find plainly both in Scripture and in all ecclesiastical records, other ministers of the word and sacraments have been.

Moreover, it cannot enter into any man’s conceit to think  it lawful, that every man which listeth should take upon him charge in the Church; and therefore a solemn admittance is of such necessity, that without it there can be no church-polity.

A number of particularities there are, which make for the more convenient being of these principal and perpetual parts in ecclesiastical polity, but yet are not of such constant use and necessity in God’s Church. Of this kind are, times and places appointed for the exercise of religion; specialties belonging to the public solemnity of the word, the sacraments, and prayer; the enlargement or abridgment of functions ministerial depending upon those two principal before-mentioned; to conclude, even whatsoever doth by way of formality and circumstance concern any public action of the Church. Now although that which the Scripture hath of things in the former kind be for ever permanent: yet in the later both much of that which the Scripture teacheth is not always needful; and much the Church of God shall always need which the Scripture teacheth not.

So as the form of polity by them set down for perpetuity is three ways faulty: faulty in omitting some things which in Scripture are of that nature, as namely the difference that ought to be of Pastors when they grow to any great multitude: faulty in requiring Doctors, Deacons, Widows, and such like, as things of perpetual necessity by the law of God, which in truth are nothing less: faulty also in urging some things by Scripture immutable, as their Lay-elders, which the Scripture neither maketh immutable nor at all teacheth, for any thing either we can as yet find or they have hitherto been able to prove. But hereof more in the books that follow.

[21.]As for those marvellous discourses whereby they adventure to argue that God must needs have done the thing which they imagine was to be done; I must confess I have often wondered at their exceeding boldness herein. When the question is whether God have delivered in Scripture (as they affirm he hath) a complete, particular, immutable form of church polity, why take they that other both presumptuous and superfluous labour to prove he should have done it; there being no way in this case to prove the deed of God, saving only by producing that evidence wherein  he hath done it? But if there be no such thing apparent upon record, they do as if one should demand a legacy by force and virtue of some written testament, wherein there being no such thing specified, he pleadeth that there it must needs be, and bringeth arguments from the love or goodwill which always the testator bore him; imagining, that these or the like proofs will convict a testament to have that in it which other men can no where by reading find. In matters which concern the actions of God, the most dutiful way on our part is to search what God hath done, and with meekness to admire that, rather than to dispute what he in congruity of reason ought to do. The ways which he hath whereby to do all things for the greatest good of his Church are more in number than we can search, other in nature than that we should presume to determine which of many should be the fittest for him to choose, till such time as we see he hath chosen of many some one; which one we then may boldly conclude to be the fittest, because he hath taken it before the rest. When we do otherwise, surely we exceed our bounds; who and where we are we forget; and therefore needful it is that our pride in such cases be controlled, and our disputes beaten back with those demands of the blessed Apostle, “How unsearchable are his judgments, and his ways past finding out! Who hath known the mind of the Lord, or who was his counsellor?”
