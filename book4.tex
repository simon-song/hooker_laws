\chapter*[The Fourth Book]{THE FOURTH BOOK. 
CONCERNING THEIR THIRD ASSERTION, THAT OUR FORM OF CHURCH POLITY IS CORRUPTED WITH POPISH ORDERS, RITES, AND CEREMONIES, BANISHED OUT OF CERTAIN REFORMED CHURCHES, WHOSE EXAMPLE THEREIN WE OUGHT TO HAVE FOLLOWED.}
\label{chap:book4}
\addcontentsline{toc}{chapter}{THE FOURTH BOOK}

THE MATTER CONTAINED IN THIS FOURTH BOOK.

I. How great use Ceremonies have in the Church.

II. The first thing they blame in the kind of our Ceremonies is, that we have not in them ancient apostolical simplicity, but a greater pomp and stateliness.

III. The second, that so many of them are the same which the Church of Rome useth; and the reasons which they bring to prove them for that cause blame-worthy.

IV. How when they go about to expound what Popish Ceremonies they mean, they contradict their own arguments against Popish Ceremonies.

V. An answer to the argument whereby they would prove, that sith we allow the customs of our fathers to be followed, we therefore may not allow such customs as the Church of Rome hath, because we cannot account of them which are of that Church as of our fathers.

VI. To their allegation, that the course of God’s own wisdom doth make against our conformity with the Church of Rome in such things.

VII. To the example of the eldest Churches which they bring for the same purpose.

VIII. That it is not our best polity (as they pretend it is) for establishment of sound religion, to have in these things no agreement with the Church of Rome being unsound.

IX. That neither the Papists upbraiding us as furnished out of their store, nor any hope which in that respect they are said to conceive, doth make any more against our ceremonies than the former allegations have done.

X. The grief which they say godly brethren conceive at such ceremonies as we have common with the Church of Rome.

XI. The third thing for which they reprove a great part of our ceremonies is, for that as we have them from the Church of Rome, so that Church had them from the Jews.

XII. The fourth, for that sundry of them have been (they say) abused unto idolatry, and are by that mean become scandalous. 

XIII. The fifth, for that we retain them still, notwithstanding the example of certain Churches reformed before us, which have cast them out.

XIV. A declaration of the proceedings of the Church of England for the establishment of things as they are.

\PRLsep

\section*{How great use Ceremonies have in the Church.}

I. SUCH was the ancient simplicity and softness of spirit which sometimes prevailed in the world, that they whose words were even as oracles amongst men, seemed evermore loth to give sentence against any thing publicly received in the Church of God, except it were wonderful apparently evil; for that they did not so much incline to that severity which delighteth to reprove the least things it seeth amiss, as to that charity which is unwilling to behold any thing that duty bindeth it to reprove. The state of this present age, wherein zeal hath drowned charity, and skill meekness, will not now suffer any man to marvel, whatsoever he shall hear reproved by whomsoever. Those rites and ceremonies of the Church therefore, which are the selfsame now that they were when holy and virtuous men maintained them against profane and deriding adversaries, her own children have at this day in derision. Whether justly or no, it shall then appear, when all things are heard which they have to allege against the outward received orders of this church. Which inasmuch as themselves do compare unto “mint and cummin,” granting them to be no part of those things which in the matter of polity are weightier, we hope that for small things their strife will neither be earnest nor long.

[2.]The sifting of that which is objected against the orders of the Church in particular, doth not belong unto this place. Here we are to discuss only those general exceptions, which have been taken at any time against them.

First therefore to the end that their nature and the use whereunto they serve may plainly appear, and so afterwards their quality the better be discerned; we are to note, that in every grand or main public duty which God requireth at the  hands of his Church, there is, besides that matter and form wherein the essence thereof consisteth, a certain outward fashion whereby the same is in decent sort administered. The substance of all religious actions is delivered from God himself in few words. For example’s sake in the sacraments. “Unto the element let the word be added, and they both do make a sacrament,” saith St. Augustine. Baptism is given by the element of water, and that prescript form of words which the Church of Christ doth use; the sacrament of the body and blood of Christ is administered in the elements of bread and wine, if those mystical words be added thereunto. But the due and decent form of administering those holy sacraments doth require a great deal more.

[3.]The end which is aimed at in setting down the outward form of all religious actions is the edification of the Church. Now men are edified, when either their understanding is taught somewhat whereof in such actions it behoveth all men to consider, or when their hearts are moved with any affection suitable thereunto; when their minds are in any sort stirred up unto that reverence, devotion, attention, and due regard, which in those cases seemeth requisite. Because therefore unto this purpose not only speech but sundry sensible means besides have always been thought necessary, and especially those means which being object to the eye, the liveliest and the most apprehensive sense of all other, have in that respect seemed the fittest to make a deep and a strong impression: from hence have risen not only a number of prayers, readings, questionings, exhortings, but even of visible signs also; which being used in performance of holy actions, are undoubtedly most effectual to open such matter, as men when they know and remember carefully, must needs be a great deal the better informed to what effect such duties serve. We must not think but that there is some ground of reason even in nature, whereby it cometh to pass that no nation under heaven either doth or ever did suffer public actions  which are of weight, whether they be civil and temporal or else spiritual and sacred, to pass without some visible solemnity: the very strangeness whereof and difference from that which is common, doth cause popular eyes to observe and to mark the same. Words, both because they are common, and do not so strongly move the fancy of man, are for the most part but slightly heard: and therefore with singular wisdom it hath been provided, that the deeds of men which are made in the presence of witnesses should pass not only with words, but also with certain sensible actions, the memory whereof is far more easy and durable than the memory of speech can be.

The things which so long experience of all ages hath confirmed and made profitable, let not us presume to condemn as follies and toys, because we sometimes know not the cause and reason of them. A wit disposed to scorn whatsoever it doth not conceive, might ask wherefore Abraham should say to his servant, “Put thy hand under my thigh and swear:” was it not sufficient for his servant to shew the religion of an oath by naming the Lord God of heaven and earth, unless that strange ceremony were added? In contracts, bargains, and conveyances, a man’s word is a token sufficient to express his will. Yet “this was the ancient manner in Israel concerning redeeming and exchanging, to establish all things; a man did pluck off his shoe and gave it his neighbour; and this was a sure witness in Israel.” Amongst the Romans in their making of a bondman free, was it not wondered wherefore so great ado should be made? The master to present his slave in some court, to take him by the hand, and not only to say in the hearing of the public magistrate, “I will that this man become free,” but after these solemn words uttered, to strike him on the cheek, to turn him round, the hair of his head to be shaved off, the magistrate to touch him thrice with a rod, in the end a cap and a white garment to be given him. To what purpose all this circumstance? Amongst the Hebrews how strange and in outward appearance almost against reason, that he which was minded to make himself a perpetual servant, should not only testify  so much in the presence of the judge, but for a visible token thereof have also his ear bored through with an awl! It were an infinite labour to prosecute these things so far as they might be exemplified both in civil and religious actions. For in both they have their necessary use and force. “The sensible things which religion hath hallowed, are resemblances framed according to things spiritually understood, whereunto they serve as a hand to lead, and a way to direct.”

[4.]And whereas it may peradventure be objected, that to add to religious duties such rites and ceremonies as are significant, is to institute new Sacraments; sure I am they will not say that Numa Pompilius did ordain a sacrament, a significant ceremony he did ordain, in commanding the priests “to execute the work of their divine service with their hands as far as to the fingers covered; thereby signifying that fidelity must be defended, and that men’s right hands are the sacred seat thereof.” Again we are also to put them in mind, that themselves do not hold all significant ceremonies for sacraments, insomuch as imposition of hands they deny to be a sacrament, and yet they give thereunto a forcible signification; for concerning it their words are these: “The party ordained by this ceremony was put in mind of his separation to the work of the Lord, that remembering himself to be taken as it were with the hand of God from amongst others, this might teach him not to account himself now his own, nor to do what himself listeth, but to consider that God hath set him about a work, which if he will discharge and accomplish, he may at the hands of God assure himself of reward; and if otherwise, of revenge.” Touching significant ceremonies,  some of them are sacraments, some as sacraments only. Sacraments are those which are signs and tokens of some general promised grace, which always really descendeth from God unto the soul that duly receiveth them; other significant tokens are only as Sacraments, yet no Sacraments: which is not our distinction, but theirs. For concerning the Apostles’ imposition of hands these are their own words; “manuum signum hoc et quasi Sacramentum usurparunt;” “they used this sign, or as it were sacrament.”

\section*{The first thing they blame in the kind of our Ceremonies is, that we have not in them ancient apostolical simplicity, but a greater pomp and stateliness.}

II. Concerning rites and ceremonies there may be fault, either in the kind or in the number and multitude of them. The first thing blamed about the kind of ours is, that in many things we have departed from the ancient simplicity of Christ and his Apostles; we have embraced more outward stateliness, we have those orders in the exercise of religion, which they who best pleased God and served him most devoutly never had. For it is out of doubt that the first state of things was best, that in the prime of Christian religion faith was soundest, the Scriptures of God were then best understood by all men, all parts of godliness did then most abound; and therefore it must needs follow, that customs, laws, and ordinances devised since are not so good for the Church of Christ, but the best way is to cut off later inventions, and to reduce things unto the ancient state wherein at the first they were. Which rule or canon we hold to be either uncertain or at leastwise unsufficient, if not both.

[2.]For in case it be certain, hard it cannot be for them to shew us, where we shall find it so exactly set down, that we may say without all controversy, “these were the orders of the Apostles’ times, these wholly and only, neither fewer nor more than these.” True it is that many things of this nature be alluded unto, yea many things declared, and many things necessarily collected out of the Apostles’ writings. But is it necessary that all the orders of the Church which were then in use should be contained in their books? Surely no. For if the tenor of their writings be well observed, it shall unto any man easily appear, that no more of them are there touched than were needful to be spoken of, sometimes  by one occasion and sometimes by another. Will they allow then of any other records besides? Well assured I am they are far enough from acknowledging that the Church ought to keep any thing as apostolical, which is not found in the Apostles’ writings, in what other records soever it be found. And therefore whereas St. Augustine affirmeth that those things which the whole Church of Christ doth hold, may well be thought to be apostolical although they be not found written; this his judgment they utterly condemn. I will not here stand in defence of St. Augustine’s opinion, which is, that such things are indeed apostolical, but yet with this exception, unless the decree of some general council have haply caused them to be received: for of positive laws and orders received throughout the whole Christian world, St. Augustine could imagine no other fountain save these two. But to let pass St. Augustine; they who condemn him herein must needs confess it a very uncertain thing what the orders of the Church were in the Apostles’ times, seeing the Scriptures do not mention them all, and other records thereof besides they utterly reject. So that in tying the Church to the orders of the Apostles’ times, they tie it to a marvellous uncertain rule; unless they require the observation of no orders but only those which are known to be apostolical by the Apostles’ own writings. But then is not this their rule of such sufficiency, that we should use it as a touchstone to try the orders of the Church by for ever.

[3.]Our end ought always to be the same; our ways and means thereunto not so. The glory of God and the good of His Church was the thing which the Apostles aimed at, and therefore ought to be the mark whereat we also level. But seeing those rites and orders may be at one time more which  at another are less available unto that purpose, what reason is there in these things to urge the state of one only age as a pattern for all to follow? It is not I am right sure their meaning, that we should now assemble our people to serve God in close and secret meetings; or that common brooks or rivers should be used for places of baptism; or that the Eucharist should be ministered after meat; or that the custom of church feasting should be renewed; or that all kind of standing provision for the ministry should be utterly taken away, and their estate made again dependent upon the voluntary devotion of men. In these things they easily perceive how unfit that were for the present, which was for the first age convenient enough. The faith, zeal, and godliness of former times is worthily had in honour; but doth this prove that the orders of the Church of Christ must be still the selfsame with theirs, that nothing may be which was not then, or that nothing which then was may lawfully since have ceased? They who recall the Church unto that which was at the first, must necessarily set bounds and limits unto their speeches. If any thing have been received repugnant unto that which was first delivered, the first things in this case must stand, the last give place unto them. But where difference is without repugnancy, that which hath been can be no prejudice to that which is.

[4.]Let the state of the people of God when they were in the house of bondage, and their manner of serving God in a strange land, be compared with that which Canaan and Jerusalem did afford, and who seeth not what huge difference there was between them? In Egypt it may be they were right glad to take some corner of a poor cottage, and there to serve God upon their knees, peradventure covered in dust and straw sometimes. Neither were they therefore the less accepted of God, but he was with them in all their afflictions, and at the length by working their admirable deliverance did testify, that they served him not in vain. Notwithstanding in the very desert they are no sooner possest of some little thing of their own, but a tabernacle is required at their hands. Being planted in the land of Canaan, and having David to be their king, when the Lord had given him rest from all his enemies, it grieved his religious mind to consider the growth of his own estate and dignity, the affairs of religion continuing  still in their former manner: “Behold now I dwell in an house of cedar-trees, and the ark of God remaineth still within curtains.” What he did purpose it was the pleasure of God that Salomon his son should perform, and perform it in manner suitable unto their present, not their ancient estate and condition. For which cause Salomon writeth unto the king of Tyrus, “The house which I build is great and wonderful, for great is our God above all gods.” Whereby it clearly appeareth that the orders of the Church of God may be acceptable unto him, as well being framed suitable to the greatness and dignity of later, as when they keep the reverend simplicity of ancienter times. Such dissimilitude therefore between us and the Apostles of Christ in the order of some outward things is no argument of default.

\section*{The second, that so many of them are the same which the Church of Rome useth; and the reasons which they bring to prove them for that cause blame-worthy.}

III. Yea, but we have framed ourselves to the customs of the church of Rome; our orders and ceremonies are papistical. It is espied that our church founders were not so careful as in this matter they should have been, but contented themselves with such discipline as they took from the church of Rome. Their error we ought to reform by abolishing all popish orders. There must be no communion nor fellowship with Papists, neither in doctrine, ceremonies, nor government. It is not enough that we are divided from the church of Rome by the single wall of doctrine, retaining as we do part of their ceremonies and almost their whole government; but government or ceremonies or whatsoever it be which is popish, away with it. This is the thing they require in us, the utter relinquishment of all things popish.

Wherein to the end we may answer them according unto their plain direct meaning, and not take advantage of doubtful speech, whereby controversies grow always endless; their main position being this, that “nothing should be placed in the Church but what God in his word hath commanded,”  they must of necessity hold all for popish which the church of Rome hath over and besides this. By popish orders, ceremonies, and government, they must therefore mean in every of these so much as the Church of Rome hath embraced without commandment of God’s word: so that whatsoever such thing we have, if the church of Rome hath it also, it goeth under the name of those things that are popish, yea although it be lawful, although agreeable to the word of God. For so they plainly affirm, saying, “Although the forms and ceremonies which they” (the church of Rome) “used were not unlawful, and that they contained nothing which is not agreeable to the word of God, yet notwithstanding neither the word of God, nor reason, nor the examples of the eldest churches both Jewish and Christian do permit us to use the same forms and ceremonies, being neither commanded of God, neither such as there may not as good as they, and rather better, be established.” The question therefore is, whether we may follow the church of Rome in those orders, rites, and ceremonies, wherein we do not think them blameable, or else ought to devise others, and to have no conformity with them, no not so much as in these things. In this sense and construction therefore as they affirm, so we deny, that whatsoever is popish we ought to abrogate.

[2.]Their arguments to prove that generally all popish orders and ceremonies ought to be clean abolished, are in sum these: “First, whereas we allow the judgment of St. Augustine, that touching those things of this kind which are not commanded or forbidden in the Scripture, we are to observe the custom of the people of God and decree of our forefathers; how can we retain the customs and constitutions of the papists in such things, who were neither the people of God nor our forefathers?” Secondly, “although the forms and ceremonies of the church of Rome were not unlawful, neither did contain any thing which is not agreeable to the word of God, yet neither the word of God, nor the examples of the eldest churches of God, nor reason, do permit us to use the same, they being heretics  and so near about us, and their orders being neither commanded of God, nor yet such but that as good or rather better may be established.” It is against the word of God to have conformity with the church of Rome in such things, as appeareth in that “the wisdom of God hath thought it a good way to keep his people from infection of idolatry and superstition, by severing them from idolaters in outward ceremonies, and therefore hath forbidden them to do things which are in themselves very lawful to be done.” And further, “whereas the Lord was careful to sever them by ceremonies from other nations, yet was he not so careful to sever them from any as from the Egyptians amongst whom they lived, and from those nations which were next neighbours unto them, because from them was the greatest fear of infection.” So that following the course which the wisdom of God doth teach, “it were more safe for us to conform our indifferent ceremonies to the Turks which are far off, than to the papists which are so near.”

Touching the example of the eldest churches of God; in one council it was decreed, “that Christians should not deck their houses with bay leaves and green boughs, because the Pagans did use so to do; and that they should not rest from their labours those days that the Pagans did; that they should not keep the first day of every month as they did. Another council decreed that Christians should not  celebrate feasts on the birthdays of the martyrs, because it was the manner of the heathen.” “ ‘O!’ saith Tertullian, ‘better is the religion of the heathen: for they use no solemnity of the Christians, neither the Lord’s day, neither the Pentecost; and if they knew them they would have nothing to do with them: for they would be afraid lest they should seem Christians; but we are not afraid to be called heathen.’ ” The same Tertullian would not have Christians to sit after they have prayed, because the idolaters did so. Whereby it appeareth, that both of particular men and of councils, in making or abolishing of ceremonies, heed hath been taken that the Christians should not be like the idolaters, no not in those things which of themselves are most indifferent to be used or not used.

The same conformity is not less opposite unto reason; first inasmuch as “contraries must be cured by their contraries, and therefore popery being anti-christianity is not healed, but by establishment of orders thereunto opposite. The way to bring a drunken man to sobriety is to carry him as far from excess of drink as may be. To rectify a crooked  stick we bend it on the contrary side, as far as it was at the first on that side from whence we draw it, and so it cometh in the end to a middle between both, which is perfect straightness. Utter inconformity therefore with the church of Rome in these things is the best and surest policy which the Church can use. While we use their ceremonies they take occasion to blaspheme, saying, that our religion cannot stand by itself, unless it lean upon the staff of their ceremonies. They hereby conceive great hope of having the rest of their popery in the end, which hope causeth them to be more frozen in their wickedness. Neither is it without cause that they have this hope, considering that which Master Bucer noteth upon the eighteenth of St. Matthew, that where these things have been left, popery hath returned; but on the other part in places which have been cleansed of these things, it hath not yet been seen that it hath had any entrance. None make such clamours for these ceremonies, as the papists and those whom they suborn; a manifest token how much they triumph and joy in these things. They breed grief of mind in a number, that are godly-minded and have anti-christianity in such detestation, that their minds are martyred with the very sight of them in the Church. Such godly brethren we ought not thus to grieve with unprofitable ceremonies, yea, ceremonies wherein there is not only no profit, but also danger of great hurt, that may grow to the Church by infection, which popish ceremonies are means to breed.”

This in effect is the sum and substance of that which they bring by way of opposition against those orders which we have common with the church of Rome; these are the reasons wherewith they would prove our ceremonies in that respect worthy of blame.

\section*{How when they go about to expound what Popish Ceremonies they mean, they contradict their own arguments against Popish Ceremonies.}

IV. Before we answer unto these things, we are to cut off that whereunto they from whom these objections proceed do oftentimes fly for defence and succour, when the force and  strength of their arguments is elided. For the ceremonies in use amongst us being in no other respect retained, saving only for that to retain them is to our seeming good and profitable, yea, so profitable and so good, that if we had either simply taken them clean away, or else removed them so as to place in their stead others, we had done worse: the plain and direct way against us herein had been only to prove, that all such ceremonies as they require to be abolished are retained by us to the hurt of the Church, or with less benefit than the abolishment of them would bring. But forasmuch as they saw how hardly they should be able to perform this, they took a more compendious way, traducing the ceremonies of our church under the name of being popish. The cause why this way seemed better unto them was, for that the name of popery is more odious than very paganism amongst divers of the more simple sort, so as whatsoever they hear named popish, they presently conceive deep hatred against it, imagining there can be nothing contained in that name but needs it must be exceeding detestable. The ears of the people they have therefore filled with strong clamour: “The Church of England is fraught with popish ceremonies: they that favour the cause of reformation maintain nothing but the sincerity of the Gospel of Jesus Christ: all such as withstand them fight for the laws of his sworn enemy, uphold the filthy relics of Antichrist, and are defenders of that which is popish.” These are the notes wherewith are drawn from the hearts of the multitude so many sighs; with these tunes their minds are exasperated against the lawful guides and governors of their souls; these are the voices that fill them with general discontentment, as though the bosom of that famous church wherein they live were more noisome than any dungeon. But when the authors of so scandalous incantations are examined, and called to account how can they justify such their dealings; when they are urged directly to answer, whether it be lawful for us to use any such ceremonies as the church of Rome useth, although the same be not commanded in the word of God; being driven to see that the use of some such ceremonies must of necessity be granted lawful, they go about to make us believe that they are just of the same opinion, and that they only think such ceremonies are not to be used when they are  unprofitable, or “when as good or better may be established.” Which answer is both idle in regard of us, and also repugnant to themselves.

[2.]It is in regard of us very vain to make this answer, because they know that what ceremonies we retain common unto the church of Rome, we therefore retain them, for that we judge them to be profitable, and to be such that others instead of them would be worse. So that when they say that we ought to abrogate such Romish ceremonies as are unprofitable, or else might have other more profitable in their stead, they trifle and they beat the air about nothing which toucheth us; unless they mean that we ought to abrogate all Romish ceremonies which in their judgment have either no use or less use than some other might have. But then must they shew some commission, whereby they are authorized to sit as judges, and we required to take their judgment for good in this case. Otherwise their sentences will not be greatly regarded, when they oppose their methinketh unto the orders of the Church of England: as in the question about surplices one of them doth; “If we look to the colour, black methinketh is more decent; if to the form, a garment down to the foot hath a great deal more comeliness in it.” If they think that we ought to prove the ceremonies commodious which we have retained, they do in this point very greatly deceive themselves. For in all right and equity, that which the Church hath received and held so long for good, that which public approbation hath ratified, must carry the benefit of presumption with it to be accounted meet and convenient. They which have stood up as yesterday to challenge it of defect, must prove their challenge. If we being defendants do answer, that the ceremonies in question are godly, comely, decent, profitable for the Church; their reply is childish and unorderly, to say, that we demand the thing in question, and shew the poverty  of our cause, the goodness whereof we are fain to beg that our adversaries would grant. For on our part this must be the answer, which orderly proceeding doth require. The burden of proving doth rest on them. In them it is frivolous to say, we ought not to use bad ceremonies of the church of Rome, and presume all such bad as it pleaseth themselves to dislike, unless we can persuade them the contrary.

[3.]Besides, they are herein opposite also to themselves. For what one thing is so common with them, as to use the custom of the church of Rome for an argument to prove, that such and such ceremonies cannot be good and profitable for us, inasmuch as that church useth them? Which usual kind of disputing sheweth, that they do not disallow only those Romish ceremonies which are unprofitable, but count all unprofitable which are Romish; that is to say, which have been devised by the church of Rome, or which are used in that church and not prescribed in the word of God. For this is the only limitation which they can use suitable unto their other positions. And therefore the cause which they yield, why they hold it lawful to retain in doctrine and in discipline some things as good, which yet are common to the church of Rome, is for that those good things are “perpetual commandments in whose place no other can come;” but ceremonies are changeable. So that their judgment in truth is, that whatsoever by the word of God is not unchangeable in the church of Rome, that church’s using is a cause why reformed churches ought to change it, and not to think it good or profitable. And lest we seem to father any thing upon them more than is properly their own, let them read even their own words, where they complain, “that we are thus constrained to be like unto the Papists in Any their ceremonies;” yea, they urge that this cause, although it were "alone, ought to move them to whom that belongeth to do them away, forasmuch as they are their ceremonies;” and that the Bishop of Salisbury doth justify this their complaint.  The clause is untrue which they add concerning the Bishop of Salisbury; but the sentence doth shew that we do them no wrong in setting down the state of the question between us thus: Whether we ought to abolish out of the church of England all such orders, rites, and ceremonies as are established in the Church of Rome, and are not prescribed in the word of God. For the affirmative whereof we are now to answer such proofs of theirs as have been before alleged.

\section*{An answer to the argument whereby they would prove, that sith we allow the customs of our fathers to be followed, we therefore may not allow such customs as the Church of Rome hath, because we cannot account of them which are of that Church as of our fathers.}

V. Let the church of Rome be what it will, let them that are of it be the people of God and our fathers in the Christian faith, or let them be otherwise; hold them for catholics or hold them for heretics; it is not a thing either one way or other in this present question greatly material. Our conformity with them in such things as have been proposed is not proved as yet unlawful by all this. St. Augustine hath said, yea and we have allowed his saying, “That the custom of the people of God and the decrees of our forefathers are  to be kept, touching those things whereof the Scripture hath neither one way nor other given us any charge.” What then? Doth it here therefore follow, that they being neither the people of God nor our forefathers, are for that cause in nothing to be followed? This consequent were good if so be it were granted, that only the custom of the people of God and the decrees of our forefathers are in such case to be observed. But then should no other kind of later laws in the Church be good; which were a gross absurdity to think. St. Augustine’s speech therefore doth import, that where we have no divine precept, if yet we have the custom of the people of God or a decree of our forefathers, this is a law and must be kept. Notwithstanding it is not denied, but that we lawfully may observe the positive constitutions of our own churches, although the same were but yesterday made by ourselves alone. Nor is there any thing in this to prove, that the church of England might not by law receive orders, rites, or customs from the church of Rome, although they were neither the people of God nor yet our forefathers. How much less when we have received from them nothing, but that which they did themselves receive from such, as we cannot deny to have been the people of God, yea such, as either we must acknowledge for our own forefathers or else disdain the race of Christ?

\section*{To their allegation, that the course of God’s own wisdom doth make against our conformity with the Church of Rome in such things.}

VI. The rites and orders wherein we follow the church of Rome are of no other kind than such as the church of Geneva itself doth follow them in. We follow the church of Rome in more things; yet they in some things of the same nature about which our present controversy is: so that the difference is not in the kind, but in the number of rites only, wherein they and we do follow the church of Rome. The use of wafer-cakes, the custom of godfathers and godmothers in baptism, are things not commanded nor forbidden in Scripture, things which have been of old and are retained in the church of Rome even at this very hour. Is conformity with Rome in such things a blemish unto the church of England, and unto churches abroad an ornament? Let them, if not for the reverence they owe unto this church, in the bowels whereof they have received I trust that precious and blessed vigour, which shall quicken them to eternal life, yet at the  leastwise for the singular affection which they do bear towards others, take heed how they strike, lest they wound whom they would not. For undoubtedly it cutteth deeper than they are aware of, when they plead that even such ceremonies of the church of Rome, as contain in them nothing which is not of itself agreeable to the word of God, ought nevertheless to be abolished; and that neither the word of God, nor reason, nor the examples of the eldest churches do permit the church of Rome to be therein followed.

[2.]Heretics they are, and they are our neighbours. By us and amongst us they lead their lives. But what then? therefore no ceremony of theirs lawful for us to use? We must yield and will that none are lawful, if God himself be a precedent against the use of any. But how appeareth it that God is so? Hereby they say it doth appear, in that “God severed his people from the heathens, but especially from the Egyptians, and such nations as were nearest neighbours unto them, by forbidding them to do those things which were in themselves very lawful to be done, yea, very profitable some, and incommodious to be forborne; such things it pleased God to forbid them, only because those heathens did them, with whom conformity in the same things might have bred infection. Thus in shaving, cutting, apparel-wearing, yea in sundry kinds of meats also, swine’s flesh, conies, and such like, they were forbidden to do so and so, because the Gentiles did so. And the end why God forbade them such things was to sever them for fear of infection by a great and an high wall from other nations, as St. Paul teacheth.” The cause of more careful separation from the nearest nations was the greatness of danger to be especially by them infected. Now papists are to us as those nations were unto Israel. Therefore if the wisdom of God be our guide, we cannot allow conformity with them, no not in any such indifferent ceremony.

[3.]Our direct answer hereunto is, that for any thing here alleged we may still doubt, whether the Lord in such indifferent ceremonies, as those whereof we dispute, did frame his  people of set purpose unto any utter dissimilitude, either with Egyptians or with any other nation else. And if God did not forbid them all such indifferent ceremonies, then our conformity with the church of Rome in some such is not hitherto as yet disproved, although papists were unto us as those heathens were unto Israel. “After the doings of the land of Egypt, wherein you dwelt, ye shall not do, saith the Lord; and after the manner of the land of Canaan, whither I will bring you, shall ye not do, neither walk in their ordinances: do after my judgments, and keep my ordinances to walk therein: I am the Lord your God.” The speech is indefinite, “ye shall not be like them:” it is not general, “ye shall not be like them in any thing, or like to them in any thing indifferent, or like unto them in any indifferent ceremony of theirs.” Seeing therefore it is not set down how far the bounds of his speech concerning dissimilitude should reach, how can any man assure us, that it extendeth farther than to those things only, wherein the nations there mentioned were idolatrous, or did against that which the law of God commandeth? Nay, doth it not seem a thing very probable, that God doth purposely add, “Do after my judgments,” as giving thereby to understand that his meaning in the former sentence was but to bar similitude in such things, as were repugnant unto the ordinances, laws, and statutes which he had given? Egyptians and Canaanites are for example’s sake named unto them, because the customs of the one they had been, and of the other they should be best acquainted with. But that wherein they might not be like unto either of them, was such peradventure as had been no whit less unlawful, although those nations had never been. So that there is no necessity to think, that God for fear of infection by reason of nearness forbade them to be like unto the Canaanites or the Egyptians, in those things which otherwise had been lawful enough.

For I would know what one thing was in those nations, and is here forbidden, being indifferent in itself, yet forbidden only because they used it. In the laws of Israel we find it written, “Ye shall not cut round the corners of your heads, neither shalt thou tear the tufts of thy beard.” These  things were usual amongst those nations, and in themselves they are indifferent. But are they indifferent being used as signs of immoderate and hopeless lamentation for the dead? In this sense it is that the law forbiddeth them. For which cause the very next words following are, “Ye shall not cut your flesh for the dead, nor make any print of a mark upon you: I am the Lord.” The like in Leviticus, where speech is of mourning for the dead; “They shall not make bald parts upon their head, nor shave off the locks of their beard, nor make any cutting in their flesh.” Again in Deuteronomy, “Ye are the children of the Lord your God; ye shall not cut yourselves, nor make you baldness between your eyes for the dead.” What is this but in effect the same which the Apostle doth more plainly express, saying, “Sorrow not as they do who have no hope?” The very light of nature itself was able to see herein a fault; that which those nations did use, having been also in use with others, the ancient Roman laws do forbid. That shaving therefore and cutting which the law doth mention was not a matter in itself indifferent, and forbidden only because it was in use amongst such idolaters as were neighbours to the people of God; but to use it had been a crime, though no other people or nation under heaven should have done it saving only themselves.

As for those laws concerning attire: “There shall no garment of linen and woollen come upon thee;” as also those touching food and diet, wherein swine’s flesh together with sundry other meats are forbidden; the use of these things had been indeed of itself harmless and indifferent: so that hereby it doth appear, how the law of God forbade in some special consideration such things as were lawful enough in themselves. But yet even here they likewise fail of that they intend. For it doth not appear that the consideration in regard whereof the law forbiddeth these things was because those nations did use them. Likely enough it is that the  Canaanites used to feed as well on sheep’s as on swine’s flesh; and therefore if the forbidding of the later had no other reason than dissimilitude with that people, they which of their own heads allege this for reason can shew I think some reason more than we are able to find why the former was not also forbidden. Might there not be some other mystery in this prohibition than they think of? Yes, some other mystery there was in it by all likelihood. For what reason is there which should but induce, and therefore much less enforce us to think, that care of dissimilitude between the people of God and the heathen nations about them, was any more the cause of forbidding them to put on garments of sundry stuff, than of charging them withal not to sow their fields with meslin; or that this was any more the cause of forbidding them to eat swine’s flesh, than of charging them withal not to eat the flesh of eagles, hawks, and the like?

Wherefore, although the church of Rome were to us, as to Israel the Egyptians and Canaanites were of old; yet doth it not follow, that the wisdom of God without respect doth teach us to erect between us and them a partition-wall of difference, in such things indifferent as have been hitherto disputed of.

\section*{To the example of the eldest Churches which they bring for the same purpose.}

VII. Neither is the example of the eldest churches a whit more available to this purpose. Notwithstanding some fault undoubtedly there is in the very resemblance of idolaters. Were it not some kind of blemish to be like unto infidels and heathens, it would not so usually be objected; men would not think it any advantage in the causes of religion to be able therewith justly to charge their adversaries as they do. Wherefore to the end that it may a little more plainly appear, what force this hath and how far the same extendeth, we are to note how all men are naturally desirous that they may seem neither to judge nor to do amiss; because every error and offence is a stain to the beauty of nature, for which cause  it blusheth thereat, but glorieth in the contrary. From thence it riseth, that they which disgrace or depress the credit of others do it either in both or in one of these. To have been in either directed by a weak and unperfect rule argueth imbecility and imperfection. Men being either led by reason or by imitation of other men’s example, if their persons be odious whose example we choose to follow, as namely if we frame our opinions to that which condemned heretics think, or direct our actions according to that which is practised and done by them; it lieth as an heavy prejudice against us, unless somewhat mightier than their bare example did move us, to think or do the same things with them. Christian men therefore having besides the common light of all men so great help of heavenly direction from above, together with the lamps of so bright examples as the Church of God doth yield, it cannot but worthily seem reproachful for us to leave both the one and the other, to become disciples unto the most hateful sort that live, to do as they do, only because we see their example before us and have a delight to follow it. Thus we may therefore safely conclude, that it is not evil simply to concur with the heathens either in opinion or in action; and that conformity with them is only then a disgrace, when either we follow them in that they think and do amiss, or follow them generally in that they do without other reason than only the liking we have to the pattern of their example; which liking doth intimate a more universal approbation of them than is allowable.

[2.]Faustus the Manichee therefore objecting against the Jews, that they forsook the idols of the Gentiles, but their temples and oblations and altars and priesthoods and all kinds of ministry of holy things they exercised even as the Gentiles did, yea, more superstitiously a great deal; against the Catholic Christians likewise, that between them and the heathens there was in many things little difference; “From them,” saith Faustus, “ye have learned to hold that one only God is the author of all; their sacrifices ye have turned into feasts of charity, their idols into martyrs whom ye honour with the like religious offices unto theirs; the ghosts of the dead ye appease with wine and delicates; the festival days of the nations ye celebrate together with them; and of their kind  of life ye have verily changed nothing:” St. Augustine’s defence in behalf of both is, that touching matters of action, Jews and Catholic Christians were free from the Gentiles’ faultiness, even in those things which were objected as tokens of their agreement with Gentiles: and concerning their consent in opinion, they did not hold the same with Gentiles because Gentiles had so taught, but because heaven and earth had so witnessed the same to be truth, that neither the one sort could err in being fully persuaded thereof, nor the other but err in case they should not consent with them.

[3.]In things of their own nature indifferent, if either councils or particular men have at any time with sound judgment misliked conformity between the Church of God and infidels, the cause thereof hath been somewhat else than only affectation of dissimilitude. They saw it necessary so to do in respect of some special accident, which the Church being not always subject unto hath not still cause to do the like. For example, in the dangerous days of trial, wherein there was no way for the truth of Jesus Christ to triumph over infidelity but through the constancy of his saints, whom yet a natural desire to save themselves from the flame might peradventure cause to join with Pagans in external customs, too far using the same as a cloak to conceal themselves in, and a mist to darken the eyes of infidels withal: for remedy hereof those laws it might be were provided, which forbad that Christians should deck their houses with boughs as the Pagans did use to do, or rest those festival days whereon  the Pagans rested, or celebrate such feasts as were, though not heathenish, yet such as the simpler sort of heathens might be beguiled in so thinking them.

[4.]As for Tertullian’s judgment concerning the rites and orders of the Church, no man having judgment can be ignorant how just exceptions may be taken against it. His opinion touching the Catholic Church was as unindifferent as touching our church the opinion of them that favour this pretended reformation is. He judged all them who did not Montanize to be but carnally minded, he judged them still over-abjectly to fawn upon the heathens, and to curry favour with infidels. Which as the catholic church did well provide that they might not do indeed, so Tertullian over-often through discontentment carpeth injuriously at them as though they did it, even when they were free from such meaning.

[5.]But if it were so, that either the judgment of these councils before alleged, or of Tertullian himself against the Christians, are in no such consideration to be understood as we have mentioned; if it were so that men are condemned as well of the one as of the other, only for using the ceremonies of a religion contrary unto their own, and that this cause is such as ought to prevail no less with us than with them: shall it not follow that seeing there is still between our religion and Paganism the selfsame contrariety, therefore we are still no less rebukeable, if we now deck our houses with boughs, or send new-year’s gifts unto our friends, or feast on those days which the Gentiles then did, or sit after prayer as they were accustomed? For so they infer upon the premises, that as great difference as commodiously may be, there should be in all outward ceremonies between the people of God and them which are not his people. Again they teach as hath been declared, that there is not as great a difference  as may be between them, except the one do avoid whatsoever rites and ceremonies uncommanded of God the other doth embrace. So that generally they teach that the very difference of spiritual condition itself between the servants of Christ and others requireth such difference in ceremonies between them, although the one be never so far disjoined in time or place from the other.

[6.]But in case the people of God and Belial do chance to be neighbours, then as the danger of infection is greater, so the same difference they say is thereby made more necessary. In this respect as the Jews were severed from the heathen, so most especially from the heathen nearest them. And in the same respect we, which ought to differ howsoever from the church of Rome, are now they say by reason of our nearness more bound to differ from them in ceremonies than from Turks. A strange kind of speech unto Christian ears, and such as I hope they themselves do acknowledge unadvisedly uttered. “We are not so much to fear infection from Turks as from papists.” What of that? we must remember that by conforming rather ourselves in that respect to Turks, we should be spreaders of a worse infection into others than any we are likely to draw from papists by our conformity with them in ceremonies. If they did hate, as Turks do, the Christians; or as Canaanites did of old the Jewish religion even in gross; the circumstance of local nearness in them unto us might haply enforce in us a duty of greater separation from them than from those other mentioned. But forasmuch as papists are so much in Christ nearer unto us than Turks, is there any reasonable man, trow you, but will judge it meeter that our ceremonies of Christian religion should be popish than Turkish or heathenish? Especially considering that we were not brought to dwell amongst them, (as Israel in Canaan,) having not been of them. For even a very part of them we were. And when God did by his good Spirit put it into our hearts, first to reform ourselves, (whence grew our separation,) and then by all good means to seek also their reformation; had we not only cut off their corruptions but also estranged ourselves from them in things indifferent, who seeth not how greatly prejudicial this might have been to  so good a cause, and what occasion it had given them to think (to their greater obduration in evil) that through a froward or wanton desire of innovation we did unconstrainedly those things for which conscience was pretended? Howsoever the case doth stand, as Juda had been rather to choose conformity in things indifferent with Israel when they were nearest opposites, than with the farthest removed Pagans; so we in the like case much rather with papists than with Turks. I might add further for more full and complete answer, so much concerning the large odds between the case of the eldest churches in regard of those heathens and ours in respect of the church of Rome, that very cavillation itself should be satisfied, and have no shift to fly unto.

\section*{That it is not our best polity (as they pretend it is) for establishment of sound religion, to have in these things no agreement with the Church of Rome being unsound.}

VIII. But that no one thing may detain us over long, I return to their reasons against our conformity with that church. That extreme dissimilitude which they urge upon us, is now commended as our best and safest policy for establishment of sound religion. The ground of which politic position is that “evils must be cured by their contraries;” and therefore the cure of the Church infected with the poison of Antichristianity must be done by that which is thereunto as contrary as may be. “A medled estate of the orders of the Gospel and the ceremonies of popery is not the best way to banish popery.”

We are contrariwise of opinion, that he which will perfectly recover a sick and restore a diseased body unto health, must not endeavour so much to bring it to a state of simple contrariety, as of fit proportion in contrariety unto those evils which are to be cured. He that will take away extreme heat by setting the body in extremity of cold, shall undoubtedly remove the disease, but together with it the diseased too. The first thing therefore in skilful cures is the knowledge of the part affected; the next is of the evil which doth affect it; the last is not only of the kind but also of the measure of contrary things whereby to remove it.


[2.]They which measure religion by dislike of the church of Rome think every man so much the more sound, by how much he can make the corruptions thereof to seem more large. And therefore some there are, namely the Arians in reformed churches of Poland, which imagine the canker to have eaten so far into the very bones and marrow of the church of Rome, as if it had not so much as a sound belief, no not concerning God himself, but that the very belief of the Trinity were a part of antichristian corruption; and that the wonderful providence of God did bring to pass that the bishop of the see of Rome should be famous for his triple crown; a sensible mark whereby the world might know him to be that mystical beast spoken of in the Revelation, to be that great and notorious Antichrist in no one respect so much as in this, that he maintaineth the doctrine of the Trinity. Wisdom therefore and skill is requisite to know, what parts are sound in that church, and what corrupted.

Neither is it to all men apparent which complain of unsound parts, with what kind of unsoundness every such part is possessed. They can say, that in doctrine, in discipline, in prayers, in sacraments, the church of Rome hath (as it hath indeed) very foul and gross corruptions; the nature whereof notwithstanding because they have not for the most part exact skill and knowledge to discern, they think that amiss many times which is not; and the salve of reformation they mightily call for, but where and what the sores are which need it, as they wot full little, so they think it not greatly material to search. Such men’s contentment must be wrought by stratagem; the usual method of art is not for them.

[3.]But with those that profess more than ordinary and common knowledge of good from evil, with them that are able to put a difference between things naught and things indifferent in the church of Rome, we are yet at controversy about the manner of removing that which is naught; whether it may not be perfectly helped, unless that also which is indifferent be cut off with it, so far till no rite or ceremony remain which the church of Rome hath, being not found in the word of God. If we think this too extreme, they reply, that to draw men from great excess, it is not amiss though we  use them unto somewhat less than is competent; and that a crooked stick is not straightened unless it be bent as far on the clean contrary side, that so it may settle itself at the length in a middle estate of evenness between both. But how can these comparisons stand them in any stead? When they urge us to extreme opposition against the church of Rome, do they mean we should be drawn unto it only for a time, and afterwards return to a mediocrity? or was it the purpose of those reformed churches, which utterly abolished all popish ceremonies, to come in the end back again to the middle point of evenness and moderation? Then have we conceived amiss of their meaning. For we have always thought their opinion to be, that utter inconformity with the church of Rome was not an extremity whereunto we should be drawn for a time, but the very mediocrity itself wherein they meant we should ever continue. Now by these comparisons it seemeth clean contrary, that howsoever they have bent themselves at first to an extreme contrariety against the Romish church, yet therein they will continue no longer than only till such time as some more moderate course for establishment of the Church may be concluded.

[4.]Yea, albeit this were not at the first their intent, yet surely now there is great cause to lead them unto it. They have seen that experience of the former policy, which may cause the authors of it to hang down their heads. When Germany had stricken off that which appeared corrupt in the doctrine of the church of Rome, but seemed nevertheless in discipline still to retain therewith very great conformity; France by that rule of policy which hath been before mentioned, took away the popish orders which Germany did retain. But process of time hath brought more light into the world; whereby men perceiving that they of the religion in France have also retained some orders which were before  in the church of Rome, and are not commanded in the word of God, there hath arisen a sect in England, which following still the very selfsame rule of policy, seeketh to reform even the French reformation, and purge out from thence also dregs of popery. These have not taken as yet such root that they are able to establish any thing. But if they had, what would spring out of their stock, and how far the unquiet wit of man might be carried with rules of such policy, God doth know. The trial which we have lived to see, may somewhat teach us what posterity is to fear. But our Lord of his infinite mercy avert whatsoever evil our swervings on the one hand or on the other may threaten unto the state of his Church!

\section*{That neither the Papists upbraiding us as furnished out of their store, nor any hope which in that respect they are said to conceive, doth make any more against our ceremonies than the former allegations have done.}

IX. That the church of Rome doth hereby take occasion to blaspheme, and to say, our religion is not able to stand of itself unless it lean upon the staff of their ceremonies, is not a matter of so great moment, that it did need to be objected, or doth deserve to receive an answer. The name of blasphemy in this place, is like the shoe of Hercules on a child’s foot. If the church of Rome do use any such kind of silly exprobration, it is no such ugly thing to the ear, that we should think the honour and credit of our religion to receive thereby any great wound. They which hereof make so perilous a matter do seem to imagine, that we have erected of late a frame of some new religion, the furniture whereof we should not have borrowed from our enemies, lest they relieving us might afterwards laugh and gibe at our poverty; whereas in truth the ceremonies which we have taken from such as were before us, are not things that belong to this or that sect, but they are the ancient rites and customs of the Church of Christ, whereof ourselves being a part, we have the selfsame interest in them which our fathers before us had, from whom the same are descended unto us. Again, in case we had been so much beholding privately unto them, doth the reputation to one church stand by saying unto another,  “I need thee not?” If some should be so vain and impotent as to mar a benefit with reproachful upbraiding, where at the least they suppose themselves to have bestowed some good turn; yet surely a wise body’s part it were not, to put out his fire, because his fond and foolish neighbour, from whom he borrowed peradventure wherewith to kindle it, might haply cast him therewith in the teeth, saying, “Were it not for me thou wouldest freeze, and not be able to heat thyself.”

[2.]As for that other argument derived from the secret affection of papists, with whom our conformity in certain ceremonies is said to put them in great hope, that their whole religion in time will have re-entrance, and therefore none are so clamorous amongst us for the observation of these ceremonies, as papists and such as papists suborn to speak for them, whereby it clearly appeareth how much they rejoice, how much they triumph in these things; our answer hereunto is still the same, that the benefit we have by such ceremonies overweigheth even this also. No man which is not exceeding partial can well deny, but that there is most just cause wherefore we should be offended greatly at the church of Rome. Notwithstanding at such times as we are to deliberate for ourselves, the freer our minds are from all distempered affections, the sounder and better is our judgment. When we are in a fretting mood at the church of Rome, and with that angry disposition enter into any cogitation of the orders and rites of our church; taking particular survey of them, we are sure to have always one eye fixed upon the countenance of our enemies, and according to the blithe or heavy aspect thereof, our other eye sheweth some other suitable token either of dislike or approbation towards our own orders. For the rule of our judgment in such case being only that of Homer, “This is the thing which our enemies would have;” what they seem contented with, even for that very cause we reject: and there is nothing but it pleaseth us much the better if we espy that it galleth them. Miserable were the state and condition of that church, the  weighty affairs whereof should be ordered by those deliberations wherein such a humour as this were predominant. We have most heartily to thank God therefore, that they amongst us to whom the first consultations of causes of this kind fell, were men which aiming at another mark, namely the glory of God and the good of this his church, took that which they judged thereunto necessary, not rejecting any good or convenient thing only because the church of Rome might perhaps like it. If we have that which is meet and right, although they be glad, we are not to envy them this their solace; we do not think it a duty of ours to be in every such thing their tormentors.

[3.]And whereas it is said that popery for want of this utter extirpation hath in some places taken root and flourished again, but hath not been able to re-establish itself in any place after provision made against it by utter evacuation of all Romish ceremonies: and therefore, as long as we hold any thing like unto them, we put them in some more hope than if all were taken away: as we deny not but this may be true, so being of two evils to choose the less, we hold it better that the friends and favourers of the church of Rome should be in some kind of hope to have a corrupt religion restored, than both we and they conceive just fear, lest under colour of rooting out popery, the most effectual means to bear up the state of religion be removed, and so a way made either for Paganism or for extreme barbarity to enter. If desire of weakening the hope of others should turn us away from the course we have taken; how much more the care of preventing our own fear withhold us from that we are urged unto! Especially seeing that our own fear we know, but we are not so certain what hope the rites and orders of our church have bred in the hearts of others.

For it is no sufficient argument thereof to say, that in  maintaining and urging these ceremonies none are so clamorous as papists and they whom papists suborn; this speech being more hard to justify than the former, and so their proof more doubtful than the thing itself which they prove. He that were certain that this is true, must have marked who they be that speak for ceremonies; he must have noted who amongst them doth speak oftenest, or is most earnest; he must have been both acquainted throughly with the religion of such, and also privy what conferences or compacts are passed in secret between them and others; which kinds of notice are not wont to be vulgar and common. Yet they which allege this would have it taken as a thing that needeth no proof, a thing which all men know and see.

And if so be it were granted them as true, what gain they by it? Sundry of them that be popish are eager in maintenance of ceremonies. Is it so strange a matter to find a good thing furthered by ill men of a sinister intent and purpose, whose forwardness is not therefore a bridle to such as favour the same cause with a better and sincerer meaning? They that seek, as they say, the removing of all popish orders out of the Church, and reckon the state of Bishops in the number of those orders, do (I doubt not) presume that the cause which they prosecute is holy. Notwithstanding it is their own ingenuous acknowledgment, that even this very cause, which they term so often by an excellency, “The Lord’s cause,” is “gratissima, most acceptable, unto some which hope for prey and spoil by it, and that our age hath store of such, and that such are the very sectaries of Dionysius the famous atheist.” Now if hereupon we should upbraid them with irreligious, as they do us with superstitious favourers; if we should follow them in their own kind of pleading, and say, that the most clamorous for this pretended reformation are either atheists, or else proctors suborned by atheists; the answer which herein they  would make unto us, let them apply unto themselves, and there an end. For they must not forbid us to presume our cause in defence of our church orders to be as good as theirs against them, till the contrary be made manifest to the world.

\section*{The grief which they say godly brethren conceive at such ceremonies as we have common with the Church of Rome.}

X. In the meanwhile sorry we are that any good and godly mind should be grieved with that which is done. But to remedy their grief lieth not so much in us as in themselves. They do not wish to be made glad with the hurt of the Church: and to remove all out of the Church whereat they shew themselves to be sorrowful, would be, as we are persuaded, hurtful if not pernicious thereunto. Till they be able to persuade the contrary, they must and will I doubt not find out some other good means to cheer up themselves. Amongst which means the example of Geneva may serve for one. Have not they the old popish custom of using godfathers and godmothers in Baptism? the old popish custom of administering the blessed sacrament of the holy Eucharist with wafer-cakes? These things the godly there can digest. Wherefore should not the godly here learn to do the like both in them and in the rest of the like nature? Some further mean peradventure it might be to assuage their grief, if so be they did consider the revenge they take on them which have been, as they interpret it, the workers of their continuance in so great grief so long. For if the maintenance of ceremonies be a corrosive to such as oppugn them, undoubtedly to such as maintain them it can be no great pleasure, when they behold how that which they reverence is oppugned. And therefore they that judge themselves martyrs when they are grieved, should think withal what they are whom they grieve. For we are still to put them in mind that the cause  doth make no difference; for that it must be presumed as good at the least on our part as on theirs, till it be in the end decided who have stood for truth and who for error. So that till then the most effectual medicine and withal the most sound to ease their grief, must not be (in our opinion) the taking away of those things whereat they are grieved, but the altering of that persuasion which they have concerning the same.

[2.]For this we therefore both pray and labour; the more because we are also persuaded, that it is but conceit in them to think, that those Romish ceremonies whereof we have hitherto spoken, are like leprous clothes, infectious unto the Church, or like soft and gentle poisons, the venom whereof being insensibly pernicious, worketh death, and yet is never felt working. Thus they say: but because they say it only, and the world hath not as yet had so great experience of their art in curing the diseases of the Church, that the bare authority of their word should persuade in a cause so weighty, they may not think much if it be required at their hands to shew, first, by what means so deadly infection can grow from similitude between us and the church of Rome in these things indifferent: secondly, for that it were infinite if the Church should provide against every such evil as may come to pass, it is not sufficient that they shew possibility of dangerous event, unless there appear some likelihood also of the same to follow in us, except we prevent it. Nor is this enough, unless it be moreover made plain, that there is no good and sufficient way of prevention, but by evacuating clean, and by emptying the Church of every such rite and ceremony, as is presently  called in question. Till this be done, their good affection towards the safety of the Church is acceptable, but the way they prescribe us to preserve it by must rest in suspense.

[3.]And lest hereat they take occasion to turn upon us the speech of the prophet Jeremy used against Babylon, “Behold we have done our endeavour to cure the diseases of Babylon, but she through her wilfulness doth rest uncured;” let them consider into what straits the Church might drive itself in being guided by this their counsel. Their axiom is, that the sound believing Church of Jesus Christ may not be like heretical churches in any of those indifferent things, which men make choice of, and do not take by prescript appointment of the word of God. In the word of God the use of bread is prescribed, as a thing without which the Eucharist may not be celebrated; but as for the kind of bread it is not denied to be a thing indifferent. Being indifferent of itself, we are by this axiom of theirs to avoid the use of unleavened bread in that sacrament, because such bread the church of Rome being heretical useth. But doth not the selfsame axiom bar us even from leavened bread also, which the church of the Grecians useth; the opinions whereof are in a number of things the same for which we condemn the church of Rome, and in some things erroneous where the church of Rome is acknowledged to be sound; as namely, in the article about proceeding of the Holy Ghost? And lest here they should say that because the Greek church is farther off, and the church of Rome nearer, we are in that respect rather to use that which the church of Rome useth not: let them imagine a reformed church in the city of Venice, where a Greek church and a popish both are. And when both these are equally near let them consider what the third shall do. Without either leavened or unleavened bread, it can have no sacrament; the word of God doth tie it to neither; and their axiom doth exclude it from both. If this constrain them, as it must, to grant that their axiom is not to take any place save in those things only where the Church hath larger scope; it resteth that they search out some stronger reason than they have as yet alleged; otherwise they constrain not us to think that the Church is tied unto any such rule or axiom, no not then when  she hath the widest field to walk in, and the greatest store of choice.

\section*{The third thing for which they reprove a great part of our ceremonies is, for that as we have them from the Church of Rome, so that Church had them from the Jews.}

XI. Against such ceremonies generally as are the same in the church of England and of Rome, we see what hath been hitherto alleged. Albeit therefore we do not find the one church’s having of such things to be sufficient cause why the other should not have them: nevertheless, in case it may be proved, that amongst the number of rites and orders common unto both, there are particulars, the use whereof is utterly unlawful in regard of some special bad and noisome quality; there is no doubt but we ought to relinquish such rites and orders, what freedom soever we have to retain the other still. As therefore we have heard their general exception against all those things, which being not commanded in the word of God, were first received in the church of Rome, and from thence have been derived into ours; so it followeth that now we proceed unto certain kinds of them, as being excepted against not only for that they are in the church of Rome, but are besides either Jewish, or abused unto idolatry, and so grown scandalous.

[2.]The church of Rome, they say, being ashamed of the simplicity of the gospel, did almost out of all religions take whatsoever had any fair and gorgeous show, borrowing in that respect from the Jews sundry of their abolished ceremonies. Thus by foolish and ridiculous imitation, all their massing furniture almost they took from the Law, lest having an altar and a priest, they should want vestments for their stage; so that whatsoever we have in common with the church of Rome, if the same be of this kind we ought to remove it. “Constantine the emperor speaking of the keeping of the feast of Easter, saith, ‘That it is an unworthy thing to have any thing common with that most spiteful company of the Jews.’ And a little after he saith, ‘That it is most absurd and against reason, that the Jews should  vaunt and glory that the Christians could not keep those things without their doctrine.’ And in another place it is said after this sort; ‘It is convenient so to order the matter, that we have nothing common with that nation.’ The council of Laodicea, which was afterwards confirmed by the sixth general council, decreed ‘that the Christians should not take unleavened bread of the Jews, or communicate with their impiety.’ ”

[3.]For the easier manifestation of truth in this point, two things there are which must be considered: namely, the causes wherefore the Church should decline from Jewish ceremonies; and how far it ought so to do. One cause is that the Jews were the deadliest and spitefullest enemies of Christianity that were in the world, and in this respect their orders so far forth to be shunned, as we have already set down in handling the matter of heathenish ceremonies. For no enemies being so venomous against Christ as Jews, they were of all other most odious, and by that mean least to be used as fit church-patterns for imitation. Another cause is the solemn abrogation of the Jews’ ordinances; which ordinances for us to resume, were to check our Lord himself which hath disannulled them. But how far this second cause doth extend, it is not on all sides fully agreed upon. And touching those things whereunto it reacheth not, although there be small cause wherefore the Church should frame itself to the Jews’ example in respect of their persons which are most hateful; yet God himself having been the author of their laws, herein they are (notwithstanding the former consideration) still worthy to be honoured, and to be followed above others, as much as the state of things will bear.

[4.]Jewish ordinances had some things natural, and of the perpetuity of those things no man doubteth. That which was positive we likewise know to have been by the coming of Christ partly necessary not to be kept, and partly indifferent to be kept or not. Of the former kind circumcision and  sacrifice were. For this point Stephen was accused, and the evidence which his accusers brought against him in judgment was, “This man ceaseth not to speak blasphemous words against this holy place and the Law, for we have heard him say that this Jesus of Nazareth shall destroy this place, and shall change the ordinances that Moses gave us.” True it is that this doctrine was then taught, which unbelievers condemning for blasphemy did therein commit that which they did condemn. The Apostles notwithstanding from whom Stephen had received it, did not so teach the abrogation, no not of those things which were necessarily to cease, but that even the Jews being Christian, might for a time continue in them. And therefore in Jerusalem the first Christian bishop not circumcised was Mark; and he not bishop till the days of Adrian the emperor, after the overthrow of Jerusalem: there having been fifteen bishops before him which were all of the circumcision.

The Christian Jews did think at the first not only themselves but the Christian Gentiles also bound, and that necessarily, to observe the whole Law. There went forth certain of the sect of Pharisees which did believe, and they coming unto Antioch, taught that it was necessary for the Gentiles to be circumcised, and to keep the Law of Moses. Whereupon there grew dissension, Paul and Barnabas disputing against them. The determination of the council held at Jerusalem concerning this matter was finally this; “Touching the Gentiles which believe, we have written and determined that they observe no such thing.” Their protestation by letters is, “Forasmuch as we have heard that certain which departed from us have troubled you with words, and cumbered your minds, saying, Ye must be circumcised and keep the Law; know that we gave them no such commandment.” Paul therefore continued still teaching the Gentiles, not only that they were not bound to observe the laws of Moses, but  that the observation of those laws which were necessarily to be abrogated, was in them altogether unlawful. In which point his doctrine was misreported, as though he had every where preached this, not only concerning the Gentiles, but also touching the Jews. Wherefore coming unto James and the rest of the clergy at Jerusalem, they told him plainly of it, saying, “Thou seest, brother, how many thousand Jews there are which believe, and they are all zealous of the Law. Now they are informed of thee, that thou teachest all the Jews which are amongst the Gentiles to forsake Moses, and sayest that they ought not to circumcise their children, neither to live after the customs.” And hereupon they give him counsel to make it apparent in the eyes of all men, that those flying reports were untrue, and that himself being a Jew kept the Law even as they did.

In some things therefore we see the Apostles did teach, that there ought not to be conformity between the Christian Jews and Gentiles. How many things this law of inconformity did comprehend, there is no need we should stand to examine. This general is true, that the Gentiles were not made conformable unto the Jews, in that which was necessarily to cease at the coming of Christ.

[5.]Touching things positive, which might either cease or continue as occasion should require, the Apostles tendering the zeal of the Jews, thought it necessary to bind even the Gentiles for a time to abstain as the Jews did, “from things offered unto idols, from blood, from strangled.” These decrees were every where delivered unto the Gentiles to be straitly observed and kept. In the other matters, where the Gentiles were free, and the Jews in their own opinion still tied, the Apostles’ doctrine unto the Jew was, “condemn not the Gentile;” unto the Gentile, “despise not the Jew.” The one sort they warned to take heed, that scrupulosity did not make them rigorous, in giving unadvised sentence against their brethren which were free; the other, that they did not become scandalous, by abusing their liberty and freedom to the offence of their weak brethren which were scrupulous. From hence therefore two conclusions there are which may evidently be drawn; the first, that whatsoever conformity of  positive laws the Apostles did bring in between the churches of Jews and Gentiles, it was in those things only which might either cease or continue a shorter or a longer time, as occasion did most require; the second, that they did not impose upon the churches of the Gentiles any part of the Jews’ ordinances with bond of necessary and perpetual observation, (as we all both by doctrine and practice acknowledge,) but only in respect of the conveniency and fitness for the present state of the Church as then it stood. The words of the council’s decree concerning the Gentiles are, “It seemed good to the Holy Ghost and to us, to lay upon you no more burden saving only those things of necessity, abstinence from idol-offerings, from strangled and blood, and from fornication.” So that in other things positive, which the coming of Christ did not necessarily extinguish, the Gentiles were left altogether free.

[6.]Neither ought it to seem unreasonable that the Gentiles should necessarily be bound and tied to Jewish ordinances, so far forth as that decree importeth. For to the Jew, who knew that their difference from other nations which were aliens and strangers from God, did especially consist in this, that God’s people had positive ordinances given to them of God himself, it seemed marvellous hard, that the Christian Gentiles should be incorporated into the same commonwealth with God’s own chosen people, and be subject to no part of his statutes, more than only the law of nature, which heathens count themselves bound unto. It was an opinion constantly received amongst the Jews, that God did deliver unto the sons of Noah seven precepts: namely, first, to live in some form of regiment under public laws; secondly, to serve and call upon the name of God; thirdly, to shun idolatry; fourthly, not to suffer effusion of blood; fifthly, to abhor all unclean knowledge in the flesh; sixthly, to commit no rapine; seventhly, and finally, not to eat of any living creature whereof the blood was not first let out.  If therefore the Gentiles would be exempt from the law of Moses, yet it might seem hard they should also cast off even those things positive which were observed before Moses, and which were not of the same kind with laws that were necessarily to cease. And peradventure hereupon the council saw it expedient to determine, that the Gentiles should, according unto the third, the seventh, and the fifth, of those precepts, abstain from things sacrificed unto idols, from strangled and blood, and from fornication. The rest the Gentiles did of their own accord observe, nature leading them thereto.

[7.]And did not nature also teach them to abstain from fornication? No doubt it did. Neither can we with reason think, that as the former two are positive, so likewise this, being meant as the Apostle doth otherwise usually understand it. But very marriage within a number of degrees being not only by the law of Moses, but also by the law of the sons of Noah (for so they took it) an unlawful discovery of nakedness; this discovery of nakedness by unlawful marriages such as Moses in the law reckoneth up, I think it for mine own part more probable to have been meant in the words of that canon, than fornication according unto the sense of the law of nature. Words must be taken according to the matter whereof they are uttered. The Apostles command to abstain from blood. Construe this meaning according to the law of nature, and it will seem that homicide only is forbidden. But construe it in reference to the law of the Jews about which the question was, and it shall easily appear to have a clean other sense, and in any man’s judgment a truer, when we expound it of eating and not of shedding blood. So if we speak of fornication, he that knoweth no law but only the law of nature must needs make thereof a narrower construction, than he which measureth the same by a law, wherein sundry kinds  even of conjugal copulation are prohibited as impure, unclean, unhonest. St. Paul himself doth term incestuous marriage fornication. If any do rather think that the Christian Gentiles themselves, through the loose and corrupt custom of those times, took simple fornication for no sin, and were in that respect offensive unto believing Jews, which by the Law had been better taught; our proposing of another conjecture is unto theirs no prejudice.

[8.]Some things therefore we see there were, wherein the Gentiles were forbidden to be like unto the Jews; some things wherein they were commanded not to be unlike. Again, some things also there were, wherein no law of God did let but that they might be either like or unlike, as occasion should require. And unto this purpose Leo saith, “Apostolical ordinance (beloved,) knowing that our Lord Jesus Christ came not into this world to undo the law, hath in such sort distinguished the mysteries of the Old Testament, that certain of them it hath chosen out to benefit evangelical knowledge withal, and for that purpose appointed that those things which before were Jewish might now be Christian customs.” The cause why the Apostles did thus conform the Christians as much as might be according to the pattern of the Jews, was to rein them in by this mean the more, and to make them cleave the better.

[9.]The Church of Christ hath had in no one thing so many and so contrary occasions of dealing as about Judaism: some having thought the whole Jewish Law wicked and damnable in itself; some not condemning it as the former sort absolutely, have notwithstanding judged it either sooner necessary to be abrogated, or further unlawful to be observed than truth can bear: some of scrupulous simplicity urging perpetual and universal observation of the law of Moses necessary,  as the Christian Jews at the first in the Apostles’ times; some as heretics, holding the same no less even after the contrary determination set down by consent of the Church at Jerusalem; finally some being herein resolute through mere infidelity, and with open professed enmity against Christ, as unbelieving Jews.

To control slanderers of the Law and Prophets, such as Marcionites and Manichees were, the Church in her liturgies hath intermingled with readings out of the New Testament lessons taken out of the Law and Prophets; whereunto Tertullian alluding, saith of the Church of Christ, “It intermingleth with evangelical and apostolical writings the Law and the Prophets; and from thence it drinketh in that faith, which with water it sealeth, clotheth with the Spirit, nourisheth with the Eucharist, with martyrdom setteth forward.” They would have wondered in those times to hear, that any man being not a favourer of heresy should term this by way of disdain, “mangling of the Gospels and Epistles.”

[10.]They which honour the Law as an image of the wisdom of God himself, are notwithstanding to know that the same had an end in Christ. But what? Was the Law so abolished with Christ, that after his ascension the office of Priests became immediately wicked, and the very name hateful, as importing the exercise of an ungodly function? No, as long as the glory of the Temple continued, and till the time of that final desolation was accomplished, the very Christian Jews did continue with their sacrifices and other parts of legal service. That very Law therefore which our Saviour was to abolish, did not so soon become unlawful to be  observed as some imagine; nor was it afterwards unlawful so far, that the very name of Altar, of Priest, of Sacrifice itself, should be banished out of the world. For though God do now hate sacrifice, whether it be heathenish or Jewish, so that we cannot have the same things which they had but with impiety; yet unless there be some greater let than the only evacuation of the Law of Moses, the names themselves may (I hope) be retained without sin, in respect of that proportion which things established by our Saviour have unto them which by him are abrogated. And so throughout all the writings of the ancient Fathers we see that the words which were do continue; the only difference is, that whereas before they had a literal, they now have a metaphorical use, and are as so many notes of remembrance unto us, that what they did signify in the letter is accomplished in the truth. And as no man can deprive the Church of this liberty, to use names whereunto the Law was accustomed, so neither are we generally forbidden the use of things which the Law hath; though it neither command us any particular rite, as it did the Jews a number, and the weightiest which it did command them are unto us in the Gospel prohibited.

[11.]Touching such as through simplicity of error did urge universal and perpetual observation of the Law of Moses at the first, we have spoken already. Against Jewish heretics and false apostles teaching afterwards the selfsame, St. Paul in every epistle commonly either disputeth or giveth warning. Jews that were zealous for the Law, but withal infidels in respect of Christianity, and to the name of Jesus Christ most spiteful enemies, did while they flourished no less persecute the Church than heathens. After their estate was overthrown, they were not that way so much to be feared. Howbeit, because they had their synagogues in every famous city almost throughout the world, and by that means great opportunity to withdraw from the Christian faith, which to do they spared no labour; this gave the church occasion to make sundry laws against them. As in the council of Laodicea  “The festival presents which Jews or heretics use to send must not be received, nor Holidays solemnized in their company.” Again, “from the Jews men ought not to receive their unleavened, nor to communicate with their impieties.” Which council was afterwards indeed confirmed by the sixth general council. But what was the true sense or meaning both of the one and the other? Were Christians here forbidden to communicate in unleavened bread because the Jews did so being enemies of the Church? He which attentively shall weigh the words will suspect, that they rather forbid communion with Jews, than imitation of them: much more, if with these two decrees be compared a third in the Council of Constantinople, “Let no man either of the clergy or laity eat the unleavened of the Jews, nor enter into any familiarity with them, nor send for them in sickness, nor take physic at their hands, nor as much as go into the bath with them. If any do otherwise being a clergyman, let him be deposed; if being a lay person, let excommunication be his punishment.”

[12.]If these canons were any argument, that they which made them did utterly condemn similitude between the Christians and Jews in things indifferent appertaining unto religion, either because the Jews were enemies unto the Church, or else for that their ceremonies were abrogated; these reasons had been as strong and effectual against their keeping the feast of Easter on the same day the Jews kept theirs, and not according to the custom of the West church. For so they did from the first beginning till Constantine’s time. For in these two things the East and West churches did interchangeably both confront the Jews and concur with them: the West church using unleavened bread, as the Jews in their passover did, but differing from them in the day whereon they kept the feast of Easter; contrariwise the East church celebrating the feast of Easter on the same day  with the Jews, but not using the same kind of bread which they did. Now if so be the East church in using leavened bread had done ill, either for that the Jews were enemies to the Church, or because Jewish ceremonies were abrogated; how should we think but that Victor the bishop of Rome (whom all judicious men do in that behalf disallow) did well to be so vehement and fierce in drawing them to the like dissimilitude for the feast of Easter? Again, if the West churches had in either of those two respects affected dissimilitude with the Jews in the feast of Easter, what reason had they to draw the Eastern church herein unto them, which reason did not enforce them to frame themselves unto it in the ceremony of leavened bread? Difference in rites should breed no controversy between one church and another; but if controversy be once bred, it must be ended. The feast of Easter being therefore litigious in the days of Constantine, who honoured of all other churches most the church of Rome, which church was the mother from whose breasts he had drawn that food, which gave him nourishment to eternal life; sith agreement was necessary, and yet impossible unless the one part were yielded unto; his desire was that of the two the Eastern church should rather yield. And to this end he useth sundry persuasive speeches.

When Stephen the Bishop of Rome going about to shew what the Catholic Church should do, had alleged what the heretics themselves did, namely, that they received such as came unto them, and offered not to baptize them anew; St. Cyprian being of a contrary mind to him about the matter at that time in question, which was, “Whether heretics converted ought to be rebaptized, yea or no?” answered the allegation of Pope Stephen with exceeding great stomach, saying, “To this degree of wretchedness the church of God and Spouse of Christ is now come, that her ways she frameth to the example of heretics; that to celebrate the Sacraments  which heavenly instruction hath delivered, light itself doth borrow from darkness, and Christians do that which Antichrists do.”

Now albeit Constantine have done that to further a better cause, which Cyprian did to countenance a worse, namely the rebaptization of heretics, and have taken advantage at the odiousness of the Jews, as Cyprian of heretics, because the Eastern church kept their feast of Easter always the fourteenth day of the month, as the Jews did, what day of the week soever it fell; or howsoever Constantine did take occasion in the handling of that cause to say, “It is unworthy to have any thing common with that spiteful nation of the Jews:” shall every motive argument used in such kind of conferences be made a rule for others still to conclude the like by, concerning all things of like nature, when as probable inducements may lead them to the contrary? Let both this and other allegations suitable unto it cease to bark any longer idly against that truth, the course and passage whereof it is not in them to hinder.

\section*{The fourth, for that sundry of them have been (they say) abused unto idolatry, and are by that mean become scandalous. }

XII. But the weightiest exception, and of all the most worthy to be respected, is against such kind of ceremonies, as have been so grossly and shamefully abused in the church of Rome, that where they remain they are scandalous, yea,  they cannot choose but be stumblingblocks and grievous causes of offence. Concerning this point therefore we are first to note, what properly it is to be scandalous or offensive; secondly, what kind of ceremonies are such; and thirdly, when they are necessarily for remedy thereof to be taken away, and when not.

[2.]The common conceit of the vulgar sort is, whensoever they see any thing which they mislike and are angry at, to think that every such thing is scandalous, and that themselves in this case are the men concerning whom our Saviour spake in so fearful manner, saying, “whosoever shall scandalize or offend any one of these little ones which believe in me” (that is, as they construe it, whosoever shall anger the meanest and simplest artisan which carrieth a good mind, by not removing out of the Church such rites and ceremonies as displease him), “better he were drowned in the bottom of the sea.” But hard were the case of the Church of Christ, if this were to scandalize. Men are scandalized when they are moved, led, and provoked unto sin. At good things evil men may take occasion to do evil; and so Christ himself was a rock of offence in Israel, they taking occasion at his poor estate and at the ignominy of his cross, to think him unworthy the name of that great and glorious Messias, whom the Prophets describe in such ample and stately terms. But that which we therefore term offensive because it inviteth men to offend, and by a dumb kind of provocation encourageth, moveth, or any way leadeth unto sin, must of necessity be acknowledged actively scandalous.

Now some things are so even by their very essence and nature, so that wheresoever they are found they are not neither can be without this force of provocation unto evil; of which kind all examples of sin and wickedness are. Thus David was scandalous in that bloody act whereby he caused the enemies of God to be blasphemous: thus the whole state of Israel scandalous, when their public disorders caused the name of God to be ill-spoken of amongst the nations. It is of this  kind that Tertullian meaneth: “Offence or scandal, if I be not deceived (saith he), is, when the example not of a good but of an evil thing doth set men forward unto sin. Good things can scandalize none save only evil minds:” good things have no scandalizing nature in them.

[3.]Yet that which is of its own nature either good or at least not evil, may by some accident become scandalous at certain times and in certain places and to certain men; the open use thereof nevertheless being otherwise without danger. The very nature of some rites and ceremonies therefore is scandalous, as it was in a number of those which the Manichees did use, and is in all such as the law of God doth forbid. Some are offensive only through the agreement of men to use them unto evil, and not else; as the most of those things indifferent which the heathens did to the service of their false gods, which another, in heart condemning their idolatry, could not do with them in show and token of approbation without being guilty of scandal given. Ceremonies of this kind are either devised at the first unto evil, as the Eunomian heretics in dishonour of the blessed Trinity brought in the laying on of water but once, to cross the custom of the church which in baptism did it thrice; or else having had a profitable use they are afterwards interpreted and wrested to the contrary, as those heretics which held the Trinity to be three distinct not persons but natures, abused the ceremony of three times laying on water in baptism unto the strengthening of their heresy. The element of water is in baptism necessary; once to lay it on or twice is indifferent. For which cause Gregory making mention thereof saith, “To dive an infant  either thrice or but once in baptism, can be no way a thing reprovable; seeing that both in three times washing the Trinity of persons, and in one the Unity of Godhead may be signified.” So that of these two ceremonies neither being hurtful in itself, both may serve unto good purpose; yet one was devised, and the other converted, unto evil.

[4.]Now whereas in the church of Rome certain ceremonies are said to have been shamefully abused unto evil, as the ceremony of crossing at baptism, of kneeling at the eucharist, of using wafer-cakes, and such like; the question is, whether for remedy of that evil wherein such ceremonies have been scandalous, and perhaps may be still unto some even amongst ourselves, whom the presence and sight of them may confirm in that former error whereto they served in times past, they are of necessity to be removed. Are these, or any other ceremonies we have common with the church of Rome, scandalous and wicked in their very nature? This no man objecteth. Are any such as have been polluted from their very birth, and instituted even at the first unto that thing which is evil? That which hath been ordained impiously at the first, may wear out that impiety in tract of time; and then what doth let but that the use thereof may stand without offence? The names of our months and of our days we are not ignorant from whence they came, and with what dishonour unto God they are said to have been devised at the first. What could be spoken against any thing more effectual to stir hatred, than that which sometime the ancient Fathers in this case speak? Yet those very names are at this day in use  throughout Christendom without hurt or scandal to any. Clear and manifest it is, that things devised by heretics, yea, devised of a very heretical purpose even against religion, and at their first devising worthy to have been withstood, may in time grow meet to be kept; as that custom, the inventors whereof were the Eunomian heretics. So that customs once established and confirmed by long use, being presently without harm, are not in regard of their corrupt original to be held scandalous.

[5.]But concerning those our ceremonies which they reckon for most popish, they are not able to avouch, that any of them was otherwise instituted than unto good, yea, so used at the first. It followeth then that they all are such, as having served to good purpose, were afterwards converted unto the contrary. And sith it is not so much as objected against us, that we retain together with them the evil wherewith they have been infected in the church of Rome, I would demand who they are whom we scandalize, by using harmless things unto that good end for which they were first instituted. Amongst ourselves that agree in the approbation of this kind of good use, no man will say that one of us is offensive and scandalous unto another. As for the favourers of the church of Rome, they know how far we herein differ and dissent from them; which thing neither we conceal, and they by their public writings also profess daily how much it grieveth them; so that of them there will not many rise up against us, as witnesses unto the indictment of scandal, whereby we might be condemned and cast, as having strengthened them in that evil wherewith they pollute themselves in the use of the same ceremonies. And concerning such as withstand the church of England herein, and hate it because it doth not sufficiently seem to hate Rome; they (I hope) are far enough from being by this mean drawn to any kind of popish error. The multitude therefore of them, unto whom we are scandalous through the use of abused ceremonies, is not so apparent, that it can justly be said in general of any one sort of men or other, we cause them to offend. If it be so, that now or then some few are espied, who, having been accustomed heretofore to the rites and ceremonies of the church of Rome, are not so scoured of their former rust as to forsake their ancient persuasion which they have had, howsoever they frame themselves to  outward obedience of laws and orders: because such may misconstrue the meaning of our ceremonies, and so take them as though they were in every sort the same they have been, shall this be thought a reason sufficient whereon to conclude that some law must necessarily be made to abolish all such ceremonies?

[6.]They answer, that there is no law of God which doth bind us to retain them. And St. Paul’s rule is, that in those things from which without hurt we may lawfully abstain, we should frame the usage of our liberty with regard to the weakness and imbecility of our brethren. Wherefore unto them which stood upon their own defence saying, “All things are lawful unto me;” he replieth, “but all things are not expedient” in regard of others. “All things are clean, all meats are lawful; but evil unto that man that eateth offensively. If for thy meat’s sake thy brother be grieved, thou walkest no longer according to charity. Destroy not him with thy meat for whom Christ died. Dissolve not for food’s sake the work of God. We that are strong must bear the imbecilities of the impotent, and not please ourselves.” It was a weakness in the Christian Jews, and a maim of judgment in them, that they thought the Gentiles polluted by the eating of those meats which themselves were afraid to touch for fear of transgressing the law of Moses; yea, hereat their hearts did so much rise, that the Apostle had just cause to fear, lest they would rather forsake Christianity than endure any fellowship with such as made no conscience of that which was unto them abominable. And for this cause mention is made of destroying the weak by meats, and of dissolving the work of God, which was his Church, a part of the living stones whereof were believing Jews. Now those weak brethren before-mentioned are said to be as the Jews were, and our ceremonies which have been abused in the church of Rome to be as the scandalous meats, from which the Gentiles are exhorted to abstain in the presence of Jews, for fear of averting them from Christian faith. Therefore, as charity did bind them to refrain from that for their brethren’s sake, which otherwise was lawful enough for them; so it  bindeth us for our brethren’s sake likewise to abolish such ceremonies, although we might lawfully else retain them.

[7.]But between these two cases there are great odds. For neither are our weak brethren as the Jews, nor the ceremonies which we use as the meats which the Gentiles used. The Jews were known to be generally weak in that respect; whereas contrariwise the imbecility of ours is not common unto so many, that we can take any such certain notice of them. It is a chance if here and there some one be found; and therefore seeing we may presume men commonly otherwise, there is no necessity that our practice should frame itself by that which the Apostle doth prescribe to the Gentiles.

Again, their use of meats was not like unto our of ceremonies, that being a matter of private action in common life, where every man was free to order that which himself did; but this a public constitution for the ordering of the Church: and we are not to look that the Church should change her public laws and ordinances, made according to that which is judged ordinarily and commonly fittest for the whole, although it chance that for some particular men the same be found inconvenient; especially when there may be other remedy also against the sores of particular inconveniences. In this case therefore where any private harm doth grow, we are not to reject instruction, as being an unmeet plaister to apply unto it; neither can we say, that he which appointeth teachers for physicians in this kind of evil, is “As if a man would set one to watch a child all day long lest he should hurt himself with a knife; whereas by taking away the knife from him, the danger is avoided, and the service of the man better employed.” For a knife may be taken away from a child, without depriving them of the benefit thereof which have years and discretion to use it. But the ceremonies which children do abuse if we remove quite and clean, as it is by some required that we should, then are they not taken from children only, but from others also; which is as though because children may perhaps hurt themselves with knives,  we should conclude, that therefore the use of knives is to be taken quite and clean even from men also.

[8.]Those particular ceremonies, which they pretend to be so scandalous, we shall in the next Book have occasion more throughly to sift, where other things also traduced in the public duties of the Church whereunto each of these appertaineth, are together with these to be touched, and such reasons to be examined as have at any time been brought either against the one or the other. In the meanwhile against the conveniency of curing such evils by instruction, strange it is that they should object the multitude of other necessary matters, wherein preachers may better bestow their time, than in giving men warning not to abuse ceremonies: a wonder it is, that they should object this, which have so many years together troubled the Church with quarrels concerning these things, and are even to this very hour so earnest in them, that if they write or speak publicly but five words, one of them is lightly about the dangerous estate of the church of England in respect of abused ceremonies. How much happier had it been for this whole Church, if they which have raised contention therein about the abuse of rites and ceremonies, had considered in due time that there is indeed store of matters fitter and better a great deal for teachers to spend time and labour in! It is through their importunate and vehement asseverations, more than through any such experience which we have had of our own, that we are forced to think it possible for one or other now and then, at leastwise in the prime of the reformation of our church, to have stumbled at some kind of ceremony: wherein forasmuch as we are contented to take this upon their credit, and to think it may be; sith also they further pretend the same to be so dangerous a snare to their souls that are at any time  taken therein; they must give our teachers leave for the saving of those souls (be they never so few) to intermingle sometime with other more necessary things admonition concerning these not unnecessary. Wherein they should in reason more easily yield this leave, considering that hereunto we shall not need to use the hundredth part of that time, which themselves think very needful to bestow in making most bitter invectives against the ceremonies of the Church.

\section*{The fifth, for that we retain them still, notwithstanding the example of certain Churches reformed before us, which have cast them out.}

XIII. But to come to the last point of all; the church of England is grievously charged with forgetfulness of her duty, which duty had been to frame herself unto the pattern of their example that went before her in the work of reformation. For “as the churches of Christ ought to be most unlike the synagogue of Antichrist in their indifferent ceremonies; so they ought to be most like one unto another, and for preservation of unity to have as much as possible may be all the same ceremonies. And therefore St. Paul, to establish this order in the church of Corinth, that they should make their gatherings for the poor upon the first day of the Sabbath, (which is our Sunday,) allegeth this for a reason, That he had so ordained in other churches.” Again, “As children of one father and servants of one family, so all churches should not only have one diet in that they have one word, but also wear as it were one livery in using the same ceremonies.” Thirdly, “This rule did the great council of Nice follow, when it ordained, that where certain at the feast of Pentecost did pray kneeling, they should pray standing: the reason whereof is added, which is, that one custom ought to be kept throughout all churches. It is true that the diversity of ceremonies ought not to cause the churches to dissent one with another; but yet it maketh most to the avoiding of dissension, that there be amongst them an unity not only in doctrine, but also in ceremonies. And therefore our form of service is to be amended, not only for that it cometh too near that of the  Papists, but also because it is so different from that of the reformed churches.” Being asked to what churches ours should conform itself, and why other reformed churches should not as well frame themselves to ours; their answer is, “that if there be any ceremonies which we have better than others, they ought to frame themselves to us; if they have better than we, then we ought to frame ourselves to them; if the ceremonies be alike commodious, the later churches should conform themselves to the first, as the younger daughter to the elder. For as St. Paul in the members, where all other things are equal, noteth it for a mark of honour above the rest, that one is called before another to the Gospel; so is it for the same cause amongst the churches. And in this respect he pincheth the Corinths, that not being the first which received the Gospel, yet they would have their several manners from other churches. Moreover, where the ceremonies are alike commodious, the fewer ought to conform themselves unto the more. Forasmuch therefore as all the churches” (so far as they know which plead after this manner) “of our confession in doctrine agree in the abrogation of divers things which we retain, our church ought either to shew that they have done evil, or else she is found to be in fault that doth not conform herself in that, which she cannot deny to be well abrogated.”

[2.]In this axiom, that preservation of peace and unity amongst Christian churches should be by all good means procured, we join most willingly and gladly with them. Neither deny we but that to the avoiding of dissension it availeth much that there be amongst them an unity as well in ceremonies as in doctrine. The only doubt is about the manner of their unity; how far churches are bound to be uniform in their ceremonies, and what way they ought to take for that purpose.

[3.]Touching the one, the rule which they have set down is, that in ceremonies indifferent, all churches ought to be one of them unto another as like as possibly they may be. Which possibly we cannot otherwise construe, than that it  doth require them to be even as like as they may be without breaking any positive ordinance of God. For the ceremonies whereof we speak, being matter of positive law, they are indifferent, if God have neither himself commanded nor forbidden them, but left them unto the Church’s discretion. So that if as great uniformity be required as is possible in these things; seeing that the law of God forbiddeth not any one of them, it followeth that from the greatest unto the least they must be in every Christian church the same, except mere impossibility of so having it be the hinderance. To us this opinion seemeth over extreme and violent: we rather incline to think it a just and reasonable cause for any church, the state whereof is free and independent, if in these things it differ from other churches, only for that it doth not judge it so fit and expedient to be framed therein by the pattern of their example, as to be otherwise framed than they. That of Gregory unto Leander is a charitable speech and a peaceable; “In una fide nil officit ecclesiæ sanctæ consuetudo diversa:” “Where the faith of the holy Church is one, a difference in customs of the Church doth no harm.” That of St. Augustine to Casulanus is somewhat more particular, and toucheth what kind of ceremonies they are, wherein one church may vary from the example of another without hurt: “Let the faith of the whole Church, how wide soever it have spread itself, be always one, although the unity of belief be famous for variety of certain ordinances, whereby that which is rightly believed suffereth no kind of let or impediment.” Calvin goeth further, “As concerning rites in particular, let the sentence of Augustine take place, which leaveth it free unto all churches to receive each their own custom. Yea sometime it profiteth and is expedient that there be difference, lest men should think that religion is tied to outward ceremonies. Always provided that there be not any emulation, nor that churches delighted with novelty affect to have that which others have not.”


[4.]They which grant it true that the diversity of ceremonies in this kind ought not to cause dissension in churches, must either acknowledge that they grant in effect nothing by these words; or if any thing be granted, there must as much be yielded unto, as we affirm against their former strict assertion. For if churches be urged by way of duty to take such ceremonies as they like not of, how can dissension be avoided? Will they say that there ought to be no dissension, because such as be urged ought to like of that whereunto they are urged? If they say this, they say just nothing. For how should any church like to be urged of duty, by such as have no authority or power over it, unto those things which being indifferent it is not of duty bound unto them? Is it their meaning, that there ought to be no dissension, because, that which churches are not bound unto, no man ought by way of duty to urge upon them; and if any man do, he standeth in the sight of both God and men most justly blameable, as a needless disturber of the peace of God’s Church, and an author of dissension? In saying this, they both condemn their own practice, when they press the church of England with so strict a bond of duty in these things; and they overthrow the ground of their practice, which is, that there ought to be in all kind of ceremonies uniformity, unless impossibility hinder it.

[5.]For proof whereof it is not enough to allege what St. Paul did about the matter of collections, or what noblemen do in the liveries of their servants, or what the council of Nice did for standing in time of prayer on certain days: because though St. Paul did will them of the church of Corinth every man to lay up somewhat by him upon the Sunday, and to  reserve it in store, till himself did come thither to send it unto the church of Jerusalem for relief of the poor there; signifying withal, that he had taken the like order with the churches of Galatia; yet the reason which he yieldeth of this order taken both in the one place and the other, sheweth the least part of his meaning to have been that whereunto his words are writhed. “Concerning collection for the saints, (he meaneth them of Jerusalem,) as I have given order to the church of Galatia, so likewise do ye,” saith the Apostle; “that is, in every first of the week let each of you lay aside by himself, and reserve according to that which God hath blessed him with, that when I come collections be not then to make; and that when I am come, whom you shall choose, them I may forthwith send away by letters to carry your beneficence unto Jerusalem.” Out of which words to conclude the duty of uniformity throughout all churches in all manner of indifferent ceremonies will be very hard, and therefore best to give it over.

[6.]But perhaps they are by so much the more loth to forsake this argument, for that it hath, though nothing else, yet the name of Scripture, to give it some kind of countenance more than the next of livery coats afforded them. For neither is it any man’s duty to clothe all his children or all his servants with one weed, nor theirs to clothe themselves so, if it were left to their own judgments, as these ceremonies are left of God to the judgment of the Church. And seeing churches are rather in this case like divers families than like divers servants of one family; because every church, the state whereof is independent upon any other, hath authority to appoint orders for itself in things indifferent: therefore of the two we may rather infer, that as one family is not abridged of liberty to be clothed in friar’s-grey for that another doth wear clay-colour, so neither are all churches bound to the selfsame indifferent ceremonies which it liketh sundry to use.

[7.]As for that canon in the council of Nice, let them but  read it and weigh it well. The ancient use of the Church throughout all Christendom was for fifty days after Easter, (which fifty days were called Pentecost, though most commonly the last day of them which is Whitsunday be so called,) in like sort on all the Sundays throughout the whole year their manner was, to stand at prayer; whereupon their meetings unto that purpose on those days had the name of Stations given them. Of which custom Tertullian speaketh in this wise; “It is not with us thought fit either to fast on the Lord’s day, or to pray kneeling. The same immunity from fasting and kneeling we keep all the time which is between the feasts of Easter and Pentecost.” This being therefore an order generally received in the Church; when some began to be singular and different from all others, and that in a ceremony which was then judged very convenient for the whole church even by the whole, those few excepted which brake out of the common pale: the council of Nice thought good to enclose them again with the rest, by a law made in this sort: “Because there are certain which will needs kneel at the time of prayer on the Lord’s-day, and in the fifty days after Easter; the holy synod judging it meet that a convenient custom be observed throughout all churches, hath decreed that standing we make our prayers to the Lord.” Whereby it plainly appeareth that in things indifferent, what the whole Church doth think convenient for the whole, the same if any part do wilfully violate, it may be reformed and inrailed again by that general authority whereunto each particular is subject; and that the spirit of singularity in a few ought to give place unto public judgment: this doth clearly enough appear, but not that all Christian churches are bound in every indifferent ceremony to be uniform; because where the whole hath not tied the parts unto one and the same thing, they being therein left  each to their own choice, may either do as other do or else otherwise, without any breach of duty at all.

[8.]Concerning those indifferent things, wherein it hath been heretofore thought good that all Christian churches should be uniform, the way which they now conceive to bring this to pass was then never thought on. For till now it hath been judged, that seeing the Law of God doth not prescribe all particular ceremonies which the Church of Christ may use; and in so great variety of them as may be found out, it is not possible that the law of nature and reason should direct all churches unto the same things, each deliberating by itself what is most convenient; the way to establish the same things indifferent throughout them all must needs be the judgment of some judicial authority drawn into one only sentence, which may be a rule for every particular to follow. And because such authority over all churches is too much to be granted unto any one mortal man, there yet remaineth that which hath been always followed as the best, the safest, the most sincere and reasonable way; namely, the verdict of the whole Church orderly taken, and set down in the assembly of some general council. But to maintain that all Christian churches ought for unity’s sake to be uniform in all ceremonies, and then to teach that the way of bringing this to pass must be by mutual imitation, so that where we have better ceremonies than others they shall be bound to follow us, and we them where theirs are better; how should we think it agreeable and consonant unto reason? For sith in things of this nature there is such variety of particular inducements, whereby one church may be led to think that better which another church led by other inducements judgeth to be worse: (for example, the East church did think it better to keep Easter-day after the manner of the Jews, the West church better to do otherwise; the Greek church judgeth it worse to use unleavened bread in the Eucharist, the Latin church leavened; one church esteemeth it not so good to receive the Eucharist sitting as standing, another church not so good standing as sitting; there being on the one side probable motives as well as on the other:) unless they add somewhat else to define more certainly what ceremonies shall stand for best, in such sort that all churches in  the world shall know them to be the best, and so know them that there may not remain any question about this point, we are not a whit the nearer for that they have hitherto said.

[9.]They themselves, although resolved in their own judgments what ceremonies are best, yet foreseeing that such as they are addicted unto be not all so clearly and so incomparably best, but others there are or may be at leastwise, when all things are well considered, as good, knew not which way smoothly to rid their hands of this matter, without providing some more certain rule to be followed for establishment of uniformity in ceremonies, when there are divers kinds of equal goodness; and therefore in this case they say, that the later churches and the fewer should conform themselves unto the elder and the more. Hereupon they conclude, that forasmuch as all the reformed churches (so far as they know), which are of our confession in doctrine, have agreed already in the abrogation of divers things which we retain; our church ought either to shew that they have done evil, or else she is found to be in fault for not conforming herself to those churches, in that which she cannot deny to be in them well abrogated. For the authority of the first churches, (and those they account to be the first in this cause which were first reformed,) they bring the comparison of younger daughters conforming themselves in attire to the example of their elder sisters; wherein there is just as much strength of reason as in the livery-coats before-mentioned. St. Paul, they say, noteth it for a mark of special honour, that Epænetus was the first man in all Achaia which did embrace the Christian faith; after the same sort he toucheth it also as a special preeminence of Junias and Andronicus, that in Christianity they were his ancients; the Corinthians he pinched with this demand, “Hath the word of God gone out from you, or hath it lighted on you alone?”

But what of all this? If any man should think that alacrity  and forwardness in good things doth add nothing unto men’s commendation, the two former speeches of St. Paul might lead him to reform his judgment. In like sort, to take down the stomach of proud conceited men, that glory as though they were able to set all others to school, there can be nothing more fit than some such words as the Apostle’s third sentence doth contain; wherein he teacheth the church of Corinth to know, that there was no such great odds between them and the rest of their brethren, that they should think themselves to be gold and the rest to be but copper. He therefore useth speech unto them to this effect: “Men instructed in the knowledge of Jesus Christ there both were before you, and are besides you in the world; ye neither are the fountain from which first, nor yet the river into which alone the word hath flowed.” But although as Epænetus was the first man in all Achaia, so Corinth had been the first church in the whole world, that received Christ; the Apostle doth not shew that in any kind of things indifferent whatsoever this should have made their example a law unto all others. Indeed the example of sundry churches for approbation of one thing doth sway much; but yet still as having the force of an example only, and not of a law. They are effectual to move any church, unless some greater thing do hinder; but they bind none, no not though they be many; saving only when they are the major part of a general assembly, and then their voices being more in number must oversway their judgments who are fewer, because in such cases the greater half is the whole. But as they stand out single each of them by itself, their number can purchase them no such authority, that the rest of the churches being fewer should be therefore bound to follow them, and to relinquish as good ceremonies as theirs for theirs.

[10.]Whereas therefore it is concluded out of these so weak premises, that the retaining of divers things in the church of England, which other reformed churches have cast out, must needs argue that we do not well, unless we can shew that they have done ill; what needed this wrest to  draw out from us an accusation of foreign churches? It is not proved as yet that if they have done well our duty is to follow them, and to forsake our own course because it differeth from theirs, although indeed it be as well for us every way as theirs for them. And if the proofs alleged for confirmation hereof had been sound, yet seeing they lead no further than only to shew, that where we can have no better ceremonies theirs must be taken; as they cannot with modesty think themselves to have found out absolutely the best which the wit of men may devise, so liking their own somewhat better than other men’s, even because they are their own, they must in equity allow us to be like unto them in this affection; which if they do, they ease us of that uncourteous burden, whereby we are charged either to condemn them or else to follow them. They grant we need not follow them, if our own ways already be better: and if our own be but equal, the law of common indulgence alloweth us to think them at the least half a thought the better because they are our own; which we may very well do, and never draw any indictment at all against theirs, but think commendably even of them also.

\section*{XIV. A declaration of the proceedings of the Church of England for the establishment of things as they are.}

XIV. To leave reformed churches therefore and their actions for Him to judge of, in whose sight they are as they are; and our desire is that they may even in his sight be found such as we ought to endeavour by all means that our own may likewise be; somewhat we are enforced to speak by way of simple declaration concerning the proceedings of the church of England in these affairs, to the end that men whose minds are free from those partial constructions, whereby the only name of difference from some other churches is thought cause sufficient to condemn ours, may the better discern whether that we have done be reasonable, yea or no. The church of England being to alter her received laws concerning such orders, rites, and ceremonies, as had been in former times an hinderance unto piety and religious service of God, was to enter into consideration first, that the change of laws, especially concerning matter of religion, must be warily proceeded in. Laws, as all other things human, are many times full of imperfection; and that which is supposed behoveful unto men, proveth oftentimes most pernicious. The wisdom which is  learned by tract of time, findeth the laws that have been in former ages established, needful in later to be abrogated. Besides, that which sometime is expedient doth not always so continue: and the number of needless laws unabolished doth weaken the force of them that are necessary. But true withal it is, that alteration though it be from worse to better hath in it inconveniences, and those weighty; unless it be in such laws as have been made upon special occasions, which occasions ceasing, laws of that kind do abrogate themselves. But when we abrogate a law as being ill made, the whole cause for which it was made still remaining, do we not herein revoke our very own deed, and upbraid ourselves with folly, yea, all that were makers of it with oversight and with error? Further, if it be a law which the custom and continual practice of many ages or years hath confirmed in the minds of men, to alter it must needs be troublesome and scandalous. It amazeth them, it causeth them to stand in doubt whether any thing be in itself by nature either good or evil, and not all things rather such as men at this or that time agree to account of them, when they behold even those things disproved, disannulled, rejected, which use had made in a manner natural. What have we to induce men unto the willing obedience and observation of laws, but the weight of so many men’s judgment as have with deliberate advice assented thereunto; the weight of that long experience, which the world hath had thereof with consent and good liking? So that to change any such law must needs with the common sort impair and weaken the force of those grounds, whereby all laws are made effectual.

[2.]Notwithstanding we do not deny alteration of laws to be sometimes a thing necessary; as when they are unnatural, or impious, or otherwise hurtful unto the public community of men, and against that good for which human societies were instituted. When the Apostles of our Lord and Saviour were ordained to alter the laws of heathenish religion received throughout the whole world, chosen I grant they were (Paul excepted) the rest ignorant, poor, simple, unschooled altogether and unlettered men; howbeit extraordinarily endued with ghostly wisdom from above before they ever undertook this enterprise; yea their authority confirmed by miracle, to  the end it might plainly appear that they were the Lord’s ambassadors, unto whose sovereign power for all flesh to stoop, for all the kingdoms of the earth to yield themselves willingly conformable in whatsoever should be required, it was their duty. In this case therefore their oppositions in maintenance of public superstition against apostolic endeavours, as that they might not condemn the ways of their ancient predecessors, that they must keep religiones traditas, the rites which from age to age had descended, that the ceremonies of religion had been ever accounted by so much holier as elder; these and the like allegations in this case were vain and frivolous.

Not to stay longer therefore in speech concerning this point, we will conclude, that as the change of such laws as have been specified is necessary, so the evidence that they are such must be great. If we have neither voice from heaven that so pronounceth of them, neither sentence of men grounded upon such manifest and clear proof, that they in whose hands it is to alter them may likewise infallibly even in heart and conscience judge them so: upon necessity to urge alteration is to trouble and disturb without necessity. As for arbitrary alterations, when laws in themselves not simply bad or unmeet are changed for better and more expedient; if the benefit of that which is newly better devised be but small, sith the custom of easiness to alter and change is so evil, no doubt but to bear a tolerable sore is better than to venture on a dangerous remedy.

[3.]Which being generally thought upon as a matter that touched nearly their whole enterprise, whereas change was notwithstanding concluded necessary, in regard of the great hurt which the Church did receive by a number of things then in use, whereupon a great deal of that which had been was now to be taken away and removed out of the Church; yet sith there are divers ways of abrogating things established, they saw it best to cut off presently such things as might in that sort be extinguished without danger, leaving the rest to be abolished by disusage through tract of time. And as this was done for the manner of abrogation: so touching the stint  or measure thereof, rites and ceremonies and other external things of like nature being hurtful unto the Church, either in respect of their quality or in regard of their number; in the former there could be no doubt or difficulty what should be done, their deliberation in the later was more hard. And therefore inasmuch as they did resolve to remove only such things of that kind as the Church might best spare, retaining the residue; their whole counsel is in this point utterly condemned, as having either proceeded from the blindness of those times, or from negligence, or from desire of honour and glory, or from an erroneous opinion that such things might be tolerated for a while; or if it did proceed (as they which would seem most favourable are content to think it possible) from a purpose, “partly the easilier to draw papists unto the Gospel” (by keeping so many orders still the same with theirs), “and partly to redeem peace thereby, the breach whereof they might fear would ensue upon more thorough alteration;” or howsoever it came to pass, the thing they did is judged evil. But such is the lot of all that deal in public affairs whether of church or commonwealth; that which men list to surmise of their doings, be it good or ill, they must beforehand patiently arm their minds to endure. Wherefore to let go private surmises, whereby the thing in itself is not made either better or worse; if just and allowable reasons might lead them to do as they did, then are these censures all frustrate.

[4.]Touching ceremonies harmless therefore in themselves, and hurtful only in respect of number: was it amiss to decree, that those things which were least needful and newliest come should be the first that were taken away, as in the abrogating of a number of saints’ days, and of other the like customs, it appeareth they did; till afterwards the Form of Common Prayer being perfected, Articles of sound Religion and Discipline agreed upon, Catechisms framed for the needful instruction of youth, churches purged of things that indeed were burdensome to the people or to the simple offensive and scandalous, all was brought at the length unto that wherein now we stand? Or was it amiss, that having this way eased  the Church as they thought of superfluity, they went not on till they had plucked up even those things also, which had taken a great deal stronger and deeper root; those things which to abrogate without constraint of manifest harm thereby arising, had been to alter unnecessarily (in their judgments) the ancient received custom of the whole Church, the universal practice of the people of God, and those very decrees of our fathers, which were not only set down by agreement of general councils, but had accordingly been put in ure and so continued in use till that very time present?

[5.]True it is, that neither councils nor customs, be they never so ancient and so general, can let the Church from taking away that thing which is hurtful to be retained. Where things have been instituted, which being convenient and good at the first, do afterwards in process of time wax otherwise; we make no doubt but they may be altered, yea, though councils or customs general have received them. And therefore it is but a needless kind of opposition which they make who thus dispute, “If in those things which are not expressed in the Scripture, that is to be observed of the Church, which is the custom of the people of God and decree of our forefathers; then how can these things at any time be varied, which heretofore have been once ordained in such sort?” Whereto we say, that things so ordained are to be kept, howbeit not necessarily any longer, than till there grow some urgent cause to ordain the contrary. For there is not any positive law of men, whether it be general or particular; received by formal express consent, as in councils, or by secret approbation, as in customs it cometh to pass; but the same may be taken away if occasion serve. Even as we all know, that many things generally kept heretofore are now in like sort generally unkept and abolished every where.

[6.]Notwithstanding till such things be abolished, what exception can there be taken against the judgment of St. Augustine, who saith, “That of things harmless, whatsoever there is which the whole Church doth observe throughout the world, to argue for any man’s immunity from observing the same, it were a point of most insolent madness?”  And surely odious it must needs have been for one Christian church to abolish that which all had received and held for the space of many ages, and that without any detriment unto religion so manifest and so great, as might in the eyes of unpartial men appear sufficient to clear them from all blame of rash and inconsiderate proceeding, if in fervour of zeal they had removed such things. Whereas contrariwise, so reasonable moderation herein used hath freed us from being deservedly subject unto that bitter kind of obloquy, whereby as the church of Rome doth under the colour of love towards those things which be harmless, maintain extremely most hurtful corruptions; so we peradventure might be upbraided, that under colour of hatred towards those things that are corrupt, we are on the other side as extreme even against most harmless ordinances. And as they are obstinate to retain that, which no man of any conscience is able well to defend; so we might be reckoned fierce and violent to tear away that, which if our own mouths did condemn, our consciences would storm and repine thereat. The Romans having banished Tarquinius the Proud, and taken a solemn oath that they never would permit any man more to reign, could not herewith content themselves, or think that tyranny was thoroughly extinguished, till they had driven one of their Consuls to depart the city, against whom they found not in the world what to object, saving only that his name was Tarquin, and that the commonwealth could not seem to have recovered perfect freedom, as long as a man of so dangerous a name was left remaining. For the church of England to have done the like in casting out of papal tyranny and superstition; to have shewed greater willingness of accepting the very ceremonies of the Turk, Christ’s professed enemy, than of the most indifferent things which the church of Rome approveth; to have left not so much as the names which the church of Rome doth give unto things innocent; to have ejected whatsoever that Church doth make account of, be it never so harmless in itself, and of never so ancient continuance, without any other crime to charge it with, than only that it hath been the hap thereof to be used  by the church of Rome, and not to be commanded in the word of God: this kind of proceeding might haply have pleased some few men, who having begun such a course themselves must needs be glad to see their example followed by us. But the Almighty which giveth wisdom and inspireth with right understanding whomsoever it pleaseth him, he foreseeing that which man’s wit had never been able to reach unto, namely, what tragedies the attempt of so extreme alteration would raise in some parts of the Christian world, did for the endless good of his Church (as we cannot choose but interpret it) use the bridle of his provident restraining hand, to stay those eager affections in some, and to settle their resolution upon a course more calm and moderate: lest as in other most ample and heretofore most flourishing dominions it hath since fallen out, so likewise if in ours it had come to pass, that the adverse part being enraged, and betaking itself to such practices as men are commonly wont to embrace, when they behold things brought to desperate extremities, and no hope left to see any other end, than only the utter oppression and clean extinguishment of one side; by this mean Christendom flaming in all parts of greatest importance at once, they all had wanted that comfort of mutual relief, whereby they are now for the time sustained (and not the least by this our church which they so much impeach) till mutual combustions, bloodsheds, and wastes, (because no other inducement will serve,) may enforce them through very faintness, after the experience of so endless miseries, to enter on all sides at the length into some such consultation, as may tend to the best reestablishment of the whole Church of Jesus Christ. To the singular good whereof it cannot but serve as a profitable direction to teach men what is most likely to prove available, when they shall quietly consider the trial that hath been thus long had of both kinds of reformation; as  well this moderate kind which the church of England hath taken, as that other more extreme and rigorous which certain churches elsewhere have better liked. In the meanwhile it may be, that suspense of judgment and exercise of charity were safer and seemlier for Christian men, than the hot pursuit of these controversies, wherein they that are most fervent to dispute be not always the most able to determine. But who are on his side, and who against him, our Lord in his good time shall reveal.

[7.]And sith thus far we have proceeded in opening the things that have been done, let not the principal doers themselves be forgotten. When the ruins of the house of God (that house which consisting of religious souls is most immediately the precious temple of the Holy Ghost) were become, not in his sight alone, but in the eyes of the whole world so exceeding great, that very superstition began even to feel itself too far grown: the first that with us made way to repair the decays thereof by beheading superstition, was King Henry the Eighth. The son and successor of which famous king as we know was Edward the Saint: in whom (for so by the event we may gather) it pleased God righteous and just to let England see what a blessing sin and iniquity would not suffer it to enjoy. Howbeit that which the wise man hath said concerning Enoch (whose days were though many in respect of ours, yet scarce as three to nine in comparison of theirs with whom he lived) the same to that admirable child most worthily may be applied, “Though he departed this world soon, yet fulfilled he much time.” But what ensued? That work which the one in such sort had begun, and the other so far proceeded in, was in short space so overthrown, as if almost it had never been: till such time as that God, whose property is to shew his mercies then greatest when they are nearest to be utterly despaired of, caused in the depth of discomfort and darkness a most glorious star to arise, and on her head settled the crown, whom himself had kept as a lamb from the slaughter of those bloody times; that the experience of his goodness in her own deliverance might cause her merciful disposition to take so much the more delight in saving others, whom the like necessity should press. What in this behalf  hath been done towards nations abroad, the parts of Christendom most afflicted can best testify. That which especially concerneth ourselves, in the present matter we treat of, is the state of reformed religion, a thing at her coming to the crown even raised as it were by miracle from the dead; a thing which we so little hoped to see, that even they which beheld it done, scarcely believed their own senses at the first beholding. Yet being then brought to pass, thus many years it hath continued, standing by no other worldly mean but that one only hand which erected it; that hand which as no kind of imminent danger could cause at the first to withhold itself, so neither have the practices so many so bloody following since been ever able to make weary. Nor can we say in this case so justly, that Aaron and Hur, the ecclesiastical and civil states, have sustained the hand which did lift itself to heaven for them, as that heaven itself hath by this hand sustained them, no aid or help having thereunto been ministered for performance of the work of reformation, other than such kind of help or aid as the Angel in the Prophet Zachary speaketh of, saying, “Neither by an army nor strength, but by my Spirit, saith the Lord of Hosts.” Which grace and favour of divine assistance having not in one thing or two shewed itself, nor for some few days or years appeared, but in such sort so long continued, our manifold sins and transgressions striving to the contrary; what can we less thereupon conclude, than that God would at leastwise by tract of time teach the world, that the thing which he blesseth, defendeth, keepeth so strangely, cannot choose but be of him? Wherefore, if any refuse to believe us disputing for the verity of religion established, let them believe God himself thus miraculously working for it, and wish life even for ever and ever unto that glorious and sacred instrument whereby he worketh.

